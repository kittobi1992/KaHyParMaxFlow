\documentclass[11pt]{llncs}
\usepackage{fullpage}
\usepackage[T1]{fontenc}
\usepackage[utf8]{inputenc}

\newcommand{\changefont}[3]{\fontfamily{#1}\fontseries{#2}\fontshape{#3}\selectfont}

\usepackage{booktabs}
\usepackage{url}
\usepackage{xspace}
\usepackage{graphicx}
\usepackage{wrapfig}
\usepackage{amsmath, amsfonts}
\usepackage{algorithm}
\usepackage[noend]{algpseudocode}

\usepackage{enumitem,amssymb}
\newlist{todolist}{itemize}{2}
\setlist[todolist]{label=$\square$}
\usepackage{pifont}
\newcommand{\cmark}{\ding{51}}%
\newcommand{\xmark}{\ding{55}}%
\newcommand{\done}{\rlap{$\square$}{\raisebox{2pt}{\large\hspace{1pt}\cmark}}%
\hspace{-2.5pt}}
\newcommand{\wontfix}{\rlap{$\square$}{\large\hspace{1pt}\xmark}}

\usepackage{pgfplots}
\pgfplotsset{compat=1.9}

\usepackage{amsopn}
\DeclareMathOperator*{\argmax}{arg\,max}
\DeclareMathOperator*{\argmin}{arg\,min}


\newcommand{\GoldbergTarjan}{\textsc{GoldbergTarjan}}
\newcommand{\EdmondKarp}{\textsc{EdmondKarp}}
\newcommand{\BoykovKolmogorov}{\textsc{BoykovKolmogorov}}
\newcommand{\IBFS}{\textsc{Ibfs}}

\title{Concept - Presentation \\ High Quality Hypergraph Partitioning \\ via Max-Flow-Min-Cut Computations}

\author{
    Tobias Heuer
}

\institute{%
Karlsruhe Institute of Technology (KIT), Karlsruhe, Germany.
}

\begin{document}
\maketitle

\section{Introduction}

\begin{todolist}
\item[\done] Problem/Task Description (high-level) and Main Contributions
    \begin{todolist}
    \item[\done] e.g. \emph{Integration of a framework based on Max-Flow-Min-Cut
          computations to improve a balanced $k$-way partition into the n-level
          hypergraph partitioner KaHyPar}
    \end{todolist}
\item[\done] Introduce \emph{hypergrapgs}
\item[\done] Define the $\epsilon$-balanced $k$-way hypergraph partitioning Problem
\item[\done] Applications
\item[\done] Introduce \emph{multilevel paradigm} 
\item Motivation: Disadvantages of \emph{FM} algorithm and why \emph{flow}-based
      approaches solve these problems
    \begin{todolist}
    \item[\done] \emph{Move}-based and only incorparates \emph{local} informations
    \item[\done] \emph{Zero}-Gain Moves
    \item Flow-based approaches are not move-based and finding the global minimum cut 
          separating two vertices $s$ and $t$
    \end{todolist}
\end{todolist}

\section{Preliminaries}

\begin{todolist}
\item Introduce most important notations
\item[\done] Define Flow Problems + Terminology
\end{todolist}

\section{Framework}

\begin{todolist}
\item High-level overview of framework (Mixed with related work)
    \begin{todolist}
    \item \emph{Active Block Scheduling}
    \item Build region around cut + \emph{Adaptive Flow Iterations}
    \item Solve flow problem on a hypergraph flow network
    \item \emph{Most Balanced Minimum Cut}
    \end{todolist}
\item[\done] Flow Networks
    \begin{todolist}
    \item[\done] Vertex Separator Analogy 
    \item[\done] Lawler Network
    \item[\done] Wong Network
    \item[\done] Heuer Network
    \item[\done] Hybrid Network
    \end{todolist}
\item Flow Problem Configuration 
    \begin{todolist}
    \item Modeling of Sanders and Schulz
    \item Optimized modeling approach
    \end{todolist}
\item Flow Algorithms
    \begin{todolist}
    \item \EdmondKarp
    \item \GoldbergTarjan
    \item \BoykovKolmogorov
    \item \IBFS
    \end{todolist}
\item MBMC on hypergraphs
\item Integration into \emph{KaHyPar}
    \begin{todolist}
    \item Flow Execution Policy
    \item Gain-Cache 
    \item Speed-Up Heuristics
    \end{todolist}
\end{todolist}

\section{Experiments}

\section{Conclusion}

\end{document}
