% Vorlage für eine Bachelorarbeit - 2012-2013 Timo Bingmann

% Dies ist nur eine Vorlage. Strikte Vorgaben wie die Bachelorarbeit auszusehen
% hat gibt es nicht. Darum können auch alle Teile angepasst werden.

\documentclass[12pt,a4paper,twoside, enabledeprecatedfontcommands]{scrartcl}

% Diese (und weitere) Eingabedateien sind in UTF-8
\usepackage[utf8]{inputenc}

% Verwende gute Type 1 Font: Latin Modern
\usepackage[T1]{fontenc}
\usepackage{lmodern}

% Sprache des Dokuments (für Silbentrennung und mehr)
\usepackage[german,english]{babel}

% Seitengröße - verwende fast die ganze A4 Seite
\usepackage[tmargin=22mm,bmargin=22mm,lmargin=20mm,rmargin=20mm]{geometry}

% Einrückung und Abstand zwischen Paragraphen
\setlength\parskip{\smallskipamount}
\setlength\parindent{0pt}

% Einige Standard-Mathematik Pakete
\usepackage{latexsym,amsmath,amssymb,mathtools,textcomp}
\usepackage{listings}

% Unterstützung für Sätze und Definitionen
\usepackage{amsthm}
\usepackage{booktabs} %for top, middle and bottomline
\usepackage{bigstrut}
    \setlength\bigstrutjot{3pt}

\newtheorem{Satz}{Satz}[section]
\newtheorem{Definition}[Satz]{Definition}
\newtheorem{Lemma}[Satz]{Lemma}

\numberwithin{equation}{section}

% Deutsches Literaturverzeichnis
\usepackage{bibgerm}

% Unterstützung zum Einbinden von Graphiken
\usepackage{graphicx}

% Pakete die tabular und array verbessern
\usepackage{array,multirow}

% Kleiner enumerate und itemize Umgebungen
\usepackage{enumitem}

\setlist[enumerate]{topsep=0pt}
\setlist[itemize]{topsep=0pt}
\setlist[description]{font=\normalfont,topsep=0pt}

\setlist[enumerate,1]{label=(\roman*)}

% TikZ für Graphiken in LaTeX
\usepackage{tikz}
\usetikzlibrary{calc}

\usepackage{pgfplots}
\usepgfplotslibrary{external}
\tikzexternalize[prefix=experiments/]

\usepackage{pifont}
\usepackage{pbox}
\usepackage{longtable}

% Aktuelle Section und Untersection am Seitenkopf
\usepackage{fancyhdr}

\fancypagestyle{plain}{
  \fancyhead{}
  \fancyfoot{}
  \fancyfoot[LE,RO]{\normalsize\thepage}
  \renewcommand{\headrulewidth}{0pt}
  \renewcommand{\footrulewidth}{0pt}
}

\fancypagestyle{normal}{
  \setlength{\headheight}{20pt}
  \setlength\footskip{32pt}
  \fancyhead{}
  \fancyhead[LE]{\normalsize\textsc{\nouppercase{\leftmark}}}
  \fancyhead[RO]{\normalsize\textsc{\nouppercase{\rightmark}}}
  \fancyfoot{}
  \fancyfoot[LE,RO]{\normalsize\thepage}
  \renewcommand{\headrulewidth}{0.4pt}
  \renewcommand{\footrulewidth}{0pt}
}

% Hyperref für Hyperlink und Sprungtexte
\usepackage{xcolor,hyperref}
\usepackage{subcaption}
\usepackage{graphicx}

\hypersetup{
  pdftitle={High Quality Hypergraph Partitioning via Max-Flow-Min-Cut Computations},
  pdfauthor={Tobias Heuer},
  pdfsubject={Stichworte, weiteres Stichwort},
  colorlinks=true,
  pdfborder={0 0 0},
  bookmarksopen=true,
  bookmarksopenlevel=1,
  bookmarksnumbered=true,
  linkcolor=blue!60!black,
  %linkcolor=black,
  citecolor=blue!60!black,
  urlcolor=blue!60!black,
  filecolor=green!60!black,
  pdfpagemode=UseNone,
  unicode=true,
}

% Paket zum Setzen von Algorithmen in Pseudocode mit kleinen Stilanpassungen
\usepackage[ruled,vlined,linesnumbered,norelsize]{algorithm2e}
\DontPrintSemicolon
\def\NlSty#1{\textnormal{\fontsize{8}{10}\selectfont{}#1}}
\SetKwSty{texttt}
\SetCommentSty{emph}
\def\listalgorithmcfname{Algorithmenverzeichnis}
\def\algorithmautorefname{Algorithmus}
\let\chapter=\section % repariert ein Problem mit algorithm2e

\newtheorem{definition}{Definition}[section]
\newtheorem{sentence}{Sentence}[section]
\newtheorem{theorem}{Theorem}[section]
\newtheorem{lemma}{Lemma}[section]
\newtheorem{problem}{Problem}[section]

\newcommand{\BigO}[1]{$\mathcal O(#1)$}

\newcommand{\HypergraphDef}{$H = (V,E,c,\omega)$}
\newcommand{\FlowNetworkDef}{$G = (V',E',c')$}
\newcommand{\ShortT}[1]{{$T_{#1}(H)$}}
\newcommand{\ShortExT}[2]{{$T_{#1}(H,#2)$}}
\newcommand{\T}[1]{{$T_{#1}(H) = (V_{#1}, E_{#1},c_{#1})$}}
\newcommand{\ExtendedT}[2]{{$T_{#1}(H,#2) = (V_{#1}, E_{#1},c_{#1})$}}
\newcommand{\Hybrid}{{$T_{\text{Hybrid}}(H,V') = (V_{\text{Hybrid}}, E_{\text{Hybrid}},c_{\text{Hybrid}})$}}
\newcommand{\ShortHybrid}{{$T_{\text{Hybrid}}(H,V')$}}


%Residual Capacity
\newcommand{\R}[1]{{r_f({#1})}}
%Vertex Seperator
\newcommand{\V}[1]{V_S(#1)}

\newcommand{\DFS}{\emph{DFS}}
\newcommand{\BFS}{\emph{BFS}}

\newcommand{\GoldbergTarjan}{\textsc{GoldbergTarjan}}
\newcommand{\EdmondKarp}{\textsc{EdmondKarp}}

%Benchmark Instances
\newcommand{\ALL}{\sc All}
\newcommand{\Dual}{\sc Dual}
\newcommand{\Primal}{\sc Primal}
\newcommand{\ISPD}{\sc Ispd98}
\newcommand{\Literal}{\sc{Literal}}
\newcommand{\SPM}{\sc Spm}
\newcommand{\DAC}{\sc Dac}

%Experiment Flow Networks
\newcommand{\ExpLawler}{T_L}
\newcommand{\ExpNodeDegree}{T_H}
\newcommand{\ExpEdgeSize}{T_G}
\newcommand{\ExpHybrid}{T_{\text{Hybrid}}}

%Flow Variants
\newcommand{\FlowVariant}[4]{$(#1\text{Flow},#2\text{CHE},#3\text{MBMC},#4\text{FM})$}

\newcounter{todocount}
\setcounter{todocount}{1}
\newcommand{\todo}[1]{{\color{red}\textbf{TODO \the\value{todocount}:}
        \emph{#1} \addtocounter{todocount}{1}}}

\begin{document}

%%%%%%%%%%%%%%%%%%%%%%%%%%%%%%%%%%%%%%%%%%%%%%%%%%%%%%%%%%%%%%%%%%%%%%

\pagestyle{empty} % keine Seitenzahlen

% Titelblatt der Arbeit
\begin{titlepage}

  \begin{center}\large
 
    \quad\includegraphics[height=17mm]{../img/logo/kit_logo_en.pdf} \hfill
    %\includegraphics[height=20mm]{../img/logo/grouplogo-algo-blue.pdf}\quad\null

    \vfill

    Master-Thesis
    \vspace*{2cm}

    {\textbf{\huge High Quality Hypergraph Partitioning \\ via Max-Flow-Min-Cut Computations} \par}
    % Siehe auch oben die Felder pdftitle={}
    % mit \par am Ende stimmt der Zeilenabstand

    \vfill

    Tobias Heuer

    \vspace*{45mm}

    \begin{tabular}{rl}
      Advisors: & Prof. Dr. Peter Sanders \\
      & M. Sc. Sebastian Schlag\\
    \end{tabular}
    
    \vspace*{10mm}

    %Institut für Theoretische Informatik, Algorithmik \\
    %Fakultät für Informatik \\
    %Karlsruher Institut für Technologie

    % English:
    Institute of Theoretical Informatics, Algorithmics \\
    Department of Informatics \\
    Karlsruhe Institute of Technology

    \vspace*{15mm}

    Date of submission: 22.12.2017

    \vspace*{12mm}
  \end{center}

\end{titlepage}

%%%%%%%%%%%%%%%%%%%%%%%%%%%%%%%%%%%%%%%%%%%%%%%%%%%%%%%%%%%%%%%%%%%%%%

\vspace*{0pt}\vfill

\hrule\medskip

Hiermit versichere ich, dass ich diese Arbeit selbständig verfasst und keine anderen, 
als die angegebenen Quellen und Hilfsmittel benutzt, die wörtlich oder inhaltlich 
übernommenen Stellen als solche kenntlich gemacht und die Satzung des Karlsruher 
Instituts für Technologie zur Sicherung guter wissenschaftlicher Praxis in der jeweils 
gültigen Fassung beachtet habe.

\bigskip

\noindent
Karlsruhe, den 22.12.2017

% Unterschrift (handgeschrieben)

\vspace*{5cm}

\clearpage

%%%%%%%%%%%%%%%%%%%%%%%%%%%%%%%%%%%%%%%%%%%%%%%%%%%%%%%%%%%%%%%%%%%%%%

\vspace*{0pt}\vfill

\selectlanguage{english}
\begin{abstract}
\centerline{\textbf{Abstract}}

Currently, algorithms based on the \emph{FM} idea \cite{fiduccia1988linear} are the 
only practical heuristics to improve a $k$-way
partition in a multilevel hypergraph partitioner. However, they are often criticized for their 
limited ability to lookahead \cite{zhao2002effective}. It might be more beneficial to move a 
hypernode with small gain, because it will induce many good moves later.
We present an alternative \emph{local search} approach based on \emph{Max-Flow-Min-Cut}
computations. The framework is inspired by the work of Sanders and Schulz \cite{sanders2011engineering} who
successfully showed that \emph{flow}-based refinement in combination with the \emph{FM}
algorithm significantly improve the quality of partitions in a multilevel graph partitioner.
In this work, we develop different modeling techniques of a hypergraph as flow network and
show how to improves the connectivity metric of a $k$-way partition by building a flow problem 
on a subset of the vertices. We integrated the framework in the hypergraph partitioner 
\emph{KaHyPar} by applying and adapting the basic framework of \cite{sanders2011engineering}.
On our large benchmark set with $3222$ instances our new configuration outperforms all 
state-of-the-art hypergraph partitioner on $70\%$ of the instances. In comparison to the latest
configuration of \emph{KaHyPar} our new approach produces $2\%$ better quality by a performance
slowdown only by a factor of $2$.

\end{abstract}

\selectlanguage{german}
\begin{abstract}
\centerline{\textbf{Zusammenfassung}}

Algorithmen basierend auf der \emph{FM}-Idee \cite{fiduccia1988linear} sind zur Zeit 
die einzigsten praktischen Heuristiken, um eine $k$-teilige
Partitionierung in einem \emph{Multilevel Hypergraph Partitioner} zu verbessern. 
Jedoch werden sie oft kritisiert für ihre limitierte Eigenschaft
vorrauszuschauen \cite{zhao2002effective}. Zum Beispiel könnte es von Vorteil sein ein
Knoten mit geringem \emph{Gain} zu verschieben, weil er vielleicht später viele bessere
Verschiebungen induziert. Wir präsentieren einen alternativen \emph{Lokale 
Suche} Ansatz basierend auf \emph{Max-Flow-Min-Cut} Berechnungen. Das Framework ist inspiriert 
durch die Arbeit von Sanders und Schulz, welche gezeigt haben, dass ein \emph{flow}-basierter
Ansatz in Kombination mit dem \emph{FM}-Algorithmus signifikant die Qualität von Partitionierungen
in einem \emph{Multilevel Graph Partitioner} verbessert \cite{sanders2011engineering}. 
In dieser Arbeit entwickeln wir verschiedene Modellierungstechnicken eines Hypergraphen 
als Flussnetzwerk und zeigen wie die \emph{connectivity metric} einer $k$-teiligen 
Partitionierung verbessert werden kann, indem ein Flussproblem auf einer
Teilmenge der Knoten aufgebaut wird. Wir haben das Framework in den
\emph{Hypergraph Partitioner KaHyPar} integriert, indem wir das Framework von \cite{sanders2011engineering}
übernommen und angepasst haben. Auf unserem großen \emph{Benchmark Set} mit $3222$ Instanzen
erzielt unsere neue Konfiguration auf $70\%$ der Instanzen eine bessere Qualität als die meisten
\emph{State-of-the-Art} Partitionierer. Im Vergleich zu der letzten Konfiguration von \emph{KaHyPar}
erreichen wir mit unserem Ansatz $2\%$ bessere Qualität mit nur doppelt so langer Laufzeit.

\end{abstract}

\vfill

\selectlanguage{english}

\vfill\vfill\vfill
\clearpage

%%%%%%%%%%%%%%%%%%%%%%%%%%%%%%%%%%%%%%%%%%%%%%%%%%%%%%%%%%%%%%%%%%%%%%

\vspace*{0pt}\vfill

\section*{Acknowledgements}

A good mentor is important for staying motivated and to have the feeling to work
on something meaningful. Since my bachelor thesis, I work together with Sebastian
Schlag in the research area of \emph{Hypergraph Partitioning}. The decision to further
work on this topic was not only a matter of interest, it was mainly a decision based
on our outstanding interpersonal working relationship. Our intellectual discussions
were characterized by the right balance of fun and the necessary serious to work towards
a common goal. I think here is the right place to thank you for the endless time, which I have spent
in your office over the last three years and which have made me a better computer scientist. \\
Further, I would like to thank my girl friend Alessa Dreixler. To be together with a 
computer scientist could be sometimes very complicated and exhausting. Especially,
if anger and frustration dominated my working day. With her understanding and motivation, 
I could continue every morning with the same energy and passion as the day before.


\vfill\vfill\vfill
\clearpage

%%%%%%%%%%%%%%%%%%%%%%%%%%%%%%%%%%%%%%%%%%%%%%%%%%%%%%%%%%%%%%%%%%%%%%

\pagestyle{normal}
% markiere sections im Seitenkopf links und subsections rechts
\renewcommand\sectionmark[1]{\markboth{\thesection\quad\MakeUppercase{#1}}{\thesection\quad\MakeUppercase{#1}}}
\renewcommand\subsectionmark[1]{\markright{\thesubsection\quad\MakeUppercase{#1}}}

% Inhaltsverzeichnis
\tableofcontents

\clearpage

%%%%%%%%%%%%%%%%%%%%%%%%%%%%%%%%%%%%%%%%%%%%%%%%%%%%%%%%%%%%%%%%%%%%%%

%\listoffigures
%\listoftables
%\listofalgorithms
%
%\clearpage

%%%%%%%%%%%%%%%%%%%%%%%%%%%%%%%%%%%%%%%%%%%%%%%%%%%%%%%%%%%%%%%%%%%%%%

\section{Introduction}

Hypergraphs are generalization of graphs, where each (hyper)edge can connect 
more than two (hyper)nodes. The $k$-way hypergraph partitioning problem is to 
patition the vertices of a hypergraph into $k$ disjoint non-empty blocks such
that the size of each block satisfy a lower and upper bound, while we simultanously 
want to minimize an objective function. \\
Classical application areas can be found in \emph{VLSI} design, parallelization of
the \emph{Sparse Matrix Vector Product} and simplying \emph{SAT} formulas 
\cite{karypis1999multilevel, mann2014formula, papa2007hypergraph}. The goal 
in \emph{VLSI} design is to partition a circuit into smaller units such that
the wires between the gates are as short as possible \cite{bulucc2016recent}.
Since a wire can connect more than two gates a hypergraph models a circuit more
accurate than a graph. In \emph{SAT} solving hypergraph partitioning is used to
decompose a formula into smaller subformulas, which can be solved easier \cite{mann2014formula}.
Beneath the classical application areas hypergraph partitioning can also be found
in more vivid areas like \emph{Warehouse Planning}. A warehouse consists of several
storage spaces where products can be placed. If we have a list of orders of the
past, we can interprete the products as vertices and the orders as hyperedges. If we
partition the hypergraph into $k$ blocks, where $k$ is the number of storage spaces,
we can place products in the warehouse such that products are close to each other if
they are often ordered together.\\
Hypergraph partitioning is a NP-hard problem \cite{lengauer2012combinatorial} and
it is even hard NP-hard to find a good approximation of solutions \cite{bui1992finding}.
The most common used heuristic in state-of-the-art hypergraph partitioner is the
\emph{multilevel paradigm} \cite{catalyurek1999hypergraph, heuer2017improving, karypis1999multilevel}.
First a sequence of smaller hypergraphs are calculated by contracting a set of hypernode pairs
in each step (\emph{coarsening phase}). If the hypergraph is small enough we can use expensive
heuristics to \emph{initial partition} the hypergraph into $k$ blocks. Afterwards, the sequence
of smaller hypergraphs is \emph{uncontracted} in reverse order and, at each level, a 
\emph{local search} heuristic is used to improve the quality of a partition 
(\emph{refinement phase}). \\
There exists several \emph{local search} heuristics for improving a partition of a hypergraph,
but only the \emph{FM} algorithm leads to a practical performance of a multilevel hypergraph
partitioner for large benchmarks \cite{papa2007hypergraph}. In general, the \emph{FM}
heuristics maintains gain values (according to the objective function) of moving a node
from its current block to an other block \cite{fiduccia1988linear}. A move is performed, 
if its gain value is maximum among all possible moves. The algorithm can be implemented 
in linear time. Since a move is performed greedily the algorithm tends to find local
optimal solutions. \\
Sanders and Schulz \cite{sanders2011engineering} successfully integrated a \emph{flow}-based refinement
algorithm in their multilevel graph partitioner. It is well known that a maximum $(s,t)$-flow
calculation yields to a minimum $(s,t)$-cutset on graphs \cite{ford1956maximal}. Their general
approach was to extract a subhypergraph around the cut and configure the source and sink sets
of the flow problem such that a maximum flow calculation on the subhypergraph leads to a 
smaller cut on the original graph. In combination with the \emph{FM} heurisitc their \emph{local
search} algorithm has the ability to find out of local optimal solutions and produces
the best partitions for a wide range of graph partitioning benchmarks.

\subsection{Problem Statement}

Currently their are no competitive alternatives to the \emph{FM} heuristic as \emph{local search}
algorithm for a multilevel hypergraph partitioner. Sanders and Schulz \cite{sanders2011engineering}
showed that \emph{flow}-based approaches can be used in a multilevel graph partitioner to obtain
high quality partitions. Their algorithm is a generic framework, which basic ideas can be
applied one-to-one to hypergraphs. However, several key challanges remain.\\
First, we have to find an appropriate model of a hypergraph as flow network. Each maximum
$(s,t)$-flow on this model should induced a minimum $(s,t)$-cutset on the hypergraph.
Further, our \emph{flow}-based approach should improve a bipartition of a subhypergraph in such
a way that the resulting bipartition yields to an improved $k$-way partition on the 
original hypergraph. Therefore, we have to find a source and sink set modelling approach
such that the above formulated contraints are satisfied. \\
The framework should be integrated into the $n$-level hypergraph partitioner 
\emph{KaHyPar}. \emph{KaHyPar} is a multilevel hypergraph partitioner in its most extreme 
version by only contracting two vertices in one level of the multilevel hierarchy
\cite{akhremtsev2017engineering,heuer2017improving,schlag2016k}. In the \emph{refinement 
phase} $n$ \emph{local searches} are instantiated. Therefore, the most challenging part is
to implement the framework in such a way that we obtain
high quality partitions and simultanously ensure that the performance reduction is within
constant factor.

\subsection{Contributions}

We present several sparsifying techniques of the state-of-the-art hypergraph flow network
modelling approach proposed by Lawler \cite{lawler1973}. Our experiments indicates that
maximum flow algorithms are up to a factor of $3$ faster with our new network. Further, 
we show that the source and sink sets of the resulting flow network of a subhypergraph 
of a already partitioned hypergraph can be configured more flexible than on graphs. More 
precisely, applying the approach of Sanders and Schulz \cite{sanders2011engineering}
directly on hypergraphs results in a minimum $(S,T)$-cutset greater or equal as with
our new technique. We integrate the framework of \cite{sanders2011engineering} into 
\emph{KaHyPar} and show that \emph{flow}-based refinement in combination with the
\emph{FM} algorithm produces on a large majority of a wide range of real world
benchmarks the best known partitions in comparison to other state-of-the-art hypergraph
partitioner. In comparison to latest quality preset of \emph{KaHyPar} our new approach
produces on average $2\%$ better partitions and is only slower by a factor of $2$.

\subsection{Outline}

We first introduce important notations and summarize related work in Section \ref{sec:preliminaries}
and \ref{sec:related_work}. Afterwards, we describe sparsifying techniques of the flow network
proposed by Lawler \cite{lawler1973} in Section \ref{sec:opt_flow_network}. In Section
\ref{sec:flow_refinement} we present our optimized source and sink set modelling
approach and describe the integration of our \emph{flow}-based refinement framework into
the $n$-level hypergraph partitioner \emph{KaHyPar}. The evaluation of our new flow network
proposed in Section \ref{sec:opt_flow_network} and framework proposed in Section 
\ref{sec:flow_refinement} is presented in Section \ref{sec:experiments}. 
Section \ref{sec:conclusion} concludes this thesis.

\newpage

%%%%%%%%%%%%%%%%%%%%%%%%%%%%%%%%%%%%%%%%%%%%%%%%%%%%%%%%%%%%%%%%%%%%%%

\section{Preliminaries}
\label{sec:preliminaries}

\subsection{Graphs}

\begin{definition}
A directed weighted graph $G = (V,E,c,\omega)$ is a set of nodes $V$ 
and a set of edges $E$ with a node weight function 
$c: V \rightarrow \mathbb{R}_{\ge 0}$ and an edge weight 
function $\omega: E \rightarrow \mathbb{R}_{\ge 0}$. A edge $e = (u,v)$ 
is a relation between two nodes $u,v \in V$.
\label{def:hypergraph}
\end{definition}

Two vertices $u$ and $v$ are \emph{adjacent}, if there exists an edge
$(u,v) \in E$. Two edges $e_1$ and $e_2$ are \emph{incident} to each other, if they
share a node. $I(v)$ denotes the set of all \emph{adjacent} nodes of 
$v$. The \emph{degree} of a node $v$ is $d(v) = |I(v)|$.

\begin{definition}
Given a directed graph $G = (V,E)$. A contraction of two nodes
$u$ and $v$ results in a new graph $G_{(u,v)} = (V\setminus\{v\},E')$, where
each edge of the form $(v,w)$ or $(w,v)$ in $E$ is replaced with an edge 
$(u,w)$ or $(w,u)$ in $E'$.
\label{def:contraction}
\end{definition}

A \emph{path} $P = (v_1,\ldots,v_k)$ is a sequence of nodes, where for
each $i \in [1,k-1]: (v_i,v_{i+1}) \in E$. A \emph{cycle} is a \emph{path}
$P = (v_1,\ldots,v_k)$ with $v_1 = v_k$. A \emph{strongly connected 
component} $C \subseteq V$ is a set of nodes where for each $u,v \in C$
exists a \emph{path} from $u$ to $v$. We can enumerate all \emph{strongly
connected components} (\emph{SCC}) in a directed graph $G$ with a linear time algorithm
proposed by Tarjan \cite{tarjan1972depth}. A directed graph $G$ without
any \emph{cycles} is called \emph{directed acyclic graph} (\emph{DAG}). On such
graphs we can define a \emph{topological order} $\gamma: V \rightarrow \mathbb{N}_+$ such
that for each $(u,v) \in E: \gamma(u) < \gamma(v)$. A \emph{topological order}
of a \emph{DAG} can be found in linear time with Kahn's algorithm \cite{kahn1962topological}.
We can transform a general directed graph $G$ into a \emph{DAG}, if we contract
each \emph{strongly connected component}. All concepts are illustrated in 
\autoref{img:scctoposort}.

\begin{figure}
\centering
\includegraphics[width=0.9\textwidth]{../img/preliminaries/scctoposort.eps}
\caption{Example of \emph{strongly connected components} of a directed graph and
         a \emph{topological order} on a \emph{directed acyclic graph}. Each 
         \emph{SCC} is marked in the same color.} 
\label{img:scctoposort}
\end{figure}

\begin{definition}
Let $G_{V'} = (V',E_{V'},c,\omega)$ be the subgraph of a graph $G$
induced by $V' \subseteq V$ with $E_{V'} = \{(u,v) \in E\ |\ u,v \in V'\}$.
\label{def:subgraph}
\end{definition}


\subsection{Flows and Applications}
\label{sec:applications}

Given a graph $G = (V,E,c)$ with capacity function $c: E \rightarrow \mathbb{R}_+$ and a source 
$s \in V$ and a sink $t \in V$. The maximum flow problem is about finding 
the maximum amount of flow from $s$ to $t$ in $G$. A flow is a function 
$f: E \rightarrow \mathbb{R}_+$, which have to satisfy the following constraints:
\begin{enumerate}
\item $\forall (u,v) \in E: f(u,v) \le c(u,v)$ (capacity constraint)
\item $\forall v \in V \setminus \{s,t\}: \sum_{(u,v) \in E} f(u,v) = \sum_{(v,u) \in E} f(v,u)$ (conservation of flow constraint)
\end{enumerate}
The capacity constraint restricts the flow on an edge $(u,v)$ by its capacity 
$c(u,v)$. Whereas the conservation of flow constraint ensures that the amount
of flow entering a node $v \in V \setminus \{s,t\}$ is the same as leaving a node.
The value of the flow is defined as $|f| = \sum_{(s,v) \in E} f(s,v) = \sum_{(v,t) \in E} f(v,t)$.
A flow $f$ is maximal, if there exists no other flow $f'$ with $|f'| > |f|$. \\
Another useful construct in connection with maximum flows, is the concept of the
\emph{residual graph} $G_f$ and the \emph{residual capacity} $r_f$ of a flow function $f$ on graph $G$.
The \emph{residual capacity} $r_f: V \times V \rightarrow \mathbb{R}_+$ is defined as follows:
\begin{enumerate}
\item $\forall (u,v) \in E: r_f(u,v) = c(u,v) - f(u,v)$
\item $\forall (u,v) \in E:$ If $f(u,v) > 0$ and $c(v,u) = 0$, then $r_f(v,u) = f(u,v)$
\end{enumerate}
For a edge $e = (u,v) \in E$ the residual capacity $r_f(u,v)$ is the remaining amount of 
flow which can be send over edge $e$. For each reverse edge $\overleftarrow{e} \notin E$ the
residual capacity $r_f(\overleftarrow{e})$ is the amount of flow which is send over $e$.
The \emph{residual graph} $G_f = (V,E_f,r_f)$ is the network containing all $(u,v) \in V \times V$
with $r_f(u,v) > 0$. More formal $E_f = \{(u,v)\ |\ r_f(u,v) > 0, (u,v) \in V \times V\}$.
\autoref{img:maximum_flow_example}~illustrates all presented concepts.

\begin{figure}
\centering
\includegraphics[width=0.9\textwidth]{../img/maximum_flow/maximum_flow_example.eps}
\caption{Figure illustrates concepts related to the maximum flow problem. A valid flow $f$ 
(red values) from $s$ to $t$ on a graph $G$ is shown on the left side. The corresponding
\emph{residual graph} $G_f$ with its \emph{residual capacities} (black values) 
is illustrated on the right side. The red highlighted path represents an \emph{augmenting path}
in $G$.}
\label{img:maximum_flow_example}
\end{figure}

The \emph{Max-Flow-Min-Cut}-Theorem is fundamental for many applications related to the maximum
flow problem \cite{ford1956maximal}.

\begin{theorem}
The value of a maximum $(s,t)$-flow obtainable in a graph $G$ is equal with the weight
of the minimum cutset in $G$ seperating $s$ and $t$.
\end{theorem}

Let $f$ be a maximum flow in a graph $G = (V,E,\omega)$ with $s \in V$ and $t \in V$. 
Further, let $A$ be the set containing all $v \in V$, which are \emph{reachable} from $s$
in $G_f$. A node $v$ is \emph{reachable} from a node $u$, if there exists a path from $u$
to $v$. Then the set of all cut edges between the bipartition $(A,V\setminus A)$ 
is a minimum-weight $(s,t)$-cutset \cite{ford2015flows}. $A$ can be calculated with a simple \BFS~in $G_f$ starting
from $s$. \\
From this analogy many solutions for related problems arose. Samples are listed below:
\begin{enumerate}
\item Maximum Bipartite-Matching
\item Minimum-Weight Vertex Seperator
\item Number of Edge-Disjoint Paths
\item Number of Vertex-Disjoint Paths
\end{enumerate}
Solutions for those problems sometimes involves a transformation $T$ of the graph $G$
into a flow network $T(G)$, such that the \emph{Max-Flow-Min-Cut}-Theorem is applicable. 
A problem important for this work is to find a minimum-weight $(s,t)$-vertex seperator
in a graph $G = (V,E,c)$ with $c: V \rightarrow \mathbb{R}_+$.

\begin{definition}
Let $G = (V,E,c)$ be a graph with $c: V \rightarrow \mathbb{R}_+$. $S \subseteq V$
is a vertex seperator for non-adjacent verticies $s \in V$ and $t \in V$ if the
removal of $S$ from graph $G$ seperates $s$ and $t$ ($s$ not \emph{reachable} from $t$).
A vertex seperator $S$ is a minimum-weight $(s,t)$-vertex seperator, if for all $S' \subseteq V$
$c(S) \le c(S')$.
\end{definition}

We can calculate a minimum-weight $(s,t)$-vertex seperator with a maximum flow
calculation in the following flow network (\todo{reference}):

\begin{definition}
\label{def:vertex_seperator_transformation}
Let $T_V$ be a transformation of a graph $G = (V,E,c)$ into 
a flow network $T_V(G) = (V_V, E_V, c_V)$ (with $c_V: E_V \rightarrow \mathbb{R}_+$). 
$T_V$ is defined as follows:
\begin{enumerate}
\item $V_V = \bigcup\limits_{v \in V}\ \{v', v''\}$
\item $\forall v \in V$ we add a directed edge $(v',v'')$
      with capacity $c_V(v',v'') = c(v)$
\item $\forall (u,v) \in E$ we add two directed edges $(u'', v')$ and 
      $(v'', u')$ with capacity $c_V(u'', v') = c_V(v'', u') = \infty$.
\end{enumerate} 
\end{definition}

The vertex seperator problem and transformation $T_V(G)$ are illustrated in \autoref{img:vertex_seperator_example}.
Obviously no edge between two adjacent nodes can be in a minimum-capacity $(s,t)$-cutset of $T_V(G)$,
because for all those edges the capacity is $\infty$. Therefore, the cutset must consist
of edges of the form $(v',v'')$. A minimum-weight $(s,t)$-vertex seperator can be calculated by
finding a maximum flow in $T_V(G)$, finding the minimum-capacity $(s,t)$-cutset with the procedure
described above and then map each cut edge $(v',v'')$ to their corresponding node $v$.\\
Given a set of sources $S$ and sinks $T$. The \emph{multi-source multi-sink} maximum flow problem is
about finding a maximum flow $f$ from all source nodes $s \in S$ to all sink nodes $t \in T$.
We can transform such a problem into a \emph{single-source single-sink} problem by adding
two additional nodes $s$ and $t$. We add a directed edge from $s$ to all source nodes $s' \in S$ 
and for all sink nodes $t' \in T$ a directed edge to $t$ with capacity 
$c(s,s') = c(t',t) = \infty$.

\begin{figure}
\centering
\includegraphics[width=0.9\textwidth]{../img/maximum_flow/vertex_seperator_example.eps}
\caption{ Illustration of the vertex seperator problem and the transformation $T_V(G)$ in which
          we can find a minimum vertex seperator with maximum flow computation. }
\label{img:vertex_seperator_example}
\end{figure}

\subsection{Hypergraphs}
\label{sec:hypergraph}

\begin{definition}
An undirected weighted hypergraph $H = (V,E,c,\omega)$ is a set of hypernodes $V$ 
and a set of hyperedges $E$ with a hypernode weight function 
$c: V \rightarrow \mathbb{R}_{\ge 0}$ and a hyperedge weight 
function $\omega: E \rightarrow \mathbb{R}_{\ge 0}$. A hyperedge $e$ 
is a subset of $V$ (formally: $\forall e \in E: e \subseteq V$).
\label{def:hypergraph}
\end{definition}

A hypergraph generalizes a graph by extending the definition of an edge, which 
can contain more than two nodes. Hyperedges are also called \emph{nets} and the hypernodes
of a net are called \emph{pins}. For a subset $V' \subseteq V$ and $E' \subseteq E$ we
define
\begin{align*}
c(V') = \sum_{v \in V'} c(v) \\
\omega(E') = \sum_{e \in E'} \omega(e)
\end{align*}

A vertex $v$ is \emph{incident} to a hyperedge $e$, if $v \in e$.
Two vertices $u$ and $v$ are \emph{adjacent}, if there exists an 
$e \in E$ such that $u,v \in e$. $I(v)$ denotes the set of all 
\emph{incident} nets of $v$. The \emph{degree} of a hypernode 
$v$ is $d(v) = |I(v)|$. The size of a net $e$ is the cardinality $|e|$.

\begin{definition}
Let $H_{V'} = (V',E_{V'},c,\omega)$ be the subhypergraph of a hypergraph $H$
induced by $V' \subseteq V$ with $E_{V'} = \{e \cap V'\ |\ e \in E: e 
\cap V' \neq \emptyset\}$.
\label{def:subhypergraph}
\end{definition}

A hypergraph $H = (V,E,c,\omega)$ can be represented as an undirected graph. 
There are two common transformations, called \emph{clique} and \emph{bipartite} 
representation \cite{HuMoerder85}. The \emph{clique} graph $G_x(H) = (V,E_x)$ models
each net $e$ as a clique between its pins. The \emph{bipartite} graph $G_*(H) = 
(V \cup E, E_*)$ contains all hypernodes and hyperedges as nodes and connects each
net $e$ with an undirected edge $\{e,v\}$ to all its pins $v \in e$. The two transformations
are illustrated in \autoref{img:hypergraph_transformation}.

\begin{figure}
\centering
\includegraphics[width=1.0\textwidth]{../img/preliminaries/hypergraph_transformation.eps}
\caption{Example of a hypergraph $H$ and its two corresponding graph representations.} 
\label{img:hypergraph_transformation}
\end{figure}

\subsection{Hypergraph Partitioning}

\begin{definition}
A $k$-way partition of a hypergraph $H$ is a partition of its hypernodes into
$k$ disjoint blocks $\Pi = \{V_1,\ldots,V_k\}$ such that $\bigcup_{i=1}^{k} V_i = V$
and $V_i \neq \emptyset$.
\label{def:kway_partition}
\end{definition}

For a $k$-way partition $\Pi = \{V_1,\ldots,V_k\}$, we define the \emph{connectivity set} of a
hyperedge $e$ with $\Lambda(e,\Pi) = \{V_i \in \Pi\ |\ V_i \cap e \neq \emptyset\}$. The \emph{connectivity}
of a net $e$ is $\lambda(e,\Pi) = |\Lambda(e,\Pi)|$. A hyperedge $e$ is \emph{cut}, if
$\lambda(e,\Pi) > 1$. $E(\Pi) = \{e\ |\ \lambda(e,\Pi) > 1\}$ is the set of all \emph{cut} 
nets. We say two blocks $V_i$ and $V_j$ are adjacent, if there exists a hyperedge
$e$ with $V_i,V_j \in \Lambda(e,\Pi)$.

\begin{definition}
For a $k$-way partition $\Pi = \{V_1,\ldots,V_k\}$ of a hypergraph $H$ 
the quotient graph $Q = (\Pi,E')$ is a undirected graph containing an 
edge between each pair of adjacent blocks of $\Pi$.
More formal, $E' = \{(V_i,V_j)\ |\ \exists e \in E: V_i,V_j \in \Lambda(e,\Pi)\}$
\label{def:quotient_graph}
\end{definition}

We say a $k$-way partition is $\epsilon$-balanced, if each block 
$V_i \in \Pi$ satisfies the \emph{balance constraint} 
$c(V_i) \le (1+\epsilon)\lceil\frac{c(V)}{k}\rceil$.

\begin{definition}
The $k$-way hypergraph partitioning problem is to find an $\epsilon$-balanced $k$-way
partition $\Pi$ of a hypergraph $H$ such that a certain objective function is minimized.
\label{def:kway_partitioning_problem}
\end{definition}

There exists several objective function in the hypergraph partitioning context,
which should either be minimized or maximized. The most popular objective function 
is the $\text{cut}$ metric (especially for \emph{graph partitioning}), which is defined as
\[\omega_H(\Pi) = \sum_{e \in E(\Pi)} \omega(e)\]
The goal is to minimize the sum of all \emph{cut} hyperedges. Another important metric
for this work is the $(\lambda - 1)$-metric or \emph{connectivity} metric, 
which is defined as
\[(\lambda - 1)_H(\Pi) = \sum_{e \in E} (\lambda(e) - 1)\omega(e)\]
The idea behind this function is to minimize the \emph{connectivity} of all hyperedges.

\newpage

%%%%%%%%%%%%%%%%%%%%%%%%%%%%%%%%%%%%%%%%%%%%%%%%%%%%%%%%%%%%%%%%%%%%%%

\section{Related Work}
\label{sec:related_work}

%%%%%%%%%%%%%%%%%%%%%%%%%%%%%%%%%%%%%%%%%%%%%%%%%%%%%%%%%%%%%%%%%%%%%%

\subsection{Maximum Flow Algorithms}
\label{sec:max_flow}

In Section \ref{sec:applications} we introduce the concept of flows in a network. We
will now present two approaches to solve the maximum flow problem.

\subsubsection{Augmenting-Path Algorithms}
\label{sec:aug_path}

An \emph{augmenting path} $P = \{v_1,\ldots,v_k\}$ is a path in the residual graph $G_f$ with $v_1 = s$ and 
$v_k = t$ \cite{edmonds1972theoretical}. \autoref{img:edmond_karp_example}~illustrates such a path.
For all edges $(u,v) \in G_f$ it holds that $r_f(u,v) > 0$.
Therefore we can increase the flow on all edges $(v_i,v_{i+1})$ by 
$\Delta f = \min_{i \in [1,\ldots,k-1]} r_f(v_i,v_{i+1})$. It can be shown that $f$ is not a
maximum flow if an augmenting path exists in $G_f$ \cite{edmonds1972theoretical}. \\
One way to calculate a maximum flow $f$ is to find augmenting paths in $G_f$ as
long as there exists one. The algorithm was established by Ford and Fulkerson \cite{ford1956maximal} and
consists of two phases. First, we search for an augmenting path $P = \{v_1,\ldots,v_k\}$
from $s$ to $t$, e.g., with a simple \DFS. Afterwards, we increase the flow on each
edge $(v_i,v_{i+1})$ by $\Delta f$ and decrease the flow on each reverse edge $(v_{i+1},v_i)$
by $\Delta f$. If the capacities are integral, the algorithm always terminates. Since we can find an augmenting
path in $G_f$ with a simple \DFS~in \BigO{|V|+|E|} time and increase the
flow on every path by at least one, the running time of the algorithm can be bounded by \BigO{|E||f_{max}|}.
We can construct instances where the running time is \BigO{|E||f_{max}|} or even instances where 
the maximum flow $|f_{max}|$ is exponential in the problem size \cite{edmonds1972theoretical}. \\
Edmond and Karp \cite{edmonds1972theoretical} improved Ford \& Fulkerson's algorithm by increasing the flow
along an augmenting path of minimal length. The shortest path from $s$ to $t$ in a 
graph with unit lengths can be found with a simple \BFS. It can be shown that the
total number of \emph{augmentations} is in \BigO{|V||E|}. The running time of Edmond \& Karp's
maximum flow algorithm is \BigO{|V||E|^2}. An exemplary execution of the algorithm
is presented in \autoref{img:edmond_karp_example}. \\
\begin{figure}
\centering
\includegraphics[width=0.9\textwidth]{../img/maximum_flow/edmond_karp_example.eps}
\caption{Execution of Edmond \& Karps maximum flow algorithm \cite{edmonds1972theoretical}.
         The network $G$ with its capacities $c$ (black values) and flow $f$ (red values) is illustrated
         on the left side. The residual graph $G_f$ with its \emph{residual capacities} $r_f$ (black values)
         is presented on the right side. In each step the current augmenting path in $G_f$ is highlighted
         by a red path. }
\label{img:edmond_karp_example}
\end{figure}
Boykov and Kolmogorov proposed a maximum flow algorithm based on augmenting path especially
designed for applications in computer vision \cite{boykov2004experimental}. Their basic idea is to 
grow two search trees simultanousely. One is starting from the source and one from the sink.
The two search trees maintain the invariant that all edges in the tree are non-saturated. More formally,
for an edge $e$ the residual capacity $r_f(e)$ must be greater than zero.
A node is added to one of the two trees if a non-saturated edge exists that connects
the node with one of the nodes in the search trees. If the two trees touch at a given node, we found
an augmenting path from the source to the sink. After we increase the flow along this
path, some of the edges in the two search trees are saturated. Therefore, the algorithm
tries to restore the search tree invariant by finding a new non-saturated edge for each node which
is connected through a saturated edge to the tree. If this is not possible, then the node is removed
from the tree. The algorithm has no polynomial complexity (worst case \BigO{|E||V|^2|f|}), but it
outperforms many state-of-the-art maximum flow algorithms on computer vision benchmarks \cite{boykov2004experimental}.\\
An extension of the algorithm of Boykov \& Kolmogorov is the \emph{incremental breadth-first search}
algorithm \cite{goldberg2011maximum}, which guarantes polynomial running time (\BigO{|V|^2|E|}).
The algorithm maintains two distance labels $d_s$ and $d_t$ for each node. For some values 
$D_s$ and $D_t$, the source tree contains all nodes up to a distance $D_s$ and the sink tree
up to distance $D_t$. Furthermore it maintains the invariant that $L = D_s + D_t + 1$ is a lower bound
for the shortest augmenting path. Initially, $d_s(s) = d_t(t) = 0$ and $D_s = D_t = 0$.
The algorithm works in passes and in a pass one of the two trees is chosen to grow. Assume, we have chosen
the source tree. Each node $u$ contained in the source tree with distance label $d_s(u) = D_s$ is
marked as \emph{active}. In a pass all \emph{active} nodes are processed. If an \emph{active} node
$u$ is adjacent to a node $v$ not contained in any of the two trees, we add $v$ to the source tree
and set $d_s(v) = d_s(u) + 1$. If $v$ is in the sink tree, we have found an augmenting path.
After augmenting along that path some of the nodes are not connected to the tree through a non-saturated
edge. For such a node $v$ the adjacency list is scanned and if an adjacent node $u$ exist with
$d_s(u) = d_s(v) - 1$ and $r_f(u,v) > 0$, the parent of $v$ is set to $u$. If such a node is not found,
we search for an adjacent node $u$ for which $d_s(u)$ is minimal and $r_f(u,v) > 0$. If such
a node is found, we set the parent of $v$ to $u$ and $d_s(v) = d_s(u) + 1$. Otherwise, $v$ is removed
from the source tree. If after a pass a node $v$ exists with $d_s(v) = D_s + 1$, $D_s$ is set to $D_s + 1$,
otherwise the algorithm terminates.

\subsubsection{Push-Relabel Algorithm}
\label{sec:push_relabel}

Goldberg and Tarjan \cite{goldberg1988new} proposed a maximum flow algorithm
not based on finding an augmenting path in the \emph{residual graph} instead, the idea is
to maintain a \emph{preflow} during the execution of the algorithm which satisfies the capacity 
constraints, but only a weakened form of the conservation of flow constraint:
\[\forall v \in V \setminus \{s,t\}: \sum_{u \in V} f(v,u) \le \sum_{u \in V} f(u,v)\]
The algorithm maintains a \emph{distance labeling} $d: V \rightarrow \mathbb{N}$ and an 
\emph{excess function} $e_f: V \rightarrow \mathbb{N}$. The \emph{distance labeling} satisfies
the following conditions: $d(s) = |V|$, $d(t) = 0$ and for each $(u,v) \in E_f$, $d(u) \le d(v) + 1$. We say an
residual edge $(u,v)$ is \emph{admissible} if $d(u) = d(v) + 1$. A node $v$ is \emph{active}
if $v \notin \{s,t\}$ and $e_f(v) > 0$.\\
Initially, all \emph{labels} and \emph{excess} values are set to zero except for the source node $s$ will be set to $d(s) = 1$
and $e_f(s) = \infty$. For each \emph{active} node $u$ the algorithm performs two update
operations, called \emph{push} and \emph{relabel}. The first operation pushes flow
over each \emph{admissible} edge $(u,v)$. After a \emph{push} $e_f(u) = e_f(u) - 
\min{(e_f(u),r_f(u,v))}$ and $e_f(v) = e_f(v) + \min{(e_f(u),r_f(u,v))}$. If there is no
\emph{admissible} edge, a \emph{relabel} operation is performed, which replaces $d(u)$ by
$\min_{(u,v) \in E_f} d(v) + 1$. The algorithm terminates, if none of the nodes is \emph{active}.
The worst case complexity of the algorithm is \BigO{n^3}. The running time can be reduced
to \BigO{n^2\log{n}} with \emph{Dynamic Trees} \cite{goldberg1988new, sleator1981data}, but this
implementation is not practical due to a large hidden constant factor.\\
The \emph{push-relabel} algorithm is one of the fastest maximum flow algorithms in practice
because there exist several speed-up techniques. The first one is
the \emph{global relabeling} heuristic which frequently updates the \emph{distance labels} by computing
the shortest path in the residual graph from all nodes to the sink \cite{cherkassky1997implementing}.
This can be done with a backward \emph{BFS} in linear time. This technique is performed periodically,
e.g., after $n$ \emph{relabel} operations. \\
The second heuristic is the \emph{gap heuristic} \cite{cherkassky1994fast,derigs1989implementing}.
If at a particular stage of the algorithm there is no node $u$ with $d(u) = g < n$, then for each node
$v$ with $g < d(v) < n$ the sink is not reachable anymore. Therefore, we can increase the \emph{distance
label} of all those nodes to $n$. To implement this heuristic, the algorithm maintains a linked list of
nodes with distance label $i$.

%%%%%%%%%%%%%%%%%%%%%%%%%%%%%%%%%%%%%%%%%%%%%%%%%%%%%%%%%%%%%%%%%%%%%%

\subsection{Modeling Flows on Hypergraphs}
\label{sec:related_lawler}

Finding a minimum $(s,t)$-cutset of a hypergraph \HypergraphDef~is close related to the problem of finding
a minimum $(s,t)$-vertex separator of the corresponding \emph{bipartite graph} representation $G_*(H)$ (see Section \ref{sec:hypergraph}).
Hu and Moerder \cite{HuMoerder85} introduce node capacities in $G_*(H)$. Each hyperedge
$e$ has a capacity equal to $\omega(e)$ and each hypernode has infinite capacity. 
Further, they show that a minimum-weight $(s,t)$-vertex separator in $G_*(H)$
is equal to a minimum-weight $(s,t)$-cutset of a hypergraph $H$. 
Finding such a separator is a flow problem and can be calculated with the flow network \ShortT{L} 
presented by Lawler \cite{lawler1973}:

\begin{definition}
Let $T_L$ be the transformation of a hypergraph \HypergraphDef~into 
a flow network \T{L} proposed by Lawler \cite{lawler1973}. \ShortT{L} is defined as follows:
\begin{enumerate}
\item $V_L = V \cup \bigcup\limits_{e \in E}\ \{\incoming{e}, \outgoing{e}\}$
\item $\forall e \in E$ we add a directed edge $(\incoming{e},\outgoing{e})$ 
      with capacity $\capa_L(\incoming{e},\outgoing{e}) = \omega(e)$
\item $\forall v \in V$ and $\forall e \in I(v)$ we add two directed edges $(v, \incoming{e})$ and 
      $(\outgoing{e}, v)$ with capacity $\capa_L(v, \incoming{e}) = \capa_L(\outgoing{e},v) = \infty$.
\end{enumerate} 
\end{definition}
\begin{figure}
\centering
\includegraphics[width=0.8\textwidth]{../img/network_transformation/lawler_transformation.eps}
\caption{Transformation of a hypergraph into the flow network \ShortT{L} \cite{lawler1973}. The
capacity of the black edges in the flow network is $\infty$.}
\label{img:lawler_transformation}
\end{figure}
An example of this transformation is shown in \autoref{img:lawler_transformation}.
\ShortT{L} is nearly equivalent to the transformation $T_V(G)$ described in Defintion \autoref{def:vertex_seperator_transformation}
except that we do not have to split the hypernodes $v \in V$.
For all $e \in E$ there exist two corresponding nodes $\incoming{e},\outgoing{e} \in V_L$. $\incoming{e}$ 
is called \emph{incoming hyperedge node} and $\outgoing{e}$ is called \emph{outgoing hyperedge node}. \\
A hypernode cannot be in a minimum-capacity $(s,t)$-vertex separator because each $v \in V$ has
infinity capacity \cite{HuMoerder85}. Therefore, a minimum-capacity $(s,t)$-cutset 
of \ShortT{L} is equal to a minimum $(s,t)$-vertex separator of $G_*(H)$.
The resulting graph \ShortT{L} has $|V_L| = |V| + 2|E|$ nodes and $|E_L| = (2\bar{e}+1)|E|$ edges, where
$\bar{e}$ is the average size of a hyperedge \cite{pistorius2003}. Using \emph{Edmond-Karps}
maximum flow algorithm (see Section \ref{sec:aug_path}) on flow network \ShortT{L} 
takes time \BigO{|V|^2|E|^2} \cite{lawler1973}. \\
A minimum-weight $(s,t)$-cutset of $H$ can be found by simply mapping the minimum-capacity
$(s,t)$-cutset to their corresponding hyperedges in $H$ (see Section \ref{sec:applications}). 
The minimum-weight $(s,t)$-bipartition consists of all vertices $v \in V$ \emph{reachable} from $s$ in the 
\emph{residual graph} of \ShortT{L} and all hypernodes not \emph{reachable}
from $s$. \\ 
In \autoref{img:lawler_augmenting_example} we illustrate the structure of \ShortT{L} and demonstrate
what happens after we augment along a path in the Lawler-Network. This figure can be used as
a reference to illustrate the proofs of Section \ref{sec:opt_flow_network}.\\
\begin{figure}[t!]
\centering
\includegraphics[width=0.75\textwidth]{../img/network_transformation/lawler_augmenting_example.eps}
\caption{Illustration of the structure of a hyperedge $e$ in \ShortT{L} and the effect of augmenting
         along a path in the network. The labeling on an edge represents the flow $f(x,y)$ and
         the capacity $\capacity{x,y}$ denoted with $f(x,y)/\capacity{x,y}$. The labeling of the
         edges in the residual graph corresponds to the residual capacity $r_f(x,y)$. The red highlighted
         path represents an augmenting path.}
\label{img:lawler_augmenting_example} 
\end{figure}

%%%%%%%%%%%%%%%%%%%%%%%%%%%%%%%%%%%%%%%%%%%%%%%%%%%%%%%%%%%%%%%%%%%%%%

\subsection{Flow-based Refinement for Graph Partitioning}
\label{sec:flow_local_search_graph} 

It seems natural to utilize maximum flow computations to improve the cut metric of a 
given partition of a graph. Lang and Rao \cite{lang2004flow} use an approach,
called \emph{Max-Flow Quotient-cut Improvement} (MQI) to improve the quality
of a graph partition when metrics such as \emph{expansion} or \emph{conductance}
are used. For a given bipartition $(S,\bar{S})$, they find the best 
improvement among all bipartitions $(S',\bar{S'})$ such that $S' \subset S$
by solving a flow problem. Andersen and Lang \cite{andersen2008algorithm}
suggested a flow-based improvement algorithm, called \emph{Improve},
which works similar as MQI, but does not restrict the output of the 
partition to $S' \subset S$. However, both techniques can not guarantee 
that the resulting bipartition is balanced and are only applicable for $k=2$. \\
Schulz and Sanders \cite{sanders2011engineering} integrate a flow-based refinement algorithm 
in the \emph{multilevel graph partitioner} \emph{KaFFPa}. Their basic idea is
to extract a region $B$ around the cut of the graph and connect the \emph{border} 
of $B$ with the source resp. sink. $B$ is defined in such a way that the flow computation
yields a feasible cut according to the \emph{balance contraint}. Many ideas of this work are used in this
thesis and adapted to hypergraphs. Therefore we will give a detailed description
of the concepts and advanced techniques to improve graph partitions.

\subsubsection{Balanced Bipartitioning}
\label{sec:balanced_bipartitioning}
\begin{figure}
\centering
\includegraphics[width=0.6\textwidth]{../img/flow_local_search/balanced_bipartitioning.eps}
\caption{Configuration of a flow problem around the cut of graph $G$ \cite{sanders2011engineering}.}
\label{img:balanced_bipartition}
\end{figure}
Let $(V_1,V_2)$ be a balanced bipartition of a graph $G = (V,E,c,\omega)$. 
Further, let $P(v) = 1$, if $v \in V_1$ and
$P(v) = 2$ otherwise. We will now explain how a given bipartition 
can be improved with flow computations. This technique can also be applied on a $k$-way 
partition by applying the approach on two adjacent blocks \cite{sanders2011engineering}. \\
Let $\delta := \{ u\ |\ \exists (u,v) \in E: P(u) \neq P(v) \}$ be the set of nodes
around the cut of $G$. For a set $B \subseteq V$ we define its border 
$\delta B := \{u \in B\ |\ \exists (u,v) \in E: v \notin B\}$.
The basic idea is to build a region $B$ around $\delta$
and connect all nodes in $\delta B \cap V_1$ to the source node $s$ and all nodes in 
$\delta B \cap V_2$ to the sink node $t$. \\
We can construct $B := B_1 \cup B_2$ with two \emph{Breadth First Searches} (\BFS). 
One is initialized with all nodes in $\delta \cap V_1$ and stops if $c(B_1)$ would 
exceed $(1+\epsilon) \lceil \frac{c(V)}{2} \rceil - c(V_2)$. The second is initialized with 
all nodes in $\delta \cap V_2$ and stops if $c(B_2)$ would exceed 
$(1+\epsilon) \lceil \frac{c(V)}{2} \rceil - c(V_1)$. The two \BFS s only touch nodes of $V_1$ resp. $V_2$
such that $B_1 \subseteq V_1$ and $B_2 \subseteq V_2$. The constraints for the weights of $B_1$
and $B_2$ guarantee that the bipartition is still balanced after a \emph{Max-Flow-Min-Cut}
computation. Connecting $s$ resp. $t$ to all border nodes $\delta B \cap V_1$ resp.
$\delta B \cap V_2$ ensures that a non-cut edge not contained in $G_B$ is not a cut edge after
assigning the minimum $(s,t)$-bipartition of subgraph $G_B$ to $G$. Consequently,
each minimum $(s,t)$-cutset in $G_B$ leads to a cut smaller or 
equal to the old cut of $G$. All concepts are illustrated in \autoref{img:balanced_bipartition}.


\subsubsection{Adaptive Flow Iterations}
\label{sec:adaptive_flow_iterations}

Sanders and Schulz \cite{sanders2011engineering} introduce several techniques to improve
this basic approach. If the \emph{Max-Flow-Min-Cut} computation on $G_B$ leads to an
improved cut, we can apply the method again. 
An extension of this approach is to iteratively adapt the
size of the flow problem based on the result of the maximum flow computation.
We define $\epsilon' := \alpha\epsilon$ for a $\alpha \ge 1$ and let the size of $B$ depend
on $\epsilon'$ rather than on $\epsilon$. If we find an improvement, we
increase $\alpha$ to $\min\{2\alpha, \alpha'\}$ where $\alpha'$ is a predefined upper bound
for $\alpha$. If not, we decrease the size of $\alpha$ to 
$\max\{\frac{\alpha}{2},1\}$. This approach is called
\emph{adaptive flow iterations} \cite{sanders2011engineering}.

\subsubsection{Most Balanced Minimum Cut}
\label{sec:related_mbmc}

Picard and Queyranne \cite{picard1980structure} show that all minimum $(s,t)$-cutsets 
are computable with one maximum $(s,t)$-flow computation.
An important concept used by them is the definition
of a \emph{closed node set} $C \subseteq V$ of a graph $G$.

\begin{definition}[Closed Node Set]
Let $G = (V,E)$ be a graph and $C \subseteq V$. $C$ is called a closed node set iff the 
condition $u \in C$ implies that for all edges $(u,v) \in E$ also $v \in C$.
\end{definition}

A \emph{closed node set} is illustrated in \autoref{img:closed_node_set}. A simple observation
is that all nodes on a cycle have to be in the same \emph{closed node set} per definition. Therefore
we can contract all \emph{Strongly Connected Components} (SCC) of $G$ with a linear time algorithm
proposed by Tarjan \cite{tarjan1972depth} and sweep over the contracted graph in reverse
topological order to enumerate all \emph{closed node sets}. If
we contract all SCCs of $G$ the resulting graph is a \emph{Directed Acyclic Graph} (DAG). Therefore, 
a topological order exists. \\
With the Theorem of Picard and Queyranne \cite{picard1980structure} we can enumerate
all minimum $(s,t)$-cuts of $G$ with one maximum flow computation.

\begin{theorem}
\label{theorem:mbmc}
There is a $1$-$1$ correspondence between the minimum $(s,t)$-cuts of a graph and the closed node
sets containing $s$ in the residual graph of a maximum $(s,t)$-flow.
\end{theorem}

All \emph{closed node sets} in the residual graph of $G$ induce a minimum $(s,t)$-cutset on $G$.
They can be calculated with the algorithm described above using the residual graph of
$G$ as input. The running time of the algorithm is \BigO{|V| + |E|}. \\
A common problem of the \emph{adaptive flow iteration} approach (see Section \ref{sec:adaptive_flow_iterations}) is
that using a large $\alpha$ often leads to cuts in $G$ that violate the balanced constraint. 
We can enumerate all minimum $(s,t)$-cutsets with one maximum 
flow computation and therefore have a higher probability to find
a feasible partition after a \emph{Max-Flow-Min-Cut} computation. We refer to this method as
\emph{Most Balanced Minimum Cut} \cite{sanders2011engineering}.
\begin{figure}
\centering
\includegraphics[width=0.9\textwidth]{../img/flow_local_search/closed_node_set.eps}
\caption{$C = \{s,a,b,c\}$ is a \emph{closed node set} of graph $G$ (left side).
         After contracting all \emph{Strongly Connected Components}, we can enumerate all
         \emph{closed node sets} of $G$ by sweeping over the contracted graph 
         in reverse topological order (right side).}
\label{img:closed_node_set}
\end{figure}


\subsubsection{Active Block Scheduling}
\label{sec:abs}
\emph{Active Block Scheduling} is a \emph{quotient graph style refinement} technique for
$k$-way partitions \cite{holtgrewe2010engineering,sanders2011engineering}. 
The algorithm is organized in rounds and executes a two-way 
local improvement algorithm on each pair of adjacent
blocks in the \emph{quotient graph} where at least one of both is \emph{active}. 
Initially all blocks are \emph{active}. A block becomes \emph{inactive}
if none of its nodes move in a round. The algorithm
terminates, if all blocks are \emph{inactive}. \\
Fiduccia and Mattheyses \cite{fiduccia1988linear} introduce a linear time
two-way local search heuristic, called \emph{FM} heuristic, 
which is fundamental for many graph partitioning algorithms.
They define the gain $g(v)$ of a node $v \in V$ as the reduction of the cut metric when
moving $v$ from its current block to the other block. By maintaining the gains of the
nodes in a special data structure, called \emph{bucket queue}, they can find a maximum
gain node in constant time. After moving a maximum gain node, they are also able to update the
data structure in time equal to the number of adjacent nodes.\\
The local improvement algorithm (for \emph{Active Block Scheduling}) can either 
be an \emph{FM} local search or a flow-based approach or even a combination of 
both as proposed by Sanders and Schulz \cite{sanders2011engineering}.

%%%%%%%%%%%%%%%%%%%%%%%%%%%%%%%%%%%%%%%%%%%%%%%%%%%%%%%%%%%%%%%%%%%%%%

\subsection{Hypergraph Partitioning}

In this section, we review how most hypergraph partitioners solve the \emph{hypergraph
partitioning problem}. The most successful
approach is the \emph{multilevel paradigm} \cite{alpert1995recent,bader2013graph,
papa2007hypergraph} which we describe in Section \ref{sec:multilevel_paradigm}.
The algorithms presented in this thesis are integrated into $n$-level hypergraph partitioner \emph{KaHyPar}. Therefore, we
give a brief overview of implementation details of this framework in Section \ref{sec:kahypar}.

\subsubsection{Multilevel Paradigm}
\label{sec:multilevel_paradigm}

The \emph{multilevel paradigm} is a three phase algorithm to solve the \emph{hypergraph 
partitioning problem} (see \autoref{img:multilevel_paradigm}). In the first stage, called
\emph{coarsening phase}, vertex matchings or clusterings are calculated which are contracted. This process is
repeated until a predefined number of hypernodes remains. The sequence of successively
smaller hypergraphs is called \emph{levels}. If the hypergraph $H$ is small enough, expensive 
algorithms can be used to initially partition $H$ into $k$ blocks (\emph{Initial Partitioning}). Afterwards, we
\emph{uncontract} each \emph{level} in reverse order of \emph{contraction} by projecting
the partition to the next \emph{level}. After \emph{uncontraction} a \emph{refinement} heuristic can be
used to improve the quality of the current partition according
to an objective function. The most commonly used \emph{refinement} algorithm is the \emph{FM}
algorithm \cite{fiduccia1988linear}.

\begin{figure}
\centering
\includegraphics[width=1.0\textwidth]{../img/hypergraph_partitioning/multilevel.eps}
\caption{Multilevel Hypergraph Partitioning}
\label{img:multilevel_paradigm}
\end{figure}


\subsubsection{$n$-Level Hypergraph Partitioning}
\label{sec:kahypar}

\emph{KaHyPar} is a multilevel hypergraph partitioner in its most extreme version, which
removes only a single vertex in one \emph{level} of the hierarchy. It seems to be the method
of choice for optimizing cut- and the $(\lambda - 1)$-metric unless speed is more important than
quality \cite{heuer2017improving}. The framework provides a \emph{direct $k$-way} \cite{akhremtsev2017engineering} 
and a \emph{recursive bisection} mode, which recursively calculates bipartitions 
(with multilevel paradigm) until the hypergraph is divided into $k$ blocks 
\cite{schlag2016k}. \emph{KaHyPar} consists of four phases: \emph{Preprocessing}
and the three phases of the \emph{multilevel paradigm}. \\
In the \emph{preprocessing} step community structures of the hypergraph are detected. The
hypergraph is transformed into a bipartite graph $G_*(H)$ (see Section \ref{sec:hypergraph}) and
a community detection algorithm is executed which optimizes \emph{modularity} \cite{fortunato2010community,
heuer2017improving}. During the \emph{coarsening phase} contractions are restricted to vertices within the same 
community. The contraction partners are chosen according to the \emph{heavy-edge} rating function
$r(u,v) := \sum_{e \in I(u) \cap I(v)} \frac{\omega(e)}{|e|-1}$ \cite{karypis1999multilevel}. The
function prefers vertices which share a large number of heavy nets with small size. The contraction
algorithm works in passes. At the beginning of each pass a random permutation of the vertices 
is generated and for each vertex $u$, the contraction partner $v$ is determined according
to the \emph{heavy-edge} rating function \cite{schlag2016k}. A pass ends if each vertex was involved
in a contraction. The passes are repeated until only
$t = 160k$ hypernodes remains.
The \emph{initial partitioning} phase uses the \emph{recursive bisection} approach to calculate
a $k$-way partition in combination with a portfolio of initial partitioning techniques 
\cite{heuer2015engineering}. In the \emph{refinement phase}, a localized \emph{FM} search is started \cite{fiduccia1988linear},
initialized with the current uncontracted vertices. The \emph{local search} maintains $k$ \emph{priority
queues} (PQ) for each block $V_i$ exactly one \cite{akhremtsev2017engineering}. After a move,
the gains of all adjacent hypernodes are updated with a \emph{delta-gain} update strategy \cite{papa2007hypergraph}.
The recalculation of all gain values at the beginning of a \emph{FM} pass is one of the main bottlenecks
of the algorithm \cite{papa2007hypergraph}. Therefore, Schlag et al. \cite{akhremtsev2017engineering,
schlag2016k} introduce a \emph{gain cache}, which prevents
expensive recalculations of the corresponding gain function. The \emph{gain cache} is maintained
with \emph{delta-gain} updates in the same way as the \emph{PQs}. Further, the \emph{local search}
is stopped as soon as an improvement during an \emph{FM} pass becomes unlikely. This model is
called \emph{adaptive stopping rule} \cite{akhremtsev2017engineering}.


\newpage

%%%%%%%%%%%%%%%%%%%%%%%%%%%%%%%%%%%%%%%%%%%%%%%%%%%%%%%%%%%%%%%%%%%%%%

\section{Optimized Approach on Modelling Flows in Hypergraphs}
\label{sec:opt_flow_network}

In Section \ref{sec:related_lawler} we have shown how a hypergraph $H$ could be transformed into 
a flow network \ShortT{L} such that every minimum-weight $(S,T)$-cutset in $H$ is a
minimum-capacity $(S,T)$-cutset in \ShortT{L} \cite{lawler1973}. However, the resulting flow
network has significantly more nodes and edges than the original hypergraph. Finding a
maximum $(S,T)$-flow is usually a very computation intensive task. 
Therefore, different modeling approaches, which reduce the number of nodes and edges,
can have a crucial impact on the running time of the flow algorithm. \\
We will present techniques to sparsify the flow network 
proposed by Lawler. First, we will show how any subset $V' \subseteq V$ of hypernodes could be removed 
from \ShortT{L} (see Section \ref{sec:heuer_network}). This approach minimizes
the number of nodes, but in some cases, the number of edges can be
significantly higher than in \ShortT{L}. But the basic idea of this technique 
can still be applied to remove low degree hypernodes from the \emph{Lawler}-Network without 
increasing the number of edges (see Section \ref{sec:degree_network}). Additionally, we show
how every hyperedge $e$ of size $2$ could be removed by inserting an undirected flow edge between
the corresponding nodes $v_1,v_2 \in e$  (see Section \ref{sec:edge_size_network}). 
Finally, we combine the two suggested approaches in a \emph{Hybrid}-Network 
(see Section \ref{sec:hybrid_network}).



\subsection{Removing Hypernodes via Clique-Expansion}
\label{sec:heuer_network}

In this Section, we show how all hypernodes of \ShortT{L} could be removed. If a hypernode $v \in V$
occurs in an augmenting path $P$ the previous node in the path must be a hyperedge node either
$e'$ or $e''$. Further, for all $e \in I(v)$ the capacity $c_L(v,e')$ is $\infty$. Therefore, 
if we push flow over a hypernode $v$, coming from a hyperedge node, we can redirect
the flow to any hyperedge node $e' \in I(v)$ during the whole maximum flow calculation, because 
$c_L(v,e') = \infty$. The following lemma is central to our first sparsifying 
technique. We will denote all incoming edges of a node $u$ with
$in(u) := \{v\ |\ (v,u) \in E\}$ and all outgoing edges with 
$out(u) := \{v\ |\ (u,v) \in E\}$. 

\begin{lemma}
\label{lemma:node_removal}
Let $G = (V,E,c)$ be a flow network and $u \in V$ a node where
$\forall v \in in(u): c(v,u) = \infty$ and $\forall w \in out(u): c(u,w) = \infty$.
Further, let $G(u) = (V\setminus\{u\}, E_u, c_u)$ be the flow network obtained by removing
$u$ and insert a directed edge between each $v \in in(u)$ and $w \in out(u)$ with $c_u(v,w) = \infty$.
Let $f$ be a maximum $(S,T)$-flow of G and $f'$ a maximum $(S,T)$-flow of
$G(u)$.
\[|f| < \infty \text{ and } u \notin S\cup T \Rightarrow |f| = |f'|\]
\end{lemma}

\begin{proof}
Let $f$ be a maximum $(S,T)$-flow of $G$. We define a maximum $(S,T)$-flow $f'$ 
of $G(u)$ as follows:
\[ 
f'(v,w) =  
  \begin{cases}
      \frac{f(v,u)f(u,w)}{\sum_{w \in out(u)} f(u,w)}, & \text{if } v \in in(u), w \in out(u) \\
      f(v,w), & \text{otherwise}
   \end{cases} 
\]
$f'$ is choosen in such a way that for all $v \in in(u): \sum_{w \in out(u)} f'(v,w) = f(v,u)$
and for all $w \in out(u): \sum_{v \in in(u)} f'(v,w) = f(u,w)$. Therefore, $f'$
satisfies the flow conservation constraint and is an valid flow function. Since
$u \notin S\cup T$, it follows that $|f| = |f'|$. \\
Let $f'$ be a maximum $(S,T)$-flow of $G(u)$. We define a maximum $(S,T)$-flow $f$
of $G$ as follows:
\[
f(v,w) =  
  \begin{cases}
      \sum_{x \in in(u)} f'(x,w), & \text{if } v = u \\
      \sum_{x \in out(u)} f'(v,x), & \text{if } w = u \\
      f'(v,w), & \text{otherwise}
   \end{cases} 
\]
The amount of flow from each $v \in in(u)$ to each $w \in out(u)$ of flow function
$f'$ is redirected over $u$ in $f$. Therefore, $f$ is an valid flow function.
Since $v \notin S\cup T$, it follows that $|f| = |f'|$.
\end{proof}

In \ShortT{L} all incoming and outgoing edges of a hypernode $v$ have a
capacity equal to $\infty$. The incoming edges are all $e'' \in I(v)$ and
the outgoing edges are all $e' \in I(v)$. Therefore, we can construct the
following network with Lemma \ref{lemma:node_removal}:

\begin{definition}
Let $T_H$ be a transformation that converts a hypergraph \HypergraphDef~into 
a flow network \ExtendedT{H}{V'} with $V' \subseteq V$. \ShortExT{H}{V'} is defined as follows:
\begin{enumerate}
\item $V_H = V\setminus V' \bigcup\limits_{e \in E}\ \{e', e''\}$
\item $\forall v \in V'$ we add a directed edge $(e_1'', e_2'),\ \forall e_1, e_2 \in I(v)$ 
      with $e_1 \neq e_2$ with capacity $c_H(e_1'', e_2') = \infty$ (Lemma \ref{lemma:node_removal}).
\item $\forall e \in E$ we add a directed edge $(e',e'')$
      with capacity $c_H(e',e'') = \omega(e)$ (same as in \ShortT{L}).
\item $\forall v \in V\setminus V'$ we add for each incident hyperedge $e \in I(v)$ two directed
      edges $(v,e')$ and $(e'',v)$ with capacity 
      $c_H(v,e') = c_H(e'',v) := \infty$ (same as in \ShortT{L}).
\end{enumerate} 
\end{definition}

An example of the transformation is shown in \autoref{img:heuer_network}.
We have to proof that a minimum-capacity $(S,T)$-cutset
of \ShortExT{H}{V'} is equal with a minimum-weight $(S,T)$-cutset of $H$. However,
we need a preparing lemma in the correctness proof.

\begin{figure}
\centering
\includegraphics[width=0.8\textwidth]{../img/network_transformation/heuer_transformation.eps}
\caption{Transformation of a hypergraph into an equivalent flow network by removing
         all hypernodes. Note, capacity of the black edges in the flow network is $\infty$.}
\label{img:heuer_network}
\end{figure}

\begin{lemma}
\label{lemma:source_and_sink_removal}
Let $G = (V,E,c)$ be a flow network and $f$ a maximum $(S,T)$-flow of $G$.
Further, let $s \in S$ be a source node with $\forall v \in out(s): c(s,v) = \infty$
and $t \in T$ be a sink node with $\forall v \in in(t): c(v,t) = \infty$.
$f_s$ is a maximum $(S',T)$-flow of $G(s)$ and $f_t$ is a maximum $(S,T')$-flow
of $G(t)$ with $S' = S\setminus \{s\} \cup out(s)$ and $T' = T \setminus \{t\}
\cup in(t)$.
\[|f| < \infty \Rightarrow |f| = |f_s| = |f_t|\]
\end{lemma}

\begin{proof}
In Section \ref{sec:applications} we have described how to solve a \emph{multi-source 
multi sink} flow problem by adding a super source node $a$ and super sink node $b$ to the network
and connect $a$ with all sources $s' \in S$ and all sinks $t' \in T$ with $b$.
For $s' \in S: c(a,s') = \infty$ and for $t' \in T: c(t',b) = \infty$. With Lemma
\ref{lemma:node_removal} follows, that we can remove $s$ from $G$ and insert
a directed edge from $a$ to each $v \in out(s)$ (equal to $G(s)$) and $|f| = |f_s|$. The new flow problem
corresponds to the \emph{multi-source multi-sink} problem with $S'$ and $T$ as source
and sink set. The proof for $G(t)$ is equivalent.
\end{proof}

As a consequence of this lemma, we could replace (or even remove) 
e.g. a source hypernode $v \in S$ of \ShortT{L} and instead add all
incoming hyperedge nodes $e' \in I(v)$ as source nodes to the flow 
problem. Because for all incoming resp. outgoing edges of vertices $v$ of 
\ShortT{L} the capacity is $\infty$.

\begin{theorem}
\label{theorem:st_cutset_equal}
A minimum-weight $(S,T)$-cutset of a hypergraph \HypergraphDef~(with $S,T \subseteq V,
S \cap T = \emptyset$) is equivalent with a minimum-capacity $(S',T')$-cutset of the
flow network \ShortExT{H}{V'} ($V' \subseteq V$) with $S' = S \setminus V'\ \cup \bigcup\limits_{e \in I(V' \cap S)} \{e'\}$ and 
$T' = T \setminus V'\ \cup \bigcup\limits_{e \in I(V'\cap T)} \{e''\}$.
\label{theorem:heuer_network}
\end{theorem}

\begin{proof}

Applying Lemma \ref{lemma:node_removal} and \ref{lemma:source_and_sink_removal}
on all nodes $v \in V'$ of flow network \ShortT{L} yields to network
\ShortExT{H}{V'} with $S'$ and $T'$ as source and sink sets. A maximum
$(S,T)$-flow $f_L$ of \ShortT{L} is then equal with a maximum $(S',T')$-flow $f_H$
of \ShortExT{H}{V'}. Since $|f_L| < \infty$, only edges between hyperedge 
nodes are contained in a minimum $(S,T)$-cutset of \ShortT{L}. Since $|f_L| = |f_H|$, the same
holds for a minimum $(S',T')$-cutset $E_{min}$ of \ShortExT{H}{V'}. Therefore, $E_{min}$
is equal with a minimum-weight $(S,T)$-cutset of $H$.

%Consider the bipartite graph representation $G_* = (V_*,E_*,c_*)$ 
%of a hypergraph $H = (V,E,c,\omega)$ presented in Section \ref{sec:hypergraph} and \ref{sec:related_lawler}, 
%where for all $v \in V: c_*(v) = \infty$ and for all $e \in E: c_*(e) = 
%\omega(e)$. A minimum-weight $(S,T)$-vertex separator in $G_*$ is equal
%with a minimum-weight $(S,T)$-cutset in $H$. A minimum-weight $(S,T)$-vertex separator can be calculated
%by finding a minimum-capacity $(S,T)$-cutset in \ShortT{L}. Let $G_H$ be the graph obtained by removing
%all $v \in V'\setminus (S \cup T)$ of $G_*$ and adding a clique between all $e \in I(v)$. 
%A minimum-weight $(S,T)$-vertex separator in $G_H$ can be calculated by finding a 
%minimum-capacity $(S,T)$-cutset in our new network \ShortExT{H}{V'\setminus (S \cup T)}.
%We will show that each vertex separator in $G_*$ is also a vertex separator in $G_H$ and
%vice versa. With Lemma \ref{lemma:lemma1} we can show that each minimum $(S,T)$-cutset
%of \ShortExT{H}{V'\setminus (S \cup T)} is equal with a minimum $(S',T')$-cutset of
%\ShortExT{H}{V'} and conclude the proof. We will denote a vertex separator of a graph $G$ with 
%$\V{G}$ and define $V'' := V' \setminus (S \cup T)$. 
%We will show, that $\V{G_*} = \V{G_H}$ with the restriction $\V{G_*} \subseteq E$ and
%$\V{G_H} \subseteq E$.
%
%Assume that $\V{G_*} \subseteq E$ is not a vertex separator in $G_H$. After removing all $e \in \V{G_*}$ of
%$G_H$, there exists still a path $P_H = \{v_1, \ldots, v_k\}$ with $v_1 \in S$ and
%$v_k \in T$ of $G_H$. We can extend $P_H$ to a path $P_*$ in $G_*$.
%We define $P_* := P_H$ and replaces every occurence of a sequence $v_i = e_1 \in E$ and
%$v_{i+1} = e_2 \in E$ with a triple $(e_1,v,e_2)$ in $P_*$, where $v \in e_1 \cap e_2 \cap V''$
%(not empty per construction). $P_*$ does not contain a vertex of $\V{G_*}$, because
%we removed all hyperedge nodes $e \in \V{G_*}$ from $G_H$ before construction of $P_*$ 
%and a hypernode is not part of  the vertex separator $\V{G_*} \subseteq E$ per defintion. 
%$P_*$ connects $S$ and $T$ in $G_*$, which is a contradiction that $\V{G_*}$ 
%is a vertex separator in $G_*$.
%
%Assume that $\V{G_H} \subseteq E$ is not a vertex separator in $G_*$. After removing all $e \in \V{G_H}$ of
%$G_*$, there exists still a path $P_* = \{v_1, \ldots, v_k\}$ with $v_1 \in S$ and
%$v_k \in T$ of $G_*$. We can extend $P_*$ to a path $P_H$ in $G_H$.
%We define $P_H := P_*$ and remove all $v \in P_* \cap V''$ from $P_H$. $G_*$ is a bipartite
%graph per definition. Therefore, each path $P_*$ in $G_*$ is an alternating path of hypernodes and
%hyperedges. The predecessor and successor of a hypernode $v \in P_* \cap V''$ must be hyperedges
%$e_1$ and $e_2$. If $v \in V''$, then $v$ is not contained $G_H$. Instead, there is
%a clique between all $e \in I(v) \Rightarrow$ $(e_1,e_2)$ is contained in $G_H$.
%$P_H$ not contain any vertex of $\V{G_H}$, because we removed all hyperedge nodes $e \in \V{G_H}$ 
%from $G_*$. $P_H$ connects $S$ and $T$ in $G_H$, which is a contradiction that 
%$\V{G_H}$ is a vertex separator in $G_H$.
%
%A minimum-weight $(S,T)$-vertex separator in $G_*$ and $G_H$ contains only hyperedges, because
%the weight of all hypernodes in $G_*$ and $G_H$ is $\infty$. Therefore, each minimum-weight
%$(S,T)$-vertex separator in $G_*$ is also a minimum-weight $(S,T)$-vertex separator in $G_H$,
%because $c(\V{G_*}) = c(\V{G_H})$. With Lemma \ref{lemma:lemma1} follows that
%a minimum-weight $(S,T)$-vertex separator in $G_*$ resp. $G_H$  can also be 
%calculated by finding a minimum-capacity $(S',T')$-cutset in \ShortT{L} resp. \ShortExT{H}{V'}. 
%Therefore, there exists a equivalence between a minimum-weight $(S,T)$-cutset 
%$E_{min}$ of $H$ and the following statements: 
%
%$E_{min}$ is a minimum$\ldots$
%\begin{enumerate}
%\item $\ldots$-weight $(S,T)$-cutset in $H$
%\item $\ldots$-weight $(S,T)$-vertex separator in $G_*$
%\item $\ldots$-capacity $(S,T)$-cutset in \ShortT{L}
%\item $\ldots$-capacity $(S',T')$-cutset in \ShortT{L} (follows from (iii) with Lemma \ref{lemma:lemma1})
%\item $\ldots$-weight $(S,T)$-vertex separator in $G_H$
%\item $\ldots$-capacity $(S,T)$-cutset in \ShortExT{H}{V''}
%\item $\ldots$-capacity $(S',T')$-cutset in \ShortExT{H}{V'} (follows from (vi) with Lemma \ref{lemma:lemma1})
%\end{enumerate}

\end{proof}

Consequently, we can find a minimum-weight $(S,T)$-cutset of $H$ by calculating
a minimum-capacity $(S',T')$-cutset of \ShortExT{H}{V'}. An open problem is how to obtain the 
corresponding minimum-weight $(S,T)$-bipartition. In \ShortT{L} all hypernodes
reachable from source nodes in the residual graph are part of the first and 
all not reachable are part of the second block of the bipartition. Since we removed 
all hypernodes $v \in V'$ in our new network, we have to reconstruct the bipartition
with the following lemma.

\begin{lemma}
\label{lemma:bipartition_construction}
Let $f$ be a maximum $(S,T)$-flow  of \ShortT{L} and $A$ be the set of all nodes reachable
from a node $s \in S$ in the residual graph.
\[ \text{If } v \in A \Leftrightarrow \exists e \in I(v): e'' \in A \]
\end{lemma}

\begin{proof}
If $e'' \in A$, then $v \in A$, because $c_L(e'',v) = \infty$ and $r_f(e'',v) = \infty$.
Assume, if $v \in A$, then $\forall e \in I(v): e'' \notin A \Rightarrow f(e'',v) = 0$ (Note, $c(e'',v) = \infty$). 
Otherwise $r_f(v, e'')$ would be greater than zero and this would imply that $e'' \in A$,
because $v \in A$. Each path $P$ in the \emph{residual graph} of \ShortT{L} from $s \in S$ to $v$ must be of the 
form $P = (s,\ldots,e',v)$. For at least one $e \in I(v)$ there must be a positive flow $f(v,e') > 0$,
otherwise edge $(e',v)$ would be not contained in the \emph{residual graph} of \ShortT{L} (Note, $c_L(e',v) = 0$).
There is a positive flow leaving node $v$, but there is no flow entering node $v$, because
$\forall e \in I(v): f(e'',v) = 0$. This violates the conservation of flow
constraint for node $v$ and therefore $f$ is not a valid flow function. There must exist at least one $e \in I(v)$
with $f(e'',v) > 0 \Rightarrow r_f(v,e'') > 0 \Rightarrow e'' \in A$.
\end{proof}

Lemma \ref{lemma:bipartition_construction} gives us an alternative construction for the minimum-weight $(S,T)$-bipartition
of $H$ for both networks \ShortT{L} and \ShortExT{H}{V'}. Regardless of the flow network, we can 
calculate a maximum flow on it and define the set $E''$, which contains all \emph{outgoing hyperedge
nodes} $e''$ \emph{reachable} from a source node $s \in S$ in the \emph{residual graph} of the flow network. 
Further, $(A := \bigcup_{e \in E''} e,\ V\setminus A)$ is a 
minimum-weight $(S,T)$-bipartition of $H$.


\subsection{Low-Degree Hypernodes}
\label{sec:degree_network}

The resulting flow network \ShortExT{H}{V} proposed in Section \ref{sec:heuer_network} has significantly
fewer nodes than the network \ShortT{L} suggested by Lawler. On the other hand, the number of
edges could be much larger. \\
Let's consider a hypernode $v \in V$. We replace $v$ in \ShortT{L} with a clique between all
hyperedges of $I(v)$. The number of edges added to \ShortExT{H}{V} depends on the degree of
$v$. Every hypernode $v \in V$ induce $d(v)(d(v) - 1)$ edges in \ShortExT{H}{V}. In \ShortT{L} a hypernode adds $2d(v)$ edges to the network with the drawback
of an additional node. A simple observation is that for all hypernodes with $d(v) \le 3$ the inequality
$d(v)(d(v) - 1) \le 2d(v)$ holds. Removing such low degree hypernodes not only reduce
the number of nodes, but also the number of edges. \\
Let $V_{d}(n) = \{v \in V\ |\ d(v) \le n\}$ be the set of all hypernodes
with degree smaller or equal $n$. Then our suggested flow network is \ShortExT{H}{V_d(3)}.

\subsection{Removing Graph Hyperedges}
\label{sec:edge_size_network}

If we want to find a minimum-weight $(S,T)$-cutset in a graph $G = (V,E,\omega)$, we do not have to transform
$G$ into an equivalent flow network. We can directly operate on the graph with capacities
$c(e) = \omega(e)$ for all $e \in E$ \cite{ford1956maximal}. Hypergraphs are generalizations of a graph, where
an edge can consist of more than two nodes. However, a hyperedge $e$ of size $2$ can still be 
interpreted as a graph edge. Instead of modelling those edges as described by Lawler \cite{lawler1973}
(see hyperedge $e_2$ in \autoref{img:lawler_transformation}), we can remove all $e',e''$ for all $e \in E$
with $|e| = 2$ and add an undirected flow edge between $v_1,v_2 \in e$ (with $v_1 \neq v_2$) with
capacity $c(\{v_1,v_2\}) = \omega(e)$.

\begin{definition}
Let $T_G$ be a transformation that converts a hypergraph \HypergraphDef~into 
a flow network \T{G}. \ShortT{G} is defined as follows:
\begin{enumerate}
\item $V_G = V \cup \bigcup\limits_{\substack{e \in E \\ |e| \neq 2}}\ \{e', e''\}$
\item $\forall e \in E$ with $|e| = 2$ and $v_1,v_2 \in e$ ($v_1 \neq v_2$) we add 
      two directed edges $(v_1,v_2)$ and $(v_2,v_1)$ to $E_G$ with capacity $c(v_1,v_2) = \omega(e)$
      and $c(v_2,v_1) = \omega(e)$
\item Let $H' = (V,E',c,\omega)$ be the hypergraph with $E' = \{e\ |\ e \in E \land |e| \neq 2\}$,
      then we add all edges of $T_L(H')$ to $E_G$ with their corresponding capacities.
\end{enumerate} 
\end{definition}

An example of transformation \ShortT{G} is shown in \autoref{img:graph_transformation}. A hyperedge
$e$ of size $2$ consists exactly of $4$ nodes and $5$ edges in \ShortT{L} (see \autoref{img:lawler_transformation}).
The same hyperedge induces $2$ nodes and $2$ edges in \ShortT{G}. 
 
\begin{figure}
\centering
\includegraphics[width=0.8\textwidth]{../img/network_transformation/graph_transformation.eps}
\caption{Transformation of a hypergraph into an equivalent flow network by inserting 
         an undirected edge with capacity $\omega(e)$ for each hyperedge of size $2$. 
         Note, capacity of the black edges in the flow network is $\infty$.}
\label{img:graph_transformation}
\end{figure}

\begin{theorem}
\label{theorem:st_cutset_equal_graph}
A minimum-weight $(S,T)$-cutset of a hypergraph \HypergraphDef~(with $S,T \subseteq V,
S \cap T = \emptyset$) is equal with a minimum-capacity $(S,T)$-cutset of the
flow network \T{G}.
\label{theorem:heuer_network}
\end{theorem}

\begin{proof}

%Each path $P_L$ of \ShortT{L} from $s \in S$ to $t \in T$ can be mapped 
%bijective to a path $P_G$ of \ShortT{G} by replacing each occurence of $(\ldots,v_1,e',e'',v_2,\ldots)$ 
%with $|e| = 2$ and $v_1,v_2 \in e$ in $P_L$ with $(\ldots,v_1,v_2,\ldots)$ in $P_G$.
%Therefore, there exists a one-to-one correspondence between all paths from $S$
%to $T$ of \ShortT{L} and \ShortT{G}. Since $c_L(e',e'') = c_G(v_1,v_2)$, each augmenting path of \ShortT{L}
%can be mapped to a unique augmenting path of \ShortT{G} and vice versa. Executing an
%augmenting path maximum flow algorithm on \ShortT{L} and simultanousely mapping each augmenting path
%to \ShortT{G} yields to a maximum $(S,T)$-flow $f$ on both networks. 

We define the bijective function $\Phi: E_L \rightarrow E_G$ as follows
\[ \Phi(e',e'') = 
   \begin{cases}
      (e',e''), \text{if } |e| \neq 2, \\
      \{v_1,v_2\}, \text{otherwise (with $v_1,v_2 \in e$ and $v_1 \neq v_2$)}
   \end{cases} \]

We will show that each $(S,T)$-cutset $A_L$ of \ShortT{L} is a $(S,T)$-cutset $\Phi(A_L)$ of
\ShortT{G} and vice versa. Per defintion $c_L(A_L) = c_G(\Phi(A_L))$ and for each $(S,T)$-cutset
$A_G$ of \ShortT{G} $c_G(A_G) = c_L(\Phi^{-1}(A_G))$. Therefore, each minimum-capacity $(S,T)$-cutset
of \ShortT{L} must be a minimum-capacity $(S,T)$-cutset of \ShortT{G} and vice versa. In the following 
let $E^* = \bigcup_{e \in E} \{(e',e'')\}$. \\
Let $A_L \subseteq E^*$ be a $(S,T)$-cutset of \ShortT{L}. Assume $\Phi(A_L)$ is not a $(S,T)$-cutset
of \ShortT{G}. Then there exists a path
$P_G = \{v_1,\ldots,v_k\}$ connecting $S$ and $T$ in \ShortT{G} not containing any edge $e \in \Phi(A_L)$.
Let $P_L$ be the path in \ShortT{L} obtained by adding edge $\Phi^{-1}(v_i,v_{i+1}) = (e',e'')$ between all
$v_i \in V$ and $v_{i+1} \in V$ in $P_G$. $\Phi^{-1}(v_i,v_{i+1}) \notin A_L$, because
$P_G$ does not contain any edge of $\Phi(A_L)$. $P_L$ connects $S$ and
$T$ in \ShortT{L}, which is a contradiction to the assumption that $A_L$ is a $(S,T)$-cutset. \\
Let $A_G \subseteq \Phi(E^*)$ be a $(S,T)$-cutset in \ShortT{G}. Let's assume $\Phi^{-1}(A_G)$ is not
a $(S,T)$-cutset of \ShortT{L}. Then there exists a path $P_L = \{v_1,\ldots,v_k\}$ 
connecting $S$ and $T$ in \ShortT{L} not containing any
edge $e \in \Phi^{-1}(A_G)$. Let $P_G$ be the path in \ShortT{G} obtained by removing each edge
$(v_i,v_{i+1})$ with $v_i = e'$ and $v_{i+1} = e''$ and $|e| = 2$ from $P_L$. Based on the construction
of \ShortT{L} the predecessor of $v_i$ and successor of $v_{i+1}$ must be hypernodes $v_1,v_2 \in e$.
Therefore, $P_G$ is a valid path in \ShortT{G} connecting $S$ and $T$ and not contain any edge of 
$A_G$. This is a contradiction to the assumption that $A_G$ is a $(S,T)$-cutset.
\end{proof} 

A minimum-weight $(S,T)$-cutset of $H$ could also be calculated with \ShortT{G}. 
Each edge $(v_1,v_2)$ with $v_1,v_2 \in V$ of the minimum-capacity
$(S,T)$-cutset of \ShortT{G} can be mapped to their corresponding hyperedge
with $\Phi^{-1}(v_1,v_2)$. Since there exists a one-one correspondence between the hypernodes
of \ShortT{L} and \ShortT{G} the corresponding bipartition are all hypernodes \emph{reachable}
from all nodes in $S$ and all not \emph{reachable} from $S$ in the \emph{residual graph}
of \ShortT{G}. 

\subsection{Combining Techniques}
\label{sec:hybrid_network}

In many real-world instances, the average hyperedge size and hypernode degree are inversely
proportional to each other. E.g., if the number of hyperedges is significantly larger than the
number of hypernodes the average hypernode degree is usually much larger than $3$. Whereas
the average hyperedge size is often equal to $2$. If the number of hyperedges is nearly equal
to the number of hypernodes, the average hypernode degree is usually smaller or equal than $3$. Whereas
the average hyperedge size is often much larger than $2$. Of course, we can construct instances
where this inversely proportional relationship cannot be observed, but in many real-world instances,
we often find the described behavior. \\
Currently, we have two different modeling approaches which either perform better on low degree
hypernode instances or small hyperedge size instances. Taking our observation from real-world instances 
into account means that either \ShortT{G} or \ShortExT{H}{V_d(3)} performs significantly better on
a specific instance. It would be preferable to combine the two approaches into one network 
which performs on most instances best. \\

\begin{definition}
Let $T_{\text{Hybrid}}$ be a transformation that converts a hypergraph \HypergraphDef~into 
a flow network \Hybrid, where $V' = \{v \in V_d(3)\ |\ \forall e \in I(v): |e| \neq 2\}$. 
\ShortHybrid~is defined as follows:
\begin{enumerate}
\item $V_{\text{Hybrid}} = V\setminus V' \bigcup\limits_{\substack{e \in E \\ |e| \neq 2}}\ \{e', e''\}$
\item $\forall v \in V'$ we add a directed edge $(e_1'', e_2'),\ \forall e_1, e_2 \in I(v)$ 
      ($e_1 \neq e_2$) with capacity $c_{\text{Hybrid}}(e_1'', e_2') = \infty$ (clique expansion).
\item $\forall e \in E$ with $|e| = 2$ and $v_1,v_2 \in e$ ($v_1 \neq v_2$) we add 
      two directed edges $(v_1,v_2)$ and $(v_2,v_1)$ with capacity $c_{\text{Hybrid}}(v_1,v_2) = \omega(e)$
      and $c_{\text{Hybrid}}(v_2,v_1) = \omega(e)$
\item $\forall e \in E$ with $|e| \neq 2$ we add a directed edge $(e',e'')$
      with capacity $c_{\text{Hybrid}}(e',e'') = \omega(e)$ (same as in \ShortT{L}).
\item $\forall v \in V\setminus V'$ we add for each incident hyperedge $e \in I(v)$ with $|e| \neq 2$ 
      two directed edges $(v,e')$ and $(e'',v)$ with capacity 
      $c_{\text{Hybrid}}(v,e') = c_{\text{Hybrid}}(e'',v) := \infty$ (same as in \ShortT{L}).
\end{enumerate} 
\end{definition}

\autoref{img:transformation_chain} summarizes all explained transformations of this section.
The proof of \autoref{theorem:st_cutset_equal_graph} can be used one-to-one to show that a minimum-capacity
$(S',T')$-cutset of \ShortExT{H}{V'} is equal with a minimum-capacity $(S',T')$-cutset of \ShortHybrid~
(for definition of $S'$ and $T'$ see \autoref{theorem:st_cutset_equal}). It follows with Lemma \ref{lemma:source_and_sink_removal}
that this is equal with a minimum-weight $(S,T)$-cutset of $H$. \\
Per definition of \ShortHybrid~we prefer a hyperedge removal over a hypernode removal. E.g., if
a hypernode has a degree smaller or equal than $3$, we only remove it, if there is no hyperedge
$e \in I(v)$ with $|e| = 2$. The reason is that a hyperedge removal always decreases the number of nodes
and edges more than a hypernode removal. \\
\begin{figure}[ht!]
\centering
\includegraphics[width=0.95\textwidth]{../img/network_transformation/hybrid_transformation.eps}
\caption{Illustration of all presented transformations of a hypergraph into a flow network.}
\label{img:transformation_chain}
\end{figure}
The minimum-weight $(S,T)$-cutset of $H$ can be calculated with the same technique described in Section
\ref{sec:edge_size_network}. Let $(A,V\setminus A)$ be the corresponding bipartition.
$A$ is the union of all reachable hypernodes from $S'$ and the union of
all reachable \emph{outgoing hyperedge nodes} $e''$ from $S'$ (see Section \ref{sec:heuer_network} 
and Lemma \ref{lemma:bipartition_construction}). 


\newpage

%%%%%%%%%%%%%%%%%%%%%%%%%%%%%%%%%%%%%%%%%%%%%%%%%%%%%%%%%%%%%%%%%%%%%%

\section{Using Max-Flow-Min-Cut Computations as Refinement Strategy}
\label{sec:flow_refinement}

We will now give a detailed overview of our flow-based refinement framework. The main
idea is to extract a subhypergraph $H_{V'}$ out of a hypergraph $H$, which is already
partitioned into $k$ blocks. $V'$ is chosen in such way that it is a subset of two
adjacent blocks $V_i$ and $V_j$. We will show how to configure
the sources $S$ and sinks $T$ of the corresponding flow network such that
a minimum $(S,T)$-bipartition of $H_{V'}$ improves the connectivity metric of $H$
(see Section \ref{sec:source_and_sink}). Further, we describe how the ideas of
Sanders and Schulz \cite{sanders2011engineering} 
can be adapted to work in an $n$-level hypergraph partitioner, called \emph{KaHyPar}. 

\subsection{Source and Sink Configuration}
\label{sec:source_and_sink}

Let $H = (V,E,c,\omega)$ be a hypergraph and $B_1$ be a bipartition of $H$.
In the following, we show how to configure the source set $S$ and sink set $T$ of the flow
network $T_L(H_{V'})$ of a subhypergraph $H_{V'}$ induced by $V' \subseteq V$. The goal is 
to improve bipartition $B_1$ of $H$ with a maximum $(S,T)$-flow calculation 
on $T_L(H_{V'})$ (with $f$ as maximum flow) such that after inserting the minimum 
$(S,T)$-bipartition of $H_{V'}$ on $H$ the resulting bipartition $B_2$ of $H$ has a 
cut less than or equal to the cut of $B_1$.
Let $E_{\text{cut}}(V',B) := \{ e \in E\ |\ \lambda(e,B) > 1 \land e \cap V' \neq \emptyset\}$ 
be the set of all cut hyperedges of bipartition $B$ of $H$ which are partially or fully contained in $H_{V'}$.
We define the cut of subhypergraph $H_{V'}$ related to a bipartition $B$ of $H$
as follows:
\[\omega_{H_{V'}}(B) := \sum_{e \in E_{\text{cut}}(V',B)} \omega(e) \]
Note, the cut $\omega_{H_{V'}}$ is defined over the
cut nets of $H$. A cut hyperedge $e$ of $H$ is not necessarily a cut hyperedge
of $H_{V'}$. If $e = \{v_1,v_2\}$ is a hyperedge with $v_1 \in V_1$ and $v_2 \in V_2$ and
$v_1 \in V'$ and $v_2 \notin V'$, then $e$ is cut in $H$, but not in $H_{V'}$, because
$v_2$ is removed from $e$ of $H_{V'}$ per definition. However, the reason that we still
define $e$ as cut hyperedge of $H_{V'}$ has to do with our problem statement, 
which we will define as follows:

\begin{problem}
\label{prob:ST} 
How do we have to define the source set $S$ and sink set $T$ for a subhypergraph $H_{V'}$ 
(with $V' \subseteq V$) and a bipartition $B_1$ such that 
after a maximum $(S,T)$-flow calculation (with $f$ as maximum flow)
the resulting minimum $(S,T)$-bipartition $B_2$ of $H$ satisfy the following conditions:
\begin{enumerate}
\item $\omega_H(B_2) \le \omega_H(B_1)$
\item $\Delta_{H} := \omega_H(B_1) - \omega_H(B_2) = \omega_{H_{V'}}(B_1) - |f| =: \Delta_{H_{V'}}$
\end{enumerate}
\end{problem}

The first condition ensures that a maximum $(S,T)$-flow calculation on $T_L(H_{V'})$ never 
decrease the cut of $H$. The existence of the second condition has practical reasons. First, we
can simply update the cut metric via $\omega_H(B_2) = w_H(B_1) - \Delta_{H_{V'}}$,
instead of summing up the weight of all cut hyperedges. Since we have to setup the subhypergraph
$H_{V'}$ before each maximum flow computation, we can implicitly calculate $\omega_{H_{V'}}(B_1)$.
Therefore, the cut metric can be updated after a \emph{Max-Flow-Min-Cut} computation
in constant time instead of \BigO{|E|}. On the other hand, we can assert the correctness of
our maximum flow algorithm. If $\Delta_H \neq \Delta_{H_{V'}}$, then with high probability our
flow algorithm is incorrect. The reason why we define $\omega_{H_{V'}}(B)$ over
the cut hyperedges of $H$ is that the equality
\[\Delta_{H} := \omega_H(B_1) - \omega_H(B_2) = \omega_{H_{V'}}(B_1) - \omega_{H_{V'}}(B_2)\]
holds only if we use the adapted defintion. 
Further, if we can show that $|f| = \omega_{H_{V'}}(B_2)$, we simultaneously show
that our source and sink set modeling approach satisfies condition (ii) 
$\Delta_H = \Delta_{H_{V'}}$.\\
We will now present a solution for our problem statement. First, we show how $S$ and $T$
can be chosen to satisfy condition (i). Afterwards, we extend $S$ and $T$ with additional
nodes to fulfil condition (ii). \\
Let $V' \subseteq V$ and $\delta B = \{ e \in E\ |\ \exists u,v \in e: u \in V'\ \land\ v \notin V' \}$
be the set of all \emph{border hyperedges} (see \autoref{img:balanced_bipartition}). 
Further, we divide $\delta B$ into two disjoint subsets:
\begin{enumerate}
\item Non-Cut hyperedges $e \in \delta B$ of $H$: $\delta B_1 = \{ e \in \delta B \ |\ e \subseteq V_1\ \lor\ e \subseteq V_2 \}$
\item Cut hyperedges $e \in \delta B$ of $H$: $\delta B_2 = \delta B \setminus \delta B_1$
\end{enumerate}
For a bipartition $(V_1,V_2)$ of $H$, we say
$v \in V_1$ is a source node of the flow network $T_L(H_{V'})$, if there exists
a hyperedge $e \in \delta B_1$ containing $v$. More formal:
\begin{align}
%S = \bigcup\limits_{\substack{e \in \delta B \\ |e \setminus V' \cap V_1| \neq 0}} (e \cap V_1 \cap V') \\
S_1 = \{ s \in V' \cap V_1\ |\ \exists e \in \delta B_1: s \in e \} \label{S1_border_hyperedges}\\
T_1 = \{ t \in V' \cap V_2\ |\ \exists e \in \delta B_1: t \in e \} \label{T1_border_hyperedges}
\end{align}
An example of a \emph{Max-Flow-Min-Cut} computation of $H_{V'}$ with $S$ and $T$ as source and
sink set is illustrated in \autoref{img:border_hyperedges}.
\begin{figure}
\centering
\includegraphics[width=0.95\textwidth]{../img/source_and_sink_set/border_hyperedges.eps}
\caption{Non-cut \emph{border hyperedges} of $H$ and $H_{V'}$ induce source and sink hypernodes
         in the flow problem.}
\label{img:border_hyperedges}
\end{figure}

\begin{lemma}
\label{cut_decrease_proof}
Let $B_1$ be a bipartition of $H$ and $T_L(H_{V'})$ the flow network of subhypergraph
$H_{V'}$ with $S$ and $T$ as defined in Equation \ref{S1_border_hyperedges} and \ref{T1_border_hyperedges} (with $V' \subseteq V$).
If $B_2$ is a bipartition obtained by a maximum $(S,T)$-flow computation on $T_L(H_{V'})$,
then the inequality $\omega_H(B_2) \le \omega_H(B_1)$ holds.
\end{lemma}

\begin{proof}
A maximum $(S,T)$-flow computation on $T_L(H_{V'})$ yields a minimum $(S,T)$-cutset on 
$H_{V'}$ \cite{ford1956maximal}. Thus, for all hyperedges $e \notin \delta B$ (fully contained in $H_{V'}$)
which are cut in $B_2$, the sum of their weight must be less or equal than the sum of all cut hyperedges
$e \notin \delta B$ of bipartition $B_1$. We have to show that a non-cut
hyperedge $e \in \delta B_1$ of $B_1 = (V_1,V_2)$ cannot become a cut hyperedge of
$B_2 = (V_1',V_2')$. Let $e \in \delta B_1$ be such a hyperedge. $e$ must be either a subset of $V_1$ or $V_2$, otherwise
$e$ is a cut hyperedge. Let $e \subseteq V_1$, then $e \cap V' \subseteq S$ (see Equation \ref{S1_border_hyperedges}). 
Defining a node $s \in S$ as source node means that it cannot change its block after a \emph{Max-Flow-Min-Cut}
computation. Therefore, $e \subseteq V_1$ and $e \subseteq V_1' \Rightarrow e$ is a non-cut
hyperedge of $B_2$. The proof for $e \subseteq V_2$ is equivalent $\Rightarrow \omega_H(B_2) 
\le \omega_H(B_1)$.
\end{proof} 
\begin{figure}[ht!]
\centering
\includegraphics[width=1.0\textwidth]{../img/source_and_sink_set/non_cut_flow_hyperedges.eps}
\caption{In this example $e_1$ and $e_3$ are cut hyperedges of the hypergraph, but non-cut nets
        of subhypergraph $H_{V'}$. Modeling the \emph{outgoing} resp.
        \emph{incoming} hyperedge node of $e_1$ resp. $e_2$ as sink resp. source ensures
        that $\Delta_H = \Delta_{H_{V'}}$.} 
\label{img:non_cut_flow_hyperedges}
\end{figure}

In the next step, we will show how $S$ and $T$ can be extended to satisfy condition (ii)
of Problem \autoref{prob:ST}. Currently, $|f| \le \omega_{H_{V'}}(B_2)$ (without a prove).
Obviously, some nodes are missing in $S$ and $T$. Consider \autoref{img:non_cut_flow_hyperedges}
to understand which nodes are missing. Transformation $1$ illustrates our current modeling
approach defined in Equation \ref{S1_border_hyperedges} and \ref{T1_border_hyperedges}. The maximum
flow on this network is $|f| = 1$, but the resulting minimum $(S,T)$-bipartition $B_2$ induce
a cut of $\omega_{H}(B_2) = 2$. This implies that $\Delta_H = 3 - 2 \neq 3 - 1 = \Delta_{H_{V'}}$.
The hyperedges $e_1$ and $e_3$ are cut nets of $H$, but non-cut hyperedges of $H_{V'}$. 
Therefore, $B_1$ induce a cut of $1$ on $H_{V'}$ if we define the cut $\omega_{H_{V'}}(B_2)$
over the cut hyperedges of $H_{V'}$ instead of the cut hyperedges of $H$. In our example, we can
remove $e_2$ from cut, but $e_1$ becomes a cut hyperedge of $H_{V'}$. Therefore, the value of the
cut of $H_{V'}$ does not change, but the cut of $H$ does. $e_1$ is already a cut hyperedge of $H$ and $B_2$ removes
$e_2$ from the cut of $H$. Therefore, $\Delta_H = 1$. However, we defined $\omega_{H_{V'}}(B_2)$
over the cut hyperedges of $H$ and currently, we hvae $|f| = 1 \neq 2 = \omega_{H_{V'}}(B_2)$. \\
Transformation $2$ illustrates the adapted modeling approach for cut hyperedges of $H$.
For each hyperedge $e \in \delta B_2$ with $e \setminus V' \cap V_1 \neq \emptyset$, 
we add the \emph{incoming hyperedge node} $\incoming{e}$ to $S$.
More formal:
\begin{align}
S = S_1 \cup \{\incoming{e} \in \delta B_2\ |\ e \setminus V' \cap V_1  \neq \emptyset\} \label{S_border_hyperedges}\\
T = T_1 \cup \{\outgoing{e} \in \delta B_2\ |\ e \setminus V' \cap V_2 \neq \emptyset \} \label{T_border_hyperedges}
\end{align}

\begin{definition}[Extension of a Subhypergraph]
\label{def:sub_extension}
We define the extension $\SubExtension$ of a subhypergraph $H_{V'}$ such that each hyperedge of $H = (V,E,c,\omega)$ which is
partially contained in $H_{V'}$ is fully contained in $\SubExtension$. More formally,
$\SubExtension = (V' \cup V'', E', c, \omega)$ with $V'' = \bigcup_{e \in \delta B} e \setminus V'$
and $E' = \{e \in E\ |\ e \subseteq (V' \cup V'')\}$.
\end{definition}

We have to show that for a maximum $(S,T)$-flow $f$ of $T_L(H_{V'})$ holds $|f| = \omega_{H_{V'}}(B_2)$.
The idea of the proof is to use the extension $\SubExtension$ of $H_{V'}$ and add 
all $V'' \cap V_1$ to $S_1$ (see Equation \ref{S1_border_hyperedges}) and all $V'' \cap V_2$ to
$T_1$ (see Equation \ref{T1_border_hyperedges}). We can show that for a maximum $(S_1,T_1)$-flow 
$f'$ of $T_L(\SubExtension)$ holds that $|f'| = \omega_{H_{V'}}(B_2)$.\\
Afterwards, we use a technique to remove all $v \in V''$ of $T_L(\SubExtension)$ and show
that the resulting flow network is $T_L(H_{V'})$ with $S$ and $T$ as source and sink set as
defined in Equation \ref{S_border_hyperedges} and \ref{T_border_hyperedges}. Moreover, for a maximum
$(S,T)$-flow $f$ then holds that $|f| = |f'| = \omega_{H_{V'}}(B_2)$. \\
Because of the complexity of the proof, we will introduce lemmas in the following which
will simplify the proof of the main theorem. Consider \autoref{img:general_source_and_sink}~if you
need an illustration for the following lemmas.

\begin{figure}
\centering
\includegraphics[width=0.95\textwidth]{../img/source_and_sink_set/general_source_and_sink.eps}
\caption{Illustration of the proof technique used in Lemma \ref{lemma:general_source_and_sink_removal}.
         The green node $s$ is the super source of the flow problem. The blue nodes are source nodes of
         the corresponding \emph{multi-source multi-sink} flow problem. The red nodes $u$ and $v$ are
         contained in $\mathcal{R}(s_2)$. Therefore, edges $(s_2,u)$ and $(s_2,v)$ are removable.}
\label{img:general_source_and_sink}
\end{figure}

\begin{lemma}[Source Edge Removal]
\label{lemma:source_edge_removal}
Let $f$ be a maximum $(S,T)$-flow of $G = (V,E,\capa)$. If there exists two edges $(s_1,v)$ and $(s_2,v)$
with infinite capacity ($s_1,s_2 \in S$) we can either remove $(s_1,v)$ or $(s_2,v)$ from $G$
without changing the amount of a maximum $(S,T)$-flow.
\end{lemma}

\begin{proof}
Let $P = (s_1,v,\ldots)$ be an augmenting path of $G$. Replacing $s_1$ in $P$ with $s_2$ yields
an augmenting path $P'$ of same length. The operation is valid because $\capacity{s_1,v} = \capacity{s_2,v} = \infty$. 
If we execute Edmond and Karp's maximum flow algorithm we can map each augmenting path 
$P$ to $P'$ and ensure that for a maximum $(S,T)$-flow $f$ follows that $f(s_1,v) = 0$. 
Consequently, there exists maximum $(S,T)$-flows where either $f(s_1,v) = 0$ or $f(s_2,v) = 0$.
Therefore, we can remove either $(s_1,v)$ or $(s_2,v)$ without changing the amount of
a maximum $(S,T)$-flow.
\end{proof}

\begin{lemma}[Sink Edge Removal]
\label{lemma:sink_edge_removal}
Let $f$ be a maximum $(S,T)$-flow of $G$. If there exists two edges $(v,t_1)$ and $(v,t_2)$
with infinite capacity ($t_1,t_2 \in T$) we can either remove $(v,t_1)$ or $(v,t_2)$ from $G$
without changing the amount of a maximum $(S,T)$-flow.
\end{lemma}

\begin{proof}
Equivalent to proof of Lemma \ref{lemma:source_edge_removal}.
\end{proof}

\begin{definition}[Removable Edges]
We denote the set of all adjacent nodes $v$ of a source node $s$ resp. sink node $t$, where
edge $(s,v)$ or $(v,t)$ is removable according to Lemma \ref{lemma:source_edge_removal}
and \ref{lemma:sink_edge_removal}, with $\mathcal{R}(s)$ resp. $\mathcal{R}(t)$.
\end{definition}

The following lemma is a generalisation of Lemma \ref{lemma:source_and_sink_removal}. We will use
the definition of $in(u)$ and $out(u)$ presented in Section \ref{sec:heuer_network}. Further,
$G_{V'}$ is a subgraph of $G = (V,E)$ induce by $V' \subseteq V$ (see Defintion \ref{def:subgraph}).

\begin{lemma}[General Source/Sink Node Removal]
\label{lemma:general_source_and_sink_removal}
Let $f$ be a maximum $(S,T)$-flow of $G = (V,E,\capa)$ with $|f| < \infty$ and
$E_s \subseteq \mathcal{R}(s)$ and $E_t \subseteq \mathcal{R}(t)$ with $s \in S$
and $t \in T$.
If $s$ is a source node where all outgoing edges have infinte capacity and
$t$ is a sink node where all incoming edges have infinte capacity, then
$|f|$ is equal with the amount of a maximum $(S',T)$-flow of $G_{V\setminus \{s\}}$ and a maximum
$(S,T')$-flow of $G_{V\setminus \{t\}}$, where $S' = (S\setminus \{s\}) \cup (out(s) \setminus E_s)$ 
and $T' = (T \setminus \{t\}) \cup (in(t) \setminus E_t)$.
\end{lemma}

\begin{proof}
$E_s$ is an arbitrary subset of $\mathcal{R}(s)$, where foreach $v \in E_s$ the edge $(s,v)$ is removable.
$S'$ is the source set without node $s$ extended with all outgoing edges of 
$s$ minus the removable edges $E_s$. With Lemma \ref{lemma:source_edge_removal}
we can remove all edges $(s,v)$ with $v \in E_s$ from $G$ and obtain flow network $G'$. Finally, we can
apply Lemma \ref{lemma:source_and_sink_removal} on $G'$ and obtain $G_{V\setminus \{s\}}$ with $(S',T)$ as source
and sink set (see \autoref{img:general_source_and_sink}). All used Lemma's did not change the amount of a maximum flow. Therefore, a maximum
$(S,T)$-flow of $G$ is equal with a maximum $(S',T)$-flow of $G_{V\setminus \{s\}}$. The proof for $t$ is
equivalent.
\end{proof}

The proof of Lemma \ref{cut_decrease_proof} can be applied one-to-one on our new source and
sink sets because $S_1 \subseteq S$ and $T_1 \subseteq T$. Therefore, $S$ and $T$ 
as defined in Equation \ref{S_border_hyperedges} and \ref{T_border_hyperedges} satisfies
condition (i) of Problem \ref{prob:ST}. We will show that for $S$ and $T$ 
the equality $\Delta_H = \Delta_{H_{V'}}$ holds.

\begin{theorem} 
\label{lemma:delta_proof}
Let $B_1 = (V_1,V_2)$ be a bipartition of $H$ and $T_L(H_{V'})$ the flow network of subhypergraph
$H_{V'}$ with $S$ and $T$ as defined in Equation \ref{S_border_hyperedges} and \ref{T_border_hyperedges} (with $V' \subseteq V$).
If $B_2$ is a bipartition obtained by a maximum $(S,T)$-flow computation on $T_L(H_{V'})$
with $f$ as maximum flow,
then $\omega_{H_{V'}}(B_2) = |f|$ ($\Rightarrow \Delta_H = \Delta_{H_{V'}}$).
\end{theorem}

\begin{proof}
Consider the extension $\SubExtension$ of subhypergraph $H_{V'}$ (see Definition \ref{def:sub_extension}).
Each maximum $(S,T)$-flow $f'$ of $T_L(\SubExtension)$ is then equal with a minimum-weight
$(S,T)$-cutset of $H_{V'}$ according to our definition of the cut $\omega_{H_{V'}}(B_2)$ over
the cut hyperedges of $H$. Because each hyperedge which is partially contained in $H_{V'}$
is fully contained in $\SubExtension$. Therefore, it holds that $|f'| = \omega_{H_{V'}}(B_2)$.
However, we have to model some restrictions into our source and sink set of $T_L(\SubExtension)$.
We will denote the source and sink set of $T_L(\SubExtension)$ with $S'$ and $T'$.
Each hypernode contained in a non-cut border hyperedge $e \in \delta B_1$ should not be able to move
such that we ensure that $e$ is not cut after a maximum $(S',T')$-flow calculation. Therefore,
we add $S_1$ and $T_1$ to $S'$ and $T'$ (see Equation \ref{S1_border_hyperedges} and \ref{T1_border_hyperedges}).
Further, all hypernodes $v \in V''$ (see Defintion \ref{def:sub_extension}) are not contained in
$H_{V'}$. Consequently, they cannot change their block if we calculate a maximum $(S,T)$-flow of
$T_L(H_{V'})$. Therefore, we add $V'' \cap V_1$ to $S'$ and $V'' \cap V_2$ to $T'$.
With $S'$ and $T'$ as source and sink set we ensure that only hypernodes $v \in V'$ are able
to move and since $S_1 \subseteq S'$ and $T_1 \subseteq T'$, we ensure that $\omega_H(B_2) \le \omega_H(B_1)$
(see Lemma \ref{cut_decrease_proof}). \\
In the following, we apply Lemma \ref{lemma:general_source_and_sink_removal} on all
hypernodes $v \in V''$ such that the flow network $T_L(\SubExtension)$ converge against
$T_L(H_{V'})$ with $S$ and $T$ as source and sink set as defined in Equation 
\ref{S_border_hyperedges} and \ref{T_border_hyperedges} \emph{without} changing the
amount of the maximum flow $f'$. Since $|f'| = \omega_{H_{V'}}(B_2)$, for a maximum $(S,T)$-flow
$f$ of $T_L(H_{V'})$ then holds $|f| = \omega_{H_{V'}}(B_2)$. Per definition a node $v \in V''$
is either a source or sink node. For each source node $s \in V''$ we have to 
define a removable subset $E_s \subseteq \mathcal{R}(s)$ such that after removing all $s \in S' \cap V''$
the resulting source set $S$ is equal to Equation \ref{S_border_hyperedges}. The technique
for removing each sink node $t \in S' \cap V''$ will be equivalent. For a source node $s \in V''$
we define $E_s = \{\incoming{e}\ |\ e \in I(s) \cap \delta B_1\}$. Remember, $\delta B_1$ contains
all non-cut border hyperedges of $H$. Thus, for each $e \in I(s) \cap \delta B_1$ exists
a source node $\bar{s} \in V'$ of $S_1$ such that edges $(s,\incoming{e})$ and $(\bar{s},\incoming{e})$
are contained in $T_L(\SubExtension)$ (see \autoref{img:delta_proof_illustration}). Therefore, $E_s$ is a removable subset of $\mathcal{R}(s)$.
For each $t \in T' \cap V''$ we define the removable subset $E_t = \{\outgoing{e}\ |\ e \in I(t) \cap \delta B_1\}$.
Applying Lemma \ref{lemma:general_source_and_sink_removal} on all source nodes $s \in V''$
with $E_s$ as removable subset and on all sink nodes $t \in V''$ with $E_t$ 
as removable subset yield flow network $T_L(H_{V'})$ with $S$ and $T$ as source and sink 
set as defined in Equation \ref{S_border_hyperedges} and \ref{T_border_hyperedges}.
A hyperedge $e \in \delta B_1$ cannot become a source or sink node because if we remove a 
source node $s \in e$ then $e'$ is in the removable subset $E_s$ (see \autoref{img:delta_proof_illustration}). The same holds
for each sink node $t \in V''$. A hyperedge $e \in \delta B_2$ becomes a source node of $S$
if we remove a source node $s \in e$ and a sink node of $T$ if we remove a 
sink node $t \in e$ because $\forall e \in \delta B_2: e',e'' \notin E_s \cup E_t$. 
Therefore, $S$ and $T$ are equal to our source and sink set defintion and
for a maximum $(S,T)$-flow $f$ it holds that $|f| = |f'| = \omega_{H_{V'}}(B_2)$.
\end{proof}

\begin{figure}
\centering
\includegraphics[width=0.95\textwidth]{../img/source_and_sink_set/delta_proof_illustration.eps}
\caption{Illustration how to remove a source node $v \in V''$. Note, the green node $s$
         is the super source of the flow problem. Consequently, all nodes connected to $s$
         are source nodes in the corresponding \emph{multi-source multi-sink} flow problem.}
\label{img:delta_proof_illustration}
\end{figure}


%\begin{figure}
%\centering
%\includegraphics[width=0.95\textwidth]{../img/source_and_sink_set/cut_border_hyperedges.eps}
%\caption{Illustration of modeling \emph{Cut Border Hyperedges} as sources and sinks. In this
%         example $e_1$ contains node from block $V_1$ and $V_2$ not contained in the flow problem. Therefore,
%         we can not remove $e_1$ from cut. Treating $e_1$ as a \emph{Border Hyperedge} would result
%         in Transformation $1$. This has the consequence that we are not able to remove $e_2$
%         from cut with a \emph{Max-Flow-Min-Cut} computation. Defining the \emph{incoming} resp.
%         \emph{outgoing} hyperedge of $e_1$ as source resp. sinks allows the corresponding hypernodes
%         of $e_1$ still to move. The consequence is that we can remove $e_2$ from cut with a
%         \emph{Max-Flow-Min-Cut} computation in Transformation $2$. }
%\label{img:cut_border_hyperedges}
%\end{figure}

We are now able to extract a subhypergraph $H_{V'}$ out of an already bipartitioned hypergraph $H$ and
calculate a minimum $(S,T)$-bipartition of $H_{V'}$ with $S$ and $T$ as defined
in Equation \ref{S_border_hyperedges} and \ref{T_border_hyperedges}. The resulting
bipartition induce a new cut on $H$ smaller or equal than the old cut. Further, we show with our
modeling technique of $S$ and $T$ that $\Delta_H$ can be calculated with the help of the amount 
of a maximum $(S,T)$-flow computation on $T_L(H_{V'})$. \\
\begin{figure}[ht!]
\centering
\includegraphics[width=0.95\textwidth]{../img/source_and_sink_set/st_modelling_summary.eps}
\caption{Illustration of modeling sources and sinks defined in Equation \ref{S_border_hyperedges}
         and \ref{T_border_hyperedges}. }
\label{img:st_modelling_summary}
\end{figure}
In Section \ref{sec:edge_size_network} we described how to remove hyperedges of size $|e| = 2$ 
by adding an undirected flow edge between the corresponding vertices $u,v \in e$. However, if
the incoming or outgoing hyperedge node is a source or a sink node, we can not directly
remove the hyperedge nodes. There are two special cases which are illustrated
in \autoref{img:edge_size_two}. This situation occurs if one of the two vertices is part
of the flow problem and one not. In case, if the incoming hyperedge node $\incoming{e}$ is a source node, 
we only remove the outgoing hyperedge node $\outgoing{e}$ and add a directed flow edge from $\incoming{e}$ 
to $v$ with capacity $\omega(e)$. In the second case, if the outgoing hyperedge node $\outgoing{e}$ is
a sink node, we only remove the incoming hyperedge node $\incoming{e}$ and add a directed flow edge from
$v$ to $\outgoing{e}$ with capacity $\omega(e)$.\\
\begin{figure}
\centering
\includegraphics[width=0.95\textwidth]{../img/source_and_sink_set/edge_size_two.eps}
\caption{Illustration of modeling hyperedges of size two if the incoming or outgoing
         hyperedge node is a source or a sink node of the flow problem.}
\label{img:edge_size_two}
\end{figure}
With the given approach we can optimize the cut metric of a given
bipartition of a hypergraph $H$. We can transfer those results to improve
a $k$-way partition $\Pi = (V_1,\ldots,V_k)$ if the objective is the connectivity
metric. Let $V' \subseteq V_i \cup V_j$ be a subset of the hypernodes of two adjacent
blocks $V_i$ and $V_j$. If we optimize the cut of
subhypergraph $H_{V'}$ we simultaneously optimize the connectivity metric of $H$.
The reduction of the cut of $H_{V'}$ is then equal with the decrease in
the connectivity metric of $H$.

\subsection{Most Balanced Minimum Cuts on Hypergraphs}
\label{sec:mbmc_hypergraphs}

Picard and Queyranne \cite{picard1980structure} show that all minimum $(s,t)$-cuts 
of a graph $G$ are computable with one maximum $(s,t)$-flow computation by 
iterating through all \emph{closed node sets} of the residual graph of $G$. 
The corresponding algorithm is presented in Section \ref{sec:related_mbmc}. \\
We can apply the same algorithm on hypergraphs. A minimum-capacity $(s,t)$-cutset of \ShortT{L}
is equal with a minimum-weight $(s,t)$-cutset of $H$. With the algorithm
of Section \ref{sec:related_mbmc} we can find all minimum-capacities
$(s,t)$-cutsets of \ShortT{L}, which are also minimum-weight $(s,t)$-cutsets
of $H$. The corresponding minimum-weight $(s,t)$-bipartitions are all
\emph{closed node sets} of the residual graph of \ShortT{L}. \\
However, when we use e.g. \ShortExT{H}{V'} (see Section \ref{sec:heuer_network})
or \ShortHybrid~(see Section \ref{sec:hybrid_network}) as underlying flow network,
some hypernodes are removed from the flow problem. It is a problem if we want to
enumerate all minimum-weight $(s,t)$-bipartitions. The solution for 
this problem is quite simple. After a maximum $(s,t)$-flow calculation
on one of the two mentioned networks we insert all removed hypernodes with
their corresponding edges again into the residual graph of our flow network.
The maximum $(s,t)$-flow is still maximal. Otherwise, we would have found an \emph{augmenting
path} on the flow network before. We are now able to compute all minimum-weight
$(s,t)$-bipartitions the same way as with \ShortT{L}.

\subsection{A direct $k$-way Flow-Based Refinement Framework}
\label{sec:flow_local_search_hypergraph}

We have described how a hypergraph $H$ could be transformed into
a flow network \ShortT{L} such that each minimum-capacity $(S,T)$-cutset of \ShortT{L} is a 
minimum-weight  $(S,T)$-cutset of $H$ (see Section \ref{sec:related_lawler}). 
Additionaly, we present techniques to sparsify the
flow network \ShortT{L} \cite{lawler1973} to reduce the complexity of 
the flow problem (see Section \ref{sec:opt_flow_network}). 
Further, we show how to configure the source and sink sets of a flow network of a 
subhypergraph $H_{V'}$ (with $V' \subseteq V$) such that a \emph{Max-Flow-Min-Cut} 
computation improves a given bipartition of $H$ (see Section \ref{sec:source_and_sink}). 
Finally, we can enumerate all minimum-weight $(s,t)$-cutsets of a subhypergraph 
$H_{V'}$ with one maximum $(S,T)$-flow calculation \cite{picard1980structure}. \\
We will now present our direct $k$-way flow-based refinement framework which we integrated
into the $n$-level hypergraph partitioner \emph{KaHyPar} \cite{heuer2017improving} 
(see Section \ref{sec:kahypar}). Our flow-based refinement approach optimizes
the \emph{connectivity} metric. We used a similiar architecture as proposed
by Sanders and Schulz \cite{sanders2011engineering} (see Section 
\ref{sec:flow_local_search_graph}). The basic concepts of the framework are
illustrated in \autoref{img:flow_framework}. \\
Our maximum flow calculations are embedded into an \emph{Active Block Scheduling}
refinement \cite{holtgrewe2010engineering} (see Section \ref{sec:abs}).
Each time we use flows to improve the connectivity metric of
a given $k$-way partition $\Pi$ we construct the quotient graph $Q$ of $\Pi$. 
Afterwards, we iterate over all edges of $Q$ in random order. For each edge
$(V_i,V_j)$ of $Q$, we build a flow problem around the cut of the bipartition
induced by $V_i$ and $V_j$. To do that we use two \BFS, one only 
touches hypernodes of $V_i$ and the second only touches hypernodes of $V_j$.
The \BFS~is initialized with all hypernodes contained in a cut hyperedge
of the bipartition $(V_i,V_j)$. A pairwise flow-based refinement is embedded
into the \emph{adaptive flow iterations} strategy \cite{sanders2011engineering}
(see Section \ref{sec:adaptive_flow_iterations}) which also determines
the size of the flow problem. \\
After we define the subhypergraph $H_{V'}$, which we use to improve the bipartition
$(V_i,V_j)$ on $H$, we construct one of the flow networks proposed in Section
\ref{sec:opt_flow_network} with sources $S$ and sinks $T$ defined in
Section \ref{sec:source_and_sink}. We implemented two maximum flow algorithms.
One is a slightly modified \emph{augmenting path} algorithm of Edmond \& Karp
\cite{edmonds1972theoretical} (see Section \ref{sec:aug_path}) 
and the second is the \emph{Push-Relabel} algorithm of
Goldberg \& Tarjan \cite{cherkassky1997implementing,goldberg1988new} 
(see Section \ref{sec:push_relabel}). Since we have a 
\emph{Multi-Source-Multi-Sink} problem, we can find several \emph{augmenting paths}
with one \BFS. After we execute a \BFS~on the residual graph, we search 
as many as possible edge-disjoint paths in the resulting \BFS-tree connecting a source $s$
with a sink $t$. Our Goldberg \& Tarjan implementation uses a \emph{FIFO} queue and
the \emph{global relabeling} and \emph{gap} heuristic \cite{cherkassky1997implementing}.
We do not use an external implementation of a maximum flow algorithm.
Since the \emph{I\textbackslash O} of writing a flow problem to memory and reading the
solution would significantly slowdown the performance of our algorithm because we have
to solve an enormous number of flow problems during the \emph{Active Block Scheduling}
refinement. After we determine a maximum $(S,T)$-flow on our flow network, we iterate over
all minimum $(S,T)$-bipartitions of $H_{V'}$ \cite{picard1980structure} and choose 
the \emph{Most Balanced Minimum Cut} (see Section \ref{sec:related_mbmc} and 
\ref{sec:mbmc_hypergraphs}) according to our \emph{balanced constraint}. \\
\emph{KaHyPar} is an $n$-level hypergraph partitioner ($|V| = n$) taking the 
multilevel paradigm to its extreme by removing only a single vertex in every level
of the hierarchy \cite{akhremtsev2017engineering} (see Section \ref{sec:kahypar}). 
During the refinement step $n$ local searches are instantiated. Therefore, 
using our flow-based refinement as local search algorithm on each level is not 
applicable, because the performance slowdown would be tremendous. Therefore,
we introduce \emph{Flow Execution Policies}. One is to execute our flow-based
refinement on each level $i$ where $i = \beta\cdot j$ with $j \in \mathbb{N}_+$ and
$\beta$ as a predefined tunning parameter. Another approach is to simulate a
multilevel partitioner with $\log(n)$ hierarchies. A flow-based refinement is then
executed on each level $i$ where $i = 2^j$ with $j \in \mathbb{N}_+$. Each policy also
performs the \emph{Active Block Scheduling} refinement on the last level of the
hierarchy. In all remaining levels where no flow is executed, we can use an 
\emph{FM}-based local search algorithm 
\cite{akhremtsev2017engineering,fiduccia1988linear,sanchis1989multiple} (see Section 
\ref{sec:abs}). \\
An observation during the implementation of this framework was that only a minority
of the pairwise refinements based on flows yields to an improvement of the connectivity
metric on hypergraph $H$. Thus, we introduce several rules which might prevent
unnecessary flow executions to improve the effectiveness ratio by simultaneously speeding up
the running time.

\begin{enumerate}
\item[(R1)] If a flow-based refinement did not lead to an improvement on two blocks in all previous
            executions, we would use flows only in the first iteration of 
            \emph{Active Block Scheduling}.
\item[(R2)] If the cut between two adjacent blocks in the quotient graph is small (e.g. $\le 10$) we
            skip the flow-based refinement on these blocks except on the last level of the hierarchy.
\item[(R3)] If the value of the cut of a minimum $(S,T)$-bipartition of $H_{V'}$ is the same 
            as the cut before, we stop the pairwise refinement.
\end{enumerate}

\begin{figure}
\centering 
\includegraphics[width=1.0\textwidth]{../img/flow_local_search/flow_framework_hypergraph.eps}
\caption{Illustration of our flow-based refinement framework for direct $k$-way hypergraph
         partitioning.}
\label{img:flow_framework}
\end{figure} 

\newpage

%%%%%%%%%%%%%%%%%%%%%%%%%%%%%%%%%%%%%%%%%%%%%%%%%%%%%%%%%%%%%%%%%%%%%%

\section{Experimental Results}
\label{sec:experiments}

\subsection{Instances}

Our full benchmark set consists of $488$ hypergraphs. We choose our benchmarks 
from three different research areas. For VLSI design we use instances from
the \emph{ISPD98 VLSI Circuit Benchmark Suite} (\ISPD) \cite{alpert1998ispd98} and add more recent
instances of the \emph{DAC 2012 Routability-Driven Placement Contest} (\DAC) \cite{viswanathan2012dac}.
Further, we interprete the Sparse Matrix instances of the \emph{Florida Sparse Matrix 
collection} (\SPM) \cite{davis2011university} as hypergraphs using the row-net model \cite{catalyurek1999hypergraph}.
The rows of each matrix are treated as hyperedges and the columns are the vertices of
the hypergraph. Our last benchmark type are SAT formulas of the \emph{International SAT
Competition 2014} \cite{belov2014application}. A common interpretation of a SAT formula 
as hypergraph is to interprete the literals as vertices and each clause as a net (\Literal) \cite{papa2007hypergraph}.
Mann and Papp \cite{mann2014formula} suggested two other hypergraph representation of
SAT formulas, called \Primal~and \Dual. The \Primal~representation treats each variable
as vertex and each clause as hyperedge. The \Dual~representation treats each clause as
vertex and the variables induced nets containing all clauses where the corresponding
variable occurs. A statiscal summary of the different instance types is presented in
Table \ref{tbl:type_properties}. \\
We divide our full benchmark set in two smaller subsets. Our \emph{parameter tunning}
benchmark set consists of $25$ hypergraphs, $5$ of each instance type (except \DAC). Additionally,
we choose a benchmark subset of $165$ instances. On our general experiments we partition
each hypergraph into $k \in \{2,4,8,16,32,64,128\}$ blocks and use for each $k$ $10$ different
\emph{seeds} with $\epsilon = 3\%$.

\begin{table}
\renewcommand{\arraystretch}{1.15}
\centering
\begin{tabular}{r|cc}
\toprule
Instance-Type & Avg. $d(v)$ & Avg. $|e|$ \\
\midrule%
\csname @@input\endcsname experiments/flow_network/instance_type_properties.tex 
\bottomrule
\end{tabular}
\caption{Average hypernode degree and hyperedge size of the different benchmark types in
         our benchmark subset.}
\label{tbl:type_properties}
\end{table}

\subsection{System and Methodology}

\begin{enumerate}
\item Used Compiler + Flags
\item Configuration of \emph{hMetis} and \emph{PaToH}
\item Describe \emph{cuber} plots
\end{enumerate}


\subsection{Flow Algorithms and Networks}

In the first experiment we want examine the impact of our sparsifying techniques (see Section \ref{sec:opt_flow_network})
on the performance of our maximum flow algorithms \GoldbergTarjan~and \EdmondKarp. 
Therefore, we first take a look at the reduction of the number of nodes and edges on different benchmark types
when using $\ExpLawler$ (see Section \ref{sec:related_lawler}), $\ExpNodeDegree$ (see Section 
\ref{sec:degree_network}), $\ExpEdgeSize$ (see Section \ref{sec:edge_size_network})
and $\ExpHybrid$ (see Section \ref{sec:hybrid_network}) . Further, we
want to evaluate the performance of the two implemented maximum flow algorithms on these
networks. \\
We evaluate the performance of the different flow networks on flow problems with
$|V'| \in \{500,1000,5000,10000,25000\}$ hypernodes. The instances are generated by executing
\emph{KaHyPar} on our benchmark subset (\todo{ref to appendix}) for $k = 2$ and five different 
seeds. After a instance is bipartitioned, we generate flow problem instances
with the above mentioned sizes and execute each possible combination of flow algorithm and
network on it. \\
The benchmark instances can be splitted into $6$ different benchmark types. The properties of these instances
in terms of the average hypernode degree and average hyperedge size is shown in \autoref{tbl:type_properties}.
Remember, $\ExpEdgeSize$ should perform best on instances with a small average hyperedge size and
$\ExpNodeDegree$ should perform best on instances with a small average hypernode degree. Based on 
\autoref{tbl:type_properties}, $\ExpEdgeSize$ should significantly reduce the number of
nodes and edges on \Primal~and \Literal~instances and $\ExpNodeDegree$ on \Dual~instances in
comparison to a our baseline $\ExpLawler$. Also both should sparsify the resulting flow network
of \ISPD~and \DAC~instances. Further, we expect that $\ExpHybrid$ combines the advantages of
both networks and performs best on all bechmark instances.\\
\autoref{fig:node_edge_distribution}~shows the predicted behaviour for flow problems of size
$25000$ hypernodes. A point on the grid is the \emph{geometric mean} of the number of 
nodes resp. edges (in the flow network) of all intances for the corresponding benchmark type.
$\ExpHybrid$ reduces the number of nodes of nearly every benchmark type by at least a factor
of $2$, except on \SPM~instances. Another observation is that instances with a large average 
hypernode degree, like \Primal~or \Literal, yield to big flow problem instances and vice 
versa (see \Dual~instances).\\
\begin{figure}
\centering
% Created by tikzDevice version 0.6.2-92-0ad2792 on 2017-11-02 18:28:28
% !TEX encoding = UTF-8 Unicode
\begin{tikzpicture}[x=1pt,y=1pt]
\definecolor[named]{fillColor}{rgb}{1.00,1.00,1.00}
\path[use as bounding box,fill=fillColor,fill opacity=0.00] (0,0) rectangle (505.89,433.62);
\begin{scope}
\path[clip] (  0.00,  0.00) rectangle (505.89,433.62);
\definecolor[named]{fillColor}{rgb}{1.00,1.00,1.00}

\path[fill=fillColor] ( -0.00,  0.00) rectangle (505.89,433.62);
\end{scope}
\begin{scope}
\path[clip] ( 52.03,138.27) rectangle (503.04,433.62);
\definecolor[named]{drawColor}{rgb}{0.75,0.75,0.75}

\path[draw=drawColor,line width= 1.5pt,line join=round,line cap=round] ( 52.03,138.27) rectangle (503.04,433.62);
\definecolor[named]{drawColor}{rgb}{0.90,0.90,0.90}

\path[draw=drawColor,line width= 0.3pt,line join=round] ( 52.03,152.39) --
	(503.04,152.39);

\path[draw=drawColor,line width= 0.3pt,line join=round] ( 52.03,192.81) --
	(503.04,192.81);

\path[draw=drawColor,line width= 0.3pt,line join=round] ( 52.03,273.64) --
	(503.04,273.64);

\path[draw=drawColor,line width= 0.3pt,line join=round] ( 52.03,345.07) --
	(503.04,345.07);

\path[draw=drawColor,line width= 0.3pt,line join=round] ( 52.03,402.22) --
	(503.04,402.22);

\path[draw=drawColor,line width= 0.3pt,line join=round] ( 92.03,138.27) --
	( 92.03,433.62);

\path[draw=drawColor,line width= 0.3pt,line join=round] (139.09,138.27) --
	(139.09,433.62);

\path[draw=drawColor,line width= 0.3pt,line join=round] (233.22,138.27) --
	(233.22,433.62);

\path[draw=drawColor,line width= 0.3pt,line join=round] (316.39,138.27) --
	(316.39,433.62);

\path[draw=drawColor,line width= 0.3pt,line join=round] (382.95,138.27) --
	(382.95,433.62);

\path[draw=drawColor,line width= 0.3pt,line join=round] (440.22,138.27) --
	(440.22,433.62);

\path[draw=drawColor,line width= 0.8pt,line join=round] ( 52.03,233.22) --
	(503.04,233.22);

\path[draw=drawColor,line width= 0.8pt,line join=round] ( 52.03,314.05) --
	(503.04,314.05);

\path[draw=drawColor,line width= 0.8pt,line join=round] ( 52.03,376.08) --
	(503.04,376.08);

\path[draw=drawColor,line width= 0.8pt,line join=round] ( 52.03,428.37) --
	(503.04,428.37);

\path[draw=drawColor,line width= 0.8pt,line join=round] (186.15,138.27) --
	(186.15,433.62);

\path[draw=drawColor,line width= 0.8pt,line join=round] (280.28,138.27) --
	(280.28,433.62);

\path[draw=drawColor,line width= 0.8pt,line join=round] (352.50,138.27) --
	(352.50,433.62);

\path[draw=drawColor,line width= 0.8pt,line join=round] (413.39,138.27) --
	(413.39,433.62);

\path[draw=drawColor,line width= 0.8pt,line join=round] (467.04,138.27) --
	(467.04,433.62);
\definecolor[named]{fillColor}{rgb}{0.78,0.49,1.00}

\path[fill=fillColor] (128.16,174.30) circle ( 11.07);

\path[fill=fillColor] (145.59,219.91) --
	(160.50,194.10) --
	(130.69,194.10) --
	cycle;

\path[fill=fillColor] ( 74.41,140.63) --
	( 96.54,140.63) --
	( 96.54,162.76) --
	( 74.41,162.76) --
	cycle;
\definecolor[named]{drawColor}{rgb}{0.78,0.49,1.00}

\path[draw=drawColor,line width= 0.4pt,line join=round,line cap=round] (286.89,339.13) -- (318.19,339.13);

\path[draw=drawColor,line width= 0.4pt,line join=round,line cap=round] (302.54,323.48) -- (302.54,354.78);

\path[draw=drawColor,line width= 0.4pt,line join=round,line cap=round] (212.10,241.49) rectangle (234.23,263.62);

\path[draw=drawColor,line width= 0.4pt,line join=round,line cap=round] (212.10,241.49) -- (234.23,263.62);

\path[draw=drawColor,line width= 0.4pt,line join=round,line cap=round] (212.10,263.62) -- (234.23,241.49);

\path[draw=drawColor,line width= 0.4pt,line join=round,line cap=round] (148.68,356.58) -- (170.81,378.71);

\path[draw=drawColor,line width= 0.4pt,line join=round,line cap=round] (148.68,378.71) -- (170.81,356.58);

\path[draw=drawColor,line width= 0.4pt,line join=round,line cap=round] (144.09,367.65) -- (175.39,367.65);

\path[draw=drawColor,line width= 0.4pt,line join=round,line cap=round] (159.74,352.00) -- (159.74,383.29);
\definecolor[named]{fillColor}{rgb}{0.00,0.75,0.77}

\path[fill=fillColor] (139.56,179.46) circle ( 11.07);

\path[fill=fillColor] (152.97,223.45) --
	(167.87,197.63) --
	(138.06,197.63) --
	cycle;

\path[fill=fillColor] (113.52,167.04) --
	(135.65,167.04) --
	(135.65,189.17) --
	(113.52,189.17) --
	cycle;
\definecolor[named]{drawColor}{rgb}{0.00,0.75,0.77}

\path[draw=drawColor,line width= 0.4pt,line join=round,line cap=round] (286.90,339.13) -- (318.20,339.13);

\path[draw=drawColor,line width= 0.4pt,line join=round,line cap=round] (302.55,323.49) -- (302.55,354.78);

\path[draw=drawColor,line width= 0.4pt,line join=round,line cap=round] (221.23,246.62) rectangle (243.37,268.75);

\path[draw=drawColor,line width= 0.4pt,line join=round,line cap=round] (221.23,246.62) -- (243.37,268.75);

\path[draw=drawColor,line width= 0.4pt,line join=round,line cap=round] (221.23,268.75) -- (243.37,246.62);

\path[draw=drawColor,line width= 0.4pt,line join=round,line cap=round] (150.68,357.85) -- (172.81,379.98);

\path[draw=drawColor,line width= 0.4pt,line join=round,line cap=round] (150.68,379.98) -- (172.81,357.85);

\path[draw=drawColor,line width= 0.4pt,line join=round,line cap=round] (146.10,368.92) -- (177.40,368.92);

\path[draw=drawColor,line width= 0.4pt,line join=round,line cap=round] (161.75,353.27) -- (161.75,384.57);
\definecolor[named]{fillColor}{rgb}{0.49,0.68,0.00}

\path[fill=fillColor] (183.06,203.29) circle ( 11.07);

\path[fill=fillColor] (192.13,241.23) --
	(207.03,215.41) --
	(177.23,215.41) --
	cycle;

\path[fill=fillColor] ( 74.47,140.63) --
	( 96.60,140.63) --
	( 96.60,162.76) --
	( 74.47,162.76) --
	cycle;
\definecolor[named]{drawColor}{rgb}{0.49,0.68,0.00}

\path[draw=drawColor,line width= 0.4pt,line join=round,line cap=round] (466.74,420.16) -- (498.04,420.16);

\path[draw=drawColor,line width= 0.4pt,line join=round,line cap=round] (482.39,404.52) -- (482.39,433.62);

\path[draw=drawColor,line width= 0.4pt,line join=round,line cap=round] (320.84,298.71) rectangle (342.97,320.85);

\path[draw=drawColor,line width= 0.4pt,line join=round,line cap=round] (320.84,298.71) -- (342.97,320.85);

\path[draw=drawColor,line width= 0.4pt,line join=round,line cap=round] (320.84,320.85) -- (342.97,298.71);

\path[draw=drawColor,line width= 0.4pt,line join=round,line cap=round] (153.06,359.41) -- (175.19,381.54);

\path[draw=drawColor,line width= 0.4pt,line join=round,line cap=round] (153.06,381.54) -- (175.19,359.41);

\path[draw=drawColor,line width= 0.4pt,line join=round,line cap=round] (148.47,370.47) -- (179.77,370.47);

\path[draw=drawColor,line width= 0.4pt,line join=round,line cap=round] (164.12,354.83) -- (164.12,386.12);
\definecolor[named]{fillColor}{rgb}{0.97,0.46,0.43}

\path[fill=fillColor] (200.35,210.60) circle ( 11.07);

\path[fill=fillColor] (202.58,245.77) --
	(217.48,219.96) --
	(187.68,219.96) --
	cycle;

\path[fill=fillColor] (113.64,167.07) --
	(135.77,167.07) --
	(135.77,189.20) --
	(113.64,189.20) --
	cycle;
\definecolor[named]{drawColor}{rgb}{0.97,0.46,0.43}

\path[draw=drawColor,line width= 0.4pt,line join=round,line cap=round] (466.90,420.19) -- (498.19,420.19);

\path[draw=drawColor,line width= 0.4pt,line join=round,line cap=round] (482.54,404.55) -- (482.54,433.62);

\path[draw=drawColor,line width= 0.4pt,line join=round,line cap=round] (331.98,302.98) rectangle (354.11,325.11);

\path[draw=drawColor,line width= 0.4pt,line join=round,line cap=round] (331.98,302.98) -- (354.11,325.11);

\path[draw=drawColor,line width= 0.4pt,line join=round,line cap=round] (331.98,325.11) -- (354.11,302.98);

\path[draw=drawColor,line width= 0.4pt,line join=round,line cap=round] (155.41,360.91) -- (177.54,383.04);

\path[draw=drawColor,line width= 0.4pt,line join=round,line cap=round] (155.41,383.04) -- (177.54,360.91);

\path[draw=drawColor,line width= 0.4pt,line join=round,line cap=round] (150.83,371.97) -- (182.13,371.97);

\path[draw=drawColor,line width= 0.4pt,line join=round,line cap=round] (166.48,356.32) -- (166.48,387.62);
\definecolor[named]{drawColor}{rgb}{1.00,0.00,0.00}
\definecolor[named]{fillColor}{rgb}{1.00,0.00,0.00}

\path[draw=drawColor,line width= 0.6pt,dash pattern=on 4pt off 4pt ,line join=round,fill=fillColor] ( 72.53,138.27) -- ( 72.53,433.62);
\end{scope}
\begin{scope}
\path[clip] (  0.00,  0.00) rectangle (505.89,433.62);
\definecolor[named]{drawColor}{rgb}{0.00,0.00,0.00}

\node[text=drawColor,anchor=base east,inner sep=0pt, outer sep=0pt, scale=  0.88] at ( 47.61,230.19) {300000};

\node[text=drawColor,anchor=base east,inner sep=0pt, outer sep=0pt, scale=  0.88] at ( 47.61,311.02) {600000};

\node[text=drawColor,anchor=base east,inner sep=0pt, outer sep=0pt, scale=  0.88] at ( 47.61,373.05) {900000};

\node[text=drawColor,anchor=base east,inner sep=0pt, outer sep=0pt, scale=  0.88] at ( 47.61,425.34) {1200000};
\end{scope}
\begin{scope}
\path[clip] (  0.00,  0.00) rectangle (505.89,433.62);
\definecolor[named]{drawColor}{rgb}{0.00,0.00,0.00}

\path[draw=drawColor,line width= 0.6pt,line join=round] ( 50.61,233.22) --
	( 52.03,233.22);

\path[draw=drawColor,line width= 0.6pt,line join=round] ( 50.61,314.05) --
	( 52.03,314.05);

\path[draw=drawColor,line width= 0.6pt,line join=round] ( 50.61,376.08) --
	( 52.03,376.08);

\path[draw=drawColor,line width= 0.6pt,line join=round] ( 50.61,428.37) --
	( 52.03,428.37);
\end{scope}
\begin{scope}
\path[clip] (  0.00,  0.00) rectangle (505.89,433.62);
\definecolor[named]{drawColor}{rgb}{0.00,0.00,0.00}

\path[draw=drawColor,line width= 0.6pt,line join=round] (186.15,136.84) --
	(186.15,138.27);

\path[draw=drawColor,line width= 0.6pt,line join=round] (280.28,136.84) --
	(280.28,138.27);

\path[draw=drawColor,line width= 0.6pt,line join=round] (352.50,136.84) --
	(352.50,138.27);

\path[draw=drawColor,line width= 0.6pt,line join=round] (413.39,136.84) --
	(413.39,138.27);

\path[draw=drawColor,line width= 0.6pt,line join=round] (467.04,136.84) --
	(467.04,138.27);
\end{scope}
\begin{scope}
\path[clip] (  0.00,  0.00) rectangle (505.89,433.62);
\definecolor[named]{drawColor}{rgb}{0.00,0.00,0.00}

\node[text=drawColor,anchor=base,inner sep=0pt, outer sep=0pt, scale=  0.88] at (186.15,127.78) {1e+05};

\node[text=drawColor,anchor=base,inner sep=0pt, outer sep=0pt, scale=  0.88] at (280.28,127.78) {2e+05};

\node[text=drawColor,anchor=base,inner sep=0pt, outer sep=0pt, scale=  0.88] at (352.50,127.78) {3e+05};

\node[text=drawColor,anchor=base,inner sep=0pt, outer sep=0pt, scale=  0.88] at (413.39,127.78) {4e+05};

\node[text=drawColor,anchor=base,inner sep=0pt, outer sep=0pt, scale=  0.88] at (467.04,127.78) {5e+05};
\end{scope}
\begin{scope}
\path[clip] (  0.00,  0.00) rectangle (505.89,433.62);
\definecolor[named]{drawColor}{rgb}{0.00,0.00,0.00}

\node[text=drawColor,anchor=base,inner sep=0pt, outer sep=0pt, scale=  0.99] at (277.54,110.97) {Number of Nodes};
\end{scope}
\begin{scope}
\path[clip] (  0.00,  0.00) rectangle (505.89,433.62);
\definecolor[named]{drawColor}{rgb}{0.00,0.00,0.00}

\node[text=drawColor,rotate= 90.00,anchor=base,inner sep=0pt, outer sep=0pt, scale=  0.99] at (  6.82,285.94) {Number of Edges};
\end{scope}
\begin{scope}
\path[clip] (  0.00,  0.00) rectangle (505.89,433.62);
\definecolor[named]{drawColor}{rgb}{1.00,1.00,1.00}
\definecolor[named]{fillColor}{rgb}{0.95,0.95,0.95}

\path[draw=drawColor,line width= 0.6pt,line join=round,line cap=round,fill=fillColor] (188.71, 78.25) rectangle (217.16,106.70);
\end{scope}
\begin{scope}
\path[clip] (  0.00,  0.00) rectangle (505.89,433.62);
\definecolor[named]{fillColor}{rgb}{0.00,0.00,0.00}

\path[fill=fillColor] (202.93, 92.47) circle ( 11.07);
\end{scope}
\begin{scope}
\path[clip] (  0.00,  0.00) rectangle (505.89,433.62);
\definecolor[named]{drawColor}{rgb}{1.00,1.00,1.00}
\definecolor[named]{fillColor}{rgb}{0.95,0.95,0.95}

\path[draw=drawColor,line width= 0.6pt,line join=round,line cap=round,fill=fillColor] (188.71, 49.79) rectangle (217.16, 78.25);
\end{scope}
\begin{scope}
\path[clip] (  0.00,  0.00) rectangle (505.89,433.62);
\definecolor[named]{fillColor}{rgb}{0.00,0.00,0.00}

\path[fill=fillColor] (202.93, 81.23) --
	(217.84, 55.41) --
	(188.03, 55.41) --
	cycle;
\end{scope}
\begin{scope}
\path[clip] (  0.00,  0.00) rectangle (505.89,433.62);
\definecolor[named]{drawColor}{rgb}{1.00,1.00,1.00}
\definecolor[named]{fillColor}{rgb}{0.95,0.95,0.95}

\path[draw=drawColor,line width= 0.6pt,line join=round,line cap=round,fill=fillColor] (246.76, 78.25) rectangle (275.22,106.70);
\end{scope}
\begin{scope}
\path[clip] (  0.00,  0.00) rectangle (505.89,433.62);
\definecolor[named]{fillColor}{rgb}{0.00,0.00,0.00}

\path[fill=fillColor] (249.93, 81.41) --
	(272.06, 81.41) --
	(272.06,103.54) --
	(249.93,103.54) --
	cycle;
\end{scope}
\begin{scope}
\path[clip] (  0.00,  0.00) rectangle (505.89,433.62);
\definecolor[named]{drawColor}{rgb}{1.00,1.00,1.00}
\definecolor[named]{fillColor}{rgb}{0.95,0.95,0.95}

\path[draw=drawColor,line width= 0.6pt,line join=round,line cap=round,fill=fillColor] (246.76, 49.79) rectangle (275.22, 78.25);
\end{scope}
\begin{scope}
\path[clip] (  0.00,  0.00) rectangle (505.89,433.62);
\definecolor[named]{drawColor}{rgb}{0.00,0.00,0.00}

\path[draw=drawColor,line width= 0.4pt,line join=round,line cap=round] (245.34, 64.02) -- (276.64, 64.02);

\path[draw=drawColor,line width= 0.4pt,line join=round,line cap=round] (260.99, 48.37) -- (260.99, 79.67);
\end{scope}
\begin{scope}
\path[clip] (  0.00,  0.00) rectangle (505.89,433.62);
\definecolor[named]{drawColor}{rgb}{1.00,1.00,1.00}
\definecolor[named]{fillColor}{rgb}{0.95,0.95,0.95}

\path[draw=drawColor,line width= 0.6pt,line join=round,line cap=round,fill=fillColor] (307.86, 78.25) rectangle (336.31,106.70);
\end{scope}
\begin{scope}
\path[clip] (  0.00,  0.00) rectangle (505.89,433.62);
\definecolor[named]{drawColor}{rgb}{0.00,0.00,0.00}

\path[draw=drawColor,line width= 0.4pt,line join=round,line cap=round] (311.02, 81.41) rectangle (333.15,103.54);

\path[draw=drawColor,line width= 0.4pt,line join=round,line cap=round] (311.02, 81.41) -- (333.15,103.54);

\path[draw=drawColor,line width= 0.4pt,line join=round,line cap=round] (311.02,103.54) -- (333.15, 81.41);
\end{scope}
\begin{scope}
\path[clip] (  0.00,  0.00) rectangle (505.89,433.62);
\definecolor[named]{drawColor}{rgb}{1.00,1.00,1.00}
\definecolor[named]{fillColor}{rgb}{0.95,0.95,0.95}

\path[draw=drawColor,line width= 0.6pt,line join=round,line cap=round,fill=fillColor] (307.86, 49.79) rectangle (336.31, 78.25);
\end{scope}
\begin{scope}
\path[clip] (  0.00,  0.00) rectangle (505.89,433.62);
\definecolor[named]{drawColor}{rgb}{0.00,0.00,0.00}

\path[draw=drawColor,line width= 0.4pt,line join=round,line cap=round] (311.02, 52.95) -- (333.15, 75.08);

\path[draw=drawColor,line width= 0.4pt,line join=round,line cap=round] (311.02, 75.08) -- (333.15, 52.95);

\path[draw=drawColor,line width= 0.4pt,line join=round,line cap=round] (306.44, 64.02) -- (337.73, 64.02);

\path[draw=drawColor,line width= 0.4pt,line join=round,line cap=round] (322.09, 48.37) -- (322.09, 79.67);
\end{scope}
\begin{scope}
\path[clip] (  0.00,  0.00) rectangle (505.89,433.62);
\definecolor[named]{drawColor}{rgb}{0.00,0.00,0.00}

\node[text=drawColor,anchor=base west,inner sep=0pt, outer sep=0pt, scale=  0.82] at (218.97, 89.63) {\DAC};
\end{scope}
\begin{scope}
\path[clip] (  0.00,  0.00) rectangle (505.89,433.62);
\definecolor[named]{drawColor}{rgb}{0.00,0.00,0.00}

\node[text=drawColor,anchor=base west,inner sep=0pt, outer sep=0pt, scale=  0.82] at (218.97, 61.18) {\ISPD};
\end{scope}
\begin{scope}
\path[clip] (  0.00,  0.00) rectangle (505.89,433.62);
\definecolor[named]{drawColor}{rgb}{0.00,0.00,0.00}

\node[text=drawColor,anchor=base west,inner sep=0pt, outer sep=0pt, scale=  0.82] at (277.02, 89.63) {\Dual};
\end{scope}
\begin{scope}
\path[clip] (  0.00,  0.00) rectangle (505.89,433.62);
\definecolor[named]{drawColor}{rgb}{0.00,0.00,0.00}

\node[text=drawColor,anchor=base west,inner sep=0pt, outer sep=0pt, scale=  0.82] at (277.02, 61.18) {\Primal};
\end{scope}
\begin{scope}
\path[clip] (  0.00,  0.00) rectangle (505.89,433.62);
\definecolor[named]{drawColor}{rgb}{0.00,0.00,0.00}

\node[text=drawColor,anchor=base west,inner sep=0pt, outer sep=0pt, scale=  0.82] at (338.12, 89.63) {\Literal};
\end{scope}
\begin{scope}
\path[clip] (  0.00,  0.00) rectangle (505.89,433.62);
\definecolor[named]{drawColor}{rgb}{0.00,0.00,0.00}

\node[text=drawColor,anchor=base west,inner sep=0pt, outer sep=0pt, scale=  0.82] at (338.12, 61.18) {\SPM};
\end{scope}
\begin{scope}
\path[clip] (  0.00,  0.00) rectangle (505.89,433.62);
\definecolor[named]{drawColor}{rgb}{1.00,1.00,1.00}
\definecolor[named]{fillColor}{rgb}{0.95,0.95,0.95}

\path[draw=drawColor,line width= 0.6pt,line join=round,line cap=round,fill=fillColor] (187.86, 12.80) rectangle (216.32, 41.26);
\end{scope}
\begin{scope}
\path[clip] (  0.00,  0.00) rectangle (505.89,433.62);
\definecolor[named]{drawColor}{rgb}{0.97,0.46,0.43}
\definecolor[named]{fillColor}{rgb}{0.97,0.46,0.43}

\path[draw=drawColor,line width= 0.4pt,line join=round,line cap=round,fill=fillColor] (202.09, 27.03) circle ( 11.07);
\end{scope}
\begin{scope}
\path[clip] (  0.00,  0.00) rectangle (505.89,433.62);
\definecolor[named]{drawColor}{rgb}{1.00,1.00,1.00}
\definecolor[named]{fillColor}{rgb}{0.95,0.95,0.95}

\path[draw=drawColor,line width= 0.6pt,line join=round,line cap=round,fill=fillColor] (229.69, 12.80) rectangle (258.14, 41.26);
\end{scope}
\begin{scope}
\path[clip] (  0.00,  0.00) rectangle (505.89,433.62);
\definecolor[named]{drawColor}{rgb}{0.49,0.68,0.00}
\definecolor[named]{fillColor}{rgb}{0.49,0.68,0.00}

\path[draw=drawColor,line width= 0.4pt,line join=round,line cap=round,fill=fillColor] (243.92, 27.03) circle ( 11.07);
\end{scope}
\begin{scope}
\path[clip] (  0.00,  0.00) rectangle (505.89,433.62);
\definecolor[named]{drawColor}{rgb}{1.00,1.00,1.00}
\definecolor[named]{fillColor}{rgb}{0.95,0.95,0.95}

\path[draw=drawColor,line width= 0.6pt,line join=round,line cap=round,fill=fillColor] (272.83, 12.80) rectangle (301.29, 41.26);
\end{scope}
\begin{scope}
\path[clip] (  0.00,  0.00) rectangle (505.89,433.62);
\definecolor[named]{drawColor}{rgb}{0.00,0.75,0.77}
\definecolor[named]{fillColor}{rgb}{0.00,0.75,0.77}

\path[draw=drawColor,line width= 0.4pt,line join=round,line cap=round,fill=fillColor] (287.06, 27.03) circle ( 11.07);
\end{scope}
\begin{scope}
\path[clip] (  0.00,  0.00) rectangle (505.89,433.62);
\definecolor[named]{drawColor}{rgb}{1.00,1.00,1.00}
\definecolor[named]{fillColor}{rgb}{0.95,0.95,0.95}

\path[draw=drawColor,line width= 0.6pt,line join=round,line cap=round,fill=fillColor] (315.27, 12.80) rectangle (343.72, 41.26);
\end{scope}
\begin{scope}
\path[clip] (  0.00,  0.00) rectangle (505.89,433.62);
\definecolor[named]{drawColor}{rgb}{0.78,0.49,1.00}
\definecolor[named]{fillColor}{rgb}{0.78,0.49,1.00}

\path[draw=drawColor,line width= 0.4pt,line join=round,line cap=round,fill=fillColor] (329.50, 27.03) circle ( 11.07);
\end{scope}
\begin{scope}
\path[clip] (  0.00,  0.00) rectangle (505.89,433.62);
\definecolor[named]{drawColor}{rgb}{0.00,0.00,0.00}

\node[text=drawColor,anchor=base west,inner sep=0pt, outer sep=0pt, scale=  0.82] at (218.12, 24.19) {$\ExpLawler$};
\end{scope}
\begin{scope}
\path[clip] (  0.00,  0.00) rectangle (505.89,433.62);
\definecolor[named]{drawColor}{rgb}{0.00,0.00,0.00}

\node[text=drawColor,anchor=base west,inner sep=0pt, outer sep=0pt, scale=  0.82] at (259.95, 24.19) {$\ExpNodeDegree$};
\end{scope}
\begin{scope}
\path[clip] (  0.00,  0.00) rectangle (505.89,433.62);
\definecolor[named]{drawColor}{rgb}{0.00,0.00,0.00}

\node[text=drawColor,anchor=base west,inner sep=0pt, outer sep=0pt, scale=  0.82] at (303.09, 24.19) {$\ExpEdgeSize$};
\end{scope}
\begin{scope}
\path[clip] (  0.00,  0.00) rectangle (505.89,433.62);
\definecolor[named]{drawColor}{rgb}{0.00,0.00,0.00}

\node[text=drawColor,anchor=base west,inner sep=0pt, outer sep=0pt, scale=  0.82] at (345.53, 24.19) {$\ExpHybrid$};
\end{scope}
\end{tikzpicture}
 %
\caption{Comparison of the number of nodes and edges on our flow networks for 
         flow problems of size $|V'| = 25000$ hypernodes on different benchmark types.
         The red dashed lines indicates $25000$ nodes.}
\label{fig:node_edge_distribution}
\end{figure} 
In \autoref{fig:max_flow_network_algo} we compare the performance of our flow algorithms on
different flow networks. The bars in the plot indicates speed ups relative to the flow algorithm
\EdmondKarp~on flow network $\ExpLawler$. The main observation is that \EdmondKarp~performs
better on small flow network instances and \GoldbergTarjan~on large flow network instances. For $|V'| \le 1000$
\EdmondKarp~is faster than \GoldbergTarjan~in most of the different bechmark types. For
$|V'| > 1000$ we can observe the opposite behaviour except for \DAC~and \Dual~instances. But the
resulting flow problems of these instances are still the smallest among all benchmark types
(see \autoref{fig:node_edge_distribution}). On the largest flow network instances \Primal~and
\Literal~for $|V'| = 25000$ \GoldbergTarjan~is up to a factor of $4$-$7$ faster than \EdmondKarp.
Further, both algorithms perform best on $\ExpHybrid$.
\begin{figure}
\centering
% Created by tikzDevice version 0.6.2-92-0ad2792 on 2017-12-31 07:49:08
% !TEX encoding = UTF-8 Unicode
\begin{tikzpicture}[x=1pt,y=1pt]
\definecolor[named]{fillColor}{rgb}{1.00,1.00,1.00}
\path[use as bounding box,fill=fillColor,fill opacity=0.00] (0,0) rectangle (505.89,578.16);
\begin{scope}
\path[clip] (  0.00,  0.00) rectangle (505.89,578.16);
\definecolor[named]{fillColor}{rgb}{1.00,1.00,1.00}

\path[fill=fillColor] (  0.00,  0.00) rectangle (505.89,578.16);
\end{scope}
\begin{scope}
\path[clip] ( 32.48,560.58) rectangle (118.67,578.16);
\definecolor[named]{drawColor}{rgb}{0.90,0.90,0.90}
\definecolor[named]{fillColor}{rgb}{0.90,0.90,0.90}

\path[draw=drawColor,line width= 0.6pt,line join=round,line cap=round,fill=fillColor] ( 32.48,560.58) rectangle (118.67,578.16);
\definecolor[named]{drawColor}{rgb}{0.00,0.00,0.00}

\node[text=drawColor,anchor=base,inner sep=0pt, outer sep=0pt, scale=  1.10] at ( 75.58,565.58) {$500$};
\end{scope}
\begin{scope}
\path[clip] (121.68,560.58) rectangle (207.87,578.16);
\definecolor[named]{drawColor}{rgb}{0.90,0.90,0.90}
\definecolor[named]{fillColor}{rgb}{0.90,0.90,0.90}

\path[draw=drawColor,line width= 0.6pt,line join=round,line cap=round,fill=fillColor] (121.68,560.58) rectangle (207.87,578.16);
\definecolor[named]{drawColor}{rgb}{0.00,0.00,0.00}

\node[text=drawColor,anchor=base,inner sep=0pt, outer sep=0pt, scale=  1.10] at (164.78,565.58) {$1000$};
\end{scope}
\begin{scope}
\path[clip] (210.88,560.58) rectangle (297.07,578.16);
\definecolor[named]{drawColor}{rgb}{0.90,0.90,0.90}
\definecolor[named]{fillColor}{rgb}{0.90,0.90,0.90}

\path[draw=drawColor,line width= 0.6pt,line join=round,line cap=round,fill=fillColor] (210.88,560.58) rectangle (297.07,578.16);
\definecolor[named]{drawColor}{rgb}{0.00,0.00,0.00}

\node[text=drawColor,anchor=base,inner sep=0pt, outer sep=0pt, scale=  1.10] at (253.98,565.58) {$5000$};
\end{scope}
\begin{scope}
\path[clip] (300.08,560.58) rectangle (386.27,578.16);
\definecolor[named]{drawColor}{rgb}{0.90,0.90,0.90}
\definecolor[named]{fillColor}{rgb}{0.90,0.90,0.90}

\path[draw=drawColor,line width= 0.6pt,line join=round,line cap=round,fill=fillColor] (300.08,560.58) rectangle (386.27,578.16);
\definecolor[named]{drawColor}{rgb}{0.00,0.00,0.00}

\node[text=drawColor,anchor=base,inner sep=0pt, outer sep=0pt, scale=  1.10] at (343.18,565.58) {$10000$};
\end{scope}
\begin{scope}
\path[clip] (389.28,560.58) rectangle (475.47,578.16);
\definecolor[named]{drawColor}{rgb}{0.90,0.90,0.90}
\definecolor[named]{fillColor}{rgb}{0.90,0.90,0.90}

\path[draw=drawColor,line width= 0.6pt,line join=round,line cap=round,fill=fillColor] (389.28,560.58) rectangle (475.47,578.16);
\definecolor[named]{drawColor}{rgb}{0.00,0.00,0.00}

\node[text=drawColor,anchor=base,inner sep=0pt, outer sep=0pt, scale=  1.10] at (432.37,565.58) {$25000$};
\end{scope}
\begin{scope}
\path[clip] ( 32.48,479.43) rectangle (118.67,560.58);
\definecolor[named]{drawColor}{rgb}{0.75,0.75,0.75}

\path[draw=drawColor,line width= 1.5pt,line join=round,line cap=round] ( 32.48,479.43) rectangle (118.67,560.58);
\definecolor[named]{drawColor}{rgb}{0.90,0.90,0.90}

\path[draw=drawColor,line width= 0.3pt,line join=round] ( 32.48,490.37) --
	(118.67,490.37);

\path[draw=drawColor,line width= 0.3pt,line join=round] ( 32.48,504.87) --
	(118.67,504.87);

\path[draw=drawColor,line width= 0.3pt,line join=round] ( 32.48,519.37) --
	(118.67,519.37);

\path[draw=drawColor,line width= 0.3pt,line join=round] ( 32.48,533.87) --
	(118.67,533.87);

\path[draw=drawColor,line width= 0.3pt,line join=round] ( 32.48,548.37) --
	(118.67,548.37);

\path[draw=drawColor,line width= 0.8pt,line join=round] ( 32.48,483.12) --
	(118.67,483.12);

\path[draw=drawColor,line width= 0.8pt,line join=round] ( 32.48,497.62) --
	(118.67,497.62);

\path[draw=drawColor,line width= 0.8pt,line join=round] ( 32.48,512.12) --
	(118.67,512.12);

\path[draw=drawColor,line width= 0.8pt,line join=round] ( 32.48,526.62) --
	(118.67,526.62);

\path[draw=drawColor,line width= 0.8pt,line join=round] ( 32.48,541.12) --
	(118.67,541.12);

\path[draw=drawColor,line width= 0.8pt,line join=round] ( 32.48,555.62) --
	(118.67,555.62);

\path[draw=drawColor,line width= 0.8pt,line join=round] ( 48.64,479.43) --
	( 48.64,560.58);

\path[draw=drawColor,line width= 0.8pt,line join=round] ( 75.58,479.43) --
	( 75.58,560.58);

\path[draw=drawColor,line width= 0.8pt,line join=round] (102.51,479.43) --
	(102.51,560.58);
\definecolor[named]{fillColor}{rgb}{0.97,0.46,0.43}

\path[fill=fillColor] ( 36.52,483.12) rectangle ( 42.58,499.11);
\definecolor[named]{fillColor}{rgb}{0.49,0.68,0.00}

\path[fill=fillColor] ( 42.58,483.12) rectangle ( 48.64,503.27);
\definecolor[named]{fillColor}{rgb}{0.00,0.75,0.77}

\path[fill=fillColor] ( 48.64,483.12) rectangle ( 54.70,498.59);
\definecolor[named]{fillColor}{rgb}{0.78,0.49,1.00}

\path[fill=fillColor] ( 54.70,483.12) rectangle ( 60.76,498.18);
\definecolor[named]{fillColor}{rgb}{0.97,0.46,0.43}

\path[fill=fillColor] ( 63.46,483.12) rectangle ( 69.52,505.63);
\definecolor[named]{fillColor}{rgb}{0.49,0.68,0.00}

\path[fill=fillColor] ( 69.52,483.12) rectangle ( 75.58,526.49);
\definecolor[named]{fillColor}{rgb}{0.00,0.75,0.77}

\path[fill=fillColor] ( 75.58,483.12) rectangle ( 81.64,505.47);
\definecolor[named]{fillColor}{rgb}{0.78,0.49,1.00}

\path[fill=fillColor] ( 81.64,483.12) rectangle ( 87.70,505.34);
\definecolor[named]{fillColor}{rgb}{0.97,0.46,0.43}

\path[fill=fillColor] ( 90.39,483.12) rectangle ( 96.45,505.76);
\definecolor[named]{fillColor}{rgb}{0.49,0.68,0.00}

\path[fill=fillColor] ( 96.45,483.12) rectangle (102.51,526.11);
\definecolor[named]{fillColor}{rgb}{0.00,0.75,0.77}

\path[fill=fillColor] (102.51,483.12) rectangle (108.57,505.44);
\definecolor[named]{fillColor}{rgb}{0.78,0.49,1.00}

\path[fill=fillColor] (108.57,483.12) rectangle (114.63,505.43);
\definecolor[named]{drawColor}{rgb}{1.00,0.00,0.00}
\definecolor[named]{fillColor}{rgb}{1.00,0.00,0.00}

\path[draw=drawColor,line width= 0.6pt,dash pattern=on 4pt off 4pt ,line join=round,fill=fillColor] ( 32.48,497.62) -- (118.67,497.62);
\definecolor[named]{drawColor}{rgb}{0.00,0.00,1.00}
\definecolor[named]{fillColor}{rgb}{0.00,0.00,1.00}

\path[draw=drawColor,line width= 0.6pt,dash pattern=on 4pt off 4pt ,line join=round,fill=fillColor] ( 32.48,512.12) -- (118.67,512.12);
\end{scope}
\begin{scope}
\path[clip] ( 32.48,395.26) rectangle (118.67,476.42);
\definecolor[named]{drawColor}{rgb}{0.75,0.75,0.75}

\path[draw=drawColor,line width= 1.5pt,line join=round,line cap=round] ( 32.48,395.26) rectangle (118.67,476.42);
\definecolor[named]{drawColor}{rgb}{0.90,0.90,0.90}

\path[draw=drawColor,line width= 0.3pt,line join=round] ( 32.48,407.91) --
	(118.67,407.91);

\path[draw=drawColor,line width= 0.3pt,line join=round] ( 32.48,425.84) --
	(118.67,425.84);

\path[draw=drawColor,line width= 0.3pt,line join=round] ( 32.48,443.76) --
	(118.67,443.76);

\path[draw=drawColor,line width= 0.3pt,line join=round] ( 32.48,461.69) --
	(118.67,461.69);

\path[draw=drawColor,line width= 0.8pt,line join=round] ( 32.48,398.95) --
	(118.67,398.95);

\path[draw=drawColor,line width= 0.8pt,line join=round] ( 32.48,416.88) --
	(118.67,416.88);

\path[draw=drawColor,line width= 0.8pt,line join=round] ( 32.48,434.80) --
	(118.67,434.80);

\path[draw=drawColor,line width= 0.8pt,line join=round] ( 32.48,452.73) --
	(118.67,452.73);

\path[draw=drawColor,line width= 0.8pt,line join=round] ( 32.48,470.65) --
	(118.67,470.65);

\path[draw=drawColor,line width= 0.8pt,line join=round] ( 48.64,395.26) --
	( 48.64,476.42);

\path[draw=drawColor,line width= 0.8pt,line join=round] ( 75.58,395.26) --
	( 75.58,476.42);

\path[draw=drawColor,line width= 0.8pt,line join=round] (102.51,395.26) --
	(102.51,476.42);
\definecolor[named]{fillColor}{rgb}{0.97,0.46,0.43}

\path[fill=fillColor] ( 36.52,398.95) rectangle ( 42.58,418.69);
\definecolor[named]{fillColor}{rgb}{0.49,0.68,0.00}

\path[fill=fillColor] ( 42.58,398.95) rectangle ( 48.64,420.27);
\definecolor[named]{fillColor}{rgb}{0.00,0.75,0.77}

\path[fill=fillColor] ( 48.64,398.95) rectangle ( 54.70,418.59);
\definecolor[named]{fillColor}{rgb}{0.78,0.49,1.00}

\path[fill=fillColor] ( 54.70,398.95) rectangle ( 60.76,417.69);
\definecolor[named]{fillColor}{rgb}{0.97,0.46,0.43}

\path[fill=fillColor] ( 63.46,398.95) rectangle ( 69.52,426.15);
\definecolor[named]{fillColor}{rgb}{0.49,0.68,0.00}

\path[fill=fillColor] ( 69.52,398.95) rectangle ( 75.58,448.34);
\definecolor[named]{fillColor}{rgb}{0.00,0.75,0.77}

\path[fill=fillColor] ( 75.58,398.95) rectangle ( 81.64,426.77);
\definecolor[named]{fillColor}{rgb}{0.78,0.49,1.00}

\path[fill=fillColor] ( 81.64,398.95) rectangle ( 87.70,426.18);
\definecolor[named]{fillColor}{rgb}{0.97,0.46,0.43}

\path[fill=fillColor] ( 90.39,398.95) rectangle ( 96.45,426.03);
\definecolor[named]{fillColor}{rgb}{0.49,0.68,0.00}

\path[fill=fillColor] ( 96.45,398.95) rectangle (102.51,448.25);
\definecolor[named]{fillColor}{rgb}{0.00,0.75,0.77}

\path[fill=fillColor] (102.51,398.95) rectangle (108.57,426.78);
\definecolor[named]{fillColor}{rgb}{0.78,0.49,1.00}

\path[fill=fillColor] (108.57,398.95) rectangle (114.63,426.26);
\definecolor[named]{drawColor}{rgb}{1.00,0.00,0.00}
\definecolor[named]{fillColor}{rgb}{1.00,0.00,0.00}

\path[draw=drawColor,line width= 0.6pt,dash pattern=on 4pt off 4pt ,line join=round,fill=fillColor] ( 32.48,416.88) -- (118.67,416.88);
\definecolor[named]{drawColor}{rgb}{0.00,0.00,1.00}
\definecolor[named]{fillColor}{rgb}{0.00,0.00,1.00}

\path[draw=drawColor,line width= 0.6pt,dash pattern=on 4pt off 4pt ,line join=round,fill=fillColor] ( 32.48,434.80) -- (118.67,434.80);
\end{scope}
\begin{scope}
\path[clip] ( 32.48,311.10) rectangle (118.67,392.25);
\definecolor[named]{drawColor}{rgb}{0.75,0.75,0.75}

\path[draw=drawColor,line width= 1.5pt,line join=round,line cap=round] ( 32.48,311.10) rectangle (118.67,392.25);
\definecolor[named]{drawColor}{rgb}{0.90,0.90,0.90}

\path[draw=drawColor,line width= 0.3pt,line join=round] ( 32.48,323.60) --
	(118.67,323.60);

\path[draw=drawColor,line width= 0.3pt,line join=round] ( 32.48,341.24) --
	(118.67,341.24);

\path[draw=drawColor,line width= 0.3pt,line join=round] ( 32.48,358.87) --
	(118.67,358.87);

\path[draw=drawColor,line width= 0.3pt,line join=round] ( 32.48,376.51) --
	(118.67,376.51);

\path[draw=drawColor,line width= 0.8pt,line join=round] ( 32.48,314.79) --
	(118.67,314.79);

\path[draw=drawColor,line width= 0.8pt,line join=round] ( 32.48,332.42) --
	(118.67,332.42);

\path[draw=drawColor,line width= 0.8pt,line join=round] ( 32.48,350.06) --
	(118.67,350.06);

\path[draw=drawColor,line width= 0.8pt,line join=round] ( 32.48,367.69) --
	(118.67,367.69);

\path[draw=drawColor,line width= 0.8pt,line join=round] ( 32.48,385.33) --
	(118.67,385.33);

\path[draw=drawColor,line width= 0.8pt,line join=round] ( 48.64,311.10) --
	( 48.64,392.25);

\path[draw=drawColor,line width= 0.8pt,line join=round] ( 75.58,311.10) --
	( 75.58,392.25);

\path[draw=drawColor,line width= 0.8pt,line join=round] (102.51,311.10) --
	(102.51,392.25);
\definecolor[named]{fillColor}{rgb}{0.97,0.46,0.43}

\path[fill=fillColor] ( 36.52,314.79) rectangle ( 42.58,335.38);
\definecolor[named]{fillColor}{rgb}{0.49,0.68,0.00}

\path[fill=fillColor] ( 42.58,314.79) rectangle ( 48.64,344.02);
\definecolor[named]{fillColor}{rgb}{0.00,0.75,0.77}

\path[fill=fillColor] ( 48.64,314.79) rectangle ( 54.70,335.55);
\definecolor[named]{fillColor}{rgb}{0.78,0.49,1.00}

\path[fill=fillColor] ( 54.70,314.79) rectangle ( 60.76,335.00);
\definecolor[named]{fillColor}{rgb}{0.97,0.46,0.43}

\path[fill=fillColor] ( 63.46,314.79) rectangle ( 69.52,339.06);
\definecolor[named]{fillColor}{rgb}{0.49,0.68,0.00}

\path[fill=fillColor] ( 69.52,314.79) rectangle ( 75.58,348.50);
\definecolor[named]{fillColor}{rgb}{0.00,0.75,0.77}

\path[fill=fillColor] ( 75.58,314.79) rectangle ( 81.64,339.81);
\definecolor[named]{fillColor}{rgb}{0.78,0.49,1.00}

\path[fill=fillColor] ( 81.64,314.79) rectangle ( 87.70,339.16);
\definecolor[named]{fillColor}{rgb}{0.97,0.46,0.43}

\path[fill=fillColor] ( 90.39,314.79) rectangle ( 96.45,339.96);
\definecolor[named]{fillColor}{rgb}{0.49,0.68,0.00}

\path[fill=fillColor] ( 96.45,314.79) rectangle (102.51,350.90);
\definecolor[named]{fillColor}{rgb}{0.00,0.75,0.77}

\path[fill=fillColor] (102.51,314.79) rectangle (108.57,340.78);
\definecolor[named]{fillColor}{rgb}{0.78,0.49,1.00}

\path[fill=fillColor] (108.57,314.79) rectangle (114.63,340.40);
\definecolor[named]{drawColor}{rgb}{1.00,0.00,0.00}
\definecolor[named]{fillColor}{rgb}{1.00,0.00,0.00}

\path[draw=drawColor,line width= 0.6pt,dash pattern=on 4pt off 4pt ,line join=round,fill=fillColor] ( 32.48,332.42) -- (118.67,332.42);
\definecolor[named]{drawColor}{rgb}{0.00,0.00,1.00}
\definecolor[named]{fillColor}{rgb}{0.00,0.00,1.00}

\path[draw=drawColor,line width= 0.6pt,dash pattern=on 4pt off 4pt ,line join=round,fill=fillColor] ( 32.48,350.06) -- (118.67,350.06);
\end{scope}
\begin{scope}
\path[clip] ( 32.48,226.93) rectangle (118.67,308.09);
\definecolor[named]{drawColor}{rgb}{0.75,0.75,0.75}

\path[draw=drawColor,line width= 1.5pt,line join=round,line cap=round] ( 32.48,226.93) rectangle (118.67,308.09);
\definecolor[named]{drawColor}{rgb}{0.90,0.90,0.90}

\path[draw=drawColor,line width= 0.3pt,line join=round] ( 32.48,240.40) --
	(118.67,240.40);

\path[draw=drawColor,line width= 0.3pt,line join=round] ( 32.48,259.96) --
	(118.67,259.96);

\path[draw=drawColor,line width= 0.3pt,line join=round] ( 32.48,279.52) --
	(118.67,279.52);

\path[draw=drawColor,line width= 0.3pt,line join=round] ( 32.48,299.08) --
	(118.67,299.08);

\path[draw=drawColor,line width= 0.8pt,line join=round] ( 32.48,230.62) --
	(118.67,230.62);

\path[draw=drawColor,line width= 0.8pt,line join=round] ( 32.48,250.18) --
	(118.67,250.18);

\path[draw=drawColor,line width= 0.8pt,line join=round] ( 32.48,269.74) --
	(118.67,269.74);

\path[draw=drawColor,line width= 0.8pt,line join=round] ( 32.48,289.30) --
	(118.67,289.30);

\path[draw=drawColor,line width= 0.8pt,line join=round] ( 48.64,226.93) --
	( 48.64,308.09);

\path[draw=drawColor,line width= 0.8pt,line join=round] ( 75.58,226.93) --
	( 75.58,308.09);

\path[draw=drawColor,line width= 0.8pt,line join=round] (102.51,226.93) --
	(102.51,308.09);
\definecolor[named]{fillColor}{rgb}{0.97,0.46,0.43}

\path[fill=fillColor] ( 36.52,230.62) rectangle ( 42.58,250.57);
\definecolor[named]{fillColor}{rgb}{0.49,0.68,0.00}

\path[fill=fillColor] ( 42.58,230.62) rectangle ( 48.64,249.43);
\definecolor[named]{fillColor}{rgb}{0.00,0.75,0.77}

\path[fill=fillColor] ( 48.64,230.62) rectangle ( 54.70,250.70);
\definecolor[named]{fillColor}{rgb}{0.78,0.49,1.00}

\path[fill=fillColor] ( 54.70,230.62) rectangle ( 60.76,250.46);
\definecolor[named]{fillColor}{rgb}{0.97,0.46,0.43}

\path[fill=fillColor] ( 63.46,230.62) rectangle ( 69.52,276.76);
\definecolor[named]{fillColor}{rgb}{0.49,0.68,0.00}

\path[fill=fillColor] ( 69.52,230.62) rectangle ( 75.58,294.82);
\definecolor[named]{fillColor}{rgb}{0.00,0.75,0.77}

\path[fill=fillColor] ( 75.58,230.62) rectangle ( 81.64,270.85);
\definecolor[named]{fillColor}{rgb}{0.78,0.49,1.00}

\path[fill=fillColor] ( 81.64,230.62) rectangle ( 87.70,274.04);
\definecolor[named]{fillColor}{rgb}{0.97,0.46,0.43}

\path[fill=fillColor] ( 90.39,230.62) rectangle ( 96.45,277.11);
\definecolor[named]{fillColor}{rgb}{0.49,0.68,0.00}

\path[fill=fillColor] ( 96.45,230.62) rectangle (102.51,294.10);
\definecolor[named]{fillColor}{rgb}{0.00,0.75,0.77}

\path[fill=fillColor] (102.51,230.62) rectangle (108.57,271.44);
\definecolor[named]{fillColor}{rgb}{0.78,0.49,1.00}

\path[fill=fillColor] (108.57,230.62) rectangle (114.63,274.23);
\definecolor[named]{drawColor}{rgb}{1.00,0.00,0.00}
\definecolor[named]{fillColor}{rgb}{1.00,0.00,0.00}

\path[draw=drawColor,line width= 0.6pt,dash pattern=on 4pt off 4pt ,line join=round,fill=fillColor] ( 32.48,250.18) -- (118.67,250.18);
\definecolor[named]{drawColor}{rgb}{0.00,0.00,1.00}
\definecolor[named]{fillColor}{rgb}{0.00,0.00,1.00}

\path[draw=drawColor,line width= 0.6pt,dash pattern=on 4pt off 4pt ,line join=round,fill=fillColor] ( 32.48,269.74) -- (118.67,269.74);
\end{scope}
\begin{scope}
\path[clip] ( 32.48,142.77) rectangle (118.67,223.92);
\definecolor[named]{drawColor}{rgb}{0.75,0.75,0.75}

\path[draw=drawColor,line width= 1.5pt,line join=round,line cap=round] ( 32.48,142.77) rectangle (118.67,223.92);
\definecolor[named]{drawColor}{rgb}{0.90,0.90,0.90}

\path[draw=drawColor,line width= 0.3pt,line join=round] ( 32.48,154.09) --
	(118.67,154.09);

\path[draw=drawColor,line width= 0.3pt,line join=round] ( 32.48,169.35) --
	(118.67,169.35);

\path[draw=drawColor,line width= 0.3pt,line join=round] ( 32.48,184.61) --
	(118.67,184.61);

\path[draw=drawColor,line width= 0.3pt,line join=round] ( 32.48,199.88) --
	(118.67,199.88);

\path[draw=drawColor,line width= 0.3pt,line join=round] ( 32.48,215.14) --
	(118.67,215.14);

\path[draw=drawColor,line width= 0.8pt,line join=round] ( 32.48,146.45) --
	(118.67,146.45);

\path[draw=drawColor,line width= 0.8pt,line join=round] ( 32.48,161.72) --
	(118.67,161.72);

\path[draw=drawColor,line width= 0.8pt,line join=round] ( 32.48,176.98) --
	(118.67,176.98);

\path[draw=drawColor,line width= 0.8pt,line join=round] ( 32.48,192.24) --
	(118.67,192.24);

\path[draw=drawColor,line width= 0.8pt,line join=round] ( 32.48,207.51) --
	(118.67,207.51);

\path[draw=drawColor,line width= 0.8pt,line join=round] ( 32.48,222.77) --
	(118.67,222.77);

\path[draw=drawColor,line width= 0.8pt,line join=round] ( 48.64,142.77) --
	( 48.64,223.92);

\path[draw=drawColor,line width= 0.8pt,line join=round] ( 75.58,142.77) --
	( 75.58,223.92);

\path[draw=drawColor,line width= 0.8pt,line join=round] (102.51,142.77) --
	(102.51,223.92);
\definecolor[named]{fillColor}{rgb}{0.97,0.46,0.43}

\path[fill=fillColor] ( 36.52,146.45) rectangle ( 42.58,162.79);
\definecolor[named]{fillColor}{rgb}{0.49,0.68,0.00}

\path[fill=fillColor] ( 42.58,146.45) rectangle ( 48.64,163.55);
\definecolor[named]{fillColor}{rgb}{0.00,0.75,0.77}

\path[fill=fillColor] ( 48.64,146.45) rectangle ( 54.70,162.17);
\definecolor[named]{fillColor}{rgb}{0.78,0.49,1.00}

\path[fill=fillColor] ( 54.70,146.45) rectangle ( 60.76,162.31);
\definecolor[named]{fillColor}{rgb}{0.97,0.46,0.43}

\path[fill=fillColor] ( 63.46,146.45) rectangle ( 69.52,177.78);
\definecolor[named]{fillColor}{rgb}{0.49,0.68,0.00}

\path[fill=fillColor] ( 69.52,146.45) rectangle ( 75.58,197.12);
\definecolor[named]{fillColor}{rgb}{0.00,0.75,0.77}

\path[fill=fillColor] ( 75.58,146.45) rectangle ( 81.64,176.74);
\definecolor[named]{fillColor}{rgb}{0.78,0.49,1.00}

\path[fill=fillColor] ( 81.64,146.45) rectangle ( 87.70,178.06);
\definecolor[named]{fillColor}{rgb}{0.97,0.46,0.43}

\path[fill=fillColor] ( 90.39,146.45) rectangle ( 96.45,178.22);
\definecolor[named]{fillColor}{rgb}{0.49,0.68,0.00}

\path[fill=fillColor] ( 96.45,146.45) rectangle (102.51,196.62);
\definecolor[named]{fillColor}{rgb}{0.00,0.75,0.77}

\path[fill=fillColor] (102.51,146.45) rectangle (108.57,176.93);
\definecolor[named]{fillColor}{rgb}{0.78,0.49,1.00}

\path[fill=fillColor] (108.57,146.45) rectangle (114.63,178.34);
\definecolor[named]{drawColor}{rgb}{1.00,0.00,0.00}
\definecolor[named]{fillColor}{rgb}{1.00,0.00,0.00}

\path[draw=drawColor,line width= 0.6pt,dash pattern=on 4pt off 4pt ,line join=round,fill=fillColor] ( 32.48,161.72) -- (118.67,161.72);
\definecolor[named]{drawColor}{rgb}{0.00,0.00,1.00}
\definecolor[named]{fillColor}{rgb}{0.00,0.00,1.00}

\path[draw=drawColor,line width= 0.6pt,dash pattern=on 4pt off 4pt ,line join=round,fill=fillColor] ( 32.48,176.98) -- (118.67,176.98);
\end{scope}
\begin{scope}
\path[clip] ( 32.48, 58.60) rectangle (118.67,139.75);
\definecolor[named]{drawColor}{rgb}{0.75,0.75,0.75}

\path[draw=drawColor,line width= 1.5pt,line join=round,line cap=round] ( 32.48, 58.60) rectangle (118.67,139.75);
\definecolor[named]{drawColor}{rgb}{0.90,0.90,0.90}

\path[draw=drawColor,line width= 0.3pt,line join=round] ( 32.48, 71.51) --
	(118.67, 71.51);

\path[draw=drawColor,line width= 0.3pt,line join=round] ( 32.48, 89.96) --
	(118.67, 89.96);

\path[draw=drawColor,line width= 0.3pt,line join=round] ( 32.48,108.40) --
	(118.67,108.40);

\path[draw=drawColor,line width= 0.3pt,line join=round] ( 32.48,126.84) --
	(118.67,126.84);

\path[draw=drawColor,line width= 0.8pt,line join=round] ( 32.48, 62.29) --
	(118.67, 62.29);

\path[draw=drawColor,line width= 0.8pt,line join=round] ( 32.48, 80.73) --
	(118.67, 80.73);

\path[draw=drawColor,line width= 0.8pt,line join=round] ( 32.48, 99.18) --
	(118.67, 99.18);

\path[draw=drawColor,line width= 0.8pt,line join=round] ( 32.48,117.62) --
	(118.67,117.62);

\path[draw=drawColor,line width= 0.8pt,line join=round] ( 32.48,136.07) --
	(118.67,136.07);

\path[draw=drawColor,line width= 0.8pt,line join=round] ( 48.64, 58.60) --
	( 48.64,139.75);

\path[draw=drawColor,line width= 0.8pt,line join=round] ( 75.58, 58.60) --
	( 75.58,139.75);

\path[draw=drawColor,line width= 0.8pt,line join=round] (102.51, 58.60) --
	(102.51,139.75);
\definecolor[named]{fillColor}{rgb}{0.97,0.46,0.43}

\path[fill=fillColor] ( 36.52, 62.29) rectangle ( 42.58,103.34);
\definecolor[named]{fillColor}{rgb}{0.49,0.68,0.00}

\path[fill=fillColor] ( 42.58, 62.29) rectangle ( 48.64,100.14);
\definecolor[named]{fillColor}{rgb}{0.00,0.75,0.77}

\path[fill=fillColor] ( 48.64, 62.29) rectangle ( 54.70,103.37);
\definecolor[named]{fillColor}{rgb}{0.78,0.49,1.00}

\path[fill=fillColor] ( 54.70, 62.29) rectangle ( 60.76,100.47);
\definecolor[named]{fillColor}{rgb}{0.97,0.46,0.43}

\path[fill=fillColor] ( 63.46, 62.29) rectangle ( 69.52,110.49);
\definecolor[named]{fillColor}{rgb}{0.49,0.68,0.00}

\path[fill=fillColor] ( 69.52, 62.29) rectangle ( 75.58,120.56);
\definecolor[named]{fillColor}{rgb}{0.00,0.75,0.77}

\path[fill=fillColor] ( 75.58, 62.29) rectangle ( 81.64,107.92);
\definecolor[named]{fillColor}{rgb}{0.78,0.49,1.00}

\path[fill=fillColor] ( 81.64, 62.29) rectangle ( 87.70,108.21);
\definecolor[named]{fillColor}{rgb}{0.97,0.46,0.43}

\path[fill=fillColor] ( 90.39, 62.29) rectangle ( 96.45,110.00);
\definecolor[named]{fillColor}{rgb}{0.49,0.68,0.00}

\path[fill=fillColor] ( 96.45, 62.29) rectangle (102.51,119.94);
\definecolor[named]{fillColor}{rgb}{0.00,0.75,0.77}

\path[fill=fillColor] (102.51, 62.29) rectangle (108.57,107.65);
\definecolor[named]{fillColor}{rgb}{0.78,0.49,1.00}

\path[fill=fillColor] (108.57, 62.29) rectangle (114.63,109.13);
\definecolor[named]{drawColor}{rgb}{1.00,0.00,0.00}
\definecolor[named]{fillColor}{rgb}{1.00,0.00,0.00}

\path[draw=drawColor,line width= 0.6pt,dash pattern=on 4pt off 4pt ,line join=round,fill=fillColor] ( 32.48, 99.18) -- (118.67, 99.18);
\definecolor[named]{drawColor}{rgb}{0.00,0.00,1.00}
\definecolor[named]{fillColor}{rgb}{0.00,0.00,1.00}

\path[draw=drawColor,line width= 0.6pt,dash pattern=on 4pt off 4pt ,line join=round,fill=fillColor] ( 32.48,136.07) -- (118.67,136.07);
\end{scope}
\begin{scope}
\path[clip] (121.68,479.43) rectangle (207.87,560.58);
\definecolor[named]{drawColor}{rgb}{0.75,0.75,0.75}

\path[draw=drawColor,line width= 1.5pt,line join=round,line cap=round] (121.68,479.43) rectangle (207.87,560.58);
\definecolor[named]{drawColor}{rgb}{0.90,0.90,0.90}

\path[draw=drawColor,line width= 0.3pt,line join=round] (121.68,490.37) --
	(207.87,490.37);

\path[draw=drawColor,line width= 0.3pt,line join=round] (121.68,504.87) --
	(207.87,504.87);

\path[draw=drawColor,line width= 0.3pt,line join=round] (121.68,519.37) --
	(207.87,519.37);

\path[draw=drawColor,line width= 0.3pt,line join=round] (121.68,533.87) --
	(207.87,533.87);

\path[draw=drawColor,line width= 0.3pt,line join=round] (121.68,548.37) --
	(207.87,548.37);

\path[draw=drawColor,line width= 0.8pt,line join=round] (121.68,483.12) --
	(207.87,483.12);

\path[draw=drawColor,line width= 0.8pt,line join=round] (121.68,497.62) --
	(207.87,497.62);

\path[draw=drawColor,line width= 0.8pt,line join=round] (121.68,512.12) --
	(207.87,512.12);

\path[draw=drawColor,line width= 0.8pt,line join=round] (121.68,526.62) --
	(207.87,526.62);

\path[draw=drawColor,line width= 0.8pt,line join=round] (121.68,541.12) --
	(207.87,541.12);

\path[draw=drawColor,line width= 0.8pt,line join=round] (121.68,555.62) --
	(207.87,555.62);

\path[draw=drawColor,line width= 0.8pt,line join=round] (137.84,479.43) --
	(137.84,560.58);

\path[draw=drawColor,line width= 0.8pt,line join=round] (164.78,479.43) --
	(164.78,560.58);

\path[draw=drawColor,line width= 0.8pt,line join=round] (191.71,479.43) --
	(191.71,560.58);
\definecolor[named]{fillColor}{rgb}{0.97,0.46,0.43}

\path[fill=fillColor] (125.72,483.12) rectangle (131.78,498.35);
\definecolor[named]{fillColor}{rgb}{0.49,0.68,0.00}

\path[fill=fillColor] (131.78,483.12) rectangle (137.84,501.72);
\definecolor[named]{fillColor}{rgb}{0.00,0.75,0.77}

\path[fill=fillColor] (137.84,483.12) rectangle (143.90,498.41);
\definecolor[named]{fillColor}{rgb}{0.78,0.49,1.00}

\path[fill=fillColor] (143.90,483.12) rectangle (149.96,498.25);
\definecolor[named]{fillColor}{rgb}{0.97,0.46,0.43}

\path[fill=fillColor] (152.66,483.12) rectangle (158.72,506.67);
\definecolor[named]{fillColor}{rgb}{0.49,0.68,0.00}

\path[fill=fillColor] (158.72,483.12) rectangle (164.78,521.93);
\definecolor[named]{fillColor}{rgb}{0.00,0.75,0.77}

\path[fill=fillColor] (164.78,483.12) rectangle (170.84,508.01);
\definecolor[named]{fillColor}{rgb}{0.78,0.49,1.00}

\path[fill=fillColor] (170.84,483.12) rectangle (176.90,507.90);
\definecolor[named]{fillColor}{rgb}{0.97,0.46,0.43}

\path[fill=fillColor] (179.59,483.12) rectangle (185.65,506.20);
\definecolor[named]{fillColor}{rgb}{0.49,0.68,0.00}

\path[fill=fillColor] (185.65,483.12) rectangle (191.71,521.30);
\definecolor[named]{fillColor}{rgb}{0.00,0.75,0.77}

\path[fill=fillColor] (191.71,483.12) rectangle (197.77,507.28);
\definecolor[named]{fillColor}{rgb}{0.78,0.49,1.00}

\path[fill=fillColor] (197.77,483.12) rectangle (203.83,507.48);
\definecolor[named]{drawColor}{rgb}{1.00,0.00,0.00}
\definecolor[named]{fillColor}{rgb}{1.00,0.00,0.00}

\path[draw=drawColor,line width= 0.6pt,dash pattern=on 4pt off 4pt ,line join=round,fill=fillColor] (121.68,497.62) -- (207.87,497.62);
\definecolor[named]{drawColor}{rgb}{0.00,0.00,1.00}
\definecolor[named]{fillColor}{rgb}{0.00,0.00,1.00}

\path[draw=drawColor,line width= 0.6pt,dash pattern=on 4pt off 4pt ,line join=round,fill=fillColor] (121.68,512.12) -- (207.87,512.12);
\end{scope}
\begin{scope}
\path[clip] (121.68,395.26) rectangle (207.87,476.42);
\definecolor[named]{drawColor}{rgb}{0.75,0.75,0.75}

\path[draw=drawColor,line width= 1.5pt,line join=round,line cap=round] (121.68,395.26) rectangle (207.87,476.42);
\definecolor[named]{drawColor}{rgb}{0.90,0.90,0.90}

\path[draw=drawColor,line width= 0.3pt,line join=round] (121.68,407.91) --
	(207.87,407.91);

\path[draw=drawColor,line width= 0.3pt,line join=round] (121.68,425.84) --
	(207.87,425.84);

\path[draw=drawColor,line width= 0.3pt,line join=round] (121.68,443.76) --
	(207.87,443.76);

\path[draw=drawColor,line width= 0.3pt,line join=round] (121.68,461.69) --
	(207.87,461.69);

\path[draw=drawColor,line width= 0.8pt,line join=round] (121.68,398.95) --
	(207.87,398.95);

\path[draw=drawColor,line width= 0.8pt,line join=round] (121.68,416.88) --
	(207.87,416.88);

\path[draw=drawColor,line width= 0.8pt,line join=round] (121.68,434.80) --
	(207.87,434.80);

\path[draw=drawColor,line width= 0.8pt,line join=round] (121.68,452.73) --
	(207.87,452.73);

\path[draw=drawColor,line width= 0.8pt,line join=round] (121.68,470.65) --
	(207.87,470.65);

\path[draw=drawColor,line width= 0.8pt,line join=round] (137.84,395.26) --
	(137.84,476.42);

\path[draw=drawColor,line width= 0.8pt,line join=round] (164.78,395.26) --
	(164.78,476.42);

\path[draw=drawColor,line width= 0.8pt,line join=round] (191.71,395.26) --
	(191.71,476.42);
\definecolor[named]{fillColor}{rgb}{0.97,0.46,0.43}

\path[fill=fillColor] (125.72,398.95) rectangle (131.78,418.14);
\definecolor[named]{fillColor}{rgb}{0.49,0.68,0.00}

\path[fill=fillColor] (131.78,398.95) rectangle (137.84,419.98);
\definecolor[named]{fillColor}{rgb}{0.00,0.75,0.77}

\path[fill=fillColor] (137.84,398.95) rectangle (143.90,418.04);
\definecolor[named]{fillColor}{rgb}{0.78,0.49,1.00}

\path[fill=fillColor] (143.90,398.95) rectangle (149.96,417.76);
\definecolor[named]{fillColor}{rgb}{0.97,0.46,0.43}

\path[fill=fillColor] (152.66,398.95) rectangle (158.72,426.35);
\definecolor[named]{fillColor}{rgb}{0.49,0.68,0.00}

\path[fill=fillColor] (158.72,398.95) rectangle (164.78,456.17);
\definecolor[named]{fillColor}{rgb}{0.00,0.75,0.77}

\path[fill=fillColor] (164.78,398.95) rectangle (170.84,428.22);
\definecolor[named]{fillColor}{rgb}{0.78,0.49,1.00}

\path[fill=fillColor] (170.84,398.95) rectangle (176.90,428.16);
\definecolor[named]{fillColor}{rgb}{0.97,0.46,0.43}

\path[fill=fillColor] (179.59,398.95) rectangle (185.65,426.02);
\definecolor[named]{fillColor}{rgb}{0.49,0.68,0.00}

\path[fill=fillColor] (185.65,398.95) rectangle (191.71,455.89);
\definecolor[named]{fillColor}{rgb}{0.00,0.75,0.77}

\path[fill=fillColor] (191.71,398.95) rectangle (197.77,427.51);
\definecolor[named]{fillColor}{rgb}{0.78,0.49,1.00}

\path[fill=fillColor] (197.77,398.95) rectangle (203.83,427.65);
\definecolor[named]{drawColor}{rgb}{1.00,0.00,0.00}
\definecolor[named]{fillColor}{rgb}{1.00,0.00,0.00}

\path[draw=drawColor,line width= 0.6pt,dash pattern=on 4pt off 4pt ,line join=round,fill=fillColor] (121.68,416.88) -- (207.87,416.88);
\definecolor[named]{drawColor}{rgb}{0.00,0.00,1.00}
\definecolor[named]{fillColor}{rgb}{0.00,0.00,1.00}

\path[draw=drawColor,line width= 0.6pt,dash pattern=on 4pt off 4pt ,line join=round,fill=fillColor] (121.68,434.80) -- (207.87,434.80);
\end{scope}
\begin{scope}
\path[clip] (121.68,311.10) rectangle (207.87,392.25);
\definecolor[named]{drawColor}{rgb}{0.75,0.75,0.75}

\path[draw=drawColor,line width= 1.5pt,line join=round,line cap=round] (121.68,311.10) rectangle (207.87,392.25);
\definecolor[named]{drawColor}{rgb}{0.90,0.90,0.90}

\path[draw=drawColor,line width= 0.3pt,line join=round] (121.68,323.60) --
	(207.87,323.60);

\path[draw=drawColor,line width= 0.3pt,line join=round] (121.68,341.24) --
	(207.87,341.24);

\path[draw=drawColor,line width= 0.3pt,line join=round] (121.68,358.87) --
	(207.87,358.87);

\path[draw=drawColor,line width= 0.3pt,line join=round] (121.68,376.51) --
	(207.87,376.51);

\path[draw=drawColor,line width= 0.8pt,line join=round] (121.68,314.79) --
	(207.87,314.79);

\path[draw=drawColor,line width= 0.8pt,line join=round] (121.68,332.42) --
	(207.87,332.42);

\path[draw=drawColor,line width= 0.8pt,line join=round] (121.68,350.06) --
	(207.87,350.06);

\path[draw=drawColor,line width= 0.8pt,line join=round] (121.68,367.69) --
	(207.87,367.69);

\path[draw=drawColor,line width= 0.8pt,line join=round] (121.68,385.33) --
	(207.87,385.33);

\path[draw=drawColor,line width= 0.8pt,line join=round] (137.84,311.10) --
	(137.84,392.25);

\path[draw=drawColor,line width= 0.8pt,line join=round] (164.78,311.10) --
	(164.78,392.25);

\path[draw=drawColor,line width= 0.8pt,line join=round] (191.71,311.10) --
	(191.71,392.25);
\definecolor[named]{fillColor}{rgb}{0.97,0.46,0.43}

\path[fill=fillColor] (125.72,314.79) rectangle (131.78,335.27);
\definecolor[named]{fillColor}{rgb}{0.49,0.68,0.00}

\path[fill=fillColor] (131.78,314.79) rectangle (137.84,346.70);
\definecolor[named]{fillColor}{rgb}{0.00,0.75,0.77}

\path[fill=fillColor] (137.84,314.79) rectangle (143.90,335.86);
\definecolor[named]{fillColor}{rgb}{0.78,0.49,1.00}

\path[fill=fillColor] (143.90,314.79) rectangle (149.96,335.42);
\definecolor[named]{fillColor}{rgb}{0.97,0.46,0.43}

\path[fill=fillColor] (152.66,314.79) rectangle (158.72,338.41);
\definecolor[named]{fillColor}{rgb}{0.49,0.68,0.00}

\path[fill=fillColor] (158.72,314.79) rectangle (164.78,349.88);
\definecolor[named]{fillColor}{rgb}{0.00,0.75,0.77}

\path[fill=fillColor] (164.78,314.79) rectangle (170.84,339.59);
\definecolor[named]{fillColor}{rgb}{0.78,0.49,1.00}

\path[fill=fillColor] (170.84,314.79) rectangle (176.90,339.21);
\definecolor[named]{fillColor}{rgb}{0.97,0.46,0.43}

\path[fill=fillColor] (179.59,314.79) rectangle (185.65,338.80);
\definecolor[named]{fillColor}{rgb}{0.49,0.68,0.00}

\path[fill=fillColor] (185.65,314.79) rectangle (191.71,352.58);
\definecolor[named]{fillColor}{rgb}{0.00,0.75,0.77}

\path[fill=fillColor] (191.71,314.79) rectangle (197.77,340.22);
\definecolor[named]{fillColor}{rgb}{0.78,0.49,1.00}

\path[fill=fillColor] (197.77,314.79) rectangle (203.83,339.98);
\definecolor[named]{drawColor}{rgb}{1.00,0.00,0.00}
\definecolor[named]{fillColor}{rgb}{1.00,0.00,0.00}

\path[draw=drawColor,line width= 0.6pt,dash pattern=on 4pt off 4pt ,line join=round,fill=fillColor] (121.68,332.42) -- (207.87,332.42);
\definecolor[named]{drawColor}{rgb}{0.00,0.00,1.00}
\definecolor[named]{fillColor}{rgb}{0.00,0.00,1.00}

\path[draw=drawColor,line width= 0.6pt,dash pattern=on 4pt off 4pt ,line join=round,fill=fillColor] (121.68,350.06) -- (207.87,350.06);
\end{scope}
\begin{scope}
\path[clip] (121.68,226.93) rectangle (207.87,308.09);
\definecolor[named]{drawColor}{rgb}{0.75,0.75,0.75}

\path[draw=drawColor,line width= 1.5pt,line join=round,line cap=round] (121.68,226.93) rectangle (207.87,308.09);
\definecolor[named]{drawColor}{rgb}{0.90,0.90,0.90}

\path[draw=drawColor,line width= 0.3pt,line join=round] (121.68,240.40) --
	(207.87,240.40);

\path[draw=drawColor,line width= 0.3pt,line join=round] (121.68,259.96) --
	(207.87,259.96);

\path[draw=drawColor,line width= 0.3pt,line join=round] (121.68,279.52) --
	(207.87,279.52);

\path[draw=drawColor,line width= 0.3pt,line join=round] (121.68,299.08) --
	(207.87,299.08);

\path[draw=drawColor,line width= 0.8pt,line join=round] (121.68,230.62) --
	(207.87,230.62);

\path[draw=drawColor,line width= 0.8pt,line join=round] (121.68,250.18) --
	(207.87,250.18);

\path[draw=drawColor,line width= 0.8pt,line join=round] (121.68,269.74) --
	(207.87,269.74);

\path[draw=drawColor,line width= 0.8pt,line join=round] (121.68,289.30) --
	(207.87,289.30);

\path[draw=drawColor,line width= 0.8pt,line join=round] (137.84,226.93) --
	(137.84,308.09);

\path[draw=drawColor,line width= 0.8pt,line join=round] (164.78,226.93) --
	(164.78,308.09);

\path[draw=drawColor,line width= 0.8pt,line join=round] (191.71,226.93) --
	(191.71,308.09);
\definecolor[named]{fillColor}{rgb}{0.97,0.46,0.43}

\path[fill=fillColor] (125.72,230.62) rectangle (131.78,250.76);
\definecolor[named]{fillColor}{rgb}{0.49,0.68,0.00}

\path[fill=fillColor] (131.78,230.62) rectangle (137.84,250.47);
\definecolor[named]{fillColor}{rgb}{0.00,0.75,0.77}

\path[fill=fillColor] (137.84,230.62) rectangle (143.90,250.86);
\definecolor[named]{fillColor}{rgb}{0.78,0.49,1.00}

\path[fill=fillColor] (143.90,230.62) rectangle (149.96,250.19);
\definecolor[named]{fillColor}{rgb}{0.97,0.46,0.43}

\path[fill=fillColor] (152.66,230.62) rectangle (158.72,284.68);
\definecolor[named]{fillColor}{rgb}{0.49,0.68,0.00}

\path[fill=fillColor] (158.72,230.62) rectangle (164.78,287.11);
\definecolor[named]{fillColor}{rgb}{0.00,0.75,0.77}

\path[fill=fillColor] (164.78,230.62) rectangle (170.84,269.54);
\definecolor[named]{fillColor}{rgb}{0.78,0.49,1.00}

\path[fill=fillColor] (170.84,230.62) rectangle (176.90,270.35);
\definecolor[named]{fillColor}{rgb}{0.97,0.46,0.43}

\path[fill=fillColor] (179.59,230.62) rectangle (185.65,285.26);
\definecolor[named]{fillColor}{rgb}{0.49,0.68,0.00}

\path[fill=fillColor] (185.65,230.62) rectangle (191.71,287.12);
\definecolor[named]{fillColor}{rgb}{0.00,0.75,0.77}

\path[fill=fillColor] (191.71,230.62) rectangle (197.77,270.42);
\definecolor[named]{fillColor}{rgb}{0.78,0.49,1.00}

\path[fill=fillColor] (197.77,230.62) rectangle (203.83,270.77);
\definecolor[named]{drawColor}{rgb}{1.00,0.00,0.00}
\definecolor[named]{fillColor}{rgb}{1.00,0.00,0.00}

\path[draw=drawColor,line width= 0.6pt,dash pattern=on 4pt off 4pt ,line join=round,fill=fillColor] (121.68,250.18) -- (207.87,250.18);
\definecolor[named]{drawColor}{rgb}{0.00,0.00,1.00}
\definecolor[named]{fillColor}{rgb}{0.00,0.00,1.00}

\path[draw=drawColor,line width= 0.6pt,dash pattern=on 4pt off 4pt ,line join=round,fill=fillColor] (121.68,269.74) -- (207.87,269.74);
\end{scope}
\begin{scope}
\path[clip] (121.68,142.77) rectangle (207.87,223.92);
\definecolor[named]{drawColor}{rgb}{0.75,0.75,0.75}

\path[draw=drawColor,line width= 1.5pt,line join=round,line cap=round] (121.68,142.77) rectangle (207.87,223.92);
\definecolor[named]{drawColor}{rgb}{0.90,0.90,0.90}

\path[draw=drawColor,line width= 0.3pt,line join=round] (121.68,154.09) --
	(207.87,154.09);

\path[draw=drawColor,line width= 0.3pt,line join=round] (121.68,169.35) --
	(207.87,169.35);

\path[draw=drawColor,line width= 0.3pt,line join=round] (121.68,184.61) --
	(207.87,184.61);

\path[draw=drawColor,line width= 0.3pt,line join=round] (121.68,199.88) --
	(207.87,199.88);

\path[draw=drawColor,line width= 0.3pt,line join=round] (121.68,215.14) --
	(207.87,215.14);

\path[draw=drawColor,line width= 0.8pt,line join=round] (121.68,146.45) --
	(207.87,146.45);

\path[draw=drawColor,line width= 0.8pt,line join=round] (121.68,161.72) --
	(207.87,161.72);

\path[draw=drawColor,line width= 0.8pt,line join=round] (121.68,176.98) --
	(207.87,176.98);

\path[draw=drawColor,line width= 0.8pt,line join=round] (121.68,192.24) --
	(207.87,192.24);

\path[draw=drawColor,line width= 0.8pt,line join=round] (121.68,207.51) --
	(207.87,207.51);

\path[draw=drawColor,line width= 0.8pt,line join=round] (121.68,222.77) --
	(207.87,222.77);

\path[draw=drawColor,line width= 0.8pt,line join=round] (137.84,142.77) --
	(137.84,223.92);

\path[draw=drawColor,line width= 0.8pt,line join=round] (164.78,142.77) --
	(164.78,223.92);

\path[draw=drawColor,line width= 0.8pt,line join=round] (191.71,142.77) --
	(191.71,223.92);
\definecolor[named]{fillColor}{rgb}{0.97,0.46,0.43}

\path[fill=fillColor] (125.72,146.45) rectangle (131.78,162.69);
\definecolor[named]{fillColor}{rgb}{0.49,0.68,0.00}

\path[fill=fillColor] (131.78,146.45) rectangle (137.84,163.81);
\definecolor[named]{fillColor}{rgb}{0.00,0.75,0.77}

\path[fill=fillColor] (137.84,146.45) rectangle (143.90,162.39);
\definecolor[named]{fillColor}{rgb}{0.78,0.49,1.00}

\path[fill=fillColor] (143.90,146.45) rectangle (149.96,162.33);
\definecolor[named]{fillColor}{rgb}{0.97,0.46,0.43}

\path[fill=fillColor] (152.66,146.45) rectangle (158.72,179.38);
\definecolor[named]{fillColor}{rgb}{0.49,0.68,0.00}

\path[fill=fillColor] (158.72,146.45) rectangle (164.78,205.03);
\definecolor[named]{fillColor}{rgb}{0.00,0.75,0.77}

\path[fill=fillColor] (164.78,146.45) rectangle (170.84,175.71);
\definecolor[named]{fillColor}{rgb}{0.78,0.49,1.00}

\path[fill=fillColor] (170.84,146.45) rectangle (176.90,176.51);
\definecolor[named]{fillColor}{rgb}{0.97,0.46,0.43}

\path[fill=fillColor] (179.59,146.45) rectangle (185.65,180.01);
\definecolor[named]{fillColor}{rgb}{0.49,0.68,0.00}

\path[fill=fillColor] (185.65,146.45) rectangle (191.71,203.81);
\definecolor[named]{fillColor}{rgb}{0.00,0.75,0.77}

\path[fill=fillColor] (191.71,146.45) rectangle (197.77,176.41);
\definecolor[named]{fillColor}{rgb}{0.78,0.49,1.00}

\path[fill=fillColor] (197.77,146.45) rectangle (203.83,177.06);
\definecolor[named]{drawColor}{rgb}{1.00,0.00,0.00}
\definecolor[named]{fillColor}{rgb}{1.00,0.00,0.00}

\path[draw=drawColor,line width= 0.6pt,dash pattern=on 4pt off 4pt ,line join=round,fill=fillColor] (121.68,161.72) -- (207.87,161.72);
\definecolor[named]{drawColor}{rgb}{0.00,0.00,1.00}
\definecolor[named]{fillColor}{rgb}{0.00,0.00,1.00}

\path[draw=drawColor,line width= 0.6pt,dash pattern=on 4pt off 4pt ,line join=round,fill=fillColor] (121.68,176.98) -- (207.87,176.98);
\end{scope}
\begin{scope}
\path[clip] (121.68, 58.60) rectangle (207.87,139.75);
\definecolor[named]{drawColor}{rgb}{0.75,0.75,0.75}

\path[draw=drawColor,line width= 1.5pt,line join=round,line cap=round] (121.68, 58.60) rectangle (207.87,139.75);
\definecolor[named]{drawColor}{rgb}{0.90,0.90,0.90}

\path[draw=drawColor,line width= 0.3pt,line join=round] (121.68, 71.51) --
	(207.87, 71.51);

\path[draw=drawColor,line width= 0.3pt,line join=round] (121.68, 89.96) --
	(207.87, 89.96);

\path[draw=drawColor,line width= 0.3pt,line join=round] (121.68,108.40) --
	(207.87,108.40);

\path[draw=drawColor,line width= 0.3pt,line join=round] (121.68,126.84) --
	(207.87,126.84);

\path[draw=drawColor,line width= 0.8pt,line join=round] (121.68, 62.29) --
	(207.87, 62.29);

\path[draw=drawColor,line width= 0.8pt,line join=round] (121.68, 80.73) --
	(207.87, 80.73);

\path[draw=drawColor,line width= 0.8pt,line join=round] (121.68, 99.18) --
	(207.87, 99.18);

\path[draw=drawColor,line width= 0.8pt,line join=round] (121.68,117.62) --
	(207.87,117.62);

\path[draw=drawColor,line width= 0.8pt,line join=round] (121.68,136.07) --
	(207.87,136.07);

\path[draw=drawColor,line width= 0.8pt,line join=round] (137.84, 58.60) --
	(137.84,139.75);

\path[draw=drawColor,line width= 0.8pt,line join=round] (164.78, 58.60) --
	(164.78,139.75);

\path[draw=drawColor,line width= 0.8pt,line join=round] (191.71, 58.60) --
	(191.71,139.75);
\definecolor[named]{fillColor}{rgb}{0.97,0.46,0.43}

\path[fill=fillColor] (125.72, 62.29) rectangle (131.78,102.06);
\definecolor[named]{fillColor}{rgb}{0.49,0.68,0.00}

\path[fill=fillColor] (131.78, 62.29) rectangle (137.84, 99.37);
\definecolor[named]{fillColor}{rgb}{0.00,0.75,0.77}

\path[fill=fillColor] (137.84, 62.29) rectangle (143.90,105.62);
\definecolor[named]{fillColor}{rgb}{0.78,0.49,1.00}

\path[fill=fillColor] (143.90, 62.29) rectangle (149.96,100.44);
\definecolor[named]{fillColor}{rgb}{0.97,0.46,0.43}

\path[fill=fillColor] (152.66, 62.29) rectangle (158.72,106.66);
\definecolor[named]{fillColor}{rgb}{0.49,0.68,0.00}

\path[fill=fillColor] (158.72, 62.29) rectangle (164.78,115.11);
\definecolor[named]{fillColor}{rgb}{0.00,0.75,0.77}

\path[fill=fillColor] (164.78, 62.29) rectangle (170.84,105.79);
\definecolor[named]{fillColor}{rgb}{0.78,0.49,1.00}

\path[fill=fillColor] (170.84, 62.29) rectangle (176.90,105.69);
\definecolor[named]{fillColor}{rgb}{0.97,0.46,0.43}

\path[fill=fillColor] (179.59, 62.29) rectangle (185.65,107.16);
\definecolor[named]{fillColor}{rgb}{0.49,0.68,0.00}

\path[fill=fillColor] (185.65, 62.29) rectangle (191.71,114.94);
\definecolor[named]{fillColor}{rgb}{0.00,0.75,0.77}

\path[fill=fillColor] (191.71, 62.29) rectangle (197.77,105.85);
\definecolor[named]{fillColor}{rgb}{0.78,0.49,1.00}

\path[fill=fillColor] (197.77, 62.29) rectangle (203.83,106.92);
\definecolor[named]{drawColor}{rgb}{1.00,0.00,0.00}
\definecolor[named]{fillColor}{rgb}{1.00,0.00,0.00}

\path[draw=drawColor,line width= 0.6pt,dash pattern=on 4pt off 4pt ,line join=round,fill=fillColor] (121.68, 99.18) -- (207.87, 99.18);
\definecolor[named]{drawColor}{rgb}{0.00,0.00,1.00}
\definecolor[named]{fillColor}{rgb}{0.00,0.00,1.00}

\path[draw=drawColor,line width= 0.6pt,dash pattern=on 4pt off 4pt ,line join=round,fill=fillColor] (121.68,136.07) -- (207.87,136.07);
\end{scope}
\begin{scope}
\path[clip] (210.88,479.43) rectangle (297.07,560.58);
\definecolor[named]{drawColor}{rgb}{0.75,0.75,0.75}

\path[draw=drawColor,line width= 1.5pt,line join=round,line cap=round] (210.88,479.43) rectangle (297.07,560.58);
\definecolor[named]{drawColor}{rgb}{0.90,0.90,0.90}

\path[draw=drawColor,line width= 0.3pt,line join=round] (210.88,490.37) --
	(297.07,490.37);

\path[draw=drawColor,line width= 0.3pt,line join=round] (210.88,504.87) --
	(297.07,504.87);

\path[draw=drawColor,line width= 0.3pt,line join=round] (210.88,519.37) --
	(297.07,519.37);

\path[draw=drawColor,line width= 0.3pt,line join=round] (210.88,533.87) --
	(297.07,533.87);

\path[draw=drawColor,line width= 0.3pt,line join=round] (210.88,548.37) --
	(297.07,548.37);

\path[draw=drawColor,line width= 0.8pt,line join=round] (210.88,483.12) --
	(297.07,483.12);

\path[draw=drawColor,line width= 0.8pt,line join=round] (210.88,497.62) --
	(297.07,497.62);

\path[draw=drawColor,line width= 0.8pt,line join=round] (210.88,512.12) --
	(297.07,512.12);

\path[draw=drawColor,line width= 0.8pt,line join=round] (210.88,526.62) --
	(297.07,526.62);

\path[draw=drawColor,line width= 0.8pt,line join=round] (210.88,541.12) --
	(297.07,541.12);

\path[draw=drawColor,line width= 0.8pt,line join=round] (210.88,555.62) --
	(297.07,555.62);

\path[draw=drawColor,line width= 0.8pt,line join=round] (227.04,479.43) --
	(227.04,560.58);

\path[draw=drawColor,line width= 0.8pt,line join=round] (253.98,479.43) --
	(253.98,560.58);

\path[draw=drawColor,line width= 0.8pt,line join=round] (280.91,479.43) --
	(280.91,560.58);
\definecolor[named]{fillColor}{rgb}{0.97,0.46,0.43}

\path[fill=fillColor] (214.92,483.12) rectangle (220.98,499.54);
\definecolor[named]{fillColor}{rgb}{0.49,0.68,0.00}

\path[fill=fillColor] (220.98,483.12) rectangle (227.04,502.86);
\definecolor[named]{fillColor}{rgb}{0.00,0.75,0.77}

\path[fill=fillColor] (227.04,483.12) rectangle (233.10,498.16);
\definecolor[named]{fillColor}{rgb}{0.78,0.49,1.00}

\path[fill=fillColor] (233.10,483.12) rectangle (239.16,498.47);
\definecolor[named]{fillColor}{rgb}{0.97,0.46,0.43}

\path[fill=fillColor] (241.86,483.12) rectangle (247.92,507.03);
\definecolor[named]{fillColor}{rgb}{0.49,0.68,0.00}

\path[fill=fillColor] (247.92,483.12) rectangle (253.98,528.43);
\definecolor[named]{fillColor}{rgb}{0.00,0.75,0.77}

\path[fill=fillColor] (253.98,483.12) rectangle (260.04,505.70);
\definecolor[named]{fillColor}{rgb}{0.78,0.49,1.00}

\path[fill=fillColor] (260.04,483.12) rectangle (266.10,505.06);
\definecolor[named]{fillColor}{rgb}{0.97,0.46,0.43}

\path[fill=fillColor] (268.79,483.12) rectangle (274.85,507.92);
\definecolor[named]{fillColor}{rgb}{0.49,0.68,0.00}

\path[fill=fillColor] (274.85,483.12) rectangle (280.91,530.79);
\definecolor[named]{fillColor}{rgb}{0.00,0.75,0.77}

\path[fill=fillColor] (280.91,483.12) rectangle (286.97,506.49);
\definecolor[named]{fillColor}{rgb}{0.78,0.49,1.00}

\path[fill=fillColor] (286.97,483.12) rectangle (293.03,505.70);
\definecolor[named]{drawColor}{rgb}{1.00,0.00,0.00}
\definecolor[named]{fillColor}{rgb}{1.00,0.00,0.00}

\path[draw=drawColor,line width= 0.6pt,dash pattern=on 4pt off 4pt ,line join=round,fill=fillColor] (210.88,497.62) -- (297.07,497.62);
\definecolor[named]{drawColor}{rgb}{0.00,0.00,1.00}
\definecolor[named]{fillColor}{rgb}{0.00,0.00,1.00}

\path[draw=drawColor,line width= 0.6pt,dash pattern=on 4pt off 4pt ,line join=round,fill=fillColor] (210.88,512.12) -- (297.07,512.12);
\end{scope}
\begin{scope}
\path[clip] (210.88,395.26) rectangle (297.07,476.42);
\definecolor[named]{drawColor}{rgb}{0.75,0.75,0.75}

\path[draw=drawColor,line width= 1.5pt,line join=round,line cap=round] (210.88,395.26) rectangle (297.07,476.42);
\definecolor[named]{drawColor}{rgb}{0.90,0.90,0.90}

\path[draw=drawColor,line width= 0.3pt,line join=round] (210.88,407.91) --
	(297.07,407.91);

\path[draw=drawColor,line width= 0.3pt,line join=round] (210.88,425.84) --
	(297.07,425.84);

\path[draw=drawColor,line width= 0.3pt,line join=round] (210.88,443.76) --
	(297.07,443.76);

\path[draw=drawColor,line width= 0.3pt,line join=round] (210.88,461.69) --
	(297.07,461.69);

\path[draw=drawColor,line width= 0.8pt,line join=round] (210.88,398.95) --
	(297.07,398.95);

\path[draw=drawColor,line width= 0.8pt,line join=round] (210.88,416.88) --
	(297.07,416.88);

\path[draw=drawColor,line width= 0.8pt,line join=round] (210.88,434.80) --
	(297.07,434.80);

\path[draw=drawColor,line width= 0.8pt,line join=round] (210.88,452.73) --
	(297.07,452.73);

\path[draw=drawColor,line width= 0.8pt,line join=round] (210.88,470.65) --
	(297.07,470.65);

\path[draw=drawColor,line width= 0.8pt,line join=round] (227.04,395.26) --
	(227.04,476.42);

\path[draw=drawColor,line width= 0.8pt,line join=round] (253.98,395.26) --
	(253.98,476.42);

\path[draw=drawColor,line width= 0.8pt,line join=round] (280.91,395.26) --
	(280.91,476.42);
\definecolor[named]{fillColor}{rgb}{0.97,0.46,0.43}

\path[fill=fillColor] (214.92,398.95) rectangle (220.98,418.62);
\definecolor[named]{fillColor}{rgb}{0.49,0.68,0.00}

\path[fill=fillColor] (220.98,398.95) rectangle (227.04,421.55);
\definecolor[named]{fillColor}{rgb}{0.00,0.75,0.77}

\path[fill=fillColor] (227.04,398.95) rectangle (233.10,417.54);
\definecolor[named]{fillColor}{rgb}{0.78,0.49,1.00}

\path[fill=fillColor] (233.10,398.95) rectangle (239.16,417.83);
\definecolor[named]{fillColor}{rgb}{0.97,0.46,0.43}

\path[fill=fillColor] (241.86,398.95) rectangle (247.92,427.26);
\definecolor[named]{fillColor}{rgb}{0.49,0.68,0.00}

\path[fill=fillColor] (247.92,398.95) rectangle (253.98,450.42);
\definecolor[named]{fillColor}{rgb}{0.00,0.75,0.77}

\path[fill=fillColor] (253.98,398.95) rectangle (260.04,424.62);
\definecolor[named]{fillColor}{rgb}{0.78,0.49,1.00}

\path[fill=fillColor] (260.04,398.95) rectangle (266.10,423.93);
\definecolor[named]{fillColor}{rgb}{0.97,0.46,0.43}

\path[fill=fillColor] (268.79,398.95) rectangle (274.85,428.19);
\definecolor[named]{fillColor}{rgb}{0.49,0.68,0.00}

\path[fill=fillColor] (274.85,398.95) rectangle (280.91,454.75);
\definecolor[named]{fillColor}{rgb}{0.00,0.75,0.77}

\path[fill=fillColor] (280.91,398.95) rectangle (286.97,424.96);
\definecolor[named]{fillColor}{rgb}{0.78,0.49,1.00}

\path[fill=fillColor] (286.97,398.95) rectangle (293.03,424.44);
\definecolor[named]{drawColor}{rgb}{1.00,0.00,0.00}
\definecolor[named]{fillColor}{rgb}{1.00,0.00,0.00}

\path[draw=drawColor,line width= 0.6pt,dash pattern=on 4pt off 4pt ,line join=round,fill=fillColor] (210.88,416.88) -- (297.07,416.88);
\definecolor[named]{drawColor}{rgb}{0.00,0.00,1.00}
\definecolor[named]{fillColor}{rgb}{0.00,0.00,1.00}

\path[draw=drawColor,line width= 0.6pt,dash pattern=on 4pt off 4pt ,line join=round,fill=fillColor] (210.88,434.80) -- (297.07,434.80);
\end{scope}
\begin{scope}
\path[clip] (210.88,311.10) rectangle (297.07,392.25);
\definecolor[named]{drawColor}{rgb}{0.75,0.75,0.75}

\path[draw=drawColor,line width= 1.5pt,line join=round,line cap=round] (210.88,311.10) rectangle (297.07,392.25);
\definecolor[named]{drawColor}{rgb}{0.90,0.90,0.90}

\path[draw=drawColor,line width= 0.3pt,line join=round] (210.88,323.60) --
	(297.07,323.60);

\path[draw=drawColor,line width= 0.3pt,line join=round] (210.88,341.24) --
	(297.07,341.24);

\path[draw=drawColor,line width= 0.3pt,line join=round] (210.88,358.87) --
	(297.07,358.87);

\path[draw=drawColor,line width= 0.3pt,line join=round] (210.88,376.51) --
	(297.07,376.51);

\path[draw=drawColor,line width= 0.8pt,line join=round] (210.88,314.79) --
	(297.07,314.79);

\path[draw=drawColor,line width= 0.8pt,line join=round] (210.88,332.42) --
	(297.07,332.42);

\path[draw=drawColor,line width= 0.8pt,line join=round] (210.88,350.06) --
	(297.07,350.06);

\path[draw=drawColor,line width= 0.8pt,line join=round] (210.88,367.69) --
	(297.07,367.69);

\path[draw=drawColor,line width= 0.8pt,line join=round] (210.88,385.33) --
	(297.07,385.33);

\path[draw=drawColor,line width= 0.8pt,line join=round] (227.04,311.10) --
	(227.04,392.25);

\path[draw=drawColor,line width= 0.8pt,line join=round] (253.98,311.10) --
	(253.98,392.25);

\path[draw=drawColor,line width= 0.8pt,line join=round] (280.91,311.10) --
	(280.91,392.25);
\definecolor[named]{fillColor}{rgb}{0.97,0.46,0.43}

\path[fill=fillColor] (214.92,314.79) rectangle (220.98,337.85);
\definecolor[named]{fillColor}{rgb}{0.49,0.68,0.00}

\path[fill=fillColor] (220.98,314.79) rectangle (227.04,351.32);
\definecolor[named]{fillColor}{rgb}{0.00,0.75,0.77}

\path[fill=fillColor] (227.04,314.79) rectangle (233.10,338.27);
\definecolor[named]{fillColor}{rgb}{0.78,0.49,1.00}

\path[fill=fillColor] (233.10,314.79) rectangle (239.16,338.46);
\definecolor[named]{fillColor}{rgb}{0.97,0.46,0.43}

\path[fill=fillColor] (241.86,314.79) rectangle (247.92,336.56);
\definecolor[named]{fillColor}{rgb}{0.49,0.68,0.00}

\path[fill=fillColor] (247.92,314.79) rectangle (253.98,343.15);
\definecolor[named]{fillColor}{rgb}{0.00,0.75,0.77}

\path[fill=fillColor] (253.98,314.79) rectangle (260.04,336.53);
\definecolor[named]{fillColor}{rgb}{0.78,0.49,1.00}

\path[fill=fillColor] (260.04,314.79) rectangle (266.10,336.27);
\definecolor[named]{fillColor}{rgb}{0.97,0.46,0.43}

\path[fill=fillColor] (268.79,314.79) rectangle (274.85,339.28);
\definecolor[named]{fillColor}{rgb}{0.49,0.68,0.00}

\path[fill=fillColor] (274.85,314.79) rectangle (280.91,355.82);
\definecolor[named]{fillColor}{rgb}{0.00,0.75,0.77}

\path[fill=fillColor] (280.91,314.79) rectangle (286.97,340.10);
\definecolor[named]{fillColor}{rgb}{0.78,0.49,1.00}

\path[fill=fillColor] (286.97,314.79) rectangle (293.03,340.18);
\definecolor[named]{drawColor}{rgb}{1.00,0.00,0.00}
\definecolor[named]{fillColor}{rgb}{1.00,0.00,0.00}

\path[draw=drawColor,line width= 0.6pt,dash pattern=on 4pt off 4pt ,line join=round,fill=fillColor] (210.88,332.42) -- (297.07,332.42);
\definecolor[named]{drawColor}{rgb}{0.00,0.00,1.00}
\definecolor[named]{fillColor}{rgb}{0.00,0.00,1.00}

\path[draw=drawColor,line width= 0.6pt,dash pattern=on 4pt off 4pt ,line join=round,fill=fillColor] (210.88,350.06) -- (297.07,350.06);
\end{scope}
\begin{scope}
\path[clip] (210.88,226.93) rectangle (297.07,308.09);
\definecolor[named]{drawColor}{rgb}{0.75,0.75,0.75}

\path[draw=drawColor,line width= 1.5pt,line join=round,line cap=round] (210.88,226.93) rectangle (297.07,308.09);
\definecolor[named]{drawColor}{rgb}{0.90,0.90,0.90}

\path[draw=drawColor,line width= 0.3pt,line join=round] (210.88,240.40) --
	(297.07,240.40);

\path[draw=drawColor,line width= 0.3pt,line join=round] (210.88,259.96) --
	(297.07,259.96);

\path[draw=drawColor,line width= 0.3pt,line join=round] (210.88,279.52) --
	(297.07,279.52);

\path[draw=drawColor,line width= 0.3pt,line join=round] (210.88,299.08) --
	(297.07,299.08);

\path[draw=drawColor,line width= 0.8pt,line join=round] (210.88,230.62) --
	(297.07,230.62);

\path[draw=drawColor,line width= 0.8pt,line join=round] (210.88,250.18) --
	(297.07,250.18);

\path[draw=drawColor,line width= 0.8pt,line join=round] (210.88,269.74) --
	(297.07,269.74);

\path[draw=drawColor,line width= 0.8pt,line join=round] (210.88,289.30) --
	(297.07,289.30);

\path[draw=drawColor,line width= 0.8pt,line join=round] (227.04,226.93) --
	(227.04,308.09);

\path[draw=drawColor,line width= 0.8pt,line join=round] (253.98,226.93) --
	(253.98,308.09);

\path[draw=drawColor,line width= 0.8pt,line join=round] (280.91,226.93) --
	(280.91,308.09);
\definecolor[named]{fillColor}{rgb}{0.97,0.46,0.43}

\path[fill=fillColor] (214.92,230.62) rectangle (220.98,251.26);
\definecolor[named]{fillColor}{rgb}{0.49,0.68,0.00}

\path[fill=fillColor] (220.98,230.62) rectangle (227.04,250.71);
\definecolor[named]{fillColor}{rgb}{0.00,0.75,0.77}

\path[fill=fillColor] (227.04,230.62) rectangle (233.10,251.78);
\definecolor[named]{fillColor}{rgb}{0.78,0.49,1.00}

\path[fill=fillColor] (233.10,230.62) rectangle (239.16,249.38);
\definecolor[named]{fillColor}{rgb}{0.97,0.46,0.43}

\path[fill=fillColor] (241.86,230.62) rectangle (247.92,284.53);
\definecolor[named]{fillColor}{rgb}{0.49,0.68,0.00}

\path[fill=fillColor] (247.92,230.62) rectangle (253.98,303.03);
\definecolor[named]{fillColor}{rgb}{0.00,0.75,0.77}

\path[fill=fillColor] (253.98,230.62) rectangle (260.04,270.52);
\definecolor[named]{fillColor}{rgb}{0.78,0.49,1.00}

\path[fill=fillColor] (260.04,230.62) rectangle (266.10,268.34);
\definecolor[named]{fillColor}{rgb}{0.97,0.46,0.43}

\path[fill=fillColor] (268.79,230.62) rectangle (274.85,285.44);
\definecolor[named]{fillColor}{rgb}{0.49,0.68,0.00}

\path[fill=fillColor] (274.85,230.62) rectangle (280.91,304.40);
\definecolor[named]{fillColor}{rgb}{0.00,0.75,0.77}

\path[fill=fillColor] (280.91,230.62) rectangle (286.97,271.05);
\definecolor[named]{fillColor}{rgb}{0.78,0.49,1.00}

\path[fill=fillColor] (286.97,230.62) rectangle (293.03,268.86);
\definecolor[named]{drawColor}{rgb}{1.00,0.00,0.00}
\definecolor[named]{fillColor}{rgb}{1.00,0.00,0.00}

\path[draw=drawColor,line width= 0.6pt,dash pattern=on 4pt off 4pt ,line join=round,fill=fillColor] (210.88,250.18) -- (297.07,250.18);
\definecolor[named]{drawColor}{rgb}{0.00,0.00,1.00}
\definecolor[named]{fillColor}{rgb}{0.00,0.00,1.00}

\path[draw=drawColor,line width= 0.6pt,dash pattern=on 4pt off 4pt ,line join=round,fill=fillColor] (210.88,269.74) -- (297.07,269.74);
\end{scope}
\begin{scope}
\path[clip] (210.88,142.77) rectangle (297.07,223.92);
\definecolor[named]{drawColor}{rgb}{0.75,0.75,0.75}

\path[draw=drawColor,line width= 1.5pt,line join=round,line cap=round] (210.88,142.77) rectangle (297.07,223.92);
\definecolor[named]{drawColor}{rgb}{0.90,0.90,0.90}

\path[draw=drawColor,line width= 0.3pt,line join=round] (210.88,154.09) --
	(297.07,154.09);

\path[draw=drawColor,line width= 0.3pt,line join=round] (210.88,169.35) --
	(297.07,169.35);

\path[draw=drawColor,line width= 0.3pt,line join=round] (210.88,184.61) --
	(297.07,184.61);

\path[draw=drawColor,line width= 0.3pt,line join=round] (210.88,199.88) --
	(297.07,199.88);

\path[draw=drawColor,line width= 0.3pt,line join=round] (210.88,215.14) --
	(297.07,215.14);

\path[draw=drawColor,line width= 0.8pt,line join=round] (210.88,146.45) --
	(297.07,146.45);

\path[draw=drawColor,line width= 0.8pt,line join=round] (210.88,161.72) --
	(297.07,161.72);

\path[draw=drawColor,line width= 0.8pt,line join=round] (210.88,176.98) --
	(297.07,176.98);

\path[draw=drawColor,line width= 0.8pt,line join=round] (210.88,192.24) --
	(297.07,192.24);

\path[draw=drawColor,line width= 0.8pt,line join=round] (210.88,207.51) --
	(297.07,207.51);

\path[draw=drawColor,line width= 0.8pt,line join=round] (210.88,222.77) --
	(297.07,222.77);

\path[draw=drawColor,line width= 0.8pt,line join=round] (227.04,142.77) --
	(227.04,223.92);

\path[draw=drawColor,line width= 0.8pt,line join=round] (253.98,142.77) --
	(253.98,223.92);

\path[draw=drawColor,line width= 0.8pt,line join=round] (280.91,142.77) --
	(280.91,223.92);
\definecolor[named]{fillColor}{rgb}{0.97,0.46,0.43}

\path[fill=fillColor] (214.92,146.45) rectangle (220.98,164.88);
\definecolor[named]{fillColor}{rgb}{0.49,0.68,0.00}

\path[fill=fillColor] (220.98,146.45) rectangle (227.04,164.36);
\definecolor[named]{fillColor}{rgb}{0.00,0.75,0.77}

\path[fill=fillColor] (227.04,146.45) rectangle (233.10,162.54);
\definecolor[named]{fillColor}{rgb}{0.78,0.49,1.00}

\path[fill=fillColor] (233.10,146.45) rectangle (239.16,161.79);
\definecolor[named]{fillColor}{rgb}{0.97,0.46,0.43}

\path[fill=fillColor] (241.86,146.45) rectangle (247.92,182.92);
\definecolor[named]{fillColor}{rgb}{0.49,0.68,0.00}

\path[fill=fillColor] (247.92,146.45) rectangle (253.98,208.10);
\definecolor[named]{fillColor}{rgb}{0.00,0.75,0.77}

\path[fill=fillColor] (253.98,146.45) rectangle (260.04,174.78);
\definecolor[named]{fillColor}{rgb}{0.78,0.49,1.00}

\path[fill=fillColor] (260.04,146.45) rectangle (266.10,174.53);
\definecolor[named]{fillColor}{rgb}{0.97,0.46,0.43}

\path[fill=fillColor] (268.79,146.45) rectangle (274.85,185.20);
\definecolor[named]{fillColor}{rgb}{0.49,0.68,0.00}

\path[fill=fillColor] (274.85,146.45) rectangle (280.91,208.59);
\definecolor[named]{fillColor}{rgb}{0.00,0.75,0.77}

\path[fill=fillColor] (280.91,146.45) rectangle (286.97,175.66);
\definecolor[named]{fillColor}{rgb}{0.78,0.49,1.00}

\path[fill=fillColor] (286.97,146.45) rectangle (293.03,175.62);
\definecolor[named]{drawColor}{rgb}{1.00,0.00,0.00}
\definecolor[named]{fillColor}{rgb}{1.00,0.00,0.00}

\path[draw=drawColor,line width= 0.6pt,dash pattern=on 4pt off 4pt ,line join=round,fill=fillColor] (210.88,161.72) -- (297.07,161.72);
\definecolor[named]{drawColor}{rgb}{0.00,0.00,1.00}
\definecolor[named]{fillColor}{rgb}{0.00,0.00,1.00}

\path[draw=drawColor,line width= 0.6pt,dash pattern=on 4pt off 4pt ,line join=round,fill=fillColor] (210.88,176.98) -- (297.07,176.98);
\end{scope}
\begin{scope}
\path[clip] (210.88, 58.60) rectangle (297.07,139.75);
\definecolor[named]{drawColor}{rgb}{0.75,0.75,0.75}

\path[draw=drawColor,line width= 1.5pt,line join=round,line cap=round] (210.88, 58.60) rectangle (297.07,139.75);
\definecolor[named]{drawColor}{rgb}{0.90,0.90,0.90}

\path[draw=drawColor,line width= 0.3pt,line join=round] (210.88, 71.51) --
	(297.07, 71.51);

\path[draw=drawColor,line width= 0.3pt,line join=round] (210.88, 89.96) --
	(297.07, 89.96);

\path[draw=drawColor,line width= 0.3pt,line join=round] (210.88,108.40) --
	(297.07,108.40);

\path[draw=drawColor,line width= 0.3pt,line join=round] (210.88,126.84) --
	(297.07,126.84);

\path[draw=drawColor,line width= 0.8pt,line join=round] (210.88, 62.29) --
	(297.07, 62.29);

\path[draw=drawColor,line width= 0.8pt,line join=round] (210.88, 80.73) --
	(297.07, 80.73);

\path[draw=drawColor,line width= 0.8pt,line join=round] (210.88, 99.18) --
	(297.07, 99.18);

\path[draw=drawColor,line width= 0.8pt,line join=round] (210.88,117.62) --
	(297.07,117.62);

\path[draw=drawColor,line width= 0.8pt,line join=round] (210.88,136.07) --
	(297.07,136.07);

\path[draw=drawColor,line width= 0.8pt,line join=round] (227.04, 58.60) --
	(227.04,139.75);

\path[draw=drawColor,line width= 0.8pt,line join=round] (253.98, 58.60) --
	(253.98,139.75);

\path[draw=drawColor,line width= 0.8pt,line join=round] (280.91, 58.60) --
	(280.91,139.75);
\definecolor[named]{fillColor}{rgb}{0.97,0.46,0.43}

\path[fill=fillColor] (214.92, 62.29) rectangle (220.98,101.01);
\definecolor[named]{fillColor}{rgb}{0.49,0.68,0.00}

\path[fill=fillColor] (220.98, 62.29) rectangle (227.04, 99.78);
\definecolor[named]{fillColor}{rgb}{0.00,0.75,0.77}

\path[fill=fillColor] (227.04, 62.29) rectangle (233.10,112.07);
\definecolor[named]{fillColor}{rgb}{0.78,0.49,1.00}

\path[fill=fillColor] (233.10, 62.29) rectangle (239.16, 99.89);
\definecolor[named]{fillColor}{rgb}{0.97,0.46,0.43}

\path[fill=fillColor] (241.86, 62.29) rectangle (247.92,103.48);
\definecolor[named]{fillColor}{rgb}{0.49,0.68,0.00}

\path[fill=fillColor] (247.92, 62.29) rectangle (253.98,109.61);
\definecolor[named]{fillColor}{rgb}{0.00,0.75,0.77}

\path[fill=fillColor] (253.98, 62.29) rectangle (260.04,103.15);
\definecolor[named]{fillColor}{rgb}{0.78,0.49,1.00}

\path[fill=fillColor] (260.04, 62.29) rectangle (266.10,101.92);
\definecolor[named]{fillColor}{rgb}{0.97,0.46,0.43}

\path[fill=fillColor] (268.79, 62.29) rectangle (274.85,104.84);
\definecolor[named]{fillColor}{rgb}{0.49,0.68,0.00}

\path[fill=fillColor] (274.85, 62.29) rectangle (280.91,109.56);
\definecolor[named]{fillColor}{rgb}{0.00,0.75,0.77}

\path[fill=fillColor] (280.91, 62.29) rectangle (286.97,103.35);
\definecolor[named]{fillColor}{rgb}{0.78,0.49,1.00}

\path[fill=fillColor] (286.97, 62.29) rectangle (293.03,103.86);
\definecolor[named]{drawColor}{rgb}{1.00,0.00,0.00}
\definecolor[named]{fillColor}{rgb}{1.00,0.00,0.00}

\path[draw=drawColor,line width= 0.6pt,dash pattern=on 4pt off 4pt ,line join=round,fill=fillColor] (210.88, 99.18) -- (297.07, 99.18);
\definecolor[named]{drawColor}{rgb}{0.00,0.00,1.00}
\definecolor[named]{fillColor}{rgb}{0.00,0.00,1.00}

\path[draw=drawColor,line width= 0.6pt,dash pattern=on 4pt off 4pt ,line join=round,fill=fillColor] (210.88,136.07) -- (297.07,136.07);
\end{scope}
\begin{scope}
\path[clip] (300.08,479.43) rectangle (386.27,560.58);
\definecolor[named]{drawColor}{rgb}{0.75,0.75,0.75}

\path[draw=drawColor,line width= 1.5pt,line join=round,line cap=round] (300.08,479.43) rectangle (386.27,560.58);
\definecolor[named]{drawColor}{rgb}{0.90,0.90,0.90}

\path[draw=drawColor,line width= 0.3pt,line join=round] (300.08,490.37) --
	(386.27,490.37);

\path[draw=drawColor,line width= 0.3pt,line join=round] (300.08,504.87) --
	(386.27,504.87);

\path[draw=drawColor,line width= 0.3pt,line join=round] (300.08,519.37) --
	(386.27,519.37);

\path[draw=drawColor,line width= 0.3pt,line join=round] (300.08,533.87) --
	(386.27,533.87);

\path[draw=drawColor,line width= 0.3pt,line join=round] (300.08,548.37) --
	(386.27,548.37);

\path[draw=drawColor,line width= 0.8pt,line join=round] (300.08,483.12) --
	(386.27,483.12);

\path[draw=drawColor,line width= 0.8pt,line join=round] (300.08,497.62) --
	(386.27,497.62);

\path[draw=drawColor,line width= 0.8pt,line join=round] (300.08,512.12) --
	(386.27,512.12);

\path[draw=drawColor,line width= 0.8pt,line join=round] (300.08,526.62) --
	(386.27,526.62);

\path[draw=drawColor,line width= 0.8pt,line join=round] (300.08,541.12) --
	(386.27,541.12);

\path[draw=drawColor,line width= 0.8pt,line join=round] (300.08,555.62) --
	(386.27,555.62);

\path[draw=drawColor,line width= 0.8pt,line join=round] (316.24,479.43) --
	(316.24,560.58);

\path[draw=drawColor,line width= 0.8pt,line join=round] (343.18,479.43) --
	(343.18,560.58);

\path[draw=drawColor,line width= 0.8pt,line join=round] (370.11,479.43) --
	(370.11,560.58);
\definecolor[named]{fillColor}{rgb}{0.97,0.46,0.43}

\path[fill=fillColor] (304.12,483.12) rectangle (310.18,501.71);
\definecolor[named]{fillColor}{rgb}{0.49,0.68,0.00}

\path[fill=fillColor] (310.18,483.12) rectangle (316.24,505.05);
\definecolor[named]{fillColor}{rgb}{0.00,0.75,0.77}

\path[fill=fillColor] (316.24,483.12) rectangle (322.30,498.55);
\definecolor[named]{fillColor}{rgb}{0.78,0.49,1.00}

\path[fill=fillColor] (322.30,483.12) rectangle (328.36,498.68);
\definecolor[named]{fillColor}{rgb}{0.97,0.46,0.43}

\path[fill=fillColor] (331.05,483.12) rectangle (337.12,512.40);
\definecolor[named]{fillColor}{rgb}{0.49,0.68,0.00}

\path[fill=fillColor] (337.12,483.12) rectangle (343.18,535.53);
\definecolor[named]{fillColor}{rgb}{0.00,0.75,0.77}

\path[fill=fillColor] (343.18,483.12) rectangle (349.24,506.75);
\definecolor[named]{fillColor}{rgb}{0.78,0.49,1.00}

\path[fill=fillColor] (349.24,483.12) rectangle (355.30,505.85);
\definecolor[named]{fillColor}{rgb}{0.97,0.46,0.43}

\path[fill=fillColor] (357.99,483.12) rectangle (364.05,513.81);
\definecolor[named]{fillColor}{rgb}{0.49,0.68,0.00}

\path[fill=fillColor] (364.05,483.12) rectangle (370.11,541.82);
\definecolor[named]{fillColor}{rgb}{0.00,0.75,0.77}

\path[fill=fillColor] (370.11,483.12) rectangle (376.17,507.63);
\definecolor[named]{fillColor}{rgb}{0.78,0.49,1.00}

\path[fill=fillColor] (376.17,483.12) rectangle (382.23,506.67);
\definecolor[named]{drawColor}{rgb}{1.00,0.00,0.00}
\definecolor[named]{fillColor}{rgb}{1.00,0.00,0.00}

\path[draw=drawColor,line width= 0.6pt,dash pattern=on 4pt off 4pt ,line join=round,fill=fillColor] (300.08,497.62) -- (386.27,497.62);
\definecolor[named]{drawColor}{rgb}{0.00,0.00,1.00}
\definecolor[named]{fillColor}{rgb}{0.00,0.00,1.00}

\path[draw=drawColor,line width= 0.6pt,dash pattern=on 4pt off 4pt ,line join=round,fill=fillColor] (300.08,512.12) -- (386.27,512.12);
\end{scope}
\begin{scope}
\path[clip] (300.08,395.26) rectangle (386.27,476.42);
\definecolor[named]{drawColor}{rgb}{0.75,0.75,0.75}

\path[draw=drawColor,line width= 1.5pt,line join=round,line cap=round] (300.08,395.26) rectangle (386.27,476.42);
\definecolor[named]{drawColor}{rgb}{0.90,0.90,0.90}

\path[draw=drawColor,line width= 0.3pt,line join=round] (300.08,407.91) --
	(386.27,407.91);

\path[draw=drawColor,line width= 0.3pt,line join=round] (300.08,425.84) --
	(386.27,425.84);

\path[draw=drawColor,line width= 0.3pt,line join=round] (300.08,443.76) --
	(386.27,443.76);

\path[draw=drawColor,line width= 0.3pt,line join=round] (300.08,461.69) --
	(386.27,461.69);

\path[draw=drawColor,line width= 0.8pt,line join=round] (300.08,398.95) --
	(386.27,398.95);

\path[draw=drawColor,line width= 0.8pt,line join=round] (300.08,416.88) --
	(386.27,416.88);

\path[draw=drawColor,line width= 0.8pt,line join=round] (300.08,434.80) --
	(386.27,434.80);

\path[draw=drawColor,line width= 0.8pt,line join=round] (300.08,452.73) --
	(386.27,452.73);

\path[draw=drawColor,line width= 0.8pt,line join=round] (300.08,470.65) --
	(386.27,470.65);

\path[draw=drawColor,line width= 0.8pt,line join=round] (316.24,395.26) --
	(316.24,476.42);

\path[draw=drawColor,line width= 0.8pt,line join=round] (343.18,395.26) --
	(343.18,476.42);

\path[draw=drawColor,line width= 0.8pt,line join=round] (370.11,395.26) --
	(370.11,476.42);
\definecolor[named]{fillColor}{rgb}{0.97,0.46,0.43}

\path[fill=fillColor] (304.12,398.95) rectangle (310.18,420.09);
\definecolor[named]{fillColor}{rgb}{0.49,0.68,0.00}

\path[fill=fillColor] (310.18,398.95) rectangle (316.24,424.06);
\definecolor[named]{fillColor}{rgb}{0.00,0.75,0.77}

\path[fill=fillColor] (316.24,398.95) rectangle (322.30,417.72);
\definecolor[named]{fillColor}{rgb}{0.78,0.49,1.00}

\path[fill=fillColor] (322.30,398.95) rectangle (328.36,417.85);
\definecolor[named]{fillColor}{rgb}{0.97,0.46,0.43}

\path[fill=fillColor] (331.05,398.95) rectangle (337.12,430.18);
\definecolor[named]{fillColor}{rgb}{0.49,0.68,0.00}

\path[fill=fillColor] (337.12,398.95) rectangle (343.18,462.17);
\definecolor[named]{fillColor}{rgb}{0.00,0.75,0.77}

\path[fill=fillColor] (343.18,398.95) rectangle (349.24,426.48);
\definecolor[named]{fillColor}{rgb}{0.78,0.49,1.00}

\path[fill=fillColor] (349.24,398.95) rectangle (355.30,424.89);
\definecolor[named]{fillColor}{rgb}{0.97,0.46,0.43}

\path[fill=fillColor] (357.99,398.95) rectangle (364.05,431.55);
\definecolor[named]{fillColor}{rgb}{0.49,0.68,0.00}

\path[fill=fillColor] (364.05,398.95) rectangle (370.11,472.73);
\definecolor[named]{fillColor}{rgb}{0.00,0.75,0.77}

\path[fill=fillColor] (370.11,398.95) rectangle (376.17,426.97);
\definecolor[named]{fillColor}{rgb}{0.78,0.49,1.00}

\path[fill=fillColor] (376.17,398.95) rectangle (382.23,425.57);
\definecolor[named]{drawColor}{rgb}{1.00,0.00,0.00}
\definecolor[named]{fillColor}{rgb}{1.00,0.00,0.00}

\path[draw=drawColor,line width= 0.6pt,dash pattern=on 4pt off 4pt ,line join=round,fill=fillColor] (300.08,416.88) -- (386.27,416.88);
\definecolor[named]{drawColor}{rgb}{0.00,0.00,1.00}
\definecolor[named]{fillColor}{rgb}{0.00,0.00,1.00}

\path[draw=drawColor,line width= 0.6pt,dash pattern=on 4pt off 4pt ,line join=round,fill=fillColor] (300.08,434.80) -- (386.27,434.80);
\end{scope}
\begin{scope}
\path[clip] (300.08,311.10) rectangle (386.27,392.25);
\definecolor[named]{drawColor}{rgb}{0.75,0.75,0.75}

\path[draw=drawColor,line width= 1.5pt,line join=round,line cap=round] (300.08,311.10) rectangle (386.27,392.25);
\definecolor[named]{drawColor}{rgb}{0.90,0.90,0.90}

\path[draw=drawColor,line width= 0.3pt,line join=round] (300.08,323.60) --
	(386.27,323.60);

\path[draw=drawColor,line width= 0.3pt,line join=round] (300.08,341.24) --
	(386.27,341.24);

\path[draw=drawColor,line width= 0.3pt,line join=round] (300.08,358.87) --
	(386.27,358.87);

\path[draw=drawColor,line width= 0.3pt,line join=round] (300.08,376.51) --
	(386.27,376.51);

\path[draw=drawColor,line width= 0.8pt,line join=round] (300.08,314.79) --
	(386.27,314.79);

\path[draw=drawColor,line width= 0.8pt,line join=round] (300.08,332.42) --
	(386.27,332.42);

\path[draw=drawColor,line width= 0.8pt,line join=round] (300.08,350.06) --
	(386.27,350.06);

\path[draw=drawColor,line width= 0.8pt,line join=round] (300.08,367.69) --
	(386.27,367.69);

\path[draw=drawColor,line width= 0.8pt,line join=round] (300.08,385.33) --
	(386.27,385.33);

\path[draw=drawColor,line width= 0.8pt,line join=round] (316.24,311.10) --
	(316.24,392.25);

\path[draw=drawColor,line width= 0.8pt,line join=round] (343.18,311.10) --
	(343.18,392.25);

\path[draw=drawColor,line width= 0.8pt,line join=round] (370.11,311.10) --
	(370.11,392.25);
\definecolor[named]{fillColor}{rgb}{0.97,0.46,0.43}

\path[fill=fillColor] (304.12,314.79) rectangle (310.18,339.97);
\definecolor[named]{fillColor}{rgb}{0.49,0.68,0.00}

\path[fill=fillColor] (310.18,314.79) rectangle (316.24,358.15);
\definecolor[named]{fillColor}{rgb}{0.00,0.75,0.77}

\path[fill=fillColor] (316.24,314.79) rectangle (322.30,339.33);
\definecolor[named]{fillColor}{rgb}{0.78,0.49,1.00}

\path[fill=fillColor] (322.30,314.79) rectangle (328.36,339.70);
\definecolor[named]{fillColor}{rgb}{0.97,0.46,0.43}

\path[fill=fillColor] (331.05,314.79) rectangle (337.12,335.11);
\definecolor[named]{fillColor}{rgb}{0.49,0.68,0.00}

\path[fill=fillColor] (337.12,314.79) rectangle (343.18,341.47);
\definecolor[named]{fillColor}{rgb}{0.00,0.75,0.77}

\path[fill=fillColor] (343.18,314.79) rectangle (349.24,334.96);
\definecolor[named]{fillColor}{rgb}{0.78,0.49,1.00}

\path[fill=fillColor] (349.24,314.79) rectangle (355.30,334.80);
\definecolor[named]{fillColor}{rgb}{0.97,0.46,0.43}

\path[fill=fillColor] (357.99,314.79) rectangle (364.05,340.65);
\definecolor[named]{fillColor}{rgb}{0.49,0.68,0.00}

\path[fill=fillColor] (364.05,314.79) rectangle (370.11,363.66);
\definecolor[named]{fillColor}{rgb}{0.00,0.75,0.77}

\path[fill=fillColor] (370.11,314.79) rectangle (376.17,341.25);
\definecolor[named]{fillColor}{rgb}{0.78,0.49,1.00}

\path[fill=fillColor] (376.17,314.79) rectangle (382.23,340.76);
\definecolor[named]{drawColor}{rgb}{1.00,0.00,0.00}
\definecolor[named]{fillColor}{rgb}{1.00,0.00,0.00}

\path[draw=drawColor,line width= 0.6pt,dash pattern=on 4pt off 4pt ,line join=round,fill=fillColor] (300.08,332.42) -- (386.27,332.42);
\definecolor[named]{drawColor}{rgb}{0.00,0.00,1.00}
\definecolor[named]{fillColor}{rgb}{0.00,0.00,1.00}

\path[draw=drawColor,line width= 0.6pt,dash pattern=on 4pt off 4pt ,line join=round,fill=fillColor] (300.08,350.06) -- (386.27,350.06);
\end{scope}
\begin{scope}
\path[clip] (300.08,226.93) rectangle (386.27,308.09);
\definecolor[named]{drawColor}{rgb}{0.75,0.75,0.75}

\path[draw=drawColor,line width= 1.5pt,line join=round,line cap=round] (300.08,226.93) rectangle (386.27,308.09);
\definecolor[named]{drawColor}{rgb}{0.90,0.90,0.90}

\path[draw=drawColor,line width= 0.3pt,line join=round] (300.08,240.40) --
	(386.27,240.40);

\path[draw=drawColor,line width= 0.3pt,line join=round] (300.08,259.96) --
	(386.27,259.96);

\path[draw=drawColor,line width= 0.3pt,line join=round] (300.08,279.52) --
	(386.27,279.52);

\path[draw=drawColor,line width= 0.3pt,line join=round] (300.08,299.08) --
	(386.27,299.08);

\path[draw=drawColor,line width= 0.8pt,line join=round] (300.08,230.62) --
	(386.27,230.62);

\path[draw=drawColor,line width= 0.8pt,line join=round] (300.08,250.18) --
	(386.27,250.18);

\path[draw=drawColor,line width= 0.8pt,line join=round] (300.08,269.74) --
	(386.27,269.74);

\path[draw=drawColor,line width= 0.8pt,line join=round] (300.08,289.30) --
	(386.27,289.30);

\path[draw=drawColor,line width= 0.8pt,line join=round] (316.24,226.93) --
	(316.24,308.09);

\path[draw=drawColor,line width= 0.8pt,line join=round] (343.18,226.93) --
	(343.18,308.09);

\path[draw=drawColor,line width= 0.8pt,line join=round] (370.11,226.93) --
	(370.11,308.09);
\definecolor[named]{fillColor}{rgb}{0.97,0.46,0.43}

\path[fill=fillColor] (304.12,230.62) rectangle (310.18,251.19);
\definecolor[named]{fillColor}{rgb}{0.49,0.68,0.00}

\path[fill=fillColor] (310.18,230.62) rectangle (316.24,250.63);
\definecolor[named]{fillColor}{rgb}{0.00,0.75,0.77}

\path[fill=fillColor] (316.24,230.62) rectangle (322.30,251.45);
\definecolor[named]{fillColor}{rgb}{0.78,0.49,1.00}

\path[fill=fillColor] (322.30,230.62) rectangle (328.36,249.31);
\definecolor[named]{fillColor}{rgb}{0.97,0.46,0.43}

\path[fill=fillColor] (331.05,230.62) rectangle (337.12,275.40);
\definecolor[named]{fillColor}{rgb}{0.49,0.68,0.00}

\path[fill=fillColor] (337.12,230.62) rectangle (343.18,293.00);
\definecolor[named]{fillColor}{rgb}{0.00,0.75,0.77}

\path[fill=fillColor] (343.18,230.62) rectangle (349.24,268.93);
\definecolor[named]{fillColor}{rgb}{0.78,0.49,1.00}

\path[fill=fillColor] (349.24,230.62) rectangle (355.30,267.28);
\definecolor[named]{fillColor}{rgb}{0.97,0.46,0.43}

\path[fill=fillColor] (357.99,230.62) rectangle (364.05,276.25);
\definecolor[named]{fillColor}{rgb}{0.49,0.68,0.00}

\path[fill=fillColor] (364.05,230.62) rectangle (370.11,294.00);
\definecolor[named]{fillColor}{rgb}{0.00,0.75,0.77}

\path[fill=fillColor] (370.11,230.62) rectangle (376.17,269.82);
\definecolor[named]{fillColor}{rgb}{0.78,0.49,1.00}

\path[fill=fillColor] (376.17,230.62) rectangle (382.23,267.96);
\definecolor[named]{drawColor}{rgb}{1.00,0.00,0.00}
\definecolor[named]{fillColor}{rgb}{1.00,0.00,0.00}

\path[draw=drawColor,line width= 0.6pt,dash pattern=on 4pt off 4pt ,line join=round,fill=fillColor] (300.08,250.18) -- (386.27,250.18);
\definecolor[named]{drawColor}{rgb}{0.00,0.00,1.00}
\definecolor[named]{fillColor}{rgb}{0.00,0.00,1.00}

\path[draw=drawColor,line width= 0.6pt,dash pattern=on 4pt off 4pt ,line join=round,fill=fillColor] (300.08,269.74) -- (386.27,269.74);
\end{scope}
\begin{scope}
\path[clip] (300.08,142.77) rectangle (386.27,223.92);
\definecolor[named]{drawColor}{rgb}{0.75,0.75,0.75}

\path[draw=drawColor,line width= 1.5pt,line join=round,line cap=round] (300.08,142.77) rectangle (386.27,223.92);
\definecolor[named]{drawColor}{rgb}{0.90,0.90,0.90}

\path[draw=drawColor,line width= 0.3pt,line join=round] (300.08,154.09) --
	(386.27,154.09);

\path[draw=drawColor,line width= 0.3pt,line join=round] (300.08,169.35) --
	(386.27,169.35);

\path[draw=drawColor,line width= 0.3pt,line join=round] (300.08,184.61) --
	(386.27,184.61);

\path[draw=drawColor,line width= 0.3pt,line join=round] (300.08,199.88) --
	(386.27,199.88);

\path[draw=drawColor,line width= 0.3pt,line join=round] (300.08,215.14) --
	(386.27,215.14);

\path[draw=drawColor,line width= 0.8pt,line join=round] (300.08,146.45) --
	(386.27,146.45);

\path[draw=drawColor,line width= 0.8pt,line join=round] (300.08,161.72) --
	(386.27,161.72);

\path[draw=drawColor,line width= 0.8pt,line join=round] (300.08,176.98) --
	(386.27,176.98);

\path[draw=drawColor,line width= 0.8pt,line join=round] (300.08,192.24) --
	(386.27,192.24);

\path[draw=drawColor,line width= 0.8pt,line join=round] (300.08,207.51) --
	(386.27,207.51);

\path[draw=drawColor,line width= 0.8pt,line join=round] (300.08,222.77) --
	(386.27,222.77);

\path[draw=drawColor,line width= 0.8pt,line join=round] (316.24,142.77) --
	(316.24,223.92);

\path[draw=drawColor,line width= 0.8pt,line join=round] (343.18,142.77) --
	(343.18,223.92);

\path[draw=drawColor,line width= 0.8pt,line join=round] (370.11,142.77) --
	(370.11,223.92);
\definecolor[named]{fillColor}{rgb}{0.97,0.46,0.43}

\path[fill=fillColor] (304.12,146.45) rectangle (310.18,164.12);
\definecolor[named]{fillColor}{rgb}{0.49,0.68,0.00}

\path[fill=fillColor] (310.18,146.45) rectangle (316.24,164.03);
\definecolor[named]{fillColor}{rgb}{0.00,0.75,0.77}

\path[fill=fillColor] (316.24,146.45) rectangle (322.30,162.91);
\definecolor[named]{fillColor}{rgb}{0.78,0.49,1.00}

\path[fill=fillColor] (322.30,146.45) rectangle (328.36,161.72);
\definecolor[named]{fillColor}{rgb}{0.97,0.46,0.43}

\path[fill=fillColor] (331.05,146.45) rectangle (337.12,180.32);
\definecolor[named]{fillColor}{rgb}{0.49,0.68,0.00}

\path[fill=fillColor] (337.12,146.45) rectangle (343.18,209.48);
\definecolor[named]{fillColor}{rgb}{0.00,0.75,0.77}

\path[fill=fillColor] (343.18,146.45) rectangle (349.24,175.17);
\definecolor[named]{fillColor}{rgb}{0.78,0.49,1.00}

\path[fill=fillColor] (349.24,146.45) rectangle (355.30,173.82);
\definecolor[named]{fillColor}{rgb}{0.97,0.46,0.43}

\path[fill=fillColor] (357.99,146.45) rectangle (364.05,182.68);
\definecolor[named]{fillColor}{rgb}{0.49,0.68,0.00}

\path[fill=fillColor] (364.05,146.45) rectangle (370.11,213.53);
\definecolor[named]{fillColor}{rgb}{0.00,0.75,0.77}

\path[fill=fillColor] (370.11,146.45) rectangle (376.17,176.24);
\definecolor[named]{fillColor}{rgb}{0.78,0.49,1.00}

\path[fill=fillColor] (376.17,146.45) rectangle (382.23,175.17);
\definecolor[named]{drawColor}{rgb}{1.00,0.00,0.00}
\definecolor[named]{fillColor}{rgb}{1.00,0.00,0.00}

\path[draw=drawColor,line width= 0.6pt,dash pattern=on 4pt off 4pt ,line join=round,fill=fillColor] (300.08,161.72) -- (386.27,161.72);
\definecolor[named]{drawColor}{rgb}{0.00,0.00,1.00}
\definecolor[named]{fillColor}{rgb}{0.00,0.00,1.00}

\path[draw=drawColor,line width= 0.6pt,dash pattern=on 4pt off 4pt ,line join=round,fill=fillColor] (300.08,176.98) -- (386.27,176.98);
\end{scope}
\begin{scope}
\path[clip] (300.08, 58.60) rectangle (386.27,139.75);
\definecolor[named]{drawColor}{rgb}{0.75,0.75,0.75}

\path[draw=drawColor,line width= 1.5pt,line join=round,line cap=round] (300.08, 58.60) rectangle (386.27,139.75);
\definecolor[named]{drawColor}{rgb}{0.90,0.90,0.90}

\path[draw=drawColor,line width= 0.3pt,line join=round] (300.08, 71.51) --
	(386.27, 71.51);

\path[draw=drawColor,line width= 0.3pt,line join=round] (300.08, 89.96) --
	(386.27, 89.96);

\path[draw=drawColor,line width= 0.3pt,line join=round] (300.08,108.40) --
	(386.27,108.40);

\path[draw=drawColor,line width= 0.3pt,line join=round] (300.08,126.84) --
	(386.27,126.84);

\path[draw=drawColor,line width= 0.8pt,line join=round] (300.08, 62.29) --
	(386.27, 62.29);

\path[draw=drawColor,line width= 0.8pt,line join=round] (300.08, 80.73) --
	(386.27, 80.73);

\path[draw=drawColor,line width= 0.8pt,line join=round] (300.08, 99.18) --
	(386.27, 99.18);

\path[draw=drawColor,line width= 0.8pt,line join=round] (300.08,117.62) --
	(386.27,117.62);

\path[draw=drawColor,line width= 0.8pt,line join=round] (300.08,136.07) --
	(386.27,136.07);

\path[draw=drawColor,line width= 0.8pt,line join=round] (316.24, 58.60) --
	(316.24,139.75);

\path[draw=drawColor,line width= 0.8pt,line join=round] (343.18, 58.60) --
	(343.18,139.75);

\path[draw=drawColor,line width= 0.8pt,line join=round] (370.11, 58.60) --
	(370.11,139.75);
\definecolor[named]{fillColor}{rgb}{0.97,0.46,0.43}

\path[fill=fillColor] (304.12, 62.29) rectangle (310.18,101.23);
\definecolor[named]{fillColor}{rgb}{0.49,0.68,0.00}

\path[fill=fillColor] (310.18, 62.29) rectangle (316.24,100.46);
\definecolor[named]{fillColor}{rgb}{0.00,0.75,0.77}

\path[fill=fillColor] (316.24, 62.29) rectangle (322.30,113.29);
\definecolor[named]{fillColor}{rgb}{0.78,0.49,1.00}

\path[fill=fillColor] (322.30, 62.29) rectangle (328.36, 99.67);
\definecolor[named]{fillColor}{rgb}{0.97,0.46,0.43}

\path[fill=fillColor] (331.05, 62.29) rectangle (337.12,102.40);
\definecolor[named]{fillColor}{rgb}{0.49,0.68,0.00}

\path[fill=fillColor] (337.12, 62.29) rectangle (343.18,109.05);
\definecolor[named]{fillColor}{rgb}{0.00,0.75,0.77}

\path[fill=fillColor] (343.18, 62.29) rectangle (349.24,102.14);
\definecolor[named]{fillColor}{rgb}{0.78,0.49,1.00}

\path[fill=fillColor] (349.24, 62.29) rectangle (355.30,101.11);
\definecolor[named]{fillColor}{rgb}{0.97,0.46,0.43}

\path[fill=fillColor] (357.99, 62.29) rectangle (364.05,103.83);
\definecolor[named]{fillColor}{rgb}{0.49,0.68,0.00}

\path[fill=fillColor] (364.05, 62.29) rectangle (370.11,109.62);
\definecolor[named]{fillColor}{rgb}{0.00,0.75,0.77}

\path[fill=fillColor] (370.11, 62.29) rectangle (376.17,101.82);
\definecolor[named]{fillColor}{rgb}{0.78,0.49,1.00}

\path[fill=fillColor] (376.17, 62.29) rectangle (382.23,102.77);
\definecolor[named]{drawColor}{rgb}{1.00,0.00,0.00}
\definecolor[named]{fillColor}{rgb}{1.00,0.00,0.00}

\path[draw=drawColor,line width= 0.6pt,dash pattern=on 4pt off 4pt ,line join=round,fill=fillColor] (300.08, 99.18) -- (386.27, 99.18);
\definecolor[named]{drawColor}{rgb}{0.00,0.00,1.00}
\definecolor[named]{fillColor}{rgb}{0.00,0.00,1.00}

\path[draw=drawColor,line width= 0.6pt,dash pattern=on 4pt off 4pt ,line join=round,fill=fillColor] (300.08,136.07) -- (386.27,136.07);
\end{scope}
\begin{scope}
\path[clip] (389.28,479.43) rectangle (475.47,560.58);
\definecolor[named]{drawColor}{rgb}{0.75,0.75,0.75}

\path[draw=drawColor,line width= 1.5pt,line join=round,line cap=round] (389.28,479.43) rectangle (475.47,560.58);
\definecolor[named]{drawColor}{rgb}{0.90,0.90,0.90}

\path[draw=drawColor,line width= 0.3pt,line join=round] (389.28,490.37) --
	(475.47,490.37);

\path[draw=drawColor,line width= 0.3pt,line join=round] (389.28,504.87) --
	(475.47,504.87);

\path[draw=drawColor,line width= 0.3pt,line join=round] (389.28,519.37) --
	(475.47,519.37);

\path[draw=drawColor,line width= 0.3pt,line join=round] (389.28,533.87) --
	(475.47,533.87);

\path[draw=drawColor,line width= 0.3pt,line join=round] (389.28,548.37) --
	(475.47,548.37);

\path[draw=drawColor,line width= 0.8pt,line join=round] (389.28,483.12) --
	(475.47,483.12);

\path[draw=drawColor,line width= 0.8pt,line join=round] (389.28,497.62) --
	(475.47,497.62);

\path[draw=drawColor,line width= 0.8pt,line join=round] (389.28,512.12) --
	(475.47,512.12);

\path[draw=drawColor,line width= 0.8pt,line join=round] (389.28,526.62) --
	(475.47,526.62);

\path[draw=drawColor,line width= 0.8pt,line join=round] (389.28,541.12) --
	(475.47,541.12);

\path[draw=drawColor,line width= 0.8pt,line join=round] (389.28,555.62) --
	(475.47,555.62);

\path[draw=drawColor,line width= 0.8pt,line join=round] (405.44,479.43) --
	(405.44,560.58);

\path[draw=drawColor,line width= 0.8pt,line join=round] (432.37,479.43) --
	(432.37,560.58);

\path[draw=drawColor,line width= 0.8pt,line join=round] (459.31,479.43) --
	(459.31,560.58);
\definecolor[named]{fillColor}{rgb}{0.97,0.46,0.43}

\path[fill=fillColor] (393.32,483.12) rectangle (399.38,509.13);
\definecolor[named]{fillColor}{rgb}{0.49,0.68,0.00}

\path[fill=fillColor] (399.38,483.12) rectangle (405.44,507.16);
\definecolor[named]{fillColor}{rgb}{0.00,0.75,0.77}

\path[fill=fillColor] (405.44,483.12) rectangle (411.50,499.12);
\definecolor[named]{fillColor}{rgb}{0.78,0.49,1.00}

\path[fill=fillColor] (411.50,483.12) rectangle (417.56,498.73);
\definecolor[named]{fillColor}{rgb}{0.97,0.46,0.43}

\path[fill=fillColor] (420.25,483.12) rectangle (426.31,510.02);
\definecolor[named]{fillColor}{rgb}{0.49,0.68,0.00}

\path[fill=fillColor] (426.31,483.12) rectangle (432.37,545.11);
\definecolor[named]{fillColor}{rgb}{0.00,0.75,0.77}

\path[fill=fillColor] (432.37,483.12) rectangle (438.43,506.55);
\definecolor[named]{fillColor}{rgb}{0.78,0.49,1.00}

\path[fill=fillColor] (438.43,483.12) rectangle (444.49,505.46);
\definecolor[named]{fillColor}{rgb}{0.97,0.46,0.43}

\path[fill=fillColor] (447.19,483.12) rectangle (453.25,513.42);
\definecolor[named]{fillColor}{rgb}{0.49,0.68,0.00}

\path[fill=fillColor] (453.25,483.12) rectangle (459.31,556.90);
\definecolor[named]{fillColor}{rgb}{0.00,0.75,0.77}

\path[fill=fillColor] (459.31,483.12) rectangle (465.37,507.93);
\definecolor[named]{fillColor}{rgb}{0.78,0.49,1.00}

\path[fill=fillColor] (465.37,483.12) rectangle (471.43,506.95);
\definecolor[named]{drawColor}{rgb}{1.00,0.00,0.00}
\definecolor[named]{fillColor}{rgb}{1.00,0.00,0.00}

\path[draw=drawColor,line width= 0.6pt,dash pattern=on 4pt off 4pt ,line join=round,fill=fillColor] (389.28,497.62) -- (475.47,497.62);
\definecolor[named]{drawColor}{rgb}{0.00,0.00,1.00}
\definecolor[named]{fillColor}{rgb}{0.00,0.00,1.00}

\path[draw=drawColor,line width= 0.6pt,dash pattern=on 4pt off 4pt ,line join=round,fill=fillColor] (389.28,512.12) -- (475.47,512.12);
\end{scope}
\begin{scope}
\path[clip] (389.28,395.26) rectangle (475.47,476.42);
\definecolor[named]{drawColor}{rgb}{0.75,0.75,0.75}

\path[draw=drawColor,line width= 1.5pt,line join=round,line cap=round] (389.28,395.26) rectangle (475.47,476.42);
\definecolor[named]{drawColor}{rgb}{0.90,0.90,0.90}

\path[draw=drawColor,line width= 0.3pt,line join=round] (389.28,407.91) --
	(475.47,407.91);

\path[draw=drawColor,line width= 0.3pt,line join=round] (389.28,425.84) --
	(475.47,425.84);

\path[draw=drawColor,line width= 0.3pt,line join=round] (389.28,443.76) --
	(475.47,443.76);

\path[draw=drawColor,line width= 0.3pt,line join=round] (389.28,461.69) --
	(475.47,461.69);

\path[draw=drawColor,line width= 0.8pt,line join=round] (389.28,398.95) --
	(475.47,398.95);

\path[draw=drawColor,line width= 0.8pt,line join=round] (389.28,416.88) --
	(475.47,416.88);

\path[draw=drawColor,line width= 0.8pt,line join=round] (389.28,434.80) --
	(475.47,434.80);

\path[draw=drawColor,line width= 0.8pt,line join=round] (389.28,452.73) --
	(475.47,452.73);

\path[draw=drawColor,line width= 0.8pt,line join=round] (389.28,470.65) --
	(475.47,470.65);

\path[draw=drawColor,line width= 0.8pt,line join=round] (405.44,395.26) --
	(405.44,476.42);

\path[draw=drawColor,line width= 0.8pt,line join=round] (432.37,395.26) --
	(432.37,476.42);

\path[draw=drawColor,line width= 0.8pt,line join=round] (459.31,395.26) --
	(459.31,476.42);
\definecolor[named]{fillColor}{rgb}{0.97,0.46,0.43}

\path[fill=fillColor] (393.32,398.95) rectangle (399.38,421.10);
\definecolor[named]{fillColor}{rgb}{0.49,0.68,0.00}

\path[fill=fillColor] (399.38,398.95) rectangle (405.44,423.22);
\definecolor[named]{fillColor}{rgb}{0.00,0.75,0.77}

\path[fill=fillColor] (405.44,398.95) rectangle (411.50,417.72);
\definecolor[named]{fillColor}{rgb}{0.78,0.49,1.00}

\path[fill=fillColor] (411.50,398.95) rectangle (417.56,417.72);
\definecolor[named]{fillColor}{rgb}{0.97,0.46,0.43}

\path[fill=fillColor] (420.25,398.95) rectangle (426.31,426.92);
\definecolor[named]{fillColor}{rgb}{0.49,0.68,0.00}

\path[fill=fillColor] (426.31,398.95) rectangle (432.37,453.06);
\definecolor[named]{fillColor}{rgb}{0.00,0.75,0.77}

\path[fill=fillColor] (432.37,398.95) rectangle (438.43,425.03);
\definecolor[named]{fillColor}{rgb}{0.78,0.49,1.00}

\path[fill=fillColor] (438.43,398.95) rectangle (444.49,423.89);
\definecolor[named]{fillColor}{rgb}{0.97,0.46,0.43}

\path[fill=fillColor] (447.19,398.95) rectangle (453.25,427.86);
\definecolor[named]{fillColor}{rgb}{0.49,0.68,0.00}

\path[fill=fillColor] (453.25,398.95) rectangle (459.31,463.64);
\definecolor[named]{fillColor}{rgb}{0.00,0.75,0.77}

\path[fill=fillColor] (459.31,398.95) rectangle (465.37,425.80);
\definecolor[named]{fillColor}{rgb}{0.78,0.49,1.00}

\path[fill=fillColor] (465.37,398.95) rectangle (471.43,424.83);
\definecolor[named]{drawColor}{rgb}{1.00,0.00,0.00}
\definecolor[named]{fillColor}{rgb}{1.00,0.00,0.00}

\path[draw=drawColor,line width= 0.6pt,dash pattern=on 4pt off 4pt ,line join=round,fill=fillColor] (389.28,416.88) -- (475.47,416.88);
\definecolor[named]{drawColor}{rgb}{0.00,0.00,1.00}
\definecolor[named]{fillColor}{rgb}{0.00,0.00,1.00}

\path[draw=drawColor,line width= 0.6pt,dash pattern=on 4pt off 4pt ,line join=round,fill=fillColor] (389.28,434.80) -- (475.47,434.80);
\end{scope}
\begin{scope}
\path[clip] (389.28,311.10) rectangle (475.47,392.25);
\definecolor[named]{drawColor}{rgb}{0.75,0.75,0.75}

\path[draw=drawColor,line width= 1.5pt,line join=round,line cap=round] (389.28,311.10) rectangle (475.47,392.25);
\definecolor[named]{drawColor}{rgb}{0.90,0.90,0.90}

\path[draw=drawColor,line width= 0.3pt,line join=round] (389.28,323.60) --
	(475.47,323.60);

\path[draw=drawColor,line width= 0.3pt,line join=round] (389.28,341.24) --
	(475.47,341.24);

\path[draw=drawColor,line width= 0.3pt,line join=round] (389.28,358.87) --
	(475.47,358.87);

\path[draw=drawColor,line width= 0.3pt,line join=round] (389.28,376.51) --
	(475.47,376.51);

\path[draw=drawColor,line width= 0.8pt,line join=round] (389.28,314.79) --
	(475.47,314.79);

\path[draw=drawColor,line width= 0.8pt,line join=round] (389.28,332.42) --
	(475.47,332.42);

\path[draw=drawColor,line width= 0.8pt,line join=round] (389.28,350.06) --
	(475.47,350.06);

\path[draw=drawColor,line width= 0.8pt,line join=round] (389.28,367.69) --
	(475.47,367.69);

\path[draw=drawColor,line width= 0.8pt,line join=round] (389.28,385.33) --
	(475.47,385.33);

\path[draw=drawColor,line width= 0.8pt,line join=round] (405.44,311.10) --
	(405.44,392.25);

\path[draw=drawColor,line width= 0.8pt,line join=round] (432.37,311.10) --
	(432.37,392.25);

\path[draw=drawColor,line width= 0.8pt,line join=round] (459.31,311.10) --
	(459.31,392.25);
\definecolor[named]{fillColor}{rgb}{0.97,0.46,0.43}

\path[fill=fillColor] (393.32,314.79) rectangle (399.38,347.61);
\definecolor[named]{fillColor}{rgb}{0.49,0.68,0.00}

\path[fill=fillColor] (399.38,314.79) rectangle (405.44,384.09);
\definecolor[named]{fillColor}{rgb}{0.00,0.75,0.77}

\path[fill=fillColor] (405.44,314.79) rectangle (411.50,342.27);
\definecolor[named]{fillColor}{rgb}{0.78,0.49,1.00}

\path[fill=fillColor] (411.50,314.79) rectangle (417.56,343.32);
\definecolor[named]{fillColor}{rgb}{0.97,0.46,0.43}

\path[fill=fillColor] (420.25,314.79) rectangle (426.31,334.43);
\definecolor[named]{fillColor}{rgb}{0.49,0.68,0.00}

\path[fill=fillColor] (426.31,314.79) rectangle (432.37,339.90);
\definecolor[named]{fillColor}{rgb}{0.00,0.75,0.77}

\path[fill=fillColor] (432.37,314.79) rectangle (438.43,334.11);
\definecolor[named]{fillColor}{rgb}{0.78,0.49,1.00}

\path[fill=fillColor] (438.43,314.79) rectangle (444.49,333.92);
\definecolor[named]{fillColor}{rgb}{0.97,0.46,0.43}

\path[fill=fillColor] (447.19,314.79) rectangle (453.25,347.78);
\definecolor[named]{fillColor}{rgb}{0.49,0.68,0.00}

\path[fill=fillColor] (453.25,314.79) rectangle (459.31,388.56);
\definecolor[named]{fillColor}{rgb}{0.00,0.75,0.77}

\path[fill=fillColor] (459.31,314.79) rectangle (465.37,344.47);
\definecolor[named]{fillColor}{rgb}{0.78,0.49,1.00}

\path[fill=fillColor] (465.37,314.79) rectangle (471.43,344.01);
\definecolor[named]{drawColor}{rgb}{1.00,0.00,0.00}
\definecolor[named]{fillColor}{rgb}{1.00,0.00,0.00}

\path[draw=drawColor,line width= 0.6pt,dash pattern=on 4pt off 4pt ,line join=round,fill=fillColor] (389.28,332.42) -- (475.47,332.42);
\definecolor[named]{drawColor}{rgb}{0.00,0.00,1.00}
\definecolor[named]{fillColor}{rgb}{0.00,0.00,1.00}

\path[draw=drawColor,line width= 0.6pt,dash pattern=on 4pt off 4pt ,line join=round,fill=fillColor] (389.28,350.06) -- (475.47,350.06);
\end{scope}
\begin{scope}
\path[clip] (389.28,226.93) rectangle (475.47,308.09);
\definecolor[named]{drawColor}{rgb}{0.75,0.75,0.75}

\path[draw=drawColor,line width= 1.5pt,line join=round,line cap=round] (389.28,226.93) rectangle (475.47,308.09);
\definecolor[named]{drawColor}{rgb}{0.90,0.90,0.90}

\path[draw=drawColor,line width= 0.3pt,line join=round] (389.28,240.40) --
	(475.47,240.40);

\path[draw=drawColor,line width= 0.3pt,line join=round] (389.28,259.96) --
	(475.47,259.96);

\path[draw=drawColor,line width= 0.3pt,line join=round] (389.28,279.52) --
	(475.47,279.52);

\path[draw=drawColor,line width= 0.3pt,line join=round] (389.28,299.08) --
	(475.47,299.08);

\path[draw=drawColor,line width= 0.8pt,line join=round] (389.28,230.62) --
	(475.47,230.62);

\path[draw=drawColor,line width= 0.8pt,line join=round] (389.28,250.18) --
	(475.47,250.18);

\path[draw=drawColor,line width= 0.8pt,line join=round] (389.28,269.74) --
	(475.47,269.74);

\path[draw=drawColor,line width= 0.8pt,line join=round] (389.28,289.30) --
	(475.47,289.30);

\path[draw=drawColor,line width= 0.8pt,line join=round] (405.44,226.93) --
	(405.44,308.09);

\path[draw=drawColor,line width= 0.8pt,line join=round] (432.37,226.93) --
	(432.37,308.09);

\path[draw=drawColor,line width= 0.8pt,line join=round] (459.31,226.93) --
	(459.31,308.09);
\definecolor[named]{fillColor}{rgb}{0.97,0.46,0.43}

\path[fill=fillColor] (393.32,230.62) rectangle (399.38,250.73);
\definecolor[named]{fillColor}{rgb}{0.49,0.68,0.00}

\path[fill=fillColor] (399.38,230.62) rectangle (405.44,250.50);
\definecolor[named]{fillColor}{rgb}{0.00,0.75,0.77}

\path[fill=fillColor] (405.44,230.62) rectangle (411.50,251.21);
\definecolor[named]{fillColor}{rgb}{0.78,0.49,1.00}

\path[fill=fillColor] (411.50,230.62) rectangle (417.56,249.40);
\definecolor[named]{fillColor}{rgb}{0.97,0.46,0.43}

\path[fill=fillColor] (420.25,230.62) rectangle (426.31,278.64);
\definecolor[named]{fillColor}{rgb}{0.49,0.68,0.00}

\path[fill=fillColor] (426.31,230.62) rectangle (432.37,300.36);
\definecolor[named]{fillColor}{rgb}{0.00,0.75,0.77}

\path[fill=fillColor] (432.37,230.62) rectangle (438.43,271.65);
\definecolor[named]{fillColor}{rgb}{0.78,0.49,1.00}

\path[fill=fillColor] (438.43,230.62) rectangle (444.49,270.07);
\definecolor[named]{fillColor}{rgb}{0.97,0.46,0.43}

\path[fill=fillColor] (447.19,230.62) rectangle (453.25,279.64);
\definecolor[named]{fillColor}{rgb}{0.49,0.68,0.00}

\path[fill=fillColor] (453.25,230.62) rectangle (459.31,300.73);
\definecolor[named]{fillColor}{rgb}{0.00,0.75,0.77}

\path[fill=fillColor] (459.31,230.62) rectangle (465.37,272.29);
\definecolor[named]{fillColor}{rgb}{0.78,0.49,1.00}

\path[fill=fillColor] (465.37,230.62) rectangle (471.43,270.49);
\definecolor[named]{drawColor}{rgb}{1.00,0.00,0.00}
\definecolor[named]{fillColor}{rgb}{1.00,0.00,0.00}

\path[draw=drawColor,line width= 0.6pt,dash pattern=on 4pt off 4pt ,line join=round,fill=fillColor] (389.28,250.18) -- (475.47,250.18);
\definecolor[named]{drawColor}{rgb}{0.00,0.00,1.00}
\definecolor[named]{fillColor}{rgb}{0.00,0.00,1.00}

\path[draw=drawColor,line width= 0.6pt,dash pattern=on 4pt off 4pt ,line join=round,fill=fillColor] (389.28,269.74) -- (475.47,269.74);
\end{scope}
\begin{scope}
\path[clip] (389.28,142.77) rectangle (475.47,223.92);
\definecolor[named]{drawColor}{rgb}{0.75,0.75,0.75}

\path[draw=drawColor,line width= 1.5pt,line join=round,line cap=round] (389.28,142.77) rectangle (475.47,223.92);
\definecolor[named]{drawColor}{rgb}{0.90,0.90,0.90}

\path[draw=drawColor,line width= 0.3pt,line join=round] (389.28,154.09) --
	(475.47,154.09);

\path[draw=drawColor,line width= 0.3pt,line join=round] (389.28,169.35) --
	(475.47,169.35);

\path[draw=drawColor,line width= 0.3pt,line join=round] (389.28,184.61) --
	(475.47,184.61);

\path[draw=drawColor,line width= 0.3pt,line join=round] (389.28,199.88) --
	(475.47,199.88);

\path[draw=drawColor,line width= 0.3pt,line join=round] (389.28,215.14) --
	(475.47,215.14);

\path[draw=drawColor,line width= 0.8pt,line join=round] (389.28,146.45) --
	(475.47,146.45);

\path[draw=drawColor,line width= 0.8pt,line join=round] (389.28,161.72) --
	(475.47,161.72);

\path[draw=drawColor,line width= 0.8pt,line join=round] (389.28,176.98) --
	(475.47,176.98);

\path[draw=drawColor,line width= 0.8pt,line join=round] (389.28,192.24) --
	(475.47,192.24);

\path[draw=drawColor,line width= 0.8pt,line join=round] (389.28,207.51) --
	(475.47,207.51);

\path[draw=drawColor,line width= 0.8pt,line join=round] (389.28,222.77) --
	(475.47,222.77);

\path[draw=drawColor,line width= 0.8pt,line join=round] (405.44,142.77) --
	(405.44,223.92);

\path[draw=drawColor,line width= 0.8pt,line join=round] (432.37,142.77) --
	(432.37,223.92);

\path[draw=drawColor,line width= 0.8pt,line join=round] (459.31,142.77) --
	(459.31,223.92);
\definecolor[named]{fillColor}{rgb}{0.97,0.46,0.43}

\path[fill=fillColor] (393.32,146.45) rectangle (399.38,164.78);
\definecolor[named]{fillColor}{rgb}{0.49,0.68,0.00}

\path[fill=fillColor] (399.38,146.45) rectangle (405.44,163.34);
\definecolor[named]{fillColor}{rgb}{0.00,0.75,0.77}

\path[fill=fillColor] (405.44,146.45) rectangle (411.50,162.66);
\definecolor[named]{fillColor}{rgb}{0.78,0.49,1.00}

\path[fill=fillColor] (411.50,146.45) rectangle (417.56,161.67);
\definecolor[named]{fillColor}{rgb}{0.97,0.46,0.43}

\path[fill=fillColor] (420.25,146.45) rectangle (426.31,177.78);
\definecolor[named]{fillColor}{rgb}{0.49,0.68,0.00}

\path[fill=fillColor] (426.31,146.45) rectangle (432.37,207.42);
\definecolor[named]{fillColor}{rgb}{0.00,0.75,0.77}

\path[fill=fillColor] (432.37,146.45) rectangle (438.43,175.06);
\definecolor[named]{fillColor}{rgb}{0.78,0.49,1.00}

\path[fill=fillColor] (438.43,146.45) rectangle (444.49,173.99);
\definecolor[named]{fillColor}{rgb}{0.97,0.46,0.43}

\path[fill=fillColor] (447.19,146.45) rectangle (453.25,180.97);
\definecolor[named]{fillColor}{rgb}{0.49,0.68,0.00}

\path[fill=fillColor] (453.25,146.45) rectangle (459.31,220.23);
\definecolor[named]{fillColor}{rgb}{0.00,0.75,0.77}

\path[fill=fillColor] (459.31,146.45) rectangle (465.37,176.63);
\definecolor[named]{fillColor}{rgb}{0.78,0.49,1.00}

\path[fill=fillColor] (465.37,146.45) rectangle (471.43,175.55);
\definecolor[named]{drawColor}{rgb}{1.00,0.00,0.00}
\definecolor[named]{fillColor}{rgb}{1.00,0.00,0.00}

\path[draw=drawColor,line width= 0.6pt,dash pattern=on 4pt off 4pt ,line join=round,fill=fillColor] (389.28,161.72) -- (475.47,161.72);
\definecolor[named]{drawColor}{rgb}{0.00,0.00,1.00}
\definecolor[named]{fillColor}{rgb}{0.00,0.00,1.00}

\path[draw=drawColor,line width= 0.6pt,dash pattern=on 4pt off 4pt ,line join=round,fill=fillColor] (389.28,176.98) -- (475.47,176.98);
\end{scope}
\begin{scope}
\path[clip] (389.28, 58.60) rectangle (475.47,139.75);
\definecolor[named]{drawColor}{rgb}{0.75,0.75,0.75}

\path[draw=drawColor,line width= 1.5pt,line join=round,line cap=round] (389.28, 58.60) rectangle (475.47,139.75);
\definecolor[named]{drawColor}{rgb}{0.90,0.90,0.90}

\path[draw=drawColor,line width= 0.3pt,line join=round] (389.28, 71.51) --
	(475.47, 71.51);

\path[draw=drawColor,line width= 0.3pt,line join=round] (389.28, 89.96) --
	(475.47, 89.96);

\path[draw=drawColor,line width= 0.3pt,line join=round] (389.28,108.40) --
	(475.47,108.40);

\path[draw=drawColor,line width= 0.3pt,line join=round] (389.28,126.84) --
	(475.47,126.84);

\path[draw=drawColor,line width= 0.8pt,line join=round] (389.28, 62.29) --
	(475.47, 62.29);

\path[draw=drawColor,line width= 0.8pt,line join=round] (389.28, 80.73) --
	(475.47, 80.73);

\path[draw=drawColor,line width= 0.8pt,line join=round] (389.28, 99.18) --
	(475.47, 99.18);

\path[draw=drawColor,line width= 0.8pt,line join=round] (389.28,117.62) --
	(475.47,117.62);

\path[draw=drawColor,line width= 0.8pt,line join=round] (389.28,136.07) --
	(475.47,136.07);

\path[draw=drawColor,line width= 0.8pt,line join=round] (405.44, 58.60) --
	(405.44,139.75);

\path[draw=drawColor,line width= 0.8pt,line join=round] (432.37, 58.60) --
	(432.37,139.75);

\path[draw=drawColor,line width= 0.8pt,line join=round] (459.31, 58.60) --
	(459.31,139.75);
\definecolor[named]{fillColor}{rgb}{0.97,0.46,0.43}

\path[fill=fillColor] (393.32, 62.29) rectangle (399.38,100.94);
\definecolor[named]{fillColor}{rgb}{0.49,0.68,0.00}

\path[fill=fillColor] (399.38, 62.29) rectangle (405.44,100.89);
\definecolor[named]{fillColor}{rgb}{0.00,0.75,0.77}

\path[fill=fillColor] (405.44, 62.29) rectangle (411.50,108.96);
\definecolor[named]{fillColor}{rgb}{0.78,0.49,1.00}

\path[fill=fillColor] (411.50, 62.29) rectangle (417.56, 99.55);
\definecolor[named]{fillColor}{rgb}{0.97,0.46,0.43}

\path[fill=fillColor] (420.25, 62.29) rectangle (426.31,101.94);
\definecolor[named]{fillColor}{rgb}{0.49,0.68,0.00}

\path[fill=fillColor] (426.31, 62.29) rectangle (432.37,105.30);
\definecolor[named]{fillColor}{rgb}{0.00,0.75,0.77}

\path[fill=fillColor] (432.37, 62.29) rectangle (438.43,101.33);
\definecolor[named]{fillColor}{rgb}{0.78,0.49,1.00}

\path[fill=fillColor] (438.43, 62.29) rectangle (444.49,100.64);
\definecolor[named]{fillColor}{rgb}{0.97,0.46,0.43}

\path[fill=fillColor] (447.19, 62.29) rectangle (453.25,103.18);
\definecolor[named]{fillColor}{rgb}{0.49,0.68,0.00}

\path[fill=fillColor] (453.25, 62.29) rectangle (459.31,107.14);
\definecolor[named]{fillColor}{rgb}{0.00,0.75,0.77}

\path[fill=fillColor] (459.31, 62.29) rectangle (465.37,100.65);
\definecolor[named]{fillColor}{rgb}{0.78,0.49,1.00}

\path[fill=fillColor] (465.37, 62.29) rectangle (471.43,101.91);
\definecolor[named]{drawColor}{rgb}{1.00,0.00,0.00}
\definecolor[named]{fillColor}{rgb}{1.00,0.00,0.00}

\path[draw=drawColor,line width= 0.6pt,dash pattern=on 4pt off 4pt ,line join=round,fill=fillColor] (389.28, 99.18) -- (475.47, 99.18);
\definecolor[named]{drawColor}{rgb}{0.00,0.00,1.00}
\definecolor[named]{fillColor}{rgb}{0.00,0.00,1.00}

\path[draw=drawColor,line width= 0.6pt,dash pattern=on 4pt off 4pt ,line join=round,fill=fillColor] (389.28,136.07) -- (475.47,136.07);
\end{scope}
\begin{scope}
\path[clip] (  0.00,  0.00) rectangle (505.89,578.16);
\definecolor[named]{drawColor}{rgb}{0.00,0.00,0.00}

\node[text=drawColor,anchor=base east,inner sep=0pt, outer sep=0pt, scale=  0.88] at ( 28.06,480.09) {0};

\node[text=drawColor,anchor=base east,inner sep=0pt, outer sep=0pt, scale=  0.88] at ( 28.06,494.59) {1};

\node[text=drawColor,anchor=base east,inner sep=0pt, outer sep=0pt, scale=  0.88] at ( 28.06,509.09) {2};

\node[text=drawColor,anchor=base east,inner sep=0pt, outer sep=0pt, scale=  0.88] at ( 28.06,523.59) {3};

\node[text=drawColor,anchor=base east,inner sep=0pt, outer sep=0pt, scale=  0.88] at ( 28.06,538.09) {4};

\node[text=drawColor,anchor=base east,inner sep=0pt, outer sep=0pt, scale=  0.88] at ( 28.06,552.59) {5};
\end{scope}
\begin{scope}
\path[clip] (  0.00,  0.00) rectangle (505.89,578.16);
\definecolor[named]{drawColor}{rgb}{0.00,0.00,0.00}

\path[draw=drawColor,line width= 0.6pt,line join=round] ( 31.06,483.12) --
	( 32.48,483.12);

\path[draw=drawColor,line width= 0.6pt,line join=round] ( 31.06,497.62) --
	( 32.48,497.62);

\path[draw=drawColor,line width= 0.6pt,line join=round] ( 31.06,512.12) --
	( 32.48,512.12);

\path[draw=drawColor,line width= 0.6pt,line join=round] ( 31.06,526.62) --
	( 32.48,526.62);

\path[draw=drawColor,line width= 0.6pt,line join=round] ( 31.06,541.12) --
	( 32.48,541.12);

\path[draw=drawColor,line width= 0.6pt,line join=round] ( 31.06,555.62) --
	( 32.48,555.62);
\end{scope}
\begin{scope}
\path[clip] (  0.00,  0.00) rectangle (505.89,578.16);
\definecolor[named]{drawColor}{rgb}{0.00,0.00,0.00}

\node[text=drawColor,anchor=base east,inner sep=0pt, outer sep=0pt, scale=  0.88] at ( 28.06,395.92) {0};

\node[text=drawColor,anchor=base east,inner sep=0pt, outer sep=0pt, scale=  0.88] at ( 28.06,413.85) {1};

\node[text=drawColor,anchor=base east,inner sep=0pt, outer sep=0pt, scale=  0.88] at ( 28.06,431.77) {2};

\node[text=drawColor,anchor=base east,inner sep=0pt, outer sep=0pt, scale=  0.88] at ( 28.06,449.70) {3};

\node[text=drawColor,anchor=base east,inner sep=0pt, outer sep=0pt, scale=  0.88] at ( 28.06,467.62) {4};
\end{scope}
\begin{scope}
\path[clip] (  0.00,  0.00) rectangle (505.89,578.16);
\definecolor[named]{drawColor}{rgb}{0.00,0.00,0.00}

\path[draw=drawColor,line width= 0.6pt,line join=round] ( 31.06,398.95) --
	( 32.48,398.95);

\path[draw=drawColor,line width= 0.6pt,line join=round] ( 31.06,416.88) --
	( 32.48,416.88);

\path[draw=drawColor,line width= 0.6pt,line join=round] ( 31.06,434.80) --
	( 32.48,434.80);

\path[draw=drawColor,line width= 0.6pt,line join=round] ( 31.06,452.73) --
	( 32.48,452.73);

\path[draw=drawColor,line width= 0.6pt,line join=round] ( 31.06,470.65) --
	( 32.48,470.65);
\end{scope}
\begin{scope}
\path[clip] (  0.00,  0.00) rectangle (505.89,578.16);
\definecolor[named]{drawColor}{rgb}{0.00,0.00,0.00}

\node[text=drawColor,anchor=base east,inner sep=0pt, outer sep=0pt, scale=  0.88] at ( 28.06,311.76) {0};

\node[text=drawColor,anchor=base east,inner sep=0pt, outer sep=0pt, scale=  0.88] at ( 28.06,329.39) {1};

\node[text=drawColor,anchor=base east,inner sep=0pt, outer sep=0pt, scale=  0.88] at ( 28.06,347.03) {2};

\node[text=drawColor,anchor=base east,inner sep=0pt, outer sep=0pt, scale=  0.88] at ( 28.06,364.66) {3};

\node[text=drawColor,anchor=base east,inner sep=0pt, outer sep=0pt, scale=  0.88] at ( 28.06,382.30) {4};
\end{scope}
\begin{scope}
\path[clip] (  0.00,  0.00) rectangle (505.89,578.16);
\definecolor[named]{drawColor}{rgb}{0.00,0.00,0.00}

\path[draw=drawColor,line width= 0.6pt,line join=round] ( 31.06,314.79) --
	( 32.48,314.79);

\path[draw=drawColor,line width= 0.6pt,line join=round] ( 31.06,332.42) --
	( 32.48,332.42);

\path[draw=drawColor,line width= 0.6pt,line join=round] ( 31.06,350.06) --
	( 32.48,350.06);

\path[draw=drawColor,line width= 0.6pt,line join=round] ( 31.06,367.69) --
	( 32.48,367.69);

\path[draw=drawColor,line width= 0.6pt,line join=round] ( 31.06,385.33) --
	( 32.48,385.33);
\end{scope}
\begin{scope}
\path[clip] (  0.00,  0.00) rectangle (505.89,578.16);
\definecolor[named]{drawColor}{rgb}{0.00,0.00,0.00}

\node[text=drawColor,anchor=base east,inner sep=0pt, outer sep=0pt, scale=  0.88] at ( 28.06,227.59) {0};

\node[text=drawColor,anchor=base east,inner sep=0pt, outer sep=0pt, scale=  0.88] at ( 28.06,247.15) {1};

\node[text=drawColor,anchor=base east,inner sep=0pt, outer sep=0pt, scale=  0.88] at ( 28.06,266.71) {2};

\node[text=drawColor,anchor=base east,inner sep=0pt, outer sep=0pt, scale=  0.88] at ( 28.06,286.27) {3};
\end{scope}
\begin{scope}
\path[clip] (  0.00,  0.00) rectangle (505.89,578.16);
\definecolor[named]{drawColor}{rgb}{0.00,0.00,0.00}

\path[draw=drawColor,line width= 0.6pt,line join=round] ( 31.06,230.62) --
	( 32.48,230.62);

\path[draw=drawColor,line width= 0.6pt,line join=round] ( 31.06,250.18) --
	( 32.48,250.18);

\path[draw=drawColor,line width= 0.6pt,line join=round] ( 31.06,269.74) --
	( 32.48,269.74);

\path[draw=drawColor,line width= 0.6pt,line join=round] ( 31.06,289.30) --
	( 32.48,289.30);
\end{scope}
\begin{scope}
\path[clip] (  0.00,  0.00) rectangle (505.89,578.16);
\definecolor[named]{drawColor}{rgb}{0.00,0.00,0.00}

\node[text=drawColor,anchor=base east,inner sep=0pt, outer sep=0pt, scale=  0.88] at ( 28.06,143.42) {0};

\node[text=drawColor,anchor=base east,inner sep=0pt, outer sep=0pt, scale=  0.88] at ( 28.06,158.69) {1};

\node[text=drawColor,anchor=base east,inner sep=0pt, outer sep=0pt, scale=  0.88] at ( 28.06,173.95) {2};

\node[text=drawColor,anchor=base east,inner sep=0pt, outer sep=0pt, scale=  0.88] at ( 28.06,189.21) {3};

\node[text=drawColor,anchor=base east,inner sep=0pt, outer sep=0pt, scale=  0.88] at ( 28.06,204.48) {4};

\node[text=drawColor,anchor=base east,inner sep=0pt, outer sep=0pt, scale=  0.88] at ( 28.06,219.74) {5};
\end{scope}
\begin{scope}
\path[clip] (  0.00,  0.00) rectangle (505.89,578.16);
\definecolor[named]{drawColor}{rgb}{0.00,0.00,0.00}

\path[draw=drawColor,line width= 0.6pt,line join=round] ( 31.06,146.45) --
	( 32.48,146.45);

\path[draw=drawColor,line width= 0.6pt,line join=round] ( 31.06,161.72) --
	( 32.48,161.72);

\path[draw=drawColor,line width= 0.6pt,line join=round] ( 31.06,176.98) --
	( 32.48,176.98);

\path[draw=drawColor,line width= 0.6pt,line join=round] ( 31.06,192.24) --
	( 32.48,192.24);

\path[draw=drawColor,line width= 0.6pt,line join=round] ( 31.06,207.51) --
	( 32.48,207.51);

\path[draw=drawColor,line width= 0.6pt,line join=round] ( 31.06,222.77) --
	( 32.48,222.77);
\end{scope}
\begin{scope}
\path[clip] (  0.00,  0.00) rectangle (505.89,578.16);
\definecolor[named]{drawColor}{rgb}{0.00,0.00,0.00}

\node[text=drawColor,anchor=base east,inner sep=0pt, outer sep=0pt, scale=  0.88] at ( 28.06, 59.26) {0.0};

\node[text=drawColor,anchor=base east,inner sep=0pt, outer sep=0pt, scale=  0.88] at ( 28.06, 77.70) {0.5};

\node[text=drawColor,anchor=base east,inner sep=0pt, outer sep=0pt, scale=  0.88] at ( 28.06, 96.15) {1.0};

\node[text=drawColor,anchor=base east,inner sep=0pt, outer sep=0pt, scale=  0.88] at ( 28.06,114.59) {1.5};

\node[text=drawColor,anchor=base east,inner sep=0pt, outer sep=0pt, scale=  0.88] at ( 28.06,133.04) {2.0};
\end{scope}
\begin{scope}
\path[clip] (  0.00,  0.00) rectangle (505.89,578.16);
\definecolor[named]{drawColor}{rgb}{0.00,0.00,0.00}

\path[draw=drawColor,line width= 0.6pt,line join=round] ( 31.06, 62.29) --
	( 32.48, 62.29);

\path[draw=drawColor,line width= 0.6pt,line join=round] ( 31.06, 80.73) --
	( 32.48, 80.73);

\path[draw=drawColor,line width= 0.6pt,line join=round] ( 31.06, 99.18) --
	( 32.48, 99.18);

\path[draw=drawColor,line width= 0.6pt,line join=round] ( 31.06,117.62) --
	( 32.48,117.62);

\path[draw=drawColor,line width= 0.6pt,line join=round] ( 31.06,136.07) --
	( 32.48,136.07);
\end{scope}
\begin{scope}
\path[clip] (475.47,479.43) rectangle (503.04,560.58);
\definecolor[named]{drawColor}{rgb}{0.90,0.90,0.90}
\definecolor[named]{fillColor}{rgb}{0.90,0.90,0.90}

\path[draw=drawColor,line width= 0.6pt,line join=round,line cap=round,fill=fillColor] (475.47,479.43) rectangle (503.04,560.58);
\definecolor[named]{drawColor}{rgb}{0.00,0.00,0.00}

\node[text=drawColor,rotate=-90.00,anchor=base,inner sep=0pt, outer sep=0pt, scale=  1.10] at (485.47,520.01) {\DAC};
\end{scope}
\begin{scope}
\path[clip] (475.47,395.26) rectangle (503.04,476.42);
\definecolor[named]{drawColor}{rgb}{0.90,0.90,0.90}
\definecolor[named]{fillColor}{rgb}{0.90,0.90,0.90}

\path[draw=drawColor,line width= 0.6pt,line join=round,line cap=round,fill=fillColor] (475.47,395.26) rectangle (503.04,476.42);
\definecolor[named]{drawColor}{rgb}{0.00,0.00,0.00}

\node[text=drawColor,rotate=-90.00,anchor=base,inner sep=0pt, outer sep=0pt, scale=  1.10] at (485.47,435.84) {\ISPD};
\end{scope}
\begin{scope}
\path[clip] (475.47,311.10) rectangle (503.04,392.25);
\definecolor[named]{drawColor}{rgb}{0.90,0.90,0.90}
\definecolor[named]{fillColor}{rgb}{0.90,0.90,0.90}

\path[draw=drawColor,line width= 0.6pt,line join=round,line cap=round,fill=fillColor] (475.47,311.10) rectangle (503.04,392.25);
\definecolor[named]{drawColor}{rgb}{0.00,0.00,0.00}

\node[text=drawColor,rotate=-90.00,anchor=base,inner sep=0pt, outer sep=0pt, scale=  1.10] at (485.47,351.67) {\Dual};
\end{scope}
\begin{scope}
\path[clip] (475.47,226.93) rectangle (503.04,308.09);
\definecolor[named]{drawColor}{rgb}{0.90,0.90,0.90}
\definecolor[named]{fillColor}{rgb}{0.90,0.90,0.90}

\path[draw=drawColor,line width= 0.6pt,line join=round,line cap=round,fill=fillColor] (475.47,226.93) rectangle (503.04,308.09);
\definecolor[named]{drawColor}{rgb}{0.00,0.00,0.00}

\node[text=drawColor,rotate=-90.00,anchor=base,inner sep=0pt, outer sep=0pt, scale=  1.10] at (485.47,267.51) {\Primal};
\end{scope}
\begin{scope}
\path[clip] (475.47,142.77) rectangle (503.04,223.92);
\definecolor[named]{drawColor}{rgb}{0.90,0.90,0.90}
\definecolor[named]{fillColor}{rgb}{0.90,0.90,0.90}

\path[draw=drawColor,line width= 0.6pt,line join=round,line cap=round,fill=fillColor] (475.47,142.77) rectangle (503.04,223.92);
\definecolor[named]{drawColor}{rgb}{0.00,0.00,0.00}

\node[text=drawColor,rotate=-90.00,anchor=base,inner sep=0pt, outer sep=0pt, scale=  1.10] at (485.47,183.34) {\Literal};
\end{scope}
\begin{scope}
\path[clip] (475.47, 58.60) rectangle (503.04,139.75);
\definecolor[named]{drawColor}{rgb}{0.90,0.90,0.90}
\definecolor[named]{fillColor}{rgb}{0.90,0.90,0.90}

\path[draw=drawColor,line width= 0.6pt,line join=round,line cap=round,fill=fillColor] (475.47, 58.60) rectangle (503.04,139.75);
\definecolor[named]{drawColor}{rgb}{0.00,0.00,0.00}

\node[text=drawColor,rotate=-90.00,anchor=base,inner sep=0pt, outer sep=0pt, scale=  1.10] at (485.47, 99.18) {\SPM};
\end{scope}
\begin{scope}
\path[clip] (  0.00,  0.00) rectangle (505.89,578.16);
\definecolor[named]{drawColor}{rgb}{0.00,0.00,0.00}

\path[draw=drawColor,line width= 0.6pt,line join=round] ( 48.64, 57.18) --
	( 48.64, 58.60);

\path[draw=drawColor,line width= 0.6pt,line join=round] ( 75.58, 57.18) --
	( 75.58, 58.60);

\path[draw=drawColor,line width= 0.6pt,line join=round] (102.51, 57.18) --
	(102.51, 58.60);
\end{scope}
\begin{scope}
\path[clip] (  0.00,  0.00) rectangle (505.89,578.16);
\definecolor[named]{drawColor}{rgb}{0.00,0.00,0.00}

\node[text=drawColor,anchor=base,inner sep=0pt, outer sep=0pt, scale=  0.88] at ( 48.64, 48.12) {$\ExpNodeDegree$};

\node[text=drawColor,anchor=base,inner sep=0pt, outer sep=0pt, scale=  0.88] at ( 75.58, 48.12) {$\ExpEdgeSize$};

\node[text=drawColor,anchor=base,inner sep=0pt, outer sep=0pt, scale=  0.88] at (102.51, 48.12) {$\ExpHybrid$};
\end{scope}
\begin{scope}
\path[clip] (  0.00,  0.00) rectangle (505.89,578.16);
\definecolor[named]{drawColor}{rgb}{0.00,0.00,0.00}

\path[draw=drawColor,line width= 0.6pt,line join=round] (137.84, 57.18) --
	(137.84, 58.60);

\path[draw=drawColor,line width= 0.6pt,line join=round] (164.78, 57.18) --
	(164.78, 58.60);

\path[draw=drawColor,line width= 0.6pt,line join=round] (191.71, 57.18) --
	(191.71, 58.60);
\end{scope}
\begin{scope}
\path[clip] (  0.00,  0.00) rectangle (505.89,578.16);
\definecolor[named]{drawColor}{rgb}{0.00,0.00,0.00}

\node[text=drawColor,anchor=base,inner sep=0pt, outer sep=0pt, scale=  0.88] at (137.84, 48.12) {$\ExpNodeDegree$};

\node[text=drawColor,anchor=base,inner sep=0pt, outer sep=0pt, scale=  0.88] at (164.78, 48.12) {$\ExpEdgeSize$};

\node[text=drawColor,anchor=base,inner sep=0pt, outer sep=0pt, scale=  0.88] at (191.71, 48.12) {$\ExpHybrid$};
\end{scope}
\begin{scope}
\path[clip] (  0.00,  0.00) rectangle (505.89,578.16);
\definecolor[named]{drawColor}{rgb}{0.00,0.00,0.00}

\path[draw=drawColor,line width= 0.6pt,line join=round] (227.04, 57.18) --
	(227.04, 58.60);

\path[draw=drawColor,line width= 0.6pt,line join=round] (253.98, 57.18) --
	(253.98, 58.60);

\path[draw=drawColor,line width= 0.6pt,line join=round] (280.91, 57.18) --
	(280.91, 58.60);
\end{scope}
\begin{scope}
\path[clip] (  0.00,  0.00) rectangle (505.89,578.16);
\definecolor[named]{drawColor}{rgb}{0.00,0.00,0.00}

\node[text=drawColor,anchor=base,inner sep=0pt, outer sep=0pt, scale=  0.88] at (227.04, 48.12) {$\ExpNodeDegree$};

\node[text=drawColor,anchor=base,inner sep=0pt, outer sep=0pt, scale=  0.88] at (253.98, 48.12) {$\ExpEdgeSize$};

\node[text=drawColor,anchor=base,inner sep=0pt, outer sep=0pt, scale=  0.88] at (280.91, 48.12) {$\ExpHybrid$};
\end{scope}
\begin{scope}
\path[clip] (  0.00,  0.00) rectangle (505.89,578.16);
\definecolor[named]{drawColor}{rgb}{0.00,0.00,0.00}

\path[draw=drawColor,line width= 0.6pt,line join=round] (316.24, 57.18) --
	(316.24, 58.60);

\path[draw=drawColor,line width= 0.6pt,line join=round] (343.18, 57.18) --
	(343.18, 58.60);

\path[draw=drawColor,line width= 0.6pt,line join=round] (370.11, 57.18) --
	(370.11, 58.60);
\end{scope}
\begin{scope}
\path[clip] (  0.00,  0.00) rectangle (505.89,578.16);
\definecolor[named]{drawColor}{rgb}{0.00,0.00,0.00}

\node[text=drawColor,anchor=base,inner sep=0pt, outer sep=0pt, scale=  0.88] at (316.24, 48.12) {$\ExpNodeDegree$};

\node[text=drawColor,anchor=base,inner sep=0pt, outer sep=0pt, scale=  0.88] at (343.18, 48.12) {$\ExpEdgeSize$};

\node[text=drawColor,anchor=base,inner sep=0pt, outer sep=0pt, scale=  0.88] at (370.11, 48.12) {$\ExpHybrid$};
\end{scope}
\begin{scope}
\path[clip] (  0.00,  0.00) rectangle (505.89,578.16);
\definecolor[named]{drawColor}{rgb}{0.00,0.00,0.00}

\path[draw=drawColor,line width= 0.6pt,line join=round] (405.44, 57.18) --
	(405.44, 58.60);

\path[draw=drawColor,line width= 0.6pt,line join=round] (432.37, 57.18) --
	(432.37, 58.60);

\path[draw=drawColor,line width= 0.6pt,line join=round] (459.31, 57.18) --
	(459.31, 58.60);
\end{scope}
\begin{scope}
\path[clip] (  0.00,  0.00) rectangle (505.89,578.16);
\definecolor[named]{drawColor}{rgb}{0.00,0.00,0.00}

\node[text=drawColor,anchor=base,inner sep=0pt, outer sep=0pt, scale=  0.88] at (405.44, 48.12) {$\ExpNodeDegree$};

\node[text=drawColor,anchor=base,inner sep=0pt, outer sep=0pt, scale=  0.88] at (432.37, 48.12) {$\ExpEdgeSize$};

\node[text=drawColor,anchor=base,inner sep=0pt, outer sep=0pt, scale=  0.88] at (459.31, 48.12) {$\ExpHybrid$};
\end{scope}
\begin{scope}
\path[clip] (  0.00,  0.00) rectangle (505.89,578.16);
\definecolor[named]{drawColor}{rgb}{0.00,0.00,0.00}

\node[text=drawColor,anchor=base,inner sep=0pt, outer sep=0pt, scale=  0.99] at (253.98, 31.30) {Flow Network};
\end{scope}
\begin{scope}
\path[clip] (  0.00,  0.00) rectangle (505.89,578.16);
\definecolor[named]{drawColor}{rgb}{0.00,0.00,0.00}

\node[text=drawColor,rotate= 90.00,anchor=base,inner sep=0pt, outer sep=0pt, scale=  0.99] at (  8.5,309.59) {Speed up relative to $\ExpLawler$};
\end{scope}
\begin{scope}
\path[clip] (  0.00,  0.00) rectangle (505.89,578.16);
\definecolor[named]{drawColor}{rgb}{1.00,1.00,1.00}
\definecolor[named]{fillColor}{rgb}{0.95,0.95,0.95}

\path[draw=drawColor,line width= 0.6pt,line join=round,line cap=round,fill=fillColor] (88.15, 12.80) rectangle (102.38, 27.03);
\end{scope}
\begin{scope}
\path[clip] (  0.00,  0.00) rectangle (505.89,578.16);
\definecolor[named]{fillColor}{rgb}{0.97,0.46,0.43}

\path[fill=fillColor] (88.86, 13.52) rectangle (101.67, 26.32);
\end{scope}
\begin{scope}
\path[clip] (  0.00,  0.00) rectangle (505.89,578.16);
\definecolor[named]{drawColor}{rgb}{1.00,1.00,1.00}
\definecolor[named]{fillColor}{rgb}{0.95,0.95,0.95}

\path[draw=drawColor,line width= 0.6pt,line join=round,line cap=round,fill=fillColor] (170.45, 12.80) rectangle (184.67, 27.03);
\end{scope}
\begin{scope}
\path[clip] (  0.00,  0.00) rectangle (505.89,578.16);
\definecolor[named]{fillColor}{rgb}{0.49,0.68,0.00}

\path[fill=fillColor] (171.16, 13.52) rectangle (183.96, 26.32);
\end{scope}
\begin{scope}
\path[clip] (  0.00,  0.00) rectangle (505.89,578.16);
\definecolor[named]{drawColor}{rgb}{1.00,1.00,1.00}
\definecolor[named]{fillColor}{rgb}{0.95,0.95,0.95}

\path[draw=drawColor,line width= 0.6pt,line join=round,line cap=round,fill=fillColor] (273.74, 12.80) rectangle (287.97, 27.03);
\end{scope}
\begin{scope}
\path[clip] (  0.00,  0.00) rectangle (505.89,578.16);
\definecolor[named]{fillColor}{rgb}{0.00,0.75,0.77}

\path[fill=fillColor] (274.45, 13.52) rectangle (287.26, 26.32);
\end{scope}
\begin{scope}
\path[clip] (  0.00,  0.00) rectangle (505.89,578.16);
\definecolor[named]{drawColor}{rgb}{1.00,1.00,1.00}
\definecolor[named]{fillColor}{rgb}{0.95,0.95,0.95}

\path[draw=drawColor,line width= 0.6pt,line join=round,line cap=round,fill=fillColor] (391.06, 12.80) rectangle (405.29, 27.03);
\end{scope}
\begin{scope}
\path[clip] (  0.00,  0.00) rectangle (505.89,578.16);
\definecolor[named]{fillColor}{rgb}{0.78,0.49,1.00}

\path[fill=fillColor] (391.77, 13.52) rectangle (404.57, 26.32);
\end{scope}
\begin{scope}
\path[clip] (  0.00,  0.00) rectangle (505.89,578.16);
\definecolor[named]{drawColor}{rgb}{0.00,0.00,0.00}

\node[text=drawColor,anchor=base west,inner sep=0pt, outer sep=0pt, scale=  0.82] at (104.18, 17.08) {\EdmondKarp};
\end{scope}
\begin{scope}
\path[clip] (  0.00,  0.00) rectangle (505.89,578.16);
\definecolor[named]{drawColor}{rgb}{0.00,0.00,0.00}

\node[text=drawColor,anchor=base west,inner sep=0pt, outer sep=0pt, scale=  0.82] at (186.48, 17.08) {\GoldbergTarjan};
\end{scope}
\begin{scope}
\path[clip] (  0.00,  0.00) rectangle (505.89,578.16);
\definecolor[named]{drawColor}{rgb}{0.00,0.00,0.00}

\node[text=drawColor,anchor=base west,inner sep=0pt, outer sep=0pt, scale=  0.82] at (289.77, 17.08) {\BoykovKolmogorov};
\end{scope}
\begin{scope}
\path[clip] (  0.00,  0.00) rectangle (505.89,578.16);
\definecolor[named]{drawColor}{rgb}{0.00,0.00,0.00}

\node[text=drawColor,anchor=base west,inner sep=0pt, outer sep=0pt, scale=  0.82] at (407.09, 17.08) {\IBFS};
\end{scope}
\end{tikzpicture}
 %
\caption{Speed up of our flow algorithms and networks relative to \EdmondKarp~on
         $\ExpLawler$ for different instance sizes and types. The red dashed line indicates the
         $($\EdmondKarp$,\ExpLawler)$ implementation and the blue dashed line
         indicates a speed up by a factor of $2$.}
\label{fig:max_flow_network_algo}
\end{figure} 
\autoref{tbl:flow_algo_network_summary}~shows the summary of our flow algorithm and network experiment
on all benchmark instances. This proofs our assumption that \EdmondKarp~works best on small instances
and \GoldbergTarjan~on large instances. However, our \emph{Max-Flow-Min-Cut} computations
are embedded in a \emph{Adapative Flow Iteration} strategy (see Section \ref{sec:adaptive_flow_iterations}).
Therefore, the running time of flow instances generated with a large $\alpha$ will dominate the
ones with small $\alpha$. Therefore, we choose \GoldbergTarjan~in combination with our flow network
$\ExpHybrid$ in the following experiments.
\begin{table}
\renewcommand{\arraystretch}{1.15}
\centering
\begin{tabular}{lr|*{4}{r@{\hspace{3mm}}}|*{4}{r@{\hspace{3mm}}}}
\toprule
 \multirow{2}{*}{\rotatebox{90}{\footnotesize{Instance}}} & \quad\quad & \multicolumn{4}{c|}{\GoldbergTarjan} & \multicolumn{4}{c}{\EdmondKarp} \\
\cmidrule{3-10}
 &  & $\ExpHybrid$ & $\ExpEdgeSize$ & $\ExpNodeDegree$ & $\ExpLawler$ & $\ExpHybrid$ & $\ExpEdgeSize$ & $\ExpNodeDegree$ & $\ExpLawler$ \\
 & $|V'|$ &  \tiny{$t[ms]$} & \tiny{$t[\%]$} & \tiny{$t[\%]$} & \tiny{$t[\%]$} & \tiny{$t[\%]$} & \tiny{$t[\%]$} & \tiny{$t[\%]$} & \tiny{$t[\%]$}
\\\midrule%
\csname @@input\endcsname experiments/flow_network/flow_network_max_flow_summary_table.tex 
\bottomrule
\end{tabular}
\caption{Running time comparison of maximum flow algorithms on different flow networks.
         Note, all values in the table are in percentage relative to \GoldbergTarjan~
         on flow network $\ExpHybrid$. In each line the fastest variant is marked bold.}
\label{tbl:flow_algo_network_summary}
\end{table}

\subsection{Configuration of the $k$-way Flow-based Refiner}
\label{sec:flow_configuration}

In this Section we examine the quality of our $k$-way flow-based refinement algorithm with
different configurations on our parameter tuning benchmark subset (\todo{ref to appendix}).
There are several configurations and tunning parameters which we have to evaluate:
\begin{itemize}
\item \emph{Max-(F)low-Min-Cut} computations as refinement algorithm (see Section \ref{sec:flow_local_search_hypergraph})
\item \emph{Adaptive Flow Iteration} parameter $\alpha'$ (see Section \ref{sec:adaptive_flow_iterations})
\item \emph{(C)ut Border Hyperedges} as sources and sinks (see Section \ref{sec:source_and_sink})
\item \emph{(M)ost Balanced Minimum Cut} heuristic (see Section \ref{sec:mbmc_hypergraphs})
\item Combining \emph{Max-(F)low-Min-Cut} computations with \emph{(FM)} refinement
\end{itemize}
In the following we will denote a configuration e.g. with \FlowVariant{+}{-}{-}{-} which indicates
which heuristic resp. technique is enabled $(+)$ or disabled $(-)$. The meaning of the 
abbreviations are explained in the enumeration above (see letters inside parenthesis). We evaluate
a configuration for $k \in \{2,4,8,16,32,64,128\}$, $\alpha' \in \{1,2,4,8,16\}$
and $10$ different seeds on our parameter tuning benchmark subset ($\epsilon = 3\%$). 
Our pairwise flow-based refinement is embedded in a $k$-way \emph{Active Block Scheduling}
refinement which is executed on each level $i$ with $i = 2^j$ ($j \in \mathbb{N}_+$) 
(see Section \ref{sec:flow_local_search_hypergraph}). As reference we use the 
latest quality configuration of \emph{KaHyPar} \cite{heuer2017improving}. \\
The results are summarized in \autoref{tbl:alpha_exp}. The values
in the column \emph{Gmean} are improvements of the connectivity metric
relative to our baseline configuration \FlowVariant{-}{-}{-}{+}. The running
time are absolute values in seconds. The first observation is that flows on
its own as refinement strategy are not strong enough to outperform the
\emph{FM} heuristic. Our strongest configuration with $\alpha' = 16$
is $2.5\%$ worse than our \emph{FM} baseline. But the result 
is still remarkable, because we only execute flows on $\log{n}$ levels
instead on $n$ as the \emph{FM} algorithm do. The running time of
scales nearly linear with parameter $\alpha'$. Using our improved source 
and sink modelling approach with \emph{Cut Border Hyperedges} (see Equation
\ref{S_final_border_hyperedges} and \ref{T_final_border_hyperedges})
significantly improves the solution quality especially for small $\alpha'$.
For small $\alpha$ most of the hypernodes are either a source or a sink. 
Introducing \emph{Cut Border Hyperedges} reduces the number of hypernode
sources and sinks by adding hyperedge sources and sinks. The quality 
improvement with this technique is therefore more effective
for small $\alpha'$, because it significantly increase the possibilities
of moving hypernodes between the blocks compared to the source and sink set
modelling approach with Equation \ref{S_border_hyperedges} and \ref{T_border_hyperedges}.
The opposite effect can be observed, if we use the \emph{Most Balanced Minimum Cut} heuristic
without \emph{Cut Border Hyperedges}. The quality improvement is more significant for large
$\alpha'$. The larger the flow problem, the higher is the number of different minimum 
$(S,T)$-cutsets and this increases the possibility to find a feasible solution according to 
our balanced constraint. If we combine both techniques, we obtain a configuration which significantly
improves the solution quality for all $\alpha'$ compared to our baseline flow configuration.
Also it outperforms our baseline \emph{FM} configuration for $\alpha' = 16$ by $0.51\%$.
If we enable \emph{FM} refinement in all levels where no flow is executed, we improve the solution
quality by nearly $2\%$ (for $\alpha' = 16$). Also the running time of this variant is faster
than all previous flow configurations, because we transfer more work to the \emph{FM} refinement.
This has as consequence that a block becomes faster \emph{inactive} during \emph{Active Block 
Scheduling} and this decreases the number of rounds of complete pairwise flow-based refinements
on the quotient graph.

\begin{table}
\renewcommand{\arraystretch}{1.15}
\centering
\begin{tabular}{|r||c|c||c|c||c|c|}
\toprule
 Config. & \multicolumn{2}{c||}{\FlowVariant{+}{-}{-}{-}} & \multicolumn{2}{c||}{\FlowVariant{+}{+}{-}{-}}  & \multicolumn{2}{c|}{\FlowVariant{+}{-}{+}{-}} \\
\midrule
$\alpha'$ & Gmean$[\%]$ & $t[s]$ & Gmean$[\%]$ & $t[s]$ & Gmean$[\%]$ & $t[s]$ \\
\midrule%
\csname @@input\endcsname experiments/flow_alpha/flow_alpha_table1.tex 
\cmidrule[1.25pt]{1-5}%
 Config. & \multicolumn{2}{c||}{\FlowVariant{+}{+}{+}{-}} & \multicolumn{2}{c||}{\FlowVariant{+}{+}{+}{+}} \\
\cmidrule{1-5}
$\alpha'$ & Gmean$[\%]$ & $t[s]$ & Gmean$[\%]$ & $t[s]$ \\
\cmidrule{1-5}%
\csname @@input\endcsname experiments/flow_alpha/flow_alpha_table2.tex 
\cmidrule{1-5}
\end{tabular}
\caption{ Table contains results for different configurations of our flow algorithm with
          increasing $\alpha'$. }
\label{tbl:alpha_exp}
\end{table}

\todo{evaluate effectiveness of flows}

\subsection{Speed-Up Heuristics}

At the end of Section \ref{sec:flow_local_search_hypergraph} we present several heuristics
to prevent unnecessary flow executions during \emph{Active Block Scheduling} ((R1)-(R3)).
The main assumption is that only a minority of \emph{Max-Flow-Min-Cut} computations
lead to an improvement on $H$. To prove that we execute \KaHyPar{MF} on our benchmark subset
(\todo{ref to appendix}) and enable one heuristic after another.\\
Table \ref{tbl:heuristics} summarizes the results of the experiment. \KaHyPar{CA} is the currently
best configuration of \emph{KaHyPar} and \KaHyPar{MF} is our baseline flow configuration of
Section \ref{sec:flow_configuration}. The index of the remaining variants of \KaHyPar{MF} 
describes which speed-up heuristics are enabled (see Section \ref{sec:flow_local_search_hypergraph}).
The values inside the table are \emph{geometric means} over the corresponding metric. On average,
enabling all speed up heuristics worsen the quality of \KaHyPar{MF} only by $0.07\%$. On the other
hand the \emph{Max-Flow-Min-Cut} computations are significantly faster by a factor of $\approx 2$.
In its final configuration \KaHyParConfig{MF}{R1,R2,R3} computes partitions with $\approx 2\%$ better
quality ($(\lambda - 1)$-metric) than \KaHyPar{CA} by a slowdown only of a factor of $\approx 2$.
In the following we will denote our final configuration \KaHyParConfig{MF}{R1,R2,R3} with
\KaHyPar{MF}.

\begin{table}
\renewcommand{\arraystretch}{1.15}
\centering
\begin{tabular}{l|cccc}
\toprule
Variant & Avg.$[\%]$ & Min.$[\%]$ & $t_{\text{flow}}[s]$ & $t[s]$ \\
\midrule%
\csname @@input\endcsname experiments/speed_up_heuristics/heuristic_table.tex 
\bottomrule
\end{tabular} 
\caption{Table shows results for our flow algorithm with different speed up heuristics.}
\label{tbl:heuristics}
\end{table}

\subsection{Comparison against other Hypergraph Partitioner}

\begin{enumerate}
\item Compare final configuration of flow refiner against sea config on the full benchmark set
\end{enumerate}





\newpage

%%%%%%%%%%%%%%%%%%%%%%%%%%%%%%%%%%%%%%%%%%%%%%%%%%%%%%%%%%%%%%%%%%%%%%

\section{Conclusion}
\label{sec:conclusion}

In this thesis, we present a novel refinement algorithm based on \emph{Max-Flow-Min-Cut}
computations for multilevel hypergraph partitioning. We integrate our framework
into the $n$-level hypergraph partitioner \emph{KaHyPar} and show that
in combination with the \emph{FM} algorithm our new approach produces the best partitions
for a wide range of applications.\\
We introduce several techniques to 
sparsify the standard hypergraph flow network \cite{lawler1973},
which consists of many edges with \emph{infinite} capacity. We present
several theoretical results, which allows us to remove such edges or even to remove
nodes. The results are of independent interest, because they are also
applicable on general flow networks with \emph{infinite} capacity edges.
The main practical implications are that we can remove any hypernode 
from the hypergraph flow network and model hyperedges of size $2$ more efficiently. 
Our final flow network combines the two techniques, which reduces 
the problem size of the resulting flow networks on various benchmark types by up to 
a factor of $2$ compared to the state-of-the-art representation and
simultanousely speed-up the running time of different maximum flow algorithms 
by the same amount.  \\
Our \emph{flow}-based refinement framework is based on ideas of Sanders and Schulz
\cite{sanders2011engineering} (developed for multilevel graph partitioning). However,
we generalize many results such that they are applicable to hypergraph partitioning.
We configure a flow problem on a subhypergraph in such a way that a \emph{Max-Flow-Min-Cut}
computation improves a given bipartition of a hypergraph. Further, we show theoretically
and practically that with our modeling approach better minimum cuts are achievable
compared to the results of Sanders and Schulz. The bipartitioning algorithm is transferred
to the direct $k$-way partitioning case by executing pairwise \emph{flow}-based refinements
on two adjacent blocks according to the \emph{active block scheduling} strategy \cite{holtgrewe2010engineering}.
We integrate the framework into the $n$-level hypergraph partitioner \emph{KaHyPar}.
The total number of \emph{flow}-based refinements throughout the multilevel hierarchy
is controlled by a \emph{flow execution policy}. In each level, where no \emph{flow}
is execeuted, the \emph{FM} algorithm is used to improve the partition. Additionally,
we develop several heuristics to prevent \emph{unpromising} \emph{Max-Flow-Min-Cut}
computations on two adjacent blocks, which speed-up our framework
by a factor of $2$. \\
The new configuration \KaHyPar{MF} produces on $97\%$ of our benchmark instances
better partitions than our old configuration \KaHyPar{CA}. On average the solution 
quality is $2.5\%$ better, while only incurring a slowdown by a factor of $2$. 
In comparison with $5$ different state-of-the-art hypergraph partitioners, \KaHyPar{MF} 
produces on $73\%$ of $3216$ benchmark instances the best partitions. 
Moreover, our partitioner has a comparable running time to the direct $k$-way
version of \emph{hMetis} and outperforms it on $84\%$ of the benchmark instances.


\subsection{Future Work}

The quality of our framework mainly depends on the number of flow executions
throughout the multilevel hierarchy and the running time of our maximum flow algorithms.
Optimizing those two basic building blocks of the framework will allow us to achieve 
better quality in the same amount of time.\\
The hypergraph flow network proposed by Lawler \cite{lawler1973} has a bipartite 
structure. Because of this structural regularity, there might be other more specialized
flow algorithms which run faster on these types of networks. Further,
one could investigate if it is possible to maintain the whole flow network over the
multilevel hierarchy instead of explicitly creating the flow network before each flow
execution. Also, it would be interesting if information from previous flow calculations can 
be used to speed-up the current flow calculation. Currently, it is possible to add edges and nodes to a flow network and speed-up
the flow computation by using informations of previous runs \cite{goldberg2015faster}. This 
would be an useful extension of the \emph{adaptive flow iteration} strategy, where we solve
a sequence of similar flow problems around the cut of two blocks of a partition. \\
Pistorius \cite{pistorius2003} describes an algorithm which implicitly 
executes \EdmondKarp~on hypergraphs using labels on the hypernodes. 
In our first version of the framework, we used a similar 
technique and implicitly executes a flow algorithm on an implicit representation of 
the underlying network. During initial experiments, it turned out that the explicit representation
was up to a factor of $2$-$3$ faster than the implicit version. 
This is due to the fact that our flow network represents a subhypergraph of the 
original hypergraph. Iterating over the edges of a node means to iterate also 
over hypernodes which are not part of the flow problem and therefore have to be ignored.
Further, many labels have to be introduced which lead to a large number of 
main memory accesses. Also the implicit flow network is not flexible enough. Developing a new 
technique to sparsify the flow network would require a new implementation of 
the implicit flow network.\\
Our current framework optimizes the \emph{connectivity} metric of a $k$-way partition.
It turned out that it is relatively simply to adapt the algorithm to different objective
functions. If we want to improve the \emph{cut} metric of a $k$-way partition with a 
\emph{Max-Flow-Min-Cut} computation on two adjacent blocks $V_i$ and $V_j$, we only have to extend the
source and sink set of the resulting flow problem with additional nodes. More precisely,
if a hyperedge contains a block $V_k$ with $V_k \notin \{V_i,V_j\}$, we add the
\emph{incoming} and \emph{outgoing} hyperedge node to the source and sink set (without a proof).
If we want to optimize \emph{sum of external degree} metric, we can use the same source and
sink set as for the \emph{connectivity metric} and double the capacity of each hyperedge $e$
with $\lambda(e,\Pi) \le 2$ (without a proof). In future versions of the framework, we want to
generalize our observations such that it can optimize any objective function, which value depends
on the \emph{connectivity} of a hyperedge.

%It is also possible to further sparsify the flow network. Assume there exists two hypernodes
%$v_1$ and $v_2$ with $d(v_1) = 3$ and $d(v_2) = 4$. Further, $|I(v_1) \cap I(v_2)| = 3$ which means
%that in each hyperedge $e$ where $v_1 \in e$ also $v_2 \in e$ except for one hyperedge $e'$
%where $v_2 \in e'$ and $v_1 \notin e'$. We remove all hypernodes with $d(v) \le 3$ in our
%\emph{hybrid} flow network. Consequently, we would remove $v_1$ and insert a clique between all incident
%hyperedges. However, $v_2$ is part of the flow network and induce $2d(v_2) = 8$ edges because $d(v_2) > 3$.
%Alternatively, we can remove $v_2$ and expand the clique between all hyperedges of $I(v_1)$
%with $e'$. In that case, we have to insert an edge from each hyperedge in $I(v_1)$ to $e'$ and vice
%versa. Since $|I(v_1)| = d(v_1) = 3$ only $2|I(v_1)| = 6$ edges are inserted and we can remove
%one hypernode. In general, an expansion of a $k$-clique to a $(k+i)$-clique induced 
%$ik$ edges from the $k$ nodes already contained in the clique to the $i$ new nodes and
%$i(k+1-1)$ edges from the $i$ new nodes to the $k$ nodes in the clique. If we can remove a 
%hypernode from the flow network by expanding a $k$-clique between hyperedge nodes to a $(k+i)$-clique,
%it is beneficial if the following inequality holds 
%\[ik + i(k+i-1) = i^2 + 2ki - i \le 2(k+i)\]
%The inequality is only satisfied for $i = 1$. In this case, we can exactly remove $2$ edges and
%$1$ node from the flow network. A possible algorithm could be to sort the hypernodes according
%to the degree and for each hypernode store a clique label which indicates between how many
%incident hyperedges already exist a clique. Afterwards, we iterate over the hypernodes and if we remove
%a hypernode, we have to update the clique label of all hypernodes in the intersection of the
%currently inserted clique. We iterate over the hypernodes until none of the hypernodes could
%be removed anymore. However, we didn't find an efficient implementation of the above-described
%algorithm. The algorithm requires a fast calculation between the intersection of several
%hyperedges. An explicit construction of the intersection hypergraph would occupy too much
%memory. 


\newpage

%%%%%%%%%%%%%%%%%%%%%%%%%%%%%%%%%%%%%%%%%%%%%%%%%%%%%%%%%%%%%%%%%%%%%%

\bibliographystyle{abbrv}
\bibliography{../literature}

\newpage

\begin{appendix}
\section{Benchmark Instances}
\label{sec:benchmark_instances}

\subsection{Parameter Tunning Benchmark Set}
\label{sec:parameter_tunning_set}

\begin{table}[ht!]
\renewcommand{\arraystretch}{1.15}
\centering
\begin{tabular}{l|cccccc}
\toprule
Type & Num & $\min{|V|}$ & Avg.$|V|$ & $\max{|V|}$ & $\min{|E|}$ & Avg.$|E|$ \\
\midrule%
\csname @@input\endcsname experiments/instances/parameter_tunning_stats1.tex 
\midrule
Type & $\max{|E|}$ & Avg.$|e|$ & Med.$|e|$ & Avg.$d(v)$ & Med.$d(v)$ & Avg.$\frac{|E|}{|V|}$ \\
\midrule%
\csname @@input\endcsname experiments/instances/parameter_tunning_stats2.tex 
\bottomrule
\end{tabular} 
\caption{Summary of the parameter tunning instances.}
\label{tbl:parameter_tunning_set}
\end{table}

\subsection{Benchmark Subset}
\label{sec:benchmark_subset}

\begin{table}[ht!]
\renewcommand{\arraystretch}{1.15}
\centering
\begin{tabular}{l|cccccc}
\toprule
Type & Num & $\min{|V|}$ & Avg.$|V|$ & $\max{|V|}$ & $\min{|E|}$ & Avg.$|E|$ \\
\midrule%
\csname @@input\endcsname experiments/instances/subset_benchmark_stats1.tex 
\midrule
Type & $\max{|E|}$ & Avg.$|e|$ & Med.$|e|$ & Avg.$d(v)$ & Med.$d(v)$ & Avg.$\frac{|E|}{|V|}$ \\
\midrule%
\csname @@input\endcsname experiments/instances/subset_benchmark_stats2.tex 
\bottomrule
\end{tabular} 
\caption{Summary of the benchmark subset instances.}
\label{tbl:benchmark_subset}
\end{table}

\newpage
\subsection{Full Benchmark Set}
\label{sec:full_benchmark_set}

\begin{table}[ht!]
\renewcommand{\arraystretch}{1.15}
\centering
\begin{tabular}{l|cccccc}
\toprule
Type & Num & $\min{|V|}$ & Avg.$|V|$ & $\max{|V|}$ & $\min{|E|}$ & Avg.$|E|$ \\
\midrule%
\csname @@input\endcsname experiments/instances/full_benchmark_stats1.tex 
\midrule
Type & $\max{|E|}$ & Avg.$|e|$ & Med.$|e|$ & Avg.$d(v)$ & Med.$d(v)$ & Avg.$\frac{|E|}{|V|}$ \\
\midrule%
\csname @@input\endcsname experiments/instances/full_benchmark_stats2.tex 
\bottomrule
\end{tabular} 
\caption{Summary of the full benchmark set instances.}
\label{tbl:full_benchmark_set}
\end{table}

\subsection{Excluded Test Instances}
\label{app:excludedinstances}

{\footnotesize  
\begin{longtable}{p{7cm}lllllll}                                                                                                                                                                                                                                                                           
Hypergraph                                                                                                                      & 2           & 4                                         & 8                                         & 16                                        & 32                                        & 64                                                 & 128                                                \\
\hline
  10pipe-q0-k.dual                                                                                                            &             &                                           &                                           & $\triangle$                               & $\triangle$                               & $\triangle$                                        & \ding{109}$\triangle$                              \\
  10pipe-q0-k.primal                                                                                                          & $\square$   & $\square$                                 & $\square$                                 & $\square$                                 & $\square$                                 & $\square$                                          & $\square$                                          \\
  11pipe-k.dual                                                                                                               & $\triangle$ & \ding{109}$\triangle$                     & \ding{109}$\triangle$                     & \ding{109}$\triangle$                     & \ding{109}$\triangle$                     & \ding{109}$\triangle$                              & \ding{109}$\triangle$                              \\
  11pipe-k                                                                                                                    &             &                                           &                                           & \ding{109}                                & \ding{109}                                & \ding{109}                                         & \ding{109}                                         \\
  11pipe-k.primal                                                                                                             & $\square$   & $\square$                                 & $\square$                                 & $\square$                                 & $\square$                                 & $\square$                                          & \ding{109}$\square$                                \\
  11pipe-q0-k.dual                                                                                                            &             &                                           &                                           &                                           & $\triangle$                               & \ding{109}$\triangle$                              & \ding{109}$\triangle$                              \\
  11pipe-q0-k.primal                                                                                                          & $\square$   & $\square$                                 & $\square$                                 & $\square$                                 & $\square$                                 & $\square$                                          & $\square$                                          \\
  9dlx-vliw-at-b-iq3.dual                                                                                                     &             &                                           &                                           &                                           &                                           &                                                    & $\triangle$                                        \\
  9dlx-vliw-at-b-iq3.primal                                                                                                   & $\square$   & $\square$                                 & $\square$                                 & $\square$                                 & $\square$                                 & $\square$                                          & $\square$                                          \\
  9vliw-m-9stages-iq3-C1-bug7.dual                                                                                            & $\triangle$ & \ding{108}\ding{109}$\triangle$ & \ding{108}\ding{109}$\triangle$ & \ding{108}\ding{109}$\triangle$ & \ding{108}\ding{109}$\triangle$ & \ding{108}\ding{109}$\triangle$          & \ding{108}\ding{109}$\triangle$          \\
  9vliw-m-9stages-iq3-C1-bug7                                                                                                 & $\triangle$ & $\triangle$                               & \ding{108}\ding{109}$\triangle$ & \ding{108}\ding{109}$\triangle$ & \ding{108}\ding{109}$\triangle$ & \ding{108}\ding{109}$\square$$\triangle$ & \ding{108}\ding{109}$\square$$\triangle$ \\
  9vliw-m-9stages-iq3-C1-bug7.primal                                                                                          & $\triangle$ & $\triangle$                               &                                           & $\triangle$                               & \ding{109}$\triangle$                     & \ding{109}$\triangle$                              & \ding{109}$\triangle$                              \\
  9vliw-m-9stages-iq3-C1-bug8.dual                                                                                            & $\triangle$ & \ding{108}\ding{109}$\triangle$ & \ding{108}\ding{109}$\triangle$ & \ding{108}\ding{109}$\triangle$ & \ding{108}\ding{109}$\triangle$ & \ding{108}\ding{109}$\triangle$          & \ding{108}\ding{109}$\triangle$          \\
  9vliw-m-9stages-iq3-C1-bug8                                                                                                 & $\triangle$ & $\triangle$                               & \ding{108}\ding{109}$\triangle$ & \ding{108}\ding{109}$\triangle$ & \ding{108}\ding{109}$\triangle$ & \ding{108}\ding{109}$\square$$\triangle$ & \ding{108}\ding{109}$\square$$\triangle$ \\
  9vliw-m-9stages-iq3-C1-bug8.primal                                                                                          & $\triangle$ & $\triangle$                               &                                           & $\triangle$                               & \ding{109}$\triangle$                     & \ding{109}$\triangle$                              & \ding{109}$\triangle$                              \\
  blocks-blocks-37-1.130-NOTKNOWN.dual                                                                                        &             & \ding{109}                                & \ding{108}\ding{109}            & \ding{108}\ding{109}            & \ding{108}\ding{109}            & \ding{108}\ding{109}                     & \ding{108}\ding{109}$\triangle$          \\
  blocks-blocks-37-1.130-NOTKNOWN                                                                                             &             & $\square$                                 & $\square$                                 & $\square$                                 & $\square$                                 & $\square$                                          & $\square$                                          \\
  blocks-blocks-37-1.130-NOTKNOWN.primal                                                                                      & $\square$   & $\square$                                 & $\square$                                 & $\square$                                 & $\square$                                 & $\square$                                          & $\square$                                          \\
  E02F20.dual                                                                                                                 &             &                                           &                                           &                                           &                                           &                                                    & \ding{109}                                         \\
  E02F22.dual                                                                                                                 &             &                                           &                                           &                                           &                                           & \ding{109}                                         & \ding{109}                                         \\
  openstacks-p30-3.085-SAT.primal                                                                                             & $\square$   & $\square$                                 & $\square$                                 & $\square$                                 & $\square$                                 & $\square$                                          & $\square$                                          \\
  openstacks-sequencedstrips-nonadl-nonnegated-os-sequencedstrips-p30-3.025-NOTKNOWN.primal                                   & $\square$   & $\square$                                 & $\square$                                 & $\square$                                 & $\square$                                 & $\square$                                          & $\square$                                          \\
  openstacks-sequencedstrips-nonadl-nonnegated-os-sequencedstrips-p30-3.085-SAT.primal                                        & $\square$   & $\square$                                 & $\square$                                 & $\square$                                 & $\square$                                 & $\square$                                          & $\square$                                          \\
  q-query-3-L100-coli.sat.dual                                                                                                &             &                                           &                                           &                                           &                                           &                                                    & $\triangle$                                        \\
  q-query-3-L150-coli.sat.dual                                                                                                &             &                                           &                                           &                                           &                                           & $\triangle$                                        & $\triangle$                                        \\
  q-query-3-L200-coli.sat.dual                                                                                                &             &                                           &                                           &                                           & $\triangle$                               & $\triangle$                                        & $\triangle$                                        \\
  q-query-3-L80-coli.sat.dual                                                                                                 &             &                                           &                                           &                                           &                                           &                                                    & $\triangle$                                        \\
  transport-transport-city-sequential-25nodes-1000size-3degree-100mindistance-3trucks-10packages-2008seed.030-NOTKNOWN.dual   &             &                                           &                                           &                                           &                                           &                                                    & $\triangle$                                        \\
  transport-transport-city-sequential-25nodes-1000size-3degree-100mindistance-3trucks-10packages-2008seed.050-NOTKNOWN.primal & $\square$   &                                           &                                           & $\square$                                 &                                           &                                                    & $\square$                                          \\
  velev-vliw-uns-2.0-uq5.dual                                                                                                 &             &                                           & $\triangle$                               & $\triangle$                               & $\triangle$                               & $\triangle$                                        & $\triangle$                                        \\
  velev-vliw-uns-2.0-uq5.primal                                                                                               & $\square$   & $\square$                                 & $\square$                                 & $\square$                                 & $\square$                                 & $\square$                                          & $\square$                                          \\
  velev-vliw-uns-4.0-9.dual                                                                                                   &             &                                           &                                           &                                           & $\triangle$                               & $\triangle$                                        & $\triangle$                                        \\
  velev-vliw-uns-4.0-9.primal                                                                                                 & $\square$   & $\square$                                 & $\square$                                 & $\square$                                 & $\square$                                 & $\square$                                          & $\square$                                               \\
\hline
  192bit                                                                                                                      & $\square$   &                                           &                                           & $\square$                                 &                                           &                                                    &                                                    \\
  appu                                                                                                                        &             &                                           &                                           &                                           &                                           & \ding{109}                                         & \ding{109}                                         \\
  ESOC                                                                                                                        & $\square$   & $\square$                                 &                                           &                                           & $\square$                                 & \ding{109}$\square$                                & $\square$                                          \\
  human-gene2                                                                                                                 &             &                                           &                                           &                                           & \ding{109}$\triangle$                     & \ding{109}$\triangle$                              & \ding{109}$\triangle$                              \\
  IMDB                                                                                                                        &             &                                           &                                           & $\triangle$                               & $\triangle$                               & $\triangle$                                        & $\triangle$                                        \\
  kron-g500-logn16                                                                                                            &             & $\triangle$                               & $\triangle$                               & $\triangle$                               & $\triangle$                               & \ding{109}$\triangle$                              & \ding{109}$\triangle$                              \\
  nlpkkt120                                                                                                                   &             & $\triangle$                               & $\triangle$                               & $\triangle$                               & $\triangle$                               & $\triangle$                              & \ding{109}$\triangle$                              \\
  Rucci1                                                                                                                      &             &                                           &                                           &                                           & $\square$                                 &                                                    &                                                    \\
  sls                                                                                                                         & $\square$   & $\square$                                 & $\square$                                 & \ding{109}$\square$                       & \ding{109}$\square$                       & \ding{109}$\square$                                & \ding{109}$\square$                                \\
  Trec14                                                                                                                      &             &                                           &                                           &                                           &                                           &                                                    & \ding{109}                                         
\end{longtable}}
\begin{table}[h!]
\centering
  \caption*{
    \begin{tabular}{|l l|}
      \hline
      $\triangle$~:              & \KaHyPar{CA}/\KaHyPar{MF} exceeded time limit                                                                                                                                                                                                            \\
      \ding{108}~:               & \hMetis{R} exceeded time limit                                                                                                                                                                                                           \\
      \ding{109}~:               & \hMetis{K} exceeded time limit                                                                                                                                                                                                           \\
      $\square$~:                & \PaToH{Q} memory allocation error                                                                                                                                                                                                        \\
      \hline
    \end{tabular}
  }
  \caption{Instances excluded from the full benchmark set evaluation.}~\label{tbl:excluded}  
\end{table}

\newpage
\section{Removing Infinite Weight Nodes of the Vertex Separator Problem}
\label{sec:infinite_weight_vertex_separator}

Finding a minimum-weight $(s,t)$-cutset of a hypergraph can be reduced to the problem 
of finding a minimum-weight $(s,t)$-vertex separator of the bipartite graph representation
of the hypergraph. The weight of the hyperedge nodes is the weight of the corresponding
hyperedge and the weight of all hypernodes is \emph{infinity}. We can calculate
a minimum-weight $(s,t)$-vertex separator with a maximum $(s,t)$-flow calculation 
on the Lawler-Transformation \ShortT{L}. In Section \ref{sec:heuer_network} 
we have shown how to remove a hypernode $v$ from \ShortT{L} by adding a 
biclique between all outgoing hyperedges $\outgoing{e}$ and all
incoming hyperedges $\incoming{e}$ with $e \in I(v)$. The correctness of the transformation
follows with Lemma \ref{lemma:node_removal}. The inserted biclique induce exactly
$d(v)^2$ edges where $d(v)$ is the degree of hypernode $v$. Since the underlying problem
is to find a minimum-weight $(s,t)$-vertex separator, we can prove that $d(v)(d(v)-1)$ edges
are sufficient to model the problem equivalent. The problem is illustrated in \autoref{img:vertex_separator_transformation}.
In the following, we show that we can remove the red edges which are exactly the edges between
two equivalent hyperedge nodes. 

\begin{lemma}
\label{lemma:vertex_separator_lemma}
Let $G = (V,E,c)$ be an undirected graph where a node $v$ exists with $c(v) = \infty$.
We can find a minimum-weight $(s,t)$-vertex separator of $G$ (weight must be smaller than 
$\infty$) with graph $G'$ which is the graph without node $v$ and having a clique 
between all adjacent nodes of $v$.
\end{lemma}

\begin{proof}
We will show that each $(s,t)$-vertex separator of $G$ is also a $(s,t)$-vertex separator
of $G'$ and vice versa. Since, the weight of each node in both network is the same, a minimum-weight
$(s,t)$-vertex separator of $G$ is also a minimum-weight $(s,t)$-vertex separator of $G'$ and vice versa.
We will call an edge of $G'$, which is part of the inserted clique, a \emph{shortcut} edge.\\
Let $V' \subseteq V$ be a $(s,t)$-vertex separator of $G$ with $c(V') < \infty$. Assume after removing
all $u \in V'$ from $G'$ there exists still a path from $s$ to $t$. If the path not contains any
\emph{shortcut} edge, the same path must connect $s$ and $t$ in $G$ because we have removed all $u \in V'$
from $G'$. This is a contradiction that $V'$ is a vertex separator of $G$. Assume the path contains
a \emph{shortcut} edge $(u,w)$ of $G'$.  Because $c(V') < \infty$ and $c(v) = \infty$,
it follows that $v \notin V'$. We can replace $(u,w)$ with two edges $(u,v)$ and $(v,w)$
and obtain a path connecting $s$ and $t$ in $G$. This is also a contradiction that $V'$ is a vertex
separator of $G$.\\
The same argumentation holds, if we want to show that each $(s,t)$-vertex separator of $G'$
is a $(s,t)$-vertex separator of $G$. If the path $P$ connecting $s$ and $t$ in $G$ contains
the removed node $v$ with edges $(u,v) \in P$ and $(v,w) \in P$ we can replace the two edges
with the \emph{shortcut} edge $(u,w)$ of $G'$. The resulting path connects $s$ and $t$ in $G'$.
This is a contradiction that $V'$ is a vertex separator of $G'$.
\end{proof}

Lemma \ref{lemma:vertex_separator_lemma} can be used to remove the \emph{infinity} capacity
edges between the same hyperedge nodes (see red edges in \autoref{img:vertex_separator_transformation}).
Instead of removing a hypernode $v$ of \ShortT{L} with Lemma \ref{lemma:node_removal}, we can apply
Lemma \ref{lemma:vertex_separator_lemma} on a hypernode $v$ of the bipartite graph representation
of the hypergraph. Afterwards, we can use the vertex separator transformation (see Defintion \ref{def:vertex_seperator_transformation})
to obtain a flow network in which the value of a maximum $(s,t)$-flow is equal with the
weight of a minimum-weight $(s,t)$-vertex separator. Both networks on the right side of 
\autoref{img:vertex_separator_transformation} are equivalent, but a removal of a hypernode
now induce $d(v)(d(v) - 1)$ edges instead of $d(v)^2$ edges in the new network.

\begin{figure}
\centering
\includegraphics[width=1.0\textwidth]{../img/network_transformation/vertex_separator_lemma.eps}
\caption{Illustration of the technique to remove a hypernode such that the removal induce 
         $d(v)(d(v) - 1)$ edges instead of $d(v)^2$ edges in the corresponding flow network.
         The edges without an explicit capacity are \emph{infinite} capacity edges.
         The transformation on the top of the figure illustrates the technique presented in
         Section \ref{sec:heuer_network}. Using Lemma \ref{lemma:vertex_separator_lemma} and
         the vertex separator transformation (see Defintion \ref{def:vertex_seperator_transformation})
         results in a equivalent flow network without the red highlighted edges.}
\label{img:vertex_separator_transformation}
\end{figure}

\newpage
\section{Detailed Flow Network and Algorithm Evaluation}

\begin{table}[ht!]
\renewcommand{\arraystretch}{1.15}
\footnotesize
\centering
\begin{tabular}{lr|C{2.5cm}|c|c|c}
\toprule
 \multirow{2}{*}{\rotatebox{90}{\tiny{Instance}}} & \quad\quad & \IBFS & \BoykovKolmogorov & \GoldbergTarjan & \EdmondKarp \\
 & $|V'|$ &  $t[ms]$ & $t[\%]$ & $t[\%]$ & $t[\%]$ 
\\\midrule% 
\csname @@input\endcsname experiments/flow_network/flow_network_max_flow_table.tex 
\bottomrule
\end{tabular}
\caption{Average running times of our maximum flow algorithms on flow network $\ExpHybrid$.
         Note, all values in the table are in percentage relative to the running time
         of the \IBFS~algorithm. In each line the fastest variant is marked bold.}
\label{tbl:flow_algo_network_detail_summary}
\end{table}

\newpage
\section{Effectiveness Tests for Flow Configurations}
\label{appendix:effectiveness_test}

To evaluate the effectiveness of our configurations presented in Section \ref{sec:flow_configuration}
we give each configuration the same amount of time to produce as many as possible partitions of
a hypergraph $H$ for a given $k$. We define
$t_{H,k}$ which is the maximum partition time of a configuration to partition $H$ in $k$
blocks. If we execute a configuration on a hypergraph $H$ for a given $k$ and $\alpha'$ the
time to produce as many as possible partitions is restricted by $3t_{H,k}$. We sum up the partition
times during execution and if that sum plus the current average partition time would exceed $3t_{H,k}$ we perform
the next run with a certain probability such that the expected running time is $3t_{H,k}$.
The effectiveness tests were proposed by Sanders and Schulz \cite{sanders2011engineering}.
The results of the tests mirrors our results of Section \ref{sec:flow_configuration}.


\begin{table}[ht]
\renewcommand{\arraystretch}{1.15}
\centering
\begin{tabular}{|r|c|c|c|}
\toprule
 Config. & \FlowVariant{+}{-}{-} & \FlowVariant{+}{+}{-}  & \FlowVariant{+}{+}{+} \\
\midrule
$\alpha'$ & Avg.$[\%]$ & Avg.$[\%]$ & Avg.$[\%]$ \\
\midrule%
\csname @@input\endcsname experiments/flow_alpha/flow_alpha_effectiveness_table.tex 
\bottomrule
\end{tabular}
\caption{ Table contains results of the effectiveness test 
          for different configurations of our flow-based refinement
          framework for increasing $\alpha'$. The quality in column \emph{Avg.} is relative
          to our baseline configuration \FlowVariant{-}{-}{+}. }
\label{tbl:alpha_effectiveness_exp}
\end{table}

\newpage
\section{Detailed Speed-Up Heuristic Evaluation}
\begin{figure}[h!]
\centering
% Created by tikzDevice version 0.6.2-92-0ad2792 on 2017-12-27 15:56:16
% !TEX encoding = UTF-8 Unicode
\begin{tikzpicture}[x=1pt,y=1pt]
\definecolor[named]{fillColor}{rgb}{1.00,1.00,1.00}
\path[use as bounding box,fill=fillColor,fill opacity=0.00] (0,0) rectangle (505.89,614.29);
\begin{scope}
\path[clip] ( 23.31,460.72) rectangle (229.63,614.29);
\definecolor[named]{drawColor}{rgb}{1.00,1.00,1.00}
\definecolor[named]{fillColor}{rgb}{1.00,1.00,1.00}

\path[draw=drawColor,line width= 0.6pt,line join=round,line cap=round,fill=fillColor] ( 23.31,460.72) rectangle (229.63,614.29);
\end{scope}
\begin{scope}
\path[clip] ( 74.44,500.63) rectangle (223.63,592.83);
\definecolor[named]{fillColor}{rgb}{1.00,1.00,1.00}

\path[fill=fillColor] ( 74.44,500.63) rectangle (223.63,592.83);
\definecolor[named]{drawColor}{rgb}{0.98,0.98,0.98}

\path[draw=drawColor,line width= 0.6pt,line join=round] ( 74.44,516.59) --
	(223.63,516.59);

\path[draw=drawColor,line width= 0.6pt,line join=round] ( 74.44,530.74) --
	(223.63,530.74);

\path[draw=drawColor,line width= 0.6pt,line join=round] ( 74.44,544.93) --
	(223.63,544.93);

\path[draw=drawColor,line width= 0.6pt,line join=round] ( 74.44,561.27) --
	(223.63,561.27);

\path[draw=drawColor,line width= 0.6pt,line join=round] ( 74.44,573.71) --
	(223.63,573.71);

\path[draw=drawColor,line width= 0.6pt,line join=round] ( 74.44,581.63) --
	(223.63,581.63);

\path[draw=drawColor,line width= 0.6pt,line join=round] (128.93,500.63) --
	(128.93,592.83);

\path[draw=drawColor,line width= 0.6pt,line join=round] (139.82,500.63) --
	(139.82,592.83);

\path[draw=drawColor,line width= 0.6pt,line join=round] (161.61,500.63) --
	(161.61,592.83);

\path[draw=drawColor,line width= 0.6pt,line join=round] (180.15,500.63) --
	(180.15,592.83);

\path[draw=drawColor,line width= 0.6pt,line join=round] (193.87,500.63) --
	(193.87,592.83);

\path[draw=drawColor,line width= 0.6pt,line join=round] (205.09,500.63) --
	(205.09,592.83);

\path[draw=drawColor,line width= 0.6pt,line join=round] (214.72,500.63) --
	(214.72,592.83);
\definecolor[named]{drawColor}{rgb}{0.75,0.75,0.75}

\path[draw=drawColor,line width= 0.6pt,dash pattern=on 1pt off 3pt ,line join=round] ( 74.44,508.31) --
	(223.63,508.31);

\path[draw=drawColor,line width= 0.6pt,dash pattern=on 1pt off 3pt ,line join=round] ( 74.44,524.86) --
	(223.63,524.86);

\path[draw=drawColor,line width= 0.6pt,dash pattern=on 1pt off 3pt ,line join=round] ( 74.44,536.62) --
	(223.63,536.62);

\path[draw=drawColor,line width= 0.6pt,dash pattern=on 1pt off 3pt ,line join=round] ( 74.44,553.24) --
	(223.63,553.24);

\path[draw=drawColor,line width= 0.6pt,dash pattern=on 1pt off 3pt ,line join=round] ( 74.44,569.30) --
	(223.63,569.30);

\path[draw=drawColor,line width= 0.6pt,dash pattern=on 1pt off 3pt ,line join=round] ( 74.44,578.12) --
	(223.63,578.12);

\path[draw=drawColor,line width= 0.6pt,dash pattern=on 1pt off 3pt ,line join=round] ( 74.44,585.15) --
	(223.63,585.15);

\path[draw=drawColor,line width= 0.6pt,dash pattern=on 1pt off 3pt ,line join=round] (150.72,500.63) --
	(150.72,592.83);

\path[draw=drawColor,line width= 0.6pt,dash pattern=on 1pt off 3pt ,line join=round] (172.51,500.63) --
	(172.51,592.83);

\path[draw=drawColor,line width= 0.6pt,dash pattern=on 1pt off 3pt ,line join=round] (187.79,500.63) --
	(187.79,592.83);

\path[draw=drawColor,line width= 0.6pt,dash pattern=on 1pt off 3pt ,line join=round] (199.96,500.63) --
	(199.96,592.83);

\path[draw=drawColor,line width= 0.6pt,dash pattern=on 1pt off 3pt ,line join=round] (210.23,500.63) --
	(210.23,592.83);

\path[draw=drawColor,line width= 0.6pt,dash pattern=on 1pt off 3pt ,line join=round] (219.21,500.63) --
	(219.21,592.83);
\definecolor[named]{drawColor}{rgb}{0.89,0.10,0.11}
\definecolor[named]{fillColor}{rgb}{0.89,0.10,0.11}

\path[draw=drawColor,line width= 0.4pt,line join=round,line cap=round,fill=fillColor] ( 81.22,555.33) circle (  1.16);

\path[draw=drawColor,line width= 0.4pt,line join=round,line cap=round,fill=fillColor] ( 84.95,554.87) circle (  1.16);

\path[draw=drawColor,line width= 0.4pt,line join=round,line cap=round,fill=fillColor] ( 87.56,554.17) circle (  1.16);

\path[draw=drawColor,line width= 0.4pt,line join=round,line cap=round,fill=fillColor] ( 89.64,550.38) circle (  1.16);

\path[draw=drawColor,line width= 0.4pt,line join=round,line cap=round,fill=fillColor] ( 91.40,548.35) circle (  1.16);

\path[draw=drawColor,line width= 0.4pt,line join=round,line cap=round,fill=fillColor] ( 92.94,548.21) circle (  1.16);

\path[draw=drawColor,line width= 0.4pt,line join=round,line cap=round,fill=fillColor] ( 94.31,547.59) circle (  1.16);

\path[draw=drawColor,line width= 0.4pt,line join=round,line cap=round,fill=fillColor] ( 95.56,546.90) circle (  1.16);

\path[draw=drawColor,line width= 0.4pt,line join=round,line cap=round,fill=fillColor] ( 96.71,546.69) circle (  1.16);

\path[draw=drawColor,line width= 0.4pt,line join=round,line cap=round,fill=fillColor] ( 97.77,546.34) circle (  1.16);

\path[draw=drawColor,line width= 0.4pt,line join=round,line cap=round,fill=fillColor] ( 98.77,545.57) circle (  1.16);

\path[draw=drawColor,line width= 0.4pt,line join=round,line cap=round,fill=fillColor] ( 99.71,545.50) circle (  1.16);

\path[draw=drawColor,line width= 0.4pt,line join=round,line cap=round,fill=fillColor] (100.59,545.49) circle (  1.16);

\path[draw=drawColor,line width= 0.4pt,line join=round,line cap=round,fill=fillColor] (101.44,545.48) circle (  1.16);

\path[draw=drawColor,line width= 0.4pt,line join=round,line cap=round,fill=fillColor] (102.24,544.99) circle (  1.16);

\path[draw=drawColor,line width= 0.4pt,line join=round,line cap=round,fill=fillColor] (103.01,544.62) circle (  1.16);

\path[draw=drawColor,line width= 0.4pt,line join=round,line cap=round,fill=fillColor] (103.75,544.42) circle (  1.16);

\path[draw=drawColor,line width= 0.4pt,line join=round,line cap=round,fill=fillColor] (104.46,544.37) circle (  1.16);

\path[draw=drawColor,line width= 0.4pt,line join=round,line cap=round,fill=fillColor] (105.14,544.18) circle (  1.16);

\path[draw=drawColor,line width= 0.4pt,line join=round,line cap=round,fill=fillColor] (105.80,544.09) circle (  1.16);

\path[draw=drawColor,line width= 0.4pt,line join=round,line cap=round,fill=fillColor] (106.44,543.97) circle (  1.16);

\path[draw=drawColor,line width= 0.4pt,line join=round,line cap=round,fill=fillColor] (107.05,543.97) circle (  1.16);

\path[draw=drawColor,line width= 0.4pt,line join=round,line cap=round,fill=fillColor] (107.65,543.40) circle (  1.16);

\path[draw=drawColor,line width= 0.4pt,line join=round,line cap=round,fill=fillColor] (108.24,543.13) circle (  1.16);

\path[draw=drawColor,line width= 0.4pt,line join=round,line cap=round,fill=fillColor] (108.80,543.12) circle (  1.16);

\path[draw=drawColor,line width= 0.4pt,line join=round,line cap=round,fill=fillColor] (109.35,543.09) circle (  1.16);

\path[draw=drawColor,line width= 0.4pt,line join=round,line cap=round,fill=fillColor] (109.89,542.80) circle (  1.16);

\path[draw=drawColor,line width= 0.4pt,line join=round,line cap=round,fill=fillColor] (110.42,542.28) circle (  1.16);

\path[draw=drawColor,line width= 0.4pt,line join=round,line cap=round,fill=fillColor] (110.93,542.27) circle (  1.16);

\path[draw=drawColor,line width= 0.4pt,line join=round,line cap=round,fill=fillColor] (111.43,542.08) circle (  1.16);

\path[draw=drawColor,line width= 0.4pt,line join=round,line cap=round,fill=fillColor] (111.92,542.03) circle (  1.16);

\path[draw=drawColor,line width= 0.4pt,line join=round,line cap=round,fill=fillColor] (112.40,542.00) circle (  1.16);

\path[draw=drawColor,line width= 0.4pt,line join=round,line cap=round,fill=fillColor] (112.87,541.93) circle (  1.16);

\path[draw=drawColor,line width= 0.4pt,line join=round,line cap=round,fill=fillColor] (113.33,541.63) circle (  1.16);

\path[draw=drawColor,line width= 0.4pt,line join=round,line cap=round,fill=fillColor] (113.78,541.57) circle (  1.16);

\path[draw=drawColor,line width= 0.4pt,line join=round,line cap=round,fill=fillColor] (114.22,541.44) circle (  1.16);

\path[draw=drawColor,line width= 0.4pt,line join=round,line cap=round,fill=fillColor] (114.65,541.38) circle (  1.16);

\path[draw=drawColor,line width= 0.4pt,line join=round,line cap=round,fill=fillColor] (115.08,541.16) circle (  1.16);

\path[draw=drawColor,line width= 0.4pt,line join=round,line cap=round,fill=fillColor] (115.50,541.14) circle (  1.16);

\path[draw=drawColor,line width= 0.4pt,line join=round,line cap=round,fill=fillColor] (115.91,540.95) circle (  1.16);

\path[draw=drawColor,line width= 0.4pt,line join=round,line cap=round,fill=fillColor] (116.32,540.92) circle (  1.16);

\path[draw=drawColor,line width= 0.4pt,line join=round,line cap=round,fill=fillColor] (116.72,540.90) circle (  1.16);

\path[draw=drawColor,line width= 0.4pt,line join=round,line cap=round,fill=fillColor] (117.11,540.86) circle (  1.16);

\path[draw=drawColor,line width= 0.4pt,line join=round,line cap=round,fill=fillColor] (117.49,540.73) circle (  1.16);

\path[draw=drawColor,line width= 0.4pt,line join=round,line cap=round,fill=fillColor] (117.87,540.67) circle (  1.16);

\path[draw=drawColor,line width= 0.4pt,line join=round,line cap=round,fill=fillColor] (118.25,540.64) circle (  1.16);

\path[draw=drawColor,line width= 0.4pt,line join=round,line cap=round,fill=fillColor] (118.62,540.59) circle (  1.16);

\path[draw=drawColor,line width= 0.4pt,line join=round,line cap=round,fill=fillColor] (118.98,540.42) circle (  1.16);

\path[draw=drawColor,line width= 0.4pt,line join=round,line cap=round,fill=fillColor] (119.34,540.41) circle (  1.16);

\path[draw=drawColor,line width= 0.4pt,line join=round,line cap=round,fill=fillColor] (119.70,540.29) circle (  1.16);

\path[draw=drawColor,line width= 0.4pt,line join=round,line cap=round,fill=fillColor] (120.05,540.17) circle (  1.16);

\path[draw=drawColor,line width= 0.4pt,line join=round,line cap=round,fill=fillColor] (120.39,540.13) circle (  1.16);

\path[draw=drawColor,line width= 0.4pt,line join=round,line cap=round,fill=fillColor] (120.73,540.11) circle (  1.16);

\path[draw=drawColor,line width= 0.4pt,line join=round,line cap=round,fill=fillColor] (121.07,540.09) circle (  1.16);

\path[draw=drawColor,line width= 0.4pt,line join=round,line cap=round,fill=fillColor] (121.40,539.87) circle (  1.16);

\path[draw=drawColor,line width= 0.4pt,line join=round,line cap=round,fill=fillColor] (121.73,539.77) circle (  1.16);

\path[draw=drawColor,line width= 0.4pt,line join=round,line cap=round,fill=fillColor] (122.05,539.75) circle (  1.16);

\path[draw=drawColor,line width= 0.4pt,line join=round,line cap=round,fill=fillColor] (122.38,539.73) circle (  1.16);

\path[draw=drawColor,line width= 0.4pt,line join=round,line cap=round,fill=fillColor] (122.69,539.63) circle (  1.16);

\path[draw=drawColor,line width= 0.4pt,line join=round,line cap=round,fill=fillColor] (123.01,539.61) circle (  1.16);

\path[draw=drawColor,line width= 0.4pt,line join=round,line cap=round,fill=fillColor] (123.32,539.55) circle (  1.16);

\path[draw=drawColor,line width= 0.4pt,line join=round,line cap=round,fill=fillColor] (123.62,539.32) circle (  1.16);

\path[draw=drawColor,line width= 0.4pt,line join=round,line cap=round,fill=fillColor] (123.93,539.08) circle (  1.16);

\path[draw=drawColor,line width= 0.4pt,line join=round,line cap=round,fill=fillColor] (124.23,539.00) circle (  1.16);

\path[draw=drawColor,line width= 0.4pt,line join=round,line cap=round,fill=fillColor] (124.52,538.99) circle (  1.16);

\path[draw=drawColor,line width= 0.4pt,line join=round,line cap=round,fill=fillColor] (124.82,538.96) circle (  1.16);

\path[draw=drawColor,line width= 0.4pt,line join=round,line cap=round,fill=fillColor] (125.11,538.93) circle (  1.16);

\path[draw=drawColor,line width= 0.4pt,line join=round,line cap=round,fill=fillColor] (125.40,538.92) circle (  1.16);

\path[draw=drawColor,line width= 0.4pt,line join=round,line cap=round,fill=fillColor] (125.68,538.90) circle (  1.16);

\path[draw=drawColor,line width= 0.4pt,line join=round,line cap=round,fill=fillColor] (125.96,538.81) circle (  1.16);

\path[draw=drawColor,line width= 0.4pt,line join=round,line cap=round,fill=fillColor] (126.24,538.76) circle (  1.16);

\path[draw=drawColor,line width= 0.4pt,line join=round,line cap=round,fill=fillColor] (126.52,538.73) circle (  1.16);

\path[draw=drawColor,line width= 0.4pt,line join=round,line cap=round,fill=fillColor] (126.80,538.72) circle (  1.16);

\path[draw=drawColor,line width= 0.4pt,line join=round,line cap=round,fill=fillColor] (127.07,538.67) circle (  1.16);

\path[draw=drawColor,line width= 0.4pt,line join=round,line cap=round,fill=fillColor] (127.34,538.62) circle (  1.16);

\path[draw=drawColor,line width= 0.4pt,line join=round,line cap=round,fill=fillColor] (127.61,538.61) circle (  1.16);

\path[draw=drawColor,line width= 0.4pt,line join=round,line cap=round,fill=fillColor] (127.87,538.58) circle (  1.16);

\path[draw=drawColor,line width= 0.4pt,line join=round,line cap=round,fill=fillColor] (128.13,538.53) circle (  1.16);

\path[draw=drawColor,line width= 0.4pt,line join=round,line cap=round,fill=fillColor] (128.40,538.53) circle (  1.16);

\path[draw=drawColor,line width= 0.4pt,line join=round,line cap=round,fill=fillColor] (128.65,538.44) circle (  1.16);

\path[draw=drawColor,line width= 0.4pt,line join=round,line cap=round,fill=fillColor] (128.91,538.43) circle (  1.16);

\path[draw=drawColor,line width= 0.4pt,line join=round,line cap=round,fill=fillColor] (129.16,538.42) circle (  1.16);

\path[draw=drawColor,line width= 0.4pt,line join=round,line cap=round,fill=fillColor] (129.42,538.42) circle (  1.16);

\path[draw=drawColor,line width= 0.4pt,line join=round,line cap=round,fill=fillColor] (129.67,538.42) circle (  1.16);

\path[draw=drawColor,line width= 0.4pt,line join=round,line cap=round,fill=fillColor] (129.91,538.39) circle (  1.16);

\path[draw=drawColor,line width= 0.4pt,line join=round,line cap=round,fill=fillColor] (130.16,538.36) circle (  1.16);

\path[draw=drawColor,line width= 0.4pt,line join=round,line cap=round,fill=fillColor] (130.41,538.32) circle (  1.16);

\path[draw=drawColor,line width= 0.4pt,line join=round,line cap=round,fill=fillColor] (130.65,538.28) circle (  1.16);

\path[draw=drawColor,line width= 0.4pt,line join=round,line cap=round,fill=fillColor] (130.89,538.16) circle (  1.16);

\path[draw=drawColor,line width= 0.4pt,line join=round,line cap=round,fill=fillColor] (131.13,538.10) circle (  1.16);

\path[draw=drawColor,line width= 0.4pt,line join=round,line cap=round,fill=fillColor] (131.36,538.09) circle (  1.16);

\path[draw=drawColor,line width= 0.4pt,line join=round,line cap=round,fill=fillColor] (131.60,538.08) circle (  1.16);

\path[draw=drawColor,line width= 0.4pt,line join=round,line cap=round,fill=fillColor] (131.83,538.05) circle (  1.16);

\path[draw=drawColor,line width= 0.4pt,line join=round,line cap=round,fill=fillColor] (132.06,538.03) circle (  1.16);

\path[draw=drawColor,line width= 0.4pt,line join=round,line cap=round,fill=fillColor] (132.30,537.99) circle (  1.16);

\path[draw=drawColor,line width= 0.4pt,line join=round,line cap=round,fill=fillColor] (132.52,537.94) circle (  1.16);

\path[draw=drawColor,line width= 0.4pt,line join=round,line cap=round,fill=fillColor] (132.75,537.90) circle (  1.16);

\path[draw=drawColor,line width= 0.4pt,line join=round,line cap=round,fill=fillColor] (132.98,537.89) circle (  1.16);

\path[draw=drawColor,line width= 0.4pt,line join=round,line cap=round,fill=fillColor] (133.20,537.87) circle (  1.16);

\path[draw=drawColor,line width= 0.4pt,line join=round,line cap=round,fill=fillColor] (133.42,537.86) circle (  1.16);

\path[draw=drawColor,line width= 0.4pt,line join=round,line cap=round,fill=fillColor] (133.64,537.85) circle (  1.16);

\path[draw=drawColor,line width= 0.4pt,line join=round,line cap=round,fill=fillColor] (133.86,537.84) circle (  1.16);

\path[draw=drawColor,line width= 0.4pt,line join=round,line cap=round,fill=fillColor] (134.08,537.82) circle (  1.16);

\path[draw=drawColor,line width= 0.4pt,line join=round,line cap=round,fill=fillColor] (134.30,537.79) circle (  1.16);

\path[draw=drawColor,line width= 0.4pt,line join=round,line cap=round,fill=fillColor] (134.51,537.78) circle (  1.16);

\path[draw=drawColor,line width= 0.4pt,line join=round,line cap=round,fill=fillColor] (134.73,537.74) circle (  1.16);

\path[draw=drawColor,line width= 0.4pt,line join=round,line cap=round,fill=fillColor] (134.94,537.74) circle (  1.16);

\path[draw=drawColor,line width= 0.4pt,line join=round,line cap=round,fill=fillColor] (135.15,537.73) circle (  1.16);

\path[draw=drawColor,line width= 0.4pt,line join=round,line cap=round,fill=fillColor] (135.36,537.72) circle (  1.16);

\path[draw=drawColor,line width= 0.4pt,line join=round,line cap=round,fill=fillColor] (135.57,537.70) circle (  1.16);

\path[draw=drawColor,line width= 0.4pt,line join=round,line cap=round,fill=fillColor] (135.78,537.65) circle (  1.16);

\path[draw=drawColor,line width= 0.4pt,line join=round,line cap=round,fill=fillColor] (135.98,537.61) circle (  1.16);

\path[draw=drawColor,line width= 0.4pt,line join=round,line cap=round,fill=fillColor] (136.19,537.61) circle (  1.16);

\path[draw=drawColor,line width= 0.4pt,line join=round,line cap=round,fill=fillColor] (136.39,537.60) circle (  1.16);

\path[draw=drawColor,line width= 0.4pt,line join=round,line cap=round,fill=fillColor] (136.60,537.58) circle (  1.16);

\path[draw=drawColor,line width= 0.4pt,line join=round,line cap=round,fill=fillColor] (136.80,537.55) circle (  1.16);

\path[draw=drawColor,line width= 0.4pt,line join=round,line cap=round,fill=fillColor] (137.00,537.54) circle (  1.16);

\path[draw=drawColor,line width= 0.4pt,line join=round,line cap=round,fill=fillColor] (137.20,537.53) circle (  1.16);

\path[draw=drawColor,line width= 0.4pt,line join=round,line cap=round,fill=fillColor] (137.40,537.52) circle (  1.16);

\path[draw=drawColor,line width= 0.4pt,line join=round,line cap=round,fill=fillColor] (137.59,537.50) circle (  1.16);

\path[draw=drawColor,line width= 0.4pt,line join=round,line cap=round,fill=fillColor] (137.79,537.47) circle (  1.16);

\path[draw=drawColor,line width= 0.4pt,line join=round,line cap=round,fill=fillColor] (137.98,537.47) circle (  1.16);

\path[draw=drawColor,line width= 0.4pt,line join=round,line cap=round,fill=fillColor] (138.18,537.44) circle (  1.16);

\path[draw=drawColor,line width= 0.4pt,line join=round,line cap=round,fill=fillColor] (138.37,537.42) circle (  1.16);

\path[draw=drawColor,line width= 0.4pt,line join=round,line cap=round,fill=fillColor] (138.56,537.41) circle (  1.16);

\path[draw=drawColor,line width= 0.4pt,line join=round,line cap=round,fill=fillColor] (138.75,537.31) circle (  1.16);

\path[draw=drawColor,line width= 0.4pt,line join=round,line cap=round,fill=fillColor] (138.94,537.22) circle (  1.16);

\path[draw=drawColor,line width= 0.4pt,line join=round,line cap=round,fill=fillColor] (139.13,537.17) circle (  1.16);

\path[draw=drawColor,line width= 0.4pt,line join=round,line cap=round,fill=fillColor] (139.32,537.17) circle (  1.16);

\path[draw=drawColor,line width= 0.4pt,line join=round,line cap=round,fill=fillColor] (139.50,537.14) circle (  1.16);

\path[draw=drawColor,line width= 0.4pt,line join=round,line cap=round,fill=fillColor] (139.69,537.11) circle (  1.16);

\path[draw=drawColor,line width= 0.4pt,line join=round,line cap=round,fill=fillColor] (139.87,537.05) circle (  1.16);

\path[draw=drawColor,line width= 0.4pt,line join=round,line cap=round,fill=fillColor] (140.06,537.05) circle (  1.16);

\path[draw=drawColor,line width= 0.4pt,line join=round,line cap=round,fill=fillColor] (140.24,537.04) circle (  1.16);

\path[draw=drawColor,line width= 0.4pt,line join=round,line cap=round,fill=fillColor] (140.42,536.91) circle (  1.16);

\path[draw=drawColor,line width= 0.4pt,line join=round,line cap=round,fill=fillColor] (140.60,536.91) circle (  1.16);

\path[draw=drawColor,line width= 0.4pt,line join=round,line cap=round,fill=fillColor] (140.78,536.91) circle (  1.16);

\path[draw=drawColor,line width= 0.4pt,line join=round,line cap=round,fill=fillColor] (140.96,536.90) circle (  1.16);

\path[draw=drawColor,line width= 0.4pt,line join=round,line cap=round,fill=fillColor] (141.14,536.89) circle (  1.16);

\path[draw=drawColor,line width= 0.4pt,line join=round,line cap=round,fill=fillColor] (141.32,536.82) circle (  1.16);

\path[draw=drawColor,line width= 0.4pt,line join=round,line cap=round,fill=fillColor] (141.50,536.69) circle (  1.16);

\path[draw=drawColor,line width= 0.4pt,line join=round,line cap=round,fill=fillColor] (141.67,536.60) circle (  1.16);

\path[draw=drawColor,line width= 0.4pt,line join=round,line cap=round,fill=fillColor] (141.85,536.59) circle (  1.16);

\path[draw=drawColor,line width= 0.4pt,line join=round,line cap=round,fill=fillColor] (142.02,536.53) circle (  1.16);

\path[draw=drawColor,line width= 0.4pt,line join=round,line cap=round,fill=fillColor] (142.20,536.49) circle (  1.16);

\path[draw=drawColor,line width= 0.4pt,line join=round,line cap=round,fill=fillColor] (142.37,536.46) circle (  1.16);

\path[draw=drawColor,line width= 0.4pt,line join=round,line cap=round,fill=fillColor] (142.54,536.37) circle (  1.16);

\path[draw=drawColor,line width= 0.4pt,line join=round,line cap=round,fill=fillColor] (142.71,536.35) circle (  1.16);

\path[draw=drawColor,line width= 0.4pt,line join=round,line cap=round,fill=fillColor] (142.88,536.32) circle (  1.16);

\path[draw=drawColor,line width= 0.4pt,line join=round,line cap=round,fill=fillColor] (143.05,536.30) circle (  1.16);

\path[draw=drawColor,line width= 0.4pt,line join=round,line cap=round,fill=fillColor] (143.22,536.30) circle (  1.16);

\path[draw=drawColor,line width= 0.4pt,line join=round,line cap=round,fill=fillColor] (143.39,536.19) circle (  1.16);

\path[draw=drawColor,line width= 0.4pt,line join=round,line cap=round,fill=fillColor] (143.56,536.19) circle (  1.16);

\path[draw=drawColor,line width= 0.4pt,line join=round,line cap=round,fill=fillColor] (143.72,536.19) circle (  1.16);

\path[draw=drawColor,line width= 0.4pt,line join=round,line cap=round,fill=fillColor] (143.89,536.16) circle (  1.16);

\path[draw=drawColor,line width= 0.4pt,line join=round,line cap=round,fill=fillColor] (144.05,536.09) circle (  1.16);

\path[draw=drawColor,line width= 0.4pt,line join=round,line cap=round,fill=fillColor] (144.22,535.99) circle (  1.16);

\path[draw=drawColor,line width= 0.4pt,line join=round,line cap=round,fill=fillColor] (144.38,535.98) circle (  1.16);

\path[draw=drawColor,line width= 0.4pt,line join=round,line cap=round,fill=fillColor] (144.55,535.96) circle (  1.16);

\path[draw=drawColor,line width= 0.4pt,line join=round,line cap=round,fill=fillColor] (144.71,535.93) circle (  1.16);

\path[draw=drawColor,line width= 0.4pt,line join=round,line cap=round,fill=fillColor] (144.87,535.93) circle (  1.16);

\path[draw=drawColor,line width= 0.4pt,line join=round,line cap=round,fill=fillColor] (145.03,535.91) circle (  1.16);

\path[draw=drawColor,line width= 0.4pt,line join=round,line cap=round,fill=fillColor] (145.19,535.90) circle (  1.16);

\path[draw=drawColor,line width= 0.4pt,line join=round,line cap=round,fill=fillColor] (145.35,535.89) circle (  1.16);

\path[draw=drawColor,line width= 0.4pt,line join=round,line cap=round,fill=fillColor] (145.51,535.79) circle (  1.16);

\path[draw=drawColor,line width= 0.4pt,line join=round,line cap=round,fill=fillColor] (145.67,535.78) circle (  1.16);

\path[draw=drawColor,line width= 0.4pt,line join=round,line cap=round,fill=fillColor] (145.83,535.71) circle (  1.16);

\path[draw=drawColor,line width= 0.4pt,line join=round,line cap=round,fill=fillColor] (145.98,535.71) circle (  1.16);

\path[draw=drawColor,line width= 0.4pt,line join=round,line cap=round,fill=fillColor] (146.14,535.67) circle (  1.16);

\path[draw=drawColor,line width= 0.4pt,line join=round,line cap=round,fill=fillColor] (146.30,535.65) circle (  1.16);

\path[draw=drawColor,line width= 0.4pt,line join=round,line cap=round,fill=fillColor] (146.45,535.64) circle (  1.16);

\path[draw=drawColor,line width= 0.4pt,line join=round,line cap=round,fill=fillColor] (146.61,535.49) circle (  1.16);

\path[draw=drawColor,line width= 0.4pt,line join=round,line cap=round,fill=fillColor] (146.76,535.48) circle (  1.16);

\path[draw=drawColor,line width= 0.4pt,line join=round,line cap=round,fill=fillColor] (146.91,535.43) circle (  1.16);

\path[draw=drawColor,line width= 0.4pt,line join=round,line cap=round,fill=fillColor] (147.07,535.43) circle (  1.16);

\path[draw=drawColor,line width= 0.4pt,line join=round,line cap=round,fill=fillColor] (147.22,535.40) circle (  1.16);

\path[draw=drawColor,line width= 0.4pt,line join=round,line cap=round,fill=fillColor] (147.37,535.37) circle (  1.16);

\path[draw=drawColor,line width= 0.4pt,line join=round,line cap=round,fill=fillColor] (147.52,535.36) circle (  1.16);

\path[draw=drawColor,line width= 0.4pt,line join=round,line cap=round,fill=fillColor] (147.67,535.36) circle (  1.16);

\path[draw=drawColor,line width= 0.4pt,line join=round,line cap=round,fill=fillColor] (147.82,535.33) circle (  1.16);

\path[draw=drawColor,line width= 0.4pt,line join=round,line cap=round,fill=fillColor] (147.97,535.33) circle (  1.16);

\path[draw=drawColor,line width= 0.4pt,line join=round,line cap=round,fill=fillColor] (148.12,535.27) circle (  1.16);

\path[draw=drawColor,line width= 0.4pt,line join=round,line cap=round,fill=fillColor] (148.27,535.26) circle (  1.16);

\path[draw=drawColor,line width= 0.4pt,line join=round,line cap=round,fill=fillColor] (148.42,535.23) circle (  1.16);

\path[draw=drawColor,line width= 0.4pt,line join=round,line cap=round,fill=fillColor] (148.57,535.21) circle (  1.16);

\path[draw=drawColor,line width= 0.4pt,line join=round,line cap=round,fill=fillColor] (148.71,535.17) circle (  1.16);

\path[draw=drawColor,line width= 0.4pt,line join=round,line cap=round,fill=fillColor] (148.86,535.15) circle (  1.16);

\path[draw=drawColor,line width= 0.4pt,line join=round,line cap=round,fill=fillColor] (149.01,535.09) circle (  1.16);

\path[draw=drawColor,line width= 0.4pt,line join=round,line cap=round,fill=fillColor] (149.15,535.07) circle (  1.16);

\path[draw=drawColor,line width= 0.4pt,line join=round,line cap=round,fill=fillColor] (149.30,535.04) circle (  1.16);

\path[draw=drawColor,line width= 0.4pt,line join=round,line cap=round,fill=fillColor] (149.44,535.03) circle (  1.16);

\path[draw=drawColor,line width= 0.4pt,line join=round,line cap=round,fill=fillColor] (149.58,534.98) circle (  1.16);

\path[draw=drawColor,line width= 0.4pt,line join=round,line cap=round,fill=fillColor] (149.73,534.95) circle (  1.16);

\path[draw=drawColor,line width= 0.4pt,line join=round,line cap=round,fill=fillColor] (149.87,534.88) circle (  1.16);

\path[draw=drawColor,line width= 0.4pt,line join=round,line cap=round,fill=fillColor] (150.01,534.88) circle (  1.16);

\path[draw=drawColor,line width= 0.4pt,line join=round,line cap=round,fill=fillColor] (150.15,534.85) circle (  1.16);

\path[draw=drawColor,line width= 0.4pt,line join=round,line cap=round,fill=fillColor] (150.30,534.85) circle (  1.16);

\path[draw=drawColor,line width= 0.4pt,line join=round,line cap=round,fill=fillColor] (150.44,534.85) circle (  1.16);

\path[draw=drawColor,line width= 0.4pt,line join=round,line cap=round,fill=fillColor] (150.58,534.84) circle (  1.16);

\path[draw=drawColor,line width= 0.4pt,line join=round,line cap=round,fill=fillColor] (150.72,534.81) circle (  1.16);

\path[draw=drawColor,line width= 0.4pt,line join=round,line cap=round,fill=fillColor] (150.86,534.81) circle (  1.16);

\path[draw=drawColor,line width= 0.4pt,line join=round,line cap=round,fill=fillColor] (151.00,534.77) circle (  1.16);

\path[draw=drawColor,line width= 0.4pt,line join=round,line cap=round,fill=fillColor] (151.13,534.76) circle (  1.16);

\path[draw=drawColor,line width= 0.4pt,line join=round,line cap=round,fill=fillColor] (151.27,534.72) circle (  1.16);

\path[draw=drawColor,line width= 0.4pt,line join=round,line cap=round,fill=fillColor] (151.41,534.69) circle (  1.16);

\path[draw=drawColor,line width= 0.4pt,line join=round,line cap=round,fill=fillColor] (151.55,534.65) circle (  1.16);

\path[draw=drawColor,line width= 0.4pt,line join=round,line cap=round,fill=fillColor] (151.68,534.65) circle (  1.16);

\path[draw=drawColor,line width= 0.4pt,line join=round,line cap=round,fill=fillColor] (151.82,534.64) circle (  1.16);

\path[draw=drawColor,line width= 0.4pt,line join=round,line cap=round,fill=fillColor] (151.96,534.63) circle (  1.16);

\path[draw=drawColor,line width= 0.4pt,line join=round,line cap=round,fill=fillColor] (152.09,534.62) circle (  1.16);

\path[draw=drawColor,line width= 0.4pt,line join=round,line cap=round,fill=fillColor] (152.23,534.60) circle (  1.16);

\path[draw=drawColor,line width= 0.4pt,line join=round,line cap=round,fill=fillColor] (152.36,534.60) circle (  1.16);

\path[draw=drawColor,line width= 0.4pt,line join=round,line cap=round,fill=fillColor] (152.49,534.59) circle (  1.16);

\path[draw=drawColor,line width= 0.4pt,line join=round,line cap=round,fill=fillColor] (152.63,534.57) circle (  1.16);

\path[draw=drawColor,line width= 0.4pt,line join=round,line cap=round,fill=fillColor] (152.76,534.55) circle (  1.16);

\path[draw=drawColor,line width= 0.4pt,line join=round,line cap=round,fill=fillColor] (152.89,534.52) circle (  1.16);

\path[draw=drawColor,line width= 0.4pt,line join=round,line cap=round,fill=fillColor] (153.03,534.48) circle (  1.16);

\path[draw=drawColor,line width= 0.4pt,line join=round,line cap=round,fill=fillColor] (153.16,534.47) circle (  1.16);

\path[draw=drawColor,line width= 0.4pt,line join=round,line cap=round,fill=fillColor] (153.29,534.45) circle (  1.16);

\path[draw=drawColor,line width= 0.4pt,line join=round,line cap=round,fill=fillColor] (153.42,534.43) circle (  1.16);

\path[draw=drawColor,line width= 0.4pt,line join=round,line cap=round,fill=fillColor] (153.55,534.43) circle (  1.16);

\path[draw=drawColor,line width= 0.4pt,line join=round,line cap=round,fill=fillColor] (153.68,534.36) circle (  1.16);

\path[draw=drawColor,line width= 0.4pt,line join=round,line cap=round,fill=fillColor] (153.81,534.33) circle (  1.16);

\path[draw=drawColor,line width= 0.4pt,line join=round,line cap=round,fill=fillColor] (153.94,534.29) circle (  1.16);

\path[draw=drawColor,line width= 0.4pt,line join=round,line cap=round,fill=fillColor] (154.07,534.25) circle (  1.16);

\path[draw=drawColor,line width= 0.4pt,line join=round,line cap=round,fill=fillColor] (154.20,534.20) circle (  1.16);

\path[draw=drawColor,line width= 0.4pt,line join=round,line cap=round,fill=fillColor] (154.33,534.19) circle (  1.16);

\path[draw=drawColor,line width= 0.4pt,line join=round,line cap=round,fill=fillColor] (154.46,534.14) circle (  1.16);

\path[draw=drawColor,line width= 0.4pt,line join=round,line cap=round,fill=fillColor] (154.59,534.10) circle (  1.16);

\path[draw=drawColor,line width= 0.4pt,line join=round,line cap=round,fill=fillColor] (154.71,534.01) circle (  1.16);

\path[draw=drawColor,line width= 0.4pt,line join=round,line cap=round,fill=fillColor] (154.84,534.01) circle (  1.16);

\path[draw=drawColor,line width= 0.4pt,line join=round,line cap=round,fill=fillColor] (154.97,534.00) circle (  1.16);

\path[draw=drawColor,line width= 0.4pt,line join=round,line cap=round,fill=fillColor] (155.09,533.99) circle (  1.16);

\path[draw=drawColor,line width= 0.4pt,line join=round,line cap=round,fill=fillColor] (155.22,533.98) circle (  1.16);

\path[draw=drawColor,line width= 0.4pt,line join=round,line cap=round,fill=fillColor] (155.35,533.98) circle (  1.16);

\path[draw=drawColor,line width= 0.4pt,line join=round,line cap=round,fill=fillColor] (155.47,533.95) circle (  1.16);

\path[draw=drawColor,line width= 0.4pt,line join=round,line cap=round,fill=fillColor] (155.60,533.94) circle (  1.16);

\path[draw=drawColor,line width= 0.4pt,line join=round,line cap=round,fill=fillColor] (155.72,533.86) circle (  1.16);

\path[draw=drawColor,line width= 0.4pt,line join=round,line cap=round,fill=fillColor] (155.85,533.83) circle (  1.16);

\path[draw=drawColor,line width= 0.4pt,line join=round,line cap=round,fill=fillColor] (155.97,533.83) circle (  1.16);

\path[draw=drawColor,line width= 0.4pt,line join=round,line cap=round,fill=fillColor] (156.09,533.80) circle (  1.16);

\path[draw=drawColor,line width= 0.4pt,line join=round,line cap=round,fill=fillColor] (156.22,533.78) circle (  1.16);

\path[draw=drawColor,line width= 0.4pt,line join=round,line cap=round,fill=fillColor] (156.34,533.77) circle (  1.16);

\path[draw=drawColor,line width= 0.4pt,line join=round,line cap=round,fill=fillColor] (156.46,533.74) circle (  1.16);

\path[draw=drawColor,line width= 0.4pt,line join=round,line cap=round,fill=fillColor] (156.58,533.73) circle (  1.16);

\path[draw=drawColor,line width= 0.4pt,line join=round,line cap=round,fill=fillColor] (156.71,533.73) circle (  1.16);

\path[draw=drawColor,line width= 0.4pt,line join=round,line cap=round,fill=fillColor] (156.83,533.70) circle (  1.16);

\path[draw=drawColor,line width= 0.4pt,line join=round,line cap=round,fill=fillColor] (156.95,533.70) circle (  1.16);

\path[draw=drawColor,line width= 0.4pt,line join=round,line cap=round,fill=fillColor] (157.07,533.59) circle (  1.16);

\path[draw=drawColor,line width= 0.4pt,line join=round,line cap=round,fill=fillColor] (157.19,533.52) circle (  1.16);

\path[draw=drawColor,line width= 0.4pt,line join=round,line cap=round,fill=fillColor] (157.31,533.51) circle (  1.16);

\path[draw=drawColor,line width= 0.4pt,line join=round,line cap=round,fill=fillColor] (157.43,533.51) circle (  1.16);

\path[draw=drawColor,line width= 0.4pt,line join=round,line cap=round,fill=fillColor] (157.55,533.51) circle (  1.16);

\path[draw=drawColor,line width= 0.4pt,line join=round,line cap=round,fill=fillColor] (157.67,533.49) circle (  1.16);

\path[draw=drawColor,line width= 0.4pt,line join=round,line cap=round,fill=fillColor] (157.79,533.44) circle (  1.16);

\path[draw=drawColor,line width= 0.4pt,line join=round,line cap=round,fill=fillColor] (157.91,533.43) circle (  1.16);

\path[draw=drawColor,line width= 0.4pt,line join=round,line cap=round,fill=fillColor] (158.02,533.40) circle (  1.16);

\path[draw=drawColor,line width= 0.4pt,line join=round,line cap=round,fill=fillColor] (158.14,533.40) circle (  1.16);

\path[draw=drawColor,line width= 0.4pt,line join=round,line cap=round,fill=fillColor] (158.26,533.38) circle (  1.16);

\path[draw=drawColor,line width= 0.4pt,line join=round,line cap=round,fill=fillColor] (158.38,533.36) circle (  1.16);

\path[draw=drawColor,line width= 0.4pt,line join=round,line cap=round,fill=fillColor] (158.50,533.35) circle (  1.16);

\path[draw=drawColor,line width= 0.4pt,line join=round,line cap=round,fill=fillColor] (158.61,533.35) circle (  1.16);

\path[draw=drawColor,line width= 0.4pt,line join=round,line cap=round,fill=fillColor] (158.73,533.31) circle (  1.16);

\path[draw=drawColor,line width= 0.4pt,line join=round,line cap=round,fill=fillColor] (158.84,533.30) circle (  1.16);

\path[draw=drawColor,line width= 0.4pt,line join=round,line cap=round,fill=fillColor] (158.96,533.25) circle (  1.16);

\path[draw=drawColor,line width= 0.4pt,line join=round,line cap=round,fill=fillColor] (159.08,533.24) circle (  1.16);

\path[draw=drawColor,line width= 0.4pt,line join=round,line cap=round,fill=fillColor] (159.19,533.13) circle (  1.16);

\path[draw=drawColor,line width= 0.4pt,line join=round,line cap=round,fill=fillColor] (159.31,533.12) circle (  1.16);

\path[draw=drawColor,line width= 0.4pt,line join=round,line cap=round,fill=fillColor] (159.42,533.10) circle (  1.16);

\path[draw=drawColor,line width= 0.4pt,line join=round,line cap=round,fill=fillColor] (159.54,533.09) circle (  1.16);

\path[draw=drawColor,line width= 0.4pt,line join=round,line cap=round,fill=fillColor] (159.65,533.03) circle (  1.16);

\path[draw=drawColor,line width= 0.4pt,line join=round,line cap=round,fill=fillColor] (159.76,533.02) circle (  1.16);

\path[draw=drawColor,line width= 0.4pt,line join=round,line cap=round,fill=fillColor] (159.88,533.02) circle (  1.16);

\path[draw=drawColor,line width= 0.4pt,line join=round,line cap=round,fill=fillColor] (159.99,532.98) circle (  1.16);

\path[draw=drawColor,line width= 0.4pt,line join=round,line cap=round,fill=fillColor] (160.10,532.97) circle (  1.16);

\path[draw=drawColor,line width= 0.4pt,line join=round,line cap=round,fill=fillColor] (160.22,532.96) circle (  1.16);

\path[draw=drawColor,line width= 0.4pt,line join=round,line cap=round,fill=fillColor] (160.33,532.95) circle (  1.16);

\path[draw=drawColor,line width= 0.4pt,line join=round,line cap=round,fill=fillColor] (160.44,532.95) circle (  1.16);

\path[draw=drawColor,line width= 0.4pt,line join=round,line cap=round,fill=fillColor] (160.55,532.95) circle (  1.16);

\path[draw=drawColor,line width= 0.4pt,line join=round,line cap=round,fill=fillColor] (160.67,532.95) circle (  1.16);

\path[draw=drawColor,line width= 0.4pt,line join=round,line cap=round,fill=fillColor] (160.78,532.93) circle (  1.16);

\path[draw=drawColor,line width= 0.4pt,line join=round,line cap=round,fill=fillColor] (160.89,532.93) circle (  1.16);

\path[draw=drawColor,line width= 0.4pt,line join=round,line cap=round,fill=fillColor] (161.00,532.93) circle (  1.16);

\path[draw=drawColor,line width= 0.4pt,line join=round,line cap=round,fill=fillColor] (161.11,532.92) circle (  1.16);

\path[draw=drawColor,line width= 0.4pt,line join=round,line cap=round,fill=fillColor] (161.22,532.88) circle (  1.16);

\path[draw=drawColor,line width= 0.4pt,line join=round,line cap=round,fill=fillColor] (161.33,532.88) circle (  1.16);

\path[draw=drawColor,line width= 0.4pt,line join=round,line cap=round,fill=fillColor] (161.44,532.87) circle (  1.16);

\path[draw=drawColor,line width= 0.4pt,line join=round,line cap=round,fill=fillColor] (161.55,532.82) circle (  1.16);

\path[draw=drawColor,line width= 0.4pt,line join=round,line cap=round,fill=fillColor] (161.66,532.81) circle (  1.16);

\path[draw=drawColor,line width= 0.4pt,line join=round,line cap=round,fill=fillColor] (161.77,532.79) circle (  1.16);

\path[draw=drawColor,line width= 0.4pt,line join=round,line cap=round,fill=fillColor] (161.88,532.79) circle (  1.16);

\path[draw=drawColor,line width= 0.4pt,line join=round,line cap=round,fill=fillColor] (161.99,532.78) circle (  1.16);

\path[draw=drawColor,line width= 0.4pt,line join=round,line cap=round,fill=fillColor] (162.10,532.77) circle (  1.16);

\path[draw=drawColor,line width= 0.4pt,line join=round,line cap=round,fill=fillColor] (162.20,532.75) circle (  1.16);

\path[draw=drawColor,line width= 0.4pt,line join=round,line cap=round,fill=fillColor] (162.31,532.67) circle (  1.16);

\path[draw=drawColor,line width= 0.4pt,line join=round,line cap=round,fill=fillColor] (162.42,532.67) circle (  1.16);

\path[draw=drawColor,line width= 0.4pt,line join=round,line cap=round,fill=fillColor] (162.53,532.67) circle (  1.16);

\path[draw=drawColor,line width= 0.4pt,line join=round,line cap=round,fill=fillColor] (162.63,532.65) circle (  1.16);

\path[draw=drawColor,line width= 0.4pt,line join=round,line cap=round,fill=fillColor] (162.74,532.65) circle (  1.16);

\path[draw=drawColor,line width= 0.4pt,line join=round,line cap=round,fill=fillColor] (162.85,532.65) circle (  1.16);

\path[draw=drawColor,line width= 0.4pt,line join=round,line cap=round,fill=fillColor] (162.95,532.62) circle (  1.16);

\path[draw=drawColor,line width= 0.4pt,line join=round,line cap=round,fill=fillColor] (163.06,532.62) circle (  1.16);

\path[draw=drawColor,line width= 0.4pt,line join=round,line cap=round,fill=fillColor] (163.17,532.61) circle (  1.16);

\path[draw=drawColor,line width= 0.4pt,line join=round,line cap=round,fill=fillColor] (163.27,532.60) circle (  1.16);

\path[draw=drawColor,line width= 0.4pt,line join=round,line cap=round,fill=fillColor] (163.38,532.60) circle (  1.16);

\path[draw=drawColor,line width= 0.4pt,line join=round,line cap=round,fill=fillColor] (163.48,532.56) circle (  1.16);

\path[draw=drawColor,line width= 0.4pt,line join=round,line cap=round,fill=fillColor] (163.59,532.55) circle (  1.16);

\path[draw=drawColor,line width= 0.4pt,line join=round,line cap=round,fill=fillColor] (163.69,532.55) circle (  1.16);

\path[draw=drawColor,line width= 0.4pt,line join=round,line cap=round,fill=fillColor] (163.80,532.52) circle (  1.16);

\path[draw=drawColor,line width= 0.4pt,line join=round,line cap=round,fill=fillColor] (163.90,532.51) circle (  1.16);

\path[draw=drawColor,line width= 0.4pt,line join=round,line cap=round,fill=fillColor] (164.01,532.50) circle (  1.16);

\path[draw=drawColor,line width= 0.4pt,line join=round,line cap=round,fill=fillColor] (164.11,532.50) circle (  1.16);

\path[draw=drawColor,line width= 0.4pt,line join=round,line cap=round,fill=fillColor] (164.21,532.36) circle (  1.16);

\path[draw=drawColor,line width= 0.4pt,line join=round,line cap=round,fill=fillColor] (164.32,532.34) circle (  1.16);

\path[draw=drawColor,line width= 0.4pt,line join=round,line cap=round,fill=fillColor] (164.42,532.34) circle (  1.16);

\path[draw=drawColor,line width= 0.4pt,line join=round,line cap=round,fill=fillColor] (164.52,532.31) circle (  1.16);

\path[draw=drawColor,line width= 0.4pt,line join=round,line cap=round,fill=fillColor] (164.63,532.31) circle (  1.16);

\path[draw=drawColor,line width= 0.4pt,line join=round,line cap=round,fill=fillColor] (164.73,532.29) circle (  1.16);

\path[draw=drawColor,line width= 0.4pt,line join=round,line cap=round,fill=fillColor] (164.83,532.27) circle (  1.16);

\path[draw=drawColor,line width= 0.4pt,line join=round,line cap=round,fill=fillColor] (164.93,532.22) circle (  1.16);

\path[draw=drawColor,line width= 0.4pt,line join=round,line cap=round,fill=fillColor] (165.04,532.22) circle (  1.16);

\path[draw=drawColor,line width= 0.4pt,line join=round,line cap=round,fill=fillColor] (165.14,532.22) circle (  1.16);

\path[draw=drawColor,line width= 0.4pt,line join=round,line cap=round,fill=fillColor] (165.24,532.21) circle (  1.16);

\path[draw=drawColor,line width= 0.4pt,line join=round,line cap=round,fill=fillColor] (165.34,532.21) circle (  1.16);

\path[draw=drawColor,line width= 0.4pt,line join=round,line cap=round,fill=fillColor] (165.44,532.21) circle (  1.16);

\path[draw=drawColor,line width= 0.4pt,line join=round,line cap=round,fill=fillColor] (165.54,532.21) circle (  1.16);

\path[draw=drawColor,line width= 0.4pt,line join=round,line cap=round,fill=fillColor] (165.64,532.20) circle (  1.16);

\path[draw=drawColor,line width= 0.4pt,line join=round,line cap=round,fill=fillColor] (165.74,532.17) circle (  1.16);

\path[draw=drawColor,line width= 0.4pt,line join=round,line cap=round,fill=fillColor] (165.85,532.15) circle (  1.16);

\path[draw=drawColor,line width= 0.4pt,line join=round,line cap=round,fill=fillColor] (165.95,532.15) circle (  1.16);

\path[draw=drawColor,line width= 0.4pt,line join=round,line cap=round,fill=fillColor] (166.05,532.13) circle (  1.16);

\path[draw=drawColor,line width= 0.4pt,line join=round,line cap=round,fill=fillColor] (166.14,532.09) circle (  1.16);

\path[draw=drawColor,line width= 0.4pt,line join=round,line cap=round,fill=fillColor] (166.24,532.07) circle (  1.16);

\path[draw=drawColor,line width= 0.4pt,line join=round,line cap=round,fill=fillColor] (166.34,532.05) circle (  1.16);

\path[draw=drawColor,line width= 0.4pt,line join=round,line cap=round,fill=fillColor] (166.44,532.05) circle (  1.16);

\path[draw=drawColor,line width= 0.4pt,line join=round,line cap=round,fill=fillColor] (166.54,532.03) circle (  1.16);

\path[draw=drawColor,line width= 0.4pt,line join=round,line cap=round,fill=fillColor] (166.64,532.02) circle (  1.16);

\path[draw=drawColor,line width= 0.4pt,line join=round,line cap=round,fill=fillColor] (166.74,531.99) circle (  1.16);

\path[draw=drawColor,line width= 0.4pt,line join=round,line cap=round,fill=fillColor] (166.84,531.98) circle (  1.16);

\path[draw=drawColor,line width= 0.4pt,line join=round,line cap=round,fill=fillColor] (166.94,531.98) circle (  1.16);

\path[draw=drawColor,line width= 0.4pt,line join=round,line cap=round,fill=fillColor] (167.03,531.97) circle (  1.16);

\path[draw=drawColor,line width= 0.4pt,line join=round,line cap=round,fill=fillColor] (167.13,531.92) circle (  1.16);

\path[draw=drawColor,line width= 0.4pt,line join=round,line cap=round,fill=fillColor] (167.23,531.90) circle (  1.16);

\path[draw=drawColor,line width= 0.4pt,line join=round,line cap=round,fill=fillColor] (167.33,531.89) circle (  1.16);

\path[draw=drawColor,line width= 0.4pt,line join=round,line cap=round,fill=fillColor] (167.42,531.89) circle (  1.16);

\path[draw=drawColor,line width= 0.4pt,line join=round,line cap=round,fill=fillColor] (167.52,531.89) circle (  1.16);

\path[draw=drawColor,line width= 0.4pt,line join=round,line cap=round,fill=fillColor] (167.62,531.87) circle (  1.16);

\path[draw=drawColor,line width= 0.4pt,line join=round,line cap=round,fill=fillColor] (167.71,531.86) circle (  1.16);

\path[draw=drawColor,line width= 0.4pt,line join=round,line cap=round,fill=fillColor] (167.81,531.85) circle (  1.16);

\path[draw=drawColor,line width= 0.4pt,line join=round,line cap=round,fill=fillColor] (167.91,531.83) circle (  1.16);

\path[draw=drawColor,line width= 0.4pt,line join=round,line cap=round,fill=fillColor] (168.00,531.81) circle (  1.16);

\path[draw=drawColor,line width= 0.4pt,line join=round,line cap=round,fill=fillColor] (168.10,531.80) circle (  1.16);

\path[draw=drawColor,line width= 0.4pt,line join=round,line cap=round,fill=fillColor] (168.20,531.79) circle (  1.16);

\path[draw=drawColor,line width= 0.4pt,line join=round,line cap=round,fill=fillColor] (168.29,531.79) circle (  1.16);

\path[draw=drawColor,line width= 0.4pt,line join=round,line cap=round,fill=fillColor] (168.39,531.74) circle (  1.16);

\path[draw=drawColor,line width= 0.4pt,line join=round,line cap=round,fill=fillColor] (168.48,531.72) circle (  1.16);

\path[draw=drawColor,line width= 0.4pt,line join=round,line cap=round,fill=fillColor] (168.58,531.69) circle (  1.16);

\path[draw=drawColor,line width= 0.4pt,line join=round,line cap=round,fill=fillColor] (168.67,531.68) circle (  1.16);

\path[draw=drawColor,line width= 0.4pt,line join=round,line cap=round,fill=fillColor] (168.77,531.64) circle (  1.16);

\path[draw=drawColor,line width= 0.4pt,line join=round,line cap=round,fill=fillColor] (168.86,531.61) circle (  1.16);

\path[draw=drawColor,line width= 0.4pt,line join=round,line cap=round,fill=fillColor] (168.95,531.58) circle (  1.16);

\path[draw=drawColor,line width= 0.4pt,line join=round,line cap=round,fill=fillColor] (169.05,531.58) circle (  1.16);

\path[draw=drawColor,line width= 0.4pt,line join=round,line cap=round,fill=fillColor] (169.14,531.58) circle (  1.16);

\path[draw=drawColor,line width= 0.4pt,line join=round,line cap=round,fill=fillColor] (169.24,531.57) circle (  1.16);

\path[draw=drawColor,line width= 0.4pt,line join=round,line cap=round,fill=fillColor] (169.33,531.56) circle (  1.16);

\path[draw=drawColor,line width= 0.4pt,line join=round,line cap=round,fill=fillColor] (169.42,531.54) circle (  1.16);

\path[draw=drawColor,line width= 0.4pt,line join=round,line cap=round,fill=fillColor] (169.52,531.54) circle (  1.16);

\path[draw=drawColor,line width= 0.4pt,line join=round,line cap=round,fill=fillColor] (169.61,531.52) circle (  1.16);

\path[draw=drawColor,line width= 0.4pt,line join=round,line cap=round,fill=fillColor] (169.70,531.49) circle (  1.16);

\path[draw=drawColor,line width= 0.4pt,line join=round,line cap=round,fill=fillColor] (169.80,531.45) circle (  1.16);

\path[draw=drawColor,line width= 0.4pt,line join=round,line cap=round,fill=fillColor] (169.89,531.45) circle (  1.16);

\path[draw=drawColor,line width= 0.4pt,line join=round,line cap=round,fill=fillColor] (169.98,531.44) circle (  1.16);

\path[draw=drawColor,line width= 0.4pt,line join=round,line cap=round,fill=fillColor] (170.07,531.44) circle (  1.16);

\path[draw=drawColor,line width= 0.4pt,line join=round,line cap=round,fill=fillColor] (170.17,531.42) circle (  1.16);

\path[draw=drawColor,line width= 0.4pt,line join=round,line cap=round,fill=fillColor] (170.26,531.41) circle (  1.16);

\path[draw=drawColor,line width= 0.4pt,line join=round,line cap=round,fill=fillColor] (170.35,531.41) circle (  1.16);

\path[draw=drawColor,line width= 0.4pt,line join=round,line cap=round,fill=fillColor] (170.44,531.40) circle (  1.16);

\path[draw=drawColor,line width= 0.4pt,line join=round,line cap=round,fill=fillColor] (170.53,531.37) circle (  1.16);

\path[draw=drawColor,line width= 0.4pt,line join=round,line cap=round,fill=fillColor] (170.62,531.37) circle (  1.16);

\path[draw=drawColor,line width= 0.4pt,line join=round,line cap=round,fill=fillColor] (170.71,531.35) circle (  1.16);

\path[draw=drawColor,line width= 0.4pt,line join=round,line cap=round,fill=fillColor] (170.81,531.33) circle (  1.16);

\path[draw=drawColor,line width= 0.4pt,line join=round,line cap=round,fill=fillColor] (170.90,531.32) circle (  1.16);

\path[draw=drawColor,line width= 0.4pt,line join=round,line cap=round,fill=fillColor] (170.99,531.32) circle (  1.16);

\path[draw=drawColor,line width= 0.4pt,line join=round,line cap=round,fill=fillColor] (171.08,531.30) circle (  1.16);

\path[draw=drawColor,line width= 0.4pt,line join=round,line cap=round,fill=fillColor] (171.17,531.29) circle (  1.16);

\path[draw=drawColor,line width= 0.4pt,line join=round,line cap=round,fill=fillColor] (171.26,531.28) circle (  1.16);

\path[draw=drawColor,line width= 0.4pt,line join=round,line cap=round,fill=fillColor] (171.35,531.27) circle (  1.16);

\path[draw=drawColor,line width= 0.4pt,line join=round,line cap=round,fill=fillColor] (171.44,531.27) circle (  1.16);

\path[draw=drawColor,line width= 0.4pt,line join=round,line cap=round,fill=fillColor] (171.53,531.23) circle (  1.16);

\path[draw=drawColor,line width= 0.4pt,line join=round,line cap=round,fill=fillColor] (171.62,531.22) circle (  1.16);

\path[draw=drawColor,line width= 0.4pt,line join=round,line cap=round,fill=fillColor] (171.71,531.21) circle (  1.16);

\path[draw=drawColor,line width= 0.4pt,line join=round,line cap=round,fill=fillColor] (171.80,531.18) circle (  1.16);

\path[draw=drawColor,line width= 0.4pt,line join=round,line cap=round,fill=fillColor] (171.89,531.16) circle (  1.16);

\path[draw=drawColor,line width= 0.4pt,line join=round,line cap=round,fill=fillColor] (171.97,531.12) circle (  1.16);

\path[draw=drawColor,line width= 0.4pt,line join=round,line cap=round,fill=fillColor] (172.06,531.11) circle (  1.16);

\path[draw=drawColor,line width= 0.4pt,line join=round,line cap=round,fill=fillColor] (172.15,531.10) circle (  1.16);

\path[draw=drawColor,line width= 0.4pt,line join=round,line cap=round,fill=fillColor] (172.24,531.09) circle (  1.16);

\path[draw=drawColor,line width= 0.4pt,line join=round,line cap=round,fill=fillColor] (172.33,531.09) circle (  1.16);

\path[draw=drawColor,line width= 0.4pt,line join=round,line cap=round,fill=fillColor] (172.42,531.08) circle (  1.16);

\path[draw=drawColor,line width= 0.4pt,line join=round,line cap=round,fill=fillColor] (172.51,531.07) circle (  1.16);

\path[draw=drawColor,line width= 0.4pt,line join=round,line cap=round,fill=fillColor] (172.59,531.07) circle (  1.16);

\path[draw=drawColor,line width= 0.4pt,line join=round,line cap=round,fill=fillColor] (172.68,531.05) circle (  1.16);

\path[draw=drawColor,line width= 0.4pt,line join=round,line cap=round,fill=fillColor] (172.77,531.03) circle (  1.16);

\path[draw=drawColor,line width= 0.4pt,line join=round,line cap=round,fill=fillColor] (172.86,531.02) circle (  1.16);

\path[draw=drawColor,line width= 0.4pt,line join=round,line cap=round,fill=fillColor] (172.94,530.96) circle (  1.16);

\path[draw=drawColor,line width= 0.4pt,line join=round,line cap=round,fill=fillColor] (173.03,530.91) circle (  1.16);

\path[draw=drawColor,line width= 0.4pt,line join=round,line cap=round,fill=fillColor] (173.12,530.90) circle (  1.16);

\path[draw=drawColor,line width= 0.4pt,line join=round,line cap=round,fill=fillColor] (173.20,530.90) circle (  1.16);

\path[draw=drawColor,line width= 0.4pt,line join=round,line cap=round,fill=fillColor] (173.29,530.89) circle (  1.16);

\path[draw=drawColor,line width= 0.4pt,line join=round,line cap=round,fill=fillColor] (173.38,530.89) circle (  1.16);

\path[draw=drawColor,line width= 0.4pt,line join=round,line cap=round,fill=fillColor] (173.46,530.88) circle (  1.16);

\path[draw=drawColor,line width= 0.4pt,line join=round,line cap=round,fill=fillColor] (173.55,530.86) circle (  1.16);

\path[draw=drawColor,line width= 0.4pt,line join=round,line cap=round,fill=fillColor] (173.64,530.86) circle (  1.16);

\path[draw=drawColor,line width= 0.4pt,line join=round,line cap=round,fill=fillColor] (173.72,530.83) circle (  1.16);

\path[draw=drawColor,line width= 0.4pt,line join=round,line cap=round,fill=fillColor] (173.81,530.83) circle (  1.16);

\path[draw=drawColor,line width= 0.4pt,line join=round,line cap=round,fill=fillColor] (173.89,530.81) circle (  1.16);

\path[draw=drawColor,line width= 0.4pt,line join=round,line cap=round,fill=fillColor] (173.98,530.77) circle (  1.16);

\path[draw=drawColor,line width= 0.4pt,line join=round,line cap=round,fill=fillColor] (174.07,530.75) circle (  1.16);

\path[draw=drawColor,line width= 0.4pt,line join=round,line cap=round,fill=fillColor] (174.15,530.74) circle (  1.16);

\path[draw=drawColor,line width= 0.4pt,line join=round,line cap=round,fill=fillColor] (174.24,530.73) circle (  1.16);

\path[draw=drawColor,line width= 0.4pt,line join=round,line cap=round,fill=fillColor] (174.32,530.70) circle (  1.16);

\path[draw=drawColor,line width= 0.4pt,line join=round,line cap=round,fill=fillColor] (174.41,530.69) circle (  1.16);

\path[draw=drawColor,line width= 0.4pt,line join=round,line cap=round,fill=fillColor] (174.49,530.67) circle (  1.16);

\path[draw=drawColor,line width= 0.4pt,line join=round,line cap=round,fill=fillColor] (174.58,530.63) circle (  1.16);

\path[draw=drawColor,line width= 0.4pt,line join=round,line cap=round,fill=fillColor] (174.66,530.62) circle (  1.16);

\path[draw=drawColor,line width= 0.4pt,line join=round,line cap=round,fill=fillColor] (174.75,530.59) circle (  1.16);

\path[draw=drawColor,line width= 0.4pt,line join=round,line cap=round,fill=fillColor] (174.83,530.55) circle (  1.16);

\path[draw=drawColor,line width= 0.4pt,line join=round,line cap=round,fill=fillColor] (174.91,530.54) circle (  1.16);

\path[draw=drawColor,line width= 0.4pt,line join=round,line cap=round,fill=fillColor] (175.00,530.53) circle (  1.16);

\path[draw=drawColor,line width= 0.4pt,line join=round,line cap=round,fill=fillColor] (175.08,530.53) circle (  1.16);

\path[draw=drawColor,line width= 0.4pt,line join=round,line cap=round,fill=fillColor] (175.17,530.52) circle (  1.16);

\path[draw=drawColor,line width= 0.4pt,line join=round,line cap=round,fill=fillColor] (175.25,530.50) circle (  1.16);

\path[draw=drawColor,line width= 0.4pt,line join=round,line cap=round,fill=fillColor] (175.33,530.49) circle (  1.16);

\path[draw=drawColor,line width= 0.4pt,line join=round,line cap=round,fill=fillColor] (175.42,530.49) circle (  1.16);

\path[draw=drawColor,line width= 0.4pt,line join=round,line cap=round,fill=fillColor] (175.50,530.48) circle (  1.16);

\path[draw=drawColor,line width= 0.4pt,line join=round,line cap=round,fill=fillColor] (175.58,530.46) circle (  1.16);

\path[draw=drawColor,line width= 0.4pt,line join=round,line cap=round,fill=fillColor] (175.67,530.41) circle (  1.16);

\path[draw=drawColor,line width= 0.4pt,line join=round,line cap=round,fill=fillColor] (175.75,530.40) circle (  1.16);

\path[draw=drawColor,line width= 0.4pt,line join=round,line cap=round,fill=fillColor] (175.83,530.39) circle (  1.16);

\path[draw=drawColor,line width= 0.4pt,line join=round,line cap=round,fill=fillColor] (175.91,530.36) circle (  1.16);

\path[draw=drawColor,line width= 0.4pt,line join=round,line cap=round,fill=fillColor] (176.00,530.36) circle (  1.16);

\path[draw=drawColor,line width= 0.4pt,line join=round,line cap=round,fill=fillColor] (176.08,530.36) circle (  1.16);

\path[draw=drawColor,line width= 0.4pt,line join=round,line cap=round,fill=fillColor] (176.16,530.36) circle (  1.16);

\path[draw=drawColor,line width= 0.4pt,line join=round,line cap=round,fill=fillColor] (176.24,530.35) circle (  1.16);

\path[draw=drawColor,line width= 0.4pt,line join=round,line cap=round,fill=fillColor] (176.33,530.31) circle (  1.16);

\path[draw=drawColor,line width= 0.4pt,line join=round,line cap=round,fill=fillColor] (176.41,530.29) circle (  1.16);

\path[draw=drawColor,line width= 0.4pt,line join=round,line cap=round,fill=fillColor] (176.49,530.27) circle (  1.16);

\path[draw=drawColor,line width= 0.4pt,line join=round,line cap=round,fill=fillColor] (176.57,530.20) circle (  1.16);

\path[draw=drawColor,line width= 0.4pt,line join=round,line cap=round,fill=fillColor] (176.65,530.20) circle (  1.16);

\path[draw=drawColor,line width= 0.4pt,line join=round,line cap=round,fill=fillColor] (176.73,530.19) circle (  1.16);

\path[draw=drawColor,line width= 0.4pt,line join=round,line cap=round,fill=fillColor] (176.82,530.15) circle (  1.16);

\path[draw=drawColor,line width= 0.4pt,line join=round,line cap=round,fill=fillColor] (176.90,530.13) circle (  1.16);

\path[draw=drawColor,line width= 0.4pt,line join=round,line cap=round,fill=fillColor] (176.98,530.11) circle (  1.16);

\path[draw=drawColor,line width= 0.4pt,line join=round,line cap=round,fill=fillColor] (177.06,530.08) circle (  1.16);

\path[draw=drawColor,line width= 0.4pt,line join=round,line cap=round,fill=fillColor] (177.14,530.08) circle (  1.16);

\path[draw=drawColor,line width= 0.4pt,line join=round,line cap=round,fill=fillColor] (177.22,530.07) circle (  1.16);

\path[draw=drawColor,line width= 0.4pt,line join=round,line cap=round,fill=fillColor] (177.30,530.07) circle (  1.16);

\path[draw=drawColor,line width= 0.4pt,line join=round,line cap=round,fill=fillColor] (177.38,530.07) circle (  1.16);

\path[draw=drawColor,line width= 0.4pt,line join=round,line cap=round,fill=fillColor] (177.46,530.07) circle (  1.16);

\path[draw=drawColor,line width= 0.4pt,line join=round,line cap=round,fill=fillColor] (177.54,530.06) circle (  1.16);

\path[draw=drawColor,line width= 0.4pt,line join=round,line cap=round,fill=fillColor] (177.62,530.06) circle (  1.16);

\path[draw=drawColor,line width= 0.4pt,line join=round,line cap=round,fill=fillColor] (177.70,529.98) circle (  1.16);

\path[draw=drawColor,line width= 0.4pt,line join=round,line cap=round,fill=fillColor] (177.78,529.97) circle (  1.16);

\path[draw=drawColor,line width= 0.4pt,line join=round,line cap=round,fill=fillColor] (177.86,529.96) circle (  1.16);

\path[draw=drawColor,line width= 0.4pt,line join=round,line cap=round,fill=fillColor] (177.94,529.93) circle (  1.16);

\path[draw=drawColor,line width= 0.4pt,line join=round,line cap=round,fill=fillColor] (178.02,529.84) circle (  1.16);

\path[draw=drawColor,line width= 0.4pt,line join=round,line cap=round,fill=fillColor] (178.10,529.80) circle (  1.16);

\path[draw=drawColor,line width= 0.4pt,line join=round,line cap=round,fill=fillColor] (178.18,529.77) circle (  1.16);

\path[draw=drawColor,line width= 0.4pt,line join=round,line cap=round,fill=fillColor] (178.26,529.76) circle (  1.16);

\path[draw=drawColor,line width= 0.4pt,line join=round,line cap=round,fill=fillColor] (178.34,529.75) circle (  1.16);

\path[draw=drawColor,line width= 0.4pt,line join=round,line cap=round,fill=fillColor] (178.42,529.73) circle (  1.16);

\path[draw=drawColor,line width= 0.4pt,line join=round,line cap=round,fill=fillColor] (178.50,529.73) circle (  1.16);

\path[draw=drawColor,line width= 0.4pt,line join=round,line cap=round,fill=fillColor] (178.57,529.72) circle (  1.16);

\path[draw=drawColor,line width= 0.4pt,line join=round,line cap=round,fill=fillColor] (178.65,529.71) circle (  1.16);

\path[draw=drawColor,line width= 0.4pt,line join=round,line cap=round,fill=fillColor] (178.73,529.70) circle (  1.16);

\path[draw=drawColor,line width= 0.4pt,line join=round,line cap=round,fill=fillColor] (178.81,529.69) circle (  1.16);

\path[draw=drawColor,line width= 0.4pt,line join=round,line cap=round,fill=fillColor] (178.89,529.68) circle (  1.16);

\path[draw=drawColor,line width= 0.4pt,line join=round,line cap=round,fill=fillColor] (178.97,529.67) circle (  1.16);

\path[draw=drawColor,line width= 0.4pt,line join=round,line cap=round,fill=fillColor] (179.04,529.66) circle (  1.16);

\path[draw=drawColor,line width= 0.4pt,line join=round,line cap=round,fill=fillColor] (179.12,529.63) circle (  1.16);

\path[draw=drawColor,line width= 0.4pt,line join=round,line cap=round,fill=fillColor] (179.20,529.63) circle (  1.16);

\path[draw=drawColor,line width= 0.4pt,line join=round,line cap=round,fill=fillColor] (179.28,529.63) circle (  1.16);

\path[draw=drawColor,line width= 0.4pt,line join=round,line cap=round,fill=fillColor] (179.36,529.62) circle (  1.16);

\path[draw=drawColor,line width= 0.4pt,line join=round,line cap=round,fill=fillColor] (179.43,529.62) circle (  1.16);

\path[draw=drawColor,line width= 0.4pt,line join=round,line cap=round,fill=fillColor] (179.51,529.62) circle (  1.16);

\path[draw=drawColor,line width= 0.4pt,line join=round,line cap=round,fill=fillColor] (179.59,529.60) circle (  1.16);

\path[draw=drawColor,line width= 0.4pt,line join=round,line cap=round,fill=fillColor] (179.67,529.58) circle (  1.16);

\path[draw=drawColor,line width= 0.4pt,line join=round,line cap=round,fill=fillColor] (179.74,529.56) circle (  1.16);

\path[draw=drawColor,line width= 0.4pt,line join=round,line cap=round,fill=fillColor] (179.82,529.53) circle (  1.16);

\path[draw=drawColor,line width= 0.4pt,line join=round,line cap=round,fill=fillColor] (179.90,529.51) circle (  1.16);

\path[draw=drawColor,line width= 0.4pt,line join=round,line cap=round,fill=fillColor] (179.97,529.49) circle (  1.16);

\path[draw=drawColor,line width= 0.4pt,line join=round,line cap=round,fill=fillColor] (180.05,529.49) circle (  1.16);

\path[draw=drawColor,line width= 0.4pt,line join=round,line cap=round,fill=fillColor] (180.13,529.48) circle (  1.16);

\path[draw=drawColor,line width= 0.4pt,line join=round,line cap=round,fill=fillColor] (180.20,529.48) circle (  1.16);

\path[draw=drawColor,line width= 0.4pt,line join=round,line cap=round,fill=fillColor] (180.28,529.48) circle (  1.16);

\path[draw=drawColor,line width= 0.4pt,line join=round,line cap=round,fill=fillColor] (180.36,529.45) circle (  1.16);

\path[draw=drawColor,line width= 0.4pt,line join=round,line cap=round,fill=fillColor] (180.43,529.45) circle (  1.16);

\path[draw=drawColor,line width= 0.4pt,line join=round,line cap=round,fill=fillColor] (180.51,529.45) circle (  1.16);

\path[draw=drawColor,line width= 0.4pt,line join=round,line cap=round,fill=fillColor] (180.58,529.44) circle (  1.16);

\path[draw=drawColor,line width= 0.4pt,line join=round,line cap=round,fill=fillColor] (180.66,529.39) circle (  1.16);

\path[draw=drawColor,line width= 0.4pt,line join=round,line cap=round,fill=fillColor] (180.74,529.37) circle (  1.16);

\path[draw=drawColor,line width= 0.4pt,line join=round,line cap=round,fill=fillColor] (180.81,529.36) circle (  1.16);

\path[draw=drawColor,line width= 0.4pt,line join=round,line cap=round,fill=fillColor] (180.89,529.34) circle (  1.16);

\path[draw=drawColor,line width= 0.4pt,line join=round,line cap=round,fill=fillColor] (180.96,529.31) circle (  1.16);

\path[draw=drawColor,line width= 0.4pt,line join=round,line cap=round,fill=fillColor] (181.04,529.30) circle (  1.16);

\path[draw=drawColor,line width= 0.4pt,line join=round,line cap=round,fill=fillColor] (181.11,529.30) circle (  1.16);

\path[draw=drawColor,line width= 0.4pt,line join=round,line cap=round,fill=fillColor] (181.19,529.30) circle (  1.16);

\path[draw=drawColor,line width= 0.4pt,line join=round,line cap=round,fill=fillColor] (181.26,529.28) circle (  1.16);

\path[draw=drawColor,line width= 0.4pt,line join=round,line cap=round,fill=fillColor] (181.34,529.26) circle (  1.16);

\path[draw=drawColor,line width= 0.4pt,line join=round,line cap=round,fill=fillColor] (181.41,529.25) circle (  1.16);

\path[draw=drawColor,line width= 0.4pt,line join=round,line cap=round,fill=fillColor] (181.49,529.23) circle (  1.16);

\path[draw=drawColor,line width= 0.4pt,line join=round,line cap=round,fill=fillColor] (181.56,529.22) circle (  1.16);

\path[draw=drawColor,line width= 0.4pt,line join=round,line cap=round,fill=fillColor] (181.64,529.22) circle (  1.16);

\path[draw=drawColor,line width= 0.4pt,line join=round,line cap=round,fill=fillColor] (181.71,529.17) circle (  1.16);

\path[draw=drawColor,line width= 0.4pt,line join=round,line cap=round,fill=fillColor] (181.79,529.11) circle (  1.16);

\path[draw=drawColor,line width= 0.4pt,line join=round,line cap=round,fill=fillColor] (181.86,529.07) circle (  1.16);

\path[draw=drawColor,line width= 0.4pt,line join=round,line cap=round,fill=fillColor] (181.94,529.06) circle (  1.16);

\path[draw=drawColor,line width= 0.4pt,line join=round,line cap=round,fill=fillColor] (182.01,529.04) circle (  1.16);

\path[draw=drawColor,line width= 0.4pt,line join=round,line cap=round,fill=fillColor] (182.08,529.00) circle (  1.16);

\path[draw=drawColor,line width= 0.4pt,line join=round,line cap=round,fill=fillColor] (182.16,528.99) circle (  1.16);

\path[draw=drawColor,line width= 0.4pt,line join=round,line cap=round,fill=fillColor] (182.23,528.98) circle (  1.16);

\path[draw=drawColor,line width= 0.4pt,line join=round,line cap=round,fill=fillColor] (182.31,528.96) circle (  1.16);

\path[draw=drawColor,line width= 0.4pt,line join=round,line cap=round,fill=fillColor] (182.38,528.95) circle (  1.16);

\path[draw=drawColor,line width= 0.4pt,line join=round,line cap=round,fill=fillColor] (182.45,528.94) circle (  1.16);

\path[draw=drawColor,line width= 0.4pt,line join=round,line cap=round,fill=fillColor] (182.53,528.92) circle (  1.16);

\path[draw=drawColor,line width= 0.4pt,line join=round,line cap=round,fill=fillColor] (182.60,528.90) circle (  1.16);

\path[draw=drawColor,line width= 0.4pt,line join=round,line cap=round,fill=fillColor] (182.67,528.88) circle (  1.16);

\path[draw=drawColor,line width= 0.4pt,line join=round,line cap=round,fill=fillColor] (182.75,528.87) circle (  1.16);

\path[draw=drawColor,line width= 0.4pt,line join=round,line cap=round,fill=fillColor] (182.82,528.86) circle (  1.16);

\path[draw=drawColor,line width= 0.4pt,line join=round,line cap=round,fill=fillColor] (182.89,528.77) circle (  1.16);

\path[draw=drawColor,line width= 0.4pt,line join=round,line cap=round,fill=fillColor] (182.96,528.76) circle (  1.16);

\path[draw=drawColor,line width= 0.4pt,line join=round,line cap=round,fill=fillColor] (183.04,528.71) circle (  1.16);

\path[draw=drawColor,line width= 0.4pt,line join=round,line cap=round,fill=fillColor] (183.11,528.70) circle (  1.16);

\path[draw=drawColor,line width= 0.4pt,line join=round,line cap=round,fill=fillColor] (183.18,528.69) circle (  1.16);

\path[draw=drawColor,line width= 0.4pt,line join=round,line cap=round,fill=fillColor] (183.26,528.69) circle (  1.16);

\path[draw=drawColor,line width= 0.4pt,line join=round,line cap=round,fill=fillColor] (183.33,528.68) circle (  1.16);

\path[draw=drawColor,line width= 0.4pt,line join=round,line cap=round,fill=fillColor] (183.40,528.67) circle (  1.16);

\path[draw=drawColor,line width= 0.4pt,line join=round,line cap=round,fill=fillColor] (183.47,528.66) circle (  1.16);

\path[draw=drawColor,line width= 0.4pt,line join=round,line cap=round,fill=fillColor] (183.54,528.66) circle (  1.16);

\path[draw=drawColor,line width= 0.4pt,line join=round,line cap=round,fill=fillColor] (183.62,528.64) circle (  1.16);

\path[draw=drawColor,line width= 0.4pt,line join=round,line cap=round,fill=fillColor] (183.69,528.60) circle (  1.16);

\path[draw=drawColor,line width= 0.4pt,line join=round,line cap=round,fill=fillColor] (183.76,528.60) circle (  1.16);

\path[draw=drawColor,line width= 0.4pt,line join=round,line cap=round,fill=fillColor] (183.83,528.53) circle (  1.16);

\path[draw=drawColor,line width= 0.4pt,line join=round,line cap=round,fill=fillColor] (183.90,528.53) circle (  1.16);

\path[draw=drawColor,line width= 0.4pt,line join=round,line cap=round,fill=fillColor] (183.98,528.50) circle (  1.16);

\path[draw=drawColor,line width= 0.4pt,line join=round,line cap=round,fill=fillColor] (184.05,528.49) circle (  1.16);

\path[draw=drawColor,line width= 0.4pt,line join=round,line cap=round,fill=fillColor] (184.12,528.48) circle (  1.16);

\path[draw=drawColor,line width= 0.4pt,line join=round,line cap=round,fill=fillColor] (184.19,528.47) circle (  1.16);

\path[draw=drawColor,line width= 0.4pt,line join=round,line cap=round,fill=fillColor] (184.26,528.46) circle (  1.16);

\path[draw=drawColor,line width= 0.4pt,line join=round,line cap=round,fill=fillColor] (184.33,528.42) circle (  1.16);

\path[draw=drawColor,line width= 0.4pt,line join=round,line cap=round,fill=fillColor] (184.40,528.42) circle (  1.16);

\path[draw=drawColor,line width= 0.4pt,line join=round,line cap=round,fill=fillColor] (184.48,528.41) circle (  1.16);

\path[draw=drawColor,line width= 0.4pt,line join=round,line cap=round,fill=fillColor] (184.55,528.40) circle (  1.16);

\path[draw=drawColor,line width= 0.4pt,line join=round,line cap=round,fill=fillColor] (184.62,528.40) circle (  1.16);

\path[draw=drawColor,line width= 0.4pt,line join=round,line cap=round,fill=fillColor] (184.69,528.40) circle (  1.16);

\path[draw=drawColor,line width= 0.4pt,line join=round,line cap=round,fill=fillColor] (184.76,528.39) circle (  1.16);

\path[draw=drawColor,line width= 0.4pt,line join=round,line cap=round,fill=fillColor] (184.83,528.36) circle (  1.16);

\path[draw=drawColor,line width= 0.4pt,line join=round,line cap=round,fill=fillColor] (184.90,528.35) circle (  1.16);

\path[draw=drawColor,line width= 0.4pt,line join=round,line cap=round,fill=fillColor] (184.97,528.35) circle (  1.16);

\path[draw=drawColor,line width= 0.4pt,line join=round,line cap=round,fill=fillColor] (185.04,528.31) circle (  1.16);

\path[draw=drawColor,line width= 0.4pt,line join=round,line cap=round,fill=fillColor] (185.11,528.31) circle (  1.16);

\path[draw=drawColor,line width= 0.4pt,line join=round,line cap=round,fill=fillColor] (185.18,528.31) circle (  1.16);

\path[draw=drawColor,line width= 0.4pt,line join=round,line cap=round,fill=fillColor] (185.25,528.28) circle (  1.16);

\path[draw=drawColor,line width= 0.4pt,line join=round,line cap=round,fill=fillColor] (185.32,528.28) circle (  1.16);

\path[draw=drawColor,line width= 0.4pt,line join=round,line cap=round,fill=fillColor] (185.39,528.25) circle (  1.16);

\path[draw=drawColor,line width= 0.4pt,line join=round,line cap=round,fill=fillColor] (185.46,528.25) circle (  1.16);

\path[draw=drawColor,line width= 0.4pt,line join=round,line cap=round,fill=fillColor] (185.53,528.22) circle (  1.16);

\path[draw=drawColor,line width= 0.4pt,line join=round,line cap=round,fill=fillColor] (185.60,528.22) circle (  1.16);

\path[draw=drawColor,line width= 0.4pt,line join=round,line cap=round,fill=fillColor] (185.67,528.19) circle (  1.16);

\path[draw=drawColor,line width= 0.4pt,line join=round,line cap=round,fill=fillColor] (185.74,528.18) circle (  1.16);

\path[draw=drawColor,line width= 0.4pt,line join=round,line cap=round,fill=fillColor] (185.81,528.17) circle (  1.16);

\path[draw=drawColor,line width= 0.4pt,line join=round,line cap=round,fill=fillColor] (185.88,528.14) circle (  1.16);

\path[draw=drawColor,line width= 0.4pt,line join=round,line cap=round,fill=fillColor] (185.95,528.14) circle (  1.16);

\path[draw=drawColor,line width= 0.4pt,line join=round,line cap=round,fill=fillColor] (186.02,528.12) circle (  1.16);

\path[draw=drawColor,line width= 0.4pt,line join=round,line cap=round,fill=fillColor] (186.09,528.09) circle (  1.16);

\path[draw=drawColor,line width= 0.4pt,line join=round,line cap=round,fill=fillColor] (186.16,528.09) circle (  1.16);

\path[draw=drawColor,line width= 0.4pt,line join=round,line cap=round,fill=fillColor] (186.22,528.04) circle (  1.16);

\path[draw=drawColor,line width= 0.4pt,line join=round,line cap=round,fill=fillColor] (186.29,528.02) circle (  1.16);

\path[draw=drawColor,line width= 0.4pt,line join=round,line cap=round,fill=fillColor] (186.36,528.01) circle (  1.16);

\path[draw=drawColor,line width= 0.4pt,line join=round,line cap=round,fill=fillColor] (186.43,528.00) circle (  1.16);

\path[draw=drawColor,line width= 0.4pt,line join=round,line cap=round,fill=fillColor] (186.50,528.00) circle (  1.16);

\path[draw=drawColor,line width= 0.4pt,line join=round,line cap=round,fill=fillColor] (186.57,527.96) circle (  1.16);

\path[draw=drawColor,line width= 0.4pt,line join=round,line cap=round,fill=fillColor] (186.64,527.96) circle (  1.16);

\path[draw=drawColor,line width= 0.4pt,line join=round,line cap=round,fill=fillColor] (186.70,527.95) circle (  1.16);

\path[draw=drawColor,line width= 0.4pt,line join=round,line cap=round,fill=fillColor] (186.77,527.94) circle (  1.16);

\path[draw=drawColor,line width= 0.4pt,line join=round,line cap=round,fill=fillColor] (186.84,527.94) circle (  1.16);

\path[draw=drawColor,line width= 0.4pt,line join=round,line cap=round,fill=fillColor] (186.91,527.92) circle (  1.16);

\path[draw=drawColor,line width= 0.4pt,line join=round,line cap=round,fill=fillColor] (186.98,527.84) circle (  1.16);

\path[draw=drawColor,line width= 0.4pt,line join=round,line cap=round,fill=fillColor] (187.05,527.84) circle (  1.16);

\path[draw=drawColor,line width= 0.4pt,line join=round,line cap=round,fill=fillColor] (187.11,527.83) circle (  1.16);

\path[draw=drawColor,line width= 0.4pt,line join=round,line cap=round,fill=fillColor] (187.18,527.80) circle (  1.16);

\path[draw=drawColor,line width= 0.4pt,line join=round,line cap=round,fill=fillColor] (187.25,527.80) circle (  1.16);

\path[draw=drawColor,line width= 0.4pt,line join=round,line cap=round,fill=fillColor] (187.32,527.75) circle (  1.16);

\path[draw=drawColor,line width= 0.4pt,line join=round,line cap=round,fill=fillColor] (187.38,527.72) circle (  1.16);

\path[draw=drawColor,line width= 0.4pt,line join=round,line cap=round,fill=fillColor] (187.45,527.68) circle (  1.16);

\path[draw=drawColor,line width= 0.4pt,line join=round,line cap=round,fill=fillColor] (187.52,527.68) circle (  1.16);

\path[draw=drawColor,line width= 0.4pt,line join=round,line cap=round,fill=fillColor] (187.59,527.68) circle (  1.16);

\path[draw=drawColor,line width= 0.4pt,line join=round,line cap=round,fill=fillColor] (187.65,527.64) circle (  1.16);

\path[draw=drawColor,line width= 0.4pt,line join=round,line cap=round,fill=fillColor] (187.72,527.63) circle (  1.16);

\path[draw=drawColor,line width= 0.4pt,line join=round,line cap=round,fill=fillColor] (187.79,527.62) circle (  1.16);

\path[draw=drawColor,line width= 0.4pt,line join=round,line cap=round,fill=fillColor] (187.86,527.60) circle (  1.16);

\path[draw=drawColor,line width= 0.4pt,line join=round,line cap=round,fill=fillColor] (187.92,527.58) circle (  1.16);

\path[draw=drawColor,line width= 0.4pt,line join=round,line cap=round,fill=fillColor] (187.99,527.55) circle (  1.16);

\path[draw=drawColor,line width= 0.4pt,line join=round,line cap=round,fill=fillColor] (188.06,527.53) circle (  1.16);

\path[draw=drawColor,line width= 0.4pt,line join=round,line cap=round,fill=fillColor] (188.12,527.53) circle (  1.16);

\path[draw=drawColor,line width= 0.4pt,line join=round,line cap=round,fill=fillColor] (188.19,527.52) circle (  1.16);

\path[draw=drawColor,line width= 0.4pt,line join=round,line cap=round,fill=fillColor] (188.26,527.52) circle (  1.16);

\path[draw=drawColor,line width= 0.4pt,line join=round,line cap=round,fill=fillColor] (188.32,527.51) circle (  1.16);

\path[draw=drawColor,line width= 0.4pt,line join=round,line cap=round,fill=fillColor] (188.39,527.47) circle (  1.16);

\path[draw=drawColor,line width= 0.4pt,line join=round,line cap=round,fill=fillColor] (188.46,527.47) circle (  1.16);

\path[draw=drawColor,line width= 0.4pt,line join=round,line cap=round,fill=fillColor] (188.52,527.46) circle (  1.16);

\path[draw=drawColor,line width= 0.4pt,line join=round,line cap=round,fill=fillColor] (188.59,527.45) circle (  1.16);

\path[draw=drawColor,line width= 0.4pt,line join=round,line cap=round,fill=fillColor] (188.66,527.43) circle (  1.16);

\path[draw=drawColor,line width= 0.4pt,line join=round,line cap=round,fill=fillColor] (188.72,527.43) circle (  1.16);

\path[draw=drawColor,line width= 0.4pt,line join=round,line cap=round,fill=fillColor] (188.79,527.42) circle (  1.16);

\path[draw=drawColor,line width= 0.4pt,line join=round,line cap=round,fill=fillColor] (188.85,527.41) circle (  1.16);

\path[draw=drawColor,line width= 0.4pt,line join=round,line cap=round,fill=fillColor] (188.92,527.38) circle (  1.16);

\path[draw=drawColor,line width= 0.4pt,line join=round,line cap=round,fill=fillColor] (188.99,527.38) circle (  1.16);

\path[draw=drawColor,line width= 0.4pt,line join=round,line cap=round,fill=fillColor] (189.05,527.35) circle (  1.16);

\path[draw=drawColor,line width= 0.4pt,line join=round,line cap=round,fill=fillColor] (189.12,527.34) circle (  1.16);

\path[draw=drawColor,line width= 0.4pt,line join=round,line cap=round,fill=fillColor] (189.18,527.34) circle (  1.16);

\path[draw=drawColor,line width= 0.4pt,line join=round,line cap=round,fill=fillColor] (189.25,527.32) circle (  1.16);

\path[draw=drawColor,line width= 0.4pt,line join=round,line cap=round,fill=fillColor] (189.31,527.32) circle (  1.16);

\path[draw=drawColor,line width= 0.4pt,line join=round,line cap=round,fill=fillColor] (189.38,527.32) circle (  1.16);

\path[draw=drawColor,line width= 0.4pt,line join=round,line cap=round,fill=fillColor] (189.45,527.31) circle (  1.16);

\path[draw=drawColor,line width= 0.4pt,line join=round,line cap=round,fill=fillColor] (189.51,527.29) circle (  1.16);

\path[draw=drawColor,line width= 0.4pt,line join=round,line cap=round,fill=fillColor] (189.58,527.29) circle (  1.16);

\path[draw=drawColor,line width= 0.4pt,line join=round,line cap=round,fill=fillColor] (189.64,527.29) circle (  1.16);

\path[draw=drawColor,line width= 0.4pt,line join=round,line cap=round,fill=fillColor] (189.71,527.25) circle (  1.16);

\path[draw=drawColor,line width= 0.4pt,line join=round,line cap=round,fill=fillColor] (189.77,527.25) circle (  1.16);

\path[draw=drawColor,line width= 0.4pt,line join=round,line cap=round,fill=fillColor] (189.84,527.23) circle (  1.16);

\path[draw=drawColor,line width= 0.4pt,line join=round,line cap=round,fill=fillColor] (189.90,527.20) circle (  1.16);

\path[draw=drawColor,line width= 0.4pt,line join=round,line cap=round,fill=fillColor] (189.97,527.19) circle (  1.16);

\path[draw=drawColor,line width= 0.4pt,line join=round,line cap=round,fill=fillColor] (190.03,527.18) circle (  1.16);

\path[draw=drawColor,line width= 0.4pt,line join=round,line cap=round,fill=fillColor] (190.10,527.16) circle (  1.16);

\path[draw=drawColor,line width= 0.4pt,line join=round,line cap=round,fill=fillColor] (190.16,527.13) circle (  1.16);

\path[draw=drawColor,line width= 0.4pt,line join=round,line cap=round,fill=fillColor] (190.23,527.13) circle (  1.16);

\path[draw=drawColor,line width= 0.4pt,line join=round,line cap=round,fill=fillColor] (190.29,527.12) circle (  1.16);

\path[draw=drawColor,line width= 0.4pt,line join=round,line cap=round,fill=fillColor] (190.35,527.07) circle (  1.16);

\path[draw=drawColor,line width= 0.4pt,line join=round,line cap=round,fill=fillColor] (190.42,527.05) circle (  1.16);

\path[draw=drawColor,line width= 0.4pt,line join=round,line cap=round,fill=fillColor] (190.48,527.03) circle (  1.16);

\path[draw=drawColor,line width= 0.4pt,line join=round,line cap=round,fill=fillColor] (190.55,527.01) circle (  1.16);

\path[draw=drawColor,line width= 0.4pt,line join=round,line cap=round,fill=fillColor] (190.61,527.00) circle (  1.16);

\path[draw=drawColor,line width= 0.4pt,line join=round,line cap=round,fill=fillColor] (190.68,526.99) circle (  1.16);

\path[draw=drawColor,line width= 0.4pt,line join=round,line cap=round,fill=fillColor] (190.74,526.97) circle (  1.16);

\path[draw=drawColor,line width= 0.4pt,line join=round,line cap=round,fill=fillColor] (190.80,526.93) circle (  1.16);

\path[draw=drawColor,line width= 0.4pt,line join=round,line cap=round,fill=fillColor] (190.87,526.93) circle (  1.16);

\path[draw=drawColor,line width= 0.4pt,line join=round,line cap=round,fill=fillColor] (190.93,526.90) circle (  1.16);

\path[draw=drawColor,line width= 0.4pt,line join=round,line cap=round,fill=fillColor] (190.99,526.90) circle (  1.16);

\path[draw=drawColor,line width= 0.4pt,line join=round,line cap=round,fill=fillColor] (191.06,526.83) circle (  1.16);

\path[draw=drawColor,line width= 0.4pt,line join=round,line cap=round,fill=fillColor] (191.12,526.82) circle (  1.16);

\path[draw=drawColor,line width= 0.4pt,line join=round,line cap=round,fill=fillColor] (191.19,526.80) circle (  1.16);

\path[draw=drawColor,line width= 0.4pt,line join=round,line cap=round,fill=fillColor] (191.25,526.77) circle (  1.16);

\path[draw=drawColor,line width= 0.4pt,line join=round,line cap=round,fill=fillColor] (191.31,526.74) circle (  1.16);

\path[draw=drawColor,line width= 0.4pt,line join=round,line cap=round,fill=fillColor] (191.38,526.73) circle (  1.16);

\path[draw=drawColor,line width= 0.4pt,line join=round,line cap=round,fill=fillColor] (191.44,526.73) circle (  1.16);

\path[draw=drawColor,line width= 0.4pt,line join=round,line cap=round,fill=fillColor] (191.50,526.70) circle (  1.16);

\path[draw=drawColor,line width= 0.4pt,line join=round,line cap=round,fill=fillColor] (191.57,526.69) circle (  1.16);

\path[draw=drawColor,line width= 0.4pt,line join=round,line cap=round,fill=fillColor] (191.63,526.69) circle (  1.16);

\path[draw=drawColor,line width= 0.4pt,line join=round,line cap=round,fill=fillColor] (191.69,526.68) circle (  1.16);

\path[draw=drawColor,line width= 0.4pt,line join=round,line cap=round,fill=fillColor] (191.75,526.67) circle (  1.16);

\path[draw=drawColor,line width= 0.4pt,line join=round,line cap=round,fill=fillColor] (191.82,526.65) circle (  1.16);

\path[draw=drawColor,line width= 0.4pt,line join=round,line cap=round,fill=fillColor] (191.88,526.65) circle (  1.16);

\path[draw=drawColor,line width= 0.4pt,line join=round,line cap=round,fill=fillColor] (191.94,526.63) circle (  1.16);

\path[draw=drawColor,line width= 0.4pt,line join=round,line cap=round,fill=fillColor] (192.01,526.61) circle (  1.16);

\path[draw=drawColor,line width= 0.4pt,line join=round,line cap=round,fill=fillColor] (192.07,526.60) circle (  1.16);

\path[draw=drawColor,line width= 0.4pt,line join=round,line cap=round,fill=fillColor] (192.13,526.59) circle (  1.16);

\path[draw=drawColor,line width= 0.4pt,line join=round,line cap=round,fill=fillColor] (192.19,526.57) circle (  1.16);

\path[draw=drawColor,line width= 0.4pt,line join=round,line cap=round,fill=fillColor] (192.26,526.56) circle (  1.16);

\path[draw=drawColor,line width= 0.4pt,line join=round,line cap=round,fill=fillColor] (192.32,526.51) circle (  1.16);

\path[draw=drawColor,line width= 0.4pt,line join=round,line cap=round,fill=fillColor] (192.38,526.50) circle (  1.16);

\path[draw=drawColor,line width= 0.4pt,line join=round,line cap=round,fill=fillColor] (192.44,526.49) circle (  1.16);

\path[draw=drawColor,line width= 0.4pt,line join=round,line cap=round,fill=fillColor] (192.51,526.49) circle (  1.16);

\path[draw=drawColor,line width= 0.4pt,line join=round,line cap=round,fill=fillColor] (192.57,526.48) circle (  1.16);

\path[draw=drawColor,line width= 0.4pt,line join=round,line cap=round,fill=fillColor] (192.63,526.43) circle (  1.16);

\path[draw=drawColor,line width= 0.4pt,line join=round,line cap=round,fill=fillColor] (192.69,526.42) circle (  1.16);

\path[draw=drawColor,line width= 0.4pt,line join=round,line cap=round,fill=fillColor] (192.75,526.42) circle (  1.16);

\path[draw=drawColor,line width= 0.4pt,line join=round,line cap=round,fill=fillColor] (192.82,526.41) circle (  1.16);

\path[draw=drawColor,line width= 0.4pt,line join=round,line cap=round,fill=fillColor] (192.88,526.35) circle (  1.16);

\path[draw=drawColor,line width= 0.4pt,line join=round,line cap=round,fill=fillColor] (192.94,526.30) circle (  1.16);

\path[draw=drawColor,line width= 0.4pt,line join=round,line cap=round,fill=fillColor] (193.00,526.28) circle (  1.16);

\path[draw=drawColor,line width= 0.4pt,line join=round,line cap=round,fill=fillColor] (193.06,526.27) circle (  1.16);

\path[draw=drawColor,line width= 0.4pt,line join=round,line cap=round,fill=fillColor] (193.13,526.23) circle (  1.16);

\path[draw=drawColor,line width= 0.4pt,line join=round,line cap=round,fill=fillColor] (193.19,526.22) circle (  1.16);

\path[draw=drawColor,line width= 0.4pt,line join=round,line cap=round,fill=fillColor] (193.25,526.19) circle (  1.16);

\path[draw=drawColor,line width= 0.4pt,line join=round,line cap=round,fill=fillColor] (193.31,526.15) circle (  1.16);

\path[draw=drawColor,line width= 0.4pt,line join=round,line cap=round,fill=fillColor] (193.37,526.14) circle (  1.16);

\path[draw=drawColor,line width= 0.4pt,line join=round,line cap=round,fill=fillColor] (193.43,526.13) circle (  1.16);

\path[draw=drawColor,line width= 0.4pt,line join=round,line cap=round,fill=fillColor] (193.49,526.13) circle (  1.16);

\path[draw=drawColor,line width= 0.4pt,line join=round,line cap=round,fill=fillColor] (193.55,526.09) circle (  1.16);

\path[draw=drawColor,line width= 0.4pt,line join=round,line cap=round,fill=fillColor] (193.62,526.07) circle (  1.16);

\path[draw=drawColor,line width= 0.4pt,line join=round,line cap=round,fill=fillColor] (193.68,526.06) circle (  1.16);

\path[draw=drawColor,line width= 0.4pt,line join=round,line cap=round,fill=fillColor] (193.74,526.05) circle (  1.16);

\path[draw=drawColor,line width= 0.4pt,line join=round,line cap=round,fill=fillColor] (193.80,526.04) circle (  1.16);

\path[draw=drawColor,line width= 0.4pt,line join=round,line cap=round,fill=fillColor] (193.86,526.02) circle (  1.16);

\path[draw=drawColor,line width= 0.4pt,line join=round,line cap=round,fill=fillColor] (193.92,526.02) circle (  1.16);

\path[draw=drawColor,line width= 0.4pt,line join=round,line cap=round,fill=fillColor] (193.98,526.00) circle (  1.16);

\path[draw=drawColor,line width= 0.4pt,line join=round,line cap=round,fill=fillColor] (194.04,526.00) circle (  1.16);

\path[draw=drawColor,line width= 0.4pt,line join=round,line cap=round,fill=fillColor] (194.10,525.98) circle (  1.16);

\path[draw=drawColor,line width= 0.4pt,line join=round,line cap=round,fill=fillColor] (194.16,525.97) circle (  1.16);

\path[draw=drawColor,line width= 0.4pt,line join=round,line cap=round,fill=fillColor] (194.22,525.96) circle (  1.16);

\path[draw=drawColor,line width= 0.4pt,line join=round,line cap=round,fill=fillColor] (194.29,525.90) circle (  1.16);

\path[draw=drawColor,line width= 0.4pt,line join=round,line cap=round,fill=fillColor] (194.35,525.88) circle (  1.16);

\path[draw=drawColor,line width= 0.4pt,line join=round,line cap=round,fill=fillColor] (194.41,525.87) circle (  1.16);

\path[draw=drawColor,line width= 0.4pt,line join=round,line cap=round,fill=fillColor] (194.47,525.87) circle (  1.16);

\path[draw=drawColor,line width= 0.4pt,line join=round,line cap=round,fill=fillColor] (194.53,525.86) circle (  1.16);

\path[draw=drawColor,line width= 0.4pt,line join=round,line cap=round,fill=fillColor] (194.59,525.86) circle (  1.16);

\path[draw=drawColor,line width= 0.4pt,line join=round,line cap=round,fill=fillColor] (194.65,525.85) circle (  1.16);

\path[draw=drawColor,line width= 0.4pt,line join=round,line cap=round,fill=fillColor] (194.71,525.83) circle (  1.16);

\path[draw=drawColor,line width= 0.4pt,line join=round,line cap=round,fill=fillColor] (194.77,525.80) circle (  1.16);

\path[draw=drawColor,line width= 0.4pt,line join=round,line cap=round,fill=fillColor] (194.83,525.79) circle (  1.16);

\path[draw=drawColor,line width= 0.4pt,line join=round,line cap=round,fill=fillColor] (194.89,525.78) circle (  1.16);

\path[draw=drawColor,line width= 0.4pt,line join=round,line cap=round,fill=fillColor] (194.95,525.76) circle (  1.16);

\path[draw=drawColor,line width= 0.4pt,line join=round,line cap=round,fill=fillColor] (195.01,525.72) circle (  1.16);

\path[draw=drawColor,line width= 0.4pt,line join=round,line cap=round,fill=fillColor] (195.07,525.70) circle (  1.16);

\path[draw=drawColor,line width= 0.4pt,line join=round,line cap=round,fill=fillColor] (195.13,525.67) circle (  1.16);

\path[draw=drawColor,line width= 0.4pt,line join=round,line cap=round,fill=fillColor] (195.19,525.67) circle (  1.16);

\path[draw=drawColor,line width= 0.4pt,line join=round,line cap=round,fill=fillColor] (195.25,525.66) circle (  1.16);

\path[draw=drawColor,line width= 0.4pt,line join=round,line cap=round,fill=fillColor] (195.31,525.66) circle (  1.16);

\path[draw=drawColor,line width= 0.4pt,line join=round,line cap=round,fill=fillColor] (195.36,525.65) circle (  1.16);

\path[draw=drawColor,line width= 0.4pt,line join=round,line cap=round,fill=fillColor] (195.42,525.64) circle (  1.16);

\path[draw=drawColor,line width= 0.4pt,line join=round,line cap=round,fill=fillColor] (195.48,525.61) circle (  1.16);

\path[draw=drawColor,line width= 0.4pt,line join=round,line cap=round,fill=fillColor] (195.54,525.61) circle (  1.16);

\path[draw=drawColor,line width= 0.4pt,line join=round,line cap=round,fill=fillColor] (195.60,525.55) circle (  1.16);

\path[draw=drawColor,line width= 0.4pt,line join=round,line cap=round,fill=fillColor] (195.66,525.52) circle (  1.16);

\path[draw=drawColor,line width= 0.4pt,line join=round,line cap=round,fill=fillColor] (195.72,525.51) circle (  1.16);

\path[draw=drawColor,line width= 0.4pt,line join=round,line cap=round,fill=fillColor] (195.78,525.50) circle (  1.16);

\path[draw=drawColor,line width= 0.4pt,line join=round,line cap=round,fill=fillColor] (195.84,525.49) circle (  1.16);

\path[draw=drawColor,line width= 0.4pt,line join=round,line cap=round,fill=fillColor] (195.90,525.41) circle (  1.16);

\path[draw=drawColor,line width= 0.4pt,line join=round,line cap=round,fill=fillColor] (195.96,525.40) circle (  1.16);

\path[draw=drawColor,line width= 0.4pt,line join=round,line cap=round,fill=fillColor] (196.02,525.35) circle (  1.16);

\path[draw=drawColor,line width= 0.4pt,line join=round,line cap=round,fill=fillColor] (196.07,525.34) circle (  1.16);

\path[draw=drawColor,line width= 0.4pt,line join=round,line cap=round,fill=fillColor] (196.13,525.34) circle (  1.16);

\path[draw=drawColor,line width= 0.4pt,line join=round,line cap=round,fill=fillColor] (196.19,525.33) circle (  1.16);

\path[draw=drawColor,line width= 0.4pt,line join=round,line cap=round,fill=fillColor] (196.25,525.32) circle (  1.16);

\path[draw=drawColor,line width= 0.4pt,line join=round,line cap=round,fill=fillColor] (196.31,525.31) circle (  1.16);

\path[draw=drawColor,line width= 0.4pt,line join=round,line cap=round,fill=fillColor] (196.37,525.29) circle (  1.16);

\path[draw=drawColor,line width= 0.4pt,line join=round,line cap=round,fill=fillColor] (196.43,525.29) circle (  1.16);

\path[draw=drawColor,line width= 0.4pt,line join=round,line cap=round,fill=fillColor] (196.48,525.28) circle (  1.16);

\path[draw=drawColor,line width= 0.4pt,line join=round,line cap=round,fill=fillColor] (196.54,525.27) circle (  1.16);

\path[draw=drawColor,line width= 0.4pt,line join=round,line cap=round,fill=fillColor] (196.60,525.25) circle (  1.16);

\path[draw=drawColor,line width= 0.4pt,line join=round,line cap=round,fill=fillColor] (196.66,525.23) circle (  1.16);

\path[draw=drawColor,line width= 0.4pt,line join=round,line cap=round,fill=fillColor] (196.72,525.23) circle (  1.16);

\path[draw=drawColor,line width= 0.4pt,line join=round,line cap=round,fill=fillColor] (196.78,525.23) circle (  1.16);

\path[draw=drawColor,line width= 0.4pt,line join=round,line cap=round,fill=fillColor] (196.83,525.22) circle (  1.16);

\path[draw=drawColor,line width= 0.4pt,line join=round,line cap=round,fill=fillColor] (196.89,525.17) circle (  1.16);

\path[draw=drawColor,line width= 0.4pt,line join=round,line cap=round,fill=fillColor] (196.95,525.16) circle (  1.16);

\path[draw=drawColor,line width= 0.4pt,line join=round,line cap=round,fill=fillColor] (197.01,525.15) circle (  1.16);

\path[draw=drawColor,line width= 0.4pt,line join=round,line cap=round,fill=fillColor] (197.07,525.10) circle (  1.16);

\path[draw=drawColor,line width= 0.4pt,line join=round,line cap=round,fill=fillColor] (197.12,525.08) circle (  1.16);

\path[draw=drawColor,line width= 0.4pt,line join=round,line cap=round,fill=fillColor] (197.18,525.05) circle (  1.16);

\path[draw=drawColor,line width= 0.4pt,line join=round,line cap=round,fill=fillColor] (197.24,525.05) circle (  1.16);

\path[draw=drawColor,line width= 0.4pt,line join=round,line cap=round,fill=fillColor] (197.30,525.04) circle (  1.16);

\path[draw=drawColor,line width= 0.4pt,line join=round,line cap=round,fill=fillColor] (197.36,525.04) circle (  1.16);

\path[draw=drawColor,line width= 0.4pt,line join=round,line cap=round,fill=fillColor] (197.41,524.98) circle (  1.16);

\path[draw=drawColor,line width= 0.4pt,line join=round,line cap=round,fill=fillColor] (197.47,524.98) circle (  1.16);

\path[draw=drawColor,line width= 0.4pt,line join=round,line cap=round,fill=fillColor] (197.53,524.91) circle (  1.16);

\path[draw=drawColor,line width= 0.4pt,line join=round,line cap=round,fill=fillColor] (197.59,524.88) circle (  1.16);

\path[draw=drawColor,line width= 0.4pt,line join=round,line cap=round,fill=fillColor] (197.64,524.87) circle (  1.16);

\path[draw=drawColor,line width= 0.4pt,line join=round,line cap=round,fill=fillColor] (197.70,524.86) circle (  1.16);

\path[draw=drawColor,line width= 0.4pt,line join=round,line cap=round,fill=fillColor] (197.76,524.85) circle (  1.16);

\path[draw=drawColor,line width= 0.4pt,line join=round,line cap=round,fill=fillColor] (197.82,524.83) circle (  1.16);

\path[draw=drawColor,line width= 0.4pt,line join=round,line cap=round,fill=fillColor] (197.87,524.80) circle (  1.16);

\path[draw=drawColor,line width= 0.4pt,line join=round,line cap=round,fill=fillColor] (197.93,524.80) circle (  1.16);

\path[draw=drawColor,line width= 0.4pt,line join=round,line cap=round,fill=fillColor] (197.99,524.76) circle (  1.16);

\path[draw=drawColor,line width= 0.4pt,line join=round,line cap=round,fill=fillColor] (198.04,524.76) circle (  1.16);

\path[draw=drawColor,line width= 0.4pt,line join=round,line cap=round,fill=fillColor] (198.10,524.72) circle (  1.16);

\path[draw=drawColor,line width= 0.4pt,line join=round,line cap=round,fill=fillColor] (198.16,524.67) circle (  1.16);

\path[draw=drawColor,line width= 0.4pt,line join=round,line cap=round,fill=fillColor] (198.22,524.64) circle (  1.16);

\path[draw=drawColor,line width= 0.4pt,line join=round,line cap=round,fill=fillColor] (198.27,524.63) circle (  1.16);

\path[draw=drawColor,line width= 0.4pt,line join=round,line cap=round,fill=fillColor] (198.33,524.63) circle (  1.16);

\path[draw=drawColor,line width= 0.4pt,line join=round,line cap=round,fill=fillColor] (198.39,524.62) circle (  1.16);

\path[draw=drawColor,line width= 0.4pt,line join=round,line cap=round,fill=fillColor] (198.44,524.62) circle (  1.16);

\path[draw=drawColor,line width= 0.4pt,line join=round,line cap=round,fill=fillColor] (198.50,524.62) circle (  1.16);

\path[draw=drawColor,line width= 0.4pt,line join=round,line cap=round,fill=fillColor] (198.56,524.62) circle (  1.16);

\path[draw=drawColor,line width= 0.4pt,line join=round,line cap=round,fill=fillColor] (198.61,524.62) circle (  1.16);

\path[draw=drawColor,line width= 0.4pt,line join=round,line cap=round,fill=fillColor] (198.67,524.62) circle (  1.16);

\path[draw=drawColor,line width= 0.4pt,line join=round,line cap=round,fill=fillColor] (198.73,524.54) circle (  1.16);

\path[draw=drawColor,line width= 0.4pt,line join=round,line cap=round,fill=fillColor] (198.78,524.51) circle (  1.16);

\path[draw=drawColor,line width= 0.4pt,line join=round,line cap=round,fill=fillColor] (198.84,524.49) circle (  1.16);

\path[draw=drawColor,line width= 0.4pt,line join=round,line cap=round,fill=fillColor] (198.90,524.44) circle (  1.16);

\path[draw=drawColor,line width= 0.4pt,line join=round,line cap=round,fill=fillColor] (198.95,524.40) circle (  1.16);

\path[draw=drawColor,line width= 0.4pt,line join=round,line cap=round,fill=fillColor] (199.01,524.39) circle (  1.16);

\path[draw=drawColor,line width= 0.4pt,line join=round,line cap=round,fill=fillColor] (199.06,524.34) circle (  1.16);

\path[draw=drawColor,line width= 0.4pt,line join=round,line cap=round,fill=fillColor] (199.12,524.33) circle (  1.16);

\path[draw=drawColor,line width= 0.4pt,line join=round,line cap=round,fill=fillColor] (199.18,524.31) circle (  1.16);

\path[draw=drawColor,line width= 0.4pt,line join=round,line cap=round,fill=fillColor] (199.23,524.29) circle (  1.16);

\path[draw=drawColor,line width= 0.4pt,line join=round,line cap=round,fill=fillColor] (199.29,524.29) circle (  1.16);

\path[draw=drawColor,line width= 0.4pt,line join=round,line cap=round,fill=fillColor] (199.34,524.26) circle (  1.16);

\path[draw=drawColor,line width= 0.4pt,line join=round,line cap=round,fill=fillColor] (199.40,524.24) circle (  1.16);

\path[draw=drawColor,line width= 0.4pt,line join=round,line cap=round,fill=fillColor] (199.46,524.24) circle (  1.16);

\path[draw=drawColor,line width= 0.4pt,line join=round,line cap=round,fill=fillColor] (199.51,524.23) circle (  1.16);

\path[draw=drawColor,line width= 0.4pt,line join=round,line cap=round,fill=fillColor] (199.57,524.20) circle (  1.16);

\path[draw=drawColor,line width= 0.4pt,line join=round,line cap=round,fill=fillColor] (199.62,524.09) circle (  1.16);

\path[draw=drawColor,line width= 0.4pt,line join=round,line cap=round,fill=fillColor] (199.68,524.08) circle (  1.16);

\path[draw=drawColor,line width= 0.4pt,line join=round,line cap=round,fill=fillColor] (199.73,524.06) circle (  1.16);

\path[draw=drawColor,line width= 0.4pt,line join=round,line cap=round,fill=fillColor] (199.79,524.05) circle (  1.16);

\path[draw=drawColor,line width= 0.4pt,line join=round,line cap=round,fill=fillColor] (199.85,524.03) circle (  1.16);

\path[draw=drawColor,line width= 0.4pt,line join=round,line cap=round,fill=fillColor] (199.90,524.02) circle (  1.16);

\path[draw=drawColor,line width= 0.4pt,line join=round,line cap=round,fill=fillColor] (199.96,524.00) circle (  1.16);

\path[draw=drawColor,line width= 0.4pt,line join=round,line cap=round,fill=fillColor] (200.01,523.99) circle (  1.16);

\path[draw=drawColor,line width= 0.4pt,line join=round,line cap=round,fill=fillColor] (200.07,523.98) circle (  1.16);

\path[draw=drawColor,line width= 0.4pt,line join=round,line cap=round,fill=fillColor] (200.12,523.97) circle (  1.16);

\path[draw=drawColor,line width= 0.4pt,line join=round,line cap=round,fill=fillColor] (200.18,523.95) circle (  1.16);

\path[draw=drawColor,line width= 0.4pt,line join=round,line cap=round,fill=fillColor] (200.23,523.85) circle (  1.16);

\path[draw=drawColor,line width= 0.4pt,line join=round,line cap=round,fill=fillColor] (200.29,523.84) circle (  1.16);

\path[draw=drawColor,line width= 0.4pt,line join=round,line cap=round,fill=fillColor] (200.34,523.80) circle (  1.16);

\path[draw=drawColor,line width= 0.4pt,line join=round,line cap=round,fill=fillColor] (200.40,523.69) circle (  1.16);

\path[draw=drawColor,line width= 0.4pt,line join=round,line cap=round,fill=fillColor] (200.45,523.65) circle (  1.16);

\path[draw=drawColor,line width= 0.4pt,line join=round,line cap=round,fill=fillColor] (200.51,523.57) circle (  1.16);

\path[draw=drawColor,line width= 0.4pt,line join=round,line cap=round,fill=fillColor] (200.56,523.52) circle (  1.16);

\path[draw=drawColor,line width= 0.4pt,line join=round,line cap=round,fill=fillColor] (200.62,523.50) circle (  1.16);

\path[draw=drawColor,line width= 0.4pt,line join=round,line cap=round,fill=fillColor] (200.67,523.46) circle (  1.16);

\path[draw=drawColor,line width= 0.4pt,line join=round,line cap=round,fill=fillColor] (200.73,523.46) circle (  1.16);

\path[draw=drawColor,line width= 0.4pt,line join=round,line cap=round,fill=fillColor] (200.78,523.44) circle (  1.16);

\path[draw=drawColor,line width= 0.4pt,line join=round,line cap=round,fill=fillColor] (200.84,523.44) circle (  1.16);

\path[draw=drawColor,line width= 0.4pt,line join=round,line cap=round,fill=fillColor] (200.89,523.43) circle (  1.16);

\path[draw=drawColor,line width= 0.4pt,line join=round,line cap=round,fill=fillColor] (200.95,523.42) circle (  1.16);

\path[draw=drawColor,line width= 0.4pt,line join=round,line cap=round,fill=fillColor] (201.00,523.39) circle (  1.16);

\path[draw=drawColor,line width= 0.4pt,line join=round,line cap=round,fill=fillColor] (201.06,523.37) circle (  1.16);

\path[draw=drawColor,line width= 0.4pt,line join=round,line cap=round,fill=fillColor] (201.11,523.37) circle (  1.16);

\path[draw=drawColor,line width= 0.4pt,line join=round,line cap=round,fill=fillColor] (201.17,523.36) circle (  1.16);

\path[draw=drawColor,line width= 0.4pt,line join=round,line cap=round,fill=fillColor] (201.22,523.33) circle (  1.16);

\path[draw=drawColor,line width= 0.4pt,line join=round,line cap=round,fill=fillColor] (201.27,523.32) circle (  1.16);

\path[draw=drawColor,line width= 0.4pt,line join=round,line cap=round,fill=fillColor] (201.33,523.32) circle (  1.16);

\path[draw=drawColor,line width= 0.4pt,line join=round,line cap=round,fill=fillColor] (201.38,523.30) circle (  1.16);

\path[draw=drawColor,line width= 0.4pt,line join=round,line cap=round,fill=fillColor] (201.44,523.22) circle (  1.16);

\path[draw=drawColor,line width= 0.4pt,line join=round,line cap=round,fill=fillColor] (201.49,523.17) circle (  1.16);

\path[draw=drawColor,line width= 0.4pt,line join=round,line cap=round,fill=fillColor] (201.55,523.15) circle (  1.16);

\path[draw=drawColor,line width= 0.4pt,line join=round,line cap=round,fill=fillColor] (201.60,523.14) circle (  1.16);

\path[draw=drawColor,line width= 0.4pt,line join=round,line cap=round,fill=fillColor] (201.65,523.14) circle (  1.16);

\path[draw=drawColor,line width= 0.4pt,line join=round,line cap=round,fill=fillColor] (201.71,523.13) circle (  1.16);

\path[draw=drawColor,line width= 0.4pt,line join=round,line cap=round,fill=fillColor] (201.76,523.12) circle (  1.16);

\path[draw=drawColor,line width= 0.4pt,line join=round,line cap=round,fill=fillColor] (201.82,523.10) circle (  1.16);

\path[draw=drawColor,line width= 0.4pt,line join=round,line cap=round,fill=fillColor] (201.87,523.08) circle (  1.16);

\path[draw=drawColor,line width= 0.4pt,line join=round,line cap=round,fill=fillColor] (201.92,523.05) circle (  1.16);

\path[draw=drawColor,line width= 0.4pt,line join=round,line cap=round,fill=fillColor] (201.98,523.04) circle (  1.16);

\path[draw=drawColor,line width= 0.4pt,line join=round,line cap=round,fill=fillColor] (202.03,523.04) circle (  1.16);

\path[draw=drawColor,line width= 0.4pt,line join=round,line cap=round,fill=fillColor] (202.09,523.02) circle (  1.16);

\path[draw=drawColor,line width= 0.4pt,line join=round,line cap=round,fill=fillColor] (202.14,523.01) circle (  1.16);

\path[draw=drawColor,line width= 0.4pt,line join=round,line cap=round,fill=fillColor] (202.19,523.00) circle (  1.16);

\path[draw=drawColor,line width= 0.4pt,line join=round,line cap=round,fill=fillColor] (202.25,522.99) circle (  1.16);

\path[draw=drawColor,line width= 0.4pt,line join=round,line cap=round,fill=fillColor] (202.30,522.95) circle (  1.16);

\path[draw=drawColor,line width= 0.4pt,line join=round,line cap=round,fill=fillColor] (202.35,522.94) circle (  1.16);

\path[draw=drawColor,line width= 0.4pt,line join=round,line cap=round,fill=fillColor] (202.41,522.91) circle (  1.16);

\path[draw=drawColor,line width= 0.4pt,line join=round,line cap=round,fill=fillColor] (202.46,522.87) circle (  1.16);

\path[draw=drawColor,line width= 0.4pt,line join=round,line cap=round,fill=fillColor] (202.51,522.86) circle (  1.16);

\path[draw=drawColor,line width= 0.4pt,line join=round,line cap=round,fill=fillColor] (202.57,522.85) circle (  1.16);

\path[draw=drawColor,line width= 0.4pt,line join=round,line cap=round,fill=fillColor] (202.62,522.79) circle (  1.16);

\path[draw=drawColor,line width= 0.4pt,line join=round,line cap=round,fill=fillColor] (202.67,522.78) circle (  1.16);

\path[draw=drawColor,line width= 0.4pt,line join=round,line cap=round,fill=fillColor] (202.73,522.77) circle (  1.16);

\path[draw=drawColor,line width= 0.4pt,line join=round,line cap=round,fill=fillColor] (202.78,522.74) circle (  1.16);

\path[draw=drawColor,line width= 0.4pt,line join=round,line cap=round,fill=fillColor] (202.83,522.73) circle (  1.16);

\path[draw=drawColor,line width= 0.4pt,line join=round,line cap=round,fill=fillColor] (202.89,522.63) circle (  1.16);

\path[draw=drawColor,line width= 0.4pt,line join=round,line cap=round,fill=fillColor] (202.94,522.57) circle (  1.16);

\path[draw=drawColor,line width= 0.4pt,line join=round,line cap=round,fill=fillColor] (202.99,522.57) circle (  1.16);

\path[draw=drawColor,line width= 0.4pt,line join=round,line cap=round,fill=fillColor] (203.05,522.53) circle (  1.16);

\path[draw=drawColor,line width= 0.4pt,line join=round,line cap=round,fill=fillColor] (203.10,522.50) circle (  1.16);

\path[draw=drawColor,line width= 0.4pt,line join=round,line cap=round,fill=fillColor] (203.15,522.46) circle (  1.16);

\path[draw=drawColor,line width= 0.4pt,line join=round,line cap=round,fill=fillColor] (203.20,522.44) circle (  1.16);

\path[draw=drawColor,line width= 0.4pt,line join=round,line cap=round,fill=fillColor] (203.26,522.41) circle (  1.16);

\path[draw=drawColor,line width= 0.4pt,line join=round,line cap=round,fill=fillColor] (203.31,522.40) circle (  1.16);

\path[draw=drawColor,line width= 0.4pt,line join=round,line cap=round,fill=fillColor] (203.36,522.38) circle (  1.16);

\path[draw=drawColor,line width= 0.4pt,line join=round,line cap=round,fill=fillColor] (203.41,522.26) circle (  1.16);

\path[draw=drawColor,line width= 0.4pt,line join=round,line cap=round,fill=fillColor] (203.47,522.25) circle (  1.16);

\path[draw=drawColor,line width= 0.4pt,line join=round,line cap=round,fill=fillColor] (203.52,522.18) circle (  1.16);

\path[draw=drawColor,line width= 0.4pt,line join=round,line cap=round,fill=fillColor] (203.57,522.15) circle (  1.16);

\path[draw=drawColor,line width= 0.4pt,line join=round,line cap=round,fill=fillColor] (203.63,522.08) circle (  1.16);

\path[draw=drawColor,line width= 0.4pt,line join=round,line cap=round,fill=fillColor] (203.68,522.06) circle (  1.16);

\path[draw=drawColor,line width= 0.4pt,line join=round,line cap=round,fill=fillColor] (203.73,522.03) circle (  1.16);

\path[draw=drawColor,line width= 0.4pt,line join=round,line cap=round,fill=fillColor] (203.78,522.02) circle (  1.16);

\path[draw=drawColor,line width= 0.4pt,line join=round,line cap=round,fill=fillColor] (203.83,522.02) circle (  1.16);

\path[draw=drawColor,line width= 0.4pt,line join=round,line cap=round,fill=fillColor] (203.89,522.01) circle (  1.16);

\path[draw=drawColor,line width= 0.4pt,line join=round,line cap=round,fill=fillColor] (203.94,521.97) circle (  1.16);

\path[draw=drawColor,line width= 0.4pt,line join=round,line cap=round,fill=fillColor] (203.99,521.96) circle (  1.16);

\path[draw=drawColor,line width= 0.4pt,line join=round,line cap=round,fill=fillColor] (204.04,521.91) circle (  1.16);

\path[draw=drawColor,line width= 0.4pt,line join=round,line cap=round,fill=fillColor] (204.10,521.89) circle (  1.16);

\path[draw=drawColor,line width= 0.4pt,line join=round,line cap=round,fill=fillColor] (204.15,521.87) circle (  1.16);

\path[draw=drawColor,line width= 0.4pt,line join=round,line cap=round,fill=fillColor] (204.20,521.83) circle (  1.16);

\path[draw=drawColor,line width= 0.4pt,line join=round,line cap=round,fill=fillColor] (204.25,521.83) circle (  1.16);

\path[draw=drawColor,line width= 0.4pt,line join=round,line cap=round,fill=fillColor] (204.30,521.80) circle (  1.16);

\path[draw=drawColor,line width= 0.4pt,line join=round,line cap=round,fill=fillColor] (204.36,521.75) circle (  1.16);

\path[draw=drawColor,line width= 0.4pt,line join=round,line cap=round,fill=fillColor] (204.41,521.70) circle (  1.16);

\path[draw=drawColor,line width= 0.4pt,line join=round,line cap=round,fill=fillColor] (204.46,521.62) circle (  1.16);

\path[draw=drawColor,line width= 0.4pt,line join=round,line cap=round,fill=fillColor] (204.51,521.59) circle (  1.16);

\path[draw=drawColor,line width= 0.4pt,line join=round,line cap=round,fill=fillColor] (204.56,521.58) circle (  1.16);

\path[draw=drawColor,line width= 0.4pt,line join=round,line cap=round,fill=fillColor] (204.62,521.46) circle (  1.16);

\path[draw=drawColor,line width= 0.4pt,line join=round,line cap=round,fill=fillColor] (204.67,521.37) circle (  1.16);

\path[draw=drawColor,line width= 0.4pt,line join=round,line cap=round,fill=fillColor] (204.72,521.31) circle (  1.16);

\path[draw=drawColor,line width= 0.4pt,line join=round,line cap=round,fill=fillColor] (204.77,521.29) circle (  1.16);

\path[draw=drawColor,line width= 0.4pt,line join=round,line cap=round,fill=fillColor] (204.82,521.28) circle (  1.16);

\path[draw=drawColor,line width= 0.4pt,line join=round,line cap=round,fill=fillColor] (204.87,521.23) circle (  1.16);

\path[draw=drawColor,line width= 0.4pt,line join=round,line cap=round,fill=fillColor] (204.93,521.18) circle (  1.16);

\path[draw=drawColor,line width= 0.4pt,line join=round,line cap=round,fill=fillColor] (204.98,521.08) circle (  1.16);

\path[draw=drawColor,line width= 0.4pt,line join=round,line cap=round,fill=fillColor] (205.03,521.06) circle (  1.16);

\path[draw=drawColor,line width= 0.4pt,line join=round,line cap=round,fill=fillColor] (205.08,521.01) circle (  1.16);

\path[draw=drawColor,line width= 0.4pt,line join=round,line cap=round,fill=fillColor] (205.13,520.96) circle (  1.16);

\path[draw=drawColor,line width= 0.4pt,line join=round,line cap=round,fill=fillColor] (205.18,520.91) circle (  1.16);

\path[draw=drawColor,line width= 0.4pt,line join=round,line cap=round,fill=fillColor] (205.23,520.89) circle (  1.16);

\path[draw=drawColor,line width= 0.4pt,line join=round,line cap=round,fill=fillColor] (205.29,520.83) circle (  1.16);

\path[draw=drawColor,line width= 0.4pt,line join=round,line cap=round,fill=fillColor] (205.34,520.76) circle (  1.16);

\path[draw=drawColor,line width= 0.4pt,line join=round,line cap=round,fill=fillColor] (205.39,520.75) circle (  1.16);

\path[draw=drawColor,line width= 0.4pt,line join=round,line cap=round,fill=fillColor] (205.44,520.72) circle (  1.16);

\path[draw=drawColor,line width= 0.4pt,line join=round,line cap=round,fill=fillColor] (205.49,520.67) circle (  1.16);

\path[draw=drawColor,line width= 0.4pt,line join=round,line cap=round,fill=fillColor] (205.54,520.59) circle (  1.16);

\path[draw=drawColor,line width= 0.4pt,line join=round,line cap=round,fill=fillColor] (205.59,520.57) circle (  1.16);

\path[draw=drawColor,line width= 0.4pt,line join=round,line cap=round,fill=fillColor] (205.64,520.52) circle (  1.16);

\path[draw=drawColor,line width= 0.4pt,line join=round,line cap=round,fill=fillColor] (205.69,520.52) circle (  1.16);

\path[draw=drawColor,line width= 0.4pt,line join=round,line cap=round,fill=fillColor] (205.75,520.29) circle (  1.16);

\path[draw=drawColor,line width= 0.4pt,line join=round,line cap=round,fill=fillColor] (205.80,520.27) circle (  1.16);

\path[draw=drawColor,line width= 0.4pt,line join=round,line cap=round,fill=fillColor] (205.85,520.26) circle (  1.16);

\path[draw=drawColor,line width= 0.4pt,line join=round,line cap=round,fill=fillColor] (205.90,520.26) circle (  1.16);

\path[draw=drawColor,line width= 0.4pt,line join=round,line cap=round,fill=fillColor] (205.95,520.21) circle (  1.16);

\path[draw=drawColor,line width= 0.4pt,line join=round,line cap=round,fill=fillColor] (206.00,520.14) circle (  1.16);

\path[draw=drawColor,line width= 0.4pt,line join=round,line cap=round,fill=fillColor] (206.05,520.01) circle (  1.16);

\path[draw=drawColor,line width= 0.4pt,line join=round,line cap=round,fill=fillColor] (206.10,519.92) circle (  1.16);

\path[draw=drawColor,line width= 0.4pt,line join=round,line cap=round,fill=fillColor] (206.15,519.91) circle (  1.16);

\path[draw=drawColor,line width= 0.4pt,line join=round,line cap=round,fill=fillColor] (206.20,519.85) circle (  1.16);

\path[draw=drawColor,line width= 0.4pt,line join=round,line cap=round,fill=fillColor] (206.25,519.83) circle (  1.16);

\path[draw=drawColor,line width= 0.4pt,line join=round,line cap=round,fill=fillColor] (206.30,519.81) circle (  1.16);

\path[draw=drawColor,line width= 0.4pt,line join=round,line cap=round,fill=fillColor] (206.35,519.75) circle (  1.16);

\path[draw=drawColor,line width= 0.4pt,line join=round,line cap=round,fill=fillColor] (206.40,519.65) circle (  1.16);

\path[draw=drawColor,line width= 0.4pt,line join=round,line cap=round,fill=fillColor] (206.45,519.65) circle (  1.16);

\path[draw=drawColor,line width= 0.4pt,line join=round,line cap=round,fill=fillColor] (206.50,519.62) circle (  1.16);

\path[draw=drawColor,line width= 0.4pt,line join=round,line cap=round,fill=fillColor] (206.55,519.52) circle (  1.16);

\path[draw=drawColor,line width= 0.4pt,line join=round,line cap=round,fill=fillColor] (206.61,519.50) circle (  1.16);

\path[draw=drawColor,line width= 0.4pt,line join=round,line cap=round,fill=fillColor] (206.66,519.32) circle (  1.16);

\path[draw=drawColor,line width= 0.4pt,line join=round,line cap=round,fill=fillColor] (206.71,519.26) circle (  1.16);

\path[draw=drawColor,line width= 0.4pt,line join=round,line cap=round,fill=fillColor] (206.76,519.13) circle (  1.16);

\path[draw=drawColor,line width= 0.4pt,line join=round,line cap=round,fill=fillColor] (206.81,519.03) circle (  1.16);

\path[draw=drawColor,line width= 0.4pt,line join=round,line cap=round,fill=fillColor] (206.86,518.89) circle (  1.16);

\path[draw=drawColor,line width= 0.4pt,line join=round,line cap=round,fill=fillColor] (206.91,518.83) circle (  1.16);

\path[draw=drawColor,line width= 0.4pt,line join=round,line cap=round,fill=fillColor] (206.96,518.81) circle (  1.16);

\path[draw=drawColor,line width= 0.4pt,line join=round,line cap=round,fill=fillColor] (207.01,518.81) circle (  1.16);

\path[draw=drawColor,line width= 0.4pt,line join=round,line cap=round,fill=fillColor] (207.06,518.80) circle (  1.16);

\path[draw=drawColor,line width= 0.4pt,line join=round,line cap=round,fill=fillColor] (207.11,518.71) circle (  1.16);

\path[draw=drawColor,line width= 0.4pt,line join=round,line cap=round,fill=fillColor] (207.16,518.68) circle (  1.16);

\path[draw=drawColor,line width= 0.4pt,line join=round,line cap=round,fill=fillColor] (207.21,518.66) circle (  1.16);

\path[draw=drawColor,line width= 0.4pt,line join=round,line cap=round,fill=fillColor] (207.26,518.58) circle (  1.16);

\path[draw=drawColor,line width= 0.4pt,line join=round,line cap=round,fill=fillColor] (207.31,518.55) circle (  1.16);

\path[draw=drawColor,line width= 0.4pt,line join=round,line cap=round,fill=fillColor] (207.36,518.52) circle (  1.16);

\path[draw=drawColor,line width= 0.4pt,line join=round,line cap=round,fill=fillColor] (207.41,518.44) circle (  1.16);

\path[draw=drawColor,line width= 0.4pt,line join=round,line cap=round,fill=fillColor] (207.46,518.44) circle (  1.16);

\path[draw=drawColor,line width= 0.4pt,line join=round,line cap=round,fill=fillColor] (207.50,518.37) circle (  1.16);

\path[draw=drawColor,line width= 0.4pt,line join=round,line cap=round,fill=fillColor] (207.55,518.34) circle (  1.16);

\path[draw=drawColor,line width= 0.4pt,line join=round,line cap=round,fill=fillColor] (207.60,518.30) circle (  1.16);

\path[draw=drawColor,line width= 0.4pt,line join=round,line cap=round,fill=fillColor] (207.65,518.30) circle (  1.16);

\path[draw=drawColor,line width= 0.4pt,line join=round,line cap=round,fill=fillColor] (207.70,518.20) circle (  1.16);

\path[draw=drawColor,line width= 0.4pt,line join=round,line cap=round,fill=fillColor] (207.75,518.16) circle (  1.16);

\path[draw=drawColor,line width= 0.4pt,line join=round,line cap=round,fill=fillColor] (207.80,517.97) circle (  1.16);

\path[draw=drawColor,line width= 0.4pt,line join=round,line cap=round,fill=fillColor] (207.85,517.91) circle (  1.16);

\path[draw=drawColor,line width= 0.4pt,line join=round,line cap=round,fill=fillColor] (207.90,517.78) circle (  1.16);

\path[draw=drawColor,line width= 0.4pt,line join=round,line cap=round,fill=fillColor] (207.95,517.53) circle (  1.16);

\path[draw=drawColor,line width= 0.4pt,line join=round,line cap=round,fill=fillColor] (208.00,517.52) circle (  1.16);

\path[draw=drawColor,line width= 0.4pt,line join=round,line cap=round,fill=fillColor] (208.05,517.43) circle (  1.16);

\path[draw=drawColor,line width= 0.4pt,line join=round,line cap=round,fill=fillColor] (208.10,517.27) circle (  1.16);

\path[draw=drawColor,line width= 0.4pt,line join=round,line cap=round,fill=fillColor] (208.15,517.23) circle (  1.16);

\path[draw=drawColor,line width= 0.4pt,line join=round,line cap=round,fill=fillColor] (208.20,517.13) circle (  1.16);

\path[draw=drawColor,line width= 0.4pt,line join=round,line cap=round,fill=fillColor] (208.25,516.77) circle (  1.16);

\path[draw=drawColor,line width= 0.4pt,line join=round,line cap=round,fill=fillColor] (208.29,516.70) circle (  1.16);

\path[draw=drawColor,line width= 0.4pt,line join=round,line cap=round,fill=fillColor] (208.34,516.70) circle (  1.16);

\path[draw=drawColor,line width= 0.4pt,line join=round,line cap=round,fill=fillColor] (208.39,516.62) circle (  1.16);

\path[draw=drawColor,line width= 0.4pt,line join=round,line cap=round,fill=fillColor] (208.44,516.42) circle (  1.16);

\path[draw=drawColor,line width= 0.4pt,line join=round,line cap=round,fill=fillColor] (208.49,516.35) circle (  1.16);

\path[draw=drawColor,line width= 0.4pt,line join=round,line cap=round,fill=fillColor] (208.54,516.11) circle (  1.16);

\path[draw=drawColor,line width= 0.4pt,line join=round,line cap=round,fill=fillColor] (208.59,516.10) circle (  1.16);

\path[draw=drawColor,line width= 0.4pt,line join=round,line cap=round,fill=fillColor] (208.64,516.07) circle (  1.16);

\path[draw=drawColor,line width= 0.4pt,line join=round,line cap=round,fill=fillColor] (208.69,515.94) circle (  1.16);

\path[draw=drawColor,line width= 0.4pt,line join=round,line cap=round,fill=fillColor] (208.74,515.92) circle (  1.16);

\path[draw=drawColor,line width= 0.4pt,line join=round,line cap=round,fill=fillColor] (208.78,515.85) circle (  1.16);

\path[draw=drawColor,line width= 0.4pt,line join=round,line cap=round,fill=fillColor] (208.83,515.83) circle (  1.16);

\path[draw=drawColor,line width= 0.4pt,line join=round,line cap=round,fill=fillColor] (208.88,515.61) circle (  1.16);

\path[draw=drawColor,line width= 0.4pt,line join=round,line cap=round,fill=fillColor] (208.93,515.47) circle (  1.16);

\path[draw=drawColor,line width= 0.4pt,line join=round,line cap=round,fill=fillColor] (208.98,514.97) circle (  1.16);

\path[draw=drawColor,line width= 0.4pt,line join=round,line cap=round,fill=fillColor] (209.03,514.87) circle (  1.16);

\path[draw=drawColor,line width= 0.4pt,line join=round,line cap=round,fill=fillColor] (209.08,514.64) circle (  1.16);

\path[draw=drawColor,line width= 0.4pt,line join=round,line cap=round,fill=fillColor] (209.12,514.45) circle (  1.16);

\path[draw=drawColor,line width= 0.4pt,line join=round,line cap=round,fill=fillColor] (209.17,514.25) circle (  1.16);

\path[draw=drawColor,line width= 0.4pt,line join=round,line cap=round,fill=fillColor] (209.22,513.28) circle (  1.16);

\path[draw=drawColor,line width= 0.4pt,line join=round,line cap=round,fill=fillColor] (209.27,513.04) circle (  1.16);

\path[draw=drawColor,line width= 0.4pt,line join=round,line cap=round,fill=fillColor] (209.32,512.79) circle (  1.16);

\path[draw=drawColor,line width= 0.4pt,line join=round,line cap=round,fill=fillColor] (209.37,512.49) circle (  1.16);

\path[draw=drawColor,line width= 0.4pt,line join=round,line cap=round,fill=fillColor] (209.42,508.31) circle (  1.16);

\path[draw=drawColor,line width= 0.4pt,line join=round,line cap=round,fill=fillColor] (209.46,508.31) circle (  1.16);

\path[draw=drawColor,line width= 0.4pt,line join=round,line cap=round,fill=fillColor] (209.51,508.31) circle (  1.16);

\path[draw=drawColor,line width= 0.4pt,line join=round,line cap=round,fill=fillColor] (209.56,508.31) circle (  1.16);

\path[draw=drawColor,line width= 0.4pt,line join=round,line cap=round,fill=fillColor] (209.61,508.31) circle (  1.16);

\path[draw=drawColor,line width= 0.4pt,line join=round,line cap=round,fill=fillColor] (209.66,508.31) circle (  1.16);

\path[draw=drawColor,line width= 0.4pt,line join=round,line cap=round,fill=fillColor] (209.70,508.31) circle (  1.16);

\path[draw=drawColor,line width= 0.4pt,line join=round,line cap=round,fill=fillColor] (209.75,508.31) circle (  1.16);

\path[draw=drawColor,line width= 0.4pt,line join=round,line cap=round,fill=fillColor] (209.80,508.31) circle (  1.16);

\path[draw=drawColor,line width= 0.4pt,line join=round,line cap=round,fill=fillColor] (209.85,508.31) circle (  1.16);

\path[draw=drawColor,line width= 0.4pt,line join=round,line cap=round,fill=fillColor] (209.90,508.31) circle (  1.16);

\path[draw=drawColor,line width= 0.4pt,line join=round,line cap=round,fill=fillColor] (209.94,508.31) circle (  1.16);

\path[draw=drawColor,line width= 0.4pt,line join=round,line cap=round,fill=fillColor] (209.99,508.31) circle (  1.16);

\path[draw=drawColor,line width= 0.4pt,line join=round,line cap=round,fill=fillColor] (210.04,508.31) circle (  1.16);

\path[draw=drawColor,line width= 0.4pt,line join=round,line cap=round,fill=fillColor] (210.09,508.31) circle (  1.16);

\path[draw=drawColor,line width= 0.4pt,line join=round,line cap=round,fill=fillColor] (210.14,508.31) circle (  1.16);

\path[draw=drawColor,line width= 0.4pt,line join=round,line cap=round,fill=fillColor] (210.18,508.31) circle (  1.16);

\path[draw=drawColor,line width= 0.4pt,line join=round,line cap=round,fill=fillColor] (210.23,508.31) circle (  1.16);

\path[draw=drawColor,line width= 0.4pt,line join=round,line cap=round,fill=fillColor] (210.28,508.31) circle (  1.16);

\path[draw=drawColor,line width= 0.4pt,line join=round,line cap=round,fill=fillColor] (210.33,508.31) circle (  1.16);

\path[draw=drawColor,line width= 0.4pt,line join=round,line cap=round,fill=fillColor] (210.38,508.31) circle (  1.16);

\path[draw=drawColor,line width= 0.4pt,line join=round,line cap=round,fill=fillColor] (210.42,508.31) circle (  1.16);

\path[draw=drawColor,line width= 0.4pt,line join=round,line cap=round,fill=fillColor] (210.47,508.31) circle (  1.16);

\path[draw=drawColor,line width= 0.4pt,line join=round,line cap=round,fill=fillColor] (210.52,508.31) circle (  1.16);

\path[draw=drawColor,line width= 0.4pt,line join=round,line cap=round,fill=fillColor] (210.57,508.31) circle (  1.16);

\path[draw=drawColor,line width= 0.4pt,line join=round,line cap=round,fill=fillColor] (210.61,508.31) circle (  1.16);

\path[draw=drawColor,line width= 0.4pt,line join=round,line cap=round,fill=fillColor] (210.66,508.31) circle (  1.16);

\path[draw=drawColor,line width= 0.4pt,line join=round,line cap=round,fill=fillColor] (210.71,508.31) circle (  1.16);

\path[draw=drawColor,line width= 0.4pt,line join=round,line cap=round,fill=fillColor] (210.76,508.31) circle (  1.16);

\path[draw=drawColor,line width= 0.4pt,line join=round,line cap=round,fill=fillColor] (210.80,508.31) circle (  1.16);

\path[draw=drawColor,line width= 0.4pt,line join=round,line cap=round,fill=fillColor] (210.85,508.31) circle (  1.16);

\path[draw=drawColor,line width= 0.4pt,line join=round,line cap=round,fill=fillColor] (210.90,508.31) circle (  1.16);

\path[draw=drawColor,line width= 0.4pt,line join=round,line cap=round,fill=fillColor] (210.95,508.31) circle (  1.16);

\path[draw=drawColor,line width= 0.4pt,line join=round,line cap=round,fill=fillColor] (210.99,508.31) circle (  1.16);

\path[draw=drawColor,line width= 0.4pt,line join=round,line cap=round,fill=fillColor] (211.04,508.31) circle (  1.16);

\path[draw=drawColor,line width= 0.4pt,line join=round,line cap=round,fill=fillColor] (211.09,508.31) circle (  1.16);

\path[draw=drawColor,line width= 0.4pt,line join=round,line cap=round,fill=fillColor] (211.13,508.31) circle (  1.16);

\path[draw=drawColor,line width= 0.4pt,line join=round,line cap=round,fill=fillColor] (211.18,508.31) circle (  1.16);

\path[draw=drawColor,line width= 0.4pt,line join=round,line cap=round,fill=fillColor] (211.23,508.31) circle (  1.16);

\path[draw=drawColor,line width= 0.4pt,line join=round,line cap=round,fill=fillColor] (211.28,508.31) circle (  1.16);

\path[draw=drawColor,line width= 0.4pt,line join=round,line cap=round,fill=fillColor] (211.32,508.31) circle (  1.16);

\path[draw=drawColor,line width= 0.4pt,line join=round,line cap=round,fill=fillColor] (211.37,508.31) circle (  1.16);

\path[draw=drawColor,line width= 0.4pt,line join=round,line cap=round,fill=fillColor] (211.42,508.31) circle (  1.16);

\path[draw=drawColor,line width= 0.4pt,line join=round,line cap=round,fill=fillColor] (211.46,508.31) circle (  1.16);

\path[draw=drawColor,line width= 0.4pt,line join=round,line cap=round,fill=fillColor] (211.51,508.31) circle (  1.16);

\path[draw=drawColor,line width= 0.4pt,line join=round,line cap=round,fill=fillColor] (211.56,508.31) circle (  1.16);

\path[draw=drawColor,line width= 0.4pt,line join=round,line cap=round,fill=fillColor] (211.60,508.31) circle (  1.16);

\path[draw=drawColor,line width= 0.4pt,line join=round,line cap=round,fill=fillColor] (211.65,508.31) circle (  1.16);

\path[draw=drawColor,line width= 0.4pt,line join=round,line cap=round,fill=fillColor] (211.70,508.31) circle (  1.16);

\path[draw=drawColor,line width= 0.4pt,line join=round,line cap=round,fill=fillColor] (211.75,508.31) circle (  1.16);

\path[draw=drawColor,line width= 0.4pt,line join=round,line cap=round,fill=fillColor] (211.79,508.31) circle (  1.16);

\path[draw=drawColor,line width= 0.4pt,line join=round,line cap=round,fill=fillColor] (211.84,508.31) circle (  1.16);

\path[draw=drawColor,line width= 0.4pt,line join=round,line cap=round,fill=fillColor] (211.89,508.31) circle (  1.16);

\path[draw=drawColor,line width= 0.4pt,line join=round,line cap=round,fill=fillColor] (211.93,508.31) circle (  1.16);

\path[draw=drawColor,line width= 0.4pt,line join=round,line cap=round,fill=fillColor] (211.98,508.31) circle (  1.16);

\path[draw=drawColor,line width= 0.4pt,line join=round,line cap=round,fill=fillColor] (212.03,508.31) circle (  1.16);

\path[draw=drawColor,line width= 0.4pt,line join=round,line cap=round,fill=fillColor] (212.07,508.31) circle (  1.16);

\path[draw=drawColor,line width= 0.4pt,line join=round,line cap=round,fill=fillColor] (212.12,508.31) circle (  1.16);

\path[draw=drawColor,line width= 0.4pt,line join=round,line cap=round,fill=fillColor] (212.16,508.31) circle (  1.16);

\path[draw=drawColor,line width= 0.4pt,line join=round,line cap=round,fill=fillColor] (212.21,508.31) circle (  1.16);

\path[draw=drawColor,line width= 0.4pt,line join=round,line cap=round,fill=fillColor] (212.26,508.31) circle (  1.16);

\path[draw=drawColor,line width= 0.4pt,line join=round,line cap=round,fill=fillColor] (212.30,508.31) circle (  1.16);

\path[draw=drawColor,line width= 0.4pt,line join=round,line cap=round,fill=fillColor] (212.35,508.31) circle (  1.16);

\path[draw=drawColor,line width= 0.4pt,line join=round,line cap=round,fill=fillColor] (212.40,508.31) circle (  1.16);

\path[draw=drawColor,line width= 0.4pt,line join=round,line cap=round,fill=fillColor] (212.44,508.31) circle (  1.16);

\path[draw=drawColor,line width= 0.4pt,line join=round,line cap=round,fill=fillColor] (212.49,508.31) circle (  1.16);

\path[draw=drawColor,line width= 0.4pt,line join=round,line cap=round,fill=fillColor] (212.54,508.31) circle (  1.16);

\path[draw=drawColor,line width= 0.4pt,line join=round,line cap=round,fill=fillColor] (212.58,508.31) circle (  1.16);

\path[draw=drawColor,line width= 0.4pt,line join=round,line cap=round,fill=fillColor] (212.63,508.31) circle (  1.16);

\path[draw=drawColor,line width= 0.4pt,line join=round,line cap=round,fill=fillColor] (212.67,508.31) circle (  1.16);

\path[draw=drawColor,line width= 0.4pt,line join=round,line cap=round,fill=fillColor] (212.72,508.31) circle (  1.16);

\path[draw=drawColor,line width= 0.4pt,line join=round,line cap=round,fill=fillColor] (212.77,508.31) circle (  1.16);

\path[draw=drawColor,line width= 0.4pt,line join=round,line cap=round,fill=fillColor] (212.81,508.31) circle (  1.16);

\path[draw=drawColor,line width= 0.4pt,line join=round,line cap=round,fill=fillColor] (212.86,508.31) circle (  1.16);

\path[draw=drawColor,line width= 0.4pt,line join=round,line cap=round,fill=fillColor] (212.91,508.31) circle (  1.16);

\path[draw=drawColor,line width= 0.4pt,line join=round,line cap=round,fill=fillColor] (212.95,508.31) circle (  1.16);

\path[draw=drawColor,line width= 0.4pt,line join=round,line cap=round,fill=fillColor] (213.00,508.31) circle (  1.16);

\path[draw=drawColor,line width= 0.4pt,line join=round,line cap=round,fill=fillColor] (213.04,508.31) circle (  1.16);

\path[draw=drawColor,line width= 0.4pt,line join=round,line cap=round,fill=fillColor] (213.09,508.31) circle (  1.16);

\path[draw=drawColor,line width= 0.4pt,line join=round,line cap=round,fill=fillColor] (213.14,508.31) circle (  1.16);

\path[draw=drawColor,line width= 0.4pt,line join=round,line cap=round,fill=fillColor] (213.18,508.31) circle (  1.16);

\path[draw=drawColor,line width= 0.4pt,line join=round,line cap=round,fill=fillColor] (213.23,508.31) circle (  1.16);

\path[draw=drawColor,line width= 0.4pt,line join=round,line cap=round,fill=fillColor] (213.27,508.31) circle (  1.16);

\path[draw=drawColor,line width= 0.4pt,line join=round,line cap=round,fill=fillColor] (213.32,508.31) circle (  1.16);

\path[draw=drawColor,line width= 0.4pt,line join=round,line cap=round,fill=fillColor] (213.36,508.31) circle (  1.16);

\path[draw=drawColor,line width= 0.4pt,line join=round,line cap=round,fill=fillColor] (213.41,508.31) circle (  1.16);

\path[draw=drawColor,line width= 0.4pt,line join=round,line cap=round,fill=fillColor] (213.46,508.31) circle (  1.16);

\path[draw=drawColor,line width= 0.4pt,line join=round,line cap=round,fill=fillColor] (213.50,508.31) circle (  1.16);

\path[draw=drawColor,line width= 0.4pt,line join=round,line cap=round,fill=fillColor] (213.55,508.31) circle (  1.16);

\path[draw=drawColor,line width= 0.4pt,line join=round,line cap=round,fill=fillColor] (213.59,508.31) circle (  1.16);

\path[draw=drawColor,line width= 0.4pt,line join=round,line cap=round,fill=fillColor] (213.64,508.31) circle (  1.16);

\path[draw=drawColor,line width= 0.4pt,line join=round,line cap=round,fill=fillColor] (213.68,508.31) circle (  1.16);

\path[draw=drawColor,line width= 0.4pt,line join=round,line cap=round,fill=fillColor] (213.73,508.31) circle (  1.16);

\path[draw=drawColor,line width= 0.4pt,line join=round,line cap=round,fill=fillColor] (213.78,508.31) circle (  1.16);

\path[draw=drawColor,line width= 0.4pt,line join=round,line cap=round,fill=fillColor] (213.82,508.31) circle (  1.16);

\path[draw=drawColor,line width= 0.4pt,line join=round,line cap=round,fill=fillColor] (213.87,508.31) circle (  1.16);

\path[draw=drawColor,line width= 0.4pt,line join=round,line cap=round,fill=fillColor] (213.91,508.31) circle (  1.16);

\path[draw=drawColor,line width= 0.4pt,line join=round,line cap=round,fill=fillColor] (213.96,508.31) circle (  1.16);

\path[draw=drawColor,line width= 0.4pt,line join=round,line cap=round,fill=fillColor] (214.00,508.31) circle (  1.16);

\path[draw=drawColor,line width= 0.4pt,line join=round,line cap=round,fill=fillColor] (214.05,508.31) circle (  1.16);

\path[draw=drawColor,line width= 0.4pt,line join=round,line cap=round,fill=fillColor] (214.09,508.31) circle (  1.16);

\path[draw=drawColor,line width= 0.4pt,line join=round,line cap=round,fill=fillColor] (214.14,508.31) circle (  1.16);

\path[draw=drawColor,line width= 0.4pt,line join=round,line cap=round,fill=fillColor] (214.18,508.31) circle (  1.16);

\path[draw=drawColor,line width= 0.4pt,line join=round,line cap=round,fill=fillColor] (214.23,508.31) circle (  1.16);

\path[draw=drawColor,line width= 0.4pt,line join=round,line cap=round,fill=fillColor] (214.27,508.31) circle (  1.16);

\path[draw=drawColor,line width= 0.4pt,line join=round,line cap=round,fill=fillColor] (214.32,508.31) circle (  1.16);

\path[draw=drawColor,line width= 0.4pt,line join=round,line cap=round,fill=fillColor] (214.36,508.31) circle (  1.16);

\path[draw=drawColor,line width= 0.4pt,line join=round,line cap=round,fill=fillColor] (214.41,508.31) circle (  1.16);

\path[draw=drawColor,line width= 0.4pt,line join=round,line cap=round,fill=fillColor] (214.45,508.31) circle (  1.16);

\path[draw=drawColor,line width= 0.4pt,line join=round,line cap=round,fill=fillColor] (214.50,508.31) circle (  1.16);

\path[draw=drawColor,line width= 0.4pt,line join=round,line cap=round,fill=fillColor] (214.54,508.31) circle (  1.16);

\path[draw=drawColor,line width= 0.4pt,line join=round,line cap=round,fill=fillColor] (214.59,508.31) circle (  1.16);

\path[draw=drawColor,line width= 0.4pt,line join=round,line cap=round,fill=fillColor] (214.63,508.31) circle (  1.16);

\path[draw=drawColor,line width= 0.4pt,line join=round,line cap=round,fill=fillColor] (214.68,508.31) circle (  1.16);

\path[draw=drawColor,line width= 0.4pt,line join=round,line cap=round,fill=fillColor] (214.72,508.31) circle (  1.16);

\path[draw=drawColor,line width= 0.4pt,line join=round,line cap=round,fill=fillColor] (214.77,508.31) circle (  1.16);

\path[draw=drawColor,line width= 0.4pt,line join=round,line cap=round,fill=fillColor] (214.81,508.31) circle (  1.16);

\path[draw=drawColor,line width= 0.4pt,line join=round,line cap=round,fill=fillColor] (214.86,508.31) circle (  1.16);

\path[draw=drawColor,line width= 0.4pt,line join=round,line cap=round,fill=fillColor] (214.90,508.31) circle (  1.16);

\path[draw=drawColor,line width= 0.4pt,line join=round,line cap=round,fill=fillColor] (214.95,508.31) circle (  1.16);

\path[draw=drawColor,line width= 0.4pt,line join=round,line cap=round,fill=fillColor] (214.99,508.31) circle (  1.16);

\path[draw=drawColor,line width= 0.4pt,line join=round,line cap=round,fill=fillColor] (215.04,508.31) circle (  1.16);

\path[draw=drawColor,line width= 0.4pt,line join=round,line cap=round,fill=fillColor] (215.08,508.31) circle (  1.16);

\path[draw=drawColor,line width= 0.4pt,line join=round,line cap=round,fill=fillColor] (215.13,508.31) circle (  1.16);

\path[draw=drawColor,line width= 0.4pt,line join=round,line cap=round,fill=fillColor] (215.17,508.31) circle (  1.16);

\path[draw=drawColor,line width= 0.4pt,line join=round,line cap=round,fill=fillColor] (215.22,508.31) circle (  1.16);

\path[draw=drawColor,line width= 0.4pt,line join=round,line cap=round,fill=fillColor] (215.26,508.31) circle (  1.16);

\path[draw=drawColor,line width= 0.4pt,line join=round,line cap=round,fill=fillColor] (215.31,508.31) circle (  1.16);

\path[draw=drawColor,line width= 0.4pt,line join=round,line cap=round,fill=fillColor] (215.35,508.31) circle (  1.16);

\path[draw=drawColor,line width= 0.4pt,line join=round,line cap=round,fill=fillColor] (215.40,508.31) circle (  1.16);

\path[draw=drawColor,line width= 0.4pt,line join=round,line cap=round,fill=fillColor] (215.44,508.31) circle (  1.16);

\path[draw=drawColor,line width= 0.4pt,line join=round,line cap=round,fill=fillColor] (215.48,508.31) circle (  1.16);

\path[draw=drawColor,line width= 0.4pt,line join=round,line cap=round,fill=fillColor] (215.53,508.31) circle (  1.16);

\path[draw=drawColor,line width= 0.4pt,line join=round,line cap=round,fill=fillColor] (215.57,508.31) circle (  1.16);

\path[draw=drawColor,line width= 0.4pt,line join=round,line cap=round,fill=fillColor] (215.62,508.31) circle (  1.16);

\path[draw=drawColor,line width= 0.4pt,line join=round,line cap=round,fill=fillColor] (215.66,508.31) circle (  1.16);

\path[draw=drawColor,line width= 0.4pt,line join=round,line cap=round,fill=fillColor] (215.71,508.31) circle (  1.16);

\path[draw=drawColor,line width= 0.4pt,line join=round,line cap=round,fill=fillColor] (215.75,508.31) circle (  1.16);

\path[draw=drawColor,line width= 0.4pt,line join=round,line cap=round,fill=fillColor] (215.79,508.31) circle (  1.16);

\path[draw=drawColor,line width= 0.4pt,line join=round,line cap=round,fill=fillColor] (215.84,508.31) circle (  1.16);

\path[draw=drawColor,line width= 0.4pt,line join=round,line cap=round,fill=fillColor] (215.88,508.31) circle (  1.16);

\path[draw=drawColor,line width= 0.4pt,line join=round,line cap=round,fill=fillColor] (215.93,508.31) circle (  1.16);

\path[draw=drawColor,line width= 0.4pt,line join=round,line cap=round,fill=fillColor] (215.97,508.31) circle (  1.16);

\path[draw=drawColor,line width= 0.4pt,line join=round,line cap=round,fill=fillColor] (216.02,508.31) circle (  1.16);

\path[draw=drawColor,line width= 0.4pt,line join=round,line cap=round,fill=fillColor] (216.06,508.31) circle (  1.16);

\path[draw=drawColor,line width= 0.4pt,line join=round,line cap=round,fill=fillColor] (216.10,508.31) circle (  1.16);

\path[draw=drawColor,line width= 0.4pt,line join=round,line cap=round,fill=fillColor] (216.15,508.31) circle (  1.16);

\path[draw=drawColor,line width= 0.4pt,line join=round,line cap=round,fill=fillColor] (216.19,508.31) circle (  1.16);

\path[draw=drawColor,line width= 0.4pt,line join=round,line cap=round,fill=fillColor] (216.24,508.31) circle (  1.16);

\path[draw=drawColor,line width= 0.4pt,line join=round,line cap=round,fill=fillColor] (216.28,508.31) circle (  1.16);

\path[draw=drawColor,line width= 0.4pt,line join=round,line cap=round,fill=fillColor] (216.32,508.31) circle (  1.16);

\path[draw=drawColor,line width= 0.4pt,line join=round,line cap=round,fill=fillColor] (216.37,508.31) circle (  1.16);

\path[draw=drawColor,line width= 0.4pt,line join=round,line cap=round,fill=fillColor] (216.41,508.31) circle (  1.16);

\path[draw=drawColor,line width= 0.4pt,line join=round,line cap=round,fill=fillColor] (216.46,508.31) circle (  1.16);

\path[draw=drawColor,line width= 0.4pt,line join=round,line cap=round,fill=fillColor] (216.50,508.31) circle (  1.16);

\path[draw=drawColor,line width= 0.4pt,line join=round,line cap=round,fill=fillColor] (216.54,508.31) circle (  1.16);

\path[draw=drawColor,line width= 0.4pt,line join=round,line cap=round,fill=fillColor] (216.59,508.31) circle (  1.16);

\path[draw=drawColor,line width= 0.4pt,line join=round,line cap=round,fill=fillColor] (216.63,508.31) circle (  1.16);

\path[draw=drawColor,line width= 0.4pt,line join=round,line cap=round,fill=fillColor] (216.68,508.31) circle (  1.16);

\path[draw=drawColor,line width= 0.4pt,line join=round,line cap=round,fill=fillColor] (216.72,508.31) circle (  1.16);

\path[draw=drawColor,line width= 0.4pt,line join=round,line cap=round,fill=fillColor] (216.76,508.31) circle (  1.16);

\path[draw=drawColor,line width= 0.4pt,line join=round,line cap=round,fill=fillColor] (216.81,508.31) circle (  1.16);

\path[draw=drawColor,line width= 0.4pt,line join=round,line cap=round,fill=fillColor] (216.85,508.31) circle (  1.16);
\definecolor[named]{drawColor}{rgb}{0.65,0.34,0.16}
\definecolor[named]{fillColor}{rgb}{0.65,0.34,0.16}

\path[draw=drawColor,line width= 0.4pt,line join=round,line cap=round,fill=fillColor] ( 81.22,563.62) circle (  1.16);

\path[draw=drawColor,line width= 0.4pt,line join=round,line cap=round,fill=fillColor] ( 84.95,546.41) circle (  1.16);

\path[draw=drawColor,line width= 0.4pt,line join=round,line cap=round,fill=fillColor] ( 87.56,534.78) circle (  1.16);

\path[draw=drawColor,line width= 0.4pt,line join=round,line cap=round,fill=fillColor] ( 89.64,534.49) circle (  1.16);

\path[draw=drawColor,line width= 0.4pt,line join=round,line cap=round,fill=fillColor] ( 91.40,533.36) circle (  1.16);

\path[draw=drawColor,line width= 0.4pt,line join=round,line cap=round,fill=fillColor] ( 92.94,532.69) circle (  1.16);

\path[draw=drawColor,line width= 0.4pt,line join=round,line cap=round,fill=fillColor] ( 94.31,532.53) circle (  1.16);

\path[draw=drawColor,line width= 0.4pt,line join=round,line cap=round,fill=fillColor] ( 95.56,532.36) circle (  1.16);

\path[draw=drawColor,line width= 0.4pt,line join=round,line cap=round,fill=fillColor] ( 96.71,531.94) circle (  1.16);

\path[draw=drawColor,line width= 0.4pt,line join=round,line cap=round,fill=fillColor] ( 97.77,531.68) circle (  1.16);

\path[draw=drawColor,line width= 0.4pt,line join=round,line cap=round,fill=fillColor] ( 98.77,531.50) circle (  1.16);

\path[draw=drawColor,line width= 0.4pt,line join=round,line cap=round,fill=fillColor] ( 99.71,531.35) circle (  1.16);

\path[draw=drawColor,line width= 0.4pt,line join=round,line cap=round,fill=fillColor] (100.59,530.56) circle (  1.16);

\path[draw=drawColor,line width= 0.4pt,line join=round,line cap=round,fill=fillColor] (101.44,530.39) circle (  1.16);

\path[draw=drawColor,line width= 0.4pt,line join=round,line cap=round,fill=fillColor] (102.24,530.35) circle (  1.16);

\path[draw=drawColor,line width= 0.4pt,line join=round,line cap=round,fill=fillColor] (103.01,530.07) circle (  1.16);

\path[draw=drawColor,line width= 0.4pt,line join=round,line cap=round,fill=fillColor] (103.75,529.86) circle (  1.16);

\path[draw=drawColor,line width= 0.4pt,line join=round,line cap=round,fill=fillColor] (104.46,529.48) circle (  1.16);

\path[draw=drawColor,line width= 0.4pt,line join=round,line cap=round,fill=fillColor] (105.14,529.45) circle (  1.16);

\path[draw=drawColor,line width= 0.4pt,line join=round,line cap=round,fill=fillColor] (105.80,529.33) circle (  1.16);

\path[draw=drawColor,line width= 0.4pt,line join=round,line cap=round,fill=fillColor] (106.44,529.26) circle (  1.16);

\path[draw=drawColor,line width= 0.4pt,line join=round,line cap=round,fill=fillColor] (107.05,529.02) circle (  1.16);

\path[draw=drawColor,line width= 0.4pt,line join=round,line cap=round,fill=fillColor] (107.65,529.01) circle (  1.16);

\path[draw=drawColor,line width= 0.4pt,line join=round,line cap=round,fill=fillColor] (108.24,528.93) circle (  1.16);

\path[draw=drawColor,line width= 0.4pt,line join=round,line cap=round,fill=fillColor] (108.80,528.79) circle (  1.16);

\path[draw=drawColor,line width= 0.4pt,line join=round,line cap=round,fill=fillColor] (109.35,528.36) circle (  1.16);

\path[draw=drawColor,line width= 0.4pt,line join=round,line cap=round,fill=fillColor] (109.89,528.14) circle (  1.16);

\path[draw=drawColor,line width= 0.4pt,line join=round,line cap=round,fill=fillColor] (110.42,527.46) circle (  1.16);

\path[draw=drawColor,line width= 0.4pt,line join=round,line cap=round,fill=fillColor] (110.93,527.28) circle (  1.16);

\path[draw=drawColor,line width= 0.4pt,line join=round,line cap=round,fill=fillColor] (111.43,527.13) circle (  1.16);

\path[draw=drawColor,line width= 0.4pt,line join=round,line cap=round,fill=fillColor] (111.92,526.79) circle (  1.16);

\path[draw=drawColor,line width= 0.4pt,line join=round,line cap=round,fill=fillColor] (112.40,526.73) circle (  1.16);

\path[draw=drawColor,line width= 0.4pt,line join=round,line cap=round,fill=fillColor] (112.87,526.55) circle (  1.16);

\path[draw=drawColor,line width= 0.4pt,line join=round,line cap=round,fill=fillColor] (113.33,526.49) circle (  1.16);

\path[draw=drawColor,line width= 0.4pt,line join=round,line cap=round,fill=fillColor] (113.78,526.46) circle (  1.16);

\path[draw=drawColor,line width= 0.4pt,line join=round,line cap=round,fill=fillColor] (114.22,526.35) circle (  1.16);

\path[draw=drawColor,line width= 0.4pt,line join=round,line cap=round,fill=fillColor] (114.65,526.12) circle (  1.16);

\path[draw=drawColor,line width= 0.4pt,line join=round,line cap=round,fill=fillColor] (115.08,525.99) circle (  1.16);

\path[draw=drawColor,line width= 0.4pt,line join=round,line cap=round,fill=fillColor] (115.50,525.96) circle (  1.16);

\path[draw=drawColor,line width= 0.4pt,line join=round,line cap=round,fill=fillColor] (115.91,525.92) circle (  1.16);

\path[draw=drawColor,line width= 0.4pt,line join=round,line cap=round,fill=fillColor] (116.32,525.84) circle (  1.16);

\path[draw=drawColor,line width= 0.4pt,line join=round,line cap=round,fill=fillColor] (116.72,525.79) circle (  1.16);

\path[draw=drawColor,line width= 0.4pt,line join=round,line cap=round,fill=fillColor] (117.11,525.74) circle (  1.16);

\path[draw=drawColor,line width= 0.4pt,line join=round,line cap=round,fill=fillColor] (117.49,525.73) circle (  1.16);

\path[draw=drawColor,line width= 0.4pt,line join=round,line cap=round,fill=fillColor] (117.87,525.64) circle (  1.16);

\path[draw=drawColor,line width= 0.4pt,line join=round,line cap=round,fill=fillColor] (118.25,525.61) circle (  1.16);

\path[draw=drawColor,line width= 0.4pt,line join=round,line cap=round,fill=fillColor] (118.62,525.57) circle (  1.16);

\path[draw=drawColor,line width= 0.4pt,line join=round,line cap=round,fill=fillColor] (118.98,525.44) circle (  1.16);

\path[draw=drawColor,line width= 0.4pt,line join=round,line cap=round,fill=fillColor] (119.34,525.38) circle (  1.16);

\path[draw=drawColor,line width= 0.4pt,line join=round,line cap=round,fill=fillColor] (119.70,525.28) circle (  1.16);

\path[draw=drawColor,line width= 0.4pt,line join=round,line cap=round,fill=fillColor] (120.05,525.18) circle (  1.16);

\path[draw=drawColor,line width= 0.4pt,line join=round,line cap=round,fill=fillColor] (120.39,525.06) circle (  1.16);

\path[draw=drawColor,line width= 0.4pt,line join=round,line cap=round,fill=fillColor] (120.73,525.06) circle (  1.16);

\path[draw=drawColor,line width= 0.4pt,line join=round,line cap=round,fill=fillColor] (121.07,525.02) circle (  1.16);

\path[draw=drawColor,line width= 0.4pt,line join=round,line cap=round,fill=fillColor] (121.40,524.99) circle (  1.16);

\path[draw=drawColor,line width= 0.4pt,line join=round,line cap=round,fill=fillColor] (121.73,524.98) circle (  1.16);

\path[draw=drawColor,line width= 0.4pt,line join=round,line cap=round,fill=fillColor] (122.05,524.94) circle (  1.16);

\path[draw=drawColor,line width= 0.4pt,line join=round,line cap=round,fill=fillColor] (122.38,524.88) circle (  1.16);

\path[draw=drawColor,line width= 0.4pt,line join=round,line cap=round,fill=fillColor] (122.69,524.88) circle (  1.16);

\path[draw=drawColor,line width= 0.4pt,line join=round,line cap=round,fill=fillColor] (123.01,524.78) circle (  1.16);

\path[draw=drawColor,line width= 0.4pt,line join=round,line cap=round,fill=fillColor] (123.32,524.76) circle (  1.16);

\path[draw=drawColor,line width= 0.4pt,line join=round,line cap=round,fill=fillColor] (123.62,524.74) circle (  1.16);

\path[draw=drawColor,line width= 0.4pt,line join=round,line cap=round,fill=fillColor] (123.93,524.67) circle (  1.16);

\path[draw=drawColor,line width= 0.4pt,line join=round,line cap=round,fill=fillColor] (124.23,524.65) circle (  1.16);

\path[draw=drawColor,line width= 0.4pt,line join=round,line cap=round,fill=fillColor] (124.52,524.62) circle (  1.16);

\path[draw=drawColor,line width= 0.4pt,line join=round,line cap=round,fill=fillColor] (124.82,524.61) circle (  1.16);

\path[draw=drawColor,line width= 0.4pt,line join=round,line cap=round,fill=fillColor] (125.11,524.60) circle (  1.16);

\path[draw=drawColor,line width= 0.4pt,line join=round,line cap=round,fill=fillColor] (125.40,524.59) circle (  1.16);

\path[draw=drawColor,line width= 0.4pt,line join=round,line cap=round,fill=fillColor] (125.68,524.49) circle (  1.16);

\path[draw=drawColor,line width= 0.4pt,line join=round,line cap=round,fill=fillColor] (125.96,524.44) circle (  1.16);

\path[draw=drawColor,line width= 0.4pt,line join=round,line cap=round,fill=fillColor] (126.24,524.36) circle (  1.16);

\path[draw=drawColor,line width= 0.4pt,line join=round,line cap=round,fill=fillColor] (126.52,524.34) circle (  1.16);

\path[draw=drawColor,line width= 0.4pt,line join=round,line cap=round,fill=fillColor] (126.80,524.20) circle (  1.16);

\path[draw=drawColor,line width= 0.4pt,line join=round,line cap=round,fill=fillColor] (127.07,524.18) circle (  1.16);

\path[draw=drawColor,line width= 0.4pt,line join=round,line cap=round,fill=fillColor] (127.34,524.16) circle (  1.16);

\path[draw=drawColor,line width= 0.4pt,line join=round,line cap=round,fill=fillColor] (127.61,524.06) circle (  1.16);

\path[draw=drawColor,line width= 0.4pt,line join=round,line cap=round,fill=fillColor] (127.87,524.05) circle (  1.16);

\path[draw=drawColor,line width= 0.4pt,line join=round,line cap=round,fill=fillColor] (128.13,524.05) circle (  1.16);

\path[draw=drawColor,line width= 0.4pt,line join=round,line cap=round,fill=fillColor] (128.40,524.00) circle (  1.16);

\path[draw=drawColor,line width= 0.4pt,line join=round,line cap=round,fill=fillColor] (128.65,523.96) circle (  1.16);

\path[draw=drawColor,line width= 0.4pt,line join=round,line cap=round,fill=fillColor] (128.91,523.92) circle (  1.16);

\path[draw=drawColor,line width= 0.4pt,line join=round,line cap=round,fill=fillColor] (129.16,523.92) circle (  1.16);

\path[draw=drawColor,line width= 0.4pt,line join=round,line cap=round,fill=fillColor] (129.42,523.92) circle (  1.16);

\path[draw=drawColor,line width= 0.4pt,line join=round,line cap=round,fill=fillColor] (129.67,523.88) circle (  1.16);

\path[draw=drawColor,line width= 0.4pt,line join=round,line cap=round,fill=fillColor] (129.91,523.86) circle (  1.16);

\path[draw=drawColor,line width= 0.4pt,line join=round,line cap=round,fill=fillColor] (130.16,523.85) circle (  1.16);

\path[draw=drawColor,line width= 0.4pt,line join=round,line cap=round,fill=fillColor] (130.41,523.79) circle (  1.16);

\path[draw=drawColor,line width= 0.4pt,line join=round,line cap=round,fill=fillColor] (130.65,523.78) circle (  1.16);

\path[draw=drawColor,line width= 0.4pt,line join=round,line cap=round,fill=fillColor] (130.89,523.76) circle (  1.16);

\path[draw=drawColor,line width= 0.4pt,line join=round,line cap=round,fill=fillColor] (131.13,523.74) circle (  1.16);

\path[draw=drawColor,line width= 0.4pt,line join=round,line cap=round,fill=fillColor] (131.36,523.66) circle (  1.16);

\path[draw=drawColor,line width= 0.4pt,line join=round,line cap=round,fill=fillColor] (131.60,523.53) circle (  1.16);

\path[draw=drawColor,line width= 0.4pt,line join=round,line cap=round,fill=fillColor] (131.83,523.52) circle (  1.16);

\path[draw=drawColor,line width= 0.4pt,line join=round,line cap=round,fill=fillColor] (132.06,523.52) circle (  1.16);

\path[draw=drawColor,line width= 0.4pt,line join=round,line cap=round,fill=fillColor] (132.30,523.37) circle (  1.16);

\path[draw=drawColor,line width= 0.4pt,line join=round,line cap=round,fill=fillColor] (132.52,523.35) circle (  1.16);

\path[draw=drawColor,line width= 0.4pt,line join=round,line cap=round,fill=fillColor] (132.75,523.33) circle (  1.16);

\path[draw=drawColor,line width= 0.4pt,line join=round,line cap=round,fill=fillColor] (132.98,523.31) circle (  1.16);

\path[draw=drawColor,line width= 0.4pt,line join=round,line cap=round,fill=fillColor] (133.20,523.30) circle (  1.16);

\path[draw=drawColor,line width= 0.4pt,line join=round,line cap=round,fill=fillColor] (133.42,523.30) circle (  1.16);

\path[draw=drawColor,line width= 0.4pt,line join=round,line cap=round,fill=fillColor] (133.64,523.29) circle (  1.16);

\path[draw=drawColor,line width= 0.4pt,line join=round,line cap=round,fill=fillColor] (133.86,523.28) circle (  1.16);

\path[draw=drawColor,line width= 0.4pt,line join=round,line cap=round,fill=fillColor] (134.08,523.25) circle (  1.16);

\path[draw=drawColor,line width= 0.4pt,line join=round,line cap=round,fill=fillColor] (134.30,523.23) circle (  1.16);

\path[draw=drawColor,line width= 0.4pt,line join=round,line cap=round,fill=fillColor] (134.51,523.20) circle (  1.16);

\path[draw=drawColor,line width= 0.4pt,line join=round,line cap=round,fill=fillColor] (134.73,523.18) circle (  1.16);

\path[draw=drawColor,line width= 0.4pt,line join=round,line cap=round,fill=fillColor] (134.94,523.10) circle (  1.16);

\path[draw=drawColor,line width= 0.4pt,line join=round,line cap=round,fill=fillColor] (135.15,523.08) circle (  1.16);

\path[draw=drawColor,line width= 0.4pt,line join=round,line cap=round,fill=fillColor] (135.36,523.04) circle (  1.16);

\path[draw=drawColor,line width= 0.4pt,line join=round,line cap=round,fill=fillColor] (135.57,522.99) circle (  1.16);

\path[draw=drawColor,line width= 0.4pt,line join=round,line cap=round,fill=fillColor] (135.78,522.97) circle (  1.16);

\path[draw=drawColor,line width= 0.4pt,line join=round,line cap=round,fill=fillColor] (135.98,522.95) circle (  1.16);

\path[draw=drawColor,line width= 0.4pt,line join=round,line cap=round,fill=fillColor] (136.19,522.94) circle (  1.16);

\path[draw=drawColor,line width= 0.4pt,line join=round,line cap=round,fill=fillColor] (136.39,522.90) circle (  1.16);

\path[draw=drawColor,line width= 0.4pt,line join=round,line cap=round,fill=fillColor] (136.60,522.88) circle (  1.16);

\path[draw=drawColor,line width= 0.4pt,line join=round,line cap=round,fill=fillColor] (136.80,522.86) circle (  1.16);

\path[draw=drawColor,line width= 0.4pt,line join=round,line cap=round,fill=fillColor] (137.00,522.84) circle (  1.16);

\path[draw=drawColor,line width= 0.4pt,line join=round,line cap=round,fill=fillColor] (137.20,522.83) circle (  1.16);

\path[draw=drawColor,line width= 0.4pt,line join=round,line cap=round,fill=fillColor] (137.40,522.82) circle (  1.16);

\path[draw=drawColor,line width= 0.4pt,line join=round,line cap=round,fill=fillColor] (137.59,522.78) circle (  1.16);

\path[draw=drawColor,line width= 0.4pt,line join=round,line cap=round,fill=fillColor] (137.79,522.78) circle (  1.16);

\path[draw=drawColor,line width= 0.4pt,line join=round,line cap=round,fill=fillColor] (137.98,522.76) circle (  1.16);

\path[draw=drawColor,line width= 0.4pt,line join=round,line cap=round,fill=fillColor] (138.18,522.72) circle (  1.16);

\path[draw=drawColor,line width= 0.4pt,line join=round,line cap=round,fill=fillColor] (138.37,522.65) circle (  1.16);

\path[draw=drawColor,line width= 0.4pt,line join=round,line cap=round,fill=fillColor] (138.56,522.62) circle (  1.16);

\path[draw=drawColor,line width= 0.4pt,line join=round,line cap=round,fill=fillColor] (138.75,522.62) circle (  1.16);

\path[draw=drawColor,line width= 0.4pt,line join=round,line cap=round,fill=fillColor] (138.94,522.60) circle (  1.16);

\path[draw=drawColor,line width= 0.4pt,line join=round,line cap=round,fill=fillColor] (139.13,522.60) circle (  1.16);

\path[draw=drawColor,line width= 0.4pt,line join=round,line cap=round,fill=fillColor] (139.32,522.56) circle (  1.16);

\path[draw=drawColor,line width= 0.4pt,line join=round,line cap=round,fill=fillColor] (139.50,522.55) circle (  1.16);

\path[draw=drawColor,line width= 0.4pt,line join=round,line cap=round,fill=fillColor] (139.69,522.50) circle (  1.16);

\path[draw=drawColor,line width= 0.4pt,line join=round,line cap=round,fill=fillColor] (139.87,522.49) circle (  1.16);

\path[draw=drawColor,line width= 0.4pt,line join=round,line cap=round,fill=fillColor] (140.06,522.45) circle (  1.16);

\path[draw=drawColor,line width= 0.4pt,line join=round,line cap=round,fill=fillColor] (140.24,522.43) circle (  1.16);

\path[draw=drawColor,line width= 0.4pt,line join=round,line cap=round,fill=fillColor] (140.42,522.37) circle (  1.16);

\path[draw=drawColor,line width= 0.4pt,line join=round,line cap=round,fill=fillColor] (140.60,522.36) circle (  1.16);

\path[draw=drawColor,line width= 0.4pt,line join=round,line cap=round,fill=fillColor] (140.78,522.36) circle (  1.16);

\path[draw=drawColor,line width= 0.4pt,line join=round,line cap=round,fill=fillColor] (140.96,522.32) circle (  1.16);

\path[draw=drawColor,line width= 0.4pt,line join=round,line cap=round,fill=fillColor] (141.14,522.29) circle (  1.16);

\path[draw=drawColor,line width= 0.4pt,line join=round,line cap=round,fill=fillColor] (141.32,522.27) circle (  1.16);

\path[draw=drawColor,line width= 0.4pt,line join=round,line cap=round,fill=fillColor] (141.50,522.24) circle (  1.16);

\path[draw=drawColor,line width= 0.4pt,line join=round,line cap=round,fill=fillColor] (141.67,522.23) circle (  1.16);

\path[draw=drawColor,line width= 0.4pt,line join=round,line cap=round,fill=fillColor] (141.85,522.20) circle (  1.16);

\path[draw=drawColor,line width= 0.4pt,line join=round,line cap=round,fill=fillColor] (142.02,522.19) circle (  1.16);

\path[draw=drawColor,line width= 0.4pt,line join=round,line cap=round,fill=fillColor] (142.20,522.17) circle (  1.16);

\path[draw=drawColor,line width= 0.4pt,line join=round,line cap=round,fill=fillColor] (142.37,522.16) circle (  1.16);

\path[draw=drawColor,line width= 0.4pt,line join=round,line cap=round,fill=fillColor] (142.54,522.14) circle (  1.16);

\path[draw=drawColor,line width= 0.4pt,line join=round,line cap=round,fill=fillColor] (142.71,522.11) circle (  1.16);

\path[draw=drawColor,line width= 0.4pt,line join=round,line cap=round,fill=fillColor] (142.88,521.99) circle (  1.16);

\path[draw=drawColor,line width= 0.4pt,line join=round,line cap=round,fill=fillColor] (143.05,521.93) circle (  1.16);

\path[draw=drawColor,line width= 0.4pt,line join=round,line cap=round,fill=fillColor] (143.22,521.87) circle (  1.16);

\path[draw=drawColor,line width= 0.4pt,line join=round,line cap=round,fill=fillColor] (143.39,521.86) circle (  1.16);

\path[draw=drawColor,line width= 0.4pt,line join=round,line cap=round,fill=fillColor] (143.56,521.82) circle (  1.16);

\path[draw=drawColor,line width= 0.4pt,line join=round,line cap=round,fill=fillColor] (143.72,521.80) circle (  1.16);

\path[draw=drawColor,line width= 0.4pt,line join=round,line cap=round,fill=fillColor] (143.89,521.79) circle (  1.16);

\path[draw=drawColor,line width= 0.4pt,line join=round,line cap=round,fill=fillColor] (144.05,521.75) circle (  1.16);

\path[draw=drawColor,line width= 0.4pt,line join=round,line cap=round,fill=fillColor] (144.22,521.75) circle (  1.16);

\path[draw=drawColor,line width= 0.4pt,line join=round,line cap=round,fill=fillColor] (144.38,521.74) circle (  1.16);

\path[draw=drawColor,line width= 0.4pt,line join=round,line cap=round,fill=fillColor] (144.55,521.73) circle (  1.16);

\path[draw=drawColor,line width= 0.4pt,line join=round,line cap=round,fill=fillColor] (144.71,521.72) circle (  1.16);

\path[draw=drawColor,line width= 0.4pt,line join=round,line cap=round,fill=fillColor] (144.87,521.71) circle (  1.16);

\path[draw=drawColor,line width= 0.4pt,line join=round,line cap=round,fill=fillColor] (145.03,521.70) circle (  1.16);

\path[draw=drawColor,line width= 0.4pt,line join=round,line cap=round,fill=fillColor] (145.19,521.69) circle (  1.16);

\path[draw=drawColor,line width= 0.4pt,line join=round,line cap=round,fill=fillColor] (145.35,521.68) circle (  1.16);

\path[draw=drawColor,line width= 0.4pt,line join=round,line cap=round,fill=fillColor] (145.51,521.60) circle (  1.16);

\path[draw=drawColor,line width= 0.4pt,line join=round,line cap=round,fill=fillColor] (145.67,521.59) circle (  1.16);

\path[draw=drawColor,line width= 0.4pt,line join=round,line cap=round,fill=fillColor] (145.83,521.56) circle (  1.16);

\path[draw=drawColor,line width= 0.4pt,line join=round,line cap=round,fill=fillColor] (145.98,521.56) circle (  1.16);

\path[draw=drawColor,line width= 0.4pt,line join=round,line cap=round,fill=fillColor] (146.14,521.55) circle (  1.16);

\path[draw=drawColor,line width= 0.4pt,line join=round,line cap=round,fill=fillColor] (146.30,521.52) circle (  1.16);

\path[draw=drawColor,line width= 0.4pt,line join=round,line cap=round,fill=fillColor] (146.45,521.50) circle (  1.16);

\path[draw=drawColor,line width= 0.4pt,line join=round,line cap=round,fill=fillColor] (146.61,521.50) circle (  1.16);

\path[draw=drawColor,line width= 0.4pt,line join=round,line cap=round,fill=fillColor] (146.76,521.45) circle (  1.16);

\path[draw=drawColor,line width= 0.4pt,line join=round,line cap=round,fill=fillColor] (146.91,521.41) circle (  1.16);

\path[draw=drawColor,line width= 0.4pt,line join=round,line cap=round,fill=fillColor] (147.07,521.41) circle (  1.16);

\path[draw=drawColor,line width= 0.4pt,line join=round,line cap=round,fill=fillColor] (147.22,521.41) circle (  1.16);

\path[draw=drawColor,line width= 0.4pt,line join=round,line cap=round,fill=fillColor] (147.37,521.40) circle (  1.16);

\path[draw=drawColor,line width= 0.4pt,line join=round,line cap=round,fill=fillColor] (147.52,521.31) circle (  1.16);

\path[draw=drawColor,line width= 0.4pt,line join=round,line cap=round,fill=fillColor] (147.67,521.31) circle (  1.16);

\path[draw=drawColor,line width= 0.4pt,line join=round,line cap=round,fill=fillColor] (147.82,521.31) circle (  1.16);

\path[draw=drawColor,line width= 0.4pt,line join=round,line cap=round,fill=fillColor] (147.97,521.29) circle (  1.16);

\path[draw=drawColor,line width= 0.4pt,line join=round,line cap=round,fill=fillColor] (148.12,521.28) circle (  1.16);

\path[draw=drawColor,line width= 0.4pt,line join=round,line cap=round,fill=fillColor] (148.27,521.28) circle (  1.16);

\path[draw=drawColor,line width= 0.4pt,line join=round,line cap=round,fill=fillColor] (148.42,521.25) circle (  1.16);

\path[draw=drawColor,line width= 0.4pt,line join=round,line cap=round,fill=fillColor] (148.57,521.23) circle (  1.16);

\path[draw=drawColor,line width= 0.4pt,line join=round,line cap=round,fill=fillColor] (148.71,521.22) circle (  1.16);

\path[draw=drawColor,line width= 0.4pt,line join=round,line cap=round,fill=fillColor] (148.86,521.21) circle (  1.16);

\path[draw=drawColor,line width= 0.4pt,line join=round,line cap=round,fill=fillColor] (149.01,521.18) circle (  1.16);

\path[draw=drawColor,line width= 0.4pt,line join=round,line cap=round,fill=fillColor] (149.15,521.10) circle (  1.16);

\path[draw=drawColor,line width= 0.4pt,line join=round,line cap=round,fill=fillColor] (149.30,521.08) circle (  1.16);

\path[draw=drawColor,line width= 0.4pt,line join=round,line cap=round,fill=fillColor] (149.44,521.00) circle (  1.16);

\path[draw=drawColor,line width= 0.4pt,line join=round,line cap=round,fill=fillColor] (149.58,520.99) circle (  1.16);

\path[draw=drawColor,line width= 0.4pt,line join=round,line cap=round,fill=fillColor] (149.73,520.97) circle (  1.16);

\path[draw=drawColor,line width= 0.4pt,line join=round,line cap=round,fill=fillColor] (149.87,520.97) circle (  1.16);

\path[draw=drawColor,line width= 0.4pt,line join=round,line cap=round,fill=fillColor] (150.01,520.96) circle (  1.16);

\path[draw=drawColor,line width= 0.4pt,line join=round,line cap=round,fill=fillColor] (150.15,520.96) circle (  1.16);

\path[draw=drawColor,line width= 0.4pt,line join=round,line cap=round,fill=fillColor] (150.30,520.95) circle (  1.16);

\path[draw=drawColor,line width= 0.4pt,line join=round,line cap=round,fill=fillColor] (150.44,520.91) circle (  1.16);

\path[draw=drawColor,line width= 0.4pt,line join=round,line cap=round,fill=fillColor] (150.58,520.87) circle (  1.16);

\path[draw=drawColor,line width= 0.4pt,line join=round,line cap=round,fill=fillColor] (150.72,520.86) circle (  1.16);

\path[draw=drawColor,line width= 0.4pt,line join=round,line cap=round,fill=fillColor] (150.86,520.85) circle (  1.16);

\path[draw=drawColor,line width= 0.4pt,line join=round,line cap=round,fill=fillColor] (151.00,520.85) circle (  1.16);

\path[draw=drawColor,line width= 0.4pt,line join=round,line cap=round,fill=fillColor] (151.13,520.80) circle (  1.16);

\path[draw=drawColor,line width= 0.4pt,line join=round,line cap=round,fill=fillColor] (151.27,520.80) circle (  1.16);

\path[draw=drawColor,line width= 0.4pt,line join=round,line cap=round,fill=fillColor] (151.41,520.79) circle (  1.16);

\path[draw=drawColor,line width= 0.4pt,line join=round,line cap=round,fill=fillColor] (151.55,520.79) circle (  1.16);

\path[draw=drawColor,line width= 0.4pt,line join=round,line cap=round,fill=fillColor] (151.68,520.78) circle (  1.16);

\path[draw=drawColor,line width= 0.4pt,line join=round,line cap=round,fill=fillColor] (151.82,520.69) circle (  1.16);

\path[draw=drawColor,line width= 0.4pt,line join=round,line cap=round,fill=fillColor] (151.96,520.64) circle (  1.16);

\path[draw=drawColor,line width= 0.4pt,line join=round,line cap=round,fill=fillColor] (152.09,520.63) circle (  1.16);

\path[draw=drawColor,line width= 0.4pt,line join=round,line cap=round,fill=fillColor] (152.23,520.60) circle (  1.16);

\path[draw=drawColor,line width= 0.4pt,line join=round,line cap=round,fill=fillColor] (152.36,520.59) circle (  1.16);

\path[draw=drawColor,line width= 0.4pt,line join=round,line cap=round,fill=fillColor] (152.49,520.57) circle (  1.16);

\path[draw=drawColor,line width= 0.4pt,line join=round,line cap=round,fill=fillColor] (152.63,520.56) circle (  1.16);

\path[draw=drawColor,line width= 0.4pt,line join=round,line cap=round,fill=fillColor] (152.76,520.56) circle (  1.16);

\path[draw=drawColor,line width= 0.4pt,line join=round,line cap=round,fill=fillColor] (152.89,520.55) circle (  1.16);

\path[draw=drawColor,line width= 0.4pt,line join=round,line cap=round,fill=fillColor] (153.03,520.54) circle (  1.16);

\path[draw=drawColor,line width= 0.4pt,line join=round,line cap=round,fill=fillColor] (153.16,520.50) circle (  1.16);

\path[draw=drawColor,line width= 0.4pt,line join=round,line cap=round,fill=fillColor] (153.29,520.48) circle (  1.16);

\path[draw=drawColor,line width= 0.4pt,line join=round,line cap=round,fill=fillColor] (153.42,520.46) circle (  1.16);

\path[draw=drawColor,line width= 0.4pt,line join=round,line cap=round,fill=fillColor] (153.55,520.45) circle (  1.16);

\path[draw=drawColor,line width= 0.4pt,line join=round,line cap=round,fill=fillColor] (153.68,520.45) circle (  1.16);

\path[draw=drawColor,line width= 0.4pt,line join=round,line cap=round,fill=fillColor] (153.81,520.41) circle (  1.16);

\path[draw=drawColor,line width= 0.4pt,line join=round,line cap=round,fill=fillColor] (153.94,520.37) circle (  1.16);

\path[draw=drawColor,line width= 0.4pt,line join=round,line cap=round,fill=fillColor] (154.07,520.37) circle (  1.16);

\path[draw=drawColor,line width= 0.4pt,line join=round,line cap=round,fill=fillColor] (154.20,520.36) circle (  1.16);

\path[draw=drawColor,line width= 0.4pt,line join=round,line cap=round,fill=fillColor] (154.33,520.35) circle (  1.16);

\path[draw=drawColor,line width= 0.4pt,line join=round,line cap=round,fill=fillColor] (154.46,520.34) circle (  1.16);

\path[draw=drawColor,line width= 0.4pt,line join=round,line cap=round,fill=fillColor] (154.59,520.29) circle (  1.16);

\path[draw=drawColor,line width= 0.4pt,line join=round,line cap=round,fill=fillColor] (154.71,520.28) circle (  1.16);

\path[draw=drawColor,line width= 0.4pt,line join=round,line cap=round,fill=fillColor] (154.84,520.25) circle (  1.16);

\path[draw=drawColor,line width= 0.4pt,line join=round,line cap=round,fill=fillColor] (154.97,520.22) circle (  1.16);

\path[draw=drawColor,line width= 0.4pt,line join=round,line cap=round,fill=fillColor] (155.09,520.16) circle (  1.16);

\path[draw=drawColor,line width= 0.4pt,line join=round,line cap=round,fill=fillColor] (155.22,520.16) circle (  1.16);

\path[draw=drawColor,line width= 0.4pt,line join=round,line cap=round,fill=fillColor] (155.35,520.14) circle (  1.16);

\path[draw=drawColor,line width= 0.4pt,line join=round,line cap=round,fill=fillColor] (155.47,520.10) circle (  1.16);

\path[draw=drawColor,line width= 0.4pt,line join=round,line cap=round,fill=fillColor] (155.60,520.07) circle (  1.16);

\path[draw=drawColor,line width= 0.4pt,line join=round,line cap=round,fill=fillColor] (155.72,520.07) circle (  1.16);

\path[draw=drawColor,line width= 0.4pt,line join=round,line cap=round,fill=fillColor] (155.85,520.06) circle (  1.16);

\path[draw=drawColor,line width= 0.4pt,line join=round,line cap=round,fill=fillColor] (155.97,520.04) circle (  1.16);

\path[draw=drawColor,line width= 0.4pt,line join=round,line cap=round,fill=fillColor] (156.09,520.04) circle (  1.16);

\path[draw=drawColor,line width= 0.4pt,line join=round,line cap=round,fill=fillColor] (156.22,520.02) circle (  1.16);

\path[draw=drawColor,line width= 0.4pt,line join=round,line cap=round,fill=fillColor] (156.34,520.01) circle (  1.16);

\path[draw=drawColor,line width= 0.4pt,line join=round,line cap=round,fill=fillColor] (156.46,519.97) circle (  1.16);

\path[draw=drawColor,line width= 0.4pt,line join=round,line cap=round,fill=fillColor] (156.58,519.95) circle (  1.16);

\path[draw=drawColor,line width= 0.4pt,line join=round,line cap=round,fill=fillColor] (156.71,519.94) circle (  1.16);

\path[draw=drawColor,line width= 0.4pt,line join=round,line cap=round,fill=fillColor] (156.83,519.93) circle (  1.16);

\path[draw=drawColor,line width= 0.4pt,line join=round,line cap=round,fill=fillColor] (156.95,519.93) circle (  1.16);

\path[draw=drawColor,line width= 0.4pt,line join=round,line cap=round,fill=fillColor] (157.07,519.93) circle (  1.16);

\path[draw=drawColor,line width= 0.4pt,line join=round,line cap=round,fill=fillColor] (157.19,519.92) circle (  1.16);

\path[draw=drawColor,line width= 0.4pt,line join=round,line cap=round,fill=fillColor] (157.31,519.91) circle (  1.16);

\path[draw=drawColor,line width= 0.4pt,line join=round,line cap=round,fill=fillColor] (157.43,519.82) circle (  1.16);

\path[draw=drawColor,line width= 0.4pt,line join=round,line cap=round,fill=fillColor] (157.55,519.82) circle (  1.16);

\path[draw=drawColor,line width= 0.4pt,line join=round,line cap=round,fill=fillColor] (157.67,519.76) circle (  1.16);

\path[draw=drawColor,line width= 0.4pt,line join=round,line cap=round,fill=fillColor] (157.79,519.76) circle (  1.16);

\path[draw=drawColor,line width= 0.4pt,line join=round,line cap=round,fill=fillColor] (157.91,519.74) circle (  1.16);

\path[draw=drawColor,line width= 0.4pt,line join=round,line cap=round,fill=fillColor] (158.02,519.72) circle (  1.16);

\path[draw=drawColor,line width= 0.4pt,line join=round,line cap=round,fill=fillColor] (158.14,519.69) circle (  1.16);

\path[draw=drawColor,line width= 0.4pt,line join=round,line cap=round,fill=fillColor] (158.26,519.68) circle (  1.16);

\path[draw=drawColor,line width= 0.4pt,line join=round,line cap=round,fill=fillColor] (158.38,519.68) circle (  1.16);

\path[draw=drawColor,line width= 0.4pt,line join=round,line cap=round,fill=fillColor] (158.50,519.67) circle (  1.16);

\path[draw=drawColor,line width= 0.4pt,line join=round,line cap=round,fill=fillColor] (158.61,519.66) circle (  1.16);

\path[draw=drawColor,line width= 0.4pt,line join=round,line cap=round,fill=fillColor] (158.73,519.65) circle (  1.16);

\path[draw=drawColor,line width= 0.4pt,line join=round,line cap=round,fill=fillColor] (158.84,519.64) circle (  1.16);

\path[draw=drawColor,line width= 0.4pt,line join=round,line cap=round,fill=fillColor] (158.96,519.64) circle (  1.16);

\path[draw=drawColor,line width= 0.4pt,line join=round,line cap=round,fill=fillColor] (159.08,519.63) circle (  1.16);

\path[draw=drawColor,line width= 0.4pt,line join=round,line cap=round,fill=fillColor] (159.19,519.60) circle (  1.16);

\path[draw=drawColor,line width= 0.4pt,line join=round,line cap=round,fill=fillColor] (159.31,519.56) circle (  1.16);

\path[draw=drawColor,line width= 0.4pt,line join=round,line cap=round,fill=fillColor] (159.42,519.56) circle (  1.16);

\path[draw=drawColor,line width= 0.4pt,line join=round,line cap=round,fill=fillColor] (159.54,519.53) circle (  1.16);

\path[draw=drawColor,line width= 0.4pt,line join=round,line cap=round,fill=fillColor] (159.65,519.52) circle (  1.16);

\path[draw=drawColor,line width= 0.4pt,line join=round,line cap=round,fill=fillColor] (159.76,519.45) circle (  1.16);

\path[draw=drawColor,line width= 0.4pt,line join=round,line cap=round,fill=fillColor] (159.88,519.43) circle (  1.16);

\path[draw=drawColor,line width= 0.4pt,line join=round,line cap=round,fill=fillColor] (159.99,519.42) circle (  1.16);

\path[draw=drawColor,line width= 0.4pt,line join=round,line cap=round,fill=fillColor] (160.10,519.42) circle (  1.16);

\path[draw=drawColor,line width= 0.4pt,line join=round,line cap=round,fill=fillColor] (160.22,519.37) circle (  1.16);

\path[draw=drawColor,line width= 0.4pt,line join=round,line cap=round,fill=fillColor] (160.33,519.35) circle (  1.16);

\path[draw=drawColor,line width= 0.4pt,line join=round,line cap=round,fill=fillColor] (160.44,519.33) circle (  1.16);

\path[draw=drawColor,line width= 0.4pt,line join=round,line cap=round,fill=fillColor] (160.55,519.32) circle (  1.16);

\path[draw=drawColor,line width= 0.4pt,line join=round,line cap=round,fill=fillColor] (160.67,519.31) circle (  1.16);

\path[draw=drawColor,line width= 0.4pt,line join=round,line cap=round,fill=fillColor] (160.78,519.25) circle (  1.16);

\path[draw=drawColor,line width= 0.4pt,line join=round,line cap=round,fill=fillColor] (160.89,519.23) circle (  1.16);

\path[draw=drawColor,line width= 0.4pt,line join=round,line cap=round,fill=fillColor] (161.00,519.21) circle (  1.16);

\path[draw=drawColor,line width= 0.4pt,line join=round,line cap=round,fill=fillColor] (161.11,519.20) circle (  1.16);

\path[draw=drawColor,line width= 0.4pt,line join=round,line cap=round,fill=fillColor] (161.22,519.17) circle (  1.16);

\path[draw=drawColor,line width= 0.4pt,line join=round,line cap=round,fill=fillColor] (161.33,519.15) circle (  1.16);

\path[draw=drawColor,line width= 0.4pt,line join=round,line cap=round,fill=fillColor] (161.44,519.15) circle (  1.16);

\path[draw=drawColor,line width= 0.4pt,line join=round,line cap=round,fill=fillColor] (161.55,519.15) circle (  1.16);

\path[draw=drawColor,line width= 0.4pt,line join=round,line cap=round,fill=fillColor] (161.66,519.15) circle (  1.16);

\path[draw=drawColor,line width= 0.4pt,line join=round,line cap=round,fill=fillColor] (161.77,519.14) circle (  1.16);

\path[draw=drawColor,line width= 0.4pt,line join=round,line cap=round,fill=fillColor] (161.88,519.11) circle (  1.16);

\path[draw=drawColor,line width= 0.4pt,line join=round,line cap=round,fill=fillColor] (161.99,519.10) circle (  1.16);

\path[draw=drawColor,line width= 0.4pt,line join=round,line cap=round,fill=fillColor] (162.10,519.10) circle (  1.16);

\path[draw=drawColor,line width= 0.4pt,line join=round,line cap=round,fill=fillColor] (162.20,519.10) circle (  1.16);

\path[draw=drawColor,line width= 0.4pt,line join=round,line cap=round,fill=fillColor] (162.31,519.09) circle (  1.16);

\path[draw=drawColor,line width= 0.4pt,line join=round,line cap=round,fill=fillColor] (162.42,519.08) circle (  1.16);

\path[draw=drawColor,line width= 0.4pt,line join=round,line cap=round,fill=fillColor] (162.53,519.07) circle (  1.16);

\path[draw=drawColor,line width= 0.4pt,line join=round,line cap=round,fill=fillColor] (162.63,519.05) circle (  1.16);

\path[draw=drawColor,line width= 0.4pt,line join=round,line cap=round,fill=fillColor] (162.74,519.03) circle (  1.16);

\path[draw=drawColor,line width= 0.4pt,line join=round,line cap=round,fill=fillColor] (162.85,519.03) circle (  1.16);

\path[draw=drawColor,line width= 0.4pt,line join=round,line cap=round,fill=fillColor] (162.95,519.03) circle (  1.16);

\path[draw=drawColor,line width= 0.4pt,line join=round,line cap=round,fill=fillColor] (163.06,519.01) circle (  1.16);

\path[draw=drawColor,line width= 0.4pt,line join=round,line cap=round,fill=fillColor] (163.17,518.99) circle (  1.16);

\path[draw=drawColor,line width= 0.4pt,line join=round,line cap=round,fill=fillColor] (163.27,518.95) circle (  1.16);

\path[draw=drawColor,line width= 0.4pt,line join=round,line cap=round,fill=fillColor] (163.38,518.94) circle (  1.16);

\path[draw=drawColor,line width= 0.4pt,line join=round,line cap=round,fill=fillColor] (163.48,518.93) circle (  1.16);

\path[draw=drawColor,line width= 0.4pt,line join=round,line cap=round,fill=fillColor] (163.59,518.87) circle (  1.16);

\path[draw=drawColor,line width= 0.4pt,line join=round,line cap=round,fill=fillColor] (163.69,518.87) circle (  1.16);

\path[draw=drawColor,line width= 0.4pt,line join=round,line cap=round,fill=fillColor] (163.80,518.86) circle (  1.16);

\path[draw=drawColor,line width= 0.4pt,line join=round,line cap=round,fill=fillColor] (163.90,518.86) circle (  1.16);

\path[draw=drawColor,line width= 0.4pt,line join=round,line cap=round,fill=fillColor] (164.01,518.86) circle (  1.16);

\path[draw=drawColor,line width= 0.4pt,line join=round,line cap=round,fill=fillColor] (164.11,518.85) circle (  1.16);

\path[draw=drawColor,line width= 0.4pt,line join=round,line cap=round,fill=fillColor] (164.21,518.85) circle (  1.16);

\path[draw=drawColor,line width= 0.4pt,line join=round,line cap=round,fill=fillColor] (164.32,518.84) circle (  1.16);

\path[draw=drawColor,line width= 0.4pt,line join=round,line cap=round,fill=fillColor] (164.42,518.83) circle (  1.16);

\path[draw=drawColor,line width= 0.4pt,line join=round,line cap=round,fill=fillColor] (164.52,518.81) circle (  1.16);

\path[draw=drawColor,line width= 0.4pt,line join=round,line cap=round,fill=fillColor] (164.63,518.80) circle (  1.16);

\path[draw=drawColor,line width= 0.4pt,line join=round,line cap=round,fill=fillColor] (164.73,518.80) circle (  1.16);

\path[draw=drawColor,line width= 0.4pt,line join=round,line cap=round,fill=fillColor] (164.83,518.79) circle (  1.16);

\path[draw=drawColor,line width= 0.4pt,line join=round,line cap=round,fill=fillColor] (164.93,518.77) circle (  1.16);

\path[draw=drawColor,line width= 0.4pt,line join=round,line cap=round,fill=fillColor] (165.04,518.74) circle (  1.16);

\path[draw=drawColor,line width= 0.4pt,line join=round,line cap=round,fill=fillColor] (165.14,518.72) circle (  1.16);

\path[draw=drawColor,line width= 0.4pt,line join=round,line cap=round,fill=fillColor] (165.24,518.70) circle (  1.16);

\path[draw=drawColor,line width= 0.4pt,line join=round,line cap=round,fill=fillColor] (165.34,518.67) circle (  1.16);

\path[draw=drawColor,line width= 0.4pt,line join=round,line cap=round,fill=fillColor] (165.44,518.67) circle (  1.16);

\path[draw=drawColor,line width= 0.4pt,line join=round,line cap=round,fill=fillColor] (165.54,518.64) circle (  1.16);

\path[draw=drawColor,line width= 0.4pt,line join=round,line cap=round,fill=fillColor] (165.64,518.62) circle (  1.16);

\path[draw=drawColor,line width= 0.4pt,line join=round,line cap=round,fill=fillColor] (165.74,518.61) circle (  1.16);

\path[draw=drawColor,line width= 0.4pt,line join=round,line cap=round,fill=fillColor] (165.85,518.60) circle (  1.16);

\path[draw=drawColor,line width= 0.4pt,line join=round,line cap=round,fill=fillColor] (165.95,518.58) circle (  1.16);

\path[draw=drawColor,line width= 0.4pt,line join=round,line cap=round,fill=fillColor] (166.05,518.57) circle (  1.16);

\path[draw=drawColor,line width= 0.4pt,line join=round,line cap=round,fill=fillColor] (166.14,518.55) circle (  1.16);

\path[draw=drawColor,line width= 0.4pt,line join=round,line cap=round,fill=fillColor] (166.24,518.54) circle (  1.16);

\path[draw=drawColor,line width= 0.4pt,line join=round,line cap=round,fill=fillColor] (166.34,518.51) circle (  1.16);

\path[draw=drawColor,line width= 0.4pt,line join=round,line cap=round,fill=fillColor] (166.44,518.50) circle (  1.16);

\path[draw=drawColor,line width= 0.4pt,line join=round,line cap=round,fill=fillColor] (166.54,518.49) circle (  1.16);

\path[draw=drawColor,line width= 0.4pt,line join=round,line cap=round,fill=fillColor] (166.64,518.47) circle (  1.16);

\path[draw=drawColor,line width= 0.4pt,line join=round,line cap=round,fill=fillColor] (166.74,518.46) circle (  1.16);

\path[draw=drawColor,line width= 0.4pt,line join=round,line cap=round,fill=fillColor] (166.84,518.42) circle (  1.16);

\path[draw=drawColor,line width= 0.4pt,line join=round,line cap=round,fill=fillColor] (166.94,518.41) circle (  1.16);

\path[draw=drawColor,line width= 0.4pt,line join=round,line cap=round,fill=fillColor] (167.03,518.41) circle (  1.16);

\path[draw=drawColor,line width= 0.4pt,line join=round,line cap=round,fill=fillColor] (167.13,518.40) circle (  1.16);

\path[draw=drawColor,line width= 0.4pt,line join=round,line cap=round,fill=fillColor] (167.23,518.40) circle (  1.16);

\path[draw=drawColor,line width= 0.4pt,line join=round,line cap=round,fill=fillColor] (167.33,518.37) circle (  1.16);

\path[draw=drawColor,line width= 0.4pt,line join=round,line cap=round,fill=fillColor] (167.42,518.31) circle (  1.16);

\path[draw=drawColor,line width= 0.4pt,line join=round,line cap=round,fill=fillColor] (167.52,518.29) circle (  1.16);

\path[draw=drawColor,line width= 0.4pt,line join=round,line cap=round,fill=fillColor] (167.62,518.29) circle (  1.16);

\path[draw=drawColor,line width= 0.4pt,line join=round,line cap=round,fill=fillColor] (167.71,518.28) circle (  1.16);

\path[draw=drawColor,line width= 0.4pt,line join=round,line cap=round,fill=fillColor] (167.81,518.27) circle (  1.16);

\path[draw=drawColor,line width= 0.4pt,line join=round,line cap=round,fill=fillColor] (167.91,518.24) circle (  1.16);

\path[draw=drawColor,line width= 0.4pt,line join=round,line cap=round,fill=fillColor] (168.00,518.24) circle (  1.16);

\path[draw=drawColor,line width= 0.4pt,line join=round,line cap=round,fill=fillColor] (168.10,518.22) circle (  1.16);

\path[draw=drawColor,line width= 0.4pt,line join=round,line cap=round,fill=fillColor] (168.20,518.21) circle (  1.16);

\path[draw=drawColor,line width= 0.4pt,line join=round,line cap=round,fill=fillColor] (168.29,518.20) circle (  1.16);

\path[draw=drawColor,line width= 0.4pt,line join=round,line cap=round,fill=fillColor] (168.39,518.19) circle (  1.16);

\path[draw=drawColor,line width= 0.4pt,line join=round,line cap=round,fill=fillColor] (168.48,518.18) circle (  1.16);

\path[draw=drawColor,line width= 0.4pt,line join=round,line cap=round,fill=fillColor] (168.58,518.14) circle (  1.16);

\path[draw=drawColor,line width= 0.4pt,line join=round,line cap=round,fill=fillColor] (168.67,518.14) circle (  1.16);

\path[draw=drawColor,line width= 0.4pt,line join=round,line cap=round,fill=fillColor] (168.77,518.14) circle (  1.16);

\path[draw=drawColor,line width= 0.4pt,line join=round,line cap=round,fill=fillColor] (168.86,518.13) circle (  1.16);

\path[draw=drawColor,line width= 0.4pt,line join=round,line cap=round,fill=fillColor] (168.95,518.09) circle (  1.16);

\path[draw=drawColor,line width= 0.4pt,line join=round,line cap=round,fill=fillColor] (169.05,518.07) circle (  1.16);

\path[draw=drawColor,line width= 0.4pt,line join=round,line cap=round,fill=fillColor] (169.14,518.07) circle (  1.16);

\path[draw=drawColor,line width= 0.4pt,line join=round,line cap=round,fill=fillColor] (169.24,518.02) circle (  1.16);

\path[draw=drawColor,line width= 0.4pt,line join=round,line cap=round,fill=fillColor] (169.33,518.02) circle (  1.16);

\path[draw=drawColor,line width= 0.4pt,line join=round,line cap=round,fill=fillColor] (169.42,518.01) circle (  1.16);

\path[draw=drawColor,line width= 0.4pt,line join=round,line cap=round,fill=fillColor] (169.52,517.98) circle (  1.16);

\path[draw=drawColor,line width= 0.4pt,line join=round,line cap=round,fill=fillColor] (169.61,517.97) circle (  1.16);

\path[draw=drawColor,line width= 0.4pt,line join=round,line cap=round,fill=fillColor] (169.70,517.96) circle (  1.16);

\path[draw=drawColor,line width= 0.4pt,line join=round,line cap=round,fill=fillColor] (169.80,517.95) circle (  1.16);

\path[draw=drawColor,line width= 0.4pt,line join=round,line cap=round,fill=fillColor] (169.89,517.94) circle (  1.16);

\path[draw=drawColor,line width= 0.4pt,line join=round,line cap=round,fill=fillColor] (169.98,517.91) circle (  1.16);

\path[draw=drawColor,line width= 0.4pt,line join=round,line cap=round,fill=fillColor] (170.07,517.90) circle (  1.16);

\path[draw=drawColor,line width= 0.4pt,line join=round,line cap=round,fill=fillColor] (170.17,517.90) circle (  1.16);

\path[draw=drawColor,line width= 0.4pt,line join=round,line cap=round,fill=fillColor] (170.26,517.90) circle (  1.16);

\path[draw=drawColor,line width= 0.4pt,line join=round,line cap=round,fill=fillColor] (170.35,517.88) circle (  1.16);

\path[draw=drawColor,line width= 0.4pt,line join=round,line cap=round,fill=fillColor] (170.44,517.88) circle (  1.16);

\path[draw=drawColor,line width= 0.4pt,line join=round,line cap=round,fill=fillColor] (170.53,517.86) circle (  1.16);

\path[draw=drawColor,line width= 0.4pt,line join=round,line cap=round,fill=fillColor] (170.62,517.86) circle (  1.16);

\path[draw=drawColor,line width= 0.4pt,line join=round,line cap=round,fill=fillColor] (170.71,517.86) circle (  1.16);

\path[draw=drawColor,line width= 0.4pt,line join=round,line cap=round,fill=fillColor] (170.81,517.85) circle (  1.16);

\path[draw=drawColor,line width= 0.4pt,line join=round,line cap=round,fill=fillColor] (170.90,517.83) circle (  1.16);

\path[draw=drawColor,line width= 0.4pt,line join=round,line cap=round,fill=fillColor] (170.99,517.82) circle (  1.16);

\path[draw=drawColor,line width= 0.4pt,line join=round,line cap=round,fill=fillColor] (171.08,517.79) circle (  1.16);

\path[draw=drawColor,line width= 0.4pt,line join=round,line cap=round,fill=fillColor] (171.17,517.78) circle (  1.16);

\path[draw=drawColor,line width= 0.4pt,line join=round,line cap=round,fill=fillColor] (171.26,517.75) circle (  1.16);

\path[draw=drawColor,line width= 0.4pt,line join=round,line cap=round,fill=fillColor] (171.35,517.73) circle (  1.16);

\path[draw=drawColor,line width= 0.4pt,line join=round,line cap=round,fill=fillColor] (171.44,517.73) circle (  1.16);

\path[draw=drawColor,line width= 0.4pt,line join=round,line cap=round,fill=fillColor] (171.53,517.72) circle (  1.16);

\path[draw=drawColor,line width= 0.4pt,line join=round,line cap=round,fill=fillColor] (171.62,517.69) circle (  1.16);

\path[draw=drawColor,line width= 0.4pt,line join=round,line cap=round,fill=fillColor] (171.71,517.65) circle (  1.16);

\path[draw=drawColor,line width= 0.4pt,line join=round,line cap=round,fill=fillColor] (171.80,517.62) circle (  1.16);

\path[draw=drawColor,line width= 0.4pt,line join=round,line cap=round,fill=fillColor] (171.89,517.42) circle (  1.16);

\path[draw=drawColor,line width= 0.4pt,line join=round,line cap=round,fill=fillColor] (171.97,517.42) circle (  1.16);

\path[draw=drawColor,line width= 0.4pt,line join=round,line cap=round,fill=fillColor] (172.06,517.39) circle (  1.16);

\path[draw=drawColor,line width= 0.4pt,line join=round,line cap=round,fill=fillColor] (172.15,517.38) circle (  1.16);

\path[draw=drawColor,line width= 0.4pt,line join=round,line cap=round,fill=fillColor] (172.24,517.35) circle (  1.16);

\path[draw=drawColor,line width= 0.4pt,line join=round,line cap=round,fill=fillColor] (172.33,517.34) circle (  1.16);

\path[draw=drawColor,line width= 0.4pt,line join=round,line cap=round,fill=fillColor] (172.42,517.30) circle (  1.16);

\path[draw=drawColor,line width= 0.4pt,line join=round,line cap=round,fill=fillColor] (172.51,517.30) circle (  1.16);

\path[draw=drawColor,line width= 0.4pt,line join=round,line cap=round,fill=fillColor] (172.59,517.29) circle (  1.16);

\path[draw=drawColor,line width= 0.4pt,line join=round,line cap=round,fill=fillColor] (172.68,517.20) circle (  1.16);

\path[draw=drawColor,line width= 0.4pt,line join=round,line cap=round,fill=fillColor] (172.77,517.18) circle (  1.16);

\path[draw=drawColor,line width= 0.4pt,line join=round,line cap=round,fill=fillColor] (172.86,517.14) circle (  1.16);

\path[draw=drawColor,line width= 0.4pt,line join=round,line cap=round,fill=fillColor] (172.94,517.13) circle (  1.16);

\path[draw=drawColor,line width= 0.4pt,line join=round,line cap=round,fill=fillColor] (173.03,517.13) circle (  1.16);

\path[draw=drawColor,line width= 0.4pt,line join=round,line cap=round,fill=fillColor] (173.12,517.11) circle (  1.16);

\path[draw=drawColor,line width= 0.4pt,line join=round,line cap=round,fill=fillColor] (173.20,517.10) circle (  1.16);

\path[draw=drawColor,line width= 0.4pt,line join=round,line cap=round,fill=fillColor] (173.29,516.98) circle (  1.16);

\path[draw=drawColor,line width= 0.4pt,line join=round,line cap=round,fill=fillColor] (173.38,516.97) circle (  1.16);

\path[draw=drawColor,line width= 0.4pt,line join=round,line cap=round,fill=fillColor] (173.46,516.95) circle (  1.16);

\path[draw=drawColor,line width= 0.4pt,line join=round,line cap=round,fill=fillColor] (173.55,516.95) circle (  1.16);

\path[draw=drawColor,line width= 0.4pt,line join=round,line cap=round,fill=fillColor] (173.64,516.94) circle (  1.16);

\path[draw=drawColor,line width= 0.4pt,line join=round,line cap=round,fill=fillColor] (173.72,516.89) circle (  1.16);

\path[draw=drawColor,line width= 0.4pt,line join=round,line cap=round,fill=fillColor] (173.81,516.88) circle (  1.16);

\path[draw=drawColor,line width= 0.4pt,line join=round,line cap=round,fill=fillColor] (173.89,516.84) circle (  1.16);

\path[draw=drawColor,line width= 0.4pt,line join=round,line cap=round,fill=fillColor] (173.98,516.82) circle (  1.16);

\path[draw=drawColor,line width= 0.4pt,line join=round,line cap=round,fill=fillColor] (174.07,516.82) circle (  1.16);

\path[draw=drawColor,line width= 0.4pt,line join=round,line cap=round,fill=fillColor] (174.15,516.82) circle (  1.16);

\path[draw=drawColor,line width= 0.4pt,line join=round,line cap=round,fill=fillColor] (174.24,516.82) circle (  1.16);

\path[draw=drawColor,line width= 0.4pt,line join=round,line cap=round,fill=fillColor] (174.32,516.80) circle (  1.16);

\path[draw=drawColor,line width= 0.4pt,line join=round,line cap=round,fill=fillColor] (174.41,516.80) circle (  1.16);

\path[draw=drawColor,line width= 0.4pt,line join=round,line cap=round,fill=fillColor] (174.49,516.79) circle (  1.16);

\path[draw=drawColor,line width= 0.4pt,line join=round,line cap=round,fill=fillColor] (174.58,516.74) circle (  1.16);

\path[draw=drawColor,line width= 0.4pt,line join=round,line cap=round,fill=fillColor] (174.66,516.74) circle (  1.16);

\path[draw=drawColor,line width= 0.4pt,line join=round,line cap=round,fill=fillColor] (174.75,516.72) circle (  1.16);

\path[draw=drawColor,line width= 0.4pt,line join=round,line cap=round,fill=fillColor] (174.83,516.71) circle (  1.16);

\path[draw=drawColor,line width= 0.4pt,line join=round,line cap=round,fill=fillColor] (174.91,516.71) circle (  1.16);

\path[draw=drawColor,line width= 0.4pt,line join=round,line cap=round,fill=fillColor] (175.00,516.67) circle (  1.16);

\path[draw=drawColor,line width= 0.4pt,line join=round,line cap=round,fill=fillColor] (175.08,516.67) circle (  1.16);

\path[draw=drawColor,line width= 0.4pt,line join=round,line cap=round,fill=fillColor] (175.17,516.66) circle (  1.16);

\path[draw=drawColor,line width= 0.4pt,line join=round,line cap=round,fill=fillColor] (175.25,516.66) circle (  1.16);

\path[draw=drawColor,line width= 0.4pt,line join=round,line cap=round,fill=fillColor] (175.33,516.65) circle (  1.16);

\path[draw=drawColor,line width= 0.4pt,line join=round,line cap=round,fill=fillColor] (175.42,516.62) circle (  1.16);

\path[draw=drawColor,line width= 0.4pt,line join=round,line cap=round,fill=fillColor] (175.50,516.61) circle (  1.16);

\path[draw=drawColor,line width= 0.4pt,line join=round,line cap=round,fill=fillColor] (175.58,516.61) circle (  1.16);

\path[draw=drawColor,line width= 0.4pt,line join=round,line cap=round,fill=fillColor] (175.67,516.60) circle (  1.16);

\path[draw=drawColor,line width= 0.4pt,line join=round,line cap=round,fill=fillColor] (175.75,516.57) circle (  1.16);

\path[draw=drawColor,line width= 0.4pt,line join=round,line cap=round,fill=fillColor] (175.83,516.53) circle (  1.16);

\path[draw=drawColor,line width= 0.4pt,line join=round,line cap=round,fill=fillColor] (175.91,516.53) circle (  1.16);

\path[draw=drawColor,line width= 0.4pt,line join=round,line cap=round,fill=fillColor] (176.00,516.53) circle (  1.16);

\path[draw=drawColor,line width= 0.4pt,line join=round,line cap=round,fill=fillColor] (176.08,516.50) circle (  1.16);

\path[draw=drawColor,line width= 0.4pt,line join=round,line cap=round,fill=fillColor] (176.16,516.49) circle (  1.16);

\path[draw=drawColor,line width= 0.4pt,line join=round,line cap=round,fill=fillColor] (176.24,516.45) circle (  1.16);

\path[draw=drawColor,line width= 0.4pt,line join=round,line cap=round,fill=fillColor] (176.33,516.41) circle (  1.16);

\path[draw=drawColor,line width= 0.4pt,line join=round,line cap=round,fill=fillColor] (176.41,516.41) circle (  1.16);

\path[draw=drawColor,line width= 0.4pt,line join=round,line cap=round,fill=fillColor] (176.49,516.41) circle (  1.16);

\path[draw=drawColor,line width= 0.4pt,line join=round,line cap=round,fill=fillColor] (176.57,516.41) circle (  1.16);

\path[draw=drawColor,line width= 0.4pt,line join=round,line cap=round,fill=fillColor] (176.65,516.39) circle (  1.16);

\path[draw=drawColor,line width= 0.4pt,line join=round,line cap=round,fill=fillColor] (176.73,516.34) circle (  1.16);

\path[draw=drawColor,line width= 0.4pt,line join=round,line cap=round,fill=fillColor] (176.82,516.33) circle (  1.16);

\path[draw=drawColor,line width= 0.4pt,line join=round,line cap=round,fill=fillColor] (176.90,516.32) circle (  1.16);

\path[draw=drawColor,line width= 0.4pt,line join=round,line cap=round,fill=fillColor] (176.98,516.31) circle (  1.16);

\path[draw=drawColor,line width= 0.4pt,line join=round,line cap=round,fill=fillColor] (177.06,516.22) circle (  1.16);

\path[draw=drawColor,line width= 0.4pt,line join=round,line cap=round,fill=fillColor] (177.14,516.18) circle (  1.16);

\path[draw=drawColor,line width= 0.4pt,line join=round,line cap=round,fill=fillColor] (177.22,516.18) circle (  1.16);

\path[draw=drawColor,line width= 0.4pt,line join=round,line cap=round,fill=fillColor] (177.30,516.17) circle (  1.16);

\path[draw=drawColor,line width= 0.4pt,line join=round,line cap=round,fill=fillColor] (177.38,516.12) circle (  1.16);

\path[draw=drawColor,line width= 0.4pt,line join=round,line cap=round,fill=fillColor] (177.46,516.11) circle (  1.16);

\path[draw=drawColor,line width= 0.4pt,line join=round,line cap=round,fill=fillColor] (177.54,516.10) circle (  1.16);

\path[draw=drawColor,line width= 0.4pt,line join=round,line cap=round,fill=fillColor] (177.62,516.07) circle (  1.16);

\path[draw=drawColor,line width= 0.4pt,line join=round,line cap=round,fill=fillColor] (177.70,515.99) circle (  1.16);

\path[draw=drawColor,line width= 0.4pt,line join=round,line cap=round,fill=fillColor] (177.78,515.98) circle (  1.16);

\path[draw=drawColor,line width= 0.4pt,line join=round,line cap=round,fill=fillColor] (177.86,515.97) circle (  1.16);

\path[draw=drawColor,line width= 0.4pt,line join=round,line cap=round,fill=fillColor] (177.94,515.93) circle (  1.16);

\path[draw=drawColor,line width= 0.4pt,line join=round,line cap=round,fill=fillColor] (178.02,515.92) circle (  1.16);

\path[draw=drawColor,line width= 0.4pt,line join=round,line cap=round,fill=fillColor] (178.10,515.90) circle (  1.16);

\path[draw=drawColor,line width= 0.4pt,line join=round,line cap=round,fill=fillColor] (178.18,515.89) circle (  1.16);

\path[draw=drawColor,line width= 0.4pt,line join=round,line cap=round,fill=fillColor] (178.26,515.86) circle (  1.16);

\path[draw=drawColor,line width= 0.4pt,line join=round,line cap=round,fill=fillColor] (178.34,515.83) circle (  1.16);

\path[draw=drawColor,line width= 0.4pt,line join=round,line cap=round,fill=fillColor] (178.42,515.74) circle (  1.16);

\path[draw=drawColor,line width= 0.4pt,line join=round,line cap=round,fill=fillColor] (178.50,515.73) circle (  1.16);

\path[draw=drawColor,line width= 0.4pt,line join=round,line cap=round,fill=fillColor] (178.57,515.68) circle (  1.16);

\path[draw=drawColor,line width= 0.4pt,line join=round,line cap=round,fill=fillColor] (178.65,515.68) circle (  1.16);

\path[draw=drawColor,line width= 0.4pt,line join=round,line cap=round,fill=fillColor] (178.73,515.66) circle (  1.16);

\path[draw=drawColor,line width= 0.4pt,line join=round,line cap=round,fill=fillColor] (178.81,515.62) circle (  1.16);

\path[draw=drawColor,line width= 0.4pt,line join=round,line cap=round,fill=fillColor] (178.89,515.59) circle (  1.16);

\path[draw=drawColor,line width= 0.4pt,line join=round,line cap=round,fill=fillColor] (178.97,515.57) circle (  1.16);

\path[draw=drawColor,line width= 0.4pt,line join=round,line cap=round,fill=fillColor] (179.04,515.54) circle (  1.16);

\path[draw=drawColor,line width= 0.4pt,line join=round,line cap=round,fill=fillColor] (179.12,515.53) circle (  1.16);

\path[draw=drawColor,line width= 0.4pt,line join=round,line cap=round,fill=fillColor] (179.20,515.52) circle (  1.16);

\path[draw=drawColor,line width= 0.4pt,line join=round,line cap=round,fill=fillColor] (179.28,515.50) circle (  1.16);

\path[draw=drawColor,line width= 0.4pt,line join=round,line cap=round,fill=fillColor] (179.36,515.50) circle (  1.16);

\path[draw=drawColor,line width= 0.4pt,line join=round,line cap=round,fill=fillColor] (179.43,515.49) circle (  1.16);

\path[draw=drawColor,line width= 0.4pt,line join=round,line cap=round,fill=fillColor] (179.51,515.48) circle (  1.16);

\path[draw=drawColor,line width= 0.4pt,line join=round,line cap=round,fill=fillColor] (179.59,515.48) circle (  1.16);

\path[draw=drawColor,line width= 0.4pt,line join=round,line cap=round,fill=fillColor] (179.67,515.45) circle (  1.16);

\path[draw=drawColor,line width= 0.4pt,line join=round,line cap=round,fill=fillColor] (179.74,515.40) circle (  1.16);

\path[draw=drawColor,line width= 0.4pt,line join=round,line cap=round,fill=fillColor] (179.82,515.40) circle (  1.16);

\path[draw=drawColor,line width= 0.4pt,line join=round,line cap=round,fill=fillColor] (179.90,515.39) circle (  1.16);

\path[draw=drawColor,line width= 0.4pt,line join=round,line cap=round,fill=fillColor] (179.97,515.36) circle (  1.16);

\path[draw=drawColor,line width= 0.4pt,line join=round,line cap=round,fill=fillColor] (180.05,515.35) circle (  1.16);

\path[draw=drawColor,line width= 0.4pt,line join=round,line cap=round,fill=fillColor] (180.13,515.29) circle (  1.16);

\path[draw=drawColor,line width= 0.4pt,line join=round,line cap=round,fill=fillColor] (180.20,515.28) circle (  1.16);

\path[draw=drawColor,line width= 0.4pt,line join=round,line cap=round,fill=fillColor] (180.28,515.23) circle (  1.16);

\path[draw=drawColor,line width= 0.4pt,line join=round,line cap=round,fill=fillColor] (180.36,515.23) circle (  1.16);

\path[draw=drawColor,line width= 0.4pt,line join=round,line cap=round,fill=fillColor] (180.43,515.15) circle (  1.16);

\path[draw=drawColor,line width= 0.4pt,line join=round,line cap=round,fill=fillColor] (180.51,515.12) circle (  1.16);

\path[draw=drawColor,line width= 0.4pt,line join=round,line cap=round,fill=fillColor] (180.58,515.12) circle (  1.16);

\path[draw=drawColor,line width= 0.4pt,line join=round,line cap=round,fill=fillColor] (180.66,515.11) circle (  1.16);

\path[draw=drawColor,line width= 0.4pt,line join=round,line cap=round,fill=fillColor] (180.74,515.07) circle (  1.16);

\path[draw=drawColor,line width= 0.4pt,line join=round,line cap=round,fill=fillColor] (180.81,515.06) circle (  1.16);

\path[draw=drawColor,line width= 0.4pt,line join=round,line cap=round,fill=fillColor] (180.89,514.98) circle (  1.16);

\path[draw=drawColor,line width= 0.4pt,line join=round,line cap=round,fill=fillColor] (180.96,514.95) circle (  1.16);

\path[draw=drawColor,line width= 0.4pt,line join=round,line cap=round,fill=fillColor] (181.04,514.92) circle (  1.16);

\path[draw=drawColor,line width= 0.4pt,line join=round,line cap=round,fill=fillColor] (181.11,514.92) circle (  1.16);

\path[draw=drawColor,line width= 0.4pt,line join=round,line cap=round,fill=fillColor] (181.19,514.89) circle (  1.16);

\path[draw=drawColor,line width= 0.4pt,line join=round,line cap=round,fill=fillColor] (181.26,514.80) circle (  1.16);

\path[draw=drawColor,line width= 0.4pt,line join=round,line cap=round,fill=fillColor] (181.34,514.79) circle (  1.16);

\path[draw=drawColor,line width= 0.4pt,line join=round,line cap=round,fill=fillColor] (181.41,514.79) circle (  1.16);

\path[draw=drawColor,line width= 0.4pt,line join=round,line cap=round,fill=fillColor] (181.49,514.77) circle (  1.16);

\path[draw=drawColor,line width= 0.4pt,line join=round,line cap=round,fill=fillColor] (181.56,514.76) circle (  1.16);

\path[draw=drawColor,line width= 0.4pt,line join=round,line cap=round,fill=fillColor] (181.64,514.76) circle (  1.16);

\path[draw=drawColor,line width= 0.4pt,line join=round,line cap=round,fill=fillColor] (181.71,514.73) circle (  1.16);

\path[draw=drawColor,line width= 0.4pt,line join=round,line cap=round,fill=fillColor] (181.79,514.72) circle (  1.16);

\path[draw=drawColor,line width= 0.4pt,line join=round,line cap=round,fill=fillColor] (181.86,514.68) circle (  1.16);

\path[draw=drawColor,line width= 0.4pt,line join=round,line cap=round,fill=fillColor] (181.94,514.67) circle (  1.16);

\path[draw=drawColor,line width= 0.4pt,line join=round,line cap=round,fill=fillColor] (182.01,514.66) circle (  1.16);

\path[draw=drawColor,line width= 0.4pt,line join=round,line cap=round,fill=fillColor] (182.08,514.63) circle (  1.16);

\path[draw=drawColor,line width= 0.4pt,line join=round,line cap=round,fill=fillColor] (182.16,514.58) circle (  1.16);

\path[draw=drawColor,line width= 0.4pt,line join=round,line cap=round,fill=fillColor] (182.23,514.56) circle (  1.16);

\path[draw=drawColor,line width= 0.4pt,line join=round,line cap=round,fill=fillColor] (182.31,514.56) circle (  1.16);

\path[draw=drawColor,line width= 0.4pt,line join=round,line cap=round,fill=fillColor] (182.38,514.53) circle (  1.16);

\path[draw=drawColor,line width= 0.4pt,line join=round,line cap=round,fill=fillColor] (182.45,514.51) circle (  1.16);

\path[draw=drawColor,line width= 0.4pt,line join=round,line cap=round,fill=fillColor] (182.53,514.47) circle (  1.16);

\path[draw=drawColor,line width= 0.4pt,line join=round,line cap=round,fill=fillColor] (182.60,514.46) circle (  1.16);

\path[draw=drawColor,line width= 0.4pt,line join=round,line cap=round,fill=fillColor] (182.67,514.43) circle (  1.16);

\path[draw=drawColor,line width= 0.4pt,line join=round,line cap=round,fill=fillColor] (182.75,514.40) circle (  1.16);

\path[draw=drawColor,line width= 0.4pt,line join=round,line cap=round,fill=fillColor] (182.82,514.40) circle (  1.16);

\path[draw=drawColor,line width= 0.4pt,line join=round,line cap=round,fill=fillColor] (182.89,514.36) circle (  1.16);

\path[draw=drawColor,line width= 0.4pt,line join=round,line cap=round,fill=fillColor] (182.96,514.30) circle (  1.16);

\path[draw=drawColor,line width= 0.4pt,line join=round,line cap=round,fill=fillColor] (183.04,514.20) circle (  1.16);

\path[draw=drawColor,line width= 0.4pt,line join=round,line cap=round,fill=fillColor] (183.11,514.19) circle (  1.16);

\path[draw=drawColor,line width= 0.4pt,line join=round,line cap=round,fill=fillColor] (183.18,514.19) circle (  1.16);

\path[draw=drawColor,line width= 0.4pt,line join=round,line cap=round,fill=fillColor] (183.26,514.19) circle (  1.16);

\path[draw=drawColor,line width= 0.4pt,line join=round,line cap=round,fill=fillColor] (183.33,514.18) circle (  1.16);

\path[draw=drawColor,line width= 0.4pt,line join=round,line cap=round,fill=fillColor] (183.40,514.06) circle (  1.16);

\path[draw=drawColor,line width= 0.4pt,line join=round,line cap=round,fill=fillColor] (183.47,514.04) circle (  1.16);

\path[draw=drawColor,line width= 0.4pt,line join=round,line cap=round,fill=fillColor] (183.54,514.00) circle (  1.16);

\path[draw=drawColor,line width= 0.4pt,line join=round,line cap=round,fill=fillColor] (183.62,513.99) circle (  1.16);

\path[draw=drawColor,line width= 0.4pt,line join=round,line cap=round,fill=fillColor] (183.69,513.96) circle (  1.16);

\path[draw=drawColor,line width= 0.4pt,line join=round,line cap=round,fill=fillColor] (183.76,513.95) circle (  1.16);

\path[draw=drawColor,line width= 0.4pt,line join=round,line cap=round,fill=fillColor] (183.83,513.93) circle (  1.16);

\path[draw=drawColor,line width= 0.4pt,line join=round,line cap=round,fill=fillColor] (183.90,513.91) circle (  1.16);

\path[draw=drawColor,line width= 0.4pt,line join=round,line cap=round,fill=fillColor] (183.98,513.89) circle (  1.16);

\path[draw=drawColor,line width= 0.4pt,line join=round,line cap=round,fill=fillColor] (184.05,513.88) circle (  1.16);

\path[draw=drawColor,line width= 0.4pt,line join=round,line cap=round,fill=fillColor] (184.12,513.87) circle (  1.16);

\path[draw=drawColor,line width= 0.4pt,line join=round,line cap=round,fill=fillColor] (184.19,513.86) circle (  1.16);

\path[draw=drawColor,line width= 0.4pt,line join=round,line cap=round,fill=fillColor] (184.26,513.86) circle (  1.16);

\path[draw=drawColor,line width= 0.4pt,line join=round,line cap=round,fill=fillColor] (184.33,513.84) circle (  1.16);

\path[draw=drawColor,line width= 0.4pt,line join=round,line cap=round,fill=fillColor] (184.40,513.83) circle (  1.16);

\path[draw=drawColor,line width= 0.4pt,line join=round,line cap=round,fill=fillColor] (184.48,513.80) circle (  1.16);

\path[draw=drawColor,line width= 0.4pt,line join=round,line cap=round,fill=fillColor] (184.55,513.79) circle (  1.16);

\path[draw=drawColor,line width= 0.4pt,line join=round,line cap=round,fill=fillColor] (184.62,513.70) circle (  1.16);

\path[draw=drawColor,line width= 0.4pt,line join=round,line cap=round,fill=fillColor] (184.69,513.68) circle (  1.16);

\path[draw=drawColor,line width= 0.4pt,line join=round,line cap=round,fill=fillColor] (184.76,513.63) circle (  1.16);

\path[draw=drawColor,line width= 0.4pt,line join=round,line cap=round,fill=fillColor] (184.83,513.62) circle (  1.16);

\path[draw=drawColor,line width= 0.4pt,line join=round,line cap=round,fill=fillColor] (184.90,513.60) circle (  1.16);

\path[draw=drawColor,line width= 0.4pt,line join=round,line cap=round,fill=fillColor] (184.97,513.59) circle (  1.16);

\path[draw=drawColor,line width= 0.4pt,line join=round,line cap=round,fill=fillColor] (185.04,513.58) circle (  1.16);

\path[draw=drawColor,line width= 0.4pt,line join=round,line cap=round,fill=fillColor] (185.11,513.53) circle (  1.16);

\path[draw=drawColor,line width= 0.4pt,line join=round,line cap=round,fill=fillColor] (185.18,513.52) circle (  1.16);

\path[draw=drawColor,line width= 0.4pt,line join=round,line cap=round,fill=fillColor] (185.25,513.48) circle (  1.16);

\path[draw=drawColor,line width= 0.4pt,line join=round,line cap=round,fill=fillColor] (185.32,513.39) circle (  1.16);

\path[draw=drawColor,line width= 0.4pt,line join=round,line cap=round,fill=fillColor] (185.39,513.37) circle (  1.16);

\path[draw=drawColor,line width= 0.4pt,line join=round,line cap=round,fill=fillColor] (185.46,513.36) circle (  1.16);

\path[draw=drawColor,line width= 0.4pt,line join=round,line cap=round,fill=fillColor] (185.53,513.35) circle (  1.16);

\path[draw=drawColor,line width= 0.4pt,line join=round,line cap=round,fill=fillColor] (185.60,513.27) circle (  1.16);

\path[draw=drawColor,line width= 0.4pt,line join=round,line cap=round,fill=fillColor] (185.67,513.19) circle (  1.16);

\path[draw=drawColor,line width= 0.4pt,line join=round,line cap=round,fill=fillColor] (185.74,513.18) circle (  1.16);

\path[draw=drawColor,line width= 0.4pt,line join=round,line cap=round,fill=fillColor] (185.81,513.12) circle (  1.16);

\path[draw=drawColor,line width= 0.4pt,line join=round,line cap=round,fill=fillColor] (185.88,513.05) circle (  1.16);

\path[draw=drawColor,line width= 0.4pt,line join=round,line cap=round,fill=fillColor] (185.95,512.84) circle (  1.16);

\path[draw=drawColor,line width= 0.4pt,line join=round,line cap=round,fill=fillColor] (186.02,512.84) circle (  1.16);

\path[draw=drawColor,line width= 0.4pt,line join=round,line cap=round,fill=fillColor] (186.09,512.78) circle (  1.16);

\path[draw=drawColor,line width= 0.4pt,line join=round,line cap=round,fill=fillColor] (186.16,512.77) circle (  1.16);

\path[draw=drawColor,line width= 0.4pt,line join=round,line cap=round,fill=fillColor] (186.22,512.68) circle (  1.16);

\path[draw=drawColor,line width= 0.4pt,line join=round,line cap=round,fill=fillColor] (186.29,512.62) circle (  1.16);

\path[draw=drawColor,line width= 0.4pt,line join=round,line cap=round,fill=fillColor] (186.36,512.60) circle (  1.16);

\path[draw=drawColor,line width= 0.4pt,line join=round,line cap=round,fill=fillColor] (186.43,512.49) circle (  1.16);

\path[draw=drawColor,line width= 0.4pt,line join=round,line cap=round,fill=fillColor] (186.50,512.47) circle (  1.16);

\path[draw=drawColor,line width= 0.4pt,line join=round,line cap=round,fill=fillColor] (186.57,512.39) circle (  1.16);

\path[draw=drawColor,line width= 0.4pt,line join=round,line cap=round,fill=fillColor] (186.64,512.29) circle (  1.16);

\path[draw=drawColor,line width= 0.4pt,line join=round,line cap=round,fill=fillColor] (186.70,512.20) circle (  1.16);

\path[draw=drawColor,line width= 0.4pt,line join=round,line cap=round,fill=fillColor] (186.77,511.94) circle (  1.16);

\path[draw=drawColor,line width= 0.4pt,line join=round,line cap=round,fill=fillColor] (186.84,511.77) circle (  1.16);

\path[draw=drawColor,line width= 0.4pt,line join=round,line cap=round,fill=fillColor] (186.91,511.62) circle (  1.16);

\path[draw=drawColor,line width= 0.4pt,line join=round,line cap=round,fill=fillColor] (186.98,511.56) circle (  1.16);

\path[draw=drawColor,line width= 0.4pt,line join=round,line cap=round,fill=fillColor] (187.05,511.31) circle (  1.16);

\path[draw=drawColor,line width= 0.4pt,line join=round,line cap=round,fill=fillColor] (187.11,510.94) circle (  1.16);

\path[draw=drawColor,line width= 0.4pt,line join=round,line cap=round,fill=fillColor] (187.18,510.90) circle (  1.16);

\path[draw=drawColor,line width= 0.4pt,line join=round,line cap=round,fill=fillColor] (187.25,510.59) circle (  1.16);

\path[draw=drawColor,line width= 0.4pt,line join=round,line cap=round,fill=fillColor] (187.32,508.31) circle (  1.16);

\path[draw=drawColor,line width= 0.4pt,line join=round,line cap=round,fill=fillColor] (187.38,508.31) circle (  1.16);

\path[draw=drawColor,line width= 0.4pt,line join=round,line cap=round,fill=fillColor] (187.45,508.31) circle (  1.16);

\path[draw=drawColor,line width= 0.4pt,line join=round,line cap=round,fill=fillColor] (187.52,508.31) circle (  1.16);

\path[draw=drawColor,line width= 0.4pt,line join=round,line cap=round,fill=fillColor] (187.59,508.31) circle (  1.16);

\path[draw=drawColor,line width= 0.4pt,line join=round,line cap=round,fill=fillColor] (187.65,508.31) circle (  1.16);

\path[draw=drawColor,line width= 0.4pt,line join=round,line cap=round,fill=fillColor] (187.72,508.31) circle (  1.16);

\path[draw=drawColor,line width= 0.4pt,line join=round,line cap=round,fill=fillColor] (187.79,508.31) circle (  1.16);

\path[draw=drawColor,line width= 0.4pt,line join=round,line cap=round,fill=fillColor] (187.86,508.31) circle (  1.16);

\path[draw=drawColor,line width= 0.4pt,line join=round,line cap=round,fill=fillColor] (187.92,508.31) circle (  1.16);

\path[draw=drawColor,line width= 0.4pt,line join=round,line cap=round,fill=fillColor] (187.99,508.31) circle (  1.16);

\path[draw=drawColor,line width= 0.4pt,line join=round,line cap=round,fill=fillColor] (188.06,508.31) circle (  1.16);

\path[draw=drawColor,line width= 0.4pt,line join=round,line cap=round,fill=fillColor] (188.12,508.31) circle (  1.16);

\path[draw=drawColor,line width= 0.4pt,line join=round,line cap=round,fill=fillColor] (188.19,508.31) circle (  1.16);

\path[draw=drawColor,line width= 0.4pt,line join=round,line cap=round,fill=fillColor] (188.26,508.31) circle (  1.16);

\path[draw=drawColor,line width= 0.4pt,line join=round,line cap=round,fill=fillColor] (188.32,508.31) circle (  1.16);

\path[draw=drawColor,line width= 0.4pt,line join=round,line cap=round,fill=fillColor] (188.39,508.31) circle (  1.16);

\path[draw=drawColor,line width= 0.4pt,line join=round,line cap=round,fill=fillColor] (188.46,508.31) circle (  1.16);

\path[draw=drawColor,line width= 0.4pt,line join=round,line cap=round,fill=fillColor] (188.52,508.31) circle (  1.16);

\path[draw=drawColor,line width= 0.4pt,line join=round,line cap=round,fill=fillColor] (188.59,508.31) circle (  1.16);

\path[draw=drawColor,line width= 0.4pt,line join=round,line cap=round,fill=fillColor] (188.66,508.31) circle (  1.16);

\path[draw=drawColor,line width= 0.4pt,line join=round,line cap=round,fill=fillColor] (188.72,508.31) circle (  1.16);

\path[draw=drawColor,line width= 0.4pt,line join=round,line cap=round,fill=fillColor] (188.79,508.31) circle (  1.16);

\path[draw=drawColor,line width= 0.4pt,line join=round,line cap=round,fill=fillColor] (188.85,508.31) circle (  1.16);

\path[draw=drawColor,line width= 0.4pt,line join=round,line cap=round,fill=fillColor] (188.92,508.31) circle (  1.16);

\path[draw=drawColor,line width= 0.4pt,line join=round,line cap=round,fill=fillColor] (188.99,508.31) circle (  1.16);

\path[draw=drawColor,line width= 0.4pt,line join=round,line cap=round,fill=fillColor] (189.05,508.31) circle (  1.16);

\path[draw=drawColor,line width= 0.4pt,line join=round,line cap=round,fill=fillColor] (189.12,508.31) circle (  1.16);

\path[draw=drawColor,line width= 0.4pt,line join=round,line cap=round,fill=fillColor] (189.18,508.31) circle (  1.16);

\path[draw=drawColor,line width= 0.4pt,line join=round,line cap=round,fill=fillColor] (189.25,508.31) circle (  1.16);

\path[draw=drawColor,line width= 0.4pt,line join=round,line cap=round,fill=fillColor] (189.31,508.31) circle (  1.16);

\path[draw=drawColor,line width= 0.4pt,line join=round,line cap=round,fill=fillColor] (189.38,508.31) circle (  1.16);

\path[draw=drawColor,line width= 0.4pt,line join=round,line cap=round,fill=fillColor] (189.45,508.31) circle (  1.16);

\path[draw=drawColor,line width= 0.4pt,line join=round,line cap=round,fill=fillColor] (189.51,508.31) circle (  1.16);

\path[draw=drawColor,line width= 0.4pt,line join=round,line cap=round,fill=fillColor] (189.58,508.31) circle (  1.16);

\path[draw=drawColor,line width= 0.4pt,line join=round,line cap=round,fill=fillColor] (189.64,508.31) circle (  1.16);

\path[draw=drawColor,line width= 0.4pt,line join=round,line cap=round,fill=fillColor] (189.71,508.31) circle (  1.16);

\path[draw=drawColor,line width= 0.4pt,line join=round,line cap=round,fill=fillColor] (189.77,508.31) circle (  1.16);

\path[draw=drawColor,line width= 0.4pt,line join=round,line cap=round,fill=fillColor] (189.84,508.31) circle (  1.16);

\path[draw=drawColor,line width= 0.4pt,line join=round,line cap=round,fill=fillColor] (189.90,508.31) circle (  1.16);

\path[draw=drawColor,line width= 0.4pt,line join=round,line cap=round,fill=fillColor] (189.97,508.31) circle (  1.16);

\path[draw=drawColor,line width= 0.4pt,line join=round,line cap=round,fill=fillColor] (190.03,508.31) circle (  1.16);

\path[draw=drawColor,line width= 0.4pt,line join=round,line cap=round,fill=fillColor] (190.10,508.31) circle (  1.16);

\path[draw=drawColor,line width= 0.4pt,line join=round,line cap=round,fill=fillColor] (190.16,508.31) circle (  1.16);

\path[draw=drawColor,line width= 0.4pt,line join=round,line cap=round,fill=fillColor] (190.23,508.31) circle (  1.16);

\path[draw=drawColor,line width= 0.4pt,line join=round,line cap=round,fill=fillColor] (190.29,508.31) circle (  1.16);

\path[draw=drawColor,line width= 0.4pt,line join=round,line cap=round,fill=fillColor] (190.35,508.31) circle (  1.16);

\path[draw=drawColor,line width= 0.4pt,line join=round,line cap=round,fill=fillColor] (190.42,508.31) circle (  1.16);

\path[draw=drawColor,line width= 0.4pt,line join=round,line cap=round,fill=fillColor] (190.48,508.31) circle (  1.16);

\path[draw=drawColor,line width= 0.4pt,line join=round,line cap=round,fill=fillColor] (190.55,508.31) circle (  1.16);

\path[draw=drawColor,line width= 0.4pt,line join=round,line cap=round,fill=fillColor] (190.61,508.31) circle (  1.16);

\path[draw=drawColor,line width= 0.4pt,line join=round,line cap=round,fill=fillColor] (190.68,508.31) circle (  1.16);

\path[draw=drawColor,line width= 0.4pt,line join=round,line cap=round,fill=fillColor] (190.74,508.31) circle (  1.16);

\path[draw=drawColor,line width= 0.4pt,line join=round,line cap=round,fill=fillColor] (190.80,508.31) circle (  1.16);

\path[draw=drawColor,line width= 0.4pt,line join=round,line cap=round,fill=fillColor] (190.87,508.31) circle (  1.16);

\path[draw=drawColor,line width= 0.4pt,line join=round,line cap=round,fill=fillColor] (190.93,508.31) circle (  1.16);

\path[draw=drawColor,line width= 0.4pt,line join=round,line cap=round,fill=fillColor] (190.99,508.31) circle (  1.16);

\path[draw=drawColor,line width= 0.4pt,line join=round,line cap=round,fill=fillColor] (191.06,508.31) circle (  1.16);

\path[draw=drawColor,line width= 0.4pt,line join=round,line cap=round,fill=fillColor] (191.12,508.31) circle (  1.16);

\path[draw=drawColor,line width= 0.4pt,line join=round,line cap=round,fill=fillColor] (191.19,508.31) circle (  1.16);

\path[draw=drawColor,line width= 0.4pt,line join=round,line cap=round,fill=fillColor] (191.25,508.31) circle (  1.16);

\path[draw=drawColor,line width= 0.4pt,line join=round,line cap=round,fill=fillColor] (191.31,508.31) circle (  1.16);

\path[draw=drawColor,line width= 0.4pt,line join=round,line cap=round,fill=fillColor] (191.38,508.31) circle (  1.16);

\path[draw=drawColor,line width= 0.4pt,line join=round,line cap=round,fill=fillColor] (191.44,508.31) circle (  1.16);

\path[draw=drawColor,line width= 0.4pt,line join=round,line cap=round,fill=fillColor] (191.50,508.31) circle (  1.16);

\path[draw=drawColor,line width= 0.4pt,line join=round,line cap=round,fill=fillColor] (191.57,508.31) circle (  1.16);

\path[draw=drawColor,line width= 0.4pt,line join=round,line cap=round,fill=fillColor] (191.63,508.31) circle (  1.16);

\path[draw=drawColor,line width= 0.4pt,line join=round,line cap=round,fill=fillColor] (191.69,508.31) circle (  1.16);

\path[draw=drawColor,line width= 0.4pt,line join=round,line cap=round,fill=fillColor] (191.75,508.31) circle (  1.16);

\path[draw=drawColor,line width= 0.4pt,line join=round,line cap=round,fill=fillColor] (191.82,508.31) circle (  1.16);

\path[draw=drawColor,line width= 0.4pt,line join=round,line cap=round,fill=fillColor] (191.88,508.31) circle (  1.16);

\path[draw=drawColor,line width= 0.4pt,line join=round,line cap=round,fill=fillColor] (191.94,508.31) circle (  1.16);

\path[draw=drawColor,line width= 0.4pt,line join=round,line cap=round,fill=fillColor] (192.01,508.31) circle (  1.16);

\path[draw=drawColor,line width= 0.4pt,line join=round,line cap=round,fill=fillColor] (192.07,508.31) circle (  1.16);

\path[draw=drawColor,line width= 0.4pt,line join=round,line cap=round,fill=fillColor] (192.13,508.31) circle (  1.16);

\path[draw=drawColor,line width= 0.4pt,line join=round,line cap=round,fill=fillColor] (192.19,508.31) circle (  1.16);

\path[draw=drawColor,line width= 0.4pt,line join=round,line cap=round,fill=fillColor] (192.26,508.31) circle (  1.16);

\path[draw=drawColor,line width= 0.4pt,line join=round,line cap=round,fill=fillColor] (192.32,508.31) circle (  1.16);

\path[draw=drawColor,line width= 0.4pt,line join=round,line cap=round,fill=fillColor] (192.38,508.31) circle (  1.16);

\path[draw=drawColor,line width= 0.4pt,line join=round,line cap=round,fill=fillColor] (192.44,508.31) circle (  1.16);

\path[draw=drawColor,line width= 0.4pt,line join=round,line cap=round,fill=fillColor] (192.51,508.31) circle (  1.16);

\path[draw=drawColor,line width= 0.4pt,line join=round,line cap=round,fill=fillColor] (192.57,508.31) circle (  1.16);

\path[draw=drawColor,line width= 0.4pt,line join=round,line cap=round,fill=fillColor] (192.63,508.31) circle (  1.16);

\path[draw=drawColor,line width= 0.4pt,line join=round,line cap=round,fill=fillColor] (192.69,508.31) circle (  1.16);

\path[draw=drawColor,line width= 0.4pt,line join=round,line cap=round,fill=fillColor] (192.75,508.31) circle (  1.16);

\path[draw=drawColor,line width= 0.4pt,line join=round,line cap=round,fill=fillColor] (192.82,508.31) circle (  1.16);

\path[draw=drawColor,line width= 0.4pt,line join=round,line cap=round,fill=fillColor] (192.88,508.31) circle (  1.16);

\path[draw=drawColor,line width= 0.4pt,line join=round,line cap=round,fill=fillColor] (192.94,508.31) circle (  1.16);

\path[draw=drawColor,line width= 0.4pt,line join=round,line cap=round,fill=fillColor] (193.00,508.31) circle (  1.16);

\path[draw=drawColor,line width= 0.4pt,line join=round,line cap=round,fill=fillColor] (193.06,508.31) circle (  1.16);

\path[draw=drawColor,line width= 0.4pt,line join=round,line cap=round,fill=fillColor] (193.13,508.31) circle (  1.16);

\path[draw=drawColor,line width= 0.4pt,line join=round,line cap=round,fill=fillColor] (193.19,508.31) circle (  1.16);

\path[draw=drawColor,line width= 0.4pt,line join=round,line cap=round,fill=fillColor] (193.25,508.31) circle (  1.16);

\path[draw=drawColor,line width= 0.4pt,line join=round,line cap=round,fill=fillColor] (193.31,508.31) circle (  1.16);

\path[draw=drawColor,line width= 0.4pt,line join=round,line cap=round,fill=fillColor] (193.37,508.31) circle (  1.16);

\path[draw=drawColor,line width= 0.4pt,line join=round,line cap=round,fill=fillColor] (193.43,508.31) circle (  1.16);

\path[draw=drawColor,line width= 0.4pt,line join=round,line cap=round,fill=fillColor] (193.49,508.31) circle (  1.16);

\path[draw=drawColor,line width= 0.4pt,line join=round,line cap=round,fill=fillColor] (193.55,508.31) circle (  1.16);

\path[draw=drawColor,line width= 0.4pt,line join=round,line cap=round,fill=fillColor] (193.62,508.31) circle (  1.16);

\path[draw=drawColor,line width= 0.4pt,line join=round,line cap=round,fill=fillColor] (193.68,508.31) circle (  1.16);

\path[draw=drawColor,line width= 0.4pt,line join=round,line cap=round,fill=fillColor] (193.74,508.31) circle (  1.16);

\path[draw=drawColor,line width= 0.4pt,line join=round,line cap=round,fill=fillColor] (193.80,508.31) circle (  1.16);

\path[draw=drawColor,line width= 0.4pt,line join=round,line cap=round,fill=fillColor] (193.86,508.31) circle (  1.16);

\path[draw=drawColor,line width= 0.4pt,line join=round,line cap=round,fill=fillColor] (193.92,508.31) circle (  1.16);

\path[draw=drawColor,line width= 0.4pt,line join=round,line cap=round,fill=fillColor] (193.98,508.31) circle (  1.16);

\path[draw=drawColor,line width= 0.4pt,line join=round,line cap=round,fill=fillColor] (194.04,508.31) circle (  1.16);

\path[draw=drawColor,line width= 0.4pt,line join=round,line cap=round,fill=fillColor] (194.10,508.31) circle (  1.16);

\path[draw=drawColor,line width= 0.4pt,line join=round,line cap=round,fill=fillColor] (194.16,508.31) circle (  1.16);

\path[draw=drawColor,line width= 0.4pt,line join=round,line cap=round,fill=fillColor] (194.22,508.31) circle (  1.16);

\path[draw=drawColor,line width= 0.4pt,line join=round,line cap=round,fill=fillColor] (194.29,508.31) circle (  1.16);

\path[draw=drawColor,line width= 0.4pt,line join=round,line cap=round,fill=fillColor] (194.35,508.31) circle (  1.16);

\path[draw=drawColor,line width= 0.4pt,line join=round,line cap=round,fill=fillColor] (194.41,508.31) circle (  1.16);

\path[draw=drawColor,line width= 0.4pt,line join=round,line cap=round,fill=fillColor] (194.47,508.31) circle (  1.16);

\path[draw=drawColor,line width= 0.4pt,line join=round,line cap=round,fill=fillColor] (194.53,508.31) circle (  1.16);

\path[draw=drawColor,line width= 0.4pt,line join=round,line cap=round,fill=fillColor] (194.59,508.31) circle (  1.16);

\path[draw=drawColor,line width= 0.4pt,line join=round,line cap=round,fill=fillColor] (194.65,508.31) circle (  1.16);

\path[draw=drawColor,line width= 0.4pt,line join=round,line cap=round,fill=fillColor] (194.71,508.31) circle (  1.16);

\path[draw=drawColor,line width= 0.4pt,line join=round,line cap=round,fill=fillColor] (194.77,508.31) circle (  1.16);

\path[draw=drawColor,line width= 0.4pt,line join=round,line cap=round,fill=fillColor] (194.83,508.31) circle (  1.16);

\path[draw=drawColor,line width= 0.4pt,line join=round,line cap=round,fill=fillColor] (194.89,508.31) circle (  1.16);

\path[draw=drawColor,line width= 0.4pt,line join=round,line cap=round,fill=fillColor] (194.95,508.31) circle (  1.16);

\path[draw=drawColor,line width= 0.4pt,line join=round,line cap=round,fill=fillColor] (195.01,508.31) circle (  1.16);

\path[draw=drawColor,line width= 0.4pt,line join=round,line cap=round,fill=fillColor] (195.07,508.31) circle (  1.16);

\path[draw=drawColor,line width= 0.4pt,line join=round,line cap=round,fill=fillColor] (195.13,508.31) circle (  1.16);

\path[draw=drawColor,line width= 0.4pt,line join=round,line cap=round,fill=fillColor] (195.19,508.31) circle (  1.16);

\path[draw=drawColor,line width= 0.4pt,line join=round,line cap=round,fill=fillColor] (195.25,508.31) circle (  1.16);

\path[draw=drawColor,line width= 0.4pt,line join=round,line cap=round,fill=fillColor] (195.31,508.31) circle (  1.16);

\path[draw=drawColor,line width= 0.4pt,line join=round,line cap=round,fill=fillColor] (195.36,508.31) circle (  1.16);

\path[draw=drawColor,line width= 0.4pt,line join=round,line cap=round,fill=fillColor] (195.42,508.31) circle (  1.16);

\path[draw=drawColor,line width= 0.4pt,line join=round,line cap=round,fill=fillColor] (195.48,508.31) circle (  1.16);

\path[draw=drawColor,line width= 0.4pt,line join=round,line cap=round,fill=fillColor] (195.54,508.31) circle (  1.16);

\path[draw=drawColor,line width= 0.4pt,line join=round,line cap=round,fill=fillColor] (195.60,508.31) circle (  1.16);

\path[draw=drawColor,line width= 0.4pt,line join=round,line cap=round,fill=fillColor] (195.66,508.31) circle (  1.16);

\path[draw=drawColor,line width= 0.4pt,line join=round,line cap=round,fill=fillColor] (195.72,508.31) circle (  1.16);

\path[draw=drawColor,line width= 0.4pt,line join=round,line cap=round,fill=fillColor] (195.78,508.31) circle (  1.16);

\path[draw=drawColor,line width= 0.4pt,line join=round,line cap=round,fill=fillColor] (195.84,508.31) circle (  1.16);

\path[draw=drawColor,line width= 0.4pt,line join=round,line cap=round,fill=fillColor] (195.90,508.31) circle (  1.16);

\path[draw=drawColor,line width= 0.4pt,line join=round,line cap=round,fill=fillColor] (195.96,508.31) circle (  1.16);

\path[draw=drawColor,line width= 0.4pt,line join=round,line cap=round,fill=fillColor] (196.02,508.31) circle (  1.16);

\path[draw=drawColor,line width= 0.4pt,line join=round,line cap=round,fill=fillColor] (196.07,508.31) circle (  1.16);

\path[draw=drawColor,line width= 0.4pt,line join=round,line cap=round,fill=fillColor] (196.13,508.31) circle (  1.16);

\path[draw=drawColor,line width= 0.4pt,line join=round,line cap=round,fill=fillColor] (196.19,508.31) circle (  1.16);

\path[draw=drawColor,line width= 0.4pt,line join=round,line cap=round,fill=fillColor] (196.25,508.31) circle (  1.16);

\path[draw=drawColor,line width= 0.4pt,line join=round,line cap=round,fill=fillColor] (196.31,508.31) circle (  1.16);

\path[draw=drawColor,line width= 0.4pt,line join=round,line cap=round,fill=fillColor] (196.37,508.31) circle (  1.16);

\path[draw=drawColor,line width= 0.4pt,line join=round,line cap=round,fill=fillColor] (196.43,508.31) circle (  1.16);

\path[draw=drawColor,line width= 0.4pt,line join=round,line cap=round,fill=fillColor] (196.48,508.31) circle (  1.16);

\path[draw=drawColor,line width= 0.4pt,line join=round,line cap=round,fill=fillColor] (196.54,508.31) circle (  1.16);

\path[draw=drawColor,line width= 0.4pt,line join=round,line cap=round,fill=fillColor] (196.60,508.31) circle (  1.16);

\path[draw=drawColor,line width= 0.4pt,line join=round,line cap=round,fill=fillColor] (196.66,508.31) circle (  1.16);

\path[draw=drawColor,line width= 0.4pt,line join=round,line cap=round,fill=fillColor] (196.72,508.31) circle (  1.16);

\path[draw=drawColor,line width= 0.4pt,line join=round,line cap=round,fill=fillColor] (196.78,508.31) circle (  1.16);

\path[draw=drawColor,line width= 0.4pt,line join=round,line cap=round,fill=fillColor] (196.83,508.31) circle (  1.16);

\path[draw=drawColor,line width= 0.4pt,line join=round,line cap=round,fill=fillColor] (196.89,508.31) circle (  1.16);

\path[draw=drawColor,line width= 0.4pt,line join=round,line cap=round,fill=fillColor] (196.95,508.31) circle (  1.16);

\path[draw=drawColor,line width= 0.4pt,line join=round,line cap=round,fill=fillColor] (197.01,508.31) circle (  1.16);

\path[draw=drawColor,line width= 0.4pt,line join=round,line cap=round,fill=fillColor] (197.07,508.31) circle (  1.16);

\path[draw=drawColor,line width= 0.4pt,line join=round,line cap=round,fill=fillColor] (197.12,508.31) circle (  1.16);

\path[draw=drawColor,line width= 0.4pt,line join=round,line cap=round,fill=fillColor] (197.18,508.31) circle (  1.16);

\path[draw=drawColor,line width= 0.4pt,line join=round,line cap=round,fill=fillColor] (197.24,508.31) circle (  1.16);

\path[draw=drawColor,line width= 0.4pt,line join=round,line cap=round,fill=fillColor] (197.30,508.31) circle (  1.16);

\path[draw=drawColor,line width= 0.4pt,line join=round,line cap=round,fill=fillColor] (197.36,508.31) circle (  1.16);

\path[draw=drawColor,line width= 0.4pt,line join=round,line cap=round,fill=fillColor] (197.41,508.31) circle (  1.16);

\path[draw=drawColor,line width= 0.4pt,line join=round,line cap=round,fill=fillColor] (197.47,508.31) circle (  1.16);

\path[draw=drawColor,line width= 0.4pt,line join=round,line cap=round,fill=fillColor] (197.53,508.31) circle (  1.16);

\path[draw=drawColor,line width= 0.4pt,line join=round,line cap=round,fill=fillColor] (197.59,508.31) circle (  1.16);

\path[draw=drawColor,line width= 0.4pt,line join=round,line cap=round,fill=fillColor] (197.64,508.31) circle (  1.16);

\path[draw=drawColor,line width= 0.4pt,line join=round,line cap=round,fill=fillColor] (197.70,508.31) circle (  1.16);

\path[draw=drawColor,line width= 0.4pt,line join=round,line cap=round,fill=fillColor] (197.76,508.31) circle (  1.16);

\path[draw=drawColor,line width= 0.4pt,line join=round,line cap=round,fill=fillColor] (197.82,508.31) circle (  1.16);

\path[draw=drawColor,line width= 0.4pt,line join=round,line cap=round,fill=fillColor] (197.87,508.31) circle (  1.16);

\path[draw=drawColor,line width= 0.4pt,line join=round,line cap=round,fill=fillColor] (197.93,508.31) circle (  1.16);

\path[draw=drawColor,line width= 0.4pt,line join=round,line cap=round,fill=fillColor] (197.99,508.31) circle (  1.16);

\path[draw=drawColor,line width= 0.4pt,line join=round,line cap=round,fill=fillColor] (198.04,508.31) circle (  1.16);

\path[draw=drawColor,line width= 0.4pt,line join=round,line cap=round,fill=fillColor] (198.10,508.31) circle (  1.16);

\path[draw=drawColor,line width= 0.4pt,line join=round,line cap=round,fill=fillColor] (198.16,508.31) circle (  1.16);

\path[draw=drawColor,line width= 0.4pt,line join=round,line cap=round,fill=fillColor] (198.22,508.31) circle (  1.16);

\path[draw=drawColor,line width= 0.4pt,line join=round,line cap=round,fill=fillColor] (198.27,508.31) circle (  1.16);

\path[draw=drawColor,line width= 0.4pt,line join=round,line cap=round,fill=fillColor] (198.33,508.31) circle (  1.16);

\path[draw=drawColor,line width= 0.4pt,line join=round,line cap=round,fill=fillColor] (198.39,508.31) circle (  1.16);

\path[draw=drawColor,line width= 0.4pt,line join=round,line cap=round,fill=fillColor] (198.44,508.31) circle (  1.16);

\path[draw=drawColor,line width= 0.4pt,line join=round,line cap=round,fill=fillColor] (198.50,508.31) circle (  1.16);

\path[draw=drawColor,line width= 0.4pt,line join=round,line cap=round,fill=fillColor] (198.56,508.31) circle (  1.16);

\path[draw=drawColor,line width= 0.4pt,line join=round,line cap=round,fill=fillColor] (198.61,508.31) circle (  1.16);

\path[draw=drawColor,line width= 0.4pt,line join=round,line cap=round,fill=fillColor] (198.67,508.31) circle (  1.16);

\path[draw=drawColor,line width= 0.4pt,line join=round,line cap=round,fill=fillColor] (198.73,508.31) circle (  1.16);

\path[draw=drawColor,line width= 0.4pt,line join=round,line cap=round,fill=fillColor] (198.78,508.31) circle (  1.16);

\path[draw=drawColor,line width= 0.4pt,line join=round,line cap=round,fill=fillColor] (198.84,508.31) circle (  1.16);

\path[draw=drawColor,line width= 0.4pt,line join=round,line cap=round,fill=fillColor] (198.90,508.31) circle (  1.16);

\path[draw=drawColor,line width= 0.4pt,line join=round,line cap=round,fill=fillColor] (198.95,508.31) circle (  1.16);

\path[draw=drawColor,line width= 0.4pt,line join=round,line cap=round,fill=fillColor] (199.01,508.31) circle (  1.16);

\path[draw=drawColor,line width= 0.4pt,line join=round,line cap=round,fill=fillColor] (199.06,508.31) circle (  1.16);

\path[draw=drawColor,line width= 0.4pt,line join=round,line cap=round,fill=fillColor] (199.12,508.31) circle (  1.16);

\path[draw=drawColor,line width= 0.4pt,line join=round,line cap=round,fill=fillColor] (199.18,508.31) circle (  1.16);

\path[draw=drawColor,line width= 0.4pt,line join=round,line cap=round,fill=fillColor] (199.23,508.31) circle (  1.16);

\path[draw=drawColor,line width= 0.4pt,line join=round,line cap=round,fill=fillColor] (199.29,508.31) circle (  1.16);

\path[draw=drawColor,line width= 0.4pt,line join=round,line cap=round,fill=fillColor] (199.34,508.31) circle (  1.16);

\path[draw=drawColor,line width= 0.4pt,line join=round,line cap=round,fill=fillColor] (199.40,508.31) circle (  1.16);

\path[draw=drawColor,line width= 0.4pt,line join=round,line cap=round,fill=fillColor] (199.46,508.31) circle (  1.16);

\path[draw=drawColor,line width= 0.4pt,line join=round,line cap=round,fill=fillColor] (199.51,508.31) circle (  1.16);

\path[draw=drawColor,line width= 0.4pt,line join=round,line cap=round,fill=fillColor] (199.57,508.31) circle (  1.16);

\path[draw=drawColor,line width= 0.4pt,line join=round,line cap=round,fill=fillColor] (199.62,508.31) circle (  1.16);

\path[draw=drawColor,line width= 0.4pt,line join=round,line cap=round,fill=fillColor] (199.68,508.31) circle (  1.16);

\path[draw=drawColor,line width= 0.4pt,line join=round,line cap=round,fill=fillColor] (199.73,508.31) circle (  1.16);

\path[draw=drawColor,line width= 0.4pt,line join=round,line cap=round,fill=fillColor] (199.79,508.31) circle (  1.16);

\path[draw=drawColor,line width= 0.4pt,line join=round,line cap=round,fill=fillColor] (199.85,508.31) circle (  1.16);

\path[draw=drawColor,line width= 0.4pt,line join=round,line cap=round,fill=fillColor] (199.90,508.31) circle (  1.16);

\path[draw=drawColor,line width= 0.4pt,line join=round,line cap=round,fill=fillColor] (199.96,508.31) circle (  1.16);

\path[draw=drawColor,line width= 0.4pt,line join=round,line cap=round,fill=fillColor] (200.01,508.31) circle (  1.16);

\path[draw=drawColor,line width= 0.4pt,line join=round,line cap=round,fill=fillColor] (200.07,508.31) circle (  1.16);

\path[draw=drawColor,line width= 0.4pt,line join=round,line cap=round,fill=fillColor] (200.12,508.31) circle (  1.16);

\path[draw=drawColor,line width= 0.4pt,line join=round,line cap=round,fill=fillColor] (200.18,508.31) circle (  1.16);

\path[draw=drawColor,line width= 0.4pt,line join=round,line cap=round,fill=fillColor] (200.23,508.31) circle (  1.16);

\path[draw=drawColor,line width= 0.4pt,line join=round,line cap=round,fill=fillColor] (200.29,508.31) circle (  1.16);

\path[draw=drawColor,line width= 0.4pt,line join=round,line cap=round,fill=fillColor] (200.34,508.31) circle (  1.16);

\path[draw=drawColor,line width= 0.4pt,line join=round,line cap=round,fill=fillColor] (200.40,508.31) circle (  1.16);

\path[draw=drawColor,line width= 0.4pt,line join=round,line cap=round,fill=fillColor] (200.45,508.31) circle (  1.16);

\path[draw=drawColor,line width= 0.4pt,line join=round,line cap=round,fill=fillColor] (200.51,508.31) circle (  1.16);

\path[draw=drawColor,line width= 0.4pt,line join=round,line cap=round,fill=fillColor] (200.56,508.31) circle (  1.16);

\path[draw=drawColor,line width= 0.4pt,line join=round,line cap=round,fill=fillColor] (200.62,508.31) circle (  1.16);

\path[draw=drawColor,line width= 0.4pt,line join=round,line cap=round,fill=fillColor] (200.67,508.31) circle (  1.16);

\path[draw=drawColor,line width= 0.4pt,line join=round,line cap=round,fill=fillColor] (200.73,508.31) circle (  1.16);

\path[draw=drawColor,line width= 0.4pt,line join=round,line cap=round,fill=fillColor] (200.78,508.31) circle (  1.16);

\path[draw=drawColor,line width= 0.4pt,line join=round,line cap=round,fill=fillColor] (200.84,508.31) circle (  1.16);

\path[draw=drawColor,line width= 0.4pt,line join=round,line cap=round,fill=fillColor] (200.89,508.31) circle (  1.16);

\path[draw=drawColor,line width= 0.4pt,line join=round,line cap=round,fill=fillColor] (200.95,508.31) circle (  1.16);

\path[draw=drawColor,line width= 0.4pt,line join=round,line cap=round,fill=fillColor] (201.00,508.31) circle (  1.16);

\path[draw=drawColor,line width= 0.4pt,line join=round,line cap=round,fill=fillColor] (201.06,508.31) circle (  1.16);

\path[draw=drawColor,line width= 0.4pt,line join=round,line cap=round,fill=fillColor] (201.11,508.31) circle (  1.16);

\path[draw=drawColor,line width= 0.4pt,line join=round,line cap=round,fill=fillColor] (201.17,508.31) circle (  1.16);

\path[draw=drawColor,line width= 0.4pt,line join=round,line cap=round,fill=fillColor] (201.22,508.31) circle (  1.16);

\path[draw=drawColor,line width= 0.4pt,line join=round,line cap=round,fill=fillColor] (201.27,508.31) circle (  1.16);

\path[draw=drawColor,line width= 0.4pt,line join=round,line cap=round,fill=fillColor] (201.33,508.31) circle (  1.16);

\path[draw=drawColor,line width= 0.4pt,line join=round,line cap=round,fill=fillColor] (201.38,508.31) circle (  1.16);

\path[draw=drawColor,line width= 0.4pt,line join=round,line cap=round,fill=fillColor] (201.44,508.31) circle (  1.16);

\path[draw=drawColor,line width= 0.4pt,line join=round,line cap=round,fill=fillColor] (201.49,508.31) circle (  1.16);

\path[draw=drawColor,line width= 0.4pt,line join=round,line cap=round,fill=fillColor] (201.55,508.31) circle (  1.16);

\path[draw=drawColor,line width= 0.4pt,line join=round,line cap=round,fill=fillColor] (201.60,508.31) circle (  1.16);

\path[draw=drawColor,line width= 0.4pt,line join=round,line cap=round,fill=fillColor] (201.65,508.31) circle (  1.16);

\path[draw=drawColor,line width= 0.4pt,line join=round,line cap=round,fill=fillColor] (201.71,508.31) circle (  1.16);

\path[draw=drawColor,line width= 0.4pt,line join=round,line cap=round,fill=fillColor] (201.76,508.31) circle (  1.16);

\path[draw=drawColor,line width= 0.4pt,line join=round,line cap=round,fill=fillColor] (201.82,508.31) circle (  1.16);

\path[draw=drawColor,line width= 0.4pt,line join=round,line cap=round,fill=fillColor] (201.87,508.31) circle (  1.16);

\path[draw=drawColor,line width= 0.4pt,line join=round,line cap=round,fill=fillColor] (201.92,508.31) circle (  1.16);

\path[draw=drawColor,line width= 0.4pt,line join=round,line cap=round,fill=fillColor] (201.98,508.31) circle (  1.16);

\path[draw=drawColor,line width= 0.4pt,line join=round,line cap=round,fill=fillColor] (202.03,508.31) circle (  1.16);

\path[draw=drawColor,line width= 0.4pt,line join=round,line cap=round,fill=fillColor] (202.09,508.31) circle (  1.16);

\path[draw=drawColor,line width= 0.4pt,line join=round,line cap=round,fill=fillColor] (202.14,508.31) circle (  1.16);

\path[draw=drawColor,line width= 0.4pt,line join=round,line cap=round,fill=fillColor] (202.19,508.31) circle (  1.16);

\path[draw=drawColor,line width= 0.4pt,line join=round,line cap=round,fill=fillColor] (202.25,508.31) circle (  1.16);

\path[draw=drawColor,line width= 0.4pt,line join=round,line cap=round,fill=fillColor] (202.30,508.31) circle (  1.16);

\path[draw=drawColor,line width= 0.4pt,line join=round,line cap=round,fill=fillColor] (202.35,508.31) circle (  1.16);

\path[draw=drawColor,line width= 0.4pt,line join=round,line cap=round,fill=fillColor] (202.41,508.31) circle (  1.16);

\path[draw=drawColor,line width= 0.4pt,line join=round,line cap=round,fill=fillColor] (202.46,508.31) circle (  1.16);

\path[draw=drawColor,line width= 0.4pt,line join=round,line cap=round,fill=fillColor] (202.51,508.31) circle (  1.16);

\path[draw=drawColor,line width= 0.4pt,line join=round,line cap=round,fill=fillColor] (202.57,508.31) circle (  1.16);

\path[draw=drawColor,line width= 0.4pt,line join=round,line cap=round,fill=fillColor] (202.62,508.31) circle (  1.16);

\path[draw=drawColor,line width= 0.4pt,line join=round,line cap=round,fill=fillColor] (202.67,508.31) circle (  1.16);

\path[draw=drawColor,line width= 0.4pt,line join=round,line cap=round,fill=fillColor] (202.73,508.31) circle (  1.16);

\path[draw=drawColor,line width= 0.4pt,line join=round,line cap=round,fill=fillColor] (202.78,508.31) circle (  1.16);

\path[draw=drawColor,line width= 0.4pt,line join=round,line cap=round,fill=fillColor] (202.83,508.31) circle (  1.16);

\path[draw=drawColor,line width= 0.4pt,line join=round,line cap=round,fill=fillColor] (202.89,508.31) circle (  1.16);

\path[draw=drawColor,line width= 0.4pt,line join=round,line cap=round,fill=fillColor] (202.94,508.31) circle (  1.16);

\path[draw=drawColor,line width= 0.4pt,line join=round,line cap=round,fill=fillColor] (202.99,508.31) circle (  1.16);

\path[draw=drawColor,line width= 0.4pt,line join=round,line cap=round,fill=fillColor] (203.05,508.31) circle (  1.16);

\path[draw=drawColor,line width= 0.4pt,line join=round,line cap=round,fill=fillColor] (203.10,508.31) circle (  1.16);

\path[draw=drawColor,line width= 0.4pt,line join=round,line cap=round,fill=fillColor] (203.15,508.31) circle (  1.16);

\path[draw=drawColor,line width= 0.4pt,line join=round,line cap=round,fill=fillColor] (203.20,508.31) circle (  1.16);

\path[draw=drawColor,line width= 0.4pt,line join=round,line cap=round,fill=fillColor] (203.26,508.31) circle (  1.16);

\path[draw=drawColor,line width= 0.4pt,line join=round,line cap=round,fill=fillColor] (203.31,508.31) circle (  1.16);

\path[draw=drawColor,line width= 0.4pt,line join=round,line cap=round,fill=fillColor] (203.36,508.31) circle (  1.16);

\path[draw=drawColor,line width= 0.4pt,line join=round,line cap=round,fill=fillColor] (203.41,508.31) circle (  1.16);

\path[draw=drawColor,line width= 0.4pt,line join=round,line cap=round,fill=fillColor] (203.47,508.31) circle (  1.16);

\path[draw=drawColor,line width= 0.4pt,line join=round,line cap=round,fill=fillColor] (203.52,508.31) circle (  1.16);

\path[draw=drawColor,line width= 0.4pt,line join=round,line cap=round,fill=fillColor] (203.57,508.31) circle (  1.16);

\path[draw=drawColor,line width= 0.4pt,line join=round,line cap=round,fill=fillColor] (203.63,508.31) circle (  1.16);

\path[draw=drawColor,line width= 0.4pt,line join=round,line cap=round,fill=fillColor] (203.68,508.31) circle (  1.16);

\path[draw=drawColor,line width= 0.4pt,line join=round,line cap=round,fill=fillColor] (203.73,508.31) circle (  1.16);

\path[draw=drawColor,line width= 0.4pt,line join=round,line cap=round,fill=fillColor] (203.78,508.31) circle (  1.16);

\path[draw=drawColor,line width= 0.4pt,line join=round,line cap=round,fill=fillColor] (203.83,508.31) circle (  1.16);

\path[draw=drawColor,line width= 0.4pt,line join=round,line cap=round,fill=fillColor] (203.89,508.31) circle (  1.16);

\path[draw=drawColor,line width= 0.4pt,line join=round,line cap=round,fill=fillColor] (203.94,508.31) circle (  1.16);

\path[draw=drawColor,line width= 0.4pt,line join=round,line cap=round,fill=fillColor] (203.99,508.31) circle (  1.16);

\path[draw=drawColor,line width= 0.4pt,line join=round,line cap=round,fill=fillColor] (204.04,508.31) circle (  1.16);

\path[draw=drawColor,line width= 0.4pt,line join=round,line cap=round,fill=fillColor] (204.10,508.31) circle (  1.16);

\path[draw=drawColor,line width= 0.4pt,line join=round,line cap=round,fill=fillColor] (204.15,508.31) circle (  1.16);

\path[draw=drawColor,line width= 0.4pt,line join=round,line cap=round,fill=fillColor] (204.20,508.31) circle (  1.16);

\path[draw=drawColor,line width= 0.4pt,line join=round,line cap=round,fill=fillColor] (204.25,508.31) circle (  1.16);

\path[draw=drawColor,line width= 0.4pt,line join=round,line cap=round,fill=fillColor] (204.30,508.31) circle (  1.16);

\path[draw=drawColor,line width= 0.4pt,line join=round,line cap=round,fill=fillColor] (204.36,508.31) circle (  1.16);

\path[draw=drawColor,line width= 0.4pt,line join=round,line cap=round,fill=fillColor] (204.41,508.31) circle (  1.16);

\path[draw=drawColor,line width= 0.4pt,line join=round,line cap=round,fill=fillColor] (204.46,508.31) circle (  1.16);

\path[draw=drawColor,line width= 0.4pt,line join=round,line cap=round,fill=fillColor] (204.51,508.31) circle (  1.16);

\path[draw=drawColor,line width= 0.4pt,line join=round,line cap=round,fill=fillColor] (204.56,508.31) circle (  1.16);

\path[draw=drawColor,line width= 0.4pt,line join=round,line cap=round,fill=fillColor] (204.62,508.31) circle (  1.16);

\path[draw=drawColor,line width= 0.4pt,line join=round,line cap=round,fill=fillColor] (204.67,508.31) circle (  1.16);

\path[draw=drawColor,line width= 0.4pt,line join=round,line cap=round,fill=fillColor] (204.72,508.31) circle (  1.16);

\path[draw=drawColor,line width= 0.4pt,line join=round,line cap=round,fill=fillColor] (204.77,508.31) circle (  1.16);

\path[draw=drawColor,line width= 0.4pt,line join=round,line cap=round,fill=fillColor] (204.82,508.31) circle (  1.16);

\path[draw=drawColor,line width= 0.4pt,line join=round,line cap=round,fill=fillColor] (204.87,508.31) circle (  1.16);

\path[draw=drawColor,line width= 0.4pt,line join=round,line cap=round,fill=fillColor] (204.93,508.31) circle (  1.16);

\path[draw=drawColor,line width= 0.4pt,line join=round,line cap=round,fill=fillColor] (204.98,508.31) circle (  1.16);

\path[draw=drawColor,line width= 0.4pt,line join=round,line cap=round,fill=fillColor] (205.03,508.31) circle (  1.16);

\path[draw=drawColor,line width= 0.4pt,line join=round,line cap=round,fill=fillColor] (205.08,508.31) circle (  1.16);

\path[draw=drawColor,line width= 0.4pt,line join=round,line cap=round,fill=fillColor] (205.13,508.31) circle (  1.16);

\path[draw=drawColor,line width= 0.4pt,line join=round,line cap=round,fill=fillColor] (205.18,508.31) circle (  1.16);

\path[draw=drawColor,line width= 0.4pt,line join=round,line cap=round,fill=fillColor] (205.23,508.31) circle (  1.16);

\path[draw=drawColor,line width= 0.4pt,line join=round,line cap=round,fill=fillColor] (205.29,508.31) circle (  1.16);

\path[draw=drawColor,line width= 0.4pt,line join=round,line cap=round,fill=fillColor] (205.34,508.31) circle (  1.16);

\path[draw=drawColor,line width= 0.4pt,line join=round,line cap=round,fill=fillColor] (205.39,508.31) circle (  1.16);

\path[draw=drawColor,line width= 0.4pt,line join=round,line cap=round,fill=fillColor] (205.44,508.31) circle (  1.16);

\path[draw=drawColor,line width= 0.4pt,line join=round,line cap=round,fill=fillColor] (205.49,508.31) circle (  1.16);

\path[draw=drawColor,line width= 0.4pt,line join=round,line cap=round,fill=fillColor] (205.54,508.31) circle (  1.16);

\path[draw=drawColor,line width= 0.4pt,line join=round,line cap=round,fill=fillColor] (205.59,508.31) circle (  1.16);

\path[draw=drawColor,line width= 0.4pt,line join=round,line cap=round,fill=fillColor] (205.64,508.31) circle (  1.16);

\path[draw=drawColor,line width= 0.4pt,line join=round,line cap=round,fill=fillColor] (205.69,508.31) circle (  1.16);

\path[draw=drawColor,line width= 0.4pt,line join=round,line cap=round,fill=fillColor] (205.75,508.31) circle (  1.16);

\path[draw=drawColor,line width= 0.4pt,line join=round,line cap=round,fill=fillColor] (205.80,508.31) circle (  1.16);

\path[draw=drawColor,line width= 0.4pt,line join=round,line cap=round,fill=fillColor] (205.85,508.31) circle (  1.16);

\path[draw=drawColor,line width= 0.4pt,line join=round,line cap=round,fill=fillColor] (205.90,508.31) circle (  1.16);

\path[draw=drawColor,line width= 0.4pt,line join=round,line cap=round,fill=fillColor] (205.95,508.31) circle (  1.16);

\path[draw=drawColor,line width= 0.4pt,line join=round,line cap=round,fill=fillColor] (206.00,508.31) circle (  1.16);

\path[draw=drawColor,line width= 0.4pt,line join=round,line cap=round,fill=fillColor] (206.05,508.31) circle (  1.16);

\path[draw=drawColor,line width= 0.4pt,line join=round,line cap=round,fill=fillColor] (206.10,508.31) circle (  1.16);

\path[draw=drawColor,line width= 0.4pt,line join=round,line cap=round,fill=fillColor] (206.15,508.31) circle (  1.16);

\path[draw=drawColor,line width= 0.4pt,line join=round,line cap=round,fill=fillColor] (206.20,508.31) circle (  1.16);

\path[draw=drawColor,line width= 0.4pt,line join=round,line cap=round,fill=fillColor] (206.25,508.31) circle (  1.16);

\path[draw=drawColor,line width= 0.4pt,line join=round,line cap=round,fill=fillColor] (206.30,508.31) circle (  1.16);

\path[draw=drawColor,line width= 0.4pt,line join=round,line cap=round,fill=fillColor] (206.35,508.31) circle (  1.16);

\path[draw=drawColor,line width= 0.4pt,line join=round,line cap=round,fill=fillColor] (206.40,508.31) circle (  1.16);

\path[draw=drawColor,line width= 0.4pt,line join=round,line cap=round,fill=fillColor] (206.45,508.31) circle (  1.16);

\path[draw=drawColor,line width= 0.4pt,line join=round,line cap=round,fill=fillColor] (206.50,508.31) circle (  1.16);

\path[draw=drawColor,line width= 0.4pt,line join=round,line cap=round,fill=fillColor] (206.55,508.31) circle (  1.16);

\path[draw=drawColor,line width= 0.4pt,line join=round,line cap=round,fill=fillColor] (206.61,508.31) circle (  1.16);

\path[draw=drawColor,line width= 0.4pt,line join=round,line cap=round,fill=fillColor] (206.66,508.31) circle (  1.16);

\path[draw=drawColor,line width= 0.4pt,line join=round,line cap=round,fill=fillColor] (206.71,508.31) circle (  1.16);

\path[draw=drawColor,line width= 0.4pt,line join=round,line cap=round,fill=fillColor] (206.76,508.31) circle (  1.16);

\path[draw=drawColor,line width= 0.4pt,line join=round,line cap=round,fill=fillColor] (206.81,508.31) circle (  1.16);

\path[draw=drawColor,line width= 0.4pt,line join=round,line cap=round,fill=fillColor] (206.86,508.31) circle (  1.16);

\path[draw=drawColor,line width= 0.4pt,line join=round,line cap=round,fill=fillColor] (206.91,508.31) circle (  1.16);

\path[draw=drawColor,line width= 0.4pt,line join=round,line cap=round,fill=fillColor] (206.96,508.31) circle (  1.16);

\path[draw=drawColor,line width= 0.4pt,line join=round,line cap=round,fill=fillColor] (207.01,508.31) circle (  1.16);

\path[draw=drawColor,line width= 0.4pt,line join=round,line cap=round,fill=fillColor] (207.06,508.31) circle (  1.16);

\path[draw=drawColor,line width= 0.4pt,line join=round,line cap=round,fill=fillColor] (207.11,508.31) circle (  1.16);

\path[draw=drawColor,line width= 0.4pt,line join=round,line cap=round,fill=fillColor] (207.16,508.31) circle (  1.16);

\path[draw=drawColor,line width= 0.4pt,line join=round,line cap=round,fill=fillColor] (207.21,508.31) circle (  1.16);

\path[draw=drawColor,line width= 0.4pt,line join=round,line cap=round,fill=fillColor] (207.26,508.31) circle (  1.16);

\path[draw=drawColor,line width= 0.4pt,line join=round,line cap=round,fill=fillColor] (207.31,508.31) circle (  1.16);

\path[draw=drawColor,line width= 0.4pt,line join=round,line cap=round,fill=fillColor] (207.36,508.31) circle (  1.16);

\path[draw=drawColor,line width= 0.4pt,line join=round,line cap=round,fill=fillColor] (207.41,508.31) circle (  1.16);

\path[draw=drawColor,line width= 0.4pt,line join=round,line cap=round,fill=fillColor] (207.46,508.31) circle (  1.16);

\path[draw=drawColor,line width= 0.4pt,line join=round,line cap=round,fill=fillColor] (207.50,508.31) circle (  1.16);

\path[draw=drawColor,line width= 0.4pt,line join=round,line cap=round,fill=fillColor] (207.55,508.31) circle (  1.16);

\path[draw=drawColor,line width= 0.4pt,line join=round,line cap=round,fill=fillColor] (207.60,508.31) circle (  1.16);

\path[draw=drawColor,line width= 0.4pt,line join=round,line cap=round,fill=fillColor] (207.65,508.31) circle (  1.16);

\path[draw=drawColor,line width= 0.4pt,line join=round,line cap=round,fill=fillColor] (207.70,508.31) circle (  1.16);

\path[draw=drawColor,line width= 0.4pt,line join=round,line cap=round,fill=fillColor] (207.75,508.31) circle (  1.16);

\path[draw=drawColor,line width= 0.4pt,line join=round,line cap=round,fill=fillColor] (207.80,508.31) circle (  1.16);

\path[draw=drawColor,line width= 0.4pt,line join=round,line cap=round,fill=fillColor] (207.85,508.31) circle (  1.16);

\path[draw=drawColor,line width= 0.4pt,line join=round,line cap=round,fill=fillColor] (207.90,508.31) circle (  1.16);

\path[draw=drawColor,line width= 0.4pt,line join=round,line cap=round,fill=fillColor] (207.95,508.31) circle (  1.16);

\path[draw=drawColor,line width= 0.4pt,line join=round,line cap=round,fill=fillColor] (208.00,508.31) circle (  1.16);

\path[draw=drawColor,line width= 0.4pt,line join=round,line cap=round,fill=fillColor] (208.05,508.31) circle (  1.16);

\path[draw=drawColor,line width= 0.4pt,line join=round,line cap=round,fill=fillColor] (208.10,508.31) circle (  1.16);

\path[draw=drawColor,line width= 0.4pt,line join=round,line cap=round,fill=fillColor] (208.15,508.31) circle (  1.16);

\path[draw=drawColor,line width= 0.4pt,line join=round,line cap=round,fill=fillColor] (208.20,508.31) circle (  1.16);

\path[draw=drawColor,line width= 0.4pt,line join=round,line cap=round,fill=fillColor] (208.25,508.31) circle (  1.16);

\path[draw=drawColor,line width= 0.4pt,line join=round,line cap=round,fill=fillColor] (208.29,508.31) circle (  1.16);

\path[draw=drawColor,line width= 0.4pt,line join=round,line cap=round,fill=fillColor] (208.34,508.31) circle (  1.16);

\path[draw=drawColor,line width= 0.4pt,line join=round,line cap=round,fill=fillColor] (208.39,508.31) circle (  1.16);

\path[draw=drawColor,line width= 0.4pt,line join=round,line cap=round,fill=fillColor] (208.44,508.31) circle (  1.16);

\path[draw=drawColor,line width= 0.4pt,line join=round,line cap=round,fill=fillColor] (208.49,508.31) circle (  1.16);

\path[draw=drawColor,line width= 0.4pt,line join=round,line cap=round,fill=fillColor] (208.54,508.31) circle (  1.16);

\path[draw=drawColor,line width= 0.4pt,line join=round,line cap=round,fill=fillColor] (208.59,508.31) circle (  1.16);

\path[draw=drawColor,line width= 0.4pt,line join=round,line cap=round,fill=fillColor] (208.64,508.31) circle (  1.16);

\path[draw=drawColor,line width= 0.4pt,line join=round,line cap=round,fill=fillColor] (208.69,508.31) circle (  1.16);

\path[draw=drawColor,line width= 0.4pt,line join=round,line cap=round,fill=fillColor] (208.74,508.31) circle (  1.16);

\path[draw=drawColor,line width= 0.4pt,line join=round,line cap=round,fill=fillColor] (208.78,508.31) circle (  1.16);

\path[draw=drawColor,line width= 0.4pt,line join=round,line cap=round,fill=fillColor] (208.83,508.31) circle (  1.16);

\path[draw=drawColor,line width= 0.4pt,line join=round,line cap=round,fill=fillColor] (208.88,508.31) circle (  1.16);

\path[draw=drawColor,line width= 0.4pt,line join=round,line cap=round,fill=fillColor] (208.93,508.31) circle (  1.16);

\path[draw=drawColor,line width= 0.4pt,line join=round,line cap=round,fill=fillColor] (208.98,508.31) circle (  1.16);

\path[draw=drawColor,line width= 0.4pt,line join=round,line cap=round,fill=fillColor] (209.03,508.31) circle (  1.16);

\path[draw=drawColor,line width= 0.4pt,line join=round,line cap=round,fill=fillColor] (209.08,508.31) circle (  1.16);

\path[draw=drawColor,line width= 0.4pt,line join=round,line cap=round,fill=fillColor] (209.12,508.31) circle (  1.16);

\path[draw=drawColor,line width= 0.4pt,line join=round,line cap=round,fill=fillColor] (209.17,508.31) circle (  1.16);

\path[draw=drawColor,line width= 0.4pt,line join=round,line cap=round,fill=fillColor] (209.22,508.31) circle (  1.16);

\path[draw=drawColor,line width= 0.4pt,line join=round,line cap=round,fill=fillColor] (209.27,508.31) circle (  1.16);

\path[draw=drawColor,line width= 0.4pt,line join=round,line cap=round,fill=fillColor] (209.32,508.31) circle (  1.16);

\path[draw=drawColor,line width= 0.4pt,line join=round,line cap=round,fill=fillColor] (209.37,508.31) circle (  1.16);

\path[draw=drawColor,line width= 0.4pt,line join=round,line cap=round,fill=fillColor] (209.42,508.31) circle (  1.16);

\path[draw=drawColor,line width= 0.4pt,line join=round,line cap=round,fill=fillColor] (209.46,508.31) circle (  1.16);

\path[draw=drawColor,line width= 0.4pt,line join=round,line cap=round,fill=fillColor] (209.51,508.31) circle (  1.16);

\path[draw=drawColor,line width= 0.4pt,line join=round,line cap=round,fill=fillColor] (209.56,508.31) circle (  1.16);

\path[draw=drawColor,line width= 0.4pt,line join=round,line cap=round,fill=fillColor] (209.61,508.31) circle (  1.16);

\path[draw=drawColor,line width= 0.4pt,line join=round,line cap=round,fill=fillColor] (209.66,508.31) circle (  1.16);

\path[draw=drawColor,line width= 0.4pt,line join=round,line cap=round,fill=fillColor] (209.70,508.31) circle (  1.16);

\path[draw=drawColor,line width= 0.4pt,line join=round,line cap=round,fill=fillColor] (209.75,508.31) circle (  1.16);

\path[draw=drawColor,line width= 0.4pt,line join=round,line cap=round,fill=fillColor] (209.80,508.31) circle (  1.16);

\path[draw=drawColor,line width= 0.4pt,line join=round,line cap=round,fill=fillColor] (209.85,508.31) circle (  1.16);

\path[draw=drawColor,line width= 0.4pt,line join=round,line cap=round,fill=fillColor] (209.90,508.31) circle (  1.16);

\path[draw=drawColor,line width= 0.4pt,line join=round,line cap=round,fill=fillColor] (209.94,508.31) circle (  1.16);

\path[draw=drawColor,line width= 0.4pt,line join=round,line cap=round,fill=fillColor] (209.99,508.31) circle (  1.16);

\path[draw=drawColor,line width= 0.4pt,line join=round,line cap=round,fill=fillColor] (210.04,508.31) circle (  1.16);

\path[draw=drawColor,line width= 0.4pt,line join=round,line cap=round,fill=fillColor] (210.09,508.31) circle (  1.16);

\path[draw=drawColor,line width= 0.4pt,line join=round,line cap=round,fill=fillColor] (210.14,508.31) circle (  1.16);

\path[draw=drawColor,line width= 0.4pt,line join=round,line cap=round,fill=fillColor] (210.18,508.31) circle (  1.16);

\path[draw=drawColor,line width= 0.4pt,line join=round,line cap=round,fill=fillColor] (210.23,508.31) circle (  1.16);

\path[draw=drawColor,line width= 0.4pt,line join=round,line cap=round,fill=fillColor] (210.28,508.31) circle (  1.16);

\path[draw=drawColor,line width= 0.4pt,line join=round,line cap=round,fill=fillColor] (210.33,508.31) circle (  1.16);

\path[draw=drawColor,line width= 0.4pt,line join=round,line cap=round,fill=fillColor] (210.38,508.31) circle (  1.16);

\path[draw=drawColor,line width= 0.4pt,line join=round,line cap=round,fill=fillColor] (210.42,508.31) circle (  1.16);

\path[draw=drawColor,line width= 0.4pt,line join=round,line cap=round,fill=fillColor] (210.47,508.31) circle (  1.16);

\path[draw=drawColor,line width= 0.4pt,line join=round,line cap=round,fill=fillColor] (210.52,508.31) circle (  1.16);

\path[draw=drawColor,line width= 0.4pt,line join=round,line cap=round,fill=fillColor] (210.57,508.31) circle (  1.16);

\path[draw=drawColor,line width= 0.4pt,line join=round,line cap=round,fill=fillColor] (210.61,508.31) circle (  1.16);

\path[draw=drawColor,line width= 0.4pt,line join=round,line cap=round,fill=fillColor] (210.66,508.31) circle (  1.16);

\path[draw=drawColor,line width= 0.4pt,line join=round,line cap=round,fill=fillColor] (210.71,508.31) circle (  1.16);

\path[draw=drawColor,line width= 0.4pt,line join=round,line cap=round,fill=fillColor] (210.76,508.31) circle (  1.16);

\path[draw=drawColor,line width= 0.4pt,line join=round,line cap=round,fill=fillColor] (210.80,508.31) circle (  1.16);

\path[draw=drawColor,line width= 0.4pt,line join=round,line cap=round,fill=fillColor] (210.85,508.31) circle (  1.16);

\path[draw=drawColor,line width= 0.4pt,line join=round,line cap=round,fill=fillColor] (210.90,508.31) circle (  1.16);

\path[draw=drawColor,line width= 0.4pt,line join=round,line cap=round,fill=fillColor] (210.95,508.31) circle (  1.16);

\path[draw=drawColor,line width= 0.4pt,line join=round,line cap=round,fill=fillColor] (210.99,508.31) circle (  1.16);

\path[draw=drawColor,line width= 0.4pt,line join=round,line cap=round,fill=fillColor] (211.04,508.31) circle (  1.16);

\path[draw=drawColor,line width= 0.4pt,line join=round,line cap=round,fill=fillColor] (211.09,508.31) circle (  1.16);

\path[draw=drawColor,line width= 0.4pt,line join=round,line cap=round,fill=fillColor] (211.13,508.31) circle (  1.16);

\path[draw=drawColor,line width= 0.4pt,line join=round,line cap=round,fill=fillColor] (211.18,508.31) circle (  1.16);

\path[draw=drawColor,line width= 0.4pt,line join=round,line cap=round,fill=fillColor] (211.23,508.31) circle (  1.16);

\path[draw=drawColor,line width= 0.4pt,line join=round,line cap=round,fill=fillColor] (211.28,508.31) circle (  1.16);

\path[draw=drawColor,line width= 0.4pt,line join=round,line cap=round,fill=fillColor] (211.32,508.31) circle (  1.16);

\path[draw=drawColor,line width= 0.4pt,line join=round,line cap=round,fill=fillColor] (211.37,508.31) circle (  1.16);

\path[draw=drawColor,line width= 0.4pt,line join=round,line cap=round,fill=fillColor] (211.42,508.31) circle (  1.16);

\path[draw=drawColor,line width= 0.4pt,line join=round,line cap=round,fill=fillColor] (211.46,508.31) circle (  1.16);

\path[draw=drawColor,line width= 0.4pt,line join=round,line cap=round,fill=fillColor] (211.51,508.31) circle (  1.16);

\path[draw=drawColor,line width= 0.4pt,line join=round,line cap=round,fill=fillColor] (211.56,508.31) circle (  1.16);

\path[draw=drawColor,line width= 0.4pt,line join=round,line cap=round,fill=fillColor] (211.60,508.31) circle (  1.16);

\path[draw=drawColor,line width= 0.4pt,line join=round,line cap=round,fill=fillColor] (211.65,508.31) circle (  1.16);

\path[draw=drawColor,line width= 0.4pt,line join=round,line cap=round,fill=fillColor] (211.70,508.31) circle (  1.16);

\path[draw=drawColor,line width= 0.4pt,line join=round,line cap=round,fill=fillColor] (211.75,508.31) circle (  1.16);

\path[draw=drawColor,line width= 0.4pt,line join=round,line cap=round,fill=fillColor] (211.79,508.31) circle (  1.16);

\path[draw=drawColor,line width= 0.4pt,line join=round,line cap=round,fill=fillColor] (211.84,508.31) circle (  1.16);

\path[draw=drawColor,line width= 0.4pt,line join=round,line cap=round,fill=fillColor] (211.89,508.31) circle (  1.16);

\path[draw=drawColor,line width= 0.4pt,line join=round,line cap=round,fill=fillColor] (211.93,508.31) circle (  1.16);

\path[draw=drawColor,line width= 0.4pt,line join=round,line cap=round,fill=fillColor] (211.98,508.31) circle (  1.16);

\path[draw=drawColor,line width= 0.4pt,line join=round,line cap=round,fill=fillColor] (212.03,508.31) circle (  1.16);

\path[draw=drawColor,line width= 0.4pt,line join=round,line cap=round,fill=fillColor] (212.07,508.31) circle (  1.16);

\path[draw=drawColor,line width= 0.4pt,line join=round,line cap=round,fill=fillColor] (212.12,508.31) circle (  1.16);

\path[draw=drawColor,line width= 0.4pt,line join=round,line cap=round,fill=fillColor] (212.16,508.31) circle (  1.16);

\path[draw=drawColor,line width= 0.4pt,line join=round,line cap=round,fill=fillColor] (212.21,508.31) circle (  1.16);

\path[draw=drawColor,line width= 0.4pt,line join=round,line cap=round,fill=fillColor] (212.26,508.31) circle (  1.16);

\path[draw=drawColor,line width= 0.4pt,line join=round,line cap=round,fill=fillColor] (212.30,508.31) circle (  1.16);

\path[draw=drawColor,line width= 0.4pt,line join=round,line cap=round,fill=fillColor] (212.35,508.31) circle (  1.16);

\path[draw=drawColor,line width= 0.4pt,line join=round,line cap=round,fill=fillColor] (212.40,508.31) circle (  1.16);

\path[draw=drawColor,line width= 0.4pt,line join=round,line cap=round,fill=fillColor] (212.44,508.31) circle (  1.16);

\path[draw=drawColor,line width= 0.4pt,line join=round,line cap=round,fill=fillColor] (212.49,508.31) circle (  1.16);

\path[draw=drawColor,line width= 0.4pt,line join=round,line cap=round,fill=fillColor] (212.54,508.31) circle (  1.16);

\path[draw=drawColor,line width= 0.4pt,line join=round,line cap=round,fill=fillColor] (212.58,508.31) circle (  1.16);

\path[draw=drawColor,line width= 0.4pt,line join=round,line cap=round,fill=fillColor] (212.63,508.31) circle (  1.16);

\path[draw=drawColor,line width= 0.4pt,line join=round,line cap=round,fill=fillColor] (212.67,508.31) circle (  1.16);

\path[draw=drawColor,line width= 0.4pt,line join=round,line cap=round,fill=fillColor] (212.72,508.31) circle (  1.16);

\path[draw=drawColor,line width= 0.4pt,line join=round,line cap=round,fill=fillColor] (212.77,508.31) circle (  1.16);

\path[draw=drawColor,line width= 0.4pt,line join=round,line cap=round,fill=fillColor] (212.81,508.31) circle (  1.16);

\path[draw=drawColor,line width= 0.4pt,line join=round,line cap=round,fill=fillColor] (212.86,508.31) circle (  1.16);

\path[draw=drawColor,line width= 0.4pt,line join=round,line cap=round,fill=fillColor] (212.91,508.31) circle (  1.16);

\path[draw=drawColor,line width= 0.4pt,line join=round,line cap=round,fill=fillColor] (212.95,508.31) circle (  1.16);

\path[draw=drawColor,line width= 0.4pt,line join=round,line cap=round,fill=fillColor] (213.00,508.31) circle (  1.16);

\path[draw=drawColor,line width= 0.4pt,line join=round,line cap=round,fill=fillColor] (213.04,508.31) circle (  1.16);

\path[draw=drawColor,line width= 0.4pt,line join=round,line cap=round,fill=fillColor] (213.09,508.31) circle (  1.16);

\path[draw=drawColor,line width= 0.4pt,line join=round,line cap=round,fill=fillColor] (213.14,508.31) circle (  1.16);

\path[draw=drawColor,line width= 0.4pt,line join=round,line cap=round,fill=fillColor] (213.18,508.31) circle (  1.16);

\path[draw=drawColor,line width= 0.4pt,line join=round,line cap=round,fill=fillColor] (213.23,508.31) circle (  1.16);

\path[draw=drawColor,line width= 0.4pt,line join=round,line cap=round,fill=fillColor] (213.27,508.31) circle (  1.16);

\path[draw=drawColor,line width= 0.4pt,line join=round,line cap=round,fill=fillColor] (213.32,508.31) circle (  1.16);

\path[draw=drawColor,line width= 0.4pt,line join=round,line cap=round,fill=fillColor] (213.36,508.31) circle (  1.16);

\path[draw=drawColor,line width= 0.4pt,line join=round,line cap=round,fill=fillColor] (213.41,508.31) circle (  1.16);

\path[draw=drawColor,line width= 0.4pt,line join=round,line cap=round,fill=fillColor] (213.46,508.31) circle (  1.16);

\path[draw=drawColor,line width= 0.4pt,line join=round,line cap=round,fill=fillColor] (213.50,508.31) circle (  1.16);

\path[draw=drawColor,line width= 0.4pt,line join=round,line cap=round,fill=fillColor] (213.55,508.31) circle (  1.16);

\path[draw=drawColor,line width= 0.4pt,line join=round,line cap=round,fill=fillColor] (213.59,508.31) circle (  1.16);

\path[draw=drawColor,line width= 0.4pt,line join=round,line cap=round,fill=fillColor] (213.64,508.31) circle (  1.16);

\path[draw=drawColor,line width= 0.4pt,line join=round,line cap=round,fill=fillColor] (213.68,508.31) circle (  1.16);

\path[draw=drawColor,line width= 0.4pt,line join=round,line cap=round,fill=fillColor] (213.73,508.31) circle (  1.16);

\path[draw=drawColor,line width= 0.4pt,line join=round,line cap=round,fill=fillColor] (213.78,508.31) circle (  1.16);

\path[draw=drawColor,line width= 0.4pt,line join=round,line cap=round,fill=fillColor] (213.82,508.31) circle (  1.16);

\path[draw=drawColor,line width= 0.4pt,line join=round,line cap=round,fill=fillColor] (213.87,508.31) circle (  1.16);

\path[draw=drawColor,line width= 0.4pt,line join=round,line cap=round,fill=fillColor] (213.91,508.31) circle (  1.16);

\path[draw=drawColor,line width= 0.4pt,line join=round,line cap=round,fill=fillColor] (213.96,508.31) circle (  1.16);

\path[draw=drawColor,line width= 0.4pt,line join=round,line cap=round,fill=fillColor] (214.00,508.31) circle (  1.16);

\path[draw=drawColor,line width= 0.4pt,line join=round,line cap=round,fill=fillColor] (214.05,508.31) circle (  1.16);

\path[draw=drawColor,line width= 0.4pt,line join=round,line cap=round,fill=fillColor] (214.09,508.31) circle (  1.16);

\path[draw=drawColor,line width= 0.4pt,line join=round,line cap=round,fill=fillColor] (214.14,508.31) circle (  1.16);

\path[draw=drawColor,line width= 0.4pt,line join=round,line cap=round,fill=fillColor] (214.18,508.31) circle (  1.16);

\path[draw=drawColor,line width= 0.4pt,line join=round,line cap=round,fill=fillColor] (214.23,508.31) circle (  1.16);

\path[draw=drawColor,line width= 0.4pt,line join=round,line cap=round,fill=fillColor] (214.27,508.31) circle (  1.16);

\path[draw=drawColor,line width= 0.4pt,line join=round,line cap=round,fill=fillColor] (214.32,508.31) circle (  1.16);

\path[draw=drawColor,line width= 0.4pt,line join=round,line cap=round,fill=fillColor] (214.36,508.31) circle (  1.16);

\path[draw=drawColor,line width= 0.4pt,line join=round,line cap=round,fill=fillColor] (214.41,508.31) circle (  1.16);

\path[draw=drawColor,line width= 0.4pt,line join=round,line cap=round,fill=fillColor] (214.45,508.31) circle (  1.16);

\path[draw=drawColor,line width= 0.4pt,line join=round,line cap=round,fill=fillColor] (214.50,508.31) circle (  1.16);

\path[draw=drawColor,line width= 0.4pt,line join=round,line cap=round,fill=fillColor] (214.54,508.31) circle (  1.16);

\path[draw=drawColor,line width= 0.4pt,line join=round,line cap=round,fill=fillColor] (214.59,508.31) circle (  1.16);

\path[draw=drawColor,line width= 0.4pt,line join=round,line cap=round,fill=fillColor] (214.63,508.31) circle (  1.16);

\path[draw=drawColor,line width= 0.4pt,line join=round,line cap=round,fill=fillColor] (214.68,508.31) circle (  1.16);

\path[draw=drawColor,line width= 0.4pt,line join=round,line cap=round,fill=fillColor] (214.72,508.31) circle (  1.16);

\path[draw=drawColor,line width= 0.4pt,line join=round,line cap=round,fill=fillColor] (214.77,508.31) circle (  1.16);

\path[draw=drawColor,line width= 0.4pt,line join=round,line cap=round,fill=fillColor] (214.81,508.31) circle (  1.16);

\path[draw=drawColor,line width= 0.4pt,line join=round,line cap=round,fill=fillColor] (214.86,508.31) circle (  1.16);

\path[draw=drawColor,line width= 0.4pt,line join=round,line cap=round,fill=fillColor] (214.90,508.31) circle (  1.16);

\path[draw=drawColor,line width= 0.4pt,line join=round,line cap=round,fill=fillColor] (214.95,508.31) circle (  1.16);

\path[draw=drawColor,line width= 0.4pt,line join=round,line cap=round,fill=fillColor] (214.99,508.31) circle (  1.16);

\path[draw=drawColor,line width= 0.4pt,line join=round,line cap=round,fill=fillColor] (215.04,508.31) circle (  1.16);

\path[draw=drawColor,line width= 0.4pt,line join=round,line cap=round,fill=fillColor] (215.08,508.31) circle (  1.16);

\path[draw=drawColor,line width= 0.4pt,line join=round,line cap=round,fill=fillColor] (215.13,508.31) circle (  1.16);

\path[draw=drawColor,line width= 0.4pt,line join=round,line cap=round,fill=fillColor] (215.17,508.31) circle (  1.16);

\path[draw=drawColor,line width= 0.4pt,line join=round,line cap=round,fill=fillColor] (215.22,508.31) circle (  1.16);

\path[draw=drawColor,line width= 0.4pt,line join=round,line cap=round,fill=fillColor] (215.26,508.31) circle (  1.16);

\path[draw=drawColor,line width= 0.4pt,line join=round,line cap=round,fill=fillColor] (215.31,508.31) circle (  1.16);

\path[draw=drawColor,line width= 0.4pt,line join=round,line cap=round,fill=fillColor] (215.35,508.31) circle (  1.16);

\path[draw=drawColor,line width= 0.4pt,line join=round,line cap=round,fill=fillColor] (215.40,508.31) circle (  1.16);

\path[draw=drawColor,line width= 0.4pt,line join=round,line cap=round,fill=fillColor] (215.44,508.31) circle (  1.16);

\path[draw=drawColor,line width= 0.4pt,line join=round,line cap=round,fill=fillColor] (215.48,508.31) circle (  1.16);

\path[draw=drawColor,line width= 0.4pt,line join=round,line cap=round,fill=fillColor] (215.53,508.31) circle (  1.16);

\path[draw=drawColor,line width= 0.4pt,line join=round,line cap=round,fill=fillColor] (215.57,508.31) circle (  1.16);

\path[draw=drawColor,line width= 0.4pt,line join=round,line cap=round,fill=fillColor] (215.62,508.31) circle (  1.16);

\path[draw=drawColor,line width= 0.4pt,line join=round,line cap=round,fill=fillColor] (215.66,508.31) circle (  1.16);

\path[draw=drawColor,line width= 0.4pt,line join=round,line cap=round,fill=fillColor] (215.71,508.31) circle (  1.16);

\path[draw=drawColor,line width= 0.4pt,line join=round,line cap=round,fill=fillColor] (215.75,508.31) circle (  1.16);

\path[draw=drawColor,line width= 0.4pt,line join=round,line cap=round,fill=fillColor] (215.79,508.31) circle (  1.16);

\path[draw=drawColor,line width= 0.4pt,line join=round,line cap=round,fill=fillColor] (215.84,508.31) circle (  1.16);

\path[draw=drawColor,line width= 0.4pt,line join=round,line cap=round,fill=fillColor] (215.88,508.31) circle (  1.16);

\path[draw=drawColor,line width= 0.4pt,line join=round,line cap=round,fill=fillColor] (215.93,508.31) circle (  1.16);

\path[draw=drawColor,line width= 0.4pt,line join=round,line cap=round,fill=fillColor] (215.97,508.31) circle (  1.16);

\path[draw=drawColor,line width= 0.4pt,line join=round,line cap=round,fill=fillColor] (216.02,508.31) circle (  1.16);

\path[draw=drawColor,line width= 0.4pt,line join=round,line cap=round,fill=fillColor] (216.06,508.31) circle (  1.16);

\path[draw=drawColor,line width= 0.4pt,line join=round,line cap=round,fill=fillColor] (216.10,508.31) circle (  1.16);

\path[draw=drawColor,line width= 0.4pt,line join=round,line cap=round,fill=fillColor] (216.15,508.31) circle (  1.16);

\path[draw=drawColor,line width= 0.4pt,line join=round,line cap=round,fill=fillColor] (216.19,508.31) circle (  1.16);

\path[draw=drawColor,line width= 0.4pt,line join=round,line cap=round,fill=fillColor] (216.24,508.31) circle (  1.16);

\path[draw=drawColor,line width= 0.4pt,line join=round,line cap=round,fill=fillColor] (216.28,508.31) circle (  1.16);

\path[draw=drawColor,line width= 0.4pt,line join=round,line cap=round,fill=fillColor] (216.32,508.31) circle (  1.16);

\path[draw=drawColor,line width= 0.4pt,line join=round,line cap=round,fill=fillColor] (216.37,508.31) circle (  1.16);

\path[draw=drawColor,line width= 0.4pt,line join=round,line cap=round,fill=fillColor] (216.41,508.31) circle (  1.16);

\path[draw=drawColor,line width= 0.4pt,line join=round,line cap=round,fill=fillColor] (216.46,508.31) circle (  1.16);

\path[draw=drawColor,line width= 0.4pt,line join=round,line cap=round,fill=fillColor] (216.50,508.31) circle (  1.16);

\path[draw=drawColor,line width= 0.4pt,line join=round,line cap=round,fill=fillColor] (216.54,508.31) circle (  1.16);

\path[draw=drawColor,line width= 0.4pt,line join=round,line cap=round,fill=fillColor] (216.59,508.31) circle (  1.16);

\path[draw=drawColor,line width= 0.4pt,line join=round,line cap=round,fill=fillColor] (216.63,508.31) circle (  1.16);

\path[draw=drawColor,line width= 0.4pt,line join=round,line cap=round,fill=fillColor] (216.68,508.31) circle (  1.16);

\path[draw=drawColor,line width= 0.4pt,line join=round,line cap=round,fill=fillColor] (216.72,508.31) circle (  1.16);

\path[draw=drawColor,line width= 0.4pt,line join=round,line cap=round,fill=fillColor] (216.76,508.31) circle (  1.16);

\path[draw=drawColor,line width= 0.4pt,line join=round,line cap=round,fill=fillColor] (216.81,508.31) circle (  1.16);

\path[draw=drawColor,line width= 0.4pt,line join=round,line cap=round,fill=fillColor] (216.85,508.31) circle (  1.16);
\definecolor[named]{drawColor}{rgb}{0.22,0.49,0.72}
\definecolor[named]{fillColor}{rgb}{0.22,0.49,0.72}

\path[draw=drawColor,line width= 0.4pt,line join=round,line cap=round,fill=fillColor] ( 81.22,563.62) circle (  1.16);

\path[draw=drawColor,line width= 0.4pt,line join=round,line cap=round,fill=fillColor] ( 84.95,546.41) circle (  1.16);

\path[draw=drawColor,line width= 0.4pt,line join=round,line cap=round,fill=fillColor] ( 87.56,535.89) circle (  1.16);

\path[draw=drawColor,line width= 0.4pt,line join=round,line cap=round,fill=fillColor] ( 89.64,535.41) circle (  1.16);

\path[draw=drawColor,line width= 0.4pt,line join=round,line cap=round,fill=fillColor] ( 91.40,534.82) circle (  1.16);

\path[draw=drawColor,line width= 0.4pt,line join=round,line cap=round,fill=fillColor] ( 92.94,533.00) circle (  1.16);

\path[draw=drawColor,line width= 0.4pt,line join=round,line cap=round,fill=fillColor] ( 94.31,532.69) circle (  1.16);

\path[draw=drawColor,line width= 0.4pt,line join=round,line cap=round,fill=fillColor] ( 95.56,532.44) circle (  1.16);

\path[draw=drawColor,line width= 0.4pt,line join=round,line cap=round,fill=fillColor] ( 96.71,532.09) circle (  1.16);

\path[draw=drawColor,line width= 0.4pt,line join=round,line cap=round,fill=fillColor] ( 97.77,531.95) circle (  1.16);

\path[draw=drawColor,line width= 0.4pt,line join=round,line cap=round,fill=fillColor] ( 98.77,531.82) circle (  1.16);

\path[draw=drawColor,line width= 0.4pt,line join=round,line cap=round,fill=fillColor] ( 99.71,531.54) circle (  1.16);

\path[draw=drawColor,line width= 0.4pt,line join=round,line cap=round,fill=fillColor] (100.59,531.53) circle (  1.16);

\path[draw=drawColor,line width= 0.4pt,line join=round,line cap=round,fill=fillColor] (101.44,531.36) circle (  1.16);

\path[draw=drawColor,line width= 0.4pt,line join=round,line cap=round,fill=fillColor] (102.24,531.32) circle (  1.16);

\path[draw=drawColor,line width= 0.4pt,line join=round,line cap=round,fill=fillColor] (103.01,531.24) circle (  1.16);

\path[draw=drawColor,line width= 0.4pt,line join=round,line cap=round,fill=fillColor] (103.75,531.14) circle (  1.16);

\path[draw=drawColor,line width= 0.4pt,line join=round,line cap=round,fill=fillColor] (104.46,531.02) circle (  1.16);

\path[draw=drawColor,line width= 0.4pt,line join=round,line cap=round,fill=fillColor] (105.14,530.21) circle (  1.16);

\path[draw=drawColor,line width= 0.4pt,line join=round,line cap=round,fill=fillColor] (105.80,530.19) circle (  1.16);

\path[draw=drawColor,line width= 0.4pt,line join=round,line cap=round,fill=fillColor] (106.44,530.06) circle (  1.16);

\path[draw=drawColor,line width= 0.4pt,line join=round,line cap=round,fill=fillColor] (107.05,530.05) circle (  1.16);

\path[draw=drawColor,line width= 0.4pt,line join=round,line cap=round,fill=fillColor] (107.65,529.46) circle (  1.16);

\path[draw=drawColor,line width= 0.4pt,line join=round,line cap=round,fill=fillColor] (108.24,529.39) circle (  1.16);

\path[draw=drawColor,line width= 0.4pt,line join=round,line cap=round,fill=fillColor] (108.80,529.38) circle (  1.16);

\path[draw=drawColor,line width= 0.4pt,line join=round,line cap=round,fill=fillColor] (109.35,529.37) circle (  1.16);

\path[draw=drawColor,line width= 0.4pt,line join=round,line cap=round,fill=fillColor] (109.89,529.27) circle (  1.16);

\path[draw=drawColor,line width= 0.4pt,line join=round,line cap=round,fill=fillColor] (110.42,529.22) circle (  1.16);

\path[draw=drawColor,line width= 0.4pt,line join=round,line cap=round,fill=fillColor] (110.93,529.19) circle (  1.16);

\path[draw=drawColor,line width= 0.4pt,line join=round,line cap=round,fill=fillColor] (111.43,529.17) circle (  1.16);

\path[draw=drawColor,line width= 0.4pt,line join=round,line cap=round,fill=fillColor] (111.92,529.08) circle (  1.16);

\path[draw=drawColor,line width= 0.4pt,line join=round,line cap=round,fill=fillColor] (112.40,529.01) circle (  1.16);

\path[draw=drawColor,line width= 0.4pt,line join=round,line cap=round,fill=fillColor] (112.87,528.69) circle (  1.16);

\path[draw=drawColor,line width= 0.4pt,line join=round,line cap=round,fill=fillColor] (113.33,528.67) circle (  1.16);

\path[draw=drawColor,line width= 0.4pt,line join=round,line cap=round,fill=fillColor] (113.78,528.36) circle (  1.16);

\path[draw=drawColor,line width= 0.4pt,line join=round,line cap=round,fill=fillColor] (114.22,528.25) circle (  1.16);

\path[draw=drawColor,line width= 0.4pt,line join=round,line cap=round,fill=fillColor] (114.65,528.22) circle (  1.16);

\path[draw=drawColor,line width= 0.4pt,line join=round,line cap=round,fill=fillColor] (115.08,528.12) circle (  1.16);

\path[draw=drawColor,line width= 0.4pt,line join=round,line cap=round,fill=fillColor] (115.50,528.09) circle (  1.16);

\path[draw=drawColor,line width= 0.4pt,line join=round,line cap=round,fill=fillColor] (115.91,528.09) circle (  1.16);

\path[draw=drawColor,line width= 0.4pt,line join=round,line cap=round,fill=fillColor] (116.32,527.96) circle (  1.16);

\path[draw=drawColor,line width= 0.4pt,line join=round,line cap=round,fill=fillColor] (116.72,527.95) circle (  1.16);

\path[draw=drawColor,line width= 0.4pt,line join=round,line cap=round,fill=fillColor] (117.11,527.93) circle (  1.16);

\path[draw=drawColor,line width= 0.4pt,line join=round,line cap=round,fill=fillColor] (117.49,527.73) circle (  1.16);

\path[draw=drawColor,line width= 0.4pt,line join=round,line cap=round,fill=fillColor] (117.87,527.63) circle (  1.16);

\path[draw=drawColor,line width= 0.4pt,line join=round,line cap=round,fill=fillColor] (118.25,527.47) circle (  1.16);

\path[draw=drawColor,line width= 0.4pt,line join=round,line cap=round,fill=fillColor] (118.62,527.46) circle (  1.16);

\path[draw=drawColor,line width= 0.4pt,line join=round,line cap=round,fill=fillColor] (118.98,527.45) circle (  1.16);

\path[draw=drawColor,line width= 0.4pt,line join=round,line cap=round,fill=fillColor] (119.34,527.34) circle (  1.16);

\path[draw=drawColor,line width= 0.4pt,line join=round,line cap=round,fill=fillColor] (119.70,527.26) circle (  1.16);

\path[draw=drawColor,line width= 0.4pt,line join=round,line cap=round,fill=fillColor] (120.05,527.23) circle (  1.16);

\path[draw=drawColor,line width= 0.4pt,line join=round,line cap=round,fill=fillColor] (120.39,527.22) circle (  1.16);

\path[draw=drawColor,line width= 0.4pt,line join=round,line cap=round,fill=fillColor] (120.73,527.21) circle (  1.16);

\path[draw=drawColor,line width= 0.4pt,line join=round,line cap=round,fill=fillColor] (121.07,527.13) circle (  1.16);

\path[draw=drawColor,line width= 0.4pt,line join=round,line cap=round,fill=fillColor] (121.40,527.10) circle (  1.16);

\path[draw=drawColor,line width= 0.4pt,line join=round,line cap=round,fill=fillColor] (121.73,526.99) circle (  1.16);

\path[draw=drawColor,line width= 0.4pt,line join=round,line cap=round,fill=fillColor] (122.05,526.96) circle (  1.16);

\path[draw=drawColor,line width= 0.4pt,line join=round,line cap=round,fill=fillColor] (122.38,526.88) circle (  1.16);

\path[draw=drawColor,line width= 0.4pt,line join=round,line cap=round,fill=fillColor] (122.69,526.73) circle (  1.16);

\path[draw=drawColor,line width= 0.4pt,line join=round,line cap=round,fill=fillColor] (123.01,526.66) circle (  1.16);

\path[draw=drawColor,line width= 0.4pt,line join=round,line cap=round,fill=fillColor] (123.32,526.65) circle (  1.16);

\path[draw=drawColor,line width= 0.4pt,line join=round,line cap=round,fill=fillColor] (123.62,526.53) circle (  1.16);

\path[draw=drawColor,line width= 0.4pt,line join=round,line cap=round,fill=fillColor] (123.93,526.46) circle (  1.16);

\path[draw=drawColor,line width= 0.4pt,line join=round,line cap=round,fill=fillColor] (124.23,526.44) circle (  1.16);

\path[draw=drawColor,line width= 0.4pt,line join=round,line cap=round,fill=fillColor] (124.52,526.42) circle (  1.16);

\path[draw=drawColor,line width= 0.4pt,line join=round,line cap=round,fill=fillColor] (124.82,526.35) circle (  1.16);

\path[draw=drawColor,line width= 0.4pt,line join=round,line cap=round,fill=fillColor] (125.11,526.33) circle (  1.16);

\path[draw=drawColor,line width= 0.4pt,line join=round,line cap=round,fill=fillColor] (125.40,526.29) circle (  1.16);

\path[draw=drawColor,line width= 0.4pt,line join=round,line cap=round,fill=fillColor] (125.68,526.25) circle (  1.16);

\path[draw=drawColor,line width= 0.4pt,line join=round,line cap=round,fill=fillColor] (125.96,526.20) circle (  1.16);

\path[draw=drawColor,line width= 0.4pt,line join=round,line cap=round,fill=fillColor] (126.24,525.98) circle (  1.16);

\path[draw=drawColor,line width= 0.4pt,line join=round,line cap=round,fill=fillColor] (126.52,525.96) circle (  1.16);

\path[draw=drawColor,line width= 0.4pt,line join=round,line cap=round,fill=fillColor] (126.80,525.96) circle (  1.16);

\path[draw=drawColor,line width= 0.4pt,line join=round,line cap=round,fill=fillColor] (127.07,525.89) circle (  1.16);

\path[draw=drawColor,line width= 0.4pt,line join=round,line cap=round,fill=fillColor] (127.34,525.88) circle (  1.16);

\path[draw=drawColor,line width= 0.4pt,line join=round,line cap=round,fill=fillColor] (127.61,525.81) circle (  1.16);

\path[draw=drawColor,line width= 0.4pt,line join=round,line cap=round,fill=fillColor] (127.87,525.79) circle (  1.16);

\path[draw=drawColor,line width= 0.4pt,line join=round,line cap=round,fill=fillColor] (128.13,525.65) circle (  1.16);

\path[draw=drawColor,line width= 0.4pt,line join=round,line cap=round,fill=fillColor] (128.40,525.62) circle (  1.16);

\path[draw=drawColor,line width= 0.4pt,line join=round,line cap=round,fill=fillColor] (128.65,525.58) circle (  1.16);

\path[draw=drawColor,line width= 0.4pt,line join=round,line cap=round,fill=fillColor] (128.91,525.54) circle (  1.16);

\path[draw=drawColor,line width= 0.4pt,line join=round,line cap=round,fill=fillColor] (129.16,525.52) circle (  1.16);

\path[draw=drawColor,line width= 0.4pt,line join=round,line cap=round,fill=fillColor] (129.42,525.49) circle (  1.16);

\path[draw=drawColor,line width= 0.4pt,line join=round,line cap=round,fill=fillColor] (129.67,525.47) circle (  1.16);

\path[draw=drawColor,line width= 0.4pt,line join=round,line cap=round,fill=fillColor] (129.91,525.42) circle (  1.16);

\path[draw=drawColor,line width= 0.4pt,line join=round,line cap=round,fill=fillColor] (130.16,525.33) circle (  1.16);

\path[draw=drawColor,line width= 0.4pt,line join=round,line cap=round,fill=fillColor] (130.41,525.33) circle (  1.16);

\path[draw=drawColor,line width= 0.4pt,line join=round,line cap=round,fill=fillColor] (130.65,525.29) circle (  1.16);

\path[draw=drawColor,line width= 0.4pt,line join=round,line cap=round,fill=fillColor] (130.89,525.28) circle (  1.16);

\path[draw=drawColor,line width= 0.4pt,line join=round,line cap=round,fill=fillColor] (131.13,525.25) circle (  1.16);

\path[draw=drawColor,line width= 0.4pt,line join=round,line cap=round,fill=fillColor] (131.36,525.22) circle (  1.16);

\path[draw=drawColor,line width= 0.4pt,line join=round,line cap=round,fill=fillColor] (131.60,525.18) circle (  1.16);

\path[draw=drawColor,line width= 0.4pt,line join=round,line cap=round,fill=fillColor] (131.83,525.18) circle (  1.16);

\path[draw=drawColor,line width= 0.4pt,line join=round,line cap=round,fill=fillColor] (132.06,525.08) circle (  1.16);

\path[draw=drawColor,line width= 0.4pt,line join=round,line cap=round,fill=fillColor] (132.30,525.06) circle (  1.16);

\path[draw=drawColor,line width= 0.4pt,line join=round,line cap=round,fill=fillColor] (132.52,525.06) circle (  1.16);

\path[draw=drawColor,line width= 0.4pt,line join=round,line cap=round,fill=fillColor] (132.75,525.01) circle (  1.16);

\path[draw=drawColor,line width= 0.4pt,line join=round,line cap=round,fill=fillColor] (132.98,525.00) circle (  1.16);

\path[draw=drawColor,line width= 0.4pt,line join=round,line cap=round,fill=fillColor] (133.20,524.87) circle (  1.16);

\path[draw=drawColor,line width= 0.4pt,line join=round,line cap=round,fill=fillColor] (133.42,524.80) circle (  1.16);

\path[draw=drawColor,line width= 0.4pt,line join=round,line cap=round,fill=fillColor] (133.64,524.79) circle (  1.16);

\path[draw=drawColor,line width= 0.4pt,line join=round,line cap=round,fill=fillColor] (133.86,524.79) circle (  1.16);

\path[draw=drawColor,line width= 0.4pt,line join=round,line cap=round,fill=fillColor] (134.08,524.71) circle (  1.16);

\path[draw=drawColor,line width= 0.4pt,line join=round,line cap=round,fill=fillColor] (134.30,524.65) circle (  1.16);

\path[draw=drawColor,line width= 0.4pt,line join=round,line cap=round,fill=fillColor] (134.51,524.62) circle (  1.16);

\path[draw=drawColor,line width= 0.4pt,line join=round,line cap=round,fill=fillColor] (134.73,524.62) circle (  1.16);

\path[draw=drawColor,line width= 0.4pt,line join=round,line cap=round,fill=fillColor] (134.94,524.61) circle (  1.16);

\path[draw=drawColor,line width= 0.4pt,line join=round,line cap=round,fill=fillColor] (135.15,524.59) circle (  1.16);

\path[draw=drawColor,line width= 0.4pt,line join=round,line cap=round,fill=fillColor] (135.36,524.56) circle (  1.16);

\path[draw=drawColor,line width= 0.4pt,line join=round,line cap=round,fill=fillColor] (135.57,524.53) circle (  1.16);

\path[draw=drawColor,line width= 0.4pt,line join=round,line cap=round,fill=fillColor] (135.78,524.51) circle (  1.16);

\path[draw=drawColor,line width= 0.4pt,line join=round,line cap=round,fill=fillColor] (135.98,524.50) circle (  1.16);

\path[draw=drawColor,line width= 0.4pt,line join=round,line cap=round,fill=fillColor] (136.19,524.45) circle (  1.16);

\path[draw=drawColor,line width= 0.4pt,line join=round,line cap=round,fill=fillColor] (136.39,524.44) circle (  1.16);

\path[draw=drawColor,line width= 0.4pt,line join=round,line cap=round,fill=fillColor] (136.60,524.44) circle (  1.16);

\path[draw=drawColor,line width= 0.4pt,line join=round,line cap=round,fill=fillColor] (136.80,524.43) circle (  1.16);

\path[draw=drawColor,line width= 0.4pt,line join=round,line cap=round,fill=fillColor] (137.00,524.29) circle (  1.16);

\path[draw=drawColor,line width= 0.4pt,line join=round,line cap=round,fill=fillColor] (137.20,524.27) circle (  1.16);

\path[draw=drawColor,line width= 0.4pt,line join=round,line cap=round,fill=fillColor] (137.40,524.26) circle (  1.16);

\path[draw=drawColor,line width= 0.4pt,line join=round,line cap=round,fill=fillColor] (137.59,524.25) circle (  1.16);

\path[draw=drawColor,line width= 0.4pt,line join=round,line cap=round,fill=fillColor] (137.79,524.19) circle (  1.16);

\path[draw=drawColor,line width= 0.4pt,line join=round,line cap=round,fill=fillColor] (137.98,524.12) circle (  1.16);

\path[draw=drawColor,line width= 0.4pt,line join=round,line cap=round,fill=fillColor] (138.18,524.11) circle (  1.16);

\path[draw=drawColor,line width= 0.4pt,line join=round,line cap=round,fill=fillColor] (138.37,524.04) circle (  1.16);

\path[draw=drawColor,line width= 0.4pt,line join=round,line cap=round,fill=fillColor] (138.56,523.95) circle (  1.16);

\path[draw=drawColor,line width= 0.4pt,line join=round,line cap=round,fill=fillColor] (138.75,523.90) circle (  1.16);

\path[draw=drawColor,line width= 0.4pt,line join=round,line cap=round,fill=fillColor] (138.94,523.86) circle (  1.16);

\path[draw=drawColor,line width= 0.4pt,line join=round,line cap=round,fill=fillColor] (139.13,523.81) circle (  1.16);

\path[draw=drawColor,line width= 0.4pt,line join=round,line cap=round,fill=fillColor] (139.32,523.78) circle (  1.16);

\path[draw=drawColor,line width= 0.4pt,line join=round,line cap=round,fill=fillColor] (139.50,523.77) circle (  1.16);

\path[draw=drawColor,line width= 0.4pt,line join=round,line cap=round,fill=fillColor] (139.69,523.75) circle (  1.16);

\path[draw=drawColor,line width= 0.4pt,line join=round,line cap=round,fill=fillColor] (139.87,523.72) circle (  1.16);

\path[draw=drawColor,line width= 0.4pt,line join=round,line cap=round,fill=fillColor] (140.06,523.72) circle (  1.16);

\path[draw=drawColor,line width= 0.4pt,line join=round,line cap=round,fill=fillColor] (140.24,523.68) circle (  1.16);

\path[draw=drawColor,line width= 0.4pt,line join=round,line cap=round,fill=fillColor] (140.42,523.66) circle (  1.16);

\path[draw=drawColor,line width= 0.4pt,line join=round,line cap=round,fill=fillColor] (140.60,523.66) circle (  1.16);

\path[draw=drawColor,line width= 0.4pt,line join=round,line cap=round,fill=fillColor] (140.78,523.60) circle (  1.16);

\path[draw=drawColor,line width= 0.4pt,line join=round,line cap=round,fill=fillColor] (140.96,523.60) circle (  1.16);

\path[draw=drawColor,line width= 0.4pt,line join=round,line cap=round,fill=fillColor] (141.14,523.59) circle (  1.16);

\path[draw=drawColor,line width= 0.4pt,line join=round,line cap=round,fill=fillColor] (141.32,523.49) circle (  1.16);

\path[draw=drawColor,line width= 0.4pt,line join=round,line cap=round,fill=fillColor] (141.50,523.46) circle (  1.16);

\path[draw=drawColor,line width= 0.4pt,line join=round,line cap=round,fill=fillColor] (141.67,523.45) circle (  1.16);

\path[draw=drawColor,line width= 0.4pt,line join=round,line cap=round,fill=fillColor] (141.85,523.45) circle (  1.16);

\path[draw=drawColor,line width= 0.4pt,line join=round,line cap=round,fill=fillColor] (142.02,523.43) circle (  1.16);

\path[draw=drawColor,line width= 0.4pt,line join=round,line cap=round,fill=fillColor] (142.20,523.32) circle (  1.16);

\path[draw=drawColor,line width= 0.4pt,line join=round,line cap=round,fill=fillColor] (142.37,523.31) circle (  1.16);

\path[draw=drawColor,line width= 0.4pt,line join=round,line cap=round,fill=fillColor] (142.54,523.30) circle (  1.16);

\path[draw=drawColor,line width= 0.4pt,line join=round,line cap=round,fill=fillColor] (142.71,523.30) circle (  1.16);

\path[draw=drawColor,line width= 0.4pt,line join=round,line cap=round,fill=fillColor] (142.88,523.29) circle (  1.16);

\path[draw=drawColor,line width= 0.4pt,line join=round,line cap=round,fill=fillColor] (143.05,523.26) circle (  1.16);

\path[draw=drawColor,line width= 0.4pt,line join=round,line cap=round,fill=fillColor] (143.22,523.26) circle (  1.16);

\path[draw=drawColor,line width= 0.4pt,line join=round,line cap=round,fill=fillColor] (143.39,523.17) circle (  1.16);

\path[draw=drawColor,line width= 0.4pt,line join=round,line cap=round,fill=fillColor] (143.56,523.13) circle (  1.16);

\path[draw=drawColor,line width= 0.4pt,line join=round,line cap=round,fill=fillColor] (143.72,523.12) circle (  1.16);

\path[draw=drawColor,line width= 0.4pt,line join=round,line cap=round,fill=fillColor] (143.89,523.12) circle (  1.16);

\path[draw=drawColor,line width= 0.4pt,line join=round,line cap=round,fill=fillColor] (144.05,523.09) circle (  1.16);

\path[draw=drawColor,line width= 0.4pt,line join=round,line cap=round,fill=fillColor] (144.22,523.08) circle (  1.16);

\path[draw=drawColor,line width= 0.4pt,line join=round,line cap=round,fill=fillColor] (144.38,523.06) circle (  1.16);

\path[draw=drawColor,line width= 0.4pt,line join=round,line cap=round,fill=fillColor] (144.55,523.03) circle (  1.16);

\path[draw=drawColor,line width= 0.4pt,line join=round,line cap=round,fill=fillColor] (144.71,523.00) circle (  1.16);

\path[draw=drawColor,line width= 0.4pt,line join=round,line cap=round,fill=fillColor] (144.87,522.99) circle (  1.16);

\path[draw=drawColor,line width= 0.4pt,line join=round,line cap=round,fill=fillColor] (145.03,522.99) circle (  1.16);

\path[draw=drawColor,line width= 0.4pt,line join=round,line cap=round,fill=fillColor] (145.19,522.95) circle (  1.16);

\path[draw=drawColor,line width= 0.4pt,line join=round,line cap=round,fill=fillColor] (145.35,522.92) circle (  1.16);

\path[draw=drawColor,line width= 0.4pt,line join=round,line cap=round,fill=fillColor] (145.51,522.84) circle (  1.16);

\path[draw=drawColor,line width= 0.4pt,line join=round,line cap=round,fill=fillColor] (145.67,522.79) circle (  1.16);

\path[draw=drawColor,line width= 0.4pt,line join=round,line cap=round,fill=fillColor] (145.83,522.79) circle (  1.16);

\path[draw=drawColor,line width= 0.4pt,line join=round,line cap=round,fill=fillColor] (145.98,522.78) circle (  1.16);

\path[draw=drawColor,line width= 0.4pt,line join=round,line cap=round,fill=fillColor] (146.14,522.77) circle (  1.16);

\path[draw=drawColor,line width= 0.4pt,line join=round,line cap=round,fill=fillColor] (146.30,522.75) circle (  1.16);

\path[draw=drawColor,line width= 0.4pt,line join=round,line cap=round,fill=fillColor] (146.45,522.68) circle (  1.16);

\path[draw=drawColor,line width= 0.4pt,line join=round,line cap=round,fill=fillColor] (146.61,522.65) circle (  1.16);

\path[draw=drawColor,line width= 0.4pt,line join=round,line cap=round,fill=fillColor] (146.76,522.64) circle (  1.16);

\path[draw=drawColor,line width= 0.4pt,line join=round,line cap=round,fill=fillColor] (146.91,522.61) circle (  1.16);

\path[draw=drawColor,line width= 0.4pt,line join=round,line cap=round,fill=fillColor] (147.07,522.58) circle (  1.16);

\path[draw=drawColor,line width= 0.4pt,line join=round,line cap=round,fill=fillColor] (147.22,522.57) circle (  1.16);

\path[draw=drawColor,line width= 0.4pt,line join=round,line cap=round,fill=fillColor] (147.37,522.48) circle (  1.16);

\path[draw=drawColor,line width= 0.4pt,line join=round,line cap=round,fill=fillColor] (147.52,522.42) circle (  1.16);

\path[draw=drawColor,line width= 0.4pt,line join=round,line cap=round,fill=fillColor] (147.67,522.40) circle (  1.16);

\path[draw=drawColor,line width= 0.4pt,line join=round,line cap=round,fill=fillColor] (147.82,522.38) circle (  1.16);

\path[draw=drawColor,line width= 0.4pt,line join=round,line cap=round,fill=fillColor] (147.97,522.36) circle (  1.16);

\path[draw=drawColor,line width= 0.4pt,line join=round,line cap=round,fill=fillColor] (148.12,522.28) circle (  1.16);

\path[draw=drawColor,line width= 0.4pt,line join=round,line cap=round,fill=fillColor] (148.27,522.27) circle (  1.16);

\path[draw=drawColor,line width= 0.4pt,line join=round,line cap=round,fill=fillColor] (148.42,522.26) circle (  1.16);

\path[draw=drawColor,line width= 0.4pt,line join=round,line cap=round,fill=fillColor] (148.57,522.22) circle (  1.16);

\path[draw=drawColor,line width= 0.4pt,line join=round,line cap=round,fill=fillColor] (148.71,522.21) circle (  1.16);

\path[draw=drawColor,line width= 0.4pt,line join=round,line cap=round,fill=fillColor] (148.86,522.20) circle (  1.16);

\path[draw=drawColor,line width= 0.4pt,line join=round,line cap=round,fill=fillColor] (149.01,522.20) circle (  1.16);

\path[draw=drawColor,line width= 0.4pt,line join=round,line cap=round,fill=fillColor] (149.15,522.19) circle (  1.16);

\path[draw=drawColor,line width= 0.4pt,line join=round,line cap=round,fill=fillColor] (149.30,522.19) circle (  1.16);

\path[draw=drawColor,line width= 0.4pt,line join=round,line cap=round,fill=fillColor] (149.44,522.16) circle (  1.16);

\path[draw=drawColor,line width= 0.4pt,line join=round,line cap=round,fill=fillColor] (149.58,522.11) circle (  1.16);

\path[draw=drawColor,line width= 0.4pt,line join=round,line cap=round,fill=fillColor] (149.73,522.11) circle (  1.16);

\path[draw=drawColor,line width= 0.4pt,line join=round,line cap=round,fill=fillColor] (149.87,522.04) circle (  1.16);

\path[draw=drawColor,line width= 0.4pt,line join=round,line cap=round,fill=fillColor] (150.01,522.04) circle (  1.16);

\path[draw=drawColor,line width= 0.4pt,line join=round,line cap=round,fill=fillColor] (150.15,522.03) circle (  1.16);

\path[draw=drawColor,line width= 0.4pt,line join=round,line cap=round,fill=fillColor] (150.30,522.00) circle (  1.16);

\path[draw=drawColor,line width= 0.4pt,line join=round,line cap=round,fill=fillColor] (150.44,521.93) circle (  1.16);

\path[draw=drawColor,line width= 0.4pt,line join=round,line cap=round,fill=fillColor] (150.58,521.93) circle (  1.16);

\path[draw=drawColor,line width= 0.4pt,line join=round,line cap=round,fill=fillColor] (150.72,521.87) circle (  1.16);

\path[draw=drawColor,line width= 0.4pt,line join=round,line cap=round,fill=fillColor] (150.86,521.84) circle (  1.16);

\path[draw=drawColor,line width= 0.4pt,line join=round,line cap=round,fill=fillColor] (151.00,521.83) circle (  1.16);

\path[draw=drawColor,line width= 0.4pt,line join=round,line cap=round,fill=fillColor] (151.13,521.81) circle (  1.16);

\path[draw=drawColor,line width= 0.4pt,line join=round,line cap=round,fill=fillColor] (151.27,521.76) circle (  1.16);

\path[draw=drawColor,line width= 0.4pt,line join=round,line cap=round,fill=fillColor] (151.41,521.75) circle (  1.16);

\path[draw=drawColor,line width= 0.4pt,line join=round,line cap=round,fill=fillColor] (151.55,521.73) circle (  1.16);

\path[draw=drawColor,line width= 0.4pt,line join=round,line cap=round,fill=fillColor] (151.68,521.72) circle (  1.16);

\path[draw=drawColor,line width= 0.4pt,line join=round,line cap=round,fill=fillColor] (151.82,521.71) circle (  1.16);

\path[draw=drawColor,line width= 0.4pt,line join=round,line cap=round,fill=fillColor] (151.96,521.67) circle (  1.16);

\path[draw=drawColor,line width= 0.4pt,line join=round,line cap=round,fill=fillColor] (152.09,521.64) circle (  1.16);

\path[draw=drawColor,line width= 0.4pt,line join=round,line cap=round,fill=fillColor] (152.23,521.64) circle (  1.16);

\path[draw=drawColor,line width= 0.4pt,line join=round,line cap=round,fill=fillColor] (152.36,521.60) circle (  1.16);

\path[draw=drawColor,line width= 0.4pt,line join=round,line cap=round,fill=fillColor] (152.49,521.55) circle (  1.16);

\path[draw=drawColor,line width= 0.4pt,line join=round,line cap=round,fill=fillColor] (152.63,521.55) circle (  1.16);

\path[draw=drawColor,line width= 0.4pt,line join=round,line cap=round,fill=fillColor] (152.76,521.55) circle (  1.16);

\path[draw=drawColor,line width= 0.4pt,line join=round,line cap=round,fill=fillColor] (152.89,521.55) circle (  1.16);

\path[draw=drawColor,line width= 0.4pt,line join=round,line cap=round,fill=fillColor] (153.03,521.52) circle (  1.16);

\path[draw=drawColor,line width= 0.4pt,line join=round,line cap=round,fill=fillColor] (153.16,521.51) circle (  1.16);

\path[draw=drawColor,line width= 0.4pt,line join=round,line cap=round,fill=fillColor] (153.29,521.51) circle (  1.16);

\path[draw=drawColor,line width= 0.4pt,line join=round,line cap=round,fill=fillColor] (153.42,521.49) circle (  1.16);

\path[draw=drawColor,line width= 0.4pt,line join=round,line cap=round,fill=fillColor] (153.55,521.48) circle (  1.16);

\path[draw=drawColor,line width= 0.4pt,line join=round,line cap=round,fill=fillColor] (153.68,521.45) circle (  1.16);

\path[draw=drawColor,line width= 0.4pt,line join=round,line cap=round,fill=fillColor] (153.81,521.40) circle (  1.16);

\path[draw=drawColor,line width= 0.4pt,line join=round,line cap=round,fill=fillColor] (153.94,521.39) circle (  1.16);

\path[draw=drawColor,line width= 0.4pt,line join=round,line cap=round,fill=fillColor] (154.07,521.39) circle (  1.16);

\path[draw=drawColor,line width= 0.4pt,line join=round,line cap=round,fill=fillColor] (154.20,521.37) circle (  1.16);

\path[draw=drawColor,line width= 0.4pt,line join=round,line cap=round,fill=fillColor] (154.33,521.35) circle (  1.16);

\path[draw=drawColor,line width= 0.4pt,line join=round,line cap=round,fill=fillColor] (154.46,521.31) circle (  1.16);

\path[draw=drawColor,line width= 0.4pt,line join=round,line cap=round,fill=fillColor] (154.59,521.31) circle (  1.16);

\path[draw=drawColor,line width= 0.4pt,line join=round,line cap=round,fill=fillColor] (154.71,521.27) circle (  1.16);

\path[draw=drawColor,line width= 0.4pt,line join=round,line cap=round,fill=fillColor] (154.84,521.27) circle (  1.16);

\path[draw=drawColor,line width= 0.4pt,line join=round,line cap=round,fill=fillColor] (154.97,521.25) circle (  1.16);

\path[draw=drawColor,line width= 0.4pt,line join=round,line cap=round,fill=fillColor] (155.09,521.23) circle (  1.16);

\path[draw=drawColor,line width= 0.4pt,line join=round,line cap=round,fill=fillColor] (155.22,521.20) circle (  1.16);

\path[draw=drawColor,line width= 0.4pt,line join=round,line cap=round,fill=fillColor] (155.35,521.18) circle (  1.16);

\path[draw=drawColor,line width= 0.4pt,line join=round,line cap=round,fill=fillColor] (155.47,521.17) circle (  1.16);

\path[draw=drawColor,line width= 0.4pt,line join=round,line cap=round,fill=fillColor] (155.60,521.17) circle (  1.16);

\path[draw=drawColor,line width= 0.4pt,line join=round,line cap=round,fill=fillColor] (155.72,521.14) circle (  1.16);

\path[draw=drawColor,line width= 0.4pt,line join=round,line cap=round,fill=fillColor] (155.85,521.14) circle (  1.16);

\path[draw=drawColor,line width= 0.4pt,line join=round,line cap=round,fill=fillColor] (155.97,521.10) circle (  1.16);

\path[draw=drawColor,line width= 0.4pt,line join=round,line cap=round,fill=fillColor] (156.09,521.06) circle (  1.16);

\path[draw=drawColor,line width= 0.4pt,line join=round,line cap=round,fill=fillColor] (156.22,521.04) circle (  1.16);

\path[draw=drawColor,line width= 0.4pt,line join=round,line cap=round,fill=fillColor] (156.34,521.04) circle (  1.16);

\path[draw=drawColor,line width= 0.4pt,line join=round,line cap=round,fill=fillColor] (156.46,521.03) circle (  1.16);

\path[draw=drawColor,line width= 0.4pt,line join=round,line cap=round,fill=fillColor] (156.58,521.02) circle (  1.16);

\path[draw=drawColor,line width= 0.4pt,line join=round,line cap=round,fill=fillColor] (156.71,521.01) circle (  1.16);

\path[draw=drawColor,line width= 0.4pt,line join=round,line cap=round,fill=fillColor] (156.83,521.01) circle (  1.16);

\path[draw=drawColor,line width= 0.4pt,line join=round,line cap=round,fill=fillColor] (156.95,521.01) circle (  1.16);

\path[draw=drawColor,line width= 0.4pt,line join=round,line cap=round,fill=fillColor] (157.07,521.00) circle (  1.16);

\path[draw=drawColor,line width= 0.4pt,line join=round,line cap=round,fill=fillColor] (157.19,520.99) circle (  1.16);

\path[draw=drawColor,line width= 0.4pt,line join=round,line cap=round,fill=fillColor] (157.31,520.99) circle (  1.16);

\path[draw=drawColor,line width= 0.4pt,line join=round,line cap=round,fill=fillColor] (157.43,520.98) circle (  1.16);

\path[draw=drawColor,line width= 0.4pt,line join=round,line cap=round,fill=fillColor] (157.55,520.98) circle (  1.16);

\path[draw=drawColor,line width= 0.4pt,line join=round,line cap=round,fill=fillColor] (157.67,520.98) circle (  1.16);

\path[draw=drawColor,line width= 0.4pt,line join=round,line cap=round,fill=fillColor] (157.79,520.96) circle (  1.16);

\path[draw=drawColor,line width= 0.4pt,line join=round,line cap=round,fill=fillColor] (157.91,520.94) circle (  1.16);

\path[draw=drawColor,line width= 0.4pt,line join=round,line cap=round,fill=fillColor] (158.02,520.91) circle (  1.16);

\path[draw=drawColor,line width= 0.4pt,line join=round,line cap=round,fill=fillColor] (158.14,520.89) circle (  1.16);

\path[draw=drawColor,line width= 0.4pt,line join=round,line cap=round,fill=fillColor] (158.26,520.88) circle (  1.16);

\path[draw=drawColor,line width= 0.4pt,line join=round,line cap=round,fill=fillColor] (158.38,520.87) circle (  1.16);

\path[draw=drawColor,line width= 0.4pt,line join=round,line cap=round,fill=fillColor] (158.50,520.86) circle (  1.16);

\path[draw=drawColor,line width= 0.4pt,line join=round,line cap=round,fill=fillColor] (158.61,520.86) circle (  1.16);

\path[draw=drawColor,line width= 0.4pt,line join=round,line cap=round,fill=fillColor] (158.73,520.86) circle (  1.16);

\path[draw=drawColor,line width= 0.4pt,line join=round,line cap=round,fill=fillColor] (158.84,520.85) circle (  1.16);

\path[draw=drawColor,line width= 0.4pt,line join=round,line cap=round,fill=fillColor] (158.96,520.84) circle (  1.16);

\path[draw=drawColor,line width= 0.4pt,line join=round,line cap=round,fill=fillColor] (159.08,520.83) circle (  1.16);

\path[draw=drawColor,line width= 0.4pt,line join=round,line cap=round,fill=fillColor] (159.19,520.82) circle (  1.16);

\path[draw=drawColor,line width= 0.4pt,line join=round,line cap=round,fill=fillColor] (159.31,520.81) circle (  1.16);

\path[draw=drawColor,line width= 0.4pt,line join=round,line cap=round,fill=fillColor] (159.42,520.76) circle (  1.16);

\path[draw=drawColor,line width= 0.4pt,line join=round,line cap=round,fill=fillColor] (159.54,520.70) circle (  1.16);

\path[draw=drawColor,line width= 0.4pt,line join=round,line cap=round,fill=fillColor] (159.65,520.68) circle (  1.16);

\path[draw=drawColor,line width= 0.4pt,line join=round,line cap=round,fill=fillColor] (159.76,520.64) circle (  1.16);

\path[draw=drawColor,line width= 0.4pt,line join=round,line cap=round,fill=fillColor] (159.88,520.62) circle (  1.16);

\path[draw=drawColor,line width= 0.4pt,line join=round,line cap=round,fill=fillColor] (159.99,520.61) circle (  1.16);

\path[draw=drawColor,line width= 0.4pt,line join=round,line cap=round,fill=fillColor] (160.10,520.60) circle (  1.16);

\path[draw=drawColor,line width= 0.4pt,line join=round,line cap=round,fill=fillColor] (160.22,520.58) circle (  1.16);

\path[draw=drawColor,line width= 0.4pt,line join=round,line cap=round,fill=fillColor] (160.33,520.53) circle (  1.16);

\path[draw=drawColor,line width= 0.4pt,line join=round,line cap=round,fill=fillColor] (160.44,520.53) circle (  1.16);

\path[draw=drawColor,line width= 0.4pt,line join=round,line cap=round,fill=fillColor] (160.55,520.52) circle (  1.16);

\path[draw=drawColor,line width= 0.4pt,line join=round,line cap=round,fill=fillColor] (160.67,520.52) circle (  1.16);

\path[draw=drawColor,line width= 0.4pt,line join=round,line cap=round,fill=fillColor] (160.78,520.49) circle (  1.16);

\path[draw=drawColor,line width= 0.4pt,line join=round,line cap=round,fill=fillColor] (160.89,520.48) circle (  1.16);

\path[draw=drawColor,line width= 0.4pt,line join=round,line cap=round,fill=fillColor] (161.00,520.47) circle (  1.16);

\path[draw=drawColor,line width= 0.4pt,line join=round,line cap=round,fill=fillColor] (161.11,520.46) circle (  1.16);

\path[draw=drawColor,line width= 0.4pt,line join=round,line cap=round,fill=fillColor] (161.22,520.46) circle (  1.16);

\path[draw=drawColor,line width= 0.4pt,line join=round,line cap=round,fill=fillColor] (161.33,520.45) circle (  1.16);

\path[draw=drawColor,line width= 0.4pt,line join=round,line cap=round,fill=fillColor] (161.44,520.45) circle (  1.16);

\path[draw=drawColor,line width= 0.4pt,line join=round,line cap=round,fill=fillColor] (161.55,520.43) circle (  1.16);

\path[draw=drawColor,line width= 0.4pt,line join=round,line cap=round,fill=fillColor] (161.66,520.43) circle (  1.16);

\path[draw=drawColor,line width= 0.4pt,line join=round,line cap=round,fill=fillColor] (161.77,520.42) circle (  1.16);

\path[draw=drawColor,line width= 0.4pt,line join=round,line cap=round,fill=fillColor] (161.88,520.40) circle (  1.16);

\path[draw=drawColor,line width= 0.4pt,line join=round,line cap=round,fill=fillColor] (161.99,520.38) circle (  1.16);

\path[draw=drawColor,line width= 0.4pt,line join=round,line cap=round,fill=fillColor] (162.10,520.32) circle (  1.16);

\path[draw=drawColor,line width= 0.4pt,line join=round,line cap=round,fill=fillColor] (162.20,520.31) circle (  1.16);

\path[draw=drawColor,line width= 0.4pt,line join=round,line cap=round,fill=fillColor] (162.31,520.30) circle (  1.16);

\path[draw=drawColor,line width= 0.4pt,line join=round,line cap=round,fill=fillColor] (162.42,520.28) circle (  1.16);

\path[draw=drawColor,line width= 0.4pt,line join=round,line cap=round,fill=fillColor] (162.53,520.28) circle (  1.16);

\path[draw=drawColor,line width= 0.4pt,line join=round,line cap=round,fill=fillColor] (162.63,520.26) circle (  1.16);

\path[draw=drawColor,line width= 0.4pt,line join=round,line cap=round,fill=fillColor] (162.74,520.23) circle (  1.16);

\path[draw=drawColor,line width= 0.4pt,line join=round,line cap=round,fill=fillColor] (162.85,520.23) circle (  1.16);

\path[draw=drawColor,line width= 0.4pt,line join=round,line cap=round,fill=fillColor] (162.95,520.23) circle (  1.16);

\path[draw=drawColor,line width= 0.4pt,line join=round,line cap=round,fill=fillColor] (163.06,520.23) circle (  1.16);

\path[draw=drawColor,line width= 0.4pt,line join=round,line cap=round,fill=fillColor] (163.17,520.23) circle (  1.16);

\path[draw=drawColor,line width= 0.4pt,line join=round,line cap=round,fill=fillColor] (163.27,520.22) circle (  1.16);

\path[draw=drawColor,line width= 0.4pt,line join=round,line cap=round,fill=fillColor] (163.38,520.17) circle (  1.16);

\path[draw=drawColor,line width= 0.4pt,line join=round,line cap=round,fill=fillColor] (163.48,520.17) circle (  1.16);

\path[draw=drawColor,line width= 0.4pt,line join=round,line cap=round,fill=fillColor] (163.59,520.13) circle (  1.16);

\path[draw=drawColor,line width= 0.4pt,line join=round,line cap=round,fill=fillColor] (163.69,520.12) circle (  1.16);

\path[draw=drawColor,line width= 0.4pt,line join=round,line cap=round,fill=fillColor] (163.80,520.05) circle (  1.16);

\path[draw=drawColor,line width= 0.4pt,line join=round,line cap=round,fill=fillColor] (163.90,520.02) circle (  1.16);

\path[draw=drawColor,line width= 0.4pt,line join=round,line cap=round,fill=fillColor] (164.01,520.01) circle (  1.16);

\path[draw=drawColor,line width= 0.4pt,line join=round,line cap=round,fill=fillColor] (164.11,519.99) circle (  1.16);

\path[draw=drawColor,line width= 0.4pt,line join=round,line cap=round,fill=fillColor] (164.21,519.98) circle (  1.16);

\path[draw=drawColor,line width= 0.4pt,line join=round,line cap=round,fill=fillColor] (164.32,519.98) circle (  1.16);

\path[draw=drawColor,line width= 0.4pt,line join=round,line cap=round,fill=fillColor] (164.42,519.92) circle (  1.16);

\path[draw=drawColor,line width= 0.4pt,line join=round,line cap=round,fill=fillColor] (164.52,519.92) circle (  1.16);

\path[draw=drawColor,line width= 0.4pt,line join=round,line cap=round,fill=fillColor] (164.63,519.91) circle (  1.16);

\path[draw=drawColor,line width= 0.4pt,line join=round,line cap=round,fill=fillColor] (164.73,519.89) circle (  1.16);

\path[draw=drawColor,line width= 0.4pt,line join=round,line cap=round,fill=fillColor] (164.83,519.89) circle (  1.16);

\path[draw=drawColor,line width= 0.4pt,line join=round,line cap=round,fill=fillColor] (164.93,519.86) circle (  1.16);

\path[draw=drawColor,line width= 0.4pt,line join=round,line cap=round,fill=fillColor] (165.04,519.86) circle (  1.16);

\path[draw=drawColor,line width= 0.4pt,line join=round,line cap=round,fill=fillColor] (165.14,519.83) circle (  1.16);

\path[draw=drawColor,line width= 0.4pt,line join=round,line cap=round,fill=fillColor] (165.24,519.81) circle (  1.16);

\path[draw=drawColor,line width= 0.4pt,line join=round,line cap=round,fill=fillColor] (165.34,519.77) circle (  1.16);

\path[draw=drawColor,line width= 0.4pt,line join=round,line cap=round,fill=fillColor] (165.44,519.74) circle (  1.16);

\path[draw=drawColor,line width= 0.4pt,line join=round,line cap=round,fill=fillColor] (165.54,519.73) circle (  1.16);

\path[draw=drawColor,line width= 0.4pt,line join=round,line cap=round,fill=fillColor] (165.64,519.73) circle (  1.16);

\path[draw=drawColor,line width= 0.4pt,line join=round,line cap=round,fill=fillColor] (165.74,519.73) circle (  1.16);

\path[draw=drawColor,line width= 0.4pt,line join=round,line cap=round,fill=fillColor] (165.85,519.72) circle (  1.16);

\path[draw=drawColor,line width= 0.4pt,line join=round,line cap=round,fill=fillColor] (165.95,519.70) circle (  1.16);

\path[draw=drawColor,line width= 0.4pt,line join=round,line cap=round,fill=fillColor] (166.05,519.69) circle (  1.16);

\path[draw=drawColor,line width= 0.4pt,line join=round,line cap=round,fill=fillColor] (166.14,519.67) circle (  1.16);

\path[draw=drawColor,line width= 0.4pt,line join=round,line cap=round,fill=fillColor] (166.24,519.63) circle (  1.16);

\path[draw=drawColor,line width= 0.4pt,line join=round,line cap=round,fill=fillColor] (166.34,519.63) circle (  1.16);

\path[draw=drawColor,line width= 0.4pt,line join=round,line cap=round,fill=fillColor] (166.44,519.63) circle (  1.16);

\path[draw=drawColor,line width= 0.4pt,line join=round,line cap=round,fill=fillColor] (166.54,519.60) circle (  1.16);

\path[draw=drawColor,line width= 0.4pt,line join=round,line cap=round,fill=fillColor] (166.64,519.59) circle (  1.16);

\path[draw=drawColor,line width= 0.4pt,line join=round,line cap=round,fill=fillColor] (166.74,519.57) circle (  1.16);

\path[draw=drawColor,line width= 0.4pt,line join=round,line cap=round,fill=fillColor] (166.84,519.56) circle (  1.16);

\path[draw=drawColor,line width= 0.4pt,line join=round,line cap=round,fill=fillColor] (166.94,519.56) circle (  1.16);

\path[draw=drawColor,line width= 0.4pt,line join=round,line cap=round,fill=fillColor] (167.03,519.56) circle (  1.16);

\path[draw=drawColor,line width= 0.4pt,line join=round,line cap=round,fill=fillColor] (167.13,519.51) circle (  1.16);

\path[draw=drawColor,line width= 0.4pt,line join=round,line cap=round,fill=fillColor] (167.23,519.47) circle (  1.16);

\path[draw=drawColor,line width= 0.4pt,line join=round,line cap=round,fill=fillColor] (167.33,519.44) circle (  1.16);

\path[draw=drawColor,line width= 0.4pt,line join=round,line cap=round,fill=fillColor] (167.42,519.44) circle (  1.16);

\path[draw=drawColor,line width= 0.4pt,line join=round,line cap=round,fill=fillColor] (167.52,519.43) circle (  1.16);

\path[draw=drawColor,line width= 0.4pt,line join=round,line cap=round,fill=fillColor] (167.62,519.43) circle (  1.16);

\path[draw=drawColor,line width= 0.4pt,line join=round,line cap=round,fill=fillColor] (167.71,519.40) circle (  1.16);

\path[draw=drawColor,line width= 0.4pt,line join=round,line cap=round,fill=fillColor] (167.81,519.37) circle (  1.16);

\path[draw=drawColor,line width= 0.4pt,line join=round,line cap=round,fill=fillColor] (167.91,519.34) circle (  1.16);

\path[draw=drawColor,line width= 0.4pt,line join=round,line cap=round,fill=fillColor] (168.00,519.34) circle (  1.16);

\path[draw=drawColor,line width= 0.4pt,line join=round,line cap=round,fill=fillColor] (168.10,519.32) circle (  1.16);

\path[draw=drawColor,line width= 0.4pt,line join=round,line cap=round,fill=fillColor] (168.20,519.30) circle (  1.16);

\path[draw=drawColor,line width= 0.4pt,line join=round,line cap=round,fill=fillColor] (168.29,519.30) circle (  1.16);

\path[draw=drawColor,line width= 0.4pt,line join=round,line cap=round,fill=fillColor] (168.39,519.26) circle (  1.16);

\path[draw=drawColor,line width= 0.4pt,line join=round,line cap=round,fill=fillColor] (168.48,519.26) circle (  1.16);

\path[draw=drawColor,line width= 0.4pt,line join=round,line cap=round,fill=fillColor] (168.58,519.25) circle (  1.16);

\path[draw=drawColor,line width= 0.4pt,line join=round,line cap=round,fill=fillColor] (168.67,519.25) circle (  1.16);

\path[draw=drawColor,line width= 0.4pt,line join=round,line cap=round,fill=fillColor] (168.77,519.25) circle (  1.16);

\path[draw=drawColor,line width= 0.4pt,line join=round,line cap=round,fill=fillColor] (168.86,519.24) circle (  1.16);

\path[draw=drawColor,line width= 0.4pt,line join=round,line cap=round,fill=fillColor] (168.95,519.23) circle (  1.16);

\path[draw=drawColor,line width= 0.4pt,line join=round,line cap=round,fill=fillColor] (169.05,519.23) circle (  1.16);

\path[draw=drawColor,line width= 0.4pt,line join=round,line cap=round,fill=fillColor] (169.14,519.22) circle (  1.16);

\path[draw=drawColor,line width= 0.4pt,line join=round,line cap=round,fill=fillColor] (169.24,519.21) circle (  1.16);

\path[draw=drawColor,line width= 0.4pt,line join=round,line cap=round,fill=fillColor] (169.33,519.18) circle (  1.16);

\path[draw=drawColor,line width= 0.4pt,line join=round,line cap=round,fill=fillColor] (169.42,519.18) circle (  1.16);

\path[draw=drawColor,line width= 0.4pt,line join=round,line cap=round,fill=fillColor] (169.52,519.17) circle (  1.16);

\path[draw=drawColor,line width= 0.4pt,line join=round,line cap=round,fill=fillColor] (169.61,519.16) circle (  1.16);

\path[draw=drawColor,line width= 0.4pt,line join=round,line cap=round,fill=fillColor] (169.70,519.16) circle (  1.16);

\path[draw=drawColor,line width= 0.4pt,line join=round,line cap=round,fill=fillColor] (169.80,519.13) circle (  1.16);

\path[draw=drawColor,line width= 0.4pt,line join=round,line cap=round,fill=fillColor] (169.89,519.13) circle (  1.16);

\path[draw=drawColor,line width= 0.4pt,line join=round,line cap=round,fill=fillColor] (169.98,519.10) circle (  1.16);

\path[draw=drawColor,line width= 0.4pt,line join=round,line cap=round,fill=fillColor] (170.07,519.09) circle (  1.16);

\path[draw=drawColor,line width= 0.4pt,line join=round,line cap=round,fill=fillColor] (170.17,519.07) circle (  1.16);

\path[draw=drawColor,line width= 0.4pt,line join=round,line cap=round,fill=fillColor] (170.26,519.03) circle (  1.16);

\path[draw=drawColor,line width= 0.4pt,line join=round,line cap=round,fill=fillColor] (170.35,519.03) circle (  1.16);

\path[draw=drawColor,line width= 0.4pt,line join=round,line cap=round,fill=fillColor] (170.44,518.98) circle (  1.16);

\path[draw=drawColor,line width= 0.4pt,line join=round,line cap=round,fill=fillColor] (170.53,518.87) circle (  1.16);

\path[draw=drawColor,line width= 0.4pt,line join=round,line cap=round,fill=fillColor] (170.62,518.87) circle (  1.16);

\path[draw=drawColor,line width= 0.4pt,line join=round,line cap=round,fill=fillColor] (170.71,518.83) circle (  1.16);

\path[draw=drawColor,line width= 0.4pt,line join=round,line cap=round,fill=fillColor] (170.81,518.82) circle (  1.16);

\path[draw=drawColor,line width= 0.4pt,line join=round,line cap=round,fill=fillColor] (170.90,518.81) circle (  1.16);

\path[draw=drawColor,line width= 0.4pt,line join=round,line cap=round,fill=fillColor] (170.99,518.80) circle (  1.16);

\path[draw=drawColor,line width= 0.4pt,line join=round,line cap=round,fill=fillColor] (171.08,518.80) circle (  1.16);

\path[draw=drawColor,line width= 0.4pt,line join=round,line cap=round,fill=fillColor] (171.17,518.79) circle (  1.16);

\path[draw=drawColor,line width= 0.4pt,line join=round,line cap=round,fill=fillColor] (171.26,518.75) circle (  1.16);

\path[draw=drawColor,line width= 0.4pt,line join=round,line cap=round,fill=fillColor] (171.35,518.73) circle (  1.16);

\path[draw=drawColor,line width= 0.4pt,line join=round,line cap=round,fill=fillColor] (171.44,518.72) circle (  1.16);

\path[draw=drawColor,line width= 0.4pt,line join=round,line cap=round,fill=fillColor] (171.53,518.70) circle (  1.16);

\path[draw=drawColor,line width= 0.4pt,line join=round,line cap=round,fill=fillColor] (171.62,518.69) circle (  1.16);

\path[draw=drawColor,line width= 0.4pt,line join=round,line cap=round,fill=fillColor] (171.71,518.63) circle (  1.16);

\path[draw=drawColor,line width= 0.4pt,line join=round,line cap=round,fill=fillColor] (171.80,518.62) circle (  1.16);

\path[draw=drawColor,line width= 0.4pt,line join=round,line cap=round,fill=fillColor] (171.89,518.62) circle (  1.16);

\path[draw=drawColor,line width= 0.4pt,line join=round,line cap=round,fill=fillColor] (171.97,518.61) circle (  1.16);

\path[draw=drawColor,line width= 0.4pt,line join=round,line cap=round,fill=fillColor] (172.06,518.59) circle (  1.16);

\path[draw=drawColor,line width= 0.4pt,line join=round,line cap=round,fill=fillColor] (172.15,518.58) circle (  1.16);

\path[draw=drawColor,line width= 0.4pt,line join=round,line cap=round,fill=fillColor] (172.24,518.57) circle (  1.16);

\path[draw=drawColor,line width= 0.4pt,line join=round,line cap=round,fill=fillColor] (172.33,518.57) circle (  1.16);

\path[draw=drawColor,line width= 0.4pt,line join=round,line cap=round,fill=fillColor] (172.42,518.55) circle (  1.16);

\path[draw=drawColor,line width= 0.4pt,line join=round,line cap=round,fill=fillColor] (172.51,518.54) circle (  1.16);

\path[draw=drawColor,line width= 0.4pt,line join=round,line cap=round,fill=fillColor] (172.59,518.53) circle (  1.16);

\path[draw=drawColor,line width= 0.4pt,line join=round,line cap=round,fill=fillColor] (172.68,518.52) circle (  1.16);

\path[draw=drawColor,line width= 0.4pt,line join=round,line cap=round,fill=fillColor] (172.77,518.51) circle (  1.16);

\path[draw=drawColor,line width= 0.4pt,line join=round,line cap=round,fill=fillColor] (172.86,518.49) circle (  1.16);

\path[draw=drawColor,line width= 0.4pt,line join=round,line cap=round,fill=fillColor] (172.94,518.46) circle (  1.16);

\path[draw=drawColor,line width= 0.4pt,line join=round,line cap=round,fill=fillColor] (173.03,518.44) circle (  1.16);

\path[draw=drawColor,line width= 0.4pt,line join=round,line cap=round,fill=fillColor] (173.12,518.44) circle (  1.16);

\path[draw=drawColor,line width= 0.4pt,line join=round,line cap=round,fill=fillColor] (173.20,518.42) circle (  1.16);

\path[draw=drawColor,line width= 0.4pt,line join=round,line cap=round,fill=fillColor] (173.29,518.40) circle (  1.16);

\path[draw=drawColor,line width= 0.4pt,line join=round,line cap=round,fill=fillColor] (173.38,518.40) circle (  1.16);

\path[draw=drawColor,line width= 0.4pt,line join=round,line cap=round,fill=fillColor] (173.46,518.39) circle (  1.16);

\path[draw=drawColor,line width= 0.4pt,line join=round,line cap=round,fill=fillColor] (173.55,518.39) circle (  1.16);

\path[draw=drawColor,line width= 0.4pt,line join=round,line cap=round,fill=fillColor] (173.64,518.38) circle (  1.16);

\path[draw=drawColor,line width= 0.4pt,line join=round,line cap=round,fill=fillColor] (173.72,518.35) circle (  1.16);

\path[draw=drawColor,line width= 0.4pt,line join=round,line cap=round,fill=fillColor] (173.81,518.32) circle (  1.16);

\path[draw=drawColor,line width= 0.4pt,line join=round,line cap=round,fill=fillColor] (173.89,518.31) circle (  1.16);

\path[draw=drawColor,line width= 0.4pt,line join=round,line cap=round,fill=fillColor] (173.98,518.31) circle (  1.16);

\path[draw=drawColor,line width= 0.4pt,line join=round,line cap=round,fill=fillColor] (174.07,518.30) circle (  1.16);

\path[draw=drawColor,line width= 0.4pt,line join=round,line cap=round,fill=fillColor] (174.15,518.29) circle (  1.16);

\path[draw=drawColor,line width= 0.4pt,line join=round,line cap=round,fill=fillColor] (174.24,518.26) circle (  1.16);

\path[draw=drawColor,line width= 0.4pt,line join=round,line cap=round,fill=fillColor] (174.32,518.24) circle (  1.16);

\path[draw=drawColor,line width= 0.4pt,line join=round,line cap=round,fill=fillColor] (174.41,518.24) circle (  1.16);

\path[draw=drawColor,line width= 0.4pt,line join=round,line cap=round,fill=fillColor] (174.49,518.22) circle (  1.16);

\path[draw=drawColor,line width= 0.4pt,line join=round,line cap=round,fill=fillColor] (174.58,518.22) circle (  1.16);

\path[draw=drawColor,line width= 0.4pt,line join=round,line cap=round,fill=fillColor] (174.66,518.20) circle (  1.16);

\path[draw=drawColor,line width= 0.4pt,line join=round,line cap=round,fill=fillColor] (174.75,518.17) circle (  1.16);

\path[draw=drawColor,line width= 0.4pt,line join=round,line cap=round,fill=fillColor] (174.83,518.09) circle (  1.16);

\path[draw=drawColor,line width= 0.4pt,line join=round,line cap=round,fill=fillColor] (174.91,518.06) circle (  1.16);

\path[draw=drawColor,line width= 0.4pt,line join=round,line cap=round,fill=fillColor] (175.00,518.06) circle (  1.16);

\path[draw=drawColor,line width= 0.4pt,line join=round,line cap=round,fill=fillColor] (175.08,518.05) circle (  1.16);

\path[draw=drawColor,line width= 0.4pt,line join=round,line cap=round,fill=fillColor] (175.17,518.04) circle (  1.16);

\path[draw=drawColor,line width= 0.4pt,line join=round,line cap=round,fill=fillColor] (175.25,518.02) circle (  1.16);

\path[draw=drawColor,line width= 0.4pt,line join=round,line cap=round,fill=fillColor] (175.33,518.02) circle (  1.16);

\path[draw=drawColor,line width= 0.4pt,line join=round,line cap=round,fill=fillColor] (175.42,518.00) circle (  1.16);

\path[draw=drawColor,line width= 0.4pt,line join=round,line cap=round,fill=fillColor] (175.50,517.99) circle (  1.16);

\path[draw=drawColor,line width= 0.4pt,line join=round,line cap=round,fill=fillColor] (175.58,517.98) circle (  1.16);

\path[draw=drawColor,line width= 0.4pt,line join=round,line cap=round,fill=fillColor] (175.67,517.98) circle (  1.16);

\path[draw=drawColor,line width= 0.4pt,line join=round,line cap=round,fill=fillColor] (175.75,517.97) circle (  1.16);

\path[draw=drawColor,line width= 0.4pt,line join=round,line cap=round,fill=fillColor] (175.83,517.97) circle (  1.16);

\path[draw=drawColor,line width= 0.4pt,line join=round,line cap=round,fill=fillColor] (175.91,517.94) circle (  1.16);

\path[draw=drawColor,line width= 0.4pt,line join=round,line cap=round,fill=fillColor] (176.00,517.92) circle (  1.16);

\path[draw=drawColor,line width= 0.4pt,line join=round,line cap=round,fill=fillColor] (176.08,517.91) circle (  1.16);

\path[draw=drawColor,line width= 0.4pt,line join=round,line cap=round,fill=fillColor] (176.16,517.91) circle (  1.16);

\path[draw=drawColor,line width= 0.4pt,line join=round,line cap=round,fill=fillColor] (176.24,517.91) circle (  1.16);

\path[draw=drawColor,line width= 0.4pt,line join=round,line cap=round,fill=fillColor] (176.33,517.89) circle (  1.16);

\path[draw=drawColor,line width= 0.4pt,line join=round,line cap=round,fill=fillColor] (176.41,517.87) circle (  1.16);

\path[draw=drawColor,line width= 0.4pt,line join=round,line cap=round,fill=fillColor] (176.49,517.86) circle (  1.16);

\path[draw=drawColor,line width= 0.4pt,line join=round,line cap=round,fill=fillColor] (176.57,517.83) circle (  1.16);

\path[draw=drawColor,line width= 0.4pt,line join=round,line cap=round,fill=fillColor] (176.65,517.82) circle (  1.16);

\path[draw=drawColor,line width= 0.4pt,line join=round,line cap=round,fill=fillColor] (176.73,517.82) circle (  1.16);

\path[draw=drawColor,line width= 0.4pt,line join=round,line cap=round,fill=fillColor] (176.82,517.82) circle (  1.16);

\path[draw=drawColor,line width= 0.4pt,line join=round,line cap=round,fill=fillColor] (176.90,517.82) circle (  1.16);

\path[draw=drawColor,line width= 0.4pt,line join=round,line cap=round,fill=fillColor] (176.98,517.74) circle (  1.16);

\path[draw=drawColor,line width= 0.4pt,line join=round,line cap=round,fill=fillColor] (177.06,517.71) circle (  1.16);

\path[draw=drawColor,line width= 0.4pt,line join=round,line cap=round,fill=fillColor] (177.14,517.70) circle (  1.16);

\path[draw=drawColor,line width= 0.4pt,line join=round,line cap=round,fill=fillColor] (177.22,517.67) circle (  1.16);

\path[draw=drawColor,line width= 0.4pt,line join=round,line cap=round,fill=fillColor] (177.30,517.66) circle (  1.16);

\path[draw=drawColor,line width= 0.4pt,line join=round,line cap=round,fill=fillColor] (177.38,517.66) circle (  1.16);

\path[draw=drawColor,line width= 0.4pt,line join=round,line cap=round,fill=fillColor] (177.46,517.66) circle (  1.16);

\path[draw=drawColor,line width= 0.4pt,line join=round,line cap=round,fill=fillColor] (177.54,517.66) circle (  1.16);

\path[draw=drawColor,line width= 0.4pt,line join=round,line cap=round,fill=fillColor] (177.62,517.65) circle (  1.16);

\path[draw=drawColor,line width= 0.4pt,line join=round,line cap=round,fill=fillColor] (177.70,517.64) circle (  1.16);

\path[draw=drawColor,line width= 0.4pt,line join=round,line cap=round,fill=fillColor] (177.78,517.64) circle (  1.16);

\path[draw=drawColor,line width= 0.4pt,line join=round,line cap=round,fill=fillColor] (177.86,517.60) circle (  1.16);

\path[draw=drawColor,line width= 0.4pt,line join=round,line cap=round,fill=fillColor] (177.94,517.58) circle (  1.16);

\path[draw=drawColor,line width= 0.4pt,line join=round,line cap=round,fill=fillColor] (178.02,517.52) circle (  1.16);

\path[draw=drawColor,line width= 0.4pt,line join=round,line cap=round,fill=fillColor] (178.10,517.51) circle (  1.16);

\path[draw=drawColor,line width= 0.4pt,line join=round,line cap=round,fill=fillColor] (178.18,517.51) circle (  1.16);

\path[draw=drawColor,line width= 0.4pt,line join=round,line cap=round,fill=fillColor] (178.26,517.47) circle (  1.16);

\path[draw=drawColor,line width= 0.4pt,line join=round,line cap=round,fill=fillColor] (178.34,517.42) circle (  1.16);

\path[draw=drawColor,line width= 0.4pt,line join=round,line cap=round,fill=fillColor] (178.42,517.40) circle (  1.16);

\path[draw=drawColor,line width= 0.4pt,line join=round,line cap=round,fill=fillColor] (178.50,517.37) circle (  1.16);

\path[draw=drawColor,line width= 0.4pt,line join=round,line cap=round,fill=fillColor] (178.57,517.36) circle (  1.16);

\path[draw=drawColor,line width= 0.4pt,line join=round,line cap=round,fill=fillColor] (178.65,517.36) circle (  1.16);

\path[draw=drawColor,line width= 0.4pt,line join=round,line cap=round,fill=fillColor] (178.73,517.34) circle (  1.16);

\path[draw=drawColor,line width= 0.4pt,line join=round,line cap=round,fill=fillColor] (178.81,517.33) circle (  1.16);

\path[draw=drawColor,line width= 0.4pt,line join=round,line cap=round,fill=fillColor] (178.89,517.28) circle (  1.16);

\path[draw=drawColor,line width= 0.4pt,line join=round,line cap=round,fill=fillColor] (178.97,517.26) circle (  1.16);

\path[draw=drawColor,line width= 0.4pt,line join=round,line cap=round,fill=fillColor] (179.04,517.26) circle (  1.16);

\path[draw=drawColor,line width= 0.4pt,line join=round,line cap=round,fill=fillColor] (179.12,517.19) circle (  1.16);

\path[draw=drawColor,line width= 0.4pt,line join=round,line cap=round,fill=fillColor] (179.20,517.19) circle (  1.16);

\path[draw=drawColor,line width= 0.4pt,line join=round,line cap=round,fill=fillColor] (179.28,517.18) circle (  1.16);

\path[draw=drawColor,line width= 0.4pt,line join=round,line cap=round,fill=fillColor] (179.36,517.14) circle (  1.16);

\path[draw=drawColor,line width= 0.4pt,line join=round,line cap=round,fill=fillColor] (179.43,517.14) circle (  1.16);

\path[draw=drawColor,line width= 0.4pt,line join=round,line cap=round,fill=fillColor] (179.51,517.13) circle (  1.16);

\path[draw=drawColor,line width= 0.4pt,line join=round,line cap=round,fill=fillColor] (179.59,517.13) circle (  1.16);

\path[draw=drawColor,line width= 0.4pt,line join=round,line cap=round,fill=fillColor] (179.67,517.13) circle (  1.16);

\path[draw=drawColor,line width= 0.4pt,line join=round,line cap=round,fill=fillColor] (179.74,517.07) circle (  1.16);

\path[draw=drawColor,line width= 0.4pt,line join=round,line cap=round,fill=fillColor] (179.82,517.04) circle (  1.16);

\path[draw=drawColor,line width= 0.4pt,line join=round,line cap=round,fill=fillColor] (179.90,517.03) circle (  1.16);

\path[draw=drawColor,line width= 0.4pt,line join=round,line cap=round,fill=fillColor] (179.97,517.01) circle (  1.16);

\path[draw=drawColor,line width= 0.4pt,line join=round,line cap=round,fill=fillColor] (180.05,516.99) circle (  1.16);

\path[draw=drawColor,line width= 0.4pt,line join=round,line cap=round,fill=fillColor] (180.13,516.99) circle (  1.16);

\path[draw=drawColor,line width= 0.4pt,line join=round,line cap=round,fill=fillColor] (180.20,516.99) circle (  1.16);

\path[draw=drawColor,line width= 0.4pt,line join=round,line cap=round,fill=fillColor] (180.28,516.93) circle (  1.16);

\path[draw=drawColor,line width= 0.4pt,line join=round,line cap=round,fill=fillColor] (180.36,516.89) circle (  1.16);

\path[draw=drawColor,line width= 0.4pt,line join=round,line cap=round,fill=fillColor] (180.43,516.86) circle (  1.16);

\path[draw=drawColor,line width= 0.4pt,line join=round,line cap=round,fill=fillColor] (180.51,516.85) circle (  1.16);

\path[draw=drawColor,line width= 0.4pt,line join=round,line cap=round,fill=fillColor] (180.58,516.84) circle (  1.16);

\path[draw=drawColor,line width= 0.4pt,line join=round,line cap=round,fill=fillColor] (180.66,516.83) circle (  1.16);

\path[draw=drawColor,line width= 0.4pt,line join=round,line cap=round,fill=fillColor] (180.74,516.82) circle (  1.16);

\path[draw=drawColor,line width= 0.4pt,line join=round,line cap=round,fill=fillColor] (180.81,516.80) circle (  1.16);

\path[draw=drawColor,line width= 0.4pt,line join=round,line cap=round,fill=fillColor] (180.89,516.80) circle (  1.16);

\path[draw=drawColor,line width= 0.4pt,line join=round,line cap=round,fill=fillColor] (180.96,516.78) circle (  1.16);

\path[draw=drawColor,line width= 0.4pt,line join=round,line cap=round,fill=fillColor] (181.04,516.78) circle (  1.16);

\path[draw=drawColor,line width= 0.4pt,line join=round,line cap=round,fill=fillColor] (181.11,516.74) circle (  1.16);

\path[draw=drawColor,line width= 0.4pt,line join=round,line cap=round,fill=fillColor] (181.19,516.74) circle (  1.16);

\path[draw=drawColor,line width= 0.4pt,line join=round,line cap=round,fill=fillColor] (181.26,516.71) circle (  1.16);

\path[draw=drawColor,line width= 0.4pt,line join=round,line cap=round,fill=fillColor] (181.34,516.71) circle (  1.16);

\path[draw=drawColor,line width= 0.4pt,line join=round,line cap=round,fill=fillColor] (181.41,516.68) circle (  1.16);

\path[draw=drawColor,line width= 0.4pt,line join=round,line cap=round,fill=fillColor] (181.49,516.63) circle (  1.16);

\path[draw=drawColor,line width= 0.4pt,line join=round,line cap=round,fill=fillColor] (181.56,516.63) circle (  1.16);

\path[draw=drawColor,line width= 0.4pt,line join=round,line cap=round,fill=fillColor] (181.64,516.62) circle (  1.16);

\path[draw=drawColor,line width= 0.4pt,line join=round,line cap=round,fill=fillColor] (181.71,516.62) circle (  1.16);

\path[draw=drawColor,line width= 0.4pt,line join=round,line cap=round,fill=fillColor] (181.79,516.59) circle (  1.16);

\path[draw=drawColor,line width= 0.4pt,line join=round,line cap=round,fill=fillColor] (181.86,516.58) circle (  1.16);

\path[draw=drawColor,line width= 0.4pt,line join=round,line cap=round,fill=fillColor] (181.94,516.56) circle (  1.16);

\path[draw=drawColor,line width= 0.4pt,line join=round,line cap=round,fill=fillColor] (182.01,516.55) circle (  1.16);

\path[draw=drawColor,line width= 0.4pt,line join=round,line cap=round,fill=fillColor] (182.08,516.54) circle (  1.16);

\path[draw=drawColor,line width= 0.4pt,line join=round,line cap=round,fill=fillColor] (182.16,516.54) circle (  1.16);

\path[draw=drawColor,line width= 0.4pt,line join=round,line cap=round,fill=fillColor] (182.23,516.51) circle (  1.16);

\path[draw=drawColor,line width= 0.4pt,line join=round,line cap=round,fill=fillColor] (182.31,516.50) circle (  1.16);

\path[draw=drawColor,line width= 0.4pt,line join=round,line cap=round,fill=fillColor] (182.38,516.49) circle (  1.16);

\path[draw=drawColor,line width= 0.4pt,line join=round,line cap=round,fill=fillColor] (182.45,516.48) circle (  1.16);

\path[draw=drawColor,line width= 0.4pt,line join=round,line cap=round,fill=fillColor] (182.53,516.48) circle (  1.16);

\path[draw=drawColor,line width= 0.4pt,line join=round,line cap=round,fill=fillColor] (182.60,516.46) circle (  1.16);

\path[draw=drawColor,line width= 0.4pt,line join=round,line cap=round,fill=fillColor] (182.67,516.46) circle (  1.16);

\path[draw=drawColor,line width= 0.4pt,line join=round,line cap=round,fill=fillColor] (182.75,516.45) circle (  1.16);

\path[draw=drawColor,line width= 0.4pt,line join=round,line cap=round,fill=fillColor] (182.82,516.44) circle (  1.16);

\path[draw=drawColor,line width= 0.4pt,line join=round,line cap=round,fill=fillColor] (182.89,516.43) circle (  1.16);

\path[draw=drawColor,line width= 0.4pt,line join=round,line cap=round,fill=fillColor] (182.96,516.41) circle (  1.16);

\path[draw=drawColor,line width= 0.4pt,line join=round,line cap=round,fill=fillColor] (183.04,516.40) circle (  1.16);

\path[draw=drawColor,line width= 0.4pt,line join=round,line cap=round,fill=fillColor] (183.11,516.37) circle (  1.16);

\path[draw=drawColor,line width= 0.4pt,line join=round,line cap=round,fill=fillColor] (183.18,516.33) circle (  1.16);

\path[draw=drawColor,line width= 0.4pt,line join=round,line cap=round,fill=fillColor] (183.26,516.31) circle (  1.16);

\path[draw=drawColor,line width= 0.4pt,line join=round,line cap=round,fill=fillColor] (183.33,516.25) circle (  1.16);

\path[draw=drawColor,line width= 0.4pt,line join=round,line cap=round,fill=fillColor] (183.40,516.25) circle (  1.16);

\path[draw=drawColor,line width= 0.4pt,line join=round,line cap=round,fill=fillColor] (183.47,516.12) circle (  1.16);

\path[draw=drawColor,line width= 0.4pt,line join=round,line cap=round,fill=fillColor] (183.54,516.10) circle (  1.16);

\path[draw=drawColor,line width= 0.4pt,line join=round,line cap=round,fill=fillColor] (183.62,516.10) circle (  1.16);

\path[draw=drawColor,line width= 0.4pt,line join=round,line cap=round,fill=fillColor] (183.69,516.01) circle (  1.16);

\path[draw=drawColor,line width= 0.4pt,line join=round,line cap=round,fill=fillColor] (183.76,515.96) circle (  1.16);

\path[draw=drawColor,line width= 0.4pt,line join=round,line cap=round,fill=fillColor] (183.83,515.94) circle (  1.16);

\path[draw=drawColor,line width= 0.4pt,line join=round,line cap=round,fill=fillColor] (183.90,515.91) circle (  1.16);

\path[draw=drawColor,line width= 0.4pt,line join=round,line cap=round,fill=fillColor] (183.98,515.90) circle (  1.16);

\path[draw=drawColor,line width= 0.4pt,line join=round,line cap=round,fill=fillColor] (184.05,515.90) circle (  1.16);

\path[draw=drawColor,line width= 0.4pt,line join=round,line cap=round,fill=fillColor] (184.12,515.89) circle (  1.16);

\path[draw=drawColor,line width= 0.4pt,line join=round,line cap=round,fill=fillColor] (184.19,515.88) circle (  1.16);

\path[draw=drawColor,line width= 0.4pt,line join=round,line cap=round,fill=fillColor] (184.26,515.72) circle (  1.16);

\path[draw=drawColor,line width= 0.4pt,line join=round,line cap=round,fill=fillColor] (184.33,515.71) circle (  1.16);

\path[draw=drawColor,line width= 0.4pt,line join=round,line cap=round,fill=fillColor] (184.40,515.69) circle (  1.16);

\path[draw=drawColor,line width= 0.4pt,line join=round,line cap=round,fill=fillColor] (184.48,515.68) circle (  1.16);

\path[draw=drawColor,line width= 0.4pt,line join=round,line cap=round,fill=fillColor] (184.55,515.68) circle (  1.16);

\path[draw=drawColor,line width= 0.4pt,line join=round,line cap=round,fill=fillColor] (184.62,515.66) circle (  1.16);

\path[draw=drawColor,line width= 0.4pt,line join=round,line cap=round,fill=fillColor] (184.69,515.65) circle (  1.16);

\path[draw=drawColor,line width= 0.4pt,line join=round,line cap=round,fill=fillColor] (184.76,515.60) circle (  1.16);

\path[draw=drawColor,line width= 0.4pt,line join=round,line cap=round,fill=fillColor] (184.83,515.59) circle (  1.16);

\path[draw=drawColor,line width= 0.4pt,line join=round,line cap=round,fill=fillColor] (184.90,515.56) circle (  1.16);

\path[draw=drawColor,line width= 0.4pt,line join=round,line cap=round,fill=fillColor] (184.97,515.52) circle (  1.16);

\path[draw=drawColor,line width= 0.4pt,line join=round,line cap=round,fill=fillColor] (185.04,515.48) circle (  1.16);

\path[draw=drawColor,line width= 0.4pt,line join=round,line cap=round,fill=fillColor] (185.11,515.45) circle (  1.16);

\path[draw=drawColor,line width= 0.4pt,line join=round,line cap=round,fill=fillColor] (185.18,515.40) circle (  1.16);

\path[draw=drawColor,line width= 0.4pt,line join=round,line cap=round,fill=fillColor] (185.25,515.39) circle (  1.16);

\path[draw=drawColor,line width= 0.4pt,line join=round,line cap=round,fill=fillColor] (185.32,515.37) circle (  1.16);

\path[draw=drawColor,line width= 0.4pt,line join=round,line cap=round,fill=fillColor] (185.39,515.37) circle (  1.16);

\path[draw=drawColor,line width= 0.4pt,line join=round,line cap=round,fill=fillColor] (185.46,515.36) circle (  1.16);

\path[draw=drawColor,line width= 0.4pt,line join=round,line cap=round,fill=fillColor] (185.53,515.35) circle (  1.16);

\path[draw=drawColor,line width= 0.4pt,line join=round,line cap=round,fill=fillColor] (185.60,515.35) circle (  1.16);

\path[draw=drawColor,line width= 0.4pt,line join=round,line cap=round,fill=fillColor] (185.67,515.29) circle (  1.16);

\path[draw=drawColor,line width= 0.4pt,line join=round,line cap=round,fill=fillColor] (185.74,515.25) circle (  1.16);

\path[draw=drawColor,line width= 0.4pt,line join=round,line cap=round,fill=fillColor] (185.81,515.24) circle (  1.16);

\path[draw=drawColor,line width= 0.4pt,line join=round,line cap=round,fill=fillColor] (185.88,515.22) circle (  1.16);

\path[draw=drawColor,line width= 0.4pt,line join=round,line cap=round,fill=fillColor] (185.95,515.17) circle (  1.16);

\path[draw=drawColor,line width= 0.4pt,line join=round,line cap=round,fill=fillColor] (186.02,515.15) circle (  1.16);

\path[draw=drawColor,line width= 0.4pt,line join=round,line cap=round,fill=fillColor] (186.09,515.15) circle (  1.16);

\path[draw=drawColor,line width= 0.4pt,line join=round,line cap=round,fill=fillColor] (186.16,515.10) circle (  1.16);

\path[draw=drawColor,line width= 0.4pt,line join=round,line cap=round,fill=fillColor] (186.22,515.10) circle (  1.16);

\path[draw=drawColor,line width= 0.4pt,line join=round,line cap=round,fill=fillColor] (186.29,515.06) circle (  1.16);

\path[draw=drawColor,line width= 0.4pt,line join=round,line cap=round,fill=fillColor] (186.36,515.04) circle (  1.16);

\path[draw=drawColor,line width= 0.4pt,line join=round,line cap=round,fill=fillColor] (186.43,514.97) circle (  1.16);

\path[draw=drawColor,line width= 0.4pt,line join=round,line cap=round,fill=fillColor] (186.50,514.94) circle (  1.16);

\path[draw=drawColor,line width= 0.4pt,line join=round,line cap=round,fill=fillColor] (186.57,514.93) circle (  1.16);

\path[draw=drawColor,line width= 0.4pt,line join=round,line cap=round,fill=fillColor] (186.64,514.86) circle (  1.16);

\path[draw=drawColor,line width= 0.4pt,line join=round,line cap=round,fill=fillColor] (186.70,514.86) circle (  1.16);

\path[draw=drawColor,line width= 0.4pt,line join=round,line cap=round,fill=fillColor] (186.77,514.85) circle (  1.16);

\path[draw=drawColor,line width= 0.4pt,line join=round,line cap=round,fill=fillColor] (186.84,514.83) circle (  1.16);

\path[draw=drawColor,line width= 0.4pt,line join=round,line cap=round,fill=fillColor] (186.91,514.82) circle (  1.16);

\path[draw=drawColor,line width= 0.4pt,line join=round,line cap=round,fill=fillColor] (186.98,514.81) circle (  1.16);

\path[draw=drawColor,line width= 0.4pt,line join=round,line cap=round,fill=fillColor] (187.05,514.79) circle (  1.16);

\path[draw=drawColor,line width= 0.4pt,line join=round,line cap=round,fill=fillColor] (187.11,514.76) circle (  1.16);

\path[draw=drawColor,line width= 0.4pt,line join=round,line cap=round,fill=fillColor] (187.18,514.75) circle (  1.16);

\path[draw=drawColor,line width= 0.4pt,line join=round,line cap=round,fill=fillColor] (187.25,514.71) circle (  1.16);

\path[draw=drawColor,line width= 0.4pt,line join=round,line cap=round,fill=fillColor] (187.32,514.71) circle (  1.16);

\path[draw=drawColor,line width= 0.4pt,line join=round,line cap=round,fill=fillColor] (187.38,514.66) circle (  1.16);

\path[draw=drawColor,line width= 0.4pt,line join=round,line cap=round,fill=fillColor] (187.45,514.66) circle (  1.16);

\path[draw=drawColor,line width= 0.4pt,line join=round,line cap=round,fill=fillColor] (187.52,514.59) circle (  1.16);

\path[draw=drawColor,line width= 0.4pt,line join=round,line cap=round,fill=fillColor] (187.59,514.58) circle (  1.16);

\path[draw=drawColor,line width= 0.4pt,line join=round,line cap=round,fill=fillColor] (187.65,514.56) circle (  1.16);

\path[draw=drawColor,line width= 0.4pt,line join=round,line cap=round,fill=fillColor] (187.72,514.56) circle (  1.16);

\path[draw=drawColor,line width= 0.4pt,line join=round,line cap=round,fill=fillColor] (187.79,514.49) circle (  1.16);

\path[draw=drawColor,line width= 0.4pt,line join=round,line cap=round,fill=fillColor] (187.86,514.43) circle (  1.16);

\path[draw=drawColor,line width= 0.4pt,line join=round,line cap=round,fill=fillColor] (187.92,514.43) circle (  1.16);

\path[draw=drawColor,line width= 0.4pt,line join=round,line cap=round,fill=fillColor] (187.99,514.30) circle (  1.16);

\path[draw=drawColor,line width= 0.4pt,line join=round,line cap=round,fill=fillColor] (188.06,514.24) circle (  1.16);

\path[draw=drawColor,line width= 0.4pt,line join=round,line cap=round,fill=fillColor] (188.12,514.19) circle (  1.16);

\path[draw=drawColor,line width= 0.4pt,line join=round,line cap=round,fill=fillColor] (188.19,514.18) circle (  1.16);

\path[draw=drawColor,line width= 0.4pt,line join=round,line cap=round,fill=fillColor] (188.26,514.07) circle (  1.16);

\path[draw=drawColor,line width= 0.4pt,line join=round,line cap=round,fill=fillColor] (188.32,514.04) circle (  1.16);

\path[draw=drawColor,line width= 0.4pt,line join=round,line cap=round,fill=fillColor] (188.39,513.97) circle (  1.16);

\path[draw=drawColor,line width= 0.4pt,line join=round,line cap=round,fill=fillColor] (188.46,513.95) circle (  1.16);

\path[draw=drawColor,line width= 0.4pt,line join=round,line cap=round,fill=fillColor] (188.52,513.93) circle (  1.16);

\path[draw=drawColor,line width= 0.4pt,line join=round,line cap=round,fill=fillColor] (188.59,513.91) circle (  1.16);

\path[draw=drawColor,line width= 0.4pt,line join=round,line cap=round,fill=fillColor] (188.66,513.88) circle (  1.16);

\path[draw=drawColor,line width= 0.4pt,line join=round,line cap=round,fill=fillColor] (188.72,513.76) circle (  1.16);

\path[draw=drawColor,line width= 0.4pt,line join=round,line cap=round,fill=fillColor] (188.79,513.75) circle (  1.16);

\path[draw=drawColor,line width= 0.4pt,line join=round,line cap=round,fill=fillColor] (188.85,513.70) circle (  1.16);

\path[draw=drawColor,line width= 0.4pt,line join=round,line cap=round,fill=fillColor] (188.92,513.68) circle (  1.16);

\path[draw=drawColor,line width= 0.4pt,line join=round,line cap=round,fill=fillColor] (188.99,513.61) circle (  1.16);

\path[draw=drawColor,line width= 0.4pt,line join=round,line cap=round,fill=fillColor] (189.05,513.56) circle (  1.16);

\path[draw=drawColor,line width= 0.4pt,line join=round,line cap=round,fill=fillColor] (189.12,513.49) circle (  1.16);

\path[draw=drawColor,line width= 0.4pt,line join=round,line cap=round,fill=fillColor] (189.18,513.38) circle (  1.16);

\path[draw=drawColor,line width= 0.4pt,line join=round,line cap=round,fill=fillColor] (189.25,513.36) circle (  1.16);

\path[draw=drawColor,line width= 0.4pt,line join=round,line cap=round,fill=fillColor] (189.31,513.28) circle (  1.16);

\path[draw=drawColor,line width= 0.4pt,line join=round,line cap=round,fill=fillColor] (189.38,513.24) circle (  1.16);

\path[draw=drawColor,line width= 0.4pt,line join=round,line cap=round,fill=fillColor] (189.45,513.18) circle (  1.16);

\path[draw=drawColor,line width= 0.4pt,line join=round,line cap=round,fill=fillColor] (189.51,513.15) circle (  1.16);

\path[draw=drawColor,line width= 0.4pt,line join=round,line cap=round,fill=fillColor] (189.58,513.09) circle (  1.16);

\path[draw=drawColor,line width= 0.4pt,line join=round,line cap=round,fill=fillColor] (189.64,512.95) circle (  1.16);

\path[draw=drawColor,line width= 0.4pt,line join=round,line cap=round,fill=fillColor] (189.71,512.93) circle (  1.16);

\path[draw=drawColor,line width= 0.4pt,line join=round,line cap=round,fill=fillColor] (189.77,512.87) circle (  1.16);

\path[draw=drawColor,line width= 0.4pt,line join=round,line cap=round,fill=fillColor] (189.84,512.78) circle (  1.16);

\path[draw=drawColor,line width= 0.4pt,line join=round,line cap=round,fill=fillColor] (189.90,512.65) circle (  1.16);

\path[draw=drawColor,line width= 0.4pt,line join=round,line cap=round,fill=fillColor] (189.97,512.62) circle (  1.16);

\path[draw=drawColor,line width= 0.4pt,line join=round,line cap=round,fill=fillColor] (190.03,512.55) circle (  1.16);

\path[draw=drawColor,line width= 0.4pt,line join=round,line cap=round,fill=fillColor] (190.10,512.47) circle (  1.16);

\path[draw=drawColor,line width= 0.4pt,line join=round,line cap=round,fill=fillColor] (190.16,512.44) circle (  1.16);

\path[draw=drawColor,line width= 0.4pt,line join=round,line cap=round,fill=fillColor] (190.23,512.40) circle (  1.16);

\path[draw=drawColor,line width= 0.4pt,line join=round,line cap=round,fill=fillColor] (190.29,512.40) circle (  1.16);

\path[draw=drawColor,line width= 0.4pt,line join=round,line cap=round,fill=fillColor] (190.35,512.26) circle (  1.16);

\path[draw=drawColor,line width= 0.4pt,line join=round,line cap=round,fill=fillColor] (190.42,512.12) circle (  1.16);

\path[draw=drawColor,line width= 0.4pt,line join=round,line cap=round,fill=fillColor] (190.48,512.07) circle (  1.16);

\path[draw=drawColor,line width= 0.4pt,line join=round,line cap=round,fill=fillColor] (190.55,512.05) circle (  1.16);

\path[draw=drawColor,line width= 0.4pt,line join=round,line cap=round,fill=fillColor] (190.61,512.00) circle (  1.16);

\path[draw=drawColor,line width= 0.4pt,line join=round,line cap=round,fill=fillColor] (190.68,511.91) circle (  1.16);

\path[draw=drawColor,line width= 0.4pt,line join=round,line cap=round,fill=fillColor] (190.74,511.81) circle (  1.16);

\path[draw=drawColor,line width= 0.4pt,line join=round,line cap=round,fill=fillColor] (190.80,511.68) circle (  1.16);

\path[draw=drawColor,line width= 0.4pt,line join=round,line cap=round,fill=fillColor] (190.87,511.54) circle (  1.16);

\path[draw=drawColor,line width= 0.4pt,line join=round,line cap=round,fill=fillColor] (190.93,511.43) circle (  1.16);

\path[draw=drawColor,line width= 0.4pt,line join=round,line cap=round,fill=fillColor] (190.99,511.33) circle (  1.16);

\path[draw=drawColor,line width= 0.4pt,line join=round,line cap=round,fill=fillColor] (191.06,511.30) circle (  1.16);

\path[draw=drawColor,line width= 0.4pt,line join=round,line cap=round,fill=fillColor] (191.12,511.02) circle (  1.16);

\path[draw=drawColor,line width= 0.4pt,line join=round,line cap=round,fill=fillColor] (191.19,510.96) circle (  1.16);

\path[draw=drawColor,line width= 0.4pt,line join=round,line cap=round,fill=fillColor] (191.25,508.31) circle (  1.16);

\path[draw=drawColor,line width= 0.4pt,line join=round,line cap=round,fill=fillColor] (191.31,508.31) circle (  1.16);

\path[draw=drawColor,line width= 0.4pt,line join=round,line cap=round,fill=fillColor] (191.38,508.31) circle (  1.16);

\path[draw=drawColor,line width= 0.4pt,line join=round,line cap=round,fill=fillColor] (191.44,508.31) circle (  1.16);

\path[draw=drawColor,line width= 0.4pt,line join=round,line cap=round,fill=fillColor] (191.50,508.31) circle (  1.16);

\path[draw=drawColor,line width= 0.4pt,line join=round,line cap=round,fill=fillColor] (191.57,508.31) circle (  1.16);

\path[draw=drawColor,line width= 0.4pt,line join=round,line cap=round,fill=fillColor] (191.63,508.31) circle (  1.16);

\path[draw=drawColor,line width= 0.4pt,line join=round,line cap=round,fill=fillColor] (191.69,508.31) circle (  1.16);

\path[draw=drawColor,line width= 0.4pt,line join=round,line cap=round,fill=fillColor] (191.75,508.31) circle (  1.16);

\path[draw=drawColor,line width= 0.4pt,line join=round,line cap=round,fill=fillColor] (191.82,508.31) circle (  1.16);

\path[draw=drawColor,line width= 0.4pt,line join=round,line cap=round,fill=fillColor] (191.88,508.31) circle (  1.16);

\path[draw=drawColor,line width= 0.4pt,line join=round,line cap=round,fill=fillColor] (191.94,508.31) circle (  1.16);

\path[draw=drawColor,line width= 0.4pt,line join=round,line cap=round,fill=fillColor] (192.01,508.31) circle (  1.16);

\path[draw=drawColor,line width= 0.4pt,line join=round,line cap=round,fill=fillColor] (192.07,508.31) circle (  1.16);

\path[draw=drawColor,line width= 0.4pt,line join=round,line cap=round,fill=fillColor] (192.13,508.31) circle (  1.16);

\path[draw=drawColor,line width= 0.4pt,line join=round,line cap=round,fill=fillColor] (192.19,508.31) circle (  1.16);

\path[draw=drawColor,line width= 0.4pt,line join=round,line cap=round,fill=fillColor] (192.26,508.31) circle (  1.16);

\path[draw=drawColor,line width= 0.4pt,line join=round,line cap=round,fill=fillColor] (192.32,508.31) circle (  1.16);

\path[draw=drawColor,line width= 0.4pt,line join=round,line cap=round,fill=fillColor] (192.38,508.31) circle (  1.16);

\path[draw=drawColor,line width= 0.4pt,line join=round,line cap=round,fill=fillColor] (192.44,508.31) circle (  1.16);

\path[draw=drawColor,line width= 0.4pt,line join=round,line cap=round,fill=fillColor] (192.51,508.31) circle (  1.16);

\path[draw=drawColor,line width= 0.4pt,line join=round,line cap=round,fill=fillColor] (192.57,508.31) circle (  1.16);

\path[draw=drawColor,line width= 0.4pt,line join=round,line cap=round,fill=fillColor] (192.63,508.31) circle (  1.16);

\path[draw=drawColor,line width= 0.4pt,line join=round,line cap=round,fill=fillColor] (192.69,508.31) circle (  1.16);

\path[draw=drawColor,line width= 0.4pt,line join=round,line cap=round,fill=fillColor] (192.75,508.31) circle (  1.16);

\path[draw=drawColor,line width= 0.4pt,line join=round,line cap=round,fill=fillColor] (192.82,508.31) circle (  1.16);

\path[draw=drawColor,line width= 0.4pt,line join=round,line cap=round,fill=fillColor] (192.88,508.31) circle (  1.16);

\path[draw=drawColor,line width= 0.4pt,line join=round,line cap=round,fill=fillColor] (192.94,508.31) circle (  1.16);

\path[draw=drawColor,line width= 0.4pt,line join=round,line cap=round,fill=fillColor] (193.00,508.31) circle (  1.16);

\path[draw=drawColor,line width= 0.4pt,line join=round,line cap=round,fill=fillColor] (193.06,508.31) circle (  1.16);

\path[draw=drawColor,line width= 0.4pt,line join=round,line cap=round,fill=fillColor] (193.13,508.31) circle (  1.16);

\path[draw=drawColor,line width= 0.4pt,line join=round,line cap=round,fill=fillColor] (193.19,508.31) circle (  1.16);

\path[draw=drawColor,line width= 0.4pt,line join=round,line cap=round,fill=fillColor] (193.25,508.31) circle (  1.16);

\path[draw=drawColor,line width= 0.4pt,line join=round,line cap=round,fill=fillColor] (193.31,508.31) circle (  1.16);

\path[draw=drawColor,line width= 0.4pt,line join=round,line cap=round,fill=fillColor] (193.37,508.31) circle (  1.16);

\path[draw=drawColor,line width= 0.4pt,line join=round,line cap=round,fill=fillColor] (193.43,508.31) circle (  1.16);

\path[draw=drawColor,line width= 0.4pt,line join=round,line cap=round,fill=fillColor] (193.49,508.31) circle (  1.16);

\path[draw=drawColor,line width= 0.4pt,line join=round,line cap=round,fill=fillColor] (193.55,508.31) circle (  1.16);

\path[draw=drawColor,line width= 0.4pt,line join=round,line cap=round,fill=fillColor] (193.62,508.31) circle (  1.16);

\path[draw=drawColor,line width= 0.4pt,line join=round,line cap=round,fill=fillColor] (193.68,508.31) circle (  1.16);

\path[draw=drawColor,line width= 0.4pt,line join=round,line cap=round,fill=fillColor] (193.74,508.31) circle (  1.16);

\path[draw=drawColor,line width= 0.4pt,line join=round,line cap=round,fill=fillColor] (193.80,508.31) circle (  1.16);

\path[draw=drawColor,line width= 0.4pt,line join=round,line cap=round,fill=fillColor] (193.86,508.31) circle (  1.16);

\path[draw=drawColor,line width= 0.4pt,line join=round,line cap=round,fill=fillColor] (193.92,508.31) circle (  1.16);

\path[draw=drawColor,line width= 0.4pt,line join=round,line cap=round,fill=fillColor] (193.98,508.31) circle (  1.16);

\path[draw=drawColor,line width= 0.4pt,line join=round,line cap=round,fill=fillColor] (194.04,508.31) circle (  1.16);

\path[draw=drawColor,line width= 0.4pt,line join=round,line cap=round,fill=fillColor] (194.10,508.31) circle (  1.16);

\path[draw=drawColor,line width= 0.4pt,line join=round,line cap=round,fill=fillColor] (194.16,508.31) circle (  1.16);

\path[draw=drawColor,line width= 0.4pt,line join=round,line cap=round,fill=fillColor] (194.22,508.31) circle (  1.16);

\path[draw=drawColor,line width= 0.4pt,line join=round,line cap=round,fill=fillColor] (194.29,508.31) circle (  1.16);

\path[draw=drawColor,line width= 0.4pt,line join=round,line cap=round,fill=fillColor] (194.35,508.31) circle (  1.16);

\path[draw=drawColor,line width= 0.4pt,line join=round,line cap=round,fill=fillColor] (194.41,508.31) circle (  1.16);

\path[draw=drawColor,line width= 0.4pt,line join=round,line cap=round,fill=fillColor] (194.47,508.31) circle (  1.16);

\path[draw=drawColor,line width= 0.4pt,line join=round,line cap=round,fill=fillColor] (194.53,508.31) circle (  1.16);

\path[draw=drawColor,line width= 0.4pt,line join=round,line cap=round,fill=fillColor] (194.59,508.31) circle (  1.16);

\path[draw=drawColor,line width= 0.4pt,line join=round,line cap=round,fill=fillColor] (194.65,508.31) circle (  1.16);

\path[draw=drawColor,line width= 0.4pt,line join=round,line cap=round,fill=fillColor] (194.71,508.31) circle (  1.16);

\path[draw=drawColor,line width= 0.4pt,line join=round,line cap=round,fill=fillColor] (194.77,508.31) circle (  1.16);

\path[draw=drawColor,line width= 0.4pt,line join=round,line cap=round,fill=fillColor] (194.83,508.31) circle (  1.16);

\path[draw=drawColor,line width= 0.4pt,line join=round,line cap=round,fill=fillColor] (194.89,508.31) circle (  1.16);

\path[draw=drawColor,line width= 0.4pt,line join=round,line cap=round,fill=fillColor] (194.95,508.31) circle (  1.16);

\path[draw=drawColor,line width= 0.4pt,line join=round,line cap=round,fill=fillColor] (195.01,508.31) circle (  1.16);

\path[draw=drawColor,line width= 0.4pt,line join=round,line cap=round,fill=fillColor] (195.07,508.31) circle (  1.16);

\path[draw=drawColor,line width= 0.4pt,line join=round,line cap=round,fill=fillColor] (195.13,508.31) circle (  1.16);

\path[draw=drawColor,line width= 0.4pt,line join=round,line cap=round,fill=fillColor] (195.19,508.31) circle (  1.16);

\path[draw=drawColor,line width= 0.4pt,line join=round,line cap=round,fill=fillColor] (195.25,508.31) circle (  1.16);

\path[draw=drawColor,line width= 0.4pt,line join=round,line cap=round,fill=fillColor] (195.31,508.31) circle (  1.16);

\path[draw=drawColor,line width= 0.4pt,line join=round,line cap=round,fill=fillColor] (195.36,508.31) circle (  1.16);

\path[draw=drawColor,line width= 0.4pt,line join=round,line cap=round,fill=fillColor] (195.42,508.31) circle (  1.16);

\path[draw=drawColor,line width= 0.4pt,line join=round,line cap=round,fill=fillColor] (195.48,508.31) circle (  1.16);

\path[draw=drawColor,line width= 0.4pt,line join=round,line cap=round,fill=fillColor] (195.54,508.31) circle (  1.16);

\path[draw=drawColor,line width= 0.4pt,line join=round,line cap=round,fill=fillColor] (195.60,508.31) circle (  1.16);

\path[draw=drawColor,line width= 0.4pt,line join=round,line cap=round,fill=fillColor] (195.66,508.31) circle (  1.16);

\path[draw=drawColor,line width= 0.4pt,line join=round,line cap=round,fill=fillColor] (195.72,508.31) circle (  1.16);

\path[draw=drawColor,line width= 0.4pt,line join=round,line cap=round,fill=fillColor] (195.78,508.31) circle (  1.16);

\path[draw=drawColor,line width= 0.4pt,line join=round,line cap=round,fill=fillColor] (195.84,508.31) circle (  1.16);

\path[draw=drawColor,line width= 0.4pt,line join=round,line cap=round,fill=fillColor] (195.90,508.31) circle (  1.16);

\path[draw=drawColor,line width= 0.4pt,line join=round,line cap=round,fill=fillColor] (195.96,508.31) circle (  1.16);

\path[draw=drawColor,line width= 0.4pt,line join=round,line cap=round,fill=fillColor] (196.02,508.31) circle (  1.16);

\path[draw=drawColor,line width= 0.4pt,line join=round,line cap=round,fill=fillColor] (196.07,508.31) circle (  1.16);

\path[draw=drawColor,line width= 0.4pt,line join=round,line cap=round,fill=fillColor] (196.13,508.31) circle (  1.16);

\path[draw=drawColor,line width= 0.4pt,line join=round,line cap=round,fill=fillColor] (196.19,508.31) circle (  1.16);

\path[draw=drawColor,line width= 0.4pt,line join=round,line cap=round,fill=fillColor] (196.25,508.31) circle (  1.16);

\path[draw=drawColor,line width= 0.4pt,line join=round,line cap=round,fill=fillColor] (196.31,508.31) circle (  1.16);

\path[draw=drawColor,line width= 0.4pt,line join=round,line cap=round,fill=fillColor] (196.37,508.31) circle (  1.16);

\path[draw=drawColor,line width= 0.4pt,line join=round,line cap=round,fill=fillColor] (196.43,508.31) circle (  1.16);

\path[draw=drawColor,line width= 0.4pt,line join=round,line cap=round,fill=fillColor] (196.48,508.31) circle (  1.16);

\path[draw=drawColor,line width= 0.4pt,line join=round,line cap=round,fill=fillColor] (196.54,508.31) circle (  1.16);

\path[draw=drawColor,line width= 0.4pt,line join=round,line cap=round,fill=fillColor] (196.60,508.31) circle (  1.16);

\path[draw=drawColor,line width= 0.4pt,line join=round,line cap=round,fill=fillColor] (196.66,508.31) circle (  1.16);

\path[draw=drawColor,line width= 0.4pt,line join=round,line cap=round,fill=fillColor] (196.72,508.31) circle (  1.16);

\path[draw=drawColor,line width= 0.4pt,line join=round,line cap=round,fill=fillColor] (196.78,508.31) circle (  1.16);

\path[draw=drawColor,line width= 0.4pt,line join=round,line cap=round,fill=fillColor] (196.83,508.31) circle (  1.16);

\path[draw=drawColor,line width= 0.4pt,line join=round,line cap=round,fill=fillColor] (196.89,508.31) circle (  1.16);

\path[draw=drawColor,line width= 0.4pt,line join=round,line cap=round,fill=fillColor] (196.95,508.31) circle (  1.16);

\path[draw=drawColor,line width= 0.4pt,line join=round,line cap=round,fill=fillColor] (197.01,508.31) circle (  1.16);

\path[draw=drawColor,line width= 0.4pt,line join=round,line cap=round,fill=fillColor] (197.07,508.31) circle (  1.16);

\path[draw=drawColor,line width= 0.4pt,line join=round,line cap=round,fill=fillColor] (197.12,508.31) circle (  1.16);

\path[draw=drawColor,line width= 0.4pt,line join=round,line cap=round,fill=fillColor] (197.18,508.31) circle (  1.16);

\path[draw=drawColor,line width= 0.4pt,line join=round,line cap=round,fill=fillColor] (197.24,508.31) circle (  1.16);

\path[draw=drawColor,line width= 0.4pt,line join=round,line cap=round,fill=fillColor] (197.30,508.31) circle (  1.16);

\path[draw=drawColor,line width= 0.4pt,line join=round,line cap=round,fill=fillColor] (197.36,508.31) circle (  1.16);

\path[draw=drawColor,line width= 0.4pt,line join=round,line cap=round,fill=fillColor] (197.41,508.31) circle (  1.16);

\path[draw=drawColor,line width= 0.4pt,line join=round,line cap=round,fill=fillColor] (197.47,508.31) circle (  1.16);

\path[draw=drawColor,line width= 0.4pt,line join=round,line cap=round,fill=fillColor] (197.53,508.31) circle (  1.16);

\path[draw=drawColor,line width= 0.4pt,line join=round,line cap=round,fill=fillColor] (197.59,508.31) circle (  1.16);

\path[draw=drawColor,line width= 0.4pt,line join=round,line cap=round,fill=fillColor] (197.64,508.31) circle (  1.16);

\path[draw=drawColor,line width= 0.4pt,line join=round,line cap=round,fill=fillColor] (197.70,508.31) circle (  1.16);

\path[draw=drawColor,line width= 0.4pt,line join=round,line cap=round,fill=fillColor] (197.76,508.31) circle (  1.16);

\path[draw=drawColor,line width= 0.4pt,line join=round,line cap=round,fill=fillColor] (197.82,508.31) circle (  1.16);

\path[draw=drawColor,line width= 0.4pt,line join=round,line cap=round,fill=fillColor] (197.87,508.31) circle (  1.16);

\path[draw=drawColor,line width= 0.4pt,line join=round,line cap=round,fill=fillColor] (197.93,508.31) circle (  1.16);

\path[draw=drawColor,line width= 0.4pt,line join=round,line cap=round,fill=fillColor] (197.99,508.31) circle (  1.16);

\path[draw=drawColor,line width= 0.4pt,line join=round,line cap=round,fill=fillColor] (198.04,508.31) circle (  1.16);

\path[draw=drawColor,line width= 0.4pt,line join=round,line cap=round,fill=fillColor] (198.10,508.31) circle (  1.16);

\path[draw=drawColor,line width= 0.4pt,line join=round,line cap=round,fill=fillColor] (198.16,508.31) circle (  1.16);

\path[draw=drawColor,line width= 0.4pt,line join=round,line cap=round,fill=fillColor] (198.22,508.31) circle (  1.16);

\path[draw=drawColor,line width= 0.4pt,line join=round,line cap=round,fill=fillColor] (198.27,508.31) circle (  1.16);

\path[draw=drawColor,line width= 0.4pt,line join=round,line cap=round,fill=fillColor] (198.33,508.31) circle (  1.16);

\path[draw=drawColor,line width= 0.4pt,line join=round,line cap=round,fill=fillColor] (198.39,508.31) circle (  1.16);

\path[draw=drawColor,line width= 0.4pt,line join=round,line cap=round,fill=fillColor] (198.44,508.31) circle (  1.16);

\path[draw=drawColor,line width= 0.4pt,line join=round,line cap=round,fill=fillColor] (198.50,508.31) circle (  1.16);

\path[draw=drawColor,line width= 0.4pt,line join=round,line cap=round,fill=fillColor] (198.56,508.31) circle (  1.16);

\path[draw=drawColor,line width= 0.4pt,line join=round,line cap=round,fill=fillColor] (198.61,508.31) circle (  1.16);

\path[draw=drawColor,line width= 0.4pt,line join=round,line cap=round,fill=fillColor] (198.67,508.31) circle (  1.16);

\path[draw=drawColor,line width= 0.4pt,line join=round,line cap=round,fill=fillColor] (198.73,508.31) circle (  1.16);

\path[draw=drawColor,line width= 0.4pt,line join=round,line cap=round,fill=fillColor] (198.78,508.31) circle (  1.16);

\path[draw=drawColor,line width= 0.4pt,line join=round,line cap=round,fill=fillColor] (198.84,508.31) circle (  1.16);

\path[draw=drawColor,line width= 0.4pt,line join=round,line cap=round,fill=fillColor] (198.90,508.31) circle (  1.16);

\path[draw=drawColor,line width= 0.4pt,line join=round,line cap=round,fill=fillColor] (198.95,508.31) circle (  1.16);

\path[draw=drawColor,line width= 0.4pt,line join=round,line cap=round,fill=fillColor] (199.01,508.31) circle (  1.16);

\path[draw=drawColor,line width= 0.4pt,line join=round,line cap=round,fill=fillColor] (199.06,508.31) circle (  1.16);

\path[draw=drawColor,line width= 0.4pt,line join=round,line cap=round,fill=fillColor] (199.12,508.31) circle (  1.16);

\path[draw=drawColor,line width= 0.4pt,line join=round,line cap=round,fill=fillColor] (199.18,508.31) circle (  1.16);

\path[draw=drawColor,line width= 0.4pt,line join=round,line cap=round,fill=fillColor] (199.23,508.31) circle (  1.16);

\path[draw=drawColor,line width= 0.4pt,line join=round,line cap=round,fill=fillColor] (199.29,508.31) circle (  1.16);

\path[draw=drawColor,line width= 0.4pt,line join=round,line cap=round,fill=fillColor] (199.34,508.31) circle (  1.16);

\path[draw=drawColor,line width= 0.4pt,line join=round,line cap=round,fill=fillColor] (199.40,508.31) circle (  1.16);

\path[draw=drawColor,line width= 0.4pt,line join=round,line cap=round,fill=fillColor] (199.46,508.31) circle (  1.16);

\path[draw=drawColor,line width= 0.4pt,line join=round,line cap=round,fill=fillColor] (199.51,508.31) circle (  1.16);

\path[draw=drawColor,line width= 0.4pt,line join=round,line cap=round,fill=fillColor] (199.57,508.31) circle (  1.16);

\path[draw=drawColor,line width= 0.4pt,line join=round,line cap=round,fill=fillColor] (199.62,508.31) circle (  1.16);

\path[draw=drawColor,line width= 0.4pt,line join=round,line cap=round,fill=fillColor] (199.68,508.31) circle (  1.16);

\path[draw=drawColor,line width= 0.4pt,line join=round,line cap=round,fill=fillColor] (199.73,508.31) circle (  1.16);

\path[draw=drawColor,line width= 0.4pt,line join=round,line cap=round,fill=fillColor] (199.79,508.31) circle (  1.16);

\path[draw=drawColor,line width= 0.4pt,line join=round,line cap=round,fill=fillColor] (199.85,508.31) circle (  1.16);

\path[draw=drawColor,line width= 0.4pt,line join=round,line cap=round,fill=fillColor] (199.90,508.31) circle (  1.16);

\path[draw=drawColor,line width= 0.4pt,line join=round,line cap=round,fill=fillColor] (199.96,508.31) circle (  1.16);

\path[draw=drawColor,line width= 0.4pt,line join=round,line cap=round,fill=fillColor] (200.01,508.31) circle (  1.16);

\path[draw=drawColor,line width= 0.4pt,line join=round,line cap=round,fill=fillColor] (200.07,508.31) circle (  1.16);

\path[draw=drawColor,line width= 0.4pt,line join=round,line cap=round,fill=fillColor] (200.12,508.31) circle (  1.16);

\path[draw=drawColor,line width= 0.4pt,line join=round,line cap=round,fill=fillColor] (200.18,508.31) circle (  1.16);

\path[draw=drawColor,line width= 0.4pt,line join=round,line cap=round,fill=fillColor] (200.23,508.31) circle (  1.16);

\path[draw=drawColor,line width= 0.4pt,line join=round,line cap=round,fill=fillColor] (200.29,508.31) circle (  1.16);

\path[draw=drawColor,line width= 0.4pt,line join=round,line cap=round,fill=fillColor] (200.34,508.31) circle (  1.16);

\path[draw=drawColor,line width= 0.4pt,line join=round,line cap=round,fill=fillColor] (200.40,508.31) circle (  1.16);

\path[draw=drawColor,line width= 0.4pt,line join=round,line cap=round,fill=fillColor] (200.45,508.31) circle (  1.16);

\path[draw=drawColor,line width= 0.4pt,line join=round,line cap=round,fill=fillColor] (200.51,508.31) circle (  1.16);

\path[draw=drawColor,line width= 0.4pt,line join=round,line cap=round,fill=fillColor] (200.56,508.31) circle (  1.16);

\path[draw=drawColor,line width= 0.4pt,line join=round,line cap=round,fill=fillColor] (200.62,508.31) circle (  1.16);

\path[draw=drawColor,line width= 0.4pt,line join=round,line cap=round,fill=fillColor] (200.67,508.31) circle (  1.16);

\path[draw=drawColor,line width= 0.4pt,line join=round,line cap=round,fill=fillColor] (200.73,508.31) circle (  1.16);

\path[draw=drawColor,line width= 0.4pt,line join=round,line cap=round,fill=fillColor] (200.78,508.31) circle (  1.16);

\path[draw=drawColor,line width= 0.4pt,line join=round,line cap=round,fill=fillColor] (200.84,508.31) circle (  1.16);

\path[draw=drawColor,line width= 0.4pt,line join=round,line cap=round,fill=fillColor] (200.89,508.31) circle (  1.16);

\path[draw=drawColor,line width= 0.4pt,line join=round,line cap=round,fill=fillColor] (200.95,508.31) circle (  1.16);

\path[draw=drawColor,line width= 0.4pt,line join=round,line cap=round,fill=fillColor] (201.00,508.31) circle (  1.16);

\path[draw=drawColor,line width= 0.4pt,line join=round,line cap=round,fill=fillColor] (201.06,508.31) circle (  1.16);

\path[draw=drawColor,line width= 0.4pt,line join=round,line cap=round,fill=fillColor] (201.11,508.31) circle (  1.16);

\path[draw=drawColor,line width= 0.4pt,line join=round,line cap=round,fill=fillColor] (201.17,508.31) circle (  1.16);

\path[draw=drawColor,line width= 0.4pt,line join=round,line cap=round,fill=fillColor] (201.22,508.31) circle (  1.16);

\path[draw=drawColor,line width= 0.4pt,line join=round,line cap=round,fill=fillColor] (201.27,508.31) circle (  1.16);

\path[draw=drawColor,line width= 0.4pt,line join=round,line cap=round,fill=fillColor] (201.33,508.31) circle (  1.16);

\path[draw=drawColor,line width= 0.4pt,line join=round,line cap=round,fill=fillColor] (201.38,508.31) circle (  1.16);

\path[draw=drawColor,line width= 0.4pt,line join=round,line cap=round,fill=fillColor] (201.44,508.31) circle (  1.16);

\path[draw=drawColor,line width= 0.4pt,line join=round,line cap=round,fill=fillColor] (201.49,508.31) circle (  1.16);

\path[draw=drawColor,line width= 0.4pt,line join=round,line cap=round,fill=fillColor] (201.55,508.31) circle (  1.16);

\path[draw=drawColor,line width= 0.4pt,line join=round,line cap=round,fill=fillColor] (201.60,508.31) circle (  1.16);

\path[draw=drawColor,line width= 0.4pt,line join=round,line cap=round,fill=fillColor] (201.65,508.31) circle (  1.16);

\path[draw=drawColor,line width= 0.4pt,line join=round,line cap=round,fill=fillColor] (201.71,508.31) circle (  1.16);

\path[draw=drawColor,line width= 0.4pt,line join=round,line cap=round,fill=fillColor] (201.76,508.31) circle (  1.16);

\path[draw=drawColor,line width= 0.4pt,line join=round,line cap=round,fill=fillColor] (201.82,508.31) circle (  1.16);

\path[draw=drawColor,line width= 0.4pt,line join=round,line cap=round,fill=fillColor] (201.87,508.31) circle (  1.16);

\path[draw=drawColor,line width= 0.4pt,line join=round,line cap=round,fill=fillColor] (201.92,508.31) circle (  1.16);

\path[draw=drawColor,line width= 0.4pt,line join=round,line cap=round,fill=fillColor] (201.98,508.31) circle (  1.16);

\path[draw=drawColor,line width= 0.4pt,line join=round,line cap=round,fill=fillColor] (202.03,508.31) circle (  1.16);

\path[draw=drawColor,line width= 0.4pt,line join=round,line cap=round,fill=fillColor] (202.09,508.31) circle (  1.16);

\path[draw=drawColor,line width= 0.4pt,line join=round,line cap=round,fill=fillColor] (202.14,508.31) circle (  1.16);

\path[draw=drawColor,line width= 0.4pt,line join=round,line cap=round,fill=fillColor] (202.19,508.31) circle (  1.16);

\path[draw=drawColor,line width= 0.4pt,line join=round,line cap=round,fill=fillColor] (202.25,508.31) circle (  1.16);

\path[draw=drawColor,line width= 0.4pt,line join=round,line cap=round,fill=fillColor] (202.30,508.31) circle (  1.16);

\path[draw=drawColor,line width= 0.4pt,line join=round,line cap=round,fill=fillColor] (202.35,508.31) circle (  1.16);

\path[draw=drawColor,line width= 0.4pt,line join=round,line cap=round,fill=fillColor] (202.41,508.31) circle (  1.16);

\path[draw=drawColor,line width= 0.4pt,line join=round,line cap=round,fill=fillColor] (202.46,508.31) circle (  1.16);

\path[draw=drawColor,line width= 0.4pt,line join=round,line cap=round,fill=fillColor] (202.51,508.31) circle (  1.16);

\path[draw=drawColor,line width= 0.4pt,line join=round,line cap=round,fill=fillColor] (202.57,508.31) circle (  1.16);

\path[draw=drawColor,line width= 0.4pt,line join=round,line cap=round,fill=fillColor] (202.62,508.31) circle (  1.16);

\path[draw=drawColor,line width= 0.4pt,line join=round,line cap=round,fill=fillColor] (202.67,508.31) circle (  1.16);

\path[draw=drawColor,line width= 0.4pt,line join=round,line cap=round,fill=fillColor] (202.73,508.31) circle (  1.16);

\path[draw=drawColor,line width= 0.4pt,line join=round,line cap=round,fill=fillColor] (202.78,508.31) circle (  1.16);

\path[draw=drawColor,line width= 0.4pt,line join=round,line cap=round,fill=fillColor] (202.83,508.31) circle (  1.16);

\path[draw=drawColor,line width= 0.4pt,line join=round,line cap=round,fill=fillColor] (202.89,508.31) circle (  1.16);

\path[draw=drawColor,line width= 0.4pt,line join=round,line cap=round,fill=fillColor] (202.94,508.31) circle (  1.16);

\path[draw=drawColor,line width= 0.4pt,line join=round,line cap=round,fill=fillColor] (202.99,508.31) circle (  1.16);

\path[draw=drawColor,line width= 0.4pt,line join=round,line cap=round,fill=fillColor] (203.05,508.31) circle (  1.16);

\path[draw=drawColor,line width= 0.4pt,line join=round,line cap=round,fill=fillColor] (203.10,508.31) circle (  1.16);

\path[draw=drawColor,line width= 0.4pt,line join=round,line cap=round,fill=fillColor] (203.15,508.31) circle (  1.16);

\path[draw=drawColor,line width= 0.4pt,line join=round,line cap=round,fill=fillColor] (203.20,508.31) circle (  1.16);

\path[draw=drawColor,line width= 0.4pt,line join=round,line cap=round,fill=fillColor] (203.26,508.31) circle (  1.16);

\path[draw=drawColor,line width= 0.4pt,line join=round,line cap=round,fill=fillColor] (203.31,508.31) circle (  1.16);

\path[draw=drawColor,line width= 0.4pt,line join=round,line cap=round,fill=fillColor] (203.36,508.31) circle (  1.16);

\path[draw=drawColor,line width= 0.4pt,line join=round,line cap=round,fill=fillColor] (203.41,508.31) circle (  1.16);

\path[draw=drawColor,line width= 0.4pt,line join=round,line cap=round,fill=fillColor] (203.47,508.31) circle (  1.16);

\path[draw=drawColor,line width= 0.4pt,line join=round,line cap=round,fill=fillColor] (203.52,508.31) circle (  1.16);

\path[draw=drawColor,line width= 0.4pt,line join=round,line cap=round,fill=fillColor] (203.57,508.31) circle (  1.16);

\path[draw=drawColor,line width= 0.4pt,line join=round,line cap=round,fill=fillColor] (203.63,508.31) circle (  1.16);

\path[draw=drawColor,line width= 0.4pt,line join=round,line cap=round,fill=fillColor] (203.68,508.31) circle (  1.16);

\path[draw=drawColor,line width= 0.4pt,line join=round,line cap=round,fill=fillColor] (203.73,508.31) circle (  1.16);

\path[draw=drawColor,line width= 0.4pt,line join=round,line cap=round,fill=fillColor] (203.78,508.31) circle (  1.16);

\path[draw=drawColor,line width= 0.4pt,line join=round,line cap=round,fill=fillColor] (203.83,508.31) circle (  1.16);

\path[draw=drawColor,line width= 0.4pt,line join=round,line cap=round,fill=fillColor] (203.89,508.31) circle (  1.16);

\path[draw=drawColor,line width= 0.4pt,line join=round,line cap=round,fill=fillColor] (203.94,508.31) circle (  1.16);

\path[draw=drawColor,line width= 0.4pt,line join=round,line cap=round,fill=fillColor] (203.99,508.31) circle (  1.16);

\path[draw=drawColor,line width= 0.4pt,line join=round,line cap=round,fill=fillColor] (204.04,508.31) circle (  1.16);

\path[draw=drawColor,line width= 0.4pt,line join=round,line cap=round,fill=fillColor] (204.10,508.31) circle (  1.16);

\path[draw=drawColor,line width= 0.4pt,line join=round,line cap=round,fill=fillColor] (204.15,508.31) circle (  1.16);

\path[draw=drawColor,line width= 0.4pt,line join=round,line cap=round,fill=fillColor] (204.20,508.31) circle (  1.16);

\path[draw=drawColor,line width= 0.4pt,line join=round,line cap=round,fill=fillColor] (204.25,508.31) circle (  1.16);

\path[draw=drawColor,line width= 0.4pt,line join=round,line cap=round,fill=fillColor] (204.30,508.31) circle (  1.16);

\path[draw=drawColor,line width= 0.4pt,line join=round,line cap=round,fill=fillColor] (204.36,508.31) circle (  1.16);

\path[draw=drawColor,line width= 0.4pt,line join=round,line cap=round,fill=fillColor] (204.41,508.31) circle (  1.16);

\path[draw=drawColor,line width= 0.4pt,line join=round,line cap=round,fill=fillColor] (204.46,508.31) circle (  1.16);

\path[draw=drawColor,line width= 0.4pt,line join=round,line cap=round,fill=fillColor] (204.51,508.31) circle (  1.16);

\path[draw=drawColor,line width= 0.4pt,line join=round,line cap=round,fill=fillColor] (204.56,508.31) circle (  1.16);

\path[draw=drawColor,line width= 0.4pt,line join=round,line cap=round,fill=fillColor] (204.62,508.31) circle (  1.16);

\path[draw=drawColor,line width= 0.4pt,line join=round,line cap=round,fill=fillColor] (204.67,508.31) circle (  1.16);

\path[draw=drawColor,line width= 0.4pt,line join=round,line cap=round,fill=fillColor] (204.72,508.31) circle (  1.16);

\path[draw=drawColor,line width= 0.4pt,line join=round,line cap=round,fill=fillColor] (204.77,508.31) circle (  1.16);

\path[draw=drawColor,line width= 0.4pt,line join=round,line cap=round,fill=fillColor] (204.82,508.31) circle (  1.16);

\path[draw=drawColor,line width= 0.4pt,line join=round,line cap=round,fill=fillColor] (204.87,508.31) circle (  1.16);

\path[draw=drawColor,line width= 0.4pt,line join=round,line cap=round,fill=fillColor] (204.93,508.31) circle (  1.16);

\path[draw=drawColor,line width= 0.4pt,line join=round,line cap=round,fill=fillColor] (204.98,508.31) circle (  1.16);

\path[draw=drawColor,line width= 0.4pt,line join=round,line cap=round,fill=fillColor] (205.03,508.31) circle (  1.16);

\path[draw=drawColor,line width= 0.4pt,line join=round,line cap=round,fill=fillColor] (205.08,508.31) circle (  1.16);

\path[draw=drawColor,line width= 0.4pt,line join=round,line cap=round,fill=fillColor] (205.13,508.31) circle (  1.16);

\path[draw=drawColor,line width= 0.4pt,line join=round,line cap=round,fill=fillColor] (205.18,508.31) circle (  1.16);

\path[draw=drawColor,line width= 0.4pt,line join=round,line cap=round,fill=fillColor] (205.23,508.31) circle (  1.16);

\path[draw=drawColor,line width= 0.4pt,line join=round,line cap=round,fill=fillColor] (205.29,508.31) circle (  1.16);

\path[draw=drawColor,line width= 0.4pt,line join=round,line cap=round,fill=fillColor] (205.34,508.31) circle (  1.16);

\path[draw=drawColor,line width= 0.4pt,line join=round,line cap=round,fill=fillColor] (205.39,508.31) circle (  1.16);

\path[draw=drawColor,line width= 0.4pt,line join=round,line cap=round,fill=fillColor] (205.44,508.31) circle (  1.16);

\path[draw=drawColor,line width= 0.4pt,line join=round,line cap=round,fill=fillColor] (205.49,508.31) circle (  1.16);

\path[draw=drawColor,line width= 0.4pt,line join=round,line cap=round,fill=fillColor] (205.54,508.31) circle (  1.16);

\path[draw=drawColor,line width= 0.4pt,line join=round,line cap=round,fill=fillColor] (205.59,508.31) circle (  1.16);

\path[draw=drawColor,line width= 0.4pt,line join=round,line cap=round,fill=fillColor] (205.64,508.31) circle (  1.16);

\path[draw=drawColor,line width= 0.4pt,line join=round,line cap=round,fill=fillColor] (205.69,508.31) circle (  1.16);

\path[draw=drawColor,line width= 0.4pt,line join=round,line cap=round,fill=fillColor] (205.75,508.31) circle (  1.16);

\path[draw=drawColor,line width= 0.4pt,line join=round,line cap=round,fill=fillColor] (205.80,508.31) circle (  1.16);

\path[draw=drawColor,line width= 0.4pt,line join=round,line cap=round,fill=fillColor] (205.85,508.31) circle (  1.16);

\path[draw=drawColor,line width= 0.4pt,line join=round,line cap=round,fill=fillColor] (205.90,508.31) circle (  1.16);

\path[draw=drawColor,line width= 0.4pt,line join=round,line cap=round,fill=fillColor] (205.95,508.31) circle (  1.16);

\path[draw=drawColor,line width= 0.4pt,line join=round,line cap=round,fill=fillColor] (206.00,508.31) circle (  1.16);

\path[draw=drawColor,line width= 0.4pt,line join=round,line cap=round,fill=fillColor] (206.05,508.31) circle (  1.16);

\path[draw=drawColor,line width= 0.4pt,line join=round,line cap=round,fill=fillColor] (206.10,508.31) circle (  1.16);

\path[draw=drawColor,line width= 0.4pt,line join=round,line cap=round,fill=fillColor] (206.15,508.31) circle (  1.16);

\path[draw=drawColor,line width= 0.4pt,line join=round,line cap=round,fill=fillColor] (206.20,508.31) circle (  1.16);

\path[draw=drawColor,line width= 0.4pt,line join=round,line cap=round,fill=fillColor] (206.25,508.31) circle (  1.16);

\path[draw=drawColor,line width= 0.4pt,line join=round,line cap=round,fill=fillColor] (206.30,508.31) circle (  1.16);

\path[draw=drawColor,line width= 0.4pt,line join=round,line cap=round,fill=fillColor] (206.35,508.31) circle (  1.16);

\path[draw=drawColor,line width= 0.4pt,line join=round,line cap=round,fill=fillColor] (206.40,508.31) circle (  1.16);

\path[draw=drawColor,line width= 0.4pt,line join=round,line cap=round,fill=fillColor] (206.45,508.31) circle (  1.16);

\path[draw=drawColor,line width= 0.4pt,line join=round,line cap=round,fill=fillColor] (206.50,508.31) circle (  1.16);

\path[draw=drawColor,line width= 0.4pt,line join=round,line cap=round,fill=fillColor] (206.55,508.31) circle (  1.16);

\path[draw=drawColor,line width= 0.4pt,line join=round,line cap=round,fill=fillColor] (206.61,508.31) circle (  1.16);

\path[draw=drawColor,line width= 0.4pt,line join=round,line cap=round,fill=fillColor] (206.66,508.31) circle (  1.16);

\path[draw=drawColor,line width= 0.4pt,line join=round,line cap=round,fill=fillColor] (206.71,508.31) circle (  1.16);

\path[draw=drawColor,line width= 0.4pt,line join=round,line cap=round,fill=fillColor] (206.76,508.31) circle (  1.16);

\path[draw=drawColor,line width= 0.4pt,line join=round,line cap=round,fill=fillColor] (206.81,508.31) circle (  1.16);

\path[draw=drawColor,line width= 0.4pt,line join=round,line cap=round,fill=fillColor] (206.86,508.31) circle (  1.16);

\path[draw=drawColor,line width= 0.4pt,line join=round,line cap=round,fill=fillColor] (206.91,508.31) circle (  1.16);

\path[draw=drawColor,line width= 0.4pt,line join=round,line cap=round,fill=fillColor] (206.96,508.31) circle (  1.16);

\path[draw=drawColor,line width= 0.4pt,line join=round,line cap=round,fill=fillColor] (207.01,508.31) circle (  1.16);

\path[draw=drawColor,line width= 0.4pt,line join=round,line cap=round,fill=fillColor] (207.06,508.31) circle (  1.16);

\path[draw=drawColor,line width= 0.4pt,line join=round,line cap=round,fill=fillColor] (207.11,508.31) circle (  1.16);

\path[draw=drawColor,line width= 0.4pt,line join=round,line cap=round,fill=fillColor] (207.16,508.31) circle (  1.16);

\path[draw=drawColor,line width= 0.4pt,line join=round,line cap=round,fill=fillColor] (207.21,508.31) circle (  1.16);

\path[draw=drawColor,line width= 0.4pt,line join=round,line cap=round,fill=fillColor] (207.26,508.31) circle (  1.16);

\path[draw=drawColor,line width= 0.4pt,line join=round,line cap=round,fill=fillColor] (207.31,508.31) circle (  1.16);

\path[draw=drawColor,line width= 0.4pt,line join=round,line cap=round,fill=fillColor] (207.36,508.31) circle (  1.16);

\path[draw=drawColor,line width= 0.4pt,line join=round,line cap=round,fill=fillColor] (207.41,508.31) circle (  1.16);

\path[draw=drawColor,line width= 0.4pt,line join=round,line cap=round,fill=fillColor] (207.46,508.31) circle (  1.16);

\path[draw=drawColor,line width= 0.4pt,line join=round,line cap=round,fill=fillColor] (207.50,508.31) circle (  1.16);

\path[draw=drawColor,line width= 0.4pt,line join=round,line cap=round,fill=fillColor] (207.55,508.31) circle (  1.16);

\path[draw=drawColor,line width= 0.4pt,line join=round,line cap=round,fill=fillColor] (207.60,508.31) circle (  1.16);

\path[draw=drawColor,line width= 0.4pt,line join=round,line cap=round,fill=fillColor] (207.65,508.31) circle (  1.16);

\path[draw=drawColor,line width= 0.4pt,line join=round,line cap=round,fill=fillColor] (207.70,508.31) circle (  1.16);

\path[draw=drawColor,line width= 0.4pt,line join=round,line cap=round,fill=fillColor] (207.75,508.31) circle (  1.16);

\path[draw=drawColor,line width= 0.4pt,line join=round,line cap=round,fill=fillColor] (207.80,508.31) circle (  1.16);

\path[draw=drawColor,line width= 0.4pt,line join=round,line cap=round,fill=fillColor] (207.85,508.31) circle (  1.16);

\path[draw=drawColor,line width= 0.4pt,line join=round,line cap=round,fill=fillColor] (207.90,508.31) circle (  1.16);

\path[draw=drawColor,line width= 0.4pt,line join=round,line cap=round,fill=fillColor] (207.95,508.31) circle (  1.16);

\path[draw=drawColor,line width= 0.4pt,line join=round,line cap=round,fill=fillColor] (208.00,508.31) circle (  1.16);

\path[draw=drawColor,line width= 0.4pt,line join=round,line cap=round,fill=fillColor] (208.05,508.31) circle (  1.16);

\path[draw=drawColor,line width= 0.4pt,line join=round,line cap=round,fill=fillColor] (208.10,508.31) circle (  1.16);

\path[draw=drawColor,line width= 0.4pt,line join=round,line cap=round,fill=fillColor] (208.15,508.31) circle (  1.16);

\path[draw=drawColor,line width= 0.4pt,line join=round,line cap=round,fill=fillColor] (208.20,508.31) circle (  1.16);

\path[draw=drawColor,line width= 0.4pt,line join=round,line cap=round,fill=fillColor] (208.25,508.31) circle (  1.16);

\path[draw=drawColor,line width= 0.4pt,line join=round,line cap=round,fill=fillColor] (208.29,508.31) circle (  1.16);

\path[draw=drawColor,line width= 0.4pt,line join=round,line cap=round,fill=fillColor] (208.34,508.31) circle (  1.16);

\path[draw=drawColor,line width= 0.4pt,line join=round,line cap=round,fill=fillColor] (208.39,508.31) circle (  1.16);

\path[draw=drawColor,line width= 0.4pt,line join=round,line cap=round,fill=fillColor] (208.44,508.31) circle (  1.16);

\path[draw=drawColor,line width= 0.4pt,line join=round,line cap=round,fill=fillColor] (208.49,508.31) circle (  1.16);

\path[draw=drawColor,line width= 0.4pt,line join=round,line cap=round,fill=fillColor] (208.54,508.31) circle (  1.16);

\path[draw=drawColor,line width= 0.4pt,line join=round,line cap=round,fill=fillColor] (208.59,508.31) circle (  1.16);

\path[draw=drawColor,line width= 0.4pt,line join=round,line cap=round,fill=fillColor] (208.64,508.31) circle (  1.16);

\path[draw=drawColor,line width= 0.4pt,line join=round,line cap=round,fill=fillColor] (208.69,508.31) circle (  1.16);

\path[draw=drawColor,line width= 0.4pt,line join=round,line cap=round,fill=fillColor] (208.74,508.31) circle (  1.16);

\path[draw=drawColor,line width= 0.4pt,line join=round,line cap=round,fill=fillColor] (208.78,508.31) circle (  1.16);

\path[draw=drawColor,line width= 0.4pt,line join=round,line cap=round,fill=fillColor] (208.83,508.31) circle (  1.16);

\path[draw=drawColor,line width= 0.4pt,line join=round,line cap=round,fill=fillColor] (208.88,508.31) circle (  1.16);

\path[draw=drawColor,line width= 0.4pt,line join=round,line cap=round,fill=fillColor] (208.93,508.31) circle (  1.16);

\path[draw=drawColor,line width= 0.4pt,line join=round,line cap=round,fill=fillColor] (208.98,508.31) circle (  1.16);

\path[draw=drawColor,line width= 0.4pt,line join=round,line cap=round,fill=fillColor] (209.03,508.31) circle (  1.16);

\path[draw=drawColor,line width= 0.4pt,line join=round,line cap=round,fill=fillColor] (209.08,508.31) circle (  1.16);

\path[draw=drawColor,line width= 0.4pt,line join=round,line cap=round,fill=fillColor] (209.12,508.31) circle (  1.16);

\path[draw=drawColor,line width= 0.4pt,line join=round,line cap=round,fill=fillColor] (209.17,508.31) circle (  1.16);

\path[draw=drawColor,line width= 0.4pt,line join=round,line cap=round,fill=fillColor] (209.22,508.31) circle (  1.16);

\path[draw=drawColor,line width= 0.4pt,line join=round,line cap=round,fill=fillColor] (209.27,508.31) circle (  1.16);

\path[draw=drawColor,line width= 0.4pt,line join=round,line cap=round,fill=fillColor] (209.32,508.31) circle (  1.16);

\path[draw=drawColor,line width= 0.4pt,line join=round,line cap=round,fill=fillColor] (209.37,508.31) circle (  1.16);

\path[draw=drawColor,line width= 0.4pt,line join=round,line cap=round,fill=fillColor] (209.42,508.31) circle (  1.16);

\path[draw=drawColor,line width= 0.4pt,line join=round,line cap=round,fill=fillColor] (209.46,508.31) circle (  1.16);

\path[draw=drawColor,line width= 0.4pt,line join=round,line cap=round,fill=fillColor] (209.51,508.31) circle (  1.16);

\path[draw=drawColor,line width= 0.4pt,line join=round,line cap=round,fill=fillColor] (209.56,508.31) circle (  1.16);

\path[draw=drawColor,line width= 0.4pt,line join=round,line cap=round,fill=fillColor] (209.61,508.31) circle (  1.16);

\path[draw=drawColor,line width= 0.4pt,line join=round,line cap=round,fill=fillColor] (209.66,508.31) circle (  1.16);

\path[draw=drawColor,line width= 0.4pt,line join=round,line cap=round,fill=fillColor] (209.70,508.31) circle (  1.16);

\path[draw=drawColor,line width= 0.4pt,line join=round,line cap=round,fill=fillColor] (209.75,508.31) circle (  1.16);

\path[draw=drawColor,line width= 0.4pt,line join=round,line cap=round,fill=fillColor] (209.80,508.31) circle (  1.16);

\path[draw=drawColor,line width= 0.4pt,line join=round,line cap=round,fill=fillColor] (209.85,508.31) circle (  1.16);

\path[draw=drawColor,line width= 0.4pt,line join=round,line cap=round,fill=fillColor] (209.90,508.31) circle (  1.16);

\path[draw=drawColor,line width= 0.4pt,line join=round,line cap=round,fill=fillColor] (209.94,508.31) circle (  1.16);

\path[draw=drawColor,line width= 0.4pt,line join=round,line cap=round,fill=fillColor] (209.99,508.31) circle (  1.16);

\path[draw=drawColor,line width= 0.4pt,line join=round,line cap=round,fill=fillColor] (210.04,508.31) circle (  1.16);

\path[draw=drawColor,line width= 0.4pt,line join=round,line cap=round,fill=fillColor] (210.09,508.31) circle (  1.16);

\path[draw=drawColor,line width= 0.4pt,line join=round,line cap=round,fill=fillColor] (210.14,508.31) circle (  1.16);

\path[draw=drawColor,line width= 0.4pt,line join=round,line cap=round,fill=fillColor] (210.18,508.31) circle (  1.16);

\path[draw=drawColor,line width= 0.4pt,line join=round,line cap=round,fill=fillColor] (210.23,508.31) circle (  1.16);

\path[draw=drawColor,line width= 0.4pt,line join=round,line cap=round,fill=fillColor] (210.28,508.31) circle (  1.16);

\path[draw=drawColor,line width= 0.4pt,line join=round,line cap=round,fill=fillColor] (210.33,508.31) circle (  1.16);

\path[draw=drawColor,line width= 0.4pt,line join=round,line cap=round,fill=fillColor] (210.38,508.31) circle (  1.16);

\path[draw=drawColor,line width= 0.4pt,line join=round,line cap=round,fill=fillColor] (210.42,508.31) circle (  1.16);

\path[draw=drawColor,line width= 0.4pt,line join=round,line cap=round,fill=fillColor] (210.47,508.31) circle (  1.16);

\path[draw=drawColor,line width= 0.4pt,line join=round,line cap=round,fill=fillColor] (210.52,508.31) circle (  1.16);

\path[draw=drawColor,line width= 0.4pt,line join=round,line cap=round,fill=fillColor] (210.57,508.31) circle (  1.16);

\path[draw=drawColor,line width= 0.4pt,line join=round,line cap=round,fill=fillColor] (210.61,508.31) circle (  1.16);

\path[draw=drawColor,line width= 0.4pt,line join=round,line cap=round,fill=fillColor] (210.66,508.31) circle (  1.16);

\path[draw=drawColor,line width= 0.4pt,line join=round,line cap=round,fill=fillColor] (210.71,508.31) circle (  1.16);

\path[draw=drawColor,line width= 0.4pt,line join=round,line cap=round,fill=fillColor] (210.76,508.31) circle (  1.16);

\path[draw=drawColor,line width= 0.4pt,line join=round,line cap=round,fill=fillColor] (210.80,508.31) circle (  1.16);

\path[draw=drawColor,line width= 0.4pt,line join=round,line cap=round,fill=fillColor] (210.85,508.31) circle (  1.16);

\path[draw=drawColor,line width= 0.4pt,line join=round,line cap=round,fill=fillColor] (210.90,508.31) circle (  1.16);

\path[draw=drawColor,line width= 0.4pt,line join=round,line cap=round,fill=fillColor] (210.95,508.31) circle (  1.16);

\path[draw=drawColor,line width= 0.4pt,line join=round,line cap=round,fill=fillColor] (210.99,508.31) circle (  1.16);

\path[draw=drawColor,line width= 0.4pt,line join=round,line cap=round,fill=fillColor] (211.04,508.31) circle (  1.16);

\path[draw=drawColor,line width= 0.4pt,line join=round,line cap=round,fill=fillColor] (211.09,508.31) circle (  1.16);

\path[draw=drawColor,line width= 0.4pt,line join=round,line cap=round,fill=fillColor] (211.13,508.31) circle (  1.16);

\path[draw=drawColor,line width= 0.4pt,line join=round,line cap=round,fill=fillColor] (211.18,508.31) circle (  1.16);

\path[draw=drawColor,line width= 0.4pt,line join=round,line cap=round,fill=fillColor] (211.23,508.31) circle (  1.16);

\path[draw=drawColor,line width= 0.4pt,line join=round,line cap=round,fill=fillColor] (211.28,508.31) circle (  1.16);

\path[draw=drawColor,line width= 0.4pt,line join=round,line cap=round,fill=fillColor] (211.32,508.31) circle (  1.16);

\path[draw=drawColor,line width= 0.4pt,line join=round,line cap=round,fill=fillColor] (211.37,508.31) circle (  1.16);

\path[draw=drawColor,line width= 0.4pt,line join=round,line cap=round,fill=fillColor] (211.42,508.31) circle (  1.16);

\path[draw=drawColor,line width= 0.4pt,line join=round,line cap=round,fill=fillColor] (211.46,508.31) circle (  1.16);

\path[draw=drawColor,line width= 0.4pt,line join=round,line cap=round,fill=fillColor] (211.51,508.31) circle (  1.16);

\path[draw=drawColor,line width= 0.4pt,line join=round,line cap=round,fill=fillColor] (211.56,508.31) circle (  1.16);

\path[draw=drawColor,line width= 0.4pt,line join=round,line cap=round,fill=fillColor] (211.60,508.31) circle (  1.16);

\path[draw=drawColor,line width= 0.4pt,line join=round,line cap=round,fill=fillColor] (211.65,508.31) circle (  1.16);

\path[draw=drawColor,line width= 0.4pt,line join=round,line cap=round,fill=fillColor] (211.70,508.31) circle (  1.16);

\path[draw=drawColor,line width= 0.4pt,line join=round,line cap=round,fill=fillColor] (211.75,508.31) circle (  1.16);

\path[draw=drawColor,line width= 0.4pt,line join=round,line cap=round,fill=fillColor] (211.79,508.31) circle (  1.16);

\path[draw=drawColor,line width= 0.4pt,line join=round,line cap=round,fill=fillColor] (211.84,508.31) circle (  1.16);

\path[draw=drawColor,line width= 0.4pt,line join=round,line cap=round,fill=fillColor] (211.89,508.31) circle (  1.16);

\path[draw=drawColor,line width= 0.4pt,line join=round,line cap=round,fill=fillColor] (211.93,508.31) circle (  1.16);

\path[draw=drawColor,line width= 0.4pt,line join=round,line cap=round,fill=fillColor] (211.98,508.31) circle (  1.16);

\path[draw=drawColor,line width= 0.4pt,line join=round,line cap=round,fill=fillColor] (212.03,508.31) circle (  1.16);

\path[draw=drawColor,line width= 0.4pt,line join=round,line cap=round,fill=fillColor] (212.07,508.31) circle (  1.16);

\path[draw=drawColor,line width= 0.4pt,line join=round,line cap=round,fill=fillColor] (212.12,508.31) circle (  1.16);

\path[draw=drawColor,line width= 0.4pt,line join=round,line cap=round,fill=fillColor] (212.16,508.31) circle (  1.16);

\path[draw=drawColor,line width= 0.4pt,line join=round,line cap=round,fill=fillColor] (212.21,508.31) circle (  1.16);

\path[draw=drawColor,line width= 0.4pt,line join=round,line cap=round,fill=fillColor] (212.26,508.31) circle (  1.16);

\path[draw=drawColor,line width= 0.4pt,line join=round,line cap=round,fill=fillColor] (212.30,508.31) circle (  1.16);

\path[draw=drawColor,line width= 0.4pt,line join=round,line cap=round,fill=fillColor] (212.35,508.31) circle (  1.16);

\path[draw=drawColor,line width= 0.4pt,line join=round,line cap=round,fill=fillColor] (212.40,508.31) circle (  1.16);

\path[draw=drawColor,line width= 0.4pt,line join=round,line cap=round,fill=fillColor] (212.44,508.31) circle (  1.16);

\path[draw=drawColor,line width= 0.4pt,line join=round,line cap=round,fill=fillColor] (212.49,508.31) circle (  1.16);

\path[draw=drawColor,line width= 0.4pt,line join=round,line cap=round,fill=fillColor] (212.54,508.31) circle (  1.16);

\path[draw=drawColor,line width= 0.4pt,line join=round,line cap=round,fill=fillColor] (212.58,508.31) circle (  1.16);

\path[draw=drawColor,line width= 0.4pt,line join=round,line cap=round,fill=fillColor] (212.63,508.31) circle (  1.16);

\path[draw=drawColor,line width= 0.4pt,line join=round,line cap=round,fill=fillColor] (212.67,508.31) circle (  1.16);

\path[draw=drawColor,line width= 0.4pt,line join=round,line cap=round,fill=fillColor] (212.72,508.31) circle (  1.16);

\path[draw=drawColor,line width= 0.4pt,line join=round,line cap=round,fill=fillColor] (212.77,508.31) circle (  1.16);

\path[draw=drawColor,line width= 0.4pt,line join=round,line cap=round,fill=fillColor] (212.81,508.31) circle (  1.16);

\path[draw=drawColor,line width= 0.4pt,line join=round,line cap=round,fill=fillColor] (212.86,508.31) circle (  1.16);

\path[draw=drawColor,line width= 0.4pt,line join=round,line cap=round,fill=fillColor] (212.91,508.31) circle (  1.16);

\path[draw=drawColor,line width= 0.4pt,line join=round,line cap=round,fill=fillColor] (212.95,508.31) circle (  1.16);

\path[draw=drawColor,line width= 0.4pt,line join=round,line cap=round,fill=fillColor] (213.00,508.31) circle (  1.16);

\path[draw=drawColor,line width= 0.4pt,line join=round,line cap=round,fill=fillColor] (213.04,508.31) circle (  1.16);

\path[draw=drawColor,line width= 0.4pt,line join=round,line cap=round,fill=fillColor] (213.09,508.31) circle (  1.16);

\path[draw=drawColor,line width= 0.4pt,line join=round,line cap=round,fill=fillColor] (213.14,508.31) circle (  1.16);

\path[draw=drawColor,line width= 0.4pt,line join=round,line cap=round,fill=fillColor] (213.18,508.31) circle (  1.16);

\path[draw=drawColor,line width= 0.4pt,line join=round,line cap=round,fill=fillColor] (213.23,508.31) circle (  1.16);

\path[draw=drawColor,line width= 0.4pt,line join=round,line cap=round,fill=fillColor] (213.27,508.31) circle (  1.16);

\path[draw=drawColor,line width= 0.4pt,line join=round,line cap=round,fill=fillColor] (213.32,508.31) circle (  1.16);

\path[draw=drawColor,line width= 0.4pt,line join=round,line cap=round,fill=fillColor] (213.36,508.31) circle (  1.16);

\path[draw=drawColor,line width= 0.4pt,line join=round,line cap=round,fill=fillColor] (213.41,508.31) circle (  1.16);

\path[draw=drawColor,line width= 0.4pt,line join=round,line cap=round,fill=fillColor] (213.46,508.31) circle (  1.16);

\path[draw=drawColor,line width= 0.4pt,line join=round,line cap=round,fill=fillColor] (213.50,508.31) circle (  1.16);

\path[draw=drawColor,line width= 0.4pt,line join=round,line cap=round,fill=fillColor] (213.55,508.31) circle (  1.16);

\path[draw=drawColor,line width= 0.4pt,line join=round,line cap=round,fill=fillColor] (213.59,508.31) circle (  1.16);

\path[draw=drawColor,line width= 0.4pt,line join=round,line cap=round,fill=fillColor] (213.64,508.31) circle (  1.16);

\path[draw=drawColor,line width= 0.4pt,line join=round,line cap=round,fill=fillColor] (213.68,508.31) circle (  1.16);

\path[draw=drawColor,line width= 0.4pt,line join=round,line cap=round,fill=fillColor] (213.73,508.31) circle (  1.16);

\path[draw=drawColor,line width= 0.4pt,line join=round,line cap=round,fill=fillColor] (213.78,508.31) circle (  1.16);

\path[draw=drawColor,line width= 0.4pt,line join=round,line cap=round,fill=fillColor] (213.82,508.31) circle (  1.16);

\path[draw=drawColor,line width= 0.4pt,line join=round,line cap=round,fill=fillColor] (213.87,508.31) circle (  1.16);

\path[draw=drawColor,line width= 0.4pt,line join=round,line cap=round,fill=fillColor] (213.91,508.31) circle (  1.16);

\path[draw=drawColor,line width= 0.4pt,line join=round,line cap=round,fill=fillColor] (213.96,508.31) circle (  1.16);

\path[draw=drawColor,line width= 0.4pt,line join=round,line cap=round,fill=fillColor] (214.00,508.31) circle (  1.16);

\path[draw=drawColor,line width= 0.4pt,line join=round,line cap=round,fill=fillColor] (214.05,508.31) circle (  1.16);

\path[draw=drawColor,line width= 0.4pt,line join=round,line cap=round,fill=fillColor] (214.09,508.31) circle (  1.16);

\path[draw=drawColor,line width= 0.4pt,line join=round,line cap=round,fill=fillColor] (214.14,508.31) circle (  1.16);

\path[draw=drawColor,line width= 0.4pt,line join=round,line cap=round,fill=fillColor] (214.18,508.31) circle (  1.16);

\path[draw=drawColor,line width= 0.4pt,line join=round,line cap=round,fill=fillColor] (214.23,508.31) circle (  1.16);

\path[draw=drawColor,line width= 0.4pt,line join=round,line cap=round,fill=fillColor] (214.27,508.31) circle (  1.16);

\path[draw=drawColor,line width= 0.4pt,line join=round,line cap=round,fill=fillColor] (214.32,508.31) circle (  1.16);

\path[draw=drawColor,line width= 0.4pt,line join=round,line cap=round,fill=fillColor] (214.36,508.31) circle (  1.16);

\path[draw=drawColor,line width= 0.4pt,line join=round,line cap=round,fill=fillColor] (214.41,508.31) circle (  1.16);

\path[draw=drawColor,line width= 0.4pt,line join=round,line cap=round,fill=fillColor] (214.45,508.31) circle (  1.16);

\path[draw=drawColor,line width= 0.4pt,line join=round,line cap=round,fill=fillColor] (214.50,508.31) circle (  1.16);

\path[draw=drawColor,line width= 0.4pt,line join=round,line cap=round,fill=fillColor] (214.54,508.31) circle (  1.16);

\path[draw=drawColor,line width= 0.4pt,line join=round,line cap=round,fill=fillColor] (214.59,508.31) circle (  1.16);

\path[draw=drawColor,line width= 0.4pt,line join=round,line cap=round,fill=fillColor] (214.63,508.31) circle (  1.16);

\path[draw=drawColor,line width= 0.4pt,line join=round,line cap=round,fill=fillColor] (214.68,508.31) circle (  1.16);

\path[draw=drawColor,line width= 0.4pt,line join=round,line cap=round,fill=fillColor] (214.72,508.31) circle (  1.16);

\path[draw=drawColor,line width= 0.4pt,line join=round,line cap=round,fill=fillColor] (214.77,508.31) circle (  1.16);

\path[draw=drawColor,line width= 0.4pt,line join=round,line cap=round,fill=fillColor] (214.81,508.31) circle (  1.16);

\path[draw=drawColor,line width= 0.4pt,line join=round,line cap=round,fill=fillColor] (214.86,508.31) circle (  1.16);

\path[draw=drawColor,line width= 0.4pt,line join=round,line cap=round,fill=fillColor] (214.90,508.31) circle (  1.16);

\path[draw=drawColor,line width= 0.4pt,line join=round,line cap=round,fill=fillColor] (214.95,508.31) circle (  1.16);

\path[draw=drawColor,line width= 0.4pt,line join=round,line cap=round,fill=fillColor] (214.99,508.31) circle (  1.16);

\path[draw=drawColor,line width= 0.4pt,line join=round,line cap=round,fill=fillColor] (215.04,508.31) circle (  1.16);

\path[draw=drawColor,line width= 0.4pt,line join=round,line cap=round,fill=fillColor] (215.08,508.31) circle (  1.16);

\path[draw=drawColor,line width= 0.4pt,line join=round,line cap=round,fill=fillColor] (215.13,508.31) circle (  1.16);

\path[draw=drawColor,line width= 0.4pt,line join=round,line cap=round,fill=fillColor] (215.17,508.31) circle (  1.16);

\path[draw=drawColor,line width= 0.4pt,line join=round,line cap=round,fill=fillColor] (215.22,508.31) circle (  1.16);

\path[draw=drawColor,line width= 0.4pt,line join=round,line cap=round,fill=fillColor] (215.26,508.31) circle (  1.16);

\path[draw=drawColor,line width= 0.4pt,line join=round,line cap=round,fill=fillColor] (215.31,508.31) circle (  1.16);

\path[draw=drawColor,line width= 0.4pt,line join=round,line cap=round,fill=fillColor] (215.35,508.31) circle (  1.16);

\path[draw=drawColor,line width= 0.4pt,line join=round,line cap=round,fill=fillColor] (215.40,508.31) circle (  1.16);

\path[draw=drawColor,line width= 0.4pt,line join=round,line cap=round,fill=fillColor] (215.44,508.31) circle (  1.16);

\path[draw=drawColor,line width= 0.4pt,line join=round,line cap=round,fill=fillColor] (215.48,508.31) circle (  1.16);

\path[draw=drawColor,line width= 0.4pt,line join=round,line cap=round,fill=fillColor] (215.53,508.31) circle (  1.16);

\path[draw=drawColor,line width= 0.4pt,line join=round,line cap=round,fill=fillColor] (215.57,508.31) circle (  1.16);

\path[draw=drawColor,line width= 0.4pt,line join=round,line cap=round,fill=fillColor] (215.62,508.31) circle (  1.16);

\path[draw=drawColor,line width= 0.4pt,line join=round,line cap=round,fill=fillColor] (215.66,508.31) circle (  1.16);

\path[draw=drawColor,line width= 0.4pt,line join=round,line cap=round,fill=fillColor] (215.71,508.31) circle (  1.16);

\path[draw=drawColor,line width= 0.4pt,line join=round,line cap=round,fill=fillColor] (215.75,508.31) circle (  1.16);

\path[draw=drawColor,line width= 0.4pt,line join=round,line cap=round,fill=fillColor] (215.79,508.31) circle (  1.16);

\path[draw=drawColor,line width= 0.4pt,line join=round,line cap=round,fill=fillColor] (215.84,508.31) circle (  1.16);

\path[draw=drawColor,line width= 0.4pt,line join=round,line cap=round,fill=fillColor] (215.88,508.31) circle (  1.16);

\path[draw=drawColor,line width= 0.4pt,line join=round,line cap=round,fill=fillColor] (215.93,508.31) circle (  1.16);

\path[draw=drawColor,line width= 0.4pt,line join=round,line cap=round,fill=fillColor] (215.97,508.31) circle (  1.16);

\path[draw=drawColor,line width= 0.4pt,line join=round,line cap=round,fill=fillColor] (216.02,508.31) circle (  1.16);

\path[draw=drawColor,line width= 0.4pt,line join=round,line cap=round,fill=fillColor] (216.06,508.31) circle (  1.16);

\path[draw=drawColor,line width= 0.4pt,line join=round,line cap=round,fill=fillColor] (216.10,508.31) circle (  1.16);

\path[draw=drawColor,line width= 0.4pt,line join=round,line cap=round,fill=fillColor] (216.15,508.31) circle (  1.16);

\path[draw=drawColor,line width= 0.4pt,line join=round,line cap=round,fill=fillColor] (216.19,508.31) circle (  1.16);

\path[draw=drawColor,line width= 0.4pt,line join=round,line cap=round,fill=fillColor] (216.24,508.31) circle (  1.16);

\path[draw=drawColor,line width= 0.4pt,line join=round,line cap=round,fill=fillColor] (216.28,508.31) circle (  1.16);

\path[draw=drawColor,line width= 0.4pt,line join=round,line cap=round,fill=fillColor] (216.32,508.31) circle (  1.16);

\path[draw=drawColor,line width= 0.4pt,line join=round,line cap=round,fill=fillColor] (216.37,508.31) circle (  1.16);

\path[draw=drawColor,line width= 0.4pt,line join=round,line cap=round,fill=fillColor] (216.41,508.31) circle (  1.16);

\path[draw=drawColor,line width= 0.4pt,line join=round,line cap=round,fill=fillColor] (216.46,508.31) circle (  1.16);

\path[draw=drawColor,line width= 0.4pt,line join=round,line cap=round,fill=fillColor] (216.50,508.31) circle (  1.16);

\path[draw=drawColor,line width= 0.4pt,line join=round,line cap=round,fill=fillColor] (216.54,508.31) circle (  1.16);

\path[draw=drawColor,line width= 0.4pt,line join=round,line cap=round,fill=fillColor] (216.59,508.31) circle (  1.16);

\path[draw=drawColor,line width= 0.4pt,line join=round,line cap=round,fill=fillColor] (216.63,508.31) circle (  1.16);

\path[draw=drawColor,line width= 0.4pt,line join=round,line cap=round,fill=fillColor] (216.68,508.31) circle (  1.16);

\path[draw=drawColor,line width= 0.4pt,line join=round,line cap=round,fill=fillColor] (216.72,508.31) circle (  1.16);

\path[draw=drawColor,line width= 0.4pt,line join=round,line cap=round,fill=fillColor] (216.76,508.31) circle (  1.16);

\path[draw=drawColor,line width= 0.4pt,line join=round,line cap=round,fill=fillColor] (216.81,508.31) circle (  1.16);

\path[draw=drawColor,line width= 0.4pt,line join=round,line cap=round,fill=fillColor] (216.85,508.31) circle (  1.16);
\definecolor[named]{drawColor}{rgb}{0.30,0.69,0.29}
\definecolor[named]{fillColor}{rgb}{0.30,0.69,0.29}

\path[draw=drawColor,line width= 0.4pt,line join=round,line cap=round,fill=fillColor] ( 81.22,563.62) circle (  1.16);

\path[draw=drawColor,line width= 0.4pt,line join=round,line cap=round,fill=fillColor] ( 84.95,546.41) circle (  1.16);

\path[draw=drawColor,line width= 0.4pt,line join=round,line cap=round,fill=fillColor] ( 87.56,538.46) circle (  1.16);

\path[draw=drawColor,line width= 0.4pt,line join=round,line cap=round,fill=fillColor] ( 89.64,537.13) circle (  1.16);

\path[draw=drawColor,line width= 0.4pt,line join=round,line cap=round,fill=fillColor] ( 91.40,535.89) circle (  1.16);

\path[draw=drawColor,line width= 0.4pt,line join=round,line cap=round,fill=fillColor] ( 92.94,534.64) circle (  1.16);

\path[draw=drawColor,line width= 0.4pt,line join=round,line cap=round,fill=fillColor] ( 94.31,534.37) circle (  1.16);

\path[draw=drawColor,line width= 0.4pt,line join=round,line cap=round,fill=fillColor] ( 95.56,533.18) circle (  1.16);

\path[draw=drawColor,line width= 0.4pt,line join=round,line cap=round,fill=fillColor] ( 96.71,533.00) circle (  1.16);

\path[draw=drawColor,line width= 0.4pt,line join=round,line cap=round,fill=fillColor] ( 97.77,532.69) circle (  1.16);

\path[draw=drawColor,line width= 0.4pt,line join=round,line cap=round,fill=fillColor] ( 98.77,532.44) circle (  1.16);

\path[draw=drawColor,line width= 0.4pt,line join=round,line cap=round,fill=fillColor] ( 99.71,531.60) circle (  1.16);

\path[draw=drawColor,line width= 0.4pt,line join=round,line cap=round,fill=fillColor] (100.59,531.53) circle (  1.16);

\path[draw=drawColor,line width= 0.4pt,line join=round,line cap=round,fill=fillColor] (101.44,531.53) circle (  1.16);

\path[draw=drawColor,line width= 0.4pt,line join=round,line cap=round,fill=fillColor] (102.24,531.30) circle (  1.16);

\path[draw=drawColor,line width= 0.4pt,line join=round,line cap=round,fill=fillColor] (103.01,531.24) circle (  1.16);

\path[draw=drawColor,line width= 0.4pt,line join=round,line cap=round,fill=fillColor] (103.75,530.80) circle (  1.16);

\path[draw=drawColor,line width= 0.4pt,line join=round,line cap=round,fill=fillColor] (104.46,530.65) circle (  1.16);

\path[draw=drawColor,line width= 0.4pt,line join=round,line cap=round,fill=fillColor] (105.14,530.58) circle (  1.16);

\path[draw=drawColor,line width= 0.4pt,line join=round,line cap=round,fill=fillColor] (105.80,530.35) circle (  1.16);

\path[draw=drawColor,line width= 0.4pt,line join=round,line cap=round,fill=fillColor] (106.44,530.30) circle (  1.16);

\path[draw=drawColor,line width= 0.4pt,line join=round,line cap=round,fill=fillColor] (107.05,530.28) circle (  1.16);

\path[draw=drawColor,line width= 0.4pt,line join=round,line cap=round,fill=fillColor] (107.65,530.26) circle (  1.16);

\path[draw=drawColor,line width= 0.4pt,line join=round,line cap=round,fill=fillColor] (108.24,530.15) circle (  1.16);

\path[draw=drawColor,line width= 0.4pt,line join=round,line cap=round,fill=fillColor] (108.80,530.05) circle (  1.16);

\path[draw=drawColor,line width= 0.4pt,line join=round,line cap=round,fill=fillColor] (109.35,529.51) circle (  1.16);

\path[draw=drawColor,line width= 0.4pt,line join=round,line cap=round,fill=fillColor] (109.89,529.29) circle (  1.16);

\path[draw=drawColor,line width= 0.4pt,line join=round,line cap=round,fill=fillColor] (110.42,529.27) circle (  1.16);

\path[draw=drawColor,line width= 0.4pt,line join=round,line cap=round,fill=fillColor] (110.93,529.21) circle (  1.16);

\path[draw=drawColor,line width= 0.4pt,line join=round,line cap=round,fill=fillColor] (111.43,529.21) circle (  1.16);

\path[draw=drawColor,line width= 0.4pt,line join=round,line cap=round,fill=fillColor] (111.92,529.19) circle (  1.16);

\path[draw=drawColor,line width= 0.4pt,line join=round,line cap=round,fill=fillColor] (112.40,529.01) circle (  1.16);

\path[draw=drawColor,line width= 0.4pt,line join=round,line cap=round,fill=fillColor] (112.87,528.92) circle (  1.16);

\path[draw=drawColor,line width= 0.4pt,line join=round,line cap=round,fill=fillColor] (113.33,528.76) circle (  1.16);

\path[draw=drawColor,line width= 0.4pt,line join=round,line cap=round,fill=fillColor] (113.78,528.66) circle (  1.16);

\path[draw=drawColor,line width= 0.4pt,line join=round,line cap=round,fill=fillColor] (114.22,528.49) circle (  1.16);

\path[draw=drawColor,line width= 0.4pt,line join=round,line cap=round,fill=fillColor] (114.65,528.37) circle (  1.16);

\path[draw=drawColor,line width= 0.4pt,line join=round,line cap=round,fill=fillColor] (115.08,528.35) circle (  1.16);

\path[draw=drawColor,line width= 0.4pt,line join=round,line cap=round,fill=fillColor] (115.50,528.02) circle (  1.16);

\path[draw=drawColor,line width= 0.4pt,line join=round,line cap=round,fill=fillColor] (115.91,527.73) circle (  1.16);

\path[draw=drawColor,line width= 0.4pt,line join=round,line cap=round,fill=fillColor] (116.32,527.67) circle (  1.16);

\path[draw=drawColor,line width= 0.4pt,line join=round,line cap=round,fill=fillColor] (116.72,527.67) circle (  1.16);

\path[draw=drawColor,line width= 0.4pt,line join=round,line cap=round,fill=fillColor] (117.11,527.61) circle (  1.16);

\path[draw=drawColor,line width= 0.4pt,line join=round,line cap=round,fill=fillColor] (117.49,527.50) circle (  1.16);

\path[draw=drawColor,line width= 0.4pt,line join=round,line cap=round,fill=fillColor] (117.87,527.50) circle (  1.16);

\path[draw=drawColor,line width= 0.4pt,line join=round,line cap=round,fill=fillColor] (118.25,527.47) circle (  1.16);

\path[draw=drawColor,line width= 0.4pt,line join=round,line cap=round,fill=fillColor] (118.62,527.34) circle (  1.16);

\path[draw=drawColor,line width= 0.4pt,line join=round,line cap=round,fill=fillColor] (118.98,527.26) circle (  1.16);

\path[draw=drawColor,line width= 0.4pt,line join=round,line cap=round,fill=fillColor] (119.34,527.22) circle (  1.16);

\path[draw=drawColor,line width= 0.4pt,line join=round,line cap=round,fill=fillColor] (119.70,527.14) circle (  1.16);

\path[draw=drawColor,line width= 0.4pt,line join=round,line cap=round,fill=fillColor] (120.05,527.13) circle (  1.16);

\path[draw=drawColor,line width= 0.4pt,line join=round,line cap=round,fill=fillColor] (120.39,527.08) circle (  1.16);

\path[draw=drawColor,line width= 0.4pt,line join=round,line cap=round,fill=fillColor] (120.73,527.03) circle (  1.16);

\path[draw=drawColor,line width= 0.4pt,line join=round,line cap=round,fill=fillColor] (121.07,527.02) circle (  1.16);

\path[draw=drawColor,line width= 0.4pt,line join=round,line cap=round,fill=fillColor] (121.40,526.93) circle (  1.16);

\path[draw=drawColor,line width= 0.4pt,line join=round,line cap=round,fill=fillColor] (121.73,526.88) circle (  1.16);

\path[draw=drawColor,line width= 0.4pt,line join=round,line cap=round,fill=fillColor] (122.05,526.85) circle (  1.16);

\path[draw=drawColor,line width= 0.4pt,line join=round,line cap=round,fill=fillColor] (122.38,526.73) circle (  1.16);

\path[draw=drawColor,line width= 0.4pt,line join=round,line cap=round,fill=fillColor] (122.69,526.65) circle (  1.16);

\path[draw=drawColor,line width= 0.4pt,line join=round,line cap=round,fill=fillColor] (123.01,526.42) circle (  1.16);

\path[draw=drawColor,line width= 0.4pt,line join=round,line cap=round,fill=fillColor] (123.32,526.24) circle (  1.16);

\path[draw=drawColor,line width= 0.4pt,line join=round,line cap=round,fill=fillColor] (123.62,526.18) circle (  1.16);

\path[draw=drawColor,line width= 0.4pt,line join=round,line cap=round,fill=fillColor] (123.93,525.98) circle (  1.16);

\path[draw=drawColor,line width= 0.4pt,line join=round,line cap=round,fill=fillColor] (124.23,525.83) circle (  1.16);

\path[draw=drawColor,line width= 0.4pt,line join=round,line cap=round,fill=fillColor] (124.52,525.79) circle (  1.16);

\path[draw=drawColor,line width= 0.4pt,line join=round,line cap=round,fill=fillColor] (124.82,525.69) circle (  1.16);

\path[draw=drawColor,line width= 0.4pt,line join=round,line cap=round,fill=fillColor] (125.11,525.54) circle (  1.16);

\path[draw=drawColor,line width= 0.4pt,line join=round,line cap=round,fill=fillColor] (125.40,525.51) circle (  1.16);

\path[draw=drawColor,line width= 0.4pt,line join=round,line cap=round,fill=fillColor] (125.68,525.50) circle (  1.16);

\path[draw=drawColor,line width= 0.4pt,line join=round,line cap=round,fill=fillColor] (125.96,525.49) circle (  1.16);

\path[draw=drawColor,line width= 0.4pt,line join=round,line cap=round,fill=fillColor] (126.24,525.49) circle (  1.16);

\path[draw=drawColor,line width= 0.4pt,line join=round,line cap=round,fill=fillColor] (126.52,525.47) circle (  1.16);

\path[draw=drawColor,line width= 0.4pt,line join=round,line cap=round,fill=fillColor] (126.80,525.35) circle (  1.16);

\path[draw=drawColor,line width= 0.4pt,line join=round,line cap=round,fill=fillColor] (127.07,525.33) circle (  1.16);

\path[draw=drawColor,line width= 0.4pt,line join=round,line cap=round,fill=fillColor] (127.34,525.33) circle (  1.16);

\path[draw=drawColor,line width= 0.4pt,line join=round,line cap=round,fill=fillColor] (127.61,525.31) circle (  1.16);

\path[draw=drawColor,line width= 0.4pt,line join=round,line cap=round,fill=fillColor] (127.87,525.27) circle (  1.16);

\path[draw=drawColor,line width= 0.4pt,line join=round,line cap=round,fill=fillColor] (128.13,525.26) circle (  1.16);

\path[draw=drawColor,line width= 0.4pt,line join=round,line cap=round,fill=fillColor] (128.40,525.26) circle (  1.16);

\path[draw=drawColor,line width= 0.4pt,line join=round,line cap=round,fill=fillColor] (128.65,525.19) circle (  1.16);

\path[draw=drawColor,line width= 0.4pt,line join=round,line cap=round,fill=fillColor] (128.91,525.15) circle (  1.16);

\path[draw=drawColor,line width= 0.4pt,line join=round,line cap=round,fill=fillColor] (129.16,525.15) circle (  1.16);

\path[draw=drawColor,line width= 0.4pt,line join=round,line cap=round,fill=fillColor] (129.42,525.08) circle (  1.16);

\path[draw=drawColor,line width= 0.4pt,line join=round,line cap=round,fill=fillColor] (129.67,525.06) circle (  1.16);

\path[draw=drawColor,line width= 0.4pt,line join=round,line cap=round,fill=fillColor] (129.91,525.04) circle (  1.16);

\path[draw=drawColor,line width= 0.4pt,line join=round,line cap=round,fill=fillColor] (130.16,524.92) circle (  1.16);

\path[draw=drawColor,line width= 0.4pt,line join=round,line cap=round,fill=fillColor] (130.41,524.91) circle (  1.16);

\path[draw=drawColor,line width= 0.4pt,line join=round,line cap=round,fill=fillColor] (130.65,524.87) circle (  1.16);

\path[draw=drawColor,line width= 0.4pt,line join=round,line cap=round,fill=fillColor] (130.89,524.79) circle (  1.16);

\path[draw=drawColor,line width= 0.4pt,line join=round,line cap=round,fill=fillColor] (131.13,524.73) circle (  1.16);

\path[draw=drawColor,line width= 0.4pt,line join=round,line cap=round,fill=fillColor] (131.36,524.69) circle (  1.16);

\path[draw=drawColor,line width= 0.4pt,line join=round,line cap=round,fill=fillColor] (131.60,524.66) circle (  1.16);

\path[draw=drawColor,line width= 0.4pt,line join=round,line cap=round,fill=fillColor] (131.83,524.62) circle (  1.16);

\path[draw=drawColor,line width= 0.4pt,line join=round,line cap=round,fill=fillColor] (132.06,524.61) circle (  1.16);

\path[draw=drawColor,line width= 0.4pt,line join=round,line cap=round,fill=fillColor] (132.30,524.56) circle (  1.16);

\path[draw=drawColor,line width= 0.4pt,line join=round,line cap=round,fill=fillColor] (132.52,524.53) circle (  1.16);

\path[draw=drawColor,line width= 0.4pt,line join=round,line cap=round,fill=fillColor] (132.75,524.28) circle (  1.16);

\path[draw=drawColor,line width= 0.4pt,line join=round,line cap=round,fill=fillColor] (132.98,524.28) circle (  1.16);

\path[draw=drawColor,line width= 0.4pt,line join=round,line cap=round,fill=fillColor] (133.20,524.27) circle (  1.16);

\path[draw=drawColor,line width= 0.4pt,line join=round,line cap=round,fill=fillColor] (133.42,524.25) circle (  1.16);

\path[draw=drawColor,line width= 0.4pt,line join=round,line cap=round,fill=fillColor] (133.64,524.17) circle (  1.16);

\path[draw=drawColor,line width= 0.4pt,line join=round,line cap=round,fill=fillColor] (133.86,524.17) circle (  1.16);

\path[draw=drawColor,line width= 0.4pt,line join=round,line cap=round,fill=fillColor] (134.08,524.08) circle (  1.16);

\path[draw=drawColor,line width= 0.4pt,line join=round,line cap=round,fill=fillColor] (134.30,524.05) circle (  1.16);

\path[draw=drawColor,line width= 0.4pt,line join=round,line cap=round,fill=fillColor] (134.51,524.04) circle (  1.16);

\path[draw=drawColor,line width= 0.4pt,line join=round,line cap=round,fill=fillColor] (134.73,523.98) circle (  1.16);

\path[draw=drawColor,line width= 0.4pt,line join=round,line cap=round,fill=fillColor] (134.94,523.89) circle (  1.16);

\path[draw=drawColor,line width= 0.4pt,line join=round,line cap=round,fill=fillColor] (135.15,523.87) circle (  1.16);

\path[draw=drawColor,line width= 0.4pt,line join=round,line cap=round,fill=fillColor] (135.36,523.87) circle (  1.16);

\path[draw=drawColor,line width= 0.4pt,line join=round,line cap=round,fill=fillColor] (135.57,523.87) circle (  1.16);

\path[draw=drawColor,line width= 0.4pt,line join=round,line cap=round,fill=fillColor] (135.78,523.86) circle (  1.16);

\path[draw=drawColor,line width= 0.4pt,line join=round,line cap=round,fill=fillColor] (135.98,523.84) circle (  1.16);

\path[draw=drawColor,line width= 0.4pt,line join=round,line cap=round,fill=fillColor] (136.19,523.83) circle (  1.16);

\path[draw=drawColor,line width= 0.4pt,line join=round,line cap=round,fill=fillColor] (136.39,523.83) circle (  1.16);

\path[draw=drawColor,line width= 0.4pt,line join=round,line cap=round,fill=fillColor] (136.60,523.82) circle (  1.16);

\path[draw=drawColor,line width= 0.4pt,line join=round,line cap=round,fill=fillColor] (136.80,523.81) circle (  1.16);

\path[draw=drawColor,line width= 0.4pt,line join=round,line cap=round,fill=fillColor] (137.00,523.75) circle (  1.16);

\path[draw=drawColor,line width= 0.4pt,line join=round,line cap=round,fill=fillColor] (137.20,523.75) circle (  1.16);

\path[draw=drawColor,line width= 0.4pt,line join=round,line cap=round,fill=fillColor] (137.40,523.72) circle (  1.16);

\path[draw=drawColor,line width= 0.4pt,line join=round,line cap=round,fill=fillColor] (137.59,523.68) circle (  1.16);

\path[draw=drawColor,line width= 0.4pt,line join=round,line cap=round,fill=fillColor] (137.79,523.68) circle (  1.16);

\path[draw=drawColor,line width= 0.4pt,line join=round,line cap=round,fill=fillColor] (137.98,523.66) circle (  1.16);

\path[draw=drawColor,line width= 0.4pt,line join=round,line cap=round,fill=fillColor] (138.18,523.63) circle (  1.16);

\path[draw=drawColor,line width= 0.4pt,line join=round,line cap=round,fill=fillColor] (138.37,523.60) circle (  1.16);

\path[draw=drawColor,line width= 0.4pt,line join=round,line cap=round,fill=fillColor] (138.56,523.57) circle (  1.16);

\path[draw=drawColor,line width= 0.4pt,line join=round,line cap=round,fill=fillColor] (138.75,523.56) circle (  1.16);

\path[draw=drawColor,line width= 0.4pt,line join=round,line cap=round,fill=fillColor] (138.94,523.55) circle (  1.16);

\path[draw=drawColor,line width= 0.4pt,line join=round,line cap=round,fill=fillColor] (139.13,523.52) circle (  1.16);

\path[draw=drawColor,line width= 0.4pt,line join=round,line cap=round,fill=fillColor] (139.32,523.52) circle (  1.16);

\path[draw=drawColor,line width= 0.4pt,line join=round,line cap=round,fill=fillColor] (139.50,523.51) circle (  1.16);

\path[draw=drawColor,line width= 0.4pt,line join=round,line cap=round,fill=fillColor] (139.69,523.49) circle (  1.16);

\path[draw=drawColor,line width= 0.4pt,line join=round,line cap=round,fill=fillColor] (139.87,523.48) circle (  1.16);

\path[draw=drawColor,line width= 0.4pt,line join=round,line cap=round,fill=fillColor] (140.06,523.45) circle (  1.16);

\path[draw=drawColor,line width= 0.4pt,line join=round,line cap=round,fill=fillColor] (140.24,523.45) circle (  1.16);

\path[draw=drawColor,line width= 0.4pt,line join=round,line cap=round,fill=fillColor] (140.42,523.44) circle (  1.16);

\path[draw=drawColor,line width= 0.4pt,line join=round,line cap=round,fill=fillColor] (140.60,523.37) circle (  1.16);

\path[draw=drawColor,line width= 0.4pt,line join=round,line cap=round,fill=fillColor] (140.78,523.36) circle (  1.16);

\path[draw=drawColor,line width= 0.4pt,line join=round,line cap=round,fill=fillColor] (140.96,523.33) circle (  1.16);

\path[draw=drawColor,line width= 0.4pt,line join=round,line cap=round,fill=fillColor] (141.14,523.30) circle (  1.16);

\path[draw=drawColor,line width= 0.4pt,line join=round,line cap=round,fill=fillColor] (141.32,523.30) circle (  1.16);

\path[draw=drawColor,line width= 0.4pt,line join=round,line cap=round,fill=fillColor] (141.50,523.22) circle (  1.16);

\path[draw=drawColor,line width= 0.4pt,line join=round,line cap=round,fill=fillColor] (141.67,523.18) circle (  1.16);

\path[draw=drawColor,line width= 0.4pt,line join=round,line cap=round,fill=fillColor] (141.85,523.11) circle (  1.16);

\path[draw=drawColor,line width= 0.4pt,line join=round,line cap=round,fill=fillColor] (142.02,523.08) circle (  1.16);

\path[draw=drawColor,line width= 0.4pt,line join=round,line cap=round,fill=fillColor] (142.20,523.08) circle (  1.16);

\path[draw=drawColor,line width= 0.4pt,line join=round,line cap=round,fill=fillColor] (142.37,523.05) circle (  1.16);

\path[draw=drawColor,line width= 0.4pt,line join=round,line cap=round,fill=fillColor] (142.54,523.03) circle (  1.16);

\path[draw=drawColor,line width= 0.4pt,line join=round,line cap=round,fill=fillColor] (142.71,522.99) circle (  1.16);

\path[draw=drawColor,line width= 0.4pt,line join=round,line cap=round,fill=fillColor] (142.88,522.99) circle (  1.16);

\path[draw=drawColor,line width= 0.4pt,line join=round,line cap=round,fill=fillColor] (143.05,522.99) circle (  1.16);

\path[draw=drawColor,line width= 0.4pt,line join=round,line cap=round,fill=fillColor] (143.22,522.98) circle (  1.16);

\path[draw=drawColor,line width= 0.4pt,line join=round,line cap=round,fill=fillColor] (143.39,522.95) circle (  1.16);

\path[draw=drawColor,line width= 0.4pt,line join=round,line cap=round,fill=fillColor] (143.56,522.86) circle (  1.16);

\path[draw=drawColor,line width= 0.4pt,line join=round,line cap=round,fill=fillColor] (143.72,522.77) circle (  1.16);

\path[draw=drawColor,line width= 0.4pt,line join=round,line cap=round,fill=fillColor] (143.89,522.77) circle (  1.16);

\path[draw=drawColor,line width= 0.4pt,line join=round,line cap=round,fill=fillColor] (144.05,522.73) circle (  1.16);

\path[draw=drawColor,line width= 0.4pt,line join=round,line cap=round,fill=fillColor] (144.22,522.71) circle (  1.16);

\path[draw=drawColor,line width= 0.4pt,line join=round,line cap=round,fill=fillColor] (144.38,522.70) circle (  1.16);

\path[draw=drawColor,line width= 0.4pt,line join=round,line cap=round,fill=fillColor] (144.55,522.68) circle (  1.16);

\path[draw=drawColor,line width= 0.4pt,line join=round,line cap=round,fill=fillColor] (144.71,522.64) circle (  1.16);

\path[draw=drawColor,line width= 0.4pt,line join=round,line cap=round,fill=fillColor] (144.87,522.60) circle (  1.16);

\path[draw=drawColor,line width= 0.4pt,line join=round,line cap=round,fill=fillColor] (145.03,522.58) circle (  1.16);

\path[draw=drawColor,line width= 0.4pt,line join=round,line cap=round,fill=fillColor] (145.19,522.57) circle (  1.16);

\path[draw=drawColor,line width= 0.4pt,line join=round,line cap=round,fill=fillColor] (145.35,522.52) circle (  1.16);

\path[draw=drawColor,line width= 0.4pt,line join=round,line cap=round,fill=fillColor] (145.51,522.51) circle (  1.16);

\path[draw=drawColor,line width= 0.4pt,line join=round,line cap=round,fill=fillColor] (145.67,522.50) circle (  1.16);

\path[draw=drawColor,line width= 0.4pt,line join=round,line cap=round,fill=fillColor] (145.83,522.46) circle (  1.16);

\path[draw=drawColor,line width= 0.4pt,line join=round,line cap=round,fill=fillColor] (145.98,522.45) circle (  1.16);

\path[draw=drawColor,line width= 0.4pt,line join=round,line cap=round,fill=fillColor] (146.14,522.44) circle (  1.16);

\path[draw=drawColor,line width= 0.4pt,line join=round,line cap=round,fill=fillColor] (146.30,522.43) circle (  1.16);

\path[draw=drawColor,line width= 0.4pt,line join=round,line cap=round,fill=fillColor] (146.45,522.40) circle (  1.16);

\path[draw=drawColor,line width= 0.4pt,line join=round,line cap=round,fill=fillColor] (146.61,522.36) circle (  1.16);

\path[draw=drawColor,line width= 0.4pt,line join=round,line cap=round,fill=fillColor] (146.76,522.34) circle (  1.16);

\path[draw=drawColor,line width= 0.4pt,line join=round,line cap=round,fill=fillColor] (146.91,522.34) circle (  1.16);

\path[draw=drawColor,line width= 0.4pt,line join=round,line cap=round,fill=fillColor] (147.07,522.34) circle (  1.16);

\path[draw=drawColor,line width= 0.4pt,line join=round,line cap=round,fill=fillColor] (147.22,522.34) circle (  1.16);

\path[draw=drawColor,line width= 0.4pt,line join=round,line cap=round,fill=fillColor] (147.37,522.33) circle (  1.16);

\path[draw=drawColor,line width= 0.4pt,line join=round,line cap=round,fill=fillColor] (147.52,522.33) circle (  1.16);

\path[draw=drawColor,line width= 0.4pt,line join=round,line cap=round,fill=fillColor] (147.67,522.31) circle (  1.16);

\path[draw=drawColor,line width= 0.4pt,line join=round,line cap=round,fill=fillColor] (147.82,522.31) circle (  1.16);

\path[draw=drawColor,line width= 0.4pt,line join=round,line cap=round,fill=fillColor] (147.97,522.28) circle (  1.16);

\path[draw=drawColor,line width= 0.4pt,line join=round,line cap=round,fill=fillColor] (148.12,522.27) circle (  1.16);

\path[draw=drawColor,line width= 0.4pt,line join=round,line cap=round,fill=fillColor] (148.27,522.26) circle (  1.16);

\path[draw=drawColor,line width= 0.4pt,line join=round,line cap=round,fill=fillColor] (148.42,522.25) circle (  1.16);

\path[draw=drawColor,line width= 0.4pt,line join=round,line cap=round,fill=fillColor] (148.57,522.23) circle (  1.16);

\path[draw=drawColor,line width= 0.4pt,line join=round,line cap=round,fill=fillColor] (148.71,522.21) circle (  1.16);

\path[draw=drawColor,line width= 0.4pt,line join=round,line cap=round,fill=fillColor] (148.86,522.13) circle (  1.16);

\path[draw=drawColor,line width= 0.4pt,line join=round,line cap=round,fill=fillColor] (149.01,522.12) circle (  1.16);

\path[draw=drawColor,line width= 0.4pt,line join=round,line cap=round,fill=fillColor] (149.15,522.10) circle (  1.16);

\path[draw=drawColor,line width= 0.4pt,line join=round,line cap=round,fill=fillColor] (149.30,522.10) circle (  1.16);

\path[draw=drawColor,line width= 0.4pt,line join=round,line cap=round,fill=fillColor] (149.44,522.09) circle (  1.16);

\path[draw=drawColor,line width= 0.4pt,line join=round,line cap=round,fill=fillColor] (149.58,522.05) circle (  1.16);

\path[draw=drawColor,line width= 0.4pt,line join=round,line cap=round,fill=fillColor] (149.73,522.05) circle (  1.16);

\path[draw=drawColor,line width= 0.4pt,line join=round,line cap=round,fill=fillColor] (149.87,522.03) circle (  1.16);

\path[draw=drawColor,line width= 0.4pt,line join=round,line cap=round,fill=fillColor] (150.01,522.02) circle (  1.16);

\path[draw=drawColor,line width= 0.4pt,line join=round,line cap=round,fill=fillColor] (150.15,522.02) circle (  1.16);

\path[draw=drawColor,line width= 0.4pt,line join=round,line cap=round,fill=fillColor] (150.30,522.01) circle (  1.16);

\path[draw=drawColor,line width= 0.4pt,line join=round,line cap=round,fill=fillColor] (150.44,521.97) circle (  1.16);

\path[draw=drawColor,line width= 0.4pt,line join=round,line cap=round,fill=fillColor] (150.58,521.97) circle (  1.16);

\path[draw=drawColor,line width= 0.4pt,line join=round,line cap=round,fill=fillColor] (150.72,521.97) circle (  1.16);

\path[draw=drawColor,line width= 0.4pt,line join=round,line cap=round,fill=fillColor] (150.86,521.94) circle (  1.16);

\path[draw=drawColor,line width= 0.4pt,line join=round,line cap=round,fill=fillColor] (151.00,521.93) circle (  1.16);

\path[draw=drawColor,line width= 0.4pt,line join=round,line cap=round,fill=fillColor] (151.13,521.93) circle (  1.16);

\path[draw=drawColor,line width= 0.4pt,line join=round,line cap=round,fill=fillColor] (151.27,521.87) circle (  1.16);

\path[draw=drawColor,line width= 0.4pt,line join=round,line cap=round,fill=fillColor] (151.41,521.82) circle (  1.16);

\path[draw=drawColor,line width= 0.4pt,line join=round,line cap=round,fill=fillColor] (151.55,521.81) circle (  1.16);

\path[draw=drawColor,line width= 0.4pt,line join=round,line cap=round,fill=fillColor] (151.68,521.81) circle (  1.16);

\path[draw=drawColor,line width= 0.4pt,line join=round,line cap=round,fill=fillColor] (151.82,521.79) circle (  1.16);

\path[draw=drawColor,line width= 0.4pt,line join=round,line cap=round,fill=fillColor] (151.96,521.77) circle (  1.16);

\path[draw=drawColor,line width= 0.4pt,line join=round,line cap=round,fill=fillColor] (152.09,521.76) circle (  1.16);

\path[draw=drawColor,line width= 0.4pt,line join=round,line cap=round,fill=fillColor] (152.23,521.74) circle (  1.16);

\path[draw=drawColor,line width= 0.4pt,line join=round,line cap=round,fill=fillColor] (152.36,521.72) circle (  1.16);

\path[draw=drawColor,line width= 0.4pt,line join=round,line cap=round,fill=fillColor] (152.49,521.70) circle (  1.16);

\path[draw=drawColor,line width= 0.4pt,line join=round,line cap=round,fill=fillColor] (152.63,521.67) circle (  1.16);

\path[draw=drawColor,line width= 0.4pt,line join=round,line cap=round,fill=fillColor] (152.76,521.65) circle (  1.16);

\path[draw=drawColor,line width= 0.4pt,line join=round,line cap=round,fill=fillColor] (152.89,521.64) circle (  1.16);

\path[draw=drawColor,line width= 0.4pt,line join=round,line cap=round,fill=fillColor] (153.03,521.64) circle (  1.16);

\path[draw=drawColor,line width= 0.4pt,line join=round,line cap=round,fill=fillColor] (153.16,521.63) circle (  1.16);

\path[draw=drawColor,line width= 0.4pt,line join=round,line cap=round,fill=fillColor] (153.29,521.58) circle (  1.16);

\path[draw=drawColor,line width= 0.4pt,line join=round,line cap=round,fill=fillColor] (153.42,521.58) circle (  1.16);

\path[draw=drawColor,line width= 0.4pt,line join=round,line cap=round,fill=fillColor] (153.55,521.55) circle (  1.16);

\path[draw=drawColor,line width= 0.4pt,line join=round,line cap=round,fill=fillColor] (153.68,521.55) circle (  1.16);

\path[draw=drawColor,line width= 0.4pt,line join=round,line cap=round,fill=fillColor] (153.81,521.53) circle (  1.16);

\path[draw=drawColor,line width= 0.4pt,line join=round,line cap=round,fill=fillColor] (153.94,521.51) circle (  1.16);

\path[draw=drawColor,line width= 0.4pt,line join=round,line cap=round,fill=fillColor] (154.07,521.51) circle (  1.16);

\path[draw=drawColor,line width= 0.4pt,line join=round,line cap=round,fill=fillColor] (154.20,521.51) circle (  1.16);

\path[draw=drawColor,line width= 0.4pt,line join=round,line cap=round,fill=fillColor] (154.33,521.50) circle (  1.16);

\path[draw=drawColor,line width= 0.4pt,line join=round,line cap=round,fill=fillColor] (154.46,521.49) circle (  1.16);

\path[draw=drawColor,line width= 0.4pt,line join=round,line cap=round,fill=fillColor] (154.59,521.49) circle (  1.16);

\path[draw=drawColor,line width= 0.4pt,line join=round,line cap=round,fill=fillColor] (154.71,521.45) circle (  1.16);

\path[draw=drawColor,line width= 0.4pt,line join=round,line cap=round,fill=fillColor] (154.84,521.39) circle (  1.16);

\path[draw=drawColor,line width= 0.4pt,line join=round,line cap=round,fill=fillColor] (154.97,521.38) circle (  1.16);

\path[draw=drawColor,line width= 0.4pt,line join=round,line cap=round,fill=fillColor] (155.09,521.32) circle (  1.16);

\path[draw=drawColor,line width= 0.4pt,line join=round,line cap=round,fill=fillColor] (155.22,521.31) circle (  1.16);

\path[draw=drawColor,line width= 0.4pt,line join=round,line cap=round,fill=fillColor] (155.35,521.31) circle (  1.16);

\path[draw=drawColor,line width= 0.4pt,line join=round,line cap=round,fill=fillColor] (155.47,521.29) circle (  1.16);

\path[draw=drawColor,line width= 0.4pt,line join=round,line cap=round,fill=fillColor] (155.60,521.29) circle (  1.16);

\path[draw=drawColor,line width= 0.4pt,line join=round,line cap=round,fill=fillColor] (155.72,521.26) circle (  1.16);

\path[draw=drawColor,line width= 0.4pt,line join=round,line cap=round,fill=fillColor] (155.85,521.26) circle (  1.16);

\path[draw=drawColor,line width= 0.4pt,line join=round,line cap=round,fill=fillColor] (155.97,521.25) circle (  1.16);

\path[draw=drawColor,line width= 0.4pt,line join=round,line cap=round,fill=fillColor] (156.09,521.25) circle (  1.16);

\path[draw=drawColor,line width= 0.4pt,line join=round,line cap=round,fill=fillColor] (156.22,521.24) circle (  1.16);

\path[draw=drawColor,line width= 0.4pt,line join=round,line cap=round,fill=fillColor] (156.34,521.23) circle (  1.16);

\path[draw=drawColor,line width= 0.4pt,line join=round,line cap=round,fill=fillColor] (156.46,521.19) circle (  1.16);

\path[draw=drawColor,line width= 0.4pt,line join=round,line cap=round,fill=fillColor] (156.58,521.19) circle (  1.16);

\path[draw=drawColor,line width= 0.4pt,line join=round,line cap=round,fill=fillColor] (156.71,521.19) circle (  1.16);

\path[draw=drawColor,line width= 0.4pt,line join=round,line cap=round,fill=fillColor] (156.83,521.16) circle (  1.16);

\path[draw=drawColor,line width= 0.4pt,line join=round,line cap=round,fill=fillColor] (156.95,521.12) circle (  1.16);

\path[draw=drawColor,line width= 0.4pt,line join=round,line cap=round,fill=fillColor] (157.07,521.09) circle (  1.16);

\path[draw=drawColor,line width= 0.4pt,line join=round,line cap=round,fill=fillColor] (157.19,521.09) circle (  1.16);

\path[draw=drawColor,line width= 0.4pt,line join=round,line cap=round,fill=fillColor] (157.31,521.04) circle (  1.16);

\path[draw=drawColor,line width= 0.4pt,line join=round,line cap=round,fill=fillColor] (157.43,521.03) circle (  1.16);

\path[draw=drawColor,line width= 0.4pt,line join=round,line cap=round,fill=fillColor] (157.55,521.02) circle (  1.16);

\path[draw=drawColor,line width= 0.4pt,line join=round,line cap=round,fill=fillColor] (157.67,521.00) circle (  1.16);

\path[draw=drawColor,line width= 0.4pt,line join=round,line cap=round,fill=fillColor] (157.79,520.99) circle (  1.16);

\path[draw=drawColor,line width= 0.4pt,line join=round,line cap=round,fill=fillColor] (157.91,520.97) circle (  1.16);

\path[draw=drawColor,line width= 0.4pt,line join=round,line cap=round,fill=fillColor] (158.02,520.97) circle (  1.16);

\path[draw=drawColor,line width= 0.4pt,line join=round,line cap=round,fill=fillColor] (158.14,520.93) circle (  1.16);

\path[draw=drawColor,line width= 0.4pt,line join=round,line cap=round,fill=fillColor] (158.26,520.92) circle (  1.16);

\path[draw=drawColor,line width= 0.4pt,line join=round,line cap=round,fill=fillColor] (158.38,520.91) circle (  1.16);

\path[draw=drawColor,line width= 0.4pt,line join=round,line cap=round,fill=fillColor] (158.50,520.89) circle (  1.16);

\path[draw=drawColor,line width= 0.4pt,line join=round,line cap=round,fill=fillColor] (158.61,520.89) circle (  1.16);

\path[draw=drawColor,line width= 0.4pt,line join=round,line cap=round,fill=fillColor] (158.73,520.87) circle (  1.16);

\path[draw=drawColor,line width= 0.4pt,line join=round,line cap=round,fill=fillColor] (158.84,520.87) circle (  1.16);

\path[draw=drawColor,line width= 0.4pt,line join=round,line cap=round,fill=fillColor] (158.96,520.86) circle (  1.16);

\path[draw=drawColor,line width= 0.4pt,line join=round,line cap=round,fill=fillColor] (159.08,520.85) circle (  1.16);

\path[draw=drawColor,line width= 0.4pt,line join=round,line cap=round,fill=fillColor] (159.19,520.84) circle (  1.16);

\path[draw=drawColor,line width= 0.4pt,line join=round,line cap=round,fill=fillColor] (159.31,520.83) circle (  1.16);

\path[draw=drawColor,line width= 0.4pt,line join=round,line cap=round,fill=fillColor] (159.42,520.82) circle (  1.16);

\path[draw=drawColor,line width= 0.4pt,line join=round,line cap=round,fill=fillColor] (159.54,520.76) circle (  1.16);

\path[draw=drawColor,line width= 0.4pt,line join=round,line cap=round,fill=fillColor] (159.65,520.75) circle (  1.16);

\path[draw=drawColor,line width= 0.4pt,line join=round,line cap=round,fill=fillColor] (159.76,520.73) circle (  1.16);

\path[draw=drawColor,line width= 0.4pt,line join=round,line cap=round,fill=fillColor] (159.88,520.73) circle (  1.16);

\path[draw=drawColor,line width= 0.4pt,line join=round,line cap=round,fill=fillColor] (159.99,520.72) circle (  1.16);

\path[draw=drawColor,line width= 0.4pt,line join=round,line cap=round,fill=fillColor] (160.10,520.72) circle (  1.16);

\path[draw=drawColor,line width= 0.4pt,line join=round,line cap=round,fill=fillColor] (160.22,520.71) circle (  1.16);

\path[draw=drawColor,line width= 0.4pt,line join=round,line cap=round,fill=fillColor] (160.33,520.70) circle (  1.16);

\path[draw=drawColor,line width= 0.4pt,line join=round,line cap=round,fill=fillColor] (160.44,520.68) circle (  1.16);

\path[draw=drawColor,line width= 0.4pt,line join=round,line cap=round,fill=fillColor] (160.55,520.64) circle (  1.16);

\path[draw=drawColor,line width= 0.4pt,line join=round,line cap=round,fill=fillColor] (160.67,520.64) circle (  1.16);

\path[draw=drawColor,line width= 0.4pt,line join=round,line cap=round,fill=fillColor] (160.78,520.54) circle (  1.16);

\path[draw=drawColor,line width= 0.4pt,line join=round,line cap=round,fill=fillColor] (160.89,520.53) circle (  1.16);

\path[draw=drawColor,line width= 0.4pt,line join=round,line cap=round,fill=fillColor] (161.00,520.51) circle (  1.16);

\path[draw=drawColor,line width= 0.4pt,line join=round,line cap=round,fill=fillColor] (161.11,520.48) circle (  1.16);

\path[draw=drawColor,line width= 0.4pt,line join=round,line cap=round,fill=fillColor] (161.22,520.46) circle (  1.16);

\path[draw=drawColor,line width= 0.4pt,line join=round,line cap=round,fill=fillColor] (161.33,520.46) circle (  1.16);

\path[draw=drawColor,line width= 0.4pt,line join=round,line cap=round,fill=fillColor] (161.44,520.43) circle (  1.16);

\path[draw=drawColor,line width= 0.4pt,line join=round,line cap=round,fill=fillColor] (161.55,520.43) circle (  1.16);

\path[draw=drawColor,line width= 0.4pt,line join=round,line cap=round,fill=fillColor] (161.66,520.38) circle (  1.16);

\path[draw=drawColor,line width= 0.4pt,line join=round,line cap=round,fill=fillColor] (161.77,520.36) circle (  1.16);

\path[draw=drawColor,line width= 0.4pt,line join=round,line cap=round,fill=fillColor] (161.88,520.35) circle (  1.16);

\path[draw=drawColor,line width= 0.4pt,line join=round,line cap=round,fill=fillColor] (161.99,520.33) circle (  1.16);

\path[draw=drawColor,line width= 0.4pt,line join=round,line cap=round,fill=fillColor] (162.10,520.32) circle (  1.16);

\path[draw=drawColor,line width= 0.4pt,line join=round,line cap=round,fill=fillColor] (162.20,520.31) circle (  1.16);

\path[draw=drawColor,line width= 0.4pt,line join=round,line cap=round,fill=fillColor] (162.31,520.28) circle (  1.16);

\path[draw=drawColor,line width= 0.4pt,line join=round,line cap=round,fill=fillColor] (162.42,520.27) circle (  1.16);

\path[draw=drawColor,line width= 0.4pt,line join=round,line cap=round,fill=fillColor] (162.53,520.26) circle (  1.16);

\path[draw=drawColor,line width= 0.4pt,line join=round,line cap=round,fill=fillColor] (162.63,520.24) circle (  1.16);

\path[draw=drawColor,line width= 0.4pt,line join=round,line cap=round,fill=fillColor] (162.74,520.19) circle (  1.16);

\path[draw=drawColor,line width= 0.4pt,line join=round,line cap=round,fill=fillColor] (162.85,520.19) circle (  1.16);

\path[draw=drawColor,line width= 0.4pt,line join=round,line cap=round,fill=fillColor] (162.95,520.19) circle (  1.16);

\path[draw=drawColor,line width= 0.4pt,line join=round,line cap=round,fill=fillColor] (163.06,520.15) circle (  1.16);

\path[draw=drawColor,line width= 0.4pt,line join=round,line cap=round,fill=fillColor] (163.17,520.15) circle (  1.16);

\path[draw=drawColor,line width= 0.4pt,line join=round,line cap=round,fill=fillColor] (163.27,520.15) circle (  1.16);

\path[draw=drawColor,line width= 0.4pt,line join=round,line cap=round,fill=fillColor] (163.38,520.13) circle (  1.16);

\path[draw=drawColor,line width= 0.4pt,line join=round,line cap=round,fill=fillColor] (163.48,520.12) circle (  1.16);

\path[draw=drawColor,line width= 0.4pt,line join=round,line cap=round,fill=fillColor] (163.59,520.11) circle (  1.16);

\path[draw=drawColor,line width= 0.4pt,line join=round,line cap=round,fill=fillColor] (163.69,520.09) circle (  1.16);

\path[draw=drawColor,line width= 0.4pt,line join=round,line cap=round,fill=fillColor] (163.80,520.08) circle (  1.16);

\path[draw=drawColor,line width= 0.4pt,line join=round,line cap=round,fill=fillColor] (163.90,520.06) circle (  1.16);

\path[draw=drawColor,line width= 0.4pt,line join=round,line cap=round,fill=fillColor] (164.01,520.06) circle (  1.16);

\path[draw=drawColor,line width= 0.4pt,line join=round,line cap=round,fill=fillColor] (164.11,520.05) circle (  1.16);

\path[draw=drawColor,line width= 0.4pt,line join=round,line cap=round,fill=fillColor] (164.21,520.04) circle (  1.16);

\path[draw=drawColor,line width= 0.4pt,line join=round,line cap=round,fill=fillColor] (164.32,520.03) circle (  1.16);

\path[draw=drawColor,line width= 0.4pt,line join=round,line cap=round,fill=fillColor] (164.42,520.01) circle (  1.16);

\path[draw=drawColor,line width= 0.4pt,line join=round,line cap=round,fill=fillColor] (164.52,519.99) circle (  1.16);

\path[draw=drawColor,line width= 0.4pt,line join=round,line cap=round,fill=fillColor] (164.63,519.98) circle (  1.16);

\path[draw=drawColor,line width= 0.4pt,line join=round,line cap=round,fill=fillColor] (164.73,519.98) circle (  1.16);

\path[draw=drawColor,line width= 0.4pt,line join=round,line cap=round,fill=fillColor] (164.83,519.96) circle (  1.16);

\path[draw=drawColor,line width= 0.4pt,line join=round,line cap=round,fill=fillColor] (164.93,519.92) circle (  1.16);

\path[draw=drawColor,line width= 0.4pt,line join=round,line cap=round,fill=fillColor] (165.04,519.88) circle (  1.16);

\path[draw=drawColor,line width= 0.4pt,line join=round,line cap=round,fill=fillColor] (165.14,519.87) circle (  1.16);

\path[draw=drawColor,line width= 0.4pt,line join=round,line cap=round,fill=fillColor] (165.24,519.87) circle (  1.16);

\path[draw=drawColor,line width= 0.4pt,line join=round,line cap=round,fill=fillColor] (165.34,519.86) circle (  1.16);

\path[draw=drawColor,line width= 0.4pt,line join=round,line cap=round,fill=fillColor] (165.44,519.85) circle (  1.16);

\path[draw=drawColor,line width= 0.4pt,line join=round,line cap=round,fill=fillColor] (165.54,519.83) circle (  1.16);

\path[draw=drawColor,line width= 0.4pt,line join=round,line cap=round,fill=fillColor] (165.64,519.81) circle (  1.16);

\path[draw=drawColor,line width= 0.4pt,line join=round,line cap=round,fill=fillColor] (165.74,519.80) circle (  1.16);

\path[draw=drawColor,line width= 0.4pt,line join=round,line cap=round,fill=fillColor] (165.85,519.78) circle (  1.16);

\path[draw=drawColor,line width= 0.4pt,line join=round,line cap=round,fill=fillColor] (165.95,519.75) circle (  1.16);

\path[draw=drawColor,line width= 0.4pt,line join=round,line cap=round,fill=fillColor] (166.05,519.74) circle (  1.16);

\path[draw=drawColor,line width= 0.4pt,line join=round,line cap=round,fill=fillColor] (166.14,519.74) circle (  1.16);

\path[draw=drawColor,line width= 0.4pt,line join=round,line cap=round,fill=fillColor] (166.24,519.73) circle (  1.16);

\path[draw=drawColor,line width= 0.4pt,line join=round,line cap=round,fill=fillColor] (166.34,519.72) circle (  1.16);

\path[draw=drawColor,line width= 0.4pt,line join=round,line cap=round,fill=fillColor] (166.44,519.71) circle (  1.16);

\path[draw=drawColor,line width= 0.4pt,line join=round,line cap=round,fill=fillColor] (166.54,519.70) circle (  1.16);

\path[draw=drawColor,line width= 0.4pt,line join=round,line cap=round,fill=fillColor] (166.64,519.65) circle (  1.16);

\path[draw=drawColor,line width= 0.4pt,line join=round,line cap=round,fill=fillColor] (166.74,519.59) circle (  1.16);

\path[draw=drawColor,line width= 0.4pt,line join=round,line cap=round,fill=fillColor] (166.84,519.58) circle (  1.16);

\path[draw=drawColor,line width= 0.4pt,line join=round,line cap=round,fill=fillColor] (166.94,519.56) circle (  1.16);

\path[draw=drawColor,line width= 0.4pt,line join=round,line cap=round,fill=fillColor] (167.03,519.56) circle (  1.16);

\path[draw=drawColor,line width= 0.4pt,line join=round,line cap=round,fill=fillColor] (167.13,519.56) circle (  1.16);

\path[draw=drawColor,line width= 0.4pt,line join=round,line cap=round,fill=fillColor] (167.23,519.55) circle (  1.16);

\path[draw=drawColor,line width= 0.4pt,line join=round,line cap=round,fill=fillColor] (167.33,519.53) circle (  1.16);

\path[draw=drawColor,line width= 0.4pt,line join=round,line cap=round,fill=fillColor] (167.42,519.53) circle (  1.16);

\path[draw=drawColor,line width= 0.4pt,line join=round,line cap=round,fill=fillColor] (167.52,519.52) circle (  1.16);

\path[draw=drawColor,line width= 0.4pt,line join=round,line cap=round,fill=fillColor] (167.62,519.51) circle (  1.16);

\path[draw=drawColor,line width= 0.4pt,line join=round,line cap=round,fill=fillColor] (167.71,519.50) circle (  1.16);

\path[draw=drawColor,line width= 0.4pt,line join=round,line cap=round,fill=fillColor] (167.81,519.45) circle (  1.16);

\path[draw=drawColor,line width= 0.4pt,line join=round,line cap=round,fill=fillColor] (167.91,519.45) circle (  1.16);

\path[draw=drawColor,line width= 0.4pt,line join=round,line cap=round,fill=fillColor] (168.00,519.43) circle (  1.16);

\path[draw=drawColor,line width= 0.4pt,line join=round,line cap=round,fill=fillColor] (168.10,519.42) circle (  1.16);

\path[draw=drawColor,line width= 0.4pt,line join=round,line cap=round,fill=fillColor] (168.20,519.40) circle (  1.16);

\path[draw=drawColor,line width= 0.4pt,line join=round,line cap=round,fill=fillColor] (168.29,519.38) circle (  1.16);

\path[draw=drawColor,line width= 0.4pt,line join=round,line cap=round,fill=fillColor] (168.39,519.35) circle (  1.16);

\path[draw=drawColor,line width= 0.4pt,line join=round,line cap=round,fill=fillColor] (168.48,519.30) circle (  1.16);

\path[draw=drawColor,line width= 0.4pt,line join=round,line cap=round,fill=fillColor] (168.58,519.29) circle (  1.16);

\path[draw=drawColor,line width= 0.4pt,line join=round,line cap=round,fill=fillColor] (168.67,519.27) circle (  1.16);

\path[draw=drawColor,line width= 0.4pt,line join=round,line cap=round,fill=fillColor] (168.77,519.23) circle (  1.16);

\path[draw=drawColor,line width= 0.4pt,line join=round,line cap=round,fill=fillColor] (168.86,519.23) circle (  1.16);

\path[draw=drawColor,line width= 0.4pt,line join=round,line cap=round,fill=fillColor] (168.95,519.23) circle (  1.16);

\path[draw=drawColor,line width= 0.4pt,line join=round,line cap=round,fill=fillColor] (169.05,519.22) circle (  1.16);

\path[draw=drawColor,line width= 0.4pt,line join=round,line cap=round,fill=fillColor] (169.14,519.22) circle (  1.16);

\path[draw=drawColor,line width= 0.4pt,line join=round,line cap=round,fill=fillColor] (169.24,519.20) circle (  1.16);

\path[draw=drawColor,line width= 0.4pt,line join=round,line cap=round,fill=fillColor] (169.33,519.20) circle (  1.16);

\path[draw=drawColor,line width= 0.4pt,line join=round,line cap=round,fill=fillColor] (169.42,519.18) circle (  1.16);

\path[draw=drawColor,line width= 0.4pt,line join=round,line cap=round,fill=fillColor] (169.52,519.18) circle (  1.16);

\path[draw=drawColor,line width= 0.4pt,line join=round,line cap=round,fill=fillColor] (169.61,519.17) circle (  1.16);

\path[draw=drawColor,line width= 0.4pt,line join=round,line cap=round,fill=fillColor] (169.70,519.13) circle (  1.16);

\path[draw=drawColor,line width= 0.4pt,line join=round,line cap=round,fill=fillColor] (169.80,519.11) circle (  1.16);

\path[draw=drawColor,line width= 0.4pt,line join=round,line cap=round,fill=fillColor] (169.89,519.10) circle (  1.16);

\path[draw=drawColor,line width= 0.4pt,line join=round,line cap=round,fill=fillColor] (169.98,519.09) circle (  1.16);

\path[draw=drawColor,line width= 0.4pt,line join=round,line cap=round,fill=fillColor] (170.07,519.09) circle (  1.16);

\path[draw=drawColor,line width= 0.4pt,line join=round,line cap=round,fill=fillColor] (170.17,519.06) circle (  1.16);

\path[draw=drawColor,line width= 0.4pt,line join=round,line cap=round,fill=fillColor] (170.26,519.05) circle (  1.16);

\path[draw=drawColor,line width= 0.4pt,line join=round,line cap=round,fill=fillColor] (170.35,519.04) circle (  1.16);

\path[draw=drawColor,line width= 0.4pt,line join=round,line cap=round,fill=fillColor] (170.44,519.02) circle (  1.16);

\path[draw=drawColor,line width= 0.4pt,line join=round,line cap=round,fill=fillColor] (170.53,519.01) circle (  1.16);

\path[draw=drawColor,line width= 0.4pt,line join=round,line cap=round,fill=fillColor] (170.62,519.01) circle (  1.16);

\path[draw=drawColor,line width= 0.4pt,line join=round,line cap=round,fill=fillColor] (170.71,518.97) circle (  1.16);

\path[draw=drawColor,line width= 0.4pt,line join=round,line cap=round,fill=fillColor] (170.81,518.96) circle (  1.16);

\path[draw=drawColor,line width= 0.4pt,line join=round,line cap=round,fill=fillColor] (170.90,518.94) circle (  1.16);

\path[draw=drawColor,line width= 0.4pt,line join=round,line cap=round,fill=fillColor] (170.99,518.94) circle (  1.16);

\path[draw=drawColor,line width= 0.4pt,line join=round,line cap=round,fill=fillColor] (171.08,518.94) circle (  1.16);

\path[draw=drawColor,line width= 0.4pt,line join=round,line cap=round,fill=fillColor] (171.17,518.94) circle (  1.16);

\path[draw=drawColor,line width= 0.4pt,line join=round,line cap=round,fill=fillColor] (171.26,518.94) circle (  1.16);

\path[draw=drawColor,line width= 0.4pt,line join=round,line cap=round,fill=fillColor] (171.35,518.91) circle (  1.16);

\path[draw=drawColor,line width= 0.4pt,line join=round,line cap=round,fill=fillColor] (171.44,518.87) circle (  1.16);

\path[draw=drawColor,line width= 0.4pt,line join=round,line cap=round,fill=fillColor] (171.53,518.86) circle (  1.16);

\path[draw=drawColor,line width= 0.4pt,line join=round,line cap=round,fill=fillColor] (171.62,518.83) circle (  1.16);

\path[draw=drawColor,line width= 0.4pt,line join=round,line cap=round,fill=fillColor] (171.71,518.82) circle (  1.16);

\path[draw=drawColor,line width= 0.4pt,line join=round,line cap=round,fill=fillColor] (171.80,518.78) circle (  1.16);

\path[draw=drawColor,line width= 0.4pt,line join=round,line cap=round,fill=fillColor] (171.89,518.78) circle (  1.16);

\path[draw=drawColor,line width= 0.4pt,line join=round,line cap=round,fill=fillColor] (171.97,518.77) circle (  1.16);

\path[draw=drawColor,line width= 0.4pt,line join=round,line cap=round,fill=fillColor] (172.06,518.77) circle (  1.16);

\path[draw=drawColor,line width= 0.4pt,line join=round,line cap=round,fill=fillColor] (172.15,518.76) circle (  1.16);

\path[draw=drawColor,line width= 0.4pt,line join=round,line cap=round,fill=fillColor] (172.24,518.75) circle (  1.16);

\path[draw=drawColor,line width= 0.4pt,line join=round,line cap=round,fill=fillColor] (172.33,518.75) circle (  1.16);

\path[draw=drawColor,line width= 0.4pt,line join=round,line cap=round,fill=fillColor] (172.42,518.73) circle (  1.16);

\path[draw=drawColor,line width= 0.4pt,line join=round,line cap=round,fill=fillColor] (172.51,518.69) circle (  1.16);

\path[draw=drawColor,line width= 0.4pt,line join=round,line cap=round,fill=fillColor] (172.59,518.69) circle (  1.16);

\path[draw=drawColor,line width= 0.4pt,line join=round,line cap=round,fill=fillColor] (172.68,518.69) circle (  1.16);

\path[draw=drawColor,line width= 0.4pt,line join=round,line cap=round,fill=fillColor] (172.77,518.67) circle (  1.16);

\path[draw=drawColor,line width= 0.4pt,line join=round,line cap=round,fill=fillColor] (172.86,518.66) circle (  1.16);

\path[draw=drawColor,line width= 0.4pt,line join=round,line cap=round,fill=fillColor] (172.94,518.66) circle (  1.16);

\path[draw=drawColor,line width= 0.4pt,line join=round,line cap=round,fill=fillColor] (173.03,518.66) circle (  1.16);

\path[draw=drawColor,line width= 0.4pt,line join=round,line cap=round,fill=fillColor] (173.12,518.64) circle (  1.16);

\path[draw=drawColor,line width= 0.4pt,line join=round,line cap=round,fill=fillColor] (173.20,518.62) circle (  1.16);

\path[draw=drawColor,line width= 0.4pt,line join=round,line cap=round,fill=fillColor] (173.29,518.61) circle (  1.16);

\path[draw=drawColor,line width= 0.4pt,line join=round,line cap=round,fill=fillColor] (173.38,518.58) circle (  1.16);

\path[draw=drawColor,line width= 0.4pt,line join=round,line cap=round,fill=fillColor] (173.46,518.56) circle (  1.16);

\path[draw=drawColor,line width= 0.4pt,line join=round,line cap=round,fill=fillColor] (173.55,518.54) circle (  1.16);

\path[draw=drawColor,line width= 0.4pt,line join=round,line cap=round,fill=fillColor] (173.64,518.52) circle (  1.16);

\path[draw=drawColor,line width= 0.4pt,line join=round,line cap=round,fill=fillColor] (173.72,518.51) circle (  1.16);

\path[draw=drawColor,line width= 0.4pt,line join=round,line cap=round,fill=fillColor] (173.81,518.49) circle (  1.16);

\path[draw=drawColor,line width= 0.4pt,line join=round,line cap=round,fill=fillColor] (173.89,518.49) circle (  1.16);

\path[draw=drawColor,line width= 0.4pt,line join=round,line cap=round,fill=fillColor] (173.98,518.48) circle (  1.16);

\path[draw=drawColor,line width= 0.4pt,line join=round,line cap=round,fill=fillColor] (174.07,518.47) circle (  1.16);

\path[draw=drawColor,line width= 0.4pt,line join=round,line cap=round,fill=fillColor] (174.15,518.46) circle (  1.16);

\path[draw=drawColor,line width= 0.4pt,line join=round,line cap=round,fill=fillColor] (174.24,518.46) circle (  1.16);

\path[draw=drawColor,line width= 0.4pt,line join=round,line cap=round,fill=fillColor] (174.32,518.44) circle (  1.16);

\path[draw=drawColor,line width= 0.4pt,line join=round,line cap=round,fill=fillColor] (174.41,518.43) circle (  1.16);

\path[draw=drawColor,line width= 0.4pt,line join=round,line cap=round,fill=fillColor] (174.49,518.42) circle (  1.16);

\path[draw=drawColor,line width= 0.4pt,line join=round,line cap=round,fill=fillColor] (174.58,518.40) circle (  1.16);

\path[draw=drawColor,line width= 0.4pt,line join=round,line cap=round,fill=fillColor] (174.66,518.40) circle (  1.16);

\path[draw=drawColor,line width= 0.4pt,line join=round,line cap=round,fill=fillColor] (174.75,518.39) circle (  1.16);

\path[draw=drawColor,line width= 0.4pt,line join=round,line cap=round,fill=fillColor] (174.83,518.38) circle (  1.16);

\path[draw=drawColor,line width= 0.4pt,line join=round,line cap=round,fill=fillColor] (174.91,518.37) circle (  1.16);

\path[draw=drawColor,line width= 0.4pt,line join=round,line cap=round,fill=fillColor] (175.00,518.34) circle (  1.16);

\path[draw=drawColor,line width= 0.4pt,line join=round,line cap=round,fill=fillColor] (175.08,518.33) circle (  1.16);

\path[draw=drawColor,line width= 0.4pt,line join=round,line cap=round,fill=fillColor] (175.17,518.30) circle (  1.16);

\path[draw=drawColor,line width= 0.4pt,line join=round,line cap=round,fill=fillColor] (175.25,518.30) circle (  1.16);

\path[draw=drawColor,line width= 0.4pt,line join=round,line cap=round,fill=fillColor] (175.33,518.27) circle (  1.16);

\path[draw=drawColor,line width= 0.4pt,line join=round,line cap=round,fill=fillColor] (175.42,518.26) circle (  1.16);

\path[draw=drawColor,line width= 0.4pt,line join=round,line cap=round,fill=fillColor] (175.50,518.24) circle (  1.16);

\path[draw=drawColor,line width= 0.4pt,line join=round,line cap=round,fill=fillColor] (175.58,518.22) circle (  1.16);

\path[draw=drawColor,line width= 0.4pt,line join=round,line cap=round,fill=fillColor] (175.67,518.22) circle (  1.16);

\path[draw=drawColor,line width= 0.4pt,line join=round,line cap=round,fill=fillColor] (175.75,518.21) circle (  1.16);

\path[draw=drawColor,line width= 0.4pt,line join=round,line cap=round,fill=fillColor] (175.83,518.20) circle (  1.16);

\path[draw=drawColor,line width= 0.4pt,line join=round,line cap=round,fill=fillColor] (175.91,518.20) circle (  1.16);

\path[draw=drawColor,line width= 0.4pt,line join=round,line cap=round,fill=fillColor] (176.00,518.20) circle (  1.16);

\path[draw=drawColor,line width= 0.4pt,line join=round,line cap=round,fill=fillColor] (176.08,518.13) circle (  1.16);

\path[draw=drawColor,line width= 0.4pt,line join=round,line cap=round,fill=fillColor] (176.16,518.12) circle (  1.16);

\path[draw=drawColor,line width= 0.4pt,line join=round,line cap=round,fill=fillColor] (176.24,518.12) circle (  1.16);

\path[draw=drawColor,line width= 0.4pt,line join=round,line cap=round,fill=fillColor] (176.33,518.12) circle (  1.16);

\path[draw=drawColor,line width= 0.4pt,line join=round,line cap=round,fill=fillColor] (176.41,518.11) circle (  1.16);

\path[draw=drawColor,line width= 0.4pt,line join=round,line cap=round,fill=fillColor] (176.49,518.08) circle (  1.16);

\path[draw=drawColor,line width= 0.4pt,line join=round,line cap=round,fill=fillColor] (176.57,518.05) circle (  1.16);

\path[draw=drawColor,line width= 0.4pt,line join=round,line cap=round,fill=fillColor] (176.65,518.04) circle (  1.16);

\path[draw=drawColor,line width= 0.4pt,line join=round,line cap=round,fill=fillColor] (176.73,518.04) circle (  1.16);

\path[draw=drawColor,line width= 0.4pt,line join=round,line cap=round,fill=fillColor] (176.82,518.03) circle (  1.16);

\path[draw=drawColor,line width= 0.4pt,line join=round,line cap=round,fill=fillColor] (176.90,518.03) circle (  1.16);

\path[draw=drawColor,line width= 0.4pt,line join=round,line cap=round,fill=fillColor] (176.98,517.98) circle (  1.16);

\path[draw=drawColor,line width= 0.4pt,line join=round,line cap=round,fill=fillColor] (177.06,517.98) circle (  1.16);

\path[draw=drawColor,line width= 0.4pt,line join=round,line cap=round,fill=fillColor] (177.14,517.97) circle (  1.16);

\path[draw=drawColor,line width= 0.4pt,line join=round,line cap=round,fill=fillColor] (177.22,517.96) circle (  1.16);

\path[draw=drawColor,line width= 0.4pt,line join=round,line cap=round,fill=fillColor] (177.30,517.91) circle (  1.16);

\path[draw=drawColor,line width= 0.4pt,line join=round,line cap=round,fill=fillColor] (177.38,517.91) circle (  1.16);

\path[draw=drawColor,line width= 0.4pt,line join=round,line cap=round,fill=fillColor] (177.46,517.91) circle (  1.16);

\path[draw=drawColor,line width= 0.4pt,line join=round,line cap=round,fill=fillColor] (177.54,517.91) circle (  1.16);

\path[draw=drawColor,line width= 0.4pt,line join=round,line cap=round,fill=fillColor] (177.62,517.89) circle (  1.16);

\path[draw=drawColor,line width= 0.4pt,line join=round,line cap=round,fill=fillColor] (177.70,517.89) circle (  1.16);

\path[draw=drawColor,line width= 0.4pt,line join=round,line cap=round,fill=fillColor] (177.78,517.89) circle (  1.16);

\path[draw=drawColor,line width= 0.4pt,line join=round,line cap=round,fill=fillColor] (177.86,517.88) circle (  1.16);

\path[draw=drawColor,line width= 0.4pt,line join=round,line cap=round,fill=fillColor] (177.94,517.87) circle (  1.16);

\path[draw=drawColor,line width= 0.4pt,line join=round,line cap=round,fill=fillColor] (178.02,517.85) circle (  1.16);

\path[draw=drawColor,line width= 0.4pt,line join=round,line cap=round,fill=fillColor] (178.10,517.84) circle (  1.16);

\path[draw=drawColor,line width= 0.4pt,line join=round,line cap=round,fill=fillColor] (178.18,517.82) circle (  1.16);

\path[draw=drawColor,line width= 0.4pt,line join=round,line cap=round,fill=fillColor] (178.26,517.79) circle (  1.16);

\path[draw=drawColor,line width= 0.4pt,line join=round,line cap=round,fill=fillColor] (178.34,517.79) circle (  1.16);

\path[draw=drawColor,line width= 0.4pt,line join=round,line cap=round,fill=fillColor] (178.42,517.78) circle (  1.16);

\path[draw=drawColor,line width= 0.4pt,line join=round,line cap=round,fill=fillColor] (178.50,517.72) circle (  1.16);

\path[draw=drawColor,line width= 0.4pt,line join=round,line cap=round,fill=fillColor] (178.57,517.69) circle (  1.16);

\path[draw=drawColor,line width= 0.4pt,line join=round,line cap=round,fill=fillColor] (178.65,517.67) circle (  1.16);

\path[draw=drawColor,line width= 0.4pt,line join=round,line cap=round,fill=fillColor] (178.73,517.66) circle (  1.16);

\path[draw=drawColor,line width= 0.4pt,line join=round,line cap=round,fill=fillColor] (178.81,517.64) circle (  1.16);

\path[draw=drawColor,line width= 0.4pt,line join=round,line cap=round,fill=fillColor] (178.89,517.64) circle (  1.16);

\path[draw=drawColor,line width= 0.4pt,line join=round,line cap=round,fill=fillColor] (178.97,517.61) circle (  1.16);

\path[draw=drawColor,line width= 0.4pt,line join=round,line cap=round,fill=fillColor] (179.04,517.58) circle (  1.16);

\path[draw=drawColor,line width= 0.4pt,line join=round,line cap=round,fill=fillColor] (179.12,517.58) circle (  1.16);

\path[draw=drawColor,line width= 0.4pt,line join=round,line cap=round,fill=fillColor] (179.20,517.54) circle (  1.16);

\path[draw=drawColor,line width= 0.4pt,line join=round,line cap=round,fill=fillColor] (179.28,517.54) circle (  1.16);

\path[draw=drawColor,line width= 0.4pt,line join=round,line cap=round,fill=fillColor] (179.36,517.53) circle (  1.16);

\path[draw=drawColor,line width= 0.4pt,line join=round,line cap=round,fill=fillColor] (179.43,517.52) circle (  1.16);

\path[draw=drawColor,line width= 0.4pt,line join=round,line cap=round,fill=fillColor] (179.51,517.47) circle (  1.16);

\path[draw=drawColor,line width= 0.4pt,line join=round,line cap=round,fill=fillColor] (179.59,517.45) circle (  1.16);

\path[draw=drawColor,line width= 0.4pt,line join=round,line cap=round,fill=fillColor] (179.67,517.44) circle (  1.16);

\path[draw=drawColor,line width= 0.4pt,line join=round,line cap=round,fill=fillColor] (179.74,517.40) circle (  1.16);

\path[draw=drawColor,line width= 0.4pt,line join=round,line cap=round,fill=fillColor] (179.82,517.40) circle (  1.16);

\path[draw=drawColor,line width= 0.4pt,line join=round,line cap=round,fill=fillColor] (179.90,517.36) circle (  1.16);

\path[draw=drawColor,line width= 0.4pt,line join=round,line cap=round,fill=fillColor] (179.97,517.36) circle (  1.16);

\path[draw=drawColor,line width= 0.4pt,line join=round,line cap=round,fill=fillColor] (180.05,517.36) circle (  1.16);

\path[draw=drawColor,line width= 0.4pt,line join=round,line cap=round,fill=fillColor] (180.13,517.35) circle (  1.16);

\path[draw=drawColor,line width= 0.4pt,line join=round,line cap=round,fill=fillColor] (180.20,517.34) circle (  1.16);

\path[draw=drawColor,line width= 0.4pt,line join=round,line cap=round,fill=fillColor] (180.28,517.34) circle (  1.16);

\path[draw=drawColor,line width= 0.4pt,line join=round,line cap=round,fill=fillColor] (180.36,517.33) circle (  1.16);

\path[draw=drawColor,line width= 0.4pt,line join=round,line cap=round,fill=fillColor] (180.43,517.30) circle (  1.16);

\path[draw=drawColor,line width= 0.4pt,line join=round,line cap=round,fill=fillColor] (180.51,517.29) circle (  1.16);

\path[draw=drawColor,line width= 0.4pt,line join=round,line cap=round,fill=fillColor] (180.58,517.28) circle (  1.16);

\path[draw=drawColor,line width= 0.4pt,line join=round,line cap=round,fill=fillColor] (180.66,517.26) circle (  1.16);

\path[draw=drawColor,line width= 0.4pt,line join=round,line cap=round,fill=fillColor] (180.74,517.26) circle (  1.16);

\path[draw=drawColor,line width= 0.4pt,line join=round,line cap=round,fill=fillColor] (180.81,517.26) circle (  1.16);

\path[draw=drawColor,line width= 0.4pt,line join=round,line cap=round,fill=fillColor] (180.89,517.26) circle (  1.16);

\path[draw=drawColor,line width= 0.4pt,line join=round,line cap=round,fill=fillColor] (180.96,517.25) circle (  1.16);

\path[draw=drawColor,line width= 0.4pt,line join=round,line cap=round,fill=fillColor] (181.04,517.25) circle (  1.16);

\path[draw=drawColor,line width= 0.4pt,line join=round,line cap=round,fill=fillColor] (181.11,517.21) circle (  1.16);

\path[draw=drawColor,line width= 0.4pt,line join=round,line cap=round,fill=fillColor] (181.19,517.21) circle (  1.16);

\path[draw=drawColor,line width= 0.4pt,line join=round,line cap=round,fill=fillColor] (181.26,517.20) circle (  1.16);

\path[draw=drawColor,line width= 0.4pt,line join=round,line cap=round,fill=fillColor] (181.34,517.19) circle (  1.16);

\path[draw=drawColor,line width= 0.4pt,line join=round,line cap=round,fill=fillColor] (181.41,517.18) circle (  1.16);

\path[draw=drawColor,line width= 0.4pt,line join=round,line cap=round,fill=fillColor] (181.49,517.15) circle (  1.16);

\path[draw=drawColor,line width= 0.4pt,line join=round,line cap=round,fill=fillColor] (181.56,517.14) circle (  1.16);

\path[draw=drawColor,line width= 0.4pt,line join=round,line cap=round,fill=fillColor] (181.64,517.14) circle (  1.16);

\path[draw=drawColor,line width= 0.4pt,line join=round,line cap=round,fill=fillColor] (181.71,517.13) circle (  1.16);

\path[draw=drawColor,line width= 0.4pt,line join=round,line cap=round,fill=fillColor] (181.79,517.13) circle (  1.16);

\path[draw=drawColor,line width= 0.4pt,line join=round,line cap=round,fill=fillColor] (181.86,517.11) circle (  1.16);

\path[draw=drawColor,line width= 0.4pt,line join=round,line cap=round,fill=fillColor] (181.94,517.10) circle (  1.16);

\path[draw=drawColor,line width= 0.4pt,line join=round,line cap=round,fill=fillColor] (182.01,517.08) circle (  1.16);

\path[draw=drawColor,line width= 0.4pt,line join=round,line cap=round,fill=fillColor] (182.08,517.07) circle (  1.16);

\path[draw=drawColor,line width= 0.4pt,line join=round,line cap=round,fill=fillColor] (182.16,517.03) circle (  1.16);

\path[draw=drawColor,line width= 0.4pt,line join=round,line cap=round,fill=fillColor] (182.23,516.99) circle (  1.16);

\path[draw=drawColor,line width= 0.4pt,line join=round,line cap=round,fill=fillColor] (182.31,516.98) circle (  1.16);

\path[draw=drawColor,line width= 0.4pt,line join=round,line cap=round,fill=fillColor] (182.38,516.97) circle (  1.16);

\path[draw=drawColor,line width= 0.4pt,line join=round,line cap=round,fill=fillColor] (182.45,516.91) circle (  1.16);

\path[draw=drawColor,line width= 0.4pt,line join=round,line cap=round,fill=fillColor] (182.53,516.90) circle (  1.16);

\path[draw=drawColor,line width= 0.4pt,line join=round,line cap=round,fill=fillColor] (182.60,516.88) circle (  1.16);

\path[draw=drawColor,line width= 0.4pt,line join=round,line cap=round,fill=fillColor] (182.67,516.87) circle (  1.16);

\path[draw=drawColor,line width= 0.4pt,line join=round,line cap=round,fill=fillColor] (182.75,516.87) circle (  1.16);

\path[draw=drawColor,line width= 0.4pt,line join=round,line cap=round,fill=fillColor] (182.82,516.87) circle (  1.16);

\path[draw=drawColor,line width= 0.4pt,line join=round,line cap=round,fill=fillColor] (182.89,516.86) circle (  1.16);

\path[draw=drawColor,line width= 0.4pt,line join=round,line cap=round,fill=fillColor] (182.96,516.84) circle (  1.16);

\path[draw=drawColor,line width= 0.4pt,line join=round,line cap=round,fill=fillColor] (183.04,516.83) circle (  1.16);

\path[draw=drawColor,line width= 0.4pt,line join=round,line cap=round,fill=fillColor] (183.11,516.82) circle (  1.16);

\path[draw=drawColor,line width= 0.4pt,line join=round,line cap=round,fill=fillColor] (183.18,516.81) circle (  1.16);

\path[draw=drawColor,line width= 0.4pt,line join=round,line cap=round,fill=fillColor] (183.26,516.80) circle (  1.16);

\path[draw=drawColor,line width= 0.4pt,line join=round,line cap=round,fill=fillColor] (183.33,516.80) circle (  1.16);

\path[draw=drawColor,line width= 0.4pt,line join=round,line cap=round,fill=fillColor] (183.40,516.80) circle (  1.16);

\path[draw=drawColor,line width= 0.4pt,line join=round,line cap=round,fill=fillColor] (183.47,516.79) circle (  1.16);

\path[draw=drawColor,line width= 0.4pt,line join=round,line cap=round,fill=fillColor] (183.54,516.79) circle (  1.16);

\path[draw=drawColor,line width= 0.4pt,line join=round,line cap=round,fill=fillColor] (183.62,516.79) circle (  1.16);

\path[draw=drawColor,line width= 0.4pt,line join=round,line cap=round,fill=fillColor] (183.69,516.78) circle (  1.16);

\path[draw=drawColor,line width= 0.4pt,line join=round,line cap=round,fill=fillColor] (183.76,516.78) circle (  1.16);

\path[draw=drawColor,line width= 0.4pt,line join=round,line cap=round,fill=fillColor] (183.83,516.75) circle (  1.16);

\path[draw=drawColor,line width= 0.4pt,line join=round,line cap=round,fill=fillColor] (183.90,516.75) circle (  1.16);

\path[draw=drawColor,line width= 0.4pt,line join=round,line cap=round,fill=fillColor] (183.98,516.74) circle (  1.16);

\path[draw=drawColor,line width= 0.4pt,line join=round,line cap=round,fill=fillColor] (184.05,516.68) circle (  1.16);

\path[draw=drawColor,line width= 0.4pt,line join=round,line cap=round,fill=fillColor] (184.12,516.68) circle (  1.16);

\path[draw=drawColor,line width= 0.4pt,line join=round,line cap=round,fill=fillColor] (184.19,516.67) circle (  1.16);

\path[draw=drawColor,line width= 0.4pt,line join=round,line cap=round,fill=fillColor] (184.26,516.67) circle (  1.16);

\path[draw=drawColor,line width= 0.4pt,line join=round,line cap=round,fill=fillColor] (184.33,516.66) circle (  1.16);

\path[draw=drawColor,line width= 0.4pt,line join=round,line cap=round,fill=fillColor] (184.40,516.63) circle (  1.16);

\path[draw=drawColor,line width= 0.4pt,line join=round,line cap=round,fill=fillColor] (184.48,516.63) circle (  1.16);

\path[draw=drawColor,line width= 0.4pt,line join=round,line cap=round,fill=fillColor] (184.55,516.62) circle (  1.16);

\path[draw=drawColor,line width= 0.4pt,line join=round,line cap=round,fill=fillColor] (184.62,516.62) circle (  1.16);

\path[draw=drawColor,line width= 0.4pt,line join=round,line cap=round,fill=fillColor] (184.69,516.61) circle (  1.16);

\path[draw=drawColor,line width= 0.4pt,line join=round,line cap=round,fill=fillColor] (184.76,516.61) circle (  1.16);

\path[draw=drawColor,line width= 0.4pt,line join=round,line cap=round,fill=fillColor] (184.83,516.56) circle (  1.16);

\path[draw=drawColor,line width= 0.4pt,line join=round,line cap=round,fill=fillColor] (184.90,516.55) circle (  1.16);

\path[draw=drawColor,line width= 0.4pt,line join=round,line cap=round,fill=fillColor] (184.97,516.47) circle (  1.16);

\path[draw=drawColor,line width= 0.4pt,line join=round,line cap=round,fill=fillColor] (185.04,516.46) circle (  1.16);

\path[draw=drawColor,line width= 0.4pt,line join=round,line cap=round,fill=fillColor] (185.11,516.46) circle (  1.16);

\path[draw=drawColor,line width= 0.4pt,line join=round,line cap=round,fill=fillColor] (185.18,516.46) circle (  1.16);

\path[draw=drawColor,line width= 0.4pt,line join=round,line cap=round,fill=fillColor] (185.25,516.44) circle (  1.16);

\path[draw=drawColor,line width= 0.4pt,line join=round,line cap=round,fill=fillColor] (185.32,516.43) circle (  1.16);

\path[draw=drawColor,line width= 0.4pt,line join=round,line cap=round,fill=fillColor] (185.39,516.42) circle (  1.16);

\path[draw=drawColor,line width= 0.4pt,line join=round,line cap=round,fill=fillColor] (185.46,516.37) circle (  1.16);

\path[draw=drawColor,line width= 0.4pt,line join=round,line cap=round,fill=fillColor] (185.53,516.33) circle (  1.16);

\path[draw=drawColor,line width= 0.4pt,line join=round,line cap=round,fill=fillColor] (185.60,516.30) circle (  1.16);

\path[draw=drawColor,line width= 0.4pt,line join=round,line cap=round,fill=fillColor] (185.67,516.25) circle (  1.16);

\path[draw=drawColor,line width= 0.4pt,line join=round,line cap=round,fill=fillColor] (185.74,516.25) circle (  1.16);

\path[draw=drawColor,line width= 0.4pt,line join=round,line cap=round,fill=fillColor] (185.81,516.24) circle (  1.16);

\path[draw=drawColor,line width= 0.4pt,line join=round,line cap=round,fill=fillColor] (185.88,516.24) circle (  1.16);

\path[draw=drawColor,line width= 0.4pt,line join=round,line cap=round,fill=fillColor] (185.95,516.20) circle (  1.16);

\path[draw=drawColor,line width= 0.4pt,line join=round,line cap=round,fill=fillColor] (186.02,516.17) circle (  1.16);

\path[draw=drawColor,line width= 0.4pt,line join=round,line cap=round,fill=fillColor] (186.09,516.13) circle (  1.16);

\path[draw=drawColor,line width= 0.4pt,line join=round,line cap=round,fill=fillColor] (186.16,516.10) circle (  1.16);

\path[draw=drawColor,line width= 0.4pt,line join=round,line cap=round,fill=fillColor] (186.22,516.07) circle (  1.16);

\path[draw=drawColor,line width= 0.4pt,line join=round,line cap=round,fill=fillColor] (186.29,516.01) circle (  1.16);

\path[draw=drawColor,line width= 0.4pt,line join=round,line cap=round,fill=fillColor] (186.36,516.01) circle (  1.16);

\path[draw=drawColor,line width= 0.4pt,line join=round,line cap=round,fill=fillColor] (186.43,515.89) circle (  1.16);

\path[draw=drawColor,line width= 0.4pt,line join=round,line cap=round,fill=fillColor] (186.50,515.86) circle (  1.16);

\path[draw=drawColor,line width= 0.4pt,line join=round,line cap=round,fill=fillColor] (186.57,515.86) circle (  1.16);

\path[draw=drawColor,line width= 0.4pt,line join=round,line cap=round,fill=fillColor] (186.64,515.85) circle (  1.16);

\path[draw=drawColor,line width= 0.4pt,line join=round,line cap=round,fill=fillColor] (186.70,515.83) circle (  1.16);

\path[draw=drawColor,line width= 0.4pt,line join=round,line cap=round,fill=fillColor] (186.77,515.82) circle (  1.16);

\path[draw=drawColor,line width= 0.4pt,line join=round,line cap=round,fill=fillColor] (186.84,515.81) circle (  1.16);

\path[draw=drawColor,line width= 0.4pt,line join=round,line cap=round,fill=fillColor] (186.91,515.78) circle (  1.16);

\path[draw=drawColor,line width= 0.4pt,line join=round,line cap=round,fill=fillColor] (186.98,515.74) circle (  1.16);

\path[draw=drawColor,line width= 0.4pt,line join=round,line cap=round,fill=fillColor] (187.05,515.70) circle (  1.16);

\path[draw=drawColor,line width= 0.4pt,line join=round,line cap=round,fill=fillColor] (187.11,515.68) circle (  1.16);

\path[draw=drawColor,line width= 0.4pt,line join=round,line cap=round,fill=fillColor] (187.18,515.59) circle (  1.16);

\path[draw=drawColor,line width= 0.4pt,line join=round,line cap=round,fill=fillColor] (187.25,515.56) circle (  1.16);

\path[draw=drawColor,line width= 0.4pt,line join=round,line cap=round,fill=fillColor] (187.32,515.48) circle (  1.16);

\path[draw=drawColor,line width= 0.4pt,line join=round,line cap=round,fill=fillColor] (187.38,515.36) circle (  1.16);

\path[draw=drawColor,line width= 0.4pt,line join=round,line cap=round,fill=fillColor] (187.45,515.35) circle (  1.16);

\path[draw=drawColor,line width= 0.4pt,line join=round,line cap=round,fill=fillColor] (187.52,515.32) circle (  1.16);

\path[draw=drawColor,line width= 0.4pt,line join=round,line cap=round,fill=fillColor] (187.59,515.31) circle (  1.16);

\path[draw=drawColor,line width= 0.4pt,line join=round,line cap=round,fill=fillColor] (187.65,515.31) circle (  1.16);

\path[draw=drawColor,line width= 0.4pt,line join=round,line cap=round,fill=fillColor] (187.72,515.27) circle (  1.16);

\path[draw=drawColor,line width= 0.4pt,line join=round,line cap=round,fill=fillColor] (187.79,515.26) circle (  1.16);

\path[draw=drawColor,line width= 0.4pt,line join=round,line cap=round,fill=fillColor] (187.86,515.20) circle (  1.16);

\path[draw=drawColor,line width= 0.4pt,line join=round,line cap=round,fill=fillColor] (187.92,515.18) circle (  1.16);

\path[draw=drawColor,line width= 0.4pt,line join=round,line cap=round,fill=fillColor] (187.99,515.16) circle (  1.16);

\path[draw=drawColor,line width= 0.4pt,line join=round,line cap=round,fill=fillColor] (188.06,515.15) circle (  1.16);

\path[draw=drawColor,line width= 0.4pt,line join=round,line cap=round,fill=fillColor] (188.12,515.07) circle (  1.16);

\path[draw=drawColor,line width= 0.4pt,line join=round,line cap=round,fill=fillColor] (188.19,515.00) circle (  1.16);

\path[draw=drawColor,line width= 0.4pt,line join=round,line cap=round,fill=fillColor] (188.26,514.99) circle (  1.16);

\path[draw=drawColor,line width= 0.4pt,line join=round,line cap=round,fill=fillColor] (188.32,514.93) circle (  1.16);

\path[draw=drawColor,line width= 0.4pt,line join=round,line cap=round,fill=fillColor] (188.39,514.90) circle (  1.16);

\path[draw=drawColor,line width= 0.4pt,line join=round,line cap=round,fill=fillColor] (188.46,514.88) circle (  1.16);

\path[draw=drawColor,line width= 0.4pt,line join=round,line cap=round,fill=fillColor] (188.52,514.86) circle (  1.16);

\path[draw=drawColor,line width= 0.4pt,line join=round,line cap=round,fill=fillColor] (188.59,514.86) circle (  1.16);

\path[draw=drawColor,line width= 0.4pt,line join=round,line cap=round,fill=fillColor] (188.66,514.86) circle (  1.16);

\path[draw=drawColor,line width= 0.4pt,line join=round,line cap=round,fill=fillColor] (188.72,514.84) circle (  1.16);

\path[draw=drawColor,line width= 0.4pt,line join=round,line cap=round,fill=fillColor] (188.79,514.82) circle (  1.16);

\path[draw=drawColor,line width= 0.4pt,line join=round,line cap=round,fill=fillColor] (188.85,514.80) circle (  1.16);

\path[draw=drawColor,line width= 0.4pt,line join=round,line cap=round,fill=fillColor] (188.92,514.76) circle (  1.16);

\path[draw=drawColor,line width= 0.4pt,line join=round,line cap=round,fill=fillColor] (188.99,514.75) circle (  1.16);

\path[draw=drawColor,line width= 0.4pt,line join=round,line cap=round,fill=fillColor] (189.05,514.74) circle (  1.16);

\path[draw=drawColor,line width= 0.4pt,line join=round,line cap=round,fill=fillColor] (189.12,514.70) circle (  1.16);

\path[draw=drawColor,line width= 0.4pt,line join=round,line cap=round,fill=fillColor] (189.18,514.66) circle (  1.16);

\path[draw=drawColor,line width= 0.4pt,line join=round,line cap=round,fill=fillColor] (189.25,514.65) circle (  1.16);

\path[draw=drawColor,line width= 0.4pt,line join=round,line cap=round,fill=fillColor] (189.31,514.63) circle (  1.16);

\path[draw=drawColor,line width= 0.4pt,line join=round,line cap=round,fill=fillColor] (189.38,514.58) circle (  1.16);

\path[draw=drawColor,line width= 0.4pt,line join=round,line cap=round,fill=fillColor] (189.45,514.56) circle (  1.16);

\path[draw=drawColor,line width= 0.4pt,line join=round,line cap=round,fill=fillColor] (189.51,514.55) circle (  1.16);

\path[draw=drawColor,line width= 0.4pt,line join=round,line cap=round,fill=fillColor] (189.58,514.52) circle (  1.16);

\path[draw=drawColor,line width= 0.4pt,line join=round,line cap=round,fill=fillColor] (189.64,514.43) circle (  1.16);

\path[draw=drawColor,line width= 0.4pt,line join=round,line cap=round,fill=fillColor] (189.71,514.43) circle (  1.16);

\path[draw=drawColor,line width= 0.4pt,line join=round,line cap=round,fill=fillColor] (189.77,514.38) circle (  1.16);

\path[draw=drawColor,line width= 0.4pt,line join=round,line cap=round,fill=fillColor] (189.84,514.36) circle (  1.16);

\path[draw=drawColor,line width= 0.4pt,line join=round,line cap=round,fill=fillColor] (189.90,514.30) circle (  1.16);

\path[draw=drawColor,line width= 0.4pt,line join=round,line cap=round,fill=fillColor] (189.97,514.25) circle (  1.16);

\path[draw=drawColor,line width= 0.4pt,line join=round,line cap=round,fill=fillColor] (190.03,514.21) circle (  1.16);

\path[draw=drawColor,line width= 0.4pt,line join=round,line cap=round,fill=fillColor] (190.10,514.18) circle (  1.16);

\path[draw=drawColor,line width= 0.4pt,line join=round,line cap=round,fill=fillColor] (190.16,514.13) circle (  1.16);

\path[draw=drawColor,line width= 0.4pt,line join=round,line cap=round,fill=fillColor] (190.23,514.13) circle (  1.16);

\path[draw=drawColor,line width= 0.4pt,line join=round,line cap=round,fill=fillColor] (190.29,514.07) circle (  1.16);

\path[draw=drawColor,line width= 0.4pt,line join=round,line cap=round,fill=fillColor] (190.35,514.07) circle (  1.16);

\path[draw=drawColor,line width= 0.4pt,line join=round,line cap=round,fill=fillColor] (190.42,514.06) circle (  1.16);

\path[draw=drawColor,line width= 0.4pt,line join=round,line cap=round,fill=fillColor] (190.48,514.05) circle (  1.16);

\path[draw=drawColor,line width= 0.4pt,line join=round,line cap=round,fill=fillColor] (190.55,514.04) circle (  1.16);

\path[draw=drawColor,line width= 0.4pt,line join=round,line cap=round,fill=fillColor] (190.61,514.01) circle (  1.16);

\path[draw=drawColor,line width= 0.4pt,line join=round,line cap=round,fill=fillColor] (190.68,513.99) circle (  1.16);

\path[draw=drawColor,line width= 0.4pt,line join=round,line cap=round,fill=fillColor] (190.74,513.97) circle (  1.16);

\path[draw=drawColor,line width= 0.4pt,line join=round,line cap=round,fill=fillColor] (190.80,513.70) circle (  1.16);

\path[draw=drawColor,line width= 0.4pt,line join=round,line cap=round,fill=fillColor] (190.87,513.68) circle (  1.16);

\path[draw=drawColor,line width= 0.4pt,line join=round,line cap=round,fill=fillColor] (190.93,513.58) circle (  1.16);

\path[draw=drawColor,line width= 0.4pt,line join=round,line cap=round,fill=fillColor] (190.99,513.58) circle (  1.16);

\path[draw=drawColor,line width= 0.4pt,line join=round,line cap=round,fill=fillColor] (191.06,513.57) circle (  1.16);

\path[draw=drawColor,line width= 0.4pt,line join=round,line cap=round,fill=fillColor] (191.12,513.53) circle (  1.16);

\path[draw=drawColor,line width= 0.4pt,line join=round,line cap=round,fill=fillColor] (191.19,513.52) circle (  1.16);

\path[draw=drawColor,line width= 0.4pt,line join=round,line cap=round,fill=fillColor] (191.25,513.50) circle (  1.16);

\path[draw=drawColor,line width= 0.4pt,line join=round,line cap=round,fill=fillColor] (191.31,513.41) circle (  1.16);

\path[draw=drawColor,line width= 0.4pt,line join=round,line cap=round,fill=fillColor] (191.38,513.38) circle (  1.16);

\path[draw=drawColor,line width= 0.4pt,line join=round,line cap=round,fill=fillColor] (191.44,513.34) circle (  1.16);

\path[draw=drawColor,line width= 0.4pt,line join=round,line cap=round,fill=fillColor] (191.50,513.31) circle (  1.16);

\path[draw=drawColor,line width= 0.4pt,line join=round,line cap=round,fill=fillColor] (191.57,513.28) circle (  1.16);

\path[draw=drawColor,line width= 0.4pt,line join=round,line cap=round,fill=fillColor] (191.63,513.24) circle (  1.16);

\path[draw=drawColor,line width= 0.4pt,line join=round,line cap=round,fill=fillColor] (191.69,513.23) circle (  1.16);

\path[draw=drawColor,line width= 0.4pt,line join=round,line cap=round,fill=fillColor] (191.75,513.21) circle (  1.16);

\path[draw=drawColor,line width= 0.4pt,line join=round,line cap=round,fill=fillColor] (191.82,513.18) circle (  1.16);

\path[draw=drawColor,line width= 0.4pt,line join=round,line cap=round,fill=fillColor] (191.88,512.95) circle (  1.16);

\path[draw=drawColor,line width= 0.4pt,line join=round,line cap=round,fill=fillColor] (191.94,512.93) circle (  1.16);

\path[draw=drawColor,line width= 0.4pt,line join=round,line cap=round,fill=fillColor] (192.01,512.79) circle (  1.16);

\path[draw=drawColor,line width= 0.4pt,line join=round,line cap=round,fill=fillColor] (192.07,512.67) circle (  1.16);

\path[draw=drawColor,line width= 0.4pt,line join=round,line cap=round,fill=fillColor] (192.13,512.67) circle (  1.16);

\path[draw=drawColor,line width= 0.4pt,line join=round,line cap=round,fill=fillColor] (192.19,512.66) circle (  1.16);

\path[draw=drawColor,line width= 0.4pt,line join=round,line cap=round,fill=fillColor] (192.26,512.56) circle (  1.16);

\path[draw=drawColor,line width= 0.4pt,line join=round,line cap=round,fill=fillColor] (192.32,512.47) circle (  1.16);

\path[draw=drawColor,line width= 0.4pt,line join=round,line cap=round,fill=fillColor] (192.38,512.44) circle (  1.16);

\path[draw=drawColor,line width= 0.4pt,line join=round,line cap=round,fill=fillColor] (192.44,512.41) circle (  1.16);

\path[draw=drawColor,line width= 0.4pt,line join=round,line cap=round,fill=fillColor] (192.51,512.40) circle (  1.16);

\path[draw=drawColor,line width= 0.4pt,line join=round,line cap=round,fill=fillColor] (192.57,512.26) circle (  1.16);

\path[draw=drawColor,line width= 0.4pt,line join=round,line cap=round,fill=fillColor] (192.63,512.24) circle (  1.16);

\path[draw=drawColor,line width= 0.4pt,line join=round,line cap=round,fill=fillColor] (192.69,512.23) circle (  1.16);

\path[draw=drawColor,line width= 0.4pt,line join=round,line cap=round,fill=fillColor] (192.75,512.00) circle (  1.16);

\path[draw=drawColor,line width= 0.4pt,line join=round,line cap=round,fill=fillColor] (192.82,511.96) circle (  1.16);

\path[draw=drawColor,line width= 0.4pt,line join=round,line cap=round,fill=fillColor] (192.88,511.90) circle (  1.16);

\path[draw=drawColor,line width= 0.4pt,line join=round,line cap=round,fill=fillColor] (192.94,511.66) circle (  1.16);

\path[draw=drawColor,line width= 0.4pt,line join=round,line cap=round,fill=fillColor] (193.00,511.65) circle (  1.16);

\path[draw=drawColor,line width= 0.4pt,line join=round,line cap=round,fill=fillColor] (193.06,511.31) circle (  1.16);

\path[draw=drawColor,line width= 0.4pt,line join=round,line cap=round,fill=fillColor] (193.13,511.26) circle (  1.16);

\path[draw=drawColor,line width= 0.4pt,line join=round,line cap=round,fill=fillColor] (193.19,510.00) circle (  1.16);

\path[draw=drawColor,line width= 0.4pt,line join=round,line cap=round,fill=fillColor] (193.25,508.31) circle (  1.16);

\path[draw=drawColor,line width= 0.4pt,line join=round,line cap=round,fill=fillColor] (193.31,508.31) circle (  1.16);

\path[draw=drawColor,line width= 0.4pt,line join=round,line cap=round,fill=fillColor] (193.37,508.31) circle (  1.16);

\path[draw=drawColor,line width= 0.4pt,line join=round,line cap=round,fill=fillColor] (193.43,508.31) circle (  1.16);

\path[draw=drawColor,line width= 0.4pt,line join=round,line cap=round,fill=fillColor] (193.49,508.31) circle (  1.16);

\path[draw=drawColor,line width= 0.4pt,line join=round,line cap=round,fill=fillColor] (193.55,508.31) circle (  1.16);

\path[draw=drawColor,line width= 0.4pt,line join=round,line cap=round,fill=fillColor] (193.62,508.31) circle (  1.16);

\path[draw=drawColor,line width= 0.4pt,line join=round,line cap=round,fill=fillColor] (193.68,508.31) circle (  1.16);

\path[draw=drawColor,line width= 0.4pt,line join=round,line cap=round,fill=fillColor] (193.74,508.31) circle (  1.16);

\path[draw=drawColor,line width= 0.4pt,line join=round,line cap=round,fill=fillColor] (193.80,508.31) circle (  1.16);

\path[draw=drawColor,line width= 0.4pt,line join=round,line cap=round,fill=fillColor] (193.86,508.31) circle (  1.16);

\path[draw=drawColor,line width= 0.4pt,line join=round,line cap=round,fill=fillColor] (193.92,508.31) circle (  1.16);

\path[draw=drawColor,line width= 0.4pt,line join=round,line cap=round,fill=fillColor] (193.98,508.31) circle (  1.16);

\path[draw=drawColor,line width= 0.4pt,line join=round,line cap=round,fill=fillColor] (194.04,508.31) circle (  1.16);

\path[draw=drawColor,line width= 0.4pt,line join=round,line cap=round,fill=fillColor] (194.10,508.31) circle (  1.16);

\path[draw=drawColor,line width= 0.4pt,line join=round,line cap=round,fill=fillColor] (194.16,508.31) circle (  1.16);

\path[draw=drawColor,line width= 0.4pt,line join=round,line cap=round,fill=fillColor] (194.22,508.31) circle (  1.16);

\path[draw=drawColor,line width= 0.4pt,line join=round,line cap=round,fill=fillColor] (194.29,508.31) circle (  1.16);

\path[draw=drawColor,line width= 0.4pt,line join=round,line cap=round,fill=fillColor] (194.35,508.31) circle (  1.16);

\path[draw=drawColor,line width= 0.4pt,line join=round,line cap=round,fill=fillColor] (194.41,508.31) circle (  1.16);

\path[draw=drawColor,line width= 0.4pt,line join=round,line cap=round,fill=fillColor] (194.47,508.31) circle (  1.16);

\path[draw=drawColor,line width= 0.4pt,line join=round,line cap=round,fill=fillColor] (194.53,508.31) circle (  1.16);

\path[draw=drawColor,line width= 0.4pt,line join=round,line cap=round,fill=fillColor] (194.59,508.31) circle (  1.16);

\path[draw=drawColor,line width= 0.4pt,line join=round,line cap=round,fill=fillColor] (194.65,508.31) circle (  1.16);

\path[draw=drawColor,line width= 0.4pt,line join=round,line cap=round,fill=fillColor] (194.71,508.31) circle (  1.16);

\path[draw=drawColor,line width= 0.4pt,line join=round,line cap=round,fill=fillColor] (194.77,508.31) circle (  1.16);

\path[draw=drawColor,line width= 0.4pt,line join=round,line cap=round,fill=fillColor] (194.83,508.31) circle (  1.16);

\path[draw=drawColor,line width= 0.4pt,line join=round,line cap=round,fill=fillColor] (194.89,508.31) circle (  1.16);

\path[draw=drawColor,line width= 0.4pt,line join=round,line cap=round,fill=fillColor] (194.95,508.31) circle (  1.16);

\path[draw=drawColor,line width= 0.4pt,line join=round,line cap=round,fill=fillColor] (195.01,508.31) circle (  1.16);

\path[draw=drawColor,line width= 0.4pt,line join=round,line cap=round,fill=fillColor] (195.07,508.31) circle (  1.16);

\path[draw=drawColor,line width= 0.4pt,line join=round,line cap=round,fill=fillColor] (195.13,508.31) circle (  1.16);

\path[draw=drawColor,line width= 0.4pt,line join=round,line cap=round,fill=fillColor] (195.19,508.31) circle (  1.16);

\path[draw=drawColor,line width= 0.4pt,line join=round,line cap=round,fill=fillColor] (195.25,508.31) circle (  1.16);

\path[draw=drawColor,line width= 0.4pt,line join=round,line cap=round,fill=fillColor] (195.31,508.31) circle (  1.16);

\path[draw=drawColor,line width= 0.4pt,line join=round,line cap=round,fill=fillColor] (195.36,508.31) circle (  1.16);

\path[draw=drawColor,line width= 0.4pt,line join=round,line cap=round,fill=fillColor] (195.42,508.31) circle (  1.16);

\path[draw=drawColor,line width= 0.4pt,line join=round,line cap=round,fill=fillColor] (195.48,508.31) circle (  1.16);

\path[draw=drawColor,line width= 0.4pt,line join=round,line cap=round,fill=fillColor] (195.54,508.31) circle (  1.16);

\path[draw=drawColor,line width= 0.4pt,line join=round,line cap=round,fill=fillColor] (195.60,508.31) circle (  1.16);

\path[draw=drawColor,line width= 0.4pt,line join=round,line cap=round,fill=fillColor] (195.66,508.31) circle (  1.16);

\path[draw=drawColor,line width= 0.4pt,line join=round,line cap=round,fill=fillColor] (195.72,508.31) circle (  1.16);

\path[draw=drawColor,line width= 0.4pt,line join=round,line cap=round,fill=fillColor] (195.78,508.31) circle (  1.16);

\path[draw=drawColor,line width= 0.4pt,line join=round,line cap=round,fill=fillColor] (195.84,508.31) circle (  1.16);

\path[draw=drawColor,line width= 0.4pt,line join=round,line cap=round,fill=fillColor] (195.90,508.31) circle (  1.16);

\path[draw=drawColor,line width= 0.4pt,line join=round,line cap=round,fill=fillColor] (195.96,508.31) circle (  1.16);

\path[draw=drawColor,line width= 0.4pt,line join=round,line cap=round,fill=fillColor] (196.02,508.31) circle (  1.16);

\path[draw=drawColor,line width= 0.4pt,line join=round,line cap=round,fill=fillColor] (196.07,508.31) circle (  1.16);

\path[draw=drawColor,line width= 0.4pt,line join=round,line cap=round,fill=fillColor] (196.13,508.31) circle (  1.16);

\path[draw=drawColor,line width= 0.4pt,line join=round,line cap=round,fill=fillColor] (196.19,508.31) circle (  1.16);

\path[draw=drawColor,line width= 0.4pt,line join=round,line cap=round,fill=fillColor] (196.25,508.31) circle (  1.16);

\path[draw=drawColor,line width= 0.4pt,line join=round,line cap=round,fill=fillColor] (196.31,508.31) circle (  1.16);

\path[draw=drawColor,line width= 0.4pt,line join=round,line cap=round,fill=fillColor] (196.37,508.31) circle (  1.16);

\path[draw=drawColor,line width= 0.4pt,line join=round,line cap=round,fill=fillColor] (196.43,508.31) circle (  1.16);

\path[draw=drawColor,line width= 0.4pt,line join=round,line cap=round,fill=fillColor] (196.48,508.31) circle (  1.16);

\path[draw=drawColor,line width= 0.4pt,line join=round,line cap=round,fill=fillColor] (196.54,508.31) circle (  1.16);

\path[draw=drawColor,line width= 0.4pt,line join=round,line cap=round,fill=fillColor] (196.60,508.31) circle (  1.16);

\path[draw=drawColor,line width= 0.4pt,line join=round,line cap=round,fill=fillColor] (196.66,508.31) circle (  1.16);

\path[draw=drawColor,line width= 0.4pt,line join=round,line cap=round,fill=fillColor] (196.72,508.31) circle (  1.16);

\path[draw=drawColor,line width= 0.4pt,line join=round,line cap=round,fill=fillColor] (196.78,508.31) circle (  1.16);

\path[draw=drawColor,line width= 0.4pt,line join=round,line cap=round,fill=fillColor] (196.83,508.31) circle (  1.16);

\path[draw=drawColor,line width= 0.4pt,line join=round,line cap=round,fill=fillColor] (196.89,508.31) circle (  1.16);

\path[draw=drawColor,line width= 0.4pt,line join=round,line cap=round,fill=fillColor] (196.95,508.31) circle (  1.16);

\path[draw=drawColor,line width= 0.4pt,line join=round,line cap=round,fill=fillColor] (197.01,508.31) circle (  1.16);

\path[draw=drawColor,line width= 0.4pt,line join=round,line cap=round,fill=fillColor] (197.07,508.31) circle (  1.16);

\path[draw=drawColor,line width= 0.4pt,line join=round,line cap=round,fill=fillColor] (197.12,508.31) circle (  1.16);

\path[draw=drawColor,line width= 0.4pt,line join=round,line cap=round,fill=fillColor] (197.18,508.31) circle (  1.16);

\path[draw=drawColor,line width= 0.4pt,line join=round,line cap=round,fill=fillColor] (197.24,508.31) circle (  1.16);

\path[draw=drawColor,line width= 0.4pt,line join=round,line cap=round,fill=fillColor] (197.30,508.31) circle (  1.16);

\path[draw=drawColor,line width= 0.4pt,line join=round,line cap=round,fill=fillColor] (197.36,508.31) circle (  1.16);

\path[draw=drawColor,line width= 0.4pt,line join=round,line cap=round,fill=fillColor] (197.41,508.31) circle (  1.16);

\path[draw=drawColor,line width= 0.4pt,line join=round,line cap=round,fill=fillColor] (197.47,508.31) circle (  1.16);

\path[draw=drawColor,line width= 0.4pt,line join=round,line cap=round,fill=fillColor] (197.53,508.31) circle (  1.16);

\path[draw=drawColor,line width= 0.4pt,line join=round,line cap=round,fill=fillColor] (197.59,508.31) circle (  1.16);

\path[draw=drawColor,line width= 0.4pt,line join=round,line cap=round,fill=fillColor] (197.64,508.31) circle (  1.16);

\path[draw=drawColor,line width= 0.4pt,line join=round,line cap=round,fill=fillColor] (197.70,508.31) circle (  1.16);

\path[draw=drawColor,line width= 0.4pt,line join=round,line cap=round,fill=fillColor] (197.76,508.31) circle (  1.16);

\path[draw=drawColor,line width= 0.4pt,line join=round,line cap=round,fill=fillColor] (197.82,508.31) circle (  1.16);

\path[draw=drawColor,line width= 0.4pt,line join=round,line cap=round,fill=fillColor] (197.87,508.31) circle (  1.16);

\path[draw=drawColor,line width= 0.4pt,line join=round,line cap=round,fill=fillColor] (197.93,508.31) circle (  1.16);

\path[draw=drawColor,line width= 0.4pt,line join=round,line cap=round,fill=fillColor] (197.99,508.31) circle (  1.16);

\path[draw=drawColor,line width= 0.4pt,line join=round,line cap=round,fill=fillColor] (198.04,508.31) circle (  1.16);

\path[draw=drawColor,line width= 0.4pt,line join=round,line cap=round,fill=fillColor] (198.10,508.31) circle (  1.16);

\path[draw=drawColor,line width= 0.4pt,line join=round,line cap=round,fill=fillColor] (198.16,508.31) circle (  1.16);

\path[draw=drawColor,line width= 0.4pt,line join=round,line cap=round,fill=fillColor] (198.22,508.31) circle (  1.16);

\path[draw=drawColor,line width= 0.4pt,line join=round,line cap=round,fill=fillColor] (198.27,508.31) circle (  1.16);

\path[draw=drawColor,line width= 0.4pt,line join=round,line cap=round,fill=fillColor] (198.33,508.31) circle (  1.16);

\path[draw=drawColor,line width= 0.4pt,line join=round,line cap=round,fill=fillColor] (198.39,508.31) circle (  1.16);

\path[draw=drawColor,line width= 0.4pt,line join=round,line cap=round,fill=fillColor] (198.44,508.31) circle (  1.16);

\path[draw=drawColor,line width= 0.4pt,line join=round,line cap=round,fill=fillColor] (198.50,508.31) circle (  1.16);

\path[draw=drawColor,line width= 0.4pt,line join=round,line cap=round,fill=fillColor] (198.56,508.31) circle (  1.16);

\path[draw=drawColor,line width= 0.4pt,line join=round,line cap=round,fill=fillColor] (198.61,508.31) circle (  1.16);

\path[draw=drawColor,line width= 0.4pt,line join=round,line cap=round,fill=fillColor] (198.67,508.31) circle (  1.16);

\path[draw=drawColor,line width= 0.4pt,line join=round,line cap=round,fill=fillColor] (198.73,508.31) circle (  1.16);

\path[draw=drawColor,line width= 0.4pt,line join=round,line cap=round,fill=fillColor] (198.78,508.31) circle (  1.16);

\path[draw=drawColor,line width= 0.4pt,line join=round,line cap=round,fill=fillColor] (198.84,508.31) circle (  1.16);

\path[draw=drawColor,line width= 0.4pt,line join=round,line cap=round,fill=fillColor] (198.90,508.31) circle (  1.16);

\path[draw=drawColor,line width= 0.4pt,line join=round,line cap=round,fill=fillColor] (198.95,508.31) circle (  1.16);

\path[draw=drawColor,line width= 0.4pt,line join=round,line cap=round,fill=fillColor] (199.01,508.31) circle (  1.16);

\path[draw=drawColor,line width= 0.4pt,line join=round,line cap=round,fill=fillColor] (199.06,508.31) circle (  1.16);

\path[draw=drawColor,line width= 0.4pt,line join=round,line cap=round,fill=fillColor] (199.12,508.31) circle (  1.16);

\path[draw=drawColor,line width= 0.4pt,line join=round,line cap=round,fill=fillColor] (199.18,508.31) circle (  1.16);

\path[draw=drawColor,line width= 0.4pt,line join=round,line cap=round,fill=fillColor] (199.23,508.31) circle (  1.16);

\path[draw=drawColor,line width= 0.4pt,line join=round,line cap=round,fill=fillColor] (199.29,508.31) circle (  1.16);

\path[draw=drawColor,line width= 0.4pt,line join=round,line cap=round,fill=fillColor] (199.34,508.31) circle (  1.16);

\path[draw=drawColor,line width= 0.4pt,line join=round,line cap=round,fill=fillColor] (199.40,508.31) circle (  1.16);

\path[draw=drawColor,line width= 0.4pt,line join=round,line cap=round,fill=fillColor] (199.46,508.31) circle (  1.16);

\path[draw=drawColor,line width= 0.4pt,line join=round,line cap=round,fill=fillColor] (199.51,508.31) circle (  1.16);

\path[draw=drawColor,line width= 0.4pt,line join=round,line cap=round,fill=fillColor] (199.57,508.31) circle (  1.16);

\path[draw=drawColor,line width= 0.4pt,line join=round,line cap=round,fill=fillColor] (199.62,508.31) circle (  1.16);

\path[draw=drawColor,line width= 0.4pt,line join=round,line cap=round,fill=fillColor] (199.68,508.31) circle (  1.16);

\path[draw=drawColor,line width= 0.4pt,line join=round,line cap=round,fill=fillColor] (199.73,508.31) circle (  1.16);

\path[draw=drawColor,line width= 0.4pt,line join=round,line cap=round,fill=fillColor] (199.79,508.31) circle (  1.16);

\path[draw=drawColor,line width= 0.4pt,line join=round,line cap=round,fill=fillColor] (199.85,508.31) circle (  1.16);

\path[draw=drawColor,line width= 0.4pt,line join=round,line cap=round,fill=fillColor] (199.90,508.31) circle (  1.16);

\path[draw=drawColor,line width= 0.4pt,line join=round,line cap=round,fill=fillColor] (199.96,508.31) circle (  1.16);

\path[draw=drawColor,line width= 0.4pt,line join=round,line cap=round,fill=fillColor] (200.01,508.31) circle (  1.16);

\path[draw=drawColor,line width= 0.4pt,line join=round,line cap=round,fill=fillColor] (200.07,508.31) circle (  1.16);

\path[draw=drawColor,line width= 0.4pt,line join=round,line cap=round,fill=fillColor] (200.12,508.31) circle (  1.16);

\path[draw=drawColor,line width= 0.4pt,line join=round,line cap=round,fill=fillColor] (200.18,508.31) circle (  1.16);

\path[draw=drawColor,line width= 0.4pt,line join=round,line cap=round,fill=fillColor] (200.23,508.31) circle (  1.16);

\path[draw=drawColor,line width= 0.4pt,line join=round,line cap=round,fill=fillColor] (200.29,508.31) circle (  1.16);

\path[draw=drawColor,line width= 0.4pt,line join=round,line cap=round,fill=fillColor] (200.34,508.31) circle (  1.16);

\path[draw=drawColor,line width= 0.4pt,line join=round,line cap=round,fill=fillColor] (200.40,508.31) circle (  1.16);

\path[draw=drawColor,line width= 0.4pt,line join=round,line cap=round,fill=fillColor] (200.45,508.31) circle (  1.16);

\path[draw=drawColor,line width= 0.4pt,line join=round,line cap=round,fill=fillColor] (200.51,508.31) circle (  1.16);

\path[draw=drawColor,line width= 0.4pt,line join=round,line cap=round,fill=fillColor] (200.56,508.31) circle (  1.16);

\path[draw=drawColor,line width= 0.4pt,line join=round,line cap=round,fill=fillColor] (200.62,508.31) circle (  1.16);

\path[draw=drawColor,line width= 0.4pt,line join=round,line cap=round,fill=fillColor] (200.67,508.31) circle (  1.16);

\path[draw=drawColor,line width= 0.4pt,line join=round,line cap=round,fill=fillColor] (200.73,508.31) circle (  1.16);

\path[draw=drawColor,line width= 0.4pt,line join=round,line cap=round,fill=fillColor] (200.78,508.31) circle (  1.16);

\path[draw=drawColor,line width= 0.4pt,line join=round,line cap=round,fill=fillColor] (200.84,508.31) circle (  1.16);

\path[draw=drawColor,line width= 0.4pt,line join=round,line cap=round,fill=fillColor] (200.89,508.31) circle (  1.16);

\path[draw=drawColor,line width= 0.4pt,line join=round,line cap=round,fill=fillColor] (200.95,508.31) circle (  1.16);

\path[draw=drawColor,line width= 0.4pt,line join=round,line cap=round,fill=fillColor] (201.00,508.31) circle (  1.16);

\path[draw=drawColor,line width= 0.4pt,line join=round,line cap=round,fill=fillColor] (201.06,508.31) circle (  1.16);

\path[draw=drawColor,line width= 0.4pt,line join=round,line cap=round,fill=fillColor] (201.11,508.31) circle (  1.16);

\path[draw=drawColor,line width= 0.4pt,line join=round,line cap=round,fill=fillColor] (201.17,508.31) circle (  1.16);

\path[draw=drawColor,line width= 0.4pt,line join=round,line cap=round,fill=fillColor] (201.22,508.31) circle (  1.16);

\path[draw=drawColor,line width= 0.4pt,line join=round,line cap=round,fill=fillColor] (201.27,508.31) circle (  1.16);

\path[draw=drawColor,line width= 0.4pt,line join=round,line cap=round,fill=fillColor] (201.33,508.31) circle (  1.16);

\path[draw=drawColor,line width= 0.4pt,line join=round,line cap=round,fill=fillColor] (201.38,508.31) circle (  1.16);

\path[draw=drawColor,line width= 0.4pt,line join=round,line cap=round,fill=fillColor] (201.44,508.31) circle (  1.16);

\path[draw=drawColor,line width= 0.4pt,line join=round,line cap=round,fill=fillColor] (201.49,508.31) circle (  1.16);

\path[draw=drawColor,line width= 0.4pt,line join=round,line cap=round,fill=fillColor] (201.55,508.31) circle (  1.16);

\path[draw=drawColor,line width= 0.4pt,line join=round,line cap=round,fill=fillColor] (201.60,508.31) circle (  1.16);

\path[draw=drawColor,line width= 0.4pt,line join=round,line cap=round,fill=fillColor] (201.65,508.31) circle (  1.16);

\path[draw=drawColor,line width= 0.4pt,line join=round,line cap=round,fill=fillColor] (201.71,508.31) circle (  1.16);

\path[draw=drawColor,line width= 0.4pt,line join=round,line cap=round,fill=fillColor] (201.76,508.31) circle (  1.16);

\path[draw=drawColor,line width= 0.4pt,line join=round,line cap=round,fill=fillColor] (201.82,508.31) circle (  1.16);

\path[draw=drawColor,line width= 0.4pt,line join=round,line cap=round,fill=fillColor] (201.87,508.31) circle (  1.16);

\path[draw=drawColor,line width= 0.4pt,line join=round,line cap=round,fill=fillColor] (201.92,508.31) circle (  1.16);

\path[draw=drawColor,line width= 0.4pt,line join=round,line cap=round,fill=fillColor] (201.98,508.31) circle (  1.16);

\path[draw=drawColor,line width= 0.4pt,line join=round,line cap=round,fill=fillColor] (202.03,508.31) circle (  1.16);

\path[draw=drawColor,line width= 0.4pt,line join=round,line cap=round,fill=fillColor] (202.09,508.31) circle (  1.16);

\path[draw=drawColor,line width= 0.4pt,line join=round,line cap=round,fill=fillColor] (202.14,508.31) circle (  1.16);

\path[draw=drawColor,line width= 0.4pt,line join=round,line cap=round,fill=fillColor] (202.19,508.31) circle (  1.16);

\path[draw=drawColor,line width= 0.4pt,line join=round,line cap=round,fill=fillColor] (202.25,508.31) circle (  1.16);

\path[draw=drawColor,line width= 0.4pt,line join=round,line cap=round,fill=fillColor] (202.30,508.31) circle (  1.16);

\path[draw=drawColor,line width= 0.4pt,line join=round,line cap=round,fill=fillColor] (202.35,508.31) circle (  1.16);

\path[draw=drawColor,line width= 0.4pt,line join=round,line cap=round,fill=fillColor] (202.41,508.31) circle (  1.16);

\path[draw=drawColor,line width= 0.4pt,line join=round,line cap=round,fill=fillColor] (202.46,508.31) circle (  1.16);

\path[draw=drawColor,line width= 0.4pt,line join=round,line cap=round,fill=fillColor] (202.51,508.31) circle (  1.16);

\path[draw=drawColor,line width= 0.4pt,line join=round,line cap=round,fill=fillColor] (202.57,508.31) circle (  1.16);

\path[draw=drawColor,line width= 0.4pt,line join=round,line cap=round,fill=fillColor] (202.62,508.31) circle (  1.16);

\path[draw=drawColor,line width= 0.4pt,line join=round,line cap=round,fill=fillColor] (202.67,508.31) circle (  1.16);

\path[draw=drawColor,line width= 0.4pt,line join=round,line cap=round,fill=fillColor] (202.73,508.31) circle (  1.16);

\path[draw=drawColor,line width= 0.4pt,line join=round,line cap=round,fill=fillColor] (202.78,508.31) circle (  1.16);

\path[draw=drawColor,line width= 0.4pt,line join=round,line cap=round,fill=fillColor] (202.83,508.31) circle (  1.16);

\path[draw=drawColor,line width= 0.4pt,line join=round,line cap=round,fill=fillColor] (202.89,508.31) circle (  1.16);

\path[draw=drawColor,line width= 0.4pt,line join=round,line cap=round,fill=fillColor] (202.94,508.31) circle (  1.16);

\path[draw=drawColor,line width= 0.4pt,line join=round,line cap=round,fill=fillColor] (202.99,508.31) circle (  1.16);

\path[draw=drawColor,line width= 0.4pt,line join=round,line cap=round,fill=fillColor] (203.05,508.31) circle (  1.16);

\path[draw=drawColor,line width= 0.4pt,line join=round,line cap=round,fill=fillColor] (203.10,508.31) circle (  1.16);

\path[draw=drawColor,line width= 0.4pt,line join=round,line cap=round,fill=fillColor] (203.15,508.31) circle (  1.16);

\path[draw=drawColor,line width= 0.4pt,line join=round,line cap=round,fill=fillColor] (203.20,508.31) circle (  1.16);

\path[draw=drawColor,line width= 0.4pt,line join=round,line cap=round,fill=fillColor] (203.26,508.31) circle (  1.16);

\path[draw=drawColor,line width= 0.4pt,line join=round,line cap=round,fill=fillColor] (203.31,508.31) circle (  1.16);

\path[draw=drawColor,line width= 0.4pt,line join=round,line cap=round,fill=fillColor] (203.36,508.31) circle (  1.16);

\path[draw=drawColor,line width= 0.4pt,line join=round,line cap=round,fill=fillColor] (203.41,508.31) circle (  1.16);

\path[draw=drawColor,line width= 0.4pt,line join=round,line cap=round,fill=fillColor] (203.47,508.31) circle (  1.16);

\path[draw=drawColor,line width= 0.4pt,line join=round,line cap=round,fill=fillColor] (203.52,508.31) circle (  1.16);

\path[draw=drawColor,line width= 0.4pt,line join=round,line cap=round,fill=fillColor] (203.57,508.31) circle (  1.16);

\path[draw=drawColor,line width= 0.4pt,line join=round,line cap=round,fill=fillColor] (203.63,508.31) circle (  1.16);

\path[draw=drawColor,line width= 0.4pt,line join=round,line cap=round,fill=fillColor] (203.68,508.31) circle (  1.16);

\path[draw=drawColor,line width= 0.4pt,line join=round,line cap=round,fill=fillColor] (203.73,508.31) circle (  1.16);

\path[draw=drawColor,line width= 0.4pt,line join=round,line cap=round,fill=fillColor] (203.78,508.31) circle (  1.16);

\path[draw=drawColor,line width= 0.4pt,line join=round,line cap=round,fill=fillColor] (203.83,508.31) circle (  1.16);

\path[draw=drawColor,line width= 0.4pt,line join=round,line cap=round,fill=fillColor] (203.89,508.31) circle (  1.16);

\path[draw=drawColor,line width= 0.4pt,line join=round,line cap=round,fill=fillColor] (203.94,508.31) circle (  1.16);

\path[draw=drawColor,line width= 0.4pt,line join=round,line cap=round,fill=fillColor] (203.99,508.31) circle (  1.16);

\path[draw=drawColor,line width= 0.4pt,line join=round,line cap=round,fill=fillColor] (204.04,508.31) circle (  1.16);

\path[draw=drawColor,line width= 0.4pt,line join=round,line cap=round,fill=fillColor] (204.10,508.31) circle (  1.16);

\path[draw=drawColor,line width= 0.4pt,line join=round,line cap=round,fill=fillColor] (204.15,508.31) circle (  1.16);

\path[draw=drawColor,line width= 0.4pt,line join=round,line cap=round,fill=fillColor] (204.20,508.31) circle (  1.16);

\path[draw=drawColor,line width= 0.4pt,line join=round,line cap=round,fill=fillColor] (204.25,508.31) circle (  1.16);

\path[draw=drawColor,line width= 0.4pt,line join=round,line cap=round,fill=fillColor] (204.30,508.31) circle (  1.16);

\path[draw=drawColor,line width= 0.4pt,line join=round,line cap=round,fill=fillColor] (204.36,508.31) circle (  1.16);

\path[draw=drawColor,line width= 0.4pt,line join=round,line cap=round,fill=fillColor] (204.41,508.31) circle (  1.16);

\path[draw=drawColor,line width= 0.4pt,line join=round,line cap=round,fill=fillColor] (204.46,508.31) circle (  1.16);

\path[draw=drawColor,line width= 0.4pt,line join=round,line cap=round,fill=fillColor] (204.51,508.31) circle (  1.16);

\path[draw=drawColor,line width= 0.4pt,line join=round,line cap=round,fill=fillColor] (204.56,508.31) circle (  1.16);

\path[draw=drawColor,line width= 0.4pt,line join=round,line cap=round,fill=fillColor] (204.62,508.31) circle (  1.16);

\path[draw=drawColor,line width= 0.4pt,line join=round,line cap=round,fill=fillColor] (204.67,508.31) circle (  1.16);

\path[draw=drawColor,line width= 0.4pt,line join=round,line cap=round,fill=fillColor] (204.72,508.31) circle (  1.16);

\path[draw=drawColor,line width= 0.4pt,line join=round,line cap=round,fill=fillColor] (204.77,508.31) circle (  1.16);

\path[draw=drawColor,line width= 0.4pt,line join=round,line cap=round,fill=fillColor] (204.82,508.31) circle (  1.16);

\path[draw=drawColor,line width= 0.4pt,line join=round,line cap=round,fill=fillColor] (204.87,508.31) circle (  1.16);

\path[draw=drawColor,line width= 0.4pt,line join=round,line cap=round,fill=fillColor] (204.93,508.31) circle (  1.16);

\path[draw=drawColor,line width= 0.4pt,line join=round,line cap=round,fill=fillColor] (204.98,508.31) circle (  1.16);

\path[draw=drawColor,line width= 0.4pt,line join=round,line cap=round,fill=fillColor] (205.03,508.31) circle (  1.16);

\path[draw=drawColor,line width= 0.4pt,line join=round,line cap=round,fill=fillColor] (205.08,508.31) circle (  1.16);

\path[draw=drawColor,line width= 0.4pt,line join=round,line cap=round,fill=fillColor] (205.13,508.31) circle (  1.16);

\path[draw=drawColor,line width= 0.4pt,line join=round,line cap=round,fill=fillColor] (205.18,508.31) circle (  1.16);

\path[draw=drawColor,line width= 0.4pt,line join=round,line cap=round,fill=fillColor] (205.23,508.31) circle (  1.16);

\path[draw=drawColor,line width= 0.4pt,line join=round,line cap=round,fill=fillColor] (205.29,508.31) circle (  1.16);

\path[draw=drawColor,line width= 0.4pt,line join=round,line cap=round,fill=fillColor] (205.34,508.31) circle (  1.16);

\path[draw=drawColor,line width= 0.4pt,line join=round,line cap=round,fill=fillColor] (205.39,508.31) circle (  1.16);

\path[draw=drawColor,line width= 0.4pt,line join=round,line cap=round,fill=fillColor] (205.44,508.31) circle (  1.16);

\path[draw=drawColor,line width= 0.4pt,line join=round,line cap=round,fill=fillColor] (205.49,508.31) circle (  1.16);

\path[draw=drawColor,line width= 0.4pt,line join=round,line cap=round,fill=fillColor] (205.54,508.31) circle (  1.16);

\path[draw=drawColor,line width= 0.4pt,line join=round,line cap=round,fill=fillColor] (205.59,508.31) circle (  1.16);

\path[draw=drawColor,line width= 0.4pt,line join=round,line cap=round,fill=fillColor] (205.64,508.31) circle (  1.16);

\path[draw=drawColor,line width= 0.4pt,line join=round,line cap=round,fill=fillColor] (205.69,508.31) circle (  1.16);

\path[draw=drawColor,line width= 0.4pt,line join=round,line cap=round,fill=fillColor] (205.75,508.31) circle (  1.16);

\path[draw=drawColor,line width= 0.4pt,line join=round,line cap=round,fill=fillColor] (205.80,508.31) circle (  1.16);

\path[draw=drawColor,line width= 0.4pt,line join=round,line cap=round,fill=fillColor] (205.85,508.31) circle (  1.16);

\path[draw=drawColor,line width= 0.4pt,line join=round,line cap=round,fill=fillColor] (205.90,508.31) circle (  1.16);

\path[draw=drawColor,line width= 0.4pt,line join=round,line cap=round,fill=fillColor] (205.95,508.31) circle (  1.16);

\path[draw=drawColor,line width= 0.4pt,line join=round,line cap=round,fill=fillColor] (206.00,508.31) circle (  1.16);

\path[draw=drawColor,line width= 0.4pt,line join=round,line cap=round,fill=fillColor] (206.05,508.31) circle (  1.16);

\path[draw=drawColor,line width= 0.4pt,line join=round,line cap=round,fill=fillColor] (206.10,508.31) circle (  1.16);

\path[draw=drawColor,line width= 0.4pt,line join=round,line cap=round,fill=fillColor] (206.15,508.31) circle (  1.16);

\path[draw=drawColor,line width= 0.4pt,line join=round,line cap=round,fill=fillColor] (206.20,508.31) circle (  1.16);

\path[draw=drawColor,line width= 0.4pt,line join=round,line cap=round,fill=fillColor] (206.25,508.31) circle (  1.16);

\path[draw=drawColor,line width= 0.4pt,line join=round,line cap=round,fill=fillColor] (206.30,508.31) circle (  1.16);

\path[draw=drawColor,line width= 0.4pt,line join=round,line cap=round,fill=fillColor] (206.35,508.31) circle (  1.16);

\path[draw=drawColor,line width= 0.4pt,line join=round,line cap=round,fill=fillColor] (206.40,508.31) circle (  1.16);

\path[draw=drawColor,line width= 0.4pt,line join=round,line cap=round,fill=fillColor] (206.45,508.31) circle (  1.16);

\path[draw=drawColor,line width= 0.4pt,line join=round,line cap=round,fill=fillColor] (206.50,508.31) circle (  1.16);

\path[draw=drawColor,line width= 0.4pt,line join=round,line cap=round,fill=fillColor] (206.55,508.31) circle (  1.16);

\path[draw=drawColor,line width= 0.4pt,line join=round,line cap=round,fill=fillColor] (206.61,508.31) circle (  1.16);

\path[draw=drawColor,line width= 0.4pt,line join=round,line cap=round,fill=fillColor] (206.66,508.31) circle (  1.16);

\path[draw=drawColor,line width= 0.4pt,line join=round,line cap=round,fill=fillColor] (206.71,508.31) circle (  1.16);

\path[draw=drawColor,line width= 0.4pt,line join=round,line cap=round,fill=fillColor] (206.76,508.31) circle (  1.16);

\path[draw=drawColor,line width= 0.4pt,line join=round,line cap=round,fill=fillColor] (206.81,508.31) circle (  1.16);

\path[draw=drawColor,line width= 0.4pt,line join=round,line cap=round,fill=fillColor] (206.86,508.31) circle (  1.16);

\path[draw=drawColor,line width= 0.4pt,line join=round,line cap=round,fill=fillColor] (206.91,508.31) circle (  1.16);

\path[draw=drawColor,line width= 0.4pt,line join=round,line cap=round,fill=fillColor] (206.96,508.31) circle (  1.16);

\path[draw=drawColor,line width= 0.4pt,line join=round,line cap=round,fill=fillColor] (207.01,508.31) circle (  1.16);

\path[draw=drawColor,line width= 0.4pt,line join=round,line cap=round,fill=fillColor] (207.06,508.31) circle (  1.16);

\path[draw=drawColor,line width= 0.4pt,line join=round,line cap=round,fill=fillColor] (207.11,508.31) circle (  1.16);

\path[draw=drawColor,line width= 0.4pt,line join=round,line cap=round,fill=fillColor] (207.16,508.31) circle (  1.16);

\path[draw=drawColor,line width= 0.4pt,line join=round,line cap=round,fill=fillColor] (207.21,508.31) circle (  1.16);

\path[draw=drawColor,line width= 0.4pt,line join=round,line cap=round,fill=fillColor] (207.26,508.31) circle (  1.16);

\path[draw=drawColor,line width= 0.4pt,line join=round,line cap=round,fill=fillColor] (207.31,508.31) circle (  1.16);

\path[draw=drawColor,line width= 0.4pt,line join=round,line cap=round,fill=fillColor] (207.36,508.31) circle (  1.16);

\path[draw=drawColor,line width= 0.4pt,line join=round,line cap=round,fill=fillColor] (207.41,508.31) circle (  1.16);

\path[draw=drawColor,line width= 0.4pt,line join=round,line cap=round,fill=fillColor] (207.46,508.31) circle (  1.16);

\path[draw=drawColor,line width= 0.4pt,line join=round,line cap=round,fill=fillColor] (207.50,508.31) circle (  1.16);

\path[draw=drawColor,line width= 0.4pt,line join=round,line cap=round,fill=fillColor] (207.55,508.31) circle (  1.16);

\path[draw=drawColor,line width= 0.4pt,line join=round,line cap=round,fill=fillColor] (207.60,508.31) circle (  1.16);

\path[draw=drawColor,line width= 0.4pt,line join=round,line cap=round,fill=fillColor] (207.65,508.31) circle (  1.16);

\path[draw=drawColor,line width= 0.4pt,line join=round,line cap=round,fill=fillColor] (207.70,508.31) circle (  1.16);

\path[draw=drawColor,line width= 0.4pt,line join=round,line cap=round,fill=fillColor] (207.75,508.31) circle (  1.16);

\path[draw=drawColor,line width= 0.4pt,line join=round,line cap=round,fill=fillColor] (207.80,508.31) circle (  1.16);

\path[draw=drawColor,line width= 0.4pt,line join=round,line cap=round,fill=fillColor] (207.85,508.31) circle (  1.16);

\path[draw=drawColor,line width= 0.4pt,line join=round,line cap=round,fill=fillColor] (207.90,508.31) circle (  1.16);

\path[draw=drawColor,line width= 0.4pt,line join=round,line cap=round,fill=fillColor] (207.95,508.31) circle (  1.16);

\path[draw=drawColor,line width= 0.4pt,line join=round,line cap=round,fill=fillColor] (208.00,508.31) circle (  1.16);

\path[draw=drawColor,line width= 0.4pt,line join=round,line cap=round,fill=fillColor] (208.05,508.31) circle (  1.16);

\path[draw=drawColor,line width= 0.4pt,line join=round,line cap=round,fill=fillColor] (208.10,508.31) circle (  1.16);

\path[draw=drawColor,line width= 0.4pt,line join=round,line cap=round,fill=fillColor] (208.15,508.31) circle (  1.16);

\path[draw=drawColor,line width= 0.4pt,line join=round,line cap=round,fill=fillColor] (208.20,508.31) circle (  1.16);

\path[draw=drawColor,line width= 0.4pt,line join=round,line cap=round,fill=fillColor] (208.25,508.31) circle (  1.16);

\path[draw=drawColor,line width= 0.4pt,line join=round,line cap=round,fill=fillColor] (208.29,508.31) circle (  1.16);

\path[draw=drawColor,line width= 0.4pt,line join=round,line cap=round,fill=fillColor] (208.34,508.31) circle (  1.16);

\path[draw=drawColor,line width= 0.4pt,line join=round,line cap=round,fill=fillColor] (208.39,508.31) circle (  1.16);

\path[draw=drawColor,line width= 0.4pt,line join=round,line cap=round,fill=fillColor] (208.44,508.31) circle (  1.16);

\path[draw=drawColor,line width= 0.4pt,line join=round,line cap=round,fill=fillColor] (208.49,508.31) circle (  1.16);

\path[draw=drawColor,line width= 0.4pt,line join=round,line cap=round,fill=fillColor] (208.54,508.31) circle (  1.16);

\path[draw=drawColor,line width= 0.4pt,line join=round,line cap=round,fill=fillColor] (208.59,508.31) circle (  1.16);

\path[draw=drawColor,line width= 0.4pt,line join=round,line cap=round,fill=fillColor] (208.64,508.31) circle (  1.16);

\path[draw=drawColor,line width= 0.4pt,line join=round,line cap=round,fill=fillColor] (208.69,508.31) circle (  1.16);

\path[draw=drawColor,line width= 0.4pt,line join=round,line cap=round,fill=fillColor] (208.74,508.31) circle (  1.16);

\path[draw=drawColor,line width= 0.4pt,line join=round,line cap=round,fill=fillColor] (208.78,508.31) circle (  1.16);

\path[draw=drawColor,line width= 0.4pt,line join=round,line cap=round,fill=fillColor] (208.83,508.31) circle (  1.16);

\path[draw=drawColor,line width= 0.4pt,line join=round,line cap=round,fill=fillColor] (208.88,508.31) circle (  1.16);

\path[draw=drawColor,line width= 0.4pt,line join=round,line cap=round,fill=fillColor] (208.93,508.31) circle (  1.16);

\path[draw=drawColor,line width= 0.4pt,line join=round,line cap=round,fill=fillColor] (208.98,508.31) circle (  1.16);

\path[draw=drawColor,line width= 0.4pt,line join=round,line cap=round,fill=fillColor] (209.03,508.31) circle (  1.16);

\path[draw=drawColor,line width= 0.4pt,line join=round,line cap=round,fill=fillColor] (209.08,508.31) circle (  1.16);

\path[draw=drawColor,line width= 0.4pt,line join=round,line cap=round,fill=fillColor] (209.12,508.31) circle (  1.16);

\path[draw=drawColor,line width= 0.4pt,line join=round,line cap=round,fill=fillColor] (209.17,508.31) circle (  1.16);

\path[draw=drawColor,line width= 0.4pt,line join=round,line cap=round,fill=fillColor] (209.22,508.31) circle (  1.16);

\path[draw=drawColor,line width= 0.4pt,line join=round,line cap=round,fill=fillColor] (209.27,508.31) circle (  1.16);

\path[draw=drawColor,line width= 0.4pt,line join=round,line cap=round,fill=fillColor] (209.32,508.31) circle (  1.16);

\path[draw=drawColor,line width= 0.4pt,line join=round,line cap=round,fill=fillColor] (209.37,508.31) circle (  1.16);

\path[draw=drawColor,line width= 0.4pt,line join=round,line cap=round,fill=fillColor] (209.42,508.31) circle (  1.16);

\path[draw=drawColor,line width= 0.4pt,line join=round,line cap=round,fill=fillColor] (209.46,508.31) circle (  1.16);

\path[draw=drawColor,line width= 0.4pt,line join=round,line cap=round,fill=fillColor] (209.51,508.31) circle (  1.16);

\path[draw=drawColor,line width= 0.4pt,line join=round,line cap=round,fill=fillColor] (209.56,508.31) circle (  1.16);

\path[draw=drawColor,line width= 0.4pt,line join=round,line cap=round,fill=fillColor] (209.61,508.31) circle (  1.16);

\path[draw=drawColor,line width= 0.4pt,line join=round,line cap=round,fill=fillColor] (209.66,508.31) circle (  1.16);

\path[draw=drawColor,line width= 0.4pt,line join=round,line cap=round,fill=fillColor] (209.70,508.31) circle (  1.16);

\path[draw=drawColor,line width= 0.4pt,line join=round,line cap=round,fill=fillColor] (209.75,508.31) circle (  1.16);

\path[draw=drawColor,line width= 0.4pt,line join=round,line cap=round,fill=fillColor] (209.80,508.31) circle (  1.16);

\path[draw=drawColor,line width= 0.4pt,line join=round,line cap=round,fill=fillColor] (209.85,508.31) circle (  1.16);

\path[draw=drawColor,line width= 0.4pt,line join=round,line cap=round,fill=fillColor] (209.90,508.31) circle (  1.16);

\path[draw=drawColor,line width= 0.4pt,line join=round,line cap=round,fill=fillColor] (209.94,508.31) circle (  1.16);

\path[draw=drawColor,line width= 0.4pt,line join=round,line cap=round,fill=fillColor] (209.99,508.31) circle (  1.16);

\path[draw=drawColor,line width= 0.4pt,line join=round,line cap=round,fill=fillColor] (210.04,508.31) circle (  1.16);

\path[draw=drawColor,line width= 0.4pt,line join=round,line cap=round,fill=fillColor] (210.09,508.31) circle (  1.16);

\path[draw=drawColor,line width= 0.4pt,line join=round,line cap=round,fill=fillColor] (210.14,508.31) circle (  1.16);

\path[draw=drawColor,line width= 0.4pt,line join=round,line cap=round,fill=fillColor] (210.18,508.31) circle (  1.16);

\path[draw=drawColor,line width= 0.4pt,line join=round,line cap=round,fill=fillColor] (210.23,508.31) circle (  1.16);

\path[draw=drawColor,line width= 0.4pt,line join=round,line cap=round,fill=fillColor] (210.28,508.31) circle (  1.16);

\path[draw=drawColor,line width= 0.4pt,line join=round,line cap=round,fill=fillColor] (210.33,508.31) circle (  1.16);

\path[draw=drawColor,line width= 0.4pt,line join=round,line cap=round,fill=fillColor] (210.38,508.31) circle (  1.16);

\path[draw=drawColor,line width= 0.4pt,line join=round,line cap=round,fill=fillColor] (210.42,508.31) circle (  1.16);

\path[draw=drawColor,line width= 0.4pt,line join=round,line cap=round,fill=fillColor] (210.47,508.31) circle (  1.16);

\path[draw=drawColor,line width= 0.4pt,line join=round,line cap=round,fill=fillColor] (210.52,508.31) circle (  1.16);

\path[draw=drawColor,line width= 0.4pt,line join=round,line cap=round,fill=fillColor] (210.57,508.31) circle (  1.16);

\path[draw=drawColor,line width= 0.4pt,line join=round,line cap=round,fill=fillColor] (210.61,508.31) circle (  1.16);

\path[draw=drawColor,line width= 0.4pt,line join=round,line cap=round,fill=fillColor] (210.66,508.31) circle (  1.16);

\path[draw=drawColor,line width= 0.4pt,line join=round,line cap=round,fill=fillColor] (210.71,508.31) circle (  1.16);

\path[draw=drawColor,line width= 0.4pt,line join=round,line cap=round,fill=fillColor] (210.76,508.31) circle (  1.16);

\path[draw=drawColor,line width= 0.4pt,line join=round,line cap=round,fill=fillColor] (210.80,508.31) circle (  1.16);

\path[draw=drawColor,line width= 0.4pt,line join=round,line cap=round,fill=fillColor] (210.85,508.31) circle (  1.16);

\path[draw=drawColor,line width= 0.4pt,line join=round,line cap=round,fill=fillColor] (210.90,508.31) circle (  1.16);

\path[draw=drawColor,line width= 0.4pt,line join=round,line cap=round,fill=fillColor] (210.95,508.31) circle (  1.16);

\path[draw=drawColor,line width= 0.4pt,line join=round,line cap=round,fill=fillColor] (210.99,508.31) circle (  1.16);

\path[draw=drawColor,line width= 0.4pt,line join=round,line cap=round,fill=fillColor] (211.04,508.31) circle (  1.16);

\path[draw=drawColor,line width= 0.4pt,line join=round,line cap=round,fill=fillColor] (211.09,508.31) circle (  1.16);

\path[draw=drawColor,line width= 0.4pt,line join=round,line cap=round,fill=fillColor] (211.13,508.31) circle (  1.16);

\path[draw=drawColor,line width= 0.4pt,line join=round,line cap=round,fill=fillColor] (211.18,508.31) circle (  1.16);

\path[draw=drawColor,line width= 0.4pt,line join=round,line cap=round,fill=fillColor] (211.23,508.31) circle (  1.16);

\path[draw=drawColor,line width= 0.4pt,line join=round,line cap=round,fill=fillColor] (211.28,508.31) circle (  1.16);

\path[draw=drawColor,line width= 0.4pt,line join=round,line cap=round,fill=fillColor] (211.32,508.31) circle (  1.16);

\path[draw=drawColor,line width= 0.4pt,line join=round,line cap=round,fill=fillColor] (211.37,508.31) circle (  1.16);

\path[draw=drawColor,line width= 0.4pt,line join=round,line cap=round,fill=fillColor] (211.42,508.31) circle (  1.16);

\path[draw=drawColor,line width= 0.4pt,line join=round,line cap=round,fill=fillColor] (211.46,508.31) circle (  1.16);

\path[draw=drawColor,line width= 0.4pt,line join=round,line cap=round,fill=fillColor] (211.51,508.31) circle (  1.16);

\path[draw=drawColor,line width= 0.4pt,line join=round,line cap=round,fill=fillColor] (211.56,508.31) circle (  1.16);

\path[draw=drawColor,line width= 0.4pt,line join=round,line cap=round,fill=fillColor] (211.60,508.31) circle (  1.16);

\path[draw=drawColor,line width= 0.4pt,line join=round,line cap=round,fill=fillColor] (211.65,508.31) circle (  1.16);

\path[draw=drawColor,line width= 0.4pt,line join=round,line cap=round,fill=fillColor] (211.70,508.31) circle (  1.16);

\path[draw=drawColor,line width= 0.4pt,line join=round,line cap=round,fill=fillColor] (211.75,508.31) circle (  1.16);

\path[draw=drawColor,line width= 0.4pt,line join=round,line cap=round,fill=fillColor] (211.79,508.31) circle (  1.16);

\path[draw=drawColor,line width= 0.4pt,line join=round,line cap=round,fill=fillColor] (211.84,508.31) circle (  1.16);

\path[draw=drawColor,line width= 0.4pt,line join=round,line cap=round,fill=fillColor] (211.89,508.31) circle (  1.16);

\path[draw=drawColor,line width= 0.4pt,line join=round,line cap=round,fill=fillColor] (211.93,508.31) circle (  1.16);

\path[draw=drawColor,line width= 0.4pt,line join=round,line cap=round,fill=fillColor] (211.98,508.31) circle (  1.16);

\path[draw=drawColor,line width= 0.4pt,line join=round,line cap=round,fill=fillColor] (212.03,508.31) circle (  1.16);

\path[draw=drawColor,line width= 0.4pt,line join=round,line cap=round,fill=fillColor] (212.07,508.31) circle (  1.16);

\path[draw=drawColor,line width= 0.4pt,line join=round,line cap=round,fill=fillColor] (212.12,508.31) circle (  1.16);

\path[draw=drawColor,line width= 0.4pt,line join=round,line cap=round,fill=fillColor] (212.16,508.31) circle (  1.16);

\path[draw=drawColor,line width= 0.4pt,line join=round,line cap=round,fill=fillColor] (212.21,508.31) circle (  1.16);

\path[draw=drawColor,line width= 0.4pt,line join=round,line cap=round,fill=fillColor] (212.26,508.31) circle (  1.16);

\path[draw=drawColor,line width= 0.4pt,line join=round,line cap=round,fill=fillColor] (212.30,508.31) circle (  1.16);

\path[draw=drawColor,line width= 0.4pt,line join=round,line cap=round,fill=fillColor] (212.35,508.31) circle (  1.16);

\path[draw=drawColor,line width= 0.4pt,line join=round,line cap=round,fill=fillColor] (212.40,508.31) circle (  1.16);

\path[draw=drawColor,line width= 0.4pt,line join=round,line cap=round,fill=fillColor] (212.44,508.31) circle (  1.16);

\path[draw=drawColor,line width= 0.4pt,line join=round,line cap=round,fill=fillColor] (212.49,508.31) circle (  1.16);

\path[draw=drawColor,line width= 0.4pt,line join=round,line cap=round,fill=fillColor] (212.54,508.31) circle (  1.16);

\path[draw=drawColor,line width= 0.4pt,line join=round,line cap=round,fill=fillColor] (212.58,508.31) circle (  1.16);

\path[draw=drawColor,line width= 0.4pt,line join=round,line cap=round,fill=fillColor] (212.63,508.31) circle (  1.16);

\path[draw=drawColor,line width= 0.4pt,line join=round,line cap=round,fill=fillColor] (212.67,508.31) circle (  1.16);

\path[draw=drawColor,line width= 0.4pt,line join=round,line cap=round,fill=fillColor] (212.72,508.31) circle (  1.16);

\path[draw=drawColor,line width= 0.4pt,line join=round,line cap=round,fill=fillColor] (212.77,508.31) circle (  1.16);

\path[draw=drawColor,line width= 0.4pt,line join=round,line cap=round,fill=fillColor] (212.81,508.31) circle (  1.16);

\path[draw=drawColor,line width= 0.4pt,line join=round,line cap=round,fill=fillColor] (212.86,508.31) circle (  1.16);

\path[draw=drawColor,line width= 0.4pt,line join=round,line cap=round,fill=fillColor] (212.91,508.31) circle (  1.16);

\path[draw=drawColor,line width= 0.4pt,line join=round,line cap=round,fill=fillColor] (212.95,508.31) circle (  1.16);

\path[draw=drawColor,line width= 0.4pt,line join=round,line cap=round,fill=fillColor] (213.00,508.31) circle (  1.16);

\path[draw=drawColor,line width= 0.4pt,line join=round,line cap=round,fill=fillColor] (213.04,508.31) circle (  1.16);

\path[draw=drawColor,line width= 0.4pt,line join=round,line cap=round,fill=fillColor] (213.09,508.31) circle (  1.16);

\path[draw=drawColor,line width= 0.4pt,line join=round,line cap=round,fill=fillColor] (213.14,508.31) circle (  1.16);

\path[draw=drawColor,line width= 0.4pt,line join=round,line cap=round,fill=fillColor] (213.18,508.31) circle (  1.16);

\path[draw=drawColor,line width= 0.4pt,line join=round,line cap=round,fill=fillColor] (213.23,508.31) circle (  1.16);

\path[draw=drawColor,line width= 0.4pt,line join=round,line cap=round,fill=fillColor] (213.27,508.31) circle (  1.16);

\path[draw=drawColor,line width= 0.4pt,line join=round,line cap=round,fill=fillColor] (213.32,508.31) circle (  1.16);

\path[draw=drawColor,line width= 0.4pt,line join=round,line cap=round,fill=fillColor] (213.36,508.31) circle (  1.16);

\path[draw=drawColor,line width= 0.4pt,line join=round,line cap=round,fill=fillColor] (213.41,508.31) circle (  1.16);

\path[draw=drawColor,line width= 0.4pt,line join=round,line cap=round,fill=fillColor] (213.46,508.31) circle (  1.16);

\path[draw=drawColor,line width= 0.4pt,line join=round,line cap=round,fill=fillColor] (213.50,508.31) circle (  1.16);

\path[draw=drawColor,line width= 0.4pt,line join=round,line cap=round,fill=fillColor] (213.55,508.31) circle (  1.16);

\path[draw=drawColor,line width= 0.4pt,line join=round,line cap=round,fill=fillColor] (213.59,508.31) circle (  1.16);

\path[draw=drawColor,line width= 0.4pt,line join=round,line cap=round,fill=fillColor] (213.64,508.31) circle (  1.16);

\path[draw=drawColor,line width= 0.4pt,line join=round,line cap=round,fill=fillColor] (213.68,508.31) circle (  1.16);

\path[draw=drawColor,line width= 0.4pt,line join=round,line cap=round,fill=fillColor] (213.73,508.31) circle (  1.16);

\path[draw=drawColor,line width= 0.4pt,line join=round,line cap=round,fill=fillColor] (213.78,508.31) circle (  1.16);

\path[draw=drawColor,line width= 0.4pt,line join=round,line cap=round,fill=fillColor] (213.82,508.31) circle (  1.16);

\path[draw=drawColor,line width= 0.4pt,line join=round,line cap=round,fill=fillColor] (213.87,508.31) circle (  1.16);

\path[draw=drawColor,line width= 0.4pt,line join=round,line cap=round,fill=fillColor] (213.91,508.31) circle (  1.16);

\path[draw=drawColor,line width= 0.4pt,line join=round,line cap=round,fill=fillColor] (213.96,508.31) circle (  1.16);

\path[draw=drawColor,line width= 0.4pt,line join=round,line cap=round,fill=fillColor] (214.00,508.31) circle (  1.16);

\path[draw=drawColor,line width= 0.4pt,line join=round,line cap=round,fill=fillColor] (214.05,508.31) circle (  1.16);

\path[draw=drawColor,line width= 0.4pt,line join=round,line cap=round,fill=fillColor] (214.09,508.31) circle (  1.16);

\path[draw=drawColor,line width= 0.4pt,line join=round,line cap=round,fill=fillColor] (214.14,508.31) circle (  1.16);

\path[draw=drawColor,line width= 0.4pt,line join=round,line cap=round,fill=fillColor] (214.18,508.31) circle (  1.16);

\path[draw=drawColor,line width= 0.4pt,line join=round,line cap=round,fill=fillColor] (214.23,508.31) circle (  1.16);

\path[draw=drawColor,line width= 0.4pt,line join=round,line cap=round,fill=fillColor] (214.27,508.31) circle (  1.16);

\path[draw=drawColor,line width= 0.4pt,line join=round,line cap=round,fill=fillColor] (214.32,508.31) circle (  1.16);

\path[draw=drawColor,line width= 0.4pt,line join=round,line cap=round,fill=fillColor] (214.36,508.31) circle (  1.16);

\path[draw=drawColor,line width= 0.4pt,line join=round,line cap=round,fill=fillColor] (214.41,508.31) circle (  1.16);

\path[draw=drawColor,line width= 0.4pt,line join=round,line cap=round,fill=fillColor] (214.45,508.31) circle (  1.16);

\path[draw=drawColor,line width= 0.4pt,line join=round,line cap=round,fill=fillColor] (214.50,508.31) circle (  1.16);

\path[draw=drawColor,line width= 0.4pt,line join=round,line cap=round,fill=fillColor] (214.54,508.31) circle (  1.16);

\path[draw=drawColor,line width= 0.4pt,line join=round,line cap=round,fill=fillColor] (214.59,508.31) circle (  1.16);

\path[draw=drawColor,line width= 0.4pt,line join=round,line cap=round,fill=fillColor] (214.63,508.31) circle (  1.16);

\path[draw=drawColor,line width= 0.4pt,line join=round,line cap=round,fill=fillColor] (214.68,508.31) circle (  1.16);

\path[draw=drawColor,line width= 0.4pt,line join=round,line cap=round,fill=fillColor] (214.72,508.31) circle (  1.16);

\path[draw=drawColor,line width= 0.4pt,line join=round,line cap=round,fill=fillColor] (214.77,508.31) circle (  1.16);

\path[draw=drawColor,line width= 0.4pt,line join=round,line cap=round,fill=fillColor] (214.81,508.31) circle (  1.16);

\path[draw=drawColor,line width= 0.4pt,line join=round,line cap=round,fill=fillColor] (214.86,508.31) circle (  1.16);

\path[draw=drawColor,line width= 0.4pt,line join=round,line cap=round,fill=fillColor] (214.90,508.31) circle (  1.16);

\path[draw=drawColor,line width= 0.4pt,line join=round,line cap=round,fill=fillColor] (214.95,508.31) circle (  1.16);

\path[draw=drawColor,line width= 0.4pt,line join=round,line cap=round,fill=fillColor] (214.99,508.31) circle (  1.16);

\path[draw=drawColor,line width= 0.4pt,line join=round,line cap=round,fill=fillColor] (215.04,508.31) circle (  1.16);

\path[draw=drawColor,line width= 0.4pt,line join=round,line cap=round,fill=fillColor] (215.08,508.31) circle (  1.16);

\path[draw=drawColor,line width= 0.4pt,line join=round,line cap=round,fill=fillColor] (215.13,508.31) circle (  1.16);

\path[draw=drawColor,line width= 0.4pt,line join=round,line cap=round,fill=fillColor] (215.17,508.31) circle (  1.16);

\path[draw=drawColor,line width= 0.4pt,line join=round,line cap=round,fill=fillColor] (215.22,508.31) circle (  1.16);

\path[draw=drawColor,line width= 0.4pt,line join=round,line cap=round,fill=fillColor] (215.26,508.31) circle (  1.16);

\path[draw=drawColor,line width= 0.4pt,line join=round,line cap=round,fill=fillColor] (215.31,508.31) circle (  1.16);

\path[draw=drawColor,line width= 0.4pt,line join=round,line cap=round,fill=fillColor] (215.35,508.31) circle (  1.16);

\path[draw=drawColor,line width= 0.4pt,line join=round,line cap=round,fill=fillColor] (215.40,508.31) circle (  1.16);

\path[draw=drawColor,line width= 0.4pt,line join=round,line cap=round,fill=fillColor] (215.44,508.31) circle (  1.16);

\path[draw=drawColor,line width= 0.4pt,line join=round,line cap=round,fill=fillColor] (215.48,508.31) circle (  1.16);

\path[draw=drawColor,line width= 0.4pt,line join=round,line cap=round,fill=fillColor] (215.53,508.31) circle (  1.16);

\path[draw=drawColor,line width= 0.4pt,line join=round,line cap=round,fill=fillColor] (215.57,508.31) circle (  1.16);

\path[draw=drawColor,line width= 0.4pt,line join=round,line cap=round,fill=fillColor] (215.62,508.31) circle (  1.16);

\path[draw=drawColor,line width= 0.4pt,line join=round,line cap=round,fill=fillColor] (215.66,508.31) circle (  1.16);

\path[draw=drawColor,line width= 0.4pt,line join=round,line cap=round,fill=fillColor] (215.71,508.31) circle (  1.16);

\path[draw=drawColor,line width= 0.4pt,line join=round,line cap=round,fill=fillColor] (215.75,508.31) circle (  1.16);

\path[draw=drawColor,line width= 0.4pt,line join=round,line cap=round,fill=fillColor] (215.79,508.31) circle (  1.16);

\path[draw=drawColor,line width= 0.4pt,line join=round,line cap=round,fill=fillColor] (215.84,508.31) circle (  1.16);

\path[draw=drawColor,line width= 0.4pt,line join=round,line cap=round,fill=fillColor] (215.88,508.31) circle (  1.16);

\path[draw=drawColor,line width= 0.4pt,line join=round,line cap=round,fill=fillColor] (215.93,508.31) circle (  1.16);

\path[draw=drawColor,line width= 0.4pt,line join=round,line cap=round,fill=fillColor] (215.97,508.31) circle (  1.16);

\path[draw=drawColor,line width= 0.4pt,line join=round,line cap=round,fill=fillColor] (216.02,508.31) circle (  1.16);

\path[draw=drawColor,line width= 0.4pt,line join=round,line cap=round,fill=fillColor] (216.06,508.31) circle (  1.16);

\path[draw=drawColor,line width= 0.4pt,line join=round,line cap=round,fill=fillColor] (216.10,508.31) circle (  1.16);

\path[draw=drawColor,line width= 0.4pt,line join=round,line cap=round,fill=fillColor] (216.15,508.31) circle (  1.16);

\path[draw=drawColor,line width= 0.4pt,line join=round,line cap=round,fill=fillColor] (216.19,508.31) circle (  1.16);

\path[draw=drawColor,line width= 0.4pt,line join=round,line cap=round,fill=fillColor] (216.24,508.31) circle (  1.16);

\path[draw=drawColor,line width= 0.4pt,line join=round,line cap=round,fill=fillColor] (216.28,508.31) circle (  1.16);

\path[draw=drawColor,line width= 0.4pt,line join=round,line cap=round,fill=fillColor] (216.32,508.31) circle (  1.16);

\path[draw=drawColor,line width= 0.4pt,line join=round,line cap=round,fill=fillColor] (216.37,508.31) circle (  1.16);

\path[draw=drawColor,line width= 0.4pt,line join=round,line cap=round,fill=fillColor] (216.41,508.31) circle (  1.16);

\path[draw=drawColor,line width= 0.4pt,line join=round,line cap=round,fill=fillColor] (216.46,508.31) circle (  1.16);

\path[draw=drawColor,line width= 0.4pt,line join=round,line cap=round,fill=fillColor] (216.50,508.31) circle (  1.16);

\path[draw=drawColor,line width= 0.4pt,line join=round,line cap=round,fill=fillColor] (216.54,508.31) circle (  1.16);

\path[draw=drawColor,line width= 0.4pt,line join=round,line cap=round,fill=fillColor] (216.59,508.31) circle (  1.16);

\path[draw=drawColor,line width= 0.4pt,line join=round,line cap=round,fill=fillColor] (216.63,508.31) circle (  1.16);

\path[draw=drawColor,line width= 0.4pt,line join=round,line cap=round,fill=fillColor] (216.68,508.31) circle (  1.16);

\path[draw=drawColor,line width= 0.4pt,line join=round,line cap=round,fill=fillColor] (216.72,508.31) circle (  1.16);

\path[draw=drawColor,line width= 0.4pt,line join=round,line cap=round,fill=fillColor] (216.76,508.31) circle (  1.16);

\path[draw=drawColor,line width= 0.4pt,line join=round,line cap=round,fill=fillColor] (216.81,508.31) circle (  1.16);

\path[draw=drawColor,line width= 0.4pt,line join=round,line cap=round,fill=fillColor] (216.85,508.31) circle (  1.16);
\definecolor[named]{drawColor}{rgb}{0.60,0.31,0.64}
\definecolor[named]{fillColor}{rgb}{0.60,0.31,0.64}

\path[draw=drawColor,line width= 0.4pt,line join=round,line cap=round,fill=fillColor] ( 81.22,563.62) circle (  1.16);

\path[draw=drawColor,line width= 0.4pt,line join=round,line cap=round,fill=fillColor] ( 84.95,546.41) circle (  1.16);

\path[draw=drawColor,line width= 0.4pt,line join=round,line cap=round,fill=fillColor] ( 87.56,540.06) circle (  1.16);

\path[draw=drawColor,line width= 0.4pt,line join=round,line cap=round,fill=fillColor] ( 89.64,534.28) circle (  1.16);

\path[draw=drawColor,line width= 0.4pt,line join=round,line cap=round,fill=fillColor] ( 91.40,533.18) circle (  1.16);

\path[draw=drawColor,line width= 0.4pt,line join=round,line cap=round,fill=fillColor] ( 92.94,533.02) circle (  1.16);

\path[draw=drawColor,line width= 0.4pt,line join=round,line cap=round,fill=fillColor] ( 94.31,533.00) circle (  1.16);

\path[draw=drawColor,line width= 0.4pt,line join=round,line cap=round,fill=fillColor] ( 95.56,532.69) circle (  1.16);

\path[draw=drawColor,line width= 0.4pt,line join=round,line cap=round,fill=fillColor] ( 96.71,531.86) circle (  1.16);

\path[draw=drawColor,line width= 0.4pt,line join=round,line cap=round,fill=fillColor] ( 97.77,530.78) circle (  1.16);

\path[draw=drawColor,line width= 0.4pt,line join=round,line cap=round,fill=fillColor] ( 98.77,530.71) circle (  1.16);

\path[draw=drawColor,line width= 0.4pt,line join=round,line cap=round,fill=fillColor] ( 99.71,530.60) circle (  1.16);

\path[draw=drawColor,line width= 0.4pt,line join=round,line cap=round,fill=fillColor] (100.59,530.46) circle (  1.16);

\path[draw=drawColor,line width= 0.4pt,line join=round,line cap=round,fill=fillColor] (101.44,530.44) circle (  1.16);

\path[draw=drawColor,line width= 0.4pt,line join=round,line cap=round,fill=fillColor] (102.24,530.35) circle (  1.16);

\path[draw=drawColor,line width= 0.4pt,line join=round,line cap=round,fill=fillColor] (103.01,530.16) circle (  1.16);

\path[draw=drawColor,line width= 0.4pt,line join=round,line cap=round,fill=fillColor] (103.75,529.99) circle (  1.16);

\path[draw=drawColor,line width= 0.4pt,line join=round,line cap=round,fill=fillColor] (104.46,529.86) circle (  1.16);

\path[draw=drawColor,line width= 0.4pt,line join=round,line cap=round,fill=fillColor] (105.14,529.68) circle (  1.16);

\path[draw=drawColor,line width= 0.4pt,line join=round,line cap=round,fill=fillColor] (105.80,529.67) circle (  1.16);

\path[draw=drawColor,line width= 0.4pt,line join=round,line cap=round,fill=fillColor] (106.44,529.51) circle (  1.16);

\path[draw=drawColor,line width= 0.4pt,line join=round,line cap=round,fill=fillColor] (107.05,529.47) circle (  1.16);

\path[draw=drawColor,line width= 0.4pt,line join=round,line cap=round,fill=fillColor] (107.65,529.41) circle (  1.16);

\path[draw=drawColor,line width= 0.4pt,line join=round,line cap=round,fill=fillColor] (108.24,529.37) circle (  1.16);

\path[draw=drawColor,line width= 0.4pt,line join=round,line cap=round,fill=fillColor] (108.80,529.37) circle (  1.16);

\path[draw=drawColor,line width= 0.4pt,line join=round,line cap=round,fill=fillColor] (109.35,529.25) circle (  1.16);

\path[draw=drawColor,line width= 0.4pt,line join=round,line cap=round,fill=fillColor] (109.89,528.91) circle (  1.16);

\path[draw=drawColor,line width= 0.4pt,line join=round,line cap=round,fill=fillColor] (110.42,528.68) circle (  1.16);

\path[draw=drawColor,line width= 0.4pt,line join=round,line cap=round,fill=fillColor] (110.93,528.61) circle (  1.16);

\path[draw=drawColor,line width= 0.4pt,line join=round,line cap=round,fill=fillColor] (111.43,528.43) circle (  1.16);

\path[draw=drawColor,line width= 0.4pt,line join=round,line cap=round,fill=fillColor] (111.92,528.40) circle (  1.16);

\path[draw=drawColor,line width= 0.4pt,line join=round,line cap=round,fill=fillColor] (112.40,528.37) circle (  1.16);

\path[draw=drawColor,line width= 0.4pt,line join=round,line cap=round,fill=fillColor] (112.87,528.33) circle (  1.16);

\path[draw=drawColor,line width= 0.4pt,line join=round,line cap=round,fill=fillColor] (113.33,528.32) circle (  1.16);

\path[draw=drawColor,line width= 0.4pt,line join=round,line cap=round,fill=fillColor] (113.78,528.23) circle (  1.16);

\path[draw=drawColor,line width= 0.4pt,line join=round,line cap=round,fill=fillColor] (114.22,528.11) circle (  1.16);

\path[draw=drawColor,line width= 0.4pt,line join=round,line cap=round,fill=fillColor] (114.65,527.81) circle (  1.16);

\path[draw=drawColor,line width= 0.4pt,line join=round,line cap=round,fill=fillColor] (115.08,527.60) circle (  1.16);

\path[draw=drawColor,line width= 0.4pt,line join=round,line cap=round,fill=fillColor] (115.50,527.60) circle (  1.16);

\path[draw=drawColor,line width= 0.4pt,line join=round,line cap=round,fill=fillColor] (115.91,527.48) circle (  1.16);

\path[draw=drawColor,line width= 0.4pt,line join=round,line cap=round,fill=fillColor] (116.32,527.44) circle (  1.16);

\path[draw=drawColor,line width= 0.4pt,line join=round,line cap=round,fill=fillColor] (116.72,527.40) circle (  1.16);

\path[draw=drawColor,line width= 0.4pt,line join=round,line cap=round,fill=fillColor] (117.11,527.34) circle (  1.16);

\path[draw=drawColor,line width= 0.4pt,line join=round,line cap=round,fill=fillColor] (117.49,527.24) circle (  1.16);

\path[draw=drawColor,line width= 0.4pt,line join=round,line cap=round,fill=fillColor] (117.87,527.24) circle (  1.16);

\path[draw=drawColor,line width= 0.4pt,line join=round,line cap=round,fill=fillColor] (118.25,527.23) circle (  1.16);

\path[draw=drawColor,line width= 0.4pt,line join=round,line cap=round,fill=fillColor] (118.62,527.18) circle (  1.16);

\path[draw=drawColor,line width= 0.4pt,line join=round,line cap=round,fill=fillColor] (118.98,527.13) circle (  1.16);

\path[draw=drawColor,line width= 0.4pt,line join=round,line cap=round,fill=fillColor] (119.34,526.90) circle (  1.16);

\path[draw=drawColor,line width= 0.4pt,line join=round,line cap=round,fill=fillColor] (119.70,526.88) circle (  1.16);

\path[draw=drawColor,line width= 0.4pt,line join=round,line cap=round,fill=fillColor] (120.05,526.85) circle (  1.16);

\path[draw=drawColor,line width= 0.4pt,line join=round,line cap=round,fill=fillColor] (120.39,526.83) circle (  1.16);

\path[draw=drawColor,line width= 0.4pt,line join=round,line cap=round,fill=fillColor] (120.73,526.80) circle (  1.16);

\path[draw=drawColor,line width= 0.4pt,line join=round,line cap=round,fill=fillColor] (121.07,526.78) circle (  1.16);

\path[draw=drawColor,line width= 0.4pt,line join=round,line cap=round,fill=fillColor] (121.40,526.76) circle (  1.16);

\path[draw=drawColor,line width= 0.4pt,line join=round,line cap=round,fill=fillColor] (121.73,526.74) circle (  1.16);

\path[draw=drawColor,line width= 0.4pt,line join=round,line cap=round,fill=fillColor] (122.05,526.64) circle (  1.16);

\path[draw=drawColor,line width= 0.4pt,line join=round,line cap=round,fill=fillColor] (122.38,526.57) circle (  1.16);

\path[draw=drawColor,line width= 0.4pt,line join=round,line cap=round,fill=fillColor] (122.69,526.49) circle (  1.16);

\path[draw=drawColor,line width= 0.4pt,line join=round,line cap=round,fill=fillColor] (123.01,526.48) circle (  1.16);

\path[draw=drawColor,line width= 0.4pt,line join=round,line cap=round,fill=fillColor] (123.32,526.43) circle (  1.16);

\path[draw=drawColor,line width= 0.4pt,line join=round,line cap=round,fill=fillColor] (123.62,526.35) circle (  1.16);

\path[draw=drawColor,line width= 0.4pt,line join=round,line cap=round,fill=fillColor] (123.93,526.29) circle (  1.16);

\path[draw=drawColor,line width= 0.4pt,line join=round,line cap=round,fill=fillColor] (124.23,526.29) circle (  1.16);

\path[draw=drawColor,line width= 0.4pt,line join=round,line cap=round,fill=fillColor] (124.52,526.27) circle (  1.16);

\path[draw=drawColor,line width= 0.4pt,line join=round,line cap=round,fill=fillColor] (124.82,526.20) circle (  1.16);

\path[draw=drawColor,line width= 0.4pt,line join=round,line cap=round,fill=fillColor] (125.11,526.16) circle (  1.16);

\path[draw=drawColor,line width= 0.4pt,line join=round,line cap=round,fill=fillColor] (125.40,526.10) circle (  1.16);

\path[draw=drawColor,line width= 0.4pt,line join=round,line cap=round,fill=fillColor] (125.68,526.01) circle (  1.16);

\path[draw=drawColor,line width= 0.4pt,line join=round,line cap=round,fill=fillColor] (125.96,525.94) circle (  1.16);

\path[draw=drawColor,line width= 0.4pt,line join=round,line cap=round,fill=fillColor] (126.24,525.90) circle (  1.16);

\path[draw=drawColor,line width= 0.4pt,line join=round,line cap=round,fill=fillColor] (126.52,525.89) circle (  1.16);

\path[draw=drawColor,line width= 0.4pt,line join=round,line cap=round,fill=fillColor] (126.80,525.85) circle (  1.16);

\path[draw=drawColor,line width= 0.4pt,line join=round,line cap=round,fill=fillColor] (127.07,525.85) circle (  1.16);

\path[draw=drawColor,line width= 0.4pt,line join=round,line cap=round,fill=fillColor] (127.34,525.83) circle (  1.16);

\path[draw=drawColor,line width= 0.4pt,line join=round,line cap=round,fill=fillColor] (127.61,525.83) circle (  1.16);

\path[draw=drawColor,line width= 0.4pt,line join=round,line cap=round,fill=fillColor] (127.87,525.82) circle (  1.16);

\path[draw=drawColor,line width= 0.4pt,line join=round,line cap=round,fill=fillColor] (128.13,525.77) circle (  1.16);

\path[draw=drawColor,line width= 0.4pt,line join=round,line cap=round,fill=fillColor] (128.40,525.77) circle (  1.16);

\path[draw=drawColor,line width= 0.4pt,line join=round,line cap=round,fill=fillColor] (128.65,525.70) circle (  1.16);

\path[draw=drawColor,line width= 0.4pt,line join=round,line cap=round,fill=fillColor] (128.91,525.68) circle (  1.16);

\path[draw=drawColor,line width= 0.4pt,line join=round,line cap=round,fill=fillColor] (129.16,525.62) circle (  1.16);

\path[draw=drawColor,line width= 0.4pt,line join=round,line cap=round,fill=fillColor] (129.42,525.57) circle (  1.16);

\path[draw=drawColor,line width= 0.4pt,line join=round,line cap=round,fill=fillColor] (129.67,525.50) circle (  1.16);

\path[draw=drawColor,line width= 0.4pt,line join=round,line cap=round,fill=fillColor] (129.91,525.36) circle (  1.16);

\path[draw=drawColor,line width= 0.4pt,line join=round,line cap=round,fill=fillColor] (130.16,525.36) circle (  1.16);

\path[draw=drawColor,line width= 0.4pt,line join=round,line cap=round,fill=fillColor] (130.41,525.30) circle (  1.16);

\path[draw=drawColor,line width= 0.4pt,line join=round,line cap=round,fill=fillColor] (130.65,525.28) circle (  1.16);

\path[draw=drawColor,line width= 0.4pt,line join=round,line cap=round,fill=fillColor] (130.89,525.26) circle (  1.16);

\path[draw=drawColor,line width= 0.4pt,line join=round,line cap=round,fill=fillColor] (131.13,525.24) circle (  1.16);

\path[draw=drawColor,line width= 0.4pt,line join=round,line cap=round,fill=fillColor] (131.36,525.23) circle (  1.16);

\path[draw=drawColor,line width= 0.4pt,line join=round,line cap=round,fill=fillColor] (131.60,525.15) circle (  1.16);

\path[draw=drawColor,line width= 0.4pt,line join=round,line cap=round,fill=fillColor] (131.83,525.06) circle (  1.16);

\path[draw=drawColor,line width= 0.4pt,line join=round,line cap=round,fill=fillColor] (132.06,524.99) circle (  1.16);

\path[draw=drawColor,line width= 0.4pt,line join=round,line cap=round,fill=fillColor] (132.30,524.97) circle (  1.16);

\path[draw=drawColor,line width= 0.4pt,line join=round,line cap=round,fill=fillColor] (132.52,524.92) circle (  1.16);

\path[draw=drawColor,line width= 0.4pt,line join=round,line cap=round,fill=fillColor] (132.75,524.75) circle (  1.16);

\path[draw=drawColor,line width= 0.4pt,line join=round,line cap=round,fill=fillColor] (132.98,524.73) circle (  1.16);

\path[draw=drawColor,line width= 0.4pt,line join=round,line cap=round,fill=fillColor] (133.20,524.72) circle (  1.16);

\path[draw=drawColor,line width= 0.4pt,line join=round,line cap=round,fill=fillColor] (133.42,524.72) circle (  1.16);

\path[draw=drawColor,line width= 0.4pt,line join=round,line cap=round,fill=fillColor] (133.64,524.69) circle (  1.16);

\path[draw=drawColor,line width= 0.4pt,line join=round,line cap=round,fill=fillColor] (133.86,524.64) circle (  1.16);

\path[draw=drawColor,line width= 0.4pt,line join=round,line cap=round,fill=fillColor] (134.08,524.63) circle (  1.16);

\path[draw=drawColor,line width= 0.4pt,line join=round,line cap=round,fill=fillColor] (134.30,524.62) circle (  1.16);

\path[draw=drawColor,line width= 0.4pt,line join=round,line cap=round,fill=fillColor] (134.51,524.62) circle (  1.16);

\path[draw=drawColor,line width= 0.4pt,line join=round,line cap=round,fill=fillColor] (134.73,524.59) circle (  1.16);

\path[draw=drawColor,line width= 0.4pt,line join=round,line cap=round,fill=fillColor] (134.94,524.51) circle (  1.16);

\path[draw=drawColor,line width= 0.4pt,line join=round,line cap=round,fill=fillColor] (135.15,524.48) circle (  1.16);

\path[draw=drawColor,line width= 0.4pt,line join=round,line cap=round,fill=fillColor] (135.36,524.44) circle (  1.16);

\path[draw=drawColor,line width= 0.4pt,line join=round,line cap=round,fill=fillColor] (135.57,524.39) circle (  1.16);

\path[draw=drawColor,line width= 0.4pt,line join=round,line cap=round,fill=fillColor] (135.78,524.36) circle (  1.16);

\path[draw=drawColor,line width= 0.4pt,line join=round,line cap=round,fill=fillColor] (135.98,524.29) circle (  1.16);

\path[draw=drawColor,line width= 0.4pt,line join=round,line cap=round,fill=fillColor] (136.19,524.20) circle (  1.16);

\path[draw=drawColor,line width= 0.4pt,line join=round,line cap=round,fill=fillColor] (136.39,524.16) circle (  1.16);

\path[draw=drawColor,line width= 0.4pt,line join=round,line cap=round,fill=fillColor] (136.60,524.10) circle (  1.16);

\path[draw=drawColor,line width= 0.4pt,line join=round,line cap=round,fill=fillColor] (136.80,524.01) circle (  1.16);

\path[draw=drawColor,line width= 0.4pt,line join=round,line cap=round,fill=fillColor] (137.00,523.99) circle (  1.16);

\path[draw=drawColor,line width= 0.4pt,line join=round,line cap=round,fill=fillColor] (137.20,523.99) circle (  1.16);

\path[draw=drawColor,line width= 0.4pt,line join=round,line cap=round,fill=fillColor] (137.40,523.96) circle (  1.16);

\path[draw=drawColor,line width= 0.4pt,line join=round,line cap=round,fill=fillColor] (137.59,523.84) circle (  1.16);

\path[draw=drawColor,line width= 0.4pt,line join=round,line cap=round,fill=fillColor] (137.79,523.81) circle (  1.16);

\path[draw=drawColor,line width= 0.4pt,line join=round,line cap=round,fill=fillColor] (137.98,523.79) circle (  1.16);

\path[draw=drawColor,line width= 0.4pt,line join=round,line cap=round,fill=fillColor] (138.18,523.75) circle (  1.16);

\path[draw=drawColor,line width= 0.4pt,line join=round,line cap=round,fill=fillColor] (138.37,523.73) circle (  1.16);

\path[draw=drawColor,line width= 0.4pt,line join=round,line cap=round,fill=fillColor] (138.56,523.68) circle (  1.16);

\path[draw=drawColor,line width= 0.4pt,line join=round,line cap=round,fill=fillColor] (138.75,523.68) circle (  1.16);

\path[draw=drawColor,line width= 0.4pt,line join=round,line cap=round,fill=fillColor] (138.94,523.68) circle (  1.16);

\path[draw=drawColor,line width= 0.4pt,line join=round,line cap=round,fill=fillColor] (139.13,523.67) circle (  1.16);

\path[draw=drawColor,line width= 0.4pt,line join=round,line cap=round,fill=fillColor] (139.32,523.66) circle (  1.16);

\path[draw=drawColor,line width= 0.4pt,line join=round,line cap=round,fill=fillColor] (139.50,523.63) circle (  1.16);

\path[draw=drawColor,line width= 0.4pt,line join=round,line cap=round,fill=fillColor] (139.69,523.63) circle (  1.16);

\path[draw=drawColor,line width= 0.4pt,line join=round,line cap=round,fill=fillColor] (139.87,523.56) circle (  1.16);

\path[draw=drawColor,line width= 0.4pt,line join=round,line cap=round,fill=fillColor] (140.06,523.55) circle (  1.16);

\path[draw=drawColor,line width= 0.4pt,line join=round,line cap=round,fill=fillColor] (140.24,523.55) circle (  1.16);

\path[draw=drawColor,line width= 0.4pt,line join=round,line cap=round,fill=fillColor] (140.42,523.53) circle (  1.16);

\path[draw=drawColor,line width= 0.4pt,line join=round,line cap=round,fill=fillColor] (140.60,523.51) circle (  1.16);

\path[draw=drawColor,line width= 0.4pt,line join=round,line cap=round,fill=fillColor] (140.78,523.48) circle (  1.16);

\path[draw=drawColor,line width= 0.4pt,line join=round,line cap=round,fill=fillColor] (140.96,523.46) circle (  1.16);

\path[draw=drawColor,line width= 0.4pt,line join=round,line cap=round,fill=fillColor] (141.14,523.46) circle (  1.16);

\path[draw=drawColor,line width= 0.4pt,line join=round,line cap=round,fill=fillColor] (141.32,523.45) circle (  1.16);

\path[draw=drawColor,line width= 0.4pt,line join=round,line cap=round,fill=fillColor] (141.50,523.42) circle (  1.16);

\path[draw=drawColor,line width= 0.4pt,line join=round,line cap=round,fill=fillColor] (141.67,523.41) circle (  1.16);

\path[draw=drawColor,line width= 0.4pt,line join=round,line cap=round,fill=fillColor] (141.85,523.37) circle (  1.16);

\path[draw=drawColor,line width= 0.4pt,line join=round,line cap=round,fill=fillColor] (142.02,523.37) circle (  1.16);

\path[draw=drawColor,line width= 0.4pt,line join=round,line cap=round,fill=fillColor] (142.20,523.35) circle (  1.16);

\path[draw=drawColor,line width= 0.4pt,line join=round,line cap=round,fill=fillColor] (142.37,523.33) circle (  1.16);

\path[draw=drawColor,line width= 0.4pt,line join=round,line cap=round,fill=fillColor] (142.54,523.30) circle (  1.16);

\path[draw=drawColor,line width= 0.4pt,line join=round,line cap=round,fill=fillColor] (142.71,523.30) circle (  1.16);

\path[draw=drawColor,line width= 0.4pt,line join=round,line cap=round,fill=fillColor] (142.88,523.29) circle (  1.16);

\path[draw=drawColor,line width= 0.4pt,line join=round,line cap=round,fill=fillColor] (143.05,523.29) circle (  1.16);

\path[draw=drawColor,line width= 0.4pt,line join=round,line cap=round,fill=fillColor] (143.22,523.25) circle (  1.16);

\path[draw=drawColor,line width= 0.4pt,line join=round,line cap=round,fill=fillColor] (143.39,523.21) circle (  1.16);

\path[draw=drawColor,line width= 0.4pt,line join=round,line cap=round,fill=fillColor] (143.56,523.18) circle (  1.16);

\path[draw=drawColor,line width= 0.4pt,line join=round,line cap=round,fill=fillColor] (143.72,523.14) circle (  1.16);

\path[draw=drawColor,line width= 0.4pt,line join=round,line cap=round,fill=fillColor] (143.89,523.04) circle (  1.16);

\path[draw=drawColor,line width= 0.4pt,line join=round,line cap=round,fill=fillColor] (144.05,522.99) circle (  1.16);

\path[draw=drawColor,line width= 0.4pt,line join=round,line cap=round,fill=fillColor] (144.22,522.96) circle (  1.16);

\path[draw=drawColor,line width= 0.4pt,line join=round,line cap=round,fill=fillColor] (144.38,522.95) circle (  1.16);

\path[draw=drawColor,line width= 0.4pt,line join=round,line cap=round,fill=fillColor] (144.55,522.95) circle (  1.16);

\path[draw=drawColor,line width= 0.4pt,line join=round,line cap=round,fill=fillColor] (144.71,522.94) circle (  1.16);

\path[draw=drawColor,line width= 0.4pt,line join=round,line cap=round,fill=fillColor] (144.87,522.91) circle (  1.16);

\path[draw=drawColor,line width= 0.4pt,line join=round,line cap=round,fill=fillColor] (145.03,522.89) circle (  1.16);

\path[draw=drawColor,line width= 0.4pt,line join=round,line cap=round,fill=fillColor] (145.19,522.88) circle (  1.16);

\path[draw=drawColor,line width= 0.4pt,line join=round,line cap=round,fill=fillColor] (145.35,522.88) circle (  1.16);

\path[draw=drawColor,line width= 0.4pt,line join=round,line cap=round,fill=fillColor] (145.51,522.87) circle (  1.16);

\path[draw=drawColor,line width= 0.4pt,line join=round,line cap=round,fill=fillColor] (145.67,522.85) circle (  1.16);

\path[draw=drawColor,line width= 0.4pt,line join=round,line cap=round,fill=fillColor] (145.83,522.84) circle (  1.16);

\path[draw=drawColor,line width= 0.4pt,line join=round,line cap=round,fill=fillColor] (145.98,522.84) circle (  1.16);

\path[draw=drawColor,line width= 0.4pt,line join=round,line cap=round,fill=fillColor] (146.14,522.77) circle (  1.16);

\path[draw=drawColor,line width= 0.4pt,line join=round,line cap=round,fill=fillColor] (146.30,522.75) circle (  1.16);

\path[draw=drawColor,line width= 0.4pt,line join=round,line cap=round,fill=fillColor] (146.45,522.68) circle (  1.16);

\path[draw=drawColor,line width= 0.4pt,line join=round,line cap=round,fill=fillColor] (146.61,522.68) circle (  1.16);

\path[draw=drawColor,line width= 0.4pt,line join=round,line cap=round,fill=fillColor] (146.76,522.63) circle (  1.16);

\path[draw=drawColor,line width= 0.4pt,line join=round,line cap=round,fill=fillColor] (146.91,522.62) circle (  1.16);

\path[draw=drawColor,line width= 0.4pt,line join=round,line cap=round,fill=fillColor] (147.07,522.62) circle (  1.16);

\path[draw=drawColor,line width= 0.4pt,line join=round,line cap=round,fill=fillColor] (147.22,522.58) circle (  1.16);

\path[draw=drawColor,line width= 0.4pt,line join=round,line cap=round,fill=fillColor] (147.37,522.57) circle (  1.16);

\path[draw=drawColor,line width= 0.4pt,line join=round,line cap=round,fill=fillColor] (147.52,522.56) circle (  1.16);

\path[draw=drawColor,line width= 0.4pt,line join=round,line cap=round,fill=fillColor] (147.67,522.52) circle (  1.16);

\path[draw=drawColor,line width= 0.4pt,line join=round,line cap=round,fill=fillColor] (147.82,522.51) circle (  1.16);

\path[draw=drawColor,line width= 0.4pt,line join=round,line cap=round,fill=fillColor] (147.97,522.49) circle (  1.16);

\path[draw=drawColor,line width= 0.4pt,line join=round,line cap=round,fill=fillColor] (148.12,522.49) circle (  1.16);

\path[draw=drawColor,line width= 0.4pt,line join=round,line cap=round,fill=fillColor] (148.27,522.46) circle (  1.16);

\path[draw=drawColor,line width= 0.4pt,line join=round,line cap=round,fill=fillColor] (148.42,522.45) circle (  1.16);

\path[draw=drawColor,line width= 0.4pt,line join=round,line cap=round,fill=fillColor] (148.57,522.42) circle (  1.16);

\path[draw=drawColor,line width= 0.4pt,line join=round,line cap=round,fill=fillColor] (148.71,522.40) circle (  1.16);

\path[draw=drawColor,line width= 0.4pt,line join=round,line cap=round,fill=fillColor] (148.86,522.40) circle (  1.16);

\path[draw=drawColor,line width= 0.4pt,line join=round,line cap=round,fill=fillColor] (149.01,522.38) circle (  1.16);

\path[draw=drawColor,line width= 0.4pt,line join=round,line cap=round,fill=fillColor] (149.15,522.32) circle (  1.16);

\path[draw=drawColor,line width= 0.4pt,line join=round,line cap=round,fill=fillColor] (149.30,522.31) circle (  1.16);

\path[draw=drawColor,line width= 0.4pt,line join=round,line cap=round,fill=fillColor] (149.44,522.31) circle (  1.16);

\path[draw=drawColor,line width= 0.4pt,line join=round,line cap=round,fill=fillColor] (149.58,522.30) circle (  1.16);

\path[draw=drawColor,line width= 0.4pt,line join=round,line cap=round,fill=fillColor] (149.73,522.30) circle (  1.16);

\path[draw=drawColor,line width= 0.4pt,line join=round,line cap=round,fill=fillColor] (149.87,522.27) circle (  1.16);

\path[draw=drawColor,line width= 0.4pt,line join=round,line cap=round,fill=fillColor] (150.01,522.27) circle (  1.16);

\path[draw=drawColor,line width= 0.4pt,line join=round,line cap=round,fill=fillColor] (150.15,522.20) circle (  1.16);

\path[draw=drawColor,line width= 0.4pt,line join=round,line cap=round,fill=fillColor] (150.30,522.20) circle (  1.16);

\path[draw=drawColor,line width= 0.4pt,line join=round,line cap=round,fill=fillColor] (150.44,522.20) circle (  1.16);

\path[draw=drawColor,line width= 0.4pt,line join=round,line cap=round,fill=fillColor] (150.58,522.16) circle (  1.16);

\path[draw=drawColor,line width= 0.4pt,line join=round,line cap=round,fill=fillColor] (150.72,522.14) circle (  1.16);

\path[draw=drawColor,line width= 0.4pt,line join=round,line cap=round,fill=fillColor] (150.86,522.12) circle (  1.16);

\path[draw=drawColor,line width= 0.4pt,line join=round,line cap=round,fill=fillColor] (151.00,522.11) circle (  1.16);

\path[draw=drawColor,line width= 0.4pt,line join=round,line cap=round,fill=fillColor] (151.13,522.11) circle (  1.16);

\path[draw=drawColor,line width= 0.4pt,line join=round,line cap=round,fill=fillColor] (151.27,522.10) circle (  1.16);

\path[draw=drawColor,line width= 0.4pt,line join=round,line cap=round,fill=fillColor] (151.41,522.10) circle (  1.16);

\path[draw=drawColor,line width= 0.4pt,line join=round,line cap=round,fill=fillColor] (151.55,522.10) circle (  1.16);

\path[draw=drawColor,line width= 0.4pt,line join=round,line cap=round,fill=fillColor] (151.68,522.07) circle (  1.16);

\path[draw=drawColor,line width= 0.4pt,line join=round,line cap=round,fill=fillColor] (151.82,522.06) circle (  1.16);

\path[draw=drawColor,line width= 0.4pt,line join=round,line cap=round,fill=fillColor] (151.96,522.04) circle (  1.16);

\path[draw=drawColor,line width= 0.4pt,line join=round,line cap=round,fill=fillColor] (152.09,522.02) circle (  1.16);

\path[draw=drawColor,line width= 0.4pt,line join=round,line cap=round,fill=fillColor] (152.23,522.01) circle (  1.16);

\path[draw=drawColor,line width= 0.4pt,line join=round,line cap=round,fill=fillColor] (152.36,521.94) circle (  1.16);

\path[draw=drawColor,line width= 0.4pt,line join=round,line cap=round,fill=fillColor] (152.49,521.94) circle (  1.16);

\path[draw=drawColor,line width= 0.4pt,line join=round,line cap=round,fill=fillColor] (152.63,521.90) circle (  1.16);

\path[draw=drawColor,line width= 0.4pt,line join=round,line cap=round,fill=fillColor] (152.76,521.89) circle (  1.16);

\path[draw=drawColor,line width= 0.4pt,line join=round,line cap=round,fill=fillColor] (152.89,521.88) circle (  1.16);

\path[draw=drawColor,line width= 0.4pt,line join=round,line cap=round,fill=fillColor] (153.03,521.87) circle (  1.16);

\path[draw=drawColor,line width= 0.4pt,line join=round,line cap=round,fill=fillColor] (153.16,521.87) circle (  1.16);

\path[draw=drawColor,line width= 0.4pt,line join=round,line cap=round,fill=fillColor] (153.29,521.86) circle (  1.16);

\path[draw=drawColor,line width= 0.4pt,line join=round,line cap=round,fill=fillColor] (153.42,521.85) circle (  1.16);

\path[draw=drawColor,line width= 0.4pt,line join=round,line cap=round,fill=fillColor] (153.55,521.85) circle (  1.16);

\path[draw=drawColor,line width= 0.4pt,line join=round,line cap=round,fill=fillColor] (153.68,521.80) circle (  1.16);

\path[draw=drawColor,line width= 0.4pt,line join=round,line cap=round,fill=fillColor] (153.81,521.79) circle (  1.16);

\path[draw=drawColor,line width= 0.4pt,line join=round,line cap=round,fill=fillColor] (153.94,521.71) circle (  1.16);

\path[draw=drawColor,line width= 0.4pt,line join=round,line cap=round,fill=fillColor] (154.07,521.71) circle (  1.16);

\path[draw=drawColor,line width= 0.4pt,line join=round,line cap=round,fill=fillColor] (154.20,521.69) circle (  1.16);

\path[draw=drawColor,line width= 0.4pt,line join=round,line cap=round,fill=fillColor] (154.33,521.68) circle (  1.16);

\path[draw=drawColor,line width= 0.4pt,line join=round,line cap=round,fill=fillColor] (154.46,521.68) circle (  1.16);

\path[draw=drawColor,line width= 0.4pt,line join=round,line cap=round,fill=fillColor] (154.59,521.66) circle (  1.16);

\path[draw=drawColor,line width= 0.4pt,line join=round,line cap=round,fill=fillColor] (154.71,521.58) circle (  1.16);

\path[draw=drawColor,line width= 0.4pt,line join=round,line cap=round,fill=fillColor] (154.84,521.57) circle (  1.16);

\path[draw=drawColor,line width= 0.4pt,line join=round,line cap=round,fill=fillColor] (154.97,521.55) circle (  1.16);

\path[draw=drawColor,line width= 0.4pt,line join=round,line cap=round,fill=fillColor] (155.09,521.54) circle (  1.16);

\path[draw=drawColor,line width= 0.4pt,line join=round,line cap=round,fill=fillColor] (155.22,521.51) circle (  1.16);

\path[draw=drawColor,line width= 0.4pt,line join=round,line cap=round,fill=fillColor] (155.35,521.46) circle (  1.16);

\path[draw=drawColor,line width= 0.4pt,line join=round,line cap=round,fill=fillColor] (155.47,521.42) circle (  1.16);

\path[draw=drawColor,line width= 0.4pt,line join=round,line cap=round,fill=fillColor] (155.60,521.41) circle (  1.16);

\path[draw=drawColor,line width= 0.4pt,line join=round,line cap=round,fill=fillColor] (155.72,521.39) circle (  1.16);

\path[draw=drawColor,line width= 0.4pt,line join=round,line cap=round,fill=fillColor] (155.85,521.39) circle (  1.16);

\path[draw=drawColor,line width= 0.4pt,line join=round,line cap=round,fill=fillColor] (155.97,521.37) circle (  1.16);

\path[draw=drawColor,line width= 0.4pt,line join=round,line cap=round,fill=fillColor] (156.09,521.36) circle (  1.16);

\path[draw=drawColor,line width= 0.4pt,line join=round,line cap=round,fill=fillColor] (156.22,521.36) circle (  1.16);

\path[draw=drawColor,line width= 0.4pt,line join=round,line cap=round,fill=fillColor] (156.34,521.31) circle (  1.16);

\path[draw=drawColor,line width= 0.4pt,line join=round,line cap=round,fill=fillColor] (156.46,521.31) circle (  1.16);

\path[draw=drawColor,line width= 0.4pt,line join=round,line cap=round,fill=fillColor] (156.58,521.30) circle (  1.16);

\path[draw=drawColor,line width= 0.4pt,line join=round,line cap=round,fill=fillColor] (156.71,521.30) circle (  1.16);

\path[draw=drawColor,line width= 0.4pt,line join=round,line cap=round,fill=fillColor] (156.83,521.28) circle (  1.16);

\path[draw=drawColor,line width= 0.4pt,line join=round,line cap=round,fill=fillColor] (156.95,521.21) circle (  1.16);

\path[draw=drawColor,line width= 0.4pt,line join=round,line cap=round,fill=fillColor] (157.07,521.20) circle (  1.16);

\path[draw=drawColor,line width= 0.4pt,line join=round,line cap=round,fill=fillColor] (157.19,521.13) circle (  1.16);

\path[draw=drawColor,line width= 0.4pt,line join=round,line cap=round,fill=fillColor] (157.31,521.11) circle (  1.16);

\path[draw=drawColor,line width= 0.4pt,line join=round,line cap=round,fill=fillColor] (157.43,521.11) circle (  1.16);

\path[draw=drawColor,line width= 0.4pt,line join=round,line cap=round,fill=fillColor] (157.55,521.10) circle (  1.16);

\path[draw=drawColor,line width= 0.4pt,line join=round,line cap=round,fill=fillColor] (157.67,521.07) circle (  1.16);

\path[draw=drawColor,line width= 0.4pt,line join=round,line cap=round,fill=fillColor] (157.79,521.04) circle (  1.16);

\path[draw=drawColor,line width= 0.4pt,line join=round,line cap=round,fill=fillColor] (157.91,521.03) circle (  1.16);

\path[draw=drawColor,line width= 0.4pt,line join=round,line cap=round,fill=fillColor] (158.02,521.02) circle (  1.16);

\path[draw=drawColor,line width= 0.4pt,line join=round,line cap=round,fill=fillColor] (158.14,520.95) circle (  1.16);

\path[draw=drawColor,line width= 0.4pt,line join=round,line cap=round,fill=fillColor] (158.26,520.92) circle (  1.16);

\path[draw=drawColor,line width= 0.4pt,line join=round,line cap=round,fill=fillColor] (158.38,520.92) circle (  1.16);

\path[draw=drawColor,line width= 0.4pt,line join=round,line cap=round,fill=fillColor] (158.50,520.92) circle (  1.16);

\path[draw=drawColor,line width= 0.4pt,line join=round,line cap=round,fill=fillColor] (158.61,520.92) circle (  1.16);

\path[draw=drawColor,line width= 0.4pt,line join=round,line cap=round,fill=fillColor] (158.73,520.87) circle (  1.16);

\path[draw=drawColor,line width= 0.4pt,line join=round,line cap=round,fill=fillColor] (158.84,520.87) circle (  1.16);

\path[draw=drawColor,line width= 0.4pt,line join=round,line cap=round,fill=fillColor] (158.96,520.84) circle (  1.16);

\path[draw=drawColor,line width= 0.4pt,line join=round,line cap=round,fill=fillColor] (159.08,520.83) circle (  1.16);

\path[draw=drawColor,line width= 0.4pt,line join=round,line cap=round,fill=fillColor] (159.19,520.83) circle (  1.16);

\path[draw=drawColor,line width= 0.4pt,line join=round,line cap=round,fill=fillColor] (159.31,520.83) circle (  1.16);

\path[draw=drawColor,line width= 0.4pt,line join=round,line cap=round,fill=fillColor] (159.42,520.81) circle (  1.16);

\path[draw=drawColor,line width= 0.4pt,line join=round,line cap=round,fill=fillColor] (159.54,520.81) circle (  1.16);

\path[draw=drawColor,line width= 0.4pt,line join=round,line cap=round,fill=fillColor] (159.65,520.75) circle (  1.16);

\path[draw=drawColor,line width= 0.4pt,line join=round,line cap=round,fill=fillColor] (159.76,520.74) circle (  1.16);

\path[draw=drawColor,line width= 0.4pt,line join=round,line cap=round,fill=fillColor] (159.88,520.72) circle (  1.16);

\path[draw=drawColor,line width= 0.4pt,line join=round,line cap=round,fill=fillColor] (159.99,520.69) circle (  1.16);

\path[draw=drawColor,line width= 0.4pt,line join=round,line cap=round,fill=fillColor] (160.10,520.66) circle (  1.16);

\path[draw=drawColor,line width= 0.4pt,line join=round,line cap=round,fill=fillColor] (160.22,520.65) circle (  1.16);

\path[draw=drawColor,line width= 0.4pt,line join=round,line cap=round,fill=fillColor] (160.33,520.64) circle (  1.16);

\path[draw=drawColor,line width= 0.4pt,line join=round,line cap=round,fill=fillColor] (160.44,520.61) circle (  1.16);

\path[draw=drawColor,line width= 0.4pt,line join=round,line cap=round,fill=fillColor] (160.55,520.61) circle (  1.16);

\path[draw=drawColor,line width= 0.4pt,line join=round,line cap=round,fill=fillColor] (160.67,520.59) circle (  1.16);

\path[draw=drawColor,line width= 0.4pt,line join=round,line cap=round,fill=fillColor] (160.78,520.59) circle (  1.16);

\path[draw=drawColor,line width= 0.4pt,line join=round,line cap=round,fill=fillColor] (160.89,520.58) circle (  1.16);

\path[draw=drawColor,line width= 0.4pt,line join=round,line cap=round,fill=fillColor] (161.00,520.57) circle (  1.16);

\path[draw=drawColor,line width= 0.4pt,line join=round,line cap=round,fill=fillColor] (161.11,520.54) circle (  1.16);

\path[draw=drawColor,line width= 0.4pt,line join=round,line cap=round,fill=fillColor] (161.22,520.49) circle (  1.16);

\path[draw=drawColor,line width= 0.4pt,line join=round,line cap=round,fill=fillColor] (161.33,520.49) circle (  1.16);

\path[draw=drawColor,line width= 0.4pt,line join=round,line cap=round,fill=fillColor] (161.44,520.47) circle (  1.16);

\path[draw=drawColor,line width= 0.4pt,line join=round,line cap=round,fill=fillColor] (161.55,520.40) circle (  1.16);

\path[draw=drawColor,line width= 0.4pt,line join=round,line cap=round,fill=fillColor] (161.66,520.39) circle (  1.16);

\path[draw=drawColor,line width= 0.4pt,line join=round,line cap=round,fill=fillColor] (161.77,520.38) circle (  1.16);

\path[draw=drawColor,line width= 0.4pt,line join=round,line cap=round,fill=fillColor] (161.88,520.37) circle (  1.16);

\path[draw=drawColor,line width= 0.4pt,line join=round,line cap=round,fill=fillColor] (161.99,520.36) circle (  1.16);

\path[draw=drawColor,line width= 0.4pt,line join=round,line cap=round,fill=fillColor] (162.10,520.35) circle (  1.16);

\path[draw=drawColor,line width= 0.4pt,line join=round,line cap=round,fill=fillColor] (162.20,520.29) circle (  1.16);

\path[draw=drawColor,line width= 0.4pt,line join=round,line cap=round,fill=fillColor] (162.31,520.29) circle (  1.16);

\path[draw=drawColor,line width= 0.4pt,line join=round,line cap=round,fill=fillColor] (162.42,520.26) circle (  1.16);

\path[draw=drawColor,line width= 0.4pt,line join=round,line cap=round,fill=fillColor] (162.53,520.25) circle (  1.16);

\path[draw=drawColor,line width= 0.4pt,line join=round,line cap=round,fill=fillColor] (162.63,520.24) circle (  1.16);

\path[draw=drawColor,line width= 0.4pt,line join=round,line cap=round,fill=fillColor] (162.74,520.23) circle (  1.16);

\path[draw=drawColor,line width= 0.4pt,line join=round,line cap=round,fill=fillColor] (162.85,520.22) circle (  1.16);

\path[draw=drawColor,line width= 0.4pt,line join=round,line cap=round,fill=fillColor] (162.95,520.21) circle (  1.16);

\path[draw=drawColor,line width= 0.4pt,line join=round,line cap=round,fill=fillColor] (163.06,520.21) circle (  1.16);

\path[draw=drawColor,line width= 0.4pt,line join=round,line cap=round,fill=fillColor] (163.17,520.14) circle (  1.16);

\path[draw=drawColor,line width= 0.4pt,line join=round,line cap=round,fill=fillColor] (163.27,520.13) circle (  1.16);

\path[draw=drawColor,line width= 0.4pt,line join=round,line cap=round,fill=fillColor] (163.38,520.13) circle (  1.16);

\path[draw=drawColor,line width= 0.4pt,line join=round,line cap=round,fill=fillColor] (163.48,520.10) circle (  1.16);

\path[draw=drawColor,line width= 0.4pt,line join=round,line cap=round,fill=fillColor] (163.59,520.10) circle (  1.16);

\path[draw=drawColor,line width= 0.4pt,line join=round,line cap=round,fill=fillColor] (163.69,520.09) circle (  1.16);

\path[draw=drawColor,line width= 0.4pt,line join=round,line cap=round,fill=fillColor] (163.80,520.06) circle (  1.16);

\path[draw=drawColor,line width= 0.4pt,line join=round,line cap=round,fill=fillColor] (163.90,520.05) circle (  1.16);

\path[draw=drawColor,line width= 0.4pt,line join=round,line cap=round,fill=fillColor] (164.01,520.01) circle (  1.16);

\path[draw=drawColor,line width= 0.4pt,line join=round,line cap=round,fill=fillColor] (164.11,519.98) circle (  1.16);

\path[draw=drawColor,line width= 0.4pt,line join=round,line cap=round,fill=fillColor] (164.21,519.96) circle (  1.16);

\path[draw=drawColor,line width= 0.4pt,line join=round,line cap=round,fill=fillColor] (164.32,519.91) circle (  1.16);

\path[draw=drawColor,line width= 0.4pt,line join=round,line cap=round,fill=fillColor] (164.42,519.90) circle (  1.16);

\path[draw=drawColor,line width= 0.4pt,line join=round,line cap=round,fill=fillColor] (164.52,519.90) circle (  1.16);

\path[draw=drawColor,line width= 0.4pt,line join=round,line cap=round,fill=fillColor] (164.63,519.89) circle (  1.16);

\path[draw=drawColor,line width= 0.4pt,line join=round,line cap=round,fill=fillColor] (164.73,519.88) circle (  1.16);

\path[draw=drawColor,line width= 0.4pt,line join=round,line cap=round,fill=fillColor] (164.83,519.87) circle (  1.16);

\path[draw=drawColor,line width= 0.4pt,line join=round,line cap=round,fill=fillColor] (164.93,519.85) circle (  1.16);

\path[draw=drawColor,line width= 0.4pt,line join=round,line cap=round,fill=fillColor] (165.04,519.84) circle (  1.16);

\path[draw=drawColor,line width= 0.4pt,line join=round,line cap=round,fill=fillColor] (165.14,519.83) circle (  1.16);

\path[draw=drawColor,line width= 0.4pt,line join=round,line cap=round,fill=fillColor] (165.24,519.81) circle (  1.16);

\path[draw=drawColor,line width= 0.4pt,line join=round,line cap=round,fill=fillColor] (165.34,519.79) circle (  1.16);

\path[draw=drawColor,line width= 0.4pt,line join=round,line cap=round,fill=fillColor] (165.44,519.78) circle (  1.16);

\path[draw=drawColor,line width= 0.4pt,line join=round,line cap=round,fill=fillColor] (165.54,519.77) circle (  1.16);

\path[draw=drawColor,line width= 0.4pt,line join=round,line cap=round,fill=fillColor] (165.64,519.75) circle (  1.16);

\path[draw=drawColor,line width= 0.4pt,line join=round,line cap=round,fill=fillColor] (165.74,519.74) circle (  1.16);

\path[draw=drawColor,line width= 0.4pt,line join=round,line cap=round,fill=fillColor] (165.85,519.73) circle (  1.16);

\path[draw=drawColor,line width= 0.4pt,line join=round,line cap=round,fill=fillColor] (165.95,519.73) circle (  1.16);

\path[draw=drawColor,line width= 0.4pt,line join=round,line cap=round,fill=fillColor] (166.05,519.70) circle (  1.16);

\path[draw=drawColor,line width= 0.4pt,line join=round,line cap=round,fill=fillColor] (166.14,519.68) circle (  1.16);

\path[draw=drawColor,line width= 0.4pt,line join=round,line cap=round,fill=fillColor] (166.24,519.67) circle (  1.16);

\path[draw=drawColor,line width= 0.4pt,line join=round,line cap=round,fill=fillColor] (166.34,519.66) circle (  1.16);

\path[draw=drawColor,line width= 0.4pt,line join=round,line cap=round,fill=fillColor] (166.44,519.65) circle (  1.16);

\path[draw=drawColor,line width= 0.4pt,line join=round,line cap=round,fill=fillColor] (166.54,519.61) circle (  1.16);

\path[draw=drawColor,line width= 0.4pt,line join=round,line cap=round,fill=fillColor] (166.64,519.61) circle (  1.16);

\path[draw=drawColor,line width= 0.4pt,line join=round,line cap=round,fill=fillColor] (166.74,519.59) circle (  1.16);

\path[draw=drawColor,line width= 0.4pt,line join=round,line cap=round,fill=fillColor] (166.84,519.58) circle (  1.16);

\path[draw=drawColor,line width= 0.4pt,line join=round,line cap=round,fill=fillColor] (166.94,519.56) circle (  1.16);

\path[draw=drawColor,line width= 0.4pt,line join=round,line cap=round,fill=fillColor] (167.03,519.56) circle (  1.16);

\path[draw=drawColor,line width= 0.4pt,line join=round,line cap=round,fill=fillColor] (167.13,519.56) circle (  1.16);

\path[draw=drawColor,line width= 0.4pt,line join=round,line cap=round,fill=fillColor] (167.23,519.54) circle (  1.16);

\path[draw=drawColor,line width= 0.4pt,line join=round,line cap=round,fill=fillColor] (167.33,519.54) circle (  1.16);

\path[draw=drawColor,line width= 0.4pt,line join=round,line cap=round,fill=fillColor] (167.42,519.53) circle (  1.16);

\path[draw=drawColor,line width= 0.4pt,line join=round,line cap=round,fill=fillColor] (167.52,519.50) circle (  1.16);

\path[draw=drawColor,line width= 0.4pt,line join=round,line cap=round,fill=fillColor] (167.62,519.50) circle (  1.16);

\path[draw=drawColor,line width= 0.4pt,line join=round,line cap=round,fill=fillColor] (167.71,519.49) circle (  1.16);

\path[draw=drawColor,line width= 0.4pt,line join=round,line cap=round,fill=fillColor] (167.81,519.49) circle (  1.16);

\path[draw=drawColor,line width= 0.4pt,line join=round,line cap=round,fill=fillColor] (167.91,519.48) circle (  1.16);

\path[draw=drawColor,line width= 0.4pt,line join=round,line cap=round,fill=fillColor] (168.00,519.45) circle (  1.16);

\path[draw=drawColor,line width= 0.4pt,line join=round,line cap=round,fill=fillColor] (168.10,519.44) circle (  1.16);

\path[draw=drawColor,line width= 0.4pt,line join=round,line cap=round,fill=fillColor] (168.20,519.40) circle (  1.16);

\path[draw=drawColor,line width= 0.4pt,line join=round,line cap=round,fill=fillColor] (168.29,519.38) circle (  1.16);

\path[draw=drawColor,line width= 0.4pt,line join=round,line cap=round,fill=fillColor] (168.39,519.37) circle (  1.16);

\path[draw=drawColor,line width= 0.4pt,line join=round,line cap=round,fill=fillColor] (168.48,519.36) circle (  1.16);

\path[draw=drawColor,line width= 0.4pt,line join=round,line cap=round,fill=fillColor] (168.58,519.36) circle (  1.16);

\path[draw=drawColor,line width= 0.4pt,line join=round,line cap=round,fill=fillColor] (168.67,519.28) circle (  1.16);

\path[draw=drawColor,line width= 0.4pt,line join=round,line cap=round,fill=fillColor] (168.77,519.25) circle (  1.16);

\path[draw=drawColor,line width= 0.4pt,line join=round,line cap=round,fill=fillColor] (168.86,519.21) circle (  1.16);

\path[draw=drawColor,line width= 0.4pt,line join=round,line cap=round,fill=fillColor] (168.95,519.18) circle (  1.16);

\path[draw=drawColor,line width= 0.4pt,line join=round,line cap=round,fill=fillColor] (169.05,519.17) circle (  1.16);

\path[draw=drawColor,line width= 0.4pt,line join=round,line cap=round,fill=fillColor] (169.14,519.16) circle (  1.16);

\path[draw=drawColor,line width= 0.4pt,line join=round,line cap=round,fill=fillColor] (169.24,519.15) circle (  1.16);

\path[draw=drawColor,line width= 0.4pt,line join=round,line cap=round,fill=fillColor] (169.33,519.12) circle (  1.16);

\path[draw=drawColor,line width= 0.4pt,line join=round,line cap=round,fill=fillColor] (169.42,519.11) circle (  1.16);

\path[draw=drawColor,line width= 0.4pt,line join=round,line cap=round,fill=fillColor] (169.52,519.10) circle (  1.16);

\path[draw=drawColor,line width= 0.4pt,line join=round,line cap=round,fill=fillColor] (169.61,519.09) circle (  1.16);

\path[draw=drawColor,line width= 0.4pt,line join=round,line cap=round,fill=fillColor] (169.70,519.09) circle (  1.16);

\path[draw=drawColor,line width= 0.4pt,line join=round,line cap=round,fill=fillColor] (169.80,519.04) circle (  1.16);

\path[draw=drawColor,line width= 0.4pt,line join=round,line cap=round,fill=fillColor] (169.89,519.03) circle (  1.16);

\path[draw=drawColor,line width= 0.4pt,line join=round,line cap=round,fill=fillColor] (169.98,519.02) circle (  1.16);

\path[draw=drawColor,line width= 0.4pt,line join=round,line cap=round,fill=fillColor] (170.07,519.01) circle (  1.16);

\path[draw=drawColor,line width= 0.4pt,line join=round,line cap=round,fill=fillColor] (170.17,518.99) circle (  1.16);

\path[draw=drawColor,line width= 0.4pt,line join=round,line cap=round,fill=fillColor] (170.26,518.95) circle (  1.16);

\path[draw=drawColor,line width= 0.4pt,line join=round,line cap=round,fill=fillColor] (170.35,518.94) circle (  1.16);

\path[draw=drawColor,line width= 0.4pt,line join=round,line cap=round,fill=fillColor] (170.44,518.93) circle (  1.16);

\path[draw=drawColor,line width= 0.4pt,line join=round,line cap=round,fill=fillColor] (170.53,518.92) circle (  1.16);

\path[draw=drawColor,line width= 0.4pt,line join=round,line cap=round,fill=fillColor] (170.62,518.92) circle (  1.16);

\path[draw=drawColor,line width= 0.4pt,line join=round,line cap=round,fill=fillColor] (170.71,518.87) circle (  1.16);

\path[draw=drawColor,line width= 0.4pt,line join=round,line cap=round,fill=fillColor] (170.81,518.84) circle (  1.16);

\path[draw=drawColor,line width= 0.4pt,line join=round,line cap=round,fill=fillColor] (170.90,518.83) circle (  1.16);

\path[draw=drawColor,line width= 0.4pt,line join=round,line cap=round,fill=fillColor] (170.99,518.83) circle (  1.16);

\path[draw=drawColor,line width= 0.4pt,line join=round,line cap=round,fill=fillColor] (171.08,518.81) circle (  1.16);

\path[draw=drawColor,line width= 0.4pt,line join=round,line cap=round,fill=fillColor] (171.17,518.81) circle (  1.16);

\path[draw=drawColor,line width= 0.4pt,line join=round,line cap=round,fill=fillColor] (171.26,518.77) circle (  1.16);

\path[draw=drawColor,line width= 0.4pt,line join=round,line cap=round,fill=fillColor] (171.35,518.76) circle (  1.16);

\path[draw=drawColor,line width= 0.4pt,line join=round,line cap=round,fill=fillColor] (171.44,518.74) circle (  1.16);

\path[draw=drawColor,line width= 0.4pt,line join=round,line cap=round,fill=fillColor] (171.53,518.73) circle (  1.16);

\path[draw=drawColor,line width= 0.4pt,line join=round,line cap=round,fill=fillColor] (171.62,518.71) circle (  1.16);

\path[draw=drawColor,line width= 0.4pt,line join=round,line cap=round,fill=fillColor] (171.71,518.69) circle (  1.16);

\path[draw=drawColor,line width= 0.4pt,line join=round,line cap=round,fill=fillColor] (171.80,518.69) circle (  1.16);

\path[draw=drawColor,line width= 0.4pt,line join=round,line cap=round,fill=fillColor] (171.89,518.69) circle (  1.16);

\path[draw=drawColor,line width= 0.4pt,line join=round,line cap=round,fill=fillColor] (171.97,518.68) circle (  1.16);

\path[draw=drawColor,line width= 0.4pt,line join=round,line cap=round,fill=fillColor] (172.06,518.67) circle (  1.16);

\path[draw=drawColor,line width= 0.4pt,line join=round,line cap=round,fill=fillColor] (172.15,518.67) circle (  1.16);

\path[draw=drawColor,line width= 0.4pt,line join=round,line cap=round,fill=fillColor] (172.24,518.65) circle (  1.16);

\path[draw=drawColor,line width= 0.4pt,line join=round,line cap=round,fill=fillColor] (172.33,518.64) circle (  1.16);

\path[draw=drawColor,line width= 0.4pt,line join=round,line cap=round,fill=fillColor] (172.42,518.64) circle (  1.16);

\path[draw=drawColor,line width= 0.4pt,line join=round,line cap=round,fill=fillColor] (172.51,518.57) circle (  1.16);

\path[draw=drawColor,line width= 0.4pt,line join=round,line cap=round,fill=fillColor] (172.59,518.53) circle (  1.16);

\path[draw=drawColor,line width= 0.4pt,line join=round,line cap=round,fill=fillColor] (172.68,518.52) circle (  1.16);

\path[draw=drawColor,line width= 0.4pt,line join=round,line cap=round,fill=fillColor] (172.77,518.51) circle (  1.16);

\path[draw=drawColor,line width= 0.4pt,line join=round,line cap=round,fill=fillColor] (172.86,518.50) circle (  1.16);

\path[draw=drawColor,line width= 0.4pt,line join=round,line cap=round,fill=fillColor] (172.94,518.50) circle (  1.16);

\path[draw=drawColor,line width= 0.4pt,line join=round,line cap=round,fill=fillColor] (173.03,518.48) circle (  1.16);

\path[draw=drawColor,line width= 0.4pt,line join=round,line cap=round,fill=fillColor] (173.12,518.48) circle (  1.16);

\path[draw=drawColor,line width= 0.4pt,line join=round,line cap=round,fill=fillColor] (173.20,518.47) circle (  1.16);

\path[draw=drawColor,line width= 0.4pt,line join=round,line cap=round,fill=fillColor] (173.29,518.46) circle (  1.16);

\path[draw=drawColor,line width= 0.4pt,line join=round,line cap=round,fill=fillColor] (173.38,518.46) circle (  1.16);

\path[draw=drawColor,line width= 0.4pt,line join=round,line cap=round,fill=fillColor] (173.46,518.45) circle (  1.16);

\path[draw=drawColor,line width= 0.4pt,line join=round,line cap=round,fill=fillColor] (173.55,518.45) circle (  1.16);

\path[draw=drawColor,line width= 0.4pt,line join=round,line cap=round,fill=fillColor] (173.64,518.44) circle (  1.16);

\path[draw=drawColor,line width= 0.4pt,line join=round,line cap=round,fill=fillColor] (173.72,518.43) circle (  1.16);

\path[draw=drawColor,line width= 0.4pt,line join=round,line cap=round,fill=fillColor] (173.81,518.41) circle (  1.16);

\path[draw=drawColor,line width= 0.4pt,line join=round,line cap=round,fill=fillColor] (173.89,518.41) circle (  1.16);

\path[draw=drawColor,line width= 0.4pt,line join=round,line cap=round,fill=fillColor] (173.98,518.41) circle (  1.16);

\path[draw=drawColor,line width= 0.4pt,line join=round,line cap=round,fill=fillColor] (174.07,518.39) circle (  1.16);

\path[draw=drawColor,line width= 0.4pt,line join=round,line cap=round,fill=fillColor] (174.15,518.39) circle (  1.16);

\path[draw=drawColor,line width= 0.4pt,line join=round,line cap=round,fill=fillColor] (174.24,518.38) circle (  1.16);

\path[draw=drawColor,line width= 0.4pt,line join=round,line cap=round,fill=fillColor] (174.32,518.37) circle (  1.16);

\path[draw=drawColor,line width= 0.4pt,line join=round,line cap=round,fill=fillColor] (174.41,518.36) circle (  1.16);

\path[draw=drawColor,line width= 0.4pt,line join=round,line cap=round,fill=fillColor] (174.49,518.33) circle (  1.16);

\path[draw=drawColor,line width= 0.4pt,line join=round,line cap=round,fill=fillColor] (174.58,518.32) circle (  1.16);

\path[draw=drawColor,line width= 0.4pt,line join=round,line cap=round,fill=fillColor] (174.66,518.29) circle (  1.16);

\path[draw=drawColor,line width= 0.4pt,line join=round,line cap=round,fill=fillColor] (174.75,518.27) circle (  1.16);

\path[draw=drawColor,line width= 0.4pt,line join=round,line cap=round,fill=fillColor] (174.83,518.27) circle (  1.16);

\path[draw=drawColor,line width= 0.4pt,line join=round,line cap=round,fill=fillColor] (174.91,518.27) circle (  1.16);

\path[draw=drawColor,line width= 0.4pt,line join=round,line cap=round,fill=fillColor] (175.00,518.27) circle (  1.16);

\path[draw=drawColor,line width= 0.4pt,line join=round,line cap=round,fill=fillColor] (175.08,518.22) circle (  1.16);

\path[draw=drawColor,line width= 0.4pt,line join=round,line cap=round,fill=fillColor] (175.17,518.21) circle (  1.16);

\path[draw=drawColor,line width= 0.4pt,line join=round,line cap=round,fill=fillColor] (175.25,518.21) circle (  1.16);

\path[draw=drawColor,line width= 0.4pt,line join=round,line cap=round,fill=fillColor] (175.33,518.20) circle (  1.16);

\path[draw=drawColor,line width= 0.4pt,line join=round,line cap=round,fill=fillColor] (175.42,518.14) circle (  1.16);

\path[draw=drawColor,line width= 0.4pt,line join=round,line cap=round,fill=fillColor] (175.50,518.13) circle (  1.16);

\path[draw=drawColor,line width= 0.4pt,line join=round,line cap=round,fill=fillColor] (175.58,518.12) circle (  1.16);

\path[draw=drawColor,line width= 0.4pt,line join=round,line cap=round,fill=fillColor] (175.67,518.11) circle (  1.16);

\path[draw=drawColor,line width= 0.4pt,line join=round,line cap=round,fill=fillColor] (175.75,518.11) circle (  1.16);

\path[draw=drawColor,line width= 0.4pt,line join=round,line cap=round,fill=fillColor] (175.83,518.05) circle (  1.16);

\path[draw=drawColor,line width= 0.4pt,line join=round,line cap=round,fill=fillColor] (175.91,518.04) circle (  1.16);

\path[draw=drawColor,line width= 0.4pt,line join=round,line cap=round,fill=fillColor] (176.00,518.00) circle (  1.16);

\path[draw=drawColor,line width= 0.4pt,line join=round,line cap=round,fill=fillColor] (176.08,517.99) circle (  1.16);

\path[draw=drawColor,line width= 0.4pt,line join=round,line cap=round,fill=fillColor] (176.16,517.98) circle (  1.16);

\path[draw=drawColor,line width= 0.4pt,line join=round,line cap=round,fill=fillColor] (176.24,517.98) circle (  1.16);

\path[draw=drawColor,line width= 0.4pt,line join=round,line cap=round,fill=fillColor] (176.33,517.97) circle (  1.16);

\path[draw=drawColor,line width= 0.4pt,line join=round,line cap=round,fill=fillColor] (176.41,517.95) circle (  1.16);

\path[draw=drawColor,line width= 0.4pt,line join=round,line cap=round,fill=fillColor] (176.49,517.92) circle (  1.16);

\path[draw=drawColor,line width= 0.4pt,line join=round,line cap=round,fill=fillColor] (176.57,517.91) circle (  1.16);

\path[draw=drawColor,line width= 0.4pt,line join=round,line cap=round,fill=fillColor] (176.65,517.90) circle (  1.16);

\path[draw=drawColor,line width= 0.4pt,line join=round,line cap=round,fill=fillColor] (176.73,517.86) circle (  1.16);

\path[draw=drawColor,line width= 0.4pt,line join=round,line cap=round,fill=fillColor] (176.82,517.85) circle (  1.16);

\path[draw=drawColor,line width= 0.4pt,line join=round,line cap=round,fill=fillColor] (176.90,517.83) circle (  1.16);

\path[draw=drawColor,line width= 0.4pt,line join=round,line cap=round,fill=fillColor] (176.98,517.81) circle (  1.16);

\path[draw=drawColor,line width= 0.4pt,line join=round,line cap=round,fill=fillColor] (177.06,517.78) circle (  1.16);

\path[draw=drawColor,line width= 0.4pt,line join=round,line cap=round,fill=fillColor] (177.14,517.78) circle (  1.16);

\path[draw=drawColor,line width= 0.4pt,line join=round,line cap=round,fill=fillColor] (177.22,517.74) circle (  1.16);

\path[draw=drawColor,line width= 0.4pt,line join=round,line cap=round,fill=fillColor] (177.30,517.73) circle (  1.16);

\path[draw=drawColor,line width= 0.4pt,line join=round,line cap=round,fill=fillColor] (177.38,517.71) circle (  1.16);

\path[draw=drawColor,line width= 0.4pt,line join=round,line cap=round,fill=fillColor] (177.46,517.69) circle (  1.16);

\path[draw=drawColor,line width= 0.4pt,line join=round,line cap=round,fill=fillColor] (177.54,517.68) circle (  1.16);

\path[draw=drawColor,line width= 0.4pt,line join=round,line cap=round,fill=fillColor] (177.62,517.68) circle (  1.16);

\path[draw=drawColor,line width= 0.4pt,line join=round,line cap=round,fill=fillColor] (177.70,517.65) circle (  1.16);

\path[draw=drawColor,line width= 0.4pt,line join=round,line cap=round,fill=fillColor] (177.78,517.64) circle (  1.16);

\path[draw=drawColor,line width= 0.4pt,line join=round,line cap=round,fill=fillColor] (177.86,517.64) circle (  1.16);

\path[draw=drawColor,line width= 0.4pt,line join=round,line cap=round,fill=fillColor] (177.94,517.64) circle (  1.16);

\path[draw=drawColor,line width= 0.4pt,line join=round,line cap=round,fill=fillColor] (178.02,517.63) circle (  1.16);

\path[draw=drawColor,line width= 0.4pt,line join=round,line cap=round,fill=fillColor] (178.10,517.62) circle (  1.16);

\path[draw=drawColor,line width= 0.4pt,line join=round,line cap=round,fill=fillColor] (178.18,517.57) circle (  1.16);

\path[draw=drawColor,line width= 0.4pt,line join=round,line cap=round,fill=fillColor] (178.26,517.55) circle (  1.16);

\path[draw=drawColor,line width= 0.4pt,line join=round,line cap=round,fill=fillColor] (178.34,517.53) circle (  1.16);

\path[draw=drawColor,line width= 0.4pt,line join=round,line cap=round,fill=fillColor] (178.42,517.53) circle (  1.16);

\path[draw=drawColor,line width= 0.4pt,line join=round,line cap=round,fill=fillColor] (178.50,517.52) circle (  1.16);

\path[draw=drawColor,line width= 0.4pt,line join=round,line cap=round,fill=fillColor] (178.57,517.51) circle (  1.16);

\path[draw=drawColor,line width= 0.4pt,line join=round,line cap=round,fill=fillColor] (178.65,517.50) circle (  1.16);

\path[draw=drawColor,line width= 0.4pt,line join=round,line cap=round,fill=fillColor] (178.73,517.50) circle (  1.16);

\path[draw=drawColor,line width= 0.4pt,line join=round,line cap=round,fill=fillColor] (178.81,517.48) circle (  1.16);

\path[draw=drawColor,line width= 0.4pt,line join=round,line cap=round,fill=fillColor] (178.89,517.45) circle (  1.16);

\path[draw=drawColor,line width= 0.4pt,line join=round,line cap=round,fill=fillColor] (178.97,517.43) circle (  1.16);

\path[draw=drawColor,line width= 0.4pt,line join=round,line cap=round,fill=fillColor] (179.04,517.42) circle (  1.16);

\path[draw=drawColor,line width= 0.4pt,line join=round,line cap=round,fill=fillColor] (179.12,517.42) circle (  1.16);

\path[draw=drawColor,line width= 0.4pt,line join=round,line cap=round,fill=fillColor] (179.20,517.41) circle (  1.16);

\path[draw=drawColor,line width= 0.4pt,line join=round,line cap=round,fill=fillColor] (179.28,517.40) circle (  1.16);

\path[draw=drawColor,line width= 0.4pt,line join=round,line cap=round,fill=fillColor] (179.36,517.39) circle (  1.16);

\path[draw=drawColor,line width= 0.4pt,line join=round,line cap=round,fill=fillColor] (179.43,517.37) circle (  1.16);

\path[draw=drawColor,line width= 0.4pt,line join=round,line cap=round,fill=fillColor] (179.51,517.36) circle (  1.16);

\path[draw=drawColor,line width= 0.4pt,line join=round,line cap=round,fill=fillColor] (179.59,517.34) circle (  1.16);

\path[draw=drawColor,line width= 0.4pt,line join=round,line cap=round,fill=fillColor] (179.67,517.33) circle (  1.16);

\path[draw=drawColor,line width= 0.4pt,line join=round,line cap=round,fill=fillColor] (179.74,517.29) circle (  1.16);

\path[draw=drawColor,line width= 0.4pt,line join=round,line cap=round,fill=fillColor] (179.82,517.28) circle (  1.16);

\path[draw=drawColor,line width= 0.4pt,line join=round,line cap=round,fill=fillColor] (179.90,517.28) circle (  1.16);

\path[draw=drawColor,line width= 0.4pt,line join=round,line cap=round,fill=fillColor] (179.97,517.27) circle (  1.16);

\path[draw=drawColor,line width= 0.4pt,line join=round,line cap=round,fill=fillColor] (180.05,517.24) circle (  1.16);

\path[draw=drawColor,line width= 0.4pt,line join=round,line cap=round,fill=fillColor] (180.13,517.22) circle (  1.16);

\path[draw=drawColor,line width= 0.4pt,line join=round,line cap=round,fill=fillColor] (180.20,517.19) circle (  1.16);

\path[draw=drawColor,line width= 0.4pt,line join=round,line cap=round,fill=fillColor] (180.28,517.18) circle (  1.16);

\path[draw=drawColor,line width= 0.4pt,line join=round,line cap=round,fill=fillColor] (180.36,517.16) circle (  1.16);

\path[draw=drawColor,line width= 0.4pt,line join=round,line cap=round,fill=fillColor] (180.43,517.16) circle (  1.16);

\path[draw=drawColor,line width= 0.4pt,line join=round,line cap=round,fill=fillColor] (180.51,517.14) circle (  1.16);

\path[draw=drawColor,line width= 0.4pt,line join=round,line cap=round,fill=fillColor] (180.58,517.14) circle (  1.16);

\path[draw=drawColor,line width= 0.4pt,line join=round,line cap=round,fill=fillColor] (180.66,517.13) circle (  1.16);

\path[draw=drawColor,line width= 0.4pt,line join=round,line cap=round,fill=fillColor] (180.74,517.12) circle (  1.16);

\path[draw=drawColor,line width= 0.4pt,line join=round,line cap=round,fill=fillColor] (180.81,517.06) circle (  1.16);

\path[draw=drawColor,line width= 0.4pt,line join=round,line cap=round,fill=fillColor] (180.89,516.98) circle (  1.16);

\path[draw=drawColor,line width= 0.4pt,line join=round,line cap=round,fill=fillColor] (180.96,516.94) circle (  1.16);

\path[draw=drawColor,line width= 0.4pt,line join=round,line cap=round,fill=fillColor] (181.04,516.93) circle (  1.16);

\path[draw=drawColor,line width= 0.4pt,line join=round,line cap=round,fill=fillColor] (181.11,516.86) circle (  1.16);

\path[draw=drawColor,line width= 0.4pt,line join=round,line cap=round,fill=fillColor] (181.19,516.81) circle (  1.16);

\path[draw=drawColor,line width= 0.4pt,line join=round,line cap=round,fill=fillColor] (181.26,516.69) circle (  1.16);

\path[draw=drawColor,line width= 0.4pt,line join=round,line cap=round,fill=fillColor] (181.34,516.68) circle (  1.16);

\path[draw=drawColor,line width= 0.4pt,line join=round,line cap=round,fill=fillColor] (181.41,516.67) circle (  1.16);

\path[draw=drawColor,line width= 0.4pt,line join=round,line cap=round,fill=fillColor] (181.49,516.67) circle (  1.16);

\path[draw=drawColor,line width= 0.4pt,line join=round,line cap=round,fill=fillColor] (181.56,516.65) circle (  1.16);

\path[draw=drawColor,line width= 0.4pt,line join=round,line cap=round,fill=fillColor] (181.64,516.64) circle (  1.16);

\path[draw=drawColor,line width= 0.4pt,line join=round,line cap=round,fill=fillColor] (181.71,516.63) circle (  1.16);

\path[draw=drawColor,line width= 0.4pt,line join=round,line cap=round,fill=fillColor] (181.79,516.56) circle (  1.16);

\path[draw=drawColor,line width= 0.4pt,line join=round,line cap=round,fill=fillColor] (181.86,516.53) circle (  1.16);

\path[draw=drawColor,line width= 0.4pt,line join=round,line cap=round,fill=fillColor] (181.94,516.50) circle (  1.16);

\path[draw=drawColor,line width= 0.4pt,line join=round,line cap=round,fill=fillColor] (182.01,516.49) circle (  1.16);

\path[draw=drawColor,line width= 0.4pt,line join=round,line cap=round,fill=fillColor] (182.08,516.45) circle (  1.16);

\path[draw=drawColor,line width= 0.4pt,line join=round,line cap=round,fill=fillColor] (182.16,516.45) circle (  1.16);

\path[draw=drawColor,line width= 0.4pt,line join=round,line cap=round,fill=fillColor] (182.23,516.44) circle (  1.16);

\path[draw=drawColor,line width= 0.4pt,line join=round,line cap=round,fill=fillColor] (182.31,516.39) circle (  1.16);

\path[draw=drawColor,line width= 0.4pt,line join=round,line cap=round,fill=fillColor] (182.38,516.39) circle (  1.16);

\path[draw=drawColor,line width= 0.4pt,line join=round,line cap=round,fill=fillColor] (182.45,516.37) circle (  1.16);

\path[draw=drawColor,line width= 0.4pt,line join=round,line cap=round,fill=fillColor] (182.53,516.31) circle (  1.16);

\path[draw=drawColor,line width= 0.4pt,line join=round,line cap=round,fill=fillColor] (182.60,516.30) circle (  1.16);

\path[draw=drawColor,line width= 0.4pt,line join=round,line cap=round,fill=fillColor] (182.67,516.25) circle (  1.16);

\path[draw=drawColor,line width= 0.4pt,line join=round,line cap=round,fill=fillColor] (182.75,516.25) circle (  1.16);

\path[draw=drawColor,line width= 0.4pt,line join=round,line cap=round,fill=fillColor] (182.82,516.24) circle (  1.16);

\path[draw=drawColor,line width= 0.4pt,line join=round,line cap=round,fill=fillColor] (182.89,516.18) circle (  1.16);

\path[draw=drawColor,line width= 0.4pt,line join=round,line cap=round,fill=fillColor] (182.96,516.16) circle (  1.16);

\path[draw=drawColor,line width= 0.4pt,line join=round,line cap=round,fill=fillColor] (183.04,516.14) circle (  1.16);

\path[draw=drawColor,line width= 0.4pt,line join=round,line cap=round,fill=fillColor] (183.11,516.14) circle (  1.16);

\path[draw=drawColor,line width= 0.4pt,line join=round,line cap=round,fill=fillColor] (183.18,516.12) circle (  1.16);

\path[draw=drawColor,line width= 0.4pt,line join=round,line cap=round,fill=fillColor] (183.26,516.11) circle (  1.16);

\path[draw=drawColor,line width= 0.4pt,line join=round,line cap=round,fill=fillColor] (183.33,516.10) circle (  1.16);

\path[draw=drawColor,line width= 0.4pt,line join=round,line cap=round,fill=fillColor] (183.40,516.09) circle (  1.16);

\path[draw=drawColor,line width= 0.4pt,line join=round,line cap=round,fill=fillColor] (183.47,516.05) circle (  1.16);

\path[draw=drawColor,line width= 0.4pt,line join=round,line cap=round,fill=fillColor] (183.54,516.02) circle (  1.16);

\path[draw=drawColor,line width= 0.4pt,line join=round,line cap=round,fill=fillColor] (183.62,516.01) circle (  1.16);

\path[draw=drawColor,line width= 0.4pt,line join=round,line cap=round,fill=fillColor] (183.69,515.83) circle (  1.16);

\path[draw=drawColor,line width= 0.4pt,line join=round,line cap=round,fill=fillColor] (183.76,515.83) circle (  1.16);

\path[draw=drawColor,line width= 0.4pt,line join=round,line cap=round,fill=fillColor] (183.83,515.79) circle (  1.16);

\path[draw=drawColor,line width= 0.4pt,line join=round,line cap=round,fill=fillColor] (183.90,515.78) circle (  1.16);

\path[draw=drawColor,line width= 0.4pt,line join=round,line cap=round,fill=fillColor] (183.98,515.78) circle (  1.16);

\path[draw=drawColor,line width= 0.4pt,line join=round,line cap=round,fill=fillColor] (184.05,515.74) circle (  1.16);

\path[draw=drawColor,line width= 0.4pt,line join=round,line cap=round,fill=fillColor] (184.12,515.73) circle (  1.16);

\path[draw=drawColor,line width= 0.4pt,line join=round,line cap=round,fill=fillColor] (184.19,515.68) circle (  1.16);

\path[draw=drawColor,line width= 0.4pt,line join=round,line cap=round,fill=fillColor] (184.26,515.66) circle (  1.16);

\path[draw=drawColor,line width= 0.4pt,line join=round,line cap=round,fill=fillColor] (184.33,515.64) circle (  1.16);

\path[draw=drawColor,line width= 0.4pt,line join=round,line cap=round,fill=fillColor] (184.40,515.60) circle (  1.16);

\path[draw=drawColor,line width= 0.4pt,line join=round,line cap=round,fill=fillColor] (184.48,515.59) circle (  1.16);

\path[draw=drawColor,line width= 0.4pt,line join=round,line cap=round,fill=fillColor] (184.55,515.56) circle (  1.16);

\path[draw=drawColor,line width= 0.4pt,line join=round,line cap=round,fill=fillColor] (184.62,515.56) circle (  1.16);

\path[draw=drawColor,line width= 0.4pt,line join=round,line cap=round,fill=fillColor] (184.69,515.54) circle (  1.16);

\path[draw=drawColor,line width= 0.4pt,line join=round,line cap=round,fill=fillColor] (184.76,515.49) circle (  1.16);

\path[draw=drawColor,line width= 0.4pt,line join=round,line cap=round,fill=fillColor] (184.83,515.47) circle (  1.16);

\path[draw=drawColor,line width= 0.4pt,line join=round,line cap=round,fill=fillColor] (184.90,515.43) circle (  1.16);

\path[draw=drawColor,line width= 0.4pt,line join=round,line cap=round,fill=fillColor] (184.97,515.39) circle (  1.16);

\path[draw=drawColor,line width= 0.4pt,line join=round,line cap=round,fill=fillColor] (185.04,515.35) circle (  1.16);

\path[draw=drawColor,line width= 0.4pt,line join=round,line cap=round,fill=fillColor] (185.11,515.29) circle (  1.16);

\path[draw=drawColor,line width= 0.4pt,line join=round,line cap=round,fill=fillColor] (185.18,515.29) circle (  1.16);

\path[draw=drawColor,line width= 0.4pt,line join=round,line cap=round,fill=fillColor] (185.25,515.28) circle (  1.16);

\path[draw=drawColor,line width= 0.4pt,line join=round,line cap=round,fill=fillColor] (185.32,515.25) circle (  1.16);

\path[draw=drawColor,line width= 0.4pt,line join=round,line cap=round,fill=fillColor] (185.39,515.25) circle (  1.16);

\path[draw=drawColor,line width= 0.4pt,line join=round,line cap=round,fill=fillColor] (185.46,515.23) circle (  1.16);

\path[draw=drawColor,line width= 0.4pt,line join=round,line cap=round,fill=fillColor] (185.53,515.21) circle (  1.16);

\path[draw=drawColor,line width= 0.4pt,line join=round,line cap=round,fill=fillColor] (185.60,515.17) circle (  1.16);

\path[draw=drawColor,line width= 0.4pt,line join=round,line cap=round,fill=fillColor] (185.67,515.17) circle (  1.16);

\path[draw=drawColor,line width= 0.4pt,line join=round,line cap=round,fill=fillColor] (185.74,515.16) circle (  1.16);

\path[draw=drawColor,line width= 0.4pt,line join=round,line cap=round,fill=fillColor] (185.81,515.16) circle (  1.16);

\path[draw=drawColor,line width= 0.4pt,line join=round,line cap=round,fill=fillColor] (185.88,515.13) circle (  1.16);

\path[draw=drawColor,line width= 0.4pt,line join=round,line cap=round,fill=fillColor] (185.95,515.10) circle (  1.16);

\path[draw=drawColor,line width= 0.4pt,line join=round,line cap=round,fill=fillColor] (186.02,515.08) circle (  1.16);

\path[draw=drawColor,line width= 0.4pt,line join=round,line cap=round,fill=fillColor] (186.09,515.05) circle (  1.16);

\path[draw=drawColor,line width= 0.4pt,line join=round,line cap=round,fill=fillColor] (186.16,515.04) circle (  1.16);

\path[draw=drawColor,line width= 0.4pt,line join=round,line cap=round,fill=fillColor] (186.22,515.02) circle (  1.16);

\path[draw=drawColor,line width= 0.4pt,line join=round,line cap=round,fill=fillColor] (186.29,515.01) circle (  1.16);

\path[draw=drawColor,line width= 0.4pt,line join=round,line cap=round,fill=fillColor] (186.36,514.99) circle (  1.16);

\path[draw=drawColor,line width= 0.4pt,line join=round,line cap=round,fill=fillColor] (186.43,514.93) circle (  1.16);

\path[draw=drawColor,line width= 0.4pt,line join=round,line cap=round,fill=fillColor] (186.50,514.90) circle (  1.16);

\path[draw=drawColor,line width= 0.4pt,line join=round,line cap=round,fill=fillColor] (186.57,514.87) circle (  1.16);

\path[draw=drawColor,line width= 0.4pt,line join=round,line cap=round,fill=fillColor] (186.64,514.86) circle (  1.16);

\path[draw=drawColor,line width= 0.4pt,line join=round,line cap=round,fill=fillColor] (186.70,514.85) circle (  1.16);

\path[draw=drawColor,line width= 0.4pt,line join=round,line cap=round,fill=fillColor] (186.77,514.78) circle (  1.16);

\path[draw=drawColor,line width= 0.4pt,line join=round,line cap=round,fill=fillColor] (186.84,514.68) circle (  1.16);

\path[draw=drawColor,line width= 0.4pt,line join=round,line cap=round,fill=fillColor] (186.91,514.68) circle (  1.16);

\path[draw=drawColor,line width= 0.4pt,line join=round,line cap=round,fill=fillColor] (186.98,514.63) circle (  1.16);

\path[draw=drawColor,line width= 0.4pt,line join=round,line cap=round,fill=fillColor] (187.05,514.61) circle (  1.16);

\path[draw=drawColor,line width= 0.4pt,line join=round,line cap=round,fill=fillColor] (187.11,514.57) circle (  1.16);

\path[draw=drawColor,line width= 0.4pt,line join=round,line cap=round,fill=fillColor] (187.18,514.55) circle (  1.16);

\path[draw=drawColor,line width= 0.4pt,line join=round,line cap=round,fill=fillColor] (187.25,514.43) circle (  1.16);

\path[draw=drawColor,line width= 0.4pt,line join=round,line cap=round,fill=fillColor] (187.32,514.40) circle (  1.16);

\path[draw=drawColor,line width= 0.4pt,line join=round,line cap=round,fill=fillColor] (187.38,514.37) circle (  1.16);

\path[draw=drawColor,line width= 0.4pt,line join=round,line cap=round,fill=fillColor] (187.45,514.35) circle (  1.16);

\path[draw=drawColor,line width= 0.4pt,line join=round,line cap=round,fill=fillColor] (187.52,514.30) circle (  1.16);

\path[draw=drawColor,line width= 0.4pt,line join=round,line cap=round,fill=fillColor] (187.59,514.29) circle (  1.16);

\path[draw=drawColor,line width= 0.4pt,line join=round,line cap=round,fill=fillColor] (187.65,514.28) circle (  1.16);

\path[draw=drawColor,line width= 0.4pt,line join=round,line cap=round,fill=fillColor] (187.72,514.23) circle (  1.16);

\path[draw=drawColor,line width= 0.4pt,line join=round,line cap=round,fill=fillColor] (187.79,514.12) circle (  1.16);

\path[draw=drawColor,line width= 0.4pt,line join=round,line cap=round,fill=fillColor] (187.86,514.12) circle (  1.16);

\path[draw=drawColor,line width= 0.4pt,line join=round,line cap=round,fill=fillColor] (187.92,514.07) circle (  1.16);

\path[draw=drawColor,line width= 0.4pt,line join=round,line cap=round,fill=fillColor] (187.99,514.01) circle (  1.16);

\path[draw=drawColor,line width= 0.4pt,line join=round,line cap=round,fill=fillColor] (188.06,514.01) circle (  1.16);

\path[draw=drawColor,line width= 0.4pt,line join=round,line cap=round,fill=fillColor] (188.12,514.01) circle (  1.16);

\path[draw=drawColor,line width= 0.4pt,line join=round,line cap=round,fill=fillColor] (188.19,513.98) circle (  1.16);

\path[draw=drawColor,line width= 0.4pt,line join=round,line cap=round,fill=fillColor] (188.26,513.93) circle (  1.16);

\path[draw=drawColor,line width= 0.4pt,line join=round,line cap=round,fill=fillColor] (188.32,513.91) circle (  1.16);

\path[draw=drawColor,line width= 0.4pt,line join=round,line cap=round,fill=fillColor] (188.39,513.89) circle (  1.16);

\path[draw=drawColor,line width= 0.4pt,line join=round,line cap=round,fill=fillColor] (188.46,513.89) circle (  1.16);

\path[draw=drawColor,line width= 0.4pt,line join=round,line cap=round,fill=fillColor] (188.52,513.85) circle (  1.16);

\path[draw=drawColor,line width= 0.4pt,line join=round,line cap=round,fill=fillColor] (188.59,513.73) circle (  1.16);

\path[draw=drawColor,line width= 0.4pt,line join=round,line cap=round,fill=fillColor] (188.66,513.66) circle (  1.16);

\path[draw=drawColor,line width= 0.4pt,line join=round,line cap=round,fill=fillColor] (188.72,513.63) circle (  1.16);

\path[draw=drawColor,line width= 0.4pt,line join=round,line cap=round,fill=fillColor] (188.79,513.54) circle (  1.16);

\path[draw=drawColor,line width= 0.4pt,line join=round,line cap=round,fill=fillColor] (188.85,513.51) circle (  1.16);

\path[draw=drawColor,line width= 0.4pt,line join=round,line cap=round,fill=fillColor] (188.92,513.42) circle (  1.16);

\path[draw=drawColor,line width= 0.4pt,line join=round,line cap=round,fill=fillColor] (188.99,513.36) circle (  1.16);

\path[draw=drawColor,line width= 0.4pt,line join=round,line cap=round,fill=fillColor] (189.05,513.34) circle (  1.16);

\path[draw=drawColor,line width= 0.4pt,line join=round,line cap=round,fill=fillColor] (189.12,513.28) circle (  1.16);

\path[draw=drawColor,line width= 0.4pt,line join=round,line cap=round,fill=fillColor] (189.18,513.24) circle (  1.16);

\path[draw=drawColor,line width= 0.4pt,line join=round,line cap=round,fill=fillColor] (189.25,513.19) circle (  1.16);

\path[draw=drawColor,line width= 0.4pt,line join=round,line cap=round,fill=fillColor] (189.31,513.18) circle (  1.16);

\path[draw=drawColor,line width= 0.4pt,line join=round,line cap=round,fill=fillColor] (189.38,513.17) circle (  1.16);

\path[draw=drawColor,line width= 0.4pt,line join=round,line cap=round,fill=fillColor] (189.45,512.95) circle (  1.16);

\path[draw=drawColor,line width= 0.4pt,line join=round,line cap=round,fill=fillColor] (189.51,512.81) circle (  1.16);

\path[draw=drawColor,line width= 0.4pt,line join=round,line cap=round,fill=fillColor] (189.58,512.78) circle (  1.16);

\path[draw=drawColor,line width= 0.4pt,line join=round,line cap=round,fill=fillColor] (189.64,512.67) circle (  1.16);

\path[draw=drawColor,line width= 0.4pt,line join=round,line cap=round,fill=fillColor] (189.71,512.58) circle (  1.16);

\path[draw=drawColor,line width= 0.4pt,line join=round,line cap=round,fill=fillColor] (189.77,512.48) circle (  1.16);

\path[draw=drawColor,line width= 0.4pt,line join=round,line cap=round,fill=fillColor] (189.84,512.46) circle (  1.16);

\path[draw=drawColor,line width= 0.4pt,line join=round,line cap=round,fill=fillColor] (189.90,512.44) circle (  1.16);

\path[draw=drawColor,line width= 0.4pt,line join=round,line cap=round,fill=fillColor] (189.97,512.43) circle (  1.16);

\path[draw=drawColor,line width= 0.4pt,line join=round,line cap=round,fill=fillColor] (190.03,512.13) circle (  1.16);

\path[draw=drawColor,line width= 0.4pt,line join=round,line cap=round,fill=fillColor] (190.10,512.09) circle (  1.16);

\path[draw=drawColor,line width= 0.4pt,line join=round,line cap=round,fill=fillColor] (190.16,511.96) circle (  1.16);

\path[draw=drawColor,line width= 0.4pt,line join=round,line cap=round,fill=fillColor] (190.23,511.86) circle (  1.16);

\path[draw=drawColor,line width= 0.4pt,line join=round,line cap=round,fill=fillColor] (190.29,511.76) circle (  1.16);

\path[draw=drawColor,line width= 0.4pt,line join=round,line cap=round,fill=fillColor] (190.35,511.60) circle (  1.16);

\path[draw=drawColor,line width= 0.4pt,line join=round,line cap=round,fill=fillColor] (190.42,511.47) circle (  1.16);

\path[draw=drawColor,line width= 0.4pt,line join=round,line cap=round,fill=fillColor] (190.48,511.46) circle (  1.16);

\path[draw=drawColor,line width= 0.4pt,line join=round,line cap=round,fill=fillColor] (190.55,511.26) circle (  1.16);

\path[draw=drawColor,line width= 0.4pt,line join=round,line cap=round,fill=fillColor] (190.61,510.82) circle (  1.16);

\path[draw=drawColor,line width= 0.4pt,line join=round,line cap=round,fill=fillColor] (190.68,508.31) circle (  1.16);

\path[draw=drawColor,line width= 0.4pt,line join=round,line cap=round,fill=fillColor] (190.74,508.31) circle (  1.16);

\path[draw=drawColor,line width= 0.4pt,line join=round,line cap=round,fill=fillColor] (190.80,508.31) circle (  1.16);

\path[draw=drawColor,line width= 0.4pt,line join=round,line cap=round,fill=fillColor] (190.87,508.31) circle (  1.16);

\path[draw=drawColor,line width= 0.4pt,line join=round,line cap=round,fill=fillColor] (190.93,508.31) circle (  1.16);

\path[draw=drawColor,line width= 0.4pt,line join=round,line cap=round,fill=fillColor] (190.99,508.31) circle (  1.16);

\path[draw=drawColor,line width= 0.4pt,line join=round,line cap=round,fill=fillColor] (191.06,508.31) circle (  1.16);

\path[draw=drawColor,line width= 0.4pt,line join=round,line cap=round,fill=fillColor] (191.12,508.31) circle (  1.16);

\path[draw=drawColor,line width= 0.4pt,line join=round,line cap=round,fill=fillColor] (191.19,508.31) circle (  1.16);

\path[draw=drawColor,line width= 0.4pt,line join=round,line cap=round,fill=fillColor] (191.25,508.31) circle (  1.16);

\path[draw=drawColor,line width= 0.4pt,line join=round,line cap=round,fill=fillColor] (191.31,508.31) circle (  1.16);

\path[draw=drawColor,line width= 0.4pt,line join=round,line cap=round,fill=fillColor] (191.38,508.31) circle (  1.16);

\path[draw=drawColor,line width= 0.4pt,line join=round,line cap=round,fill=fillColor] (191.44,508.31) circle (  1.16);

\path[draw=drawColor,line width= 0.4pt,line join=round,line cap=round,fill=fillColor] (191.50,508.31) circle (  1.16);

\path[draw=drawColor,line width= 0.4pt,line join=round,line cap=round,fill=fillColor] (191.57,508.31) circle (  1.16);

\path[draw=drawColor,line width= 0.4pt,line join=round,line cap=round,fill=fillColor] (191.63,508.31) circle (  1.16);

\path[draw=drawColor,line width= 0.4pt,line join=round,line cap=round,fill=fillColor] (191.69,508.31) circle (  1.16);

\path[draw=drawColor,line width= 0.4pt,line join=round,line cap=round,fill=fillColor] (191.75,508.31) circle (  1.16);

\path[draw=drawColor,line width= 0.4pt,line join=round,line cap=round,fill=fillColor] (191.82,508.31) circle (  1.16);

\path[draw=drawColor,line width= 0.4pt,line join=round,line cap=round,fill=fillColor] (191.88,508.31) circle (  1.16);

\path[draw=drawColor,line width= 0.4pt,line join=round,line cap=round,fill=fillColor] (191.94,508.31) circle (  1.16);

\path[draw=drawColor,line width= 0.4pt,line join=round,line cap=round,fill=fillColor] (192.01,508.31) circle (  1.16);

\path[draw=drawColor,line width= 0.4pt,line join=round,line cap=round,fill=fillColor] (192.07,508.31) circle (  1.16);

\path[draw=drawColor,line width= 0.4pt,line join=round,line cap=round,fill=fillColor] (192.13,508.31) circle (  1.16);

\path[draw=drawColor,line width= 0.4pt,line join=round,line cap=round,fill=fillColor] (192.19,508.31) circle (  1.16);

\path[draw=drawColor,line width= 0.4pt,line join=round,line cap=round,fill=fillColor] (192.26,508.31) circle (  1.16);

\path[draw=drawColor,line width= 0.4pt,line join=round,line cap=round,fill=fillColor] (192.32,508.31) circle (  1.16);

\path[draw=drawColor,line width= 0.4pt,line join=round,line cap=round,fill=fillColor] (192.38,508.31) circle (  1.16);

\path[draw=drawColor,line width= 0.4pt,line join=round,line cap=round,fill=fillColor] (192.44,508.31) circle (  1.16);

\path[draw=drawColor,line width= 0.4pt,line join=round,line cap=round,fill=fillColor] (192.51,508.31) circle (  1.16);

\path[draw=drawColor,line width= 0.4pt,line join=round,line cap=round,fill=fillColor] (192.57,508.31) circle (  1.16);

\path[draw=drawColor,line width= 0.4pt,line join=round,line cap=round,fill=fillColor] (192.63,508.31) circle (  1.16);

\path[draw=drawColor,line width= 0.4pt,line join=round,line cap=round,fill=fillColor] (192.69,508.31) circle (  1.16);

\path[draw=drawColor,line width= 0.4pt,line join=round,line cap=round,fill=fillColor] (192.75,508.31) circle (  1.16);

\path[draw=drawColor,line width= 0.4pt,line join=round,line cap=round,fill=fillColor] (192.82,508.31) circle (  1.16);

\path[draw=drawColor,line width= 0.4pt,line join=round,line cap=round,fill=fillColor] (192.88,508.31) circle (  1.16);

\path[draw=drawColor,line width= 0.4pt,line join=round,line cap=round,fill=fillColor] (192.94,508.31) circle (  1.16);

\path[draw=drawColor,line width= 0.4pt,line join=round,line cap=round,fill=fillColor] (193.00,508.31) circle (  1.16);

\path[draw=drawColor,line width= 0.4pt,line join=round,line cap=round,fill=fillColor] (193.06,508.31) circle (  1.16);

\path[draw=drawColor,line width= 0.4pt,line join=round,line cap=round,fill=fillColor] (193.13,508.31) circle (  1.16);

\path[draw=drawColor,line width= 0.4pt,line join=round,line cap=round,fill=fillColor] (193.19,508.31) circle (  1.16);

\path[draw=drawColor,line width= 0.4pt,line join=round,line cap=round,fill=fillColor] (193.25,508.31) circle (  1.16);

\path[draw=drawColor,line width= 0.4pt,line join=round,line cap=round,fill=fillColor] (193.31,508.31) circle (  1.16);

\path[draw=drawColor,line width= 0.4pt,line join=round,line cap=round,fill=fillColor] (193.37,508.31) circle (  1.16);

\path[draw=drawColor,line width= 0.4pt,line join=round,line cap=round,fill=fillColor] (193.43,508.31) circle (  1.16);

\path[draw=drawColor,line width= 0.4pt,line join=round,line cap=round,fill=fillColor] (193.49,508.31) circle (  1.16);

\path[draw=drawColor,line width= 0.4pt,line join=round,line cap=round,fill=fillColor] (193.55,508.31) circle (  1.16);

\path[draw=drawColor,line width= 0.4pt,line join=round,line cap=round,fill=fillColor] (193.62,508.31) circle (  1.16);

\path[draw=drawColor,line width= 0.4pt,line join=round,line cap=round,fill=fillColor] (193.68,508.31) circle (  1.16);

\path[draw=drawColor,line width= 0.4pt,line join=round,line cap=round,fill=fillColor] (193.74,508.31) circle (  1.16);

\path[draw=drawColor,line width= 0.4pt,line join=round,line cap=round,fill=fillColor] (193.80,508.31) circle (  1.16);

\path[draw=drawColor,line width= 0.4pt,line join=round,line cap=round,fill=fillColor] (193.86,508.31) circle (  1.16);

\path[draw=drawColor,line width= 0.4pt,line join=round,line cap=round,fill=fillColor] (193.92,508.31) circle (  1.16);

\path[draw=drawColor,line width= 0.4pt,line join=round,line cap=round,fill=fillColor] (193.98,508.31) circle (  1.16);

\path[draw=drawColor,line width= 0.4pt,line join=round,line cap=round,fill=fillColor] (194.04,508.31) circle (  1.16);

\path[draw=drawColor,line width= 0.4pt,line join=round,line cap=round,fill=fillColor] (194.10,508.31) circle (  1.16);

\path[draw=drawColor,line width= 0.4pt,line join=round,line cap=round,fill=fillColor] (194.16,508.31) circle (  1.16);

\path[draw=drawColor,line width= 0.4pt,line join=round,line cap=round,fill=fillColor] (194.22,508.31) circle (  1.16);

\path[draw=drawColor,line width= 0.4pt,line join=round,line cap=round,fill=fillColor] (194.29,508.31) circle (  1.16);

\path[draw=drawColor,line width= 0.4pt,line join=round,line cap=round,fill=fillColor] (194.35,508.31) circle (  1.16);

\path[draw=drawColor,line width= 0.4pt,line join=round,line cap=round,fill=fillColor] (194.41,508.31) circle (  1.16);

\path[draw=drawColor,line width= 0.4pt,line join=round,line cap=round,fill=fillColor] (194.47,508.31) circle (  1.16);

\path[draw=drawColor,line width= 0.4pt,line join=round,line cap=round,fill=fillColor] (194.53,508.31) circle (  1.16);

\path[draw=drawColor,line width= 0.4pt,line join=round,line cap=round,fill=fillColor] (194.59,508.31) circle (  1.16);

\path[draw=drawColor,line width= 0.4pt,line join=round,line cap=round,fill=fillColor] (194.65,508.31) circle (  1.16);

\path[draw=drawColor,line width= 0.4pt,line join=round,line cap=round,fill=fillColor] (194.71,508.31) circle (  1.16);

\path[draw=drawColor,line width= 0.4pt,line join=round,line cap=round,fill=fillColor] (194.77,508.31) circle (  1.16);

\path[draw=drawColor,line width= 0.4pt,line join=round,line cap=round,fill=fillColor] (194.83,508.31) circle (  1.16);

\path[draw=drawColor,line width= 0.4pt,line join=round,line cap=round,fill=fillColor] (194.89,508.31) circle (  1.16);

\path[draw=drawColor,line width= 0.4pt,line join=round,line cap=round,fill=fillColor] (194.95,508.31) circle (  1.16);

\path[draw=drawColor,line width= 0.4pt,line join=round,line cap=round,fill=fillColor] (195.01,508.31) circle (  1.16);

\path[draw=drawColor,line width= 0.4pt,line join=round,line cap=round,fill=fillColor] (195.07,508.31) circle (  1.16);

\path[draw=drawColor,line width= 0.4pt,line join=round,line cap=round,fill=fillColor] (195.13,508.31) circle (  1.16);

\path[draw=drawColor,line width= 0.4pt,line join=round,line cap=round,fill=fillColor] (195.19,508.31) circle (  1.16);

\path[draw=drawColor,line width= 0.4pt,line join=round,line cap=round,fill=fillColor] (195.25,508.31) circle (  1.16);

\path[draw=drawColor,line width= 0.4pt,line join=round,line cap=round,fill=fillColor] (195.31,508.31) circle (  1.16);

\path[draw=drawColor,line width= 0.4pt,line join=round,line cap=round,fill=fillColor] (195.36,508.31) circle (  1.16);

\path[draw=drawColor,line width= 0.4pt,line join=round,line cap=round,fill=fillColor] (195.42,508.31) circle (  1.16);

\path[draw=drawColor,line width= 0.4pt,line join=round,line cap=round,fill=fillColor] (195.48,508.31) circle (  1.16);

\path[draw=drawColor,line width= 0.4pt,line join=round,line cap=round,fill=fillColor] (195.54,508.31) circle (  1.16);

\path[draw=drawColor,line width= 0.4pt,line join=round,line cap=round,fill=fillColor] (195.60,508.31) circle (  1.16);

\path[draw=drawColor,line width= 0.4pt,line join=round,line cap=round,fill=fillColor] (195.66,508.31) circle (  1.16);

\path[draw=drawColor,line width= 0.4pt,line join=round,line cap=round,fill=fillColor] (195.72,508.31) circle (  1.16);

\path[draw=drawColor,line width= 0.4pt,line join=round,line cap=round,fill=fillColor] (195.78,508.31) circle (  1.16);

\path[draw=drawColor,line width= 0.4pt,line join=round,line cap=round,fill=fillColor] (195.84,508.31) circle (  1.16);

\path[draw=drawColor,line width= 0.4pt,line join=round,line cap=round,fill=fillColor] (195.90,508.31) circle (  1.16);

\path[draw=drawColor,line width= 0.4pt,line join=round,line cap=round,fill=fillColor] (195.96,508.31) circle (  1.16);

\path[draw=drawColor,line width= 0.4pt,line join=round,line cap=round,fill=fillColor] (196.02,508.31) circle (  1.16);

\path[draw=drawColor,line width= 0.4pt,line join=round,line cap=round,fill=fillColor] (196.07,508.31) circle (  1.16);

\path[draw=drawColor,line width= 0.4pt,line join=round,line cap=round,fill=fillColor] (196.13,508.31) circle (  1.16);

\path[draw=drawColor,line width= 0.4pt,line join=round,line cap=round,fill=fillColor] (196.19,508.31) circle (  1.16);

\path[draw=drawColor,line width= 0.4pt,line join=round,line cap=round,fill=fillColor] (196.25,508.31) circle (  1.16);

\path[draw=drawColor,line width= 0.4pt,line join=round,line cap=round,fill=fillColor] (196.31,508.31) circle (  1.16);

\path[draw=drawColor,line width= 0.4pt,line join=round,line cap=round,fill=fillColor] (196.37,508.31) circle (  1.16);

\path[draw=drawColor,line width= 0.4pt,line join=round,line cap=round,fill=fillColor] (196.43,508.31) circle (  1.16);

\path[draw=drawColor,line width= 0.4pt,line join=round,line cap=round,fill=fillColor] (196.48,508.31) circle (  1.16);

\path[draw=drawColor,line width= 0.4pt,line join=round,line cap=round,fill=fillColor] (196.54,508.31) circle (  1.16);

\path[draw=drawColor,line width= 0.4pt,line join=round,line cap=round,fill=fillColor] (196.60,508.31) circle (  1.16);

\path[draw=drawColor,line width= 0.4pt,line join=round,line cap=round,fill=fillColor] (196.66,508.31) circle (  1.16);

\path[draw=drawColor,line width= 0.4pt,line join=round,line cap=round,fill=fillColor] (196.72,508.31) circle (  1.16);

\path[draw=drawColor,line width= 0.4pt,line join=round,line cap=round,fill=fillColor] (196.78,508.31) circle (  1.16);

\path[draw=drawColor,line width= 0.4pt,line join=round,line cap=round,fill=fillColor] (196.83,508.31) circle (  1.16);

\path[draw=drawColor,line width= 0.4pt,line join=round,line cap=round,fill=fillColor] (196.89,508.31) circle (  1.16);

\path[draw=drawColor,line width= 0.4pt,line join=round,line cap=round,fill=fillColor] (196.95,508.31) circle (  1.16);

\path[draw=drawColor,line width= 0.4pt,line join=round,line cap=round,fill=fillColor] (197.01,508.31) circle (  1.16);

\path[draw=drawColor,line width= 0.4pt,line join=round,line cap=round,fill=fillColor] (197.07,508.31) circle (  1.16);

\path[draw=drawColor,line width= 0.4pt,line join=round,line cap=round,fill=fillColor] (197.12,508.31) circle (  1.16);

\path[draw=drawColor,line width= 0.4pt,line join=round,line cap=round,fill=fillColor] (197.18,508.31) circle (  1.16);

\path[draw=drawColor,line width= 0.4pt,line join=round,line cap=round,fill=fillColor] (197.24,508.31) circle (  1.16);

\path[draw=drawColor,line width= 0.4pt,line join=round,line cap=round,fill=fillColor] (197.30,508.31) circle (  1.16);

\path[draw=drawColor,line width= 0.4pt,line join=round,line cap=round,fill=fillColor] (197.36,508.31) circle (  1.16);

\path[draw=drawColor,line width= 0.4pt,line join=round,line cap=round,fill=fillColor] (197.41,508.31) circle (  1.16);

\path[draw=drawColor,line width= 0.4pt,line join=round,line cap=round,fill=fillColor] (197.47,508.31) circle (  1.16);

\path[draw=drawColor,line width= 0.4pt,line join=round,line cap=round,fill=fillColor] (197.53,508.31) circle (  1.16);

\path[draw=drawColor,line width= 0.4pt,line join=round,line cap=round,fill=fillColor] (197.59,508.31) circle (  1.16);

\path[draw=drawColor,line width= 0.4pt,line join=round,line cap=round,fill=fillColor] (197.64,508.31) circle (  1.16);

\path[draw=drawColor,line width= 0.4pt,line join=round,line cap=round,fill=fillColor] (197.70,508.31) circle (  1.16);

\path[draw=drawColor,line width= 0.4pt,line join=round,line cap=round,fill=fillColor] (197.76,508.31) circle (  1.16);

\path[draw=drawColor,line width= 0.4pt,line join=round,line cap=round,fill=fillColor] (197.82,508.31) circle (  1.16);

\path[draw=drawColor,line width= 0.4pt,line join=round,line cap=round,fill=fillColor] (197.87,508.31) circle (  1.16);

\path[draw=drawColor,line width= 0.4pt,line join=round,line cap=round,fill=fillColor] (197.93,508.31) circle (  1.16);

\path[draw=drawColor,line width= 0.4pt,line join=round,line cap=round,fill=fillColor] (197.99,508.31) circle (  1.16);

\path[draw=drawColor,line width= 0.4pt,line join=round,line cap=round,fill=fillColor] (198.04,508.31) circle (  1.16);

\path[draw=drawColor,line width= 0.4pt,line join=round,line cap=round,fill=fillColor] (198.10,508.31) circle (  1.16);

\path[draw=drawColor,line width= 0.4pt,line join=round,line cap=round,fill=fillColor] (198.16,508.31) circle (  1.16);

\path[draw=drawColor,line width= 0.4pt,line join=round,line cap=round,fill=fillColor] (198.22,508.31) circle (  1.16);

\path[draw=drawColor,line width= 0.4pt,line join=round,line cap=round,fill=fillColor] (198.27,508.31) circle (  1.16);

\path[draw=drawColor,line width= 0.4pt,line join=round,line cap=round,fill=fillColor] (198.33,508.31) circle (  1.16);

\path[draw=drawColor,line width= 0.4pt,line join=round,line cap=round,fill=fillColor] (198.39,508.31) circle (  1.16);

\path[draw=drawColor,line width= 0.4pt,line join=round,line cap=round,fill=fillColor] (198.44,508.31) circle (  1.16);

\path[draw=drawColor,line width= 0.4pt,line join=round,line cap=round,fill=fillColor] (198.50,508.31) circle (  1.16);

\path[draw=drawColor,line width= 0.4pt,line join=round,line cap=round,fill=fillColor] (198.56,508.31) circle (  1.16);

\path[draw=drawColor,line width= 0.4pt,line join=round,line cap=round,fill=fillColor] (198.61,508.31) circle (  1.16);

\path[draw=drawColor,line width= 0.4pt,line join=round,line cap=round,fill=fillColor] (198.67,508.31) circle (  1.16);

\path[draw=drawColor,line width= 0.4pt,line join=round,line cap=round,fill=fillColor] (198.73,508.31) circle (  1.16);

\path[draw=drawColor,line width= 0.4pt,line join=round,line cap=round,fill=fillColor] (198.78,508.31) circle (  1.16);

\path[draw=drawColor,line width= 0.4pt,line join=round,line cap=round,fill=fillColor] (198.84,508.31) circle (  1.16);

\path[draw=drawColor,line width= 0.4pt,line join=round,line cap=round,fill=fillColor] (198.90,508.31) circle (  1.16);

\path[draw=drawColor,line width= 0.4pt,line join=round,line cap=round,fill=fillColor] (198.95,508.31) circle (  1.16);

\path[draw=drawColor,line width= 0.4pt,line join=round,line cap=round,fill=fillColor] (199.01,508.31) circle (  1.16);

\path[draw=drawColor,line width= 0.4pt,line join=round,line cap=round,fill=fillColor] (199.06,508.31) circle (  1.16);

\path[draw=drawColor,line width= 0.4pt,line join=round,line cap=round,fill=fillColor] (199.12,508.31) circle (  1.16);

\path[draw=drawColor,line width= 0.4pt,line join=round,line cap=round,fill=fillColor] (199.18,508.31) circle (  1.16);

\path[draw=drawColor,line width= 0.4pt,line join=round,line cap=round,fill=fillColor] (199.23,508.31) circle (  1.16);

\path[draw=drawColor,line width= 0.4pt,line join=round,line cap=round,fill=fillColor] (199.29,508.31) circle (  1.16);

\path[draw=drawColor,line width= 0.4pt,line join=round,line cap=round,fill=fillColor] (199.34,508.31) circle (  1.16);

\path[draw=drawColor,line width= 0.4pt,line join=round,line cap=round,fill=fillColor] (199.40,508.31) circle (  1.16);

\path[draw=drawColor,line width= 0.4pt,line join=round,line cap=round,fill=fillColor] (199.46,508.31) circle (  1.16);

\path[draw=drawColor,line width= 0.4pt,line join=round,line cap=round,fill=fillColor] (199.51,508.31) circle (  1.16);

\path[draw=drawColor,line width= 0.4pt,line join=round,line cap=round,fill=fillColor] (199.57,508.31) circle (  1.16);

\path[draw=drawColor,line width= 0.4pt,line join=round,line cap=round,fill=fillColor] (199.62,508.31) circle (  1.16);

\path[draw=drawColor,line width= 0.4pt,line join=round,line cap=round,fill=fillColor] (199.68,508.31) circle (  1.16);

\path[draw=drawColor,line width= 0.4pt,line join=round,line cap=round,fill=fillColor] (199.73,508.31) circle (  1.16);

\path[draw=drawColor,line width= 0.4pt,line join=round,line cap=round,fill=fillColor] (199.79,508.31) circle (  1.16);

\path[draw=drawColor,line width= 0.4pt,line join=round,line cap=round,fill=fillColor] (199.85,508.31) circle (  1.16);

\path[draw=drawColor,line width= 0.4pt,line join=round,line cap=round,fill=fillColor] (199.90,508.31) circle (  1.16);

\path[draw=drawColor,line width= 0.4pt,line join=round,line cap=round,fill=fillColor] (199.96,508.31) circle (  1.16);

\path[draw=drawColor,line width= 0.4pt,line join=round,line cap=round,fill=fillColor] (200.01,508.31) circle (  1.16);

\path[draw=drawColor,line width= 0.4pt,line join=round,line cap=round,fill=fillColor] (200.07,508.31) circle (  1.16);

\path[draw=drawColor,line width= 0.4pt,line join=round,line cap=round,fill=fillColor] (200.12,508.31) circle (  1.16);

\path[draw=drawColor,line width= 0.4pt,line join=round,line cap=round,fill=fillColor] (200.18,508.31) circle (  1.16);

\path[draw=drawColor,line width= 0.4pt,line join=round,line cap=round,fill=fillColor] (200.23,508.31) circle (  1.16);

\path[draw=drawColor,line width= 0.4pt,line join=round,line cap=round,fill=fillColor] (200.29,508.31) circle (  1.16);

\path[draw=drawColor,line width= 0.4pt,line join=round,line cap=round,fill=fillColor] (200.34,508.31) circle (  1.16);

\path[draw=drawColor,line width= 0.4pt,line join=round,line cap=round,fill=fillColor] (200.40,508.31) circle (  1.16);

\path[draw=drawColor,line width= 0.4pt,line join=round,line cap=round,fill=fillColor] (200.45,508.31) circle (  1.16);

\path[draw=drawColor,line width= 0.4pt,line join=round,line cap=round,fill=fillColor] (200.51,508.31) circle (  1.16);

\path[draw=drawColor,line width= 0.4pt,line join=round,line cap=round,fill=fillColor] (200.56,508.31) circle (  1.16);

\path[draw=drawColor,line width= 0.4pt,line join=round,line cap=round,fill=fillColor] (200.62,508.31) circle (  1.16);

\path[draw=drawColor,line width= 0.4pt,line join=round,line cap=round,fill=fillColor] (200.67,508.31) circle (  1.16);

\path[draw=drawColor,line width= 0.4pt,line join=round,line cap=round,fill=fillColor] (200.73,508.31) circle (  1.16);

\path[draw=drawColor,line width= 0.4pt,line join=round,line cap=round,fill=fillColor] (200.78,508.31) circle (  1.16);

\path[draw=drawColor,line width= 0.4pt,line join=round,line cap=round,fill=fillColor] (200.84,508.31) circle (  1.16);

\path[draw=drawColor,line width= 0.4pt,line join=round,line cap=round,fill=fillColor] (200.89,508.31) circle (  1.16);

\path[draw=drawColor,line width= 0.4pt,line join=round,line cap=round,fill=fillColor] (200.95,508.31) circle (  1.16);

\path[draw=drawColor,line width= 0.4pt,line join=round,line cap=round,fill=fillColor] (201.00,508.31) circle (  1.16);

\path[draw=drawColor,line width= 0.4pt,line join=round,line cap=round,fill=fillColor] (201.06,508.31) circle (  1.16);

\path[draw=drawColor,line width= 0.4pt,line join=round,line cap=round,fill=fillColor] (201.11,508.31) circle (  1.16);

\path[draw=drawColor,line width= 0.4pt,line join=round,line cap=round,fill=fillColor] (201.17,508.31) circle (  1.16);

\path[draw=drawColor,line width= 0.4pt,line join=round,line cap=round,fill=fillColor] (201.22,508.31) circle (  1.16);

\path[draw=drawColor,line width= 0.4pt,line join=round,line cap=round,fill=fillColor] (201.27,508.31) circle (  1.16);

\path[draw=drawColor,line width= 0.4pt,line join=round,line cap=round,fill=fillColor] (201.33,508.31) circle (  1.16);

\path[draw=drawColor,line width= 0.4pt,line join=round,line cap=round,fill=fillColor] (201.38,508.31) circle (  1.16);

\path[draw=drawColor,line width= 0.4pt,line join=round,line cap=round,fill=fillColor] (201.44,508.31) circle (  1.16);

\path[draw=drawColor,line width= 0.4pt,line join=round,line cap=round,fill=fillColor] (201.49,508.31) circle (  1.16);

\path[draw=drawColor,line width= 0.4pt,line join=round,line cap=round,fill=fillColor] (201.55,508.31) circle (  1.16);

\path[draw=drawColor,line width= 0.4pt,line join=round,line cap=round,fill=fillColor] (201.60,508.31) circle (  1.16);

\path[draw=drawColor,line width= 0.4pt,line join=round,line cap=round,fill=fillColor] (201.65,508.31) circle (  1.16);

\path[draw=drawColor,line width= 0.4pt,line join=round,line cap=round,fill=fillColor] (201.71,508.31) circle (  1.16);

\path[draw=drawColor,line width= 0.4pt,line join=round,line cap=round,fill=fillColor] (201.76,508.31) circle (  1.16);

\path[draw=drawColor,line width= 0.4pt,line join=round,line cap=round,fill=fillColor] (201.82,508.31) circle (  1.16);

\path[draw=drawColor,line width= 0.4pt,line join=round,line cap=round,fill=fillColor] (201.87,508.31) circle (  1.16);

\path[draw=drawColor,line width= 0.4pt,line join=round,line cap=round,fill=fillColor] (201.92,508.31) circle (  1.16);

\path[draw=drawColor,line width= 0.4pt,line join=round,line cap=round,fill=fillColor] (201.98,508.31) circle (  1.16);

\path[draw=drawColor,line width= 0.4pt,line join=round,line cap=round,fill=fillColor] (202.03,508.31) circle (  1.16);

\path[draw=drawColor,line width= 0.4pt,line join=round,line cap=round,fill=fillColor] (202.09,508.31) circle (  1.16);

\path[draw=drawColor,line width= 0.4pt,line join=round,line cap=round,fill=fillColor] (202.14,508.31) circle (  1.16);

\path[draw=drawColor,line width= 0.4pt,line join=round,line cap=round,fill=fillColor] (202.19,508.31) circle (  1.16);

\path[draw=drawColor,line width= 0.4pt,line join=round,line cap=round,fill=fillColor] (202.25,508.31) circle (  1.16);

\path[draw=drawColor,line width= 0.4pt,line join=round,line cap=round,fill=fillColor] (202.30,508.31) circle (  1.16);

\path[draw=drawColor,line width= 0.4pt,line join=round,line cap=round,fill=fillColor] (202.35,508.31) circle (  1.16);

\path[draw=drawColor,line width= 0.4pt,line join=round,line cap=round,fill=fillColor] (202.41,508.31) circle (  1.16);

\path[draw=drawColor,line width= 0.4pt,line join=round,line cap=round,fill=fillColor] (202.46,508.31) circle (  1.16);

\path[draw=drawColor,line width= 0.4pt,line join=round,line cap=round,fill=fillColor] (202.51,508.31) circle (  1.16);

\path[draw=drawColor,line width= 0.4pt,line join=round,line cap=round,fill=fillColor] (202.57,508.31) circle (  1.16);

\path[draw=drawColor,line width= 0.4pt,line join=round,line cap=round,fill=fillColor] (202.62,508.31) circle (  1.16);

\path[draw=drawColor,line width= 0.4pt,line join=round,line cap=round,fill=fillColor] (202.67,508.31) circle (  1.16);

\path[draw=drawColor,line width= 0.4pt,line join=round,line cap=round,fill=fillColor] (202.73,508.31) circle (  1.16);

\path[draw=drawColor,line width= 0.4pt,line join=round,line cap=round,fill=fillColor] (202.78,508.31) circle (  1.16);

\path[draw=drawColor,line width= 0.4pt,line join=round,line cap=round,fill=fillColor] (202.83,508.31) circle (  1.16);

\path[draw=drawColor,line width= 0.4pt,line join=round,line cap=round,fill=fillColor] (202.89,508.31) circle (  1.16);

\path[draw=drawColor,line width= 0.4pt,line join=round,line cap=round,fill=fillColor] (202.94,508.31) circle (  1.16);

\path[draw=drawColor,line width= 0.4pt,line join=round,line cap=round,fill=fillColor] (202.99,508.31) circle (  1.16);

\path[draw=drawColor,line width= 0.4pt,line join=round,line cap=round,fill=fillColor] (203.05,508.31) circle (  1.16);

\path[draw=drawColor,line width= 0.4pt,line join=round,line cap=round,fill=fillColor] (203.10,508.31) circle (  1.16);

\path[draw=drawColor,line width= 0.4pt,line join=round,line cap=round,fill=fillColor] (203.15,508.31) circle (  1.16);

\path[draw=drawColor,line width= 0.4pt,line join=round,line cap=round,fill=fillColor] (203.20,508.31) circle (  1.16);

\path[draw=drawColor,line width= 0.4pt,line join=round,line cap=round,fill=fillColor] (203.26,508.31) circle (  1.16);

\path[draw=drawColor,line width= 0.4pt,line join=round,line cap=round,fill=fillColor] (203.31,508.31) circle (  1.16);

\path[draw=drawColor,line width= 0.4pt,line join=round,line cap=round,fill=fillColor] (203.36,508.31) circle (  1.16);

\path[draw=drawColor,line width= 0.4pt,line join=round,line cap=round,fill=fillColor] (203.41,508.31) circle (  1.16);

\path[draw=drawColor,line width= 0.4pt,line join=round,line cap=round,fill=fillColor] (203.47,508.31) circle (  1.16);

\path[draw=drawColor,line width= 0.4pt,line join=round,line cap=round,fill=fillColor] (203.52,508.31) circle (  1.16);

\path[draw=drawColor,line width= 0.4pt,line join=round,line cap=round,fill=fillColor] (203.57,508.31) circle (  1.16);

\path[draw=drawColor,line width= 0.4pt,line join=round,line cap=round,fill=fillColor] (203.63,508.31) circle (  1.16);

\path[draw=drawColor,line width= 0.4pt,line join=round,line cap=round,fill=fillColor] (203.68,508.31) circle (  1.16);

\path[draw=drawColor,line width= 0.4pt,line join=round,line cap=round,fill=fillColor] (203.73,508.31) circle (  1.16);

\path[draw=drawColor,line width= 0.4pt,line join=round,line cap=round,fill=fillColor] (203.78,508.31) circle (  1.16);

\path[draw=drawColor,line width= 0.4pt,line join=round,line cap=round,fill=fillColor] (203.83,508.31) circle (  1.16);

\path[draw=drawColor,line width= 0.4pt,line join=round,line cap=round,fill=fillColor] (203.89,508.31) circle (  1.16);

\path[draw=drawColor,line width= 0.4pt,line join=round,line cap=round,fill=fillColor] (203.94,508.31) circle (  1.16);

\path[draw=drawColor,line width= 0.4pt,line join=round,line cap=round,fill=fillColor] (203.99,508.31) circle (  1.16);

\path[draw=drawColor,line width= 0.4pt,line join=round,line cap=round,fill=fillColor] (204.04,508.31) circle (  1.16);

\path[draw=drawColor,line width= 0.4pt,line join=round,line cap=round,fill=fillColor] (204.10,508.31) circle (  1.16);

\path[draw=drawColor,line width= 0.4pt,line join=round,line cap=round,fill=fillColor] (204.15,508.31) circle (  1.16);

\path[draw=drawColor,line width= 0.4pt,line join=round,line cap=round,fill=fillColor] (204.20,508.31) circle (  1.16);

\path[draw=drawColor,line width= 0.4pt,line join=round,line cap=round,fill=fillColor] (204.25,508.31) circle (  1.16);

\path[draw=drawColor,line width= 0.4pt,line join=round,line cap=round,fill=fillColor] (204.30,508.31) circle (  1.16);

\path[draw=drawColor,line width= 0.4pt,line join=round,line cap=round,fill=fillColor] (204.36,508.31) circle (  1.16);

\path[draw=drawColor,line width= 0.4pt,line join=round,line cap=round,fill=fillColor] (204.41,508.31) circle (  1.16);

\path[draw=drawColor,line width= 0.4pt,line join=round,line cap=round,fill=fillColor] (204.46,508.31) circle (  1.16);

\path[draw=drawColor,line width= 0.4pt,line join=round,line cap=round,fill=fillColor] (204.51,508.31) circle (  1.16);

\path[draw=drawColor,line width= 0.4pt,line join=round,line cap=round,fill=fillColor] (204.56,508.31) circle (  1.16);

\path[draw=drawColor,line width= 0.4pt,line join=round,line cap=round,fill=fillColor] (204.62,508.31) circle (  1.16);

\path[draw=drawColor,line width= 0.4pt,line join=round,line cap=round,fill=fillColor] (204.67,508.31) circle (  1.16);

\path[draw=drawColor,line width= 0.4pt,line join=round,line cap=round,fill=fillColor] (204.72,508.31) circle (  1.16);

\path[draw=drawColor,line width= 0.4pt,line join=round,line cap=round,fill=fillColor] (204.77,508.31) circle (  1.16);

\path[draw=drawColor,line width= 0.4pt,line join=round,line cap=round,fill=fillColor] (204.82,508.31) circle (  1.16);

\path[draw=drawColor,line width= 0.4pt,line join=round,line cap=round,fill=fillColor] (204.87,508.31) circle (  1.16);

\path[draw=drawColor,line width= 0.4pt,line join=round,line cap=round,fill=fillColor] (204.93,508.31) circle (  1.16);

\path[draw=drawColor,line width= 0.4pt,line join=round,line cap=round,fill=fillColor] (204.98,508.31) circle (  1.16);

\path[draw=drawColor,line width= 0.4pt,line join=round,line cap=round,fill=fillColor] (205.03,508.31) circle (  1.16);

\path[draw=drawColor,line width= 0.4pt,line join=round,line cap=round,fill=fillColor] (205.08,508.31) circle (  1.16);

\path[draw=drawColor,line width= 0.4pt,line join=round,line cap=round,fill=fillColor] (205.13,508.31) circle (  1.16);

\path[draw=drawColor,line width= 0.4pt,line join=round,line cap=round,fill=fillColor] (205.18,508.31) circle (  1.16);

\path[draw=drawColor,line width= 0.4pt,line join=round,line cap=round,fill=fillColor] (205.23,508.31) circle (  1.16);

\path[draw=drawColor,line width= 0.4pt,line join=round,line cap=round,fill=fillColor] (205.29,508.31) circle (  1.16);

\path[draw=drawColor,line width= 0.4pt,line join=round,line cap=round,fill=fillColor] (205.34,508.31) circle (  1.16);

\path[draw=drawColor,line width= 0.4pt,line join=round,line cap=round,fill=fillColor] (205.39,508.31) circle (  1.16);

\path[draw=drawColor,line width= 0.4pt,line join=round,line cap=round,fill=fillColor] (205.44,508.31) circle (  1.16);

\path[draw=drawColor,line width= 0.4pt,line join=round,line cap=round,fill=fillColor] (205.49,508.31) circle (  1.16);

\path[draw=drawColor,line width= 0.4pt,line join=round,line cap=round,fill=fillColor] (205.54,508.31) circle (  1.16);

\path[draw=drawColor,line width= 0.4pt,line join=round,line cap=round,fill=fillColor] (205.59,508.31) circle (  1.16);

\path[draw=drawColor,line width= 0.4pt,line join=round,line cap=round,fill=fillColor] (205.64,508.31) circle (  1.16);

\path[draw=drawColor,line width= 0.4pt,line join=round,line cap=round,fill=fillColor] (205.69,508.31) circle (  1.16);

\path[draw=drawColor,line width= 0.4pt,line join=round,line cap=round,fill=fillColor] (205.75,508.31) circle (  1.16);

\path[draw=drawColor,line width= 0.4pt,line join=round,line cap=round,fill=fillColor] (205.80,508.31) circle (  1.16);

\path[draw=drawColor,line width= 0.4pt,line join=round,line cap=round,fill=fillColor] (205.85,508.31) circle (  1.16);

\path[draw=drawColor,line width= 0.4pt,line join=round,line cap=round,fill=fillColor] (205.90,508.31) circle (  1.16);

\path[draw=drawColor,line width= 0.4pt,line join=round,line cap=round,fill=fillColor] (205.95,508.31) circle (  1.16);

\path[draw=drawColor,line width= 0.4pt,line join=round,line cap=round,fill=fillColor] (206.00,508.31) circle (  1.16);

\path[draw=drawColor,line width= 0.4pt,line join=round,line cap=round,fill=fillColor] (206.05,508.31) circle (  1.16);

\path[draw=drawColor,line width= 0.4pt,line join=round,line cap=round,fill=fillColor] (206.10,508.31) circle (  1.16);

\path[draw=drawColor,line width= 0.4pt,line join=round,line cap=round,fill=fillColor] (206.15,508.31) circle (  1.16);

\path[draw=drawColor,line width= 0.4pt,line join=round,line cap=round,fill=fillColor] (206.20,508.31) circle (  1.16);

\path[draw=drawColor,line width= 0.4pt,line join=round,line cap=round,fill=fillColor] (206.25,508.31) circle (  1.16);

\path[draw=drawColor,line width= 0.4pt,line join=round,line cap=round,fill=fillColor] (206.30,508.31) circle (  1.16);

\path[draw=drawColor,line width= 0.4pt,line join=round,line cap=round,fill=fillColor] (206.35,508.31) circle (  1.16);

\path[draw=drawColor,line width= 0.4pt,line join=round,line cap=round,fill=fillColor] (206.40,508.31) circle (  1.16);

\path[draw=drawColor,line width= 0.4pt,line join=round,line cap=round,fill=fillColor] (206.45,508.31) circle (  1.16);

\path[draw=drawColor,line width= 0.4pt,line join=round,line cap=round,fill=fillColor] (206.50,508.31) circle (  1.16);

\path[draw=drawColor,line width= 0.4pt,line join=round,line cap=round,fill=fillColor] (206.55,508.31) circle (  1.16);

\path[draw=drawColor,line width= 0.4pt,line join=round,line cap=round,fill=fillColor] (206.61,508.31) circle (  1.16);

\path[draw=drawColor,line width= 0.4pt,line join=round,line cap=round,fill=fillColor] (206.66,508.31) circle (  1.16);

\path[draw=drawColor,line width= 0.4pt,line join=round,line cap=round,fill=fillColor] (206.71,508.31) circle (  1.16);

\path[draw=drawColor,line width= 0.4pt,line join=round,line cap=round,fill=fillColor] (206.76,508.31) circle (  1.16);

\path[draw=drawColor,line width= 0.4pt,line join=round,line cap=round,fill=fillColor] (206.81,508.31) circle (  1.16);

\path[draw=drawColor,line width= 0.4pt,line join=round,line cap=round,fill=fillColor] (206.86,508.31) circle (  1.16);

\path[draw=drawColor,line width= 0.4pt,line join=round,line cap=round,fill=fillColor] (206.91,508.31) circle (  1.16);

\path[draw=drawColor,line width= 0.4pt,line join=round,line cap=round,fill=fillColor] (206.96,508.31) circle (  1.16);

\path[draw=drawColor,line width= 0.4pt,line join=round,line cap=round,fill=fillColor] (207.01,508.31) circle (  1.16);

\path[draw=drawColor,line width= 0.4pt,line join=round,line cap=round,fill=fillColor] (207.06,508.31) circle (  1.16);

\path[draw=drawColor,line width= 0.4pt,line join=round,line cap=round,fill=fillColor] (207.11,508.31) circle (  1.16);

\path[draw=drawColor,line width= 0.4pt,line join=round,line cap=round,fill=fillColor] (207.16,508.31) circle (  1.16);

\path[draw=drawColor,line width= 0.4pt,line join=round,line cap=round,fill=fillColor] (207.21,508.31) circle (  1.16);

\path[draw=drawColor,line width= 0.4pt,line join=round,line cap=round,fill=fillColor] (207.26,508.31) circle (  1.16);

\path[draw=drawColor,line width= 0.4pt,line join=round,line cap=round,fill=fillColor] (207.31,508.31) circle (  1.16);

\path[draw=drawColor,line width= 0.4pt,line join=round,line cap=round,fill=fillColor] (207.36,508.31) circle (  1.16);

\path[draw=drawColor,line width= 0.4pt,line join=round,line cap=round,fill=fillColor] (207.41,508.31) circle (  1.16);

\path[draw=drawColor,line width= 0.4pt,line join=round,line cap=round,fill=fillColor] (207.46,508.31) circle (  1.16);

\path[draw=drawColor,line width= 0.4pt,line join=round,line cap=round,fill=fillColor] (207.50,508.31) circle (  1.16);

\path[draw=drawColor,line width= 0.4pt,line join=round,line cap=round,fill=fillColor] (207.55,508.31) circle (  1.16);

\path[draw=drawColor,line width= 0.4pt,line join=round,line cap=round,fill=fillColor] (207.60,508.31) circle (  1.16);

\path[draw=drawColor,line width= 0.4pt,line join=round,line cap=round,fill=fillColor] (207.65,508.31) circle (  1.16);

\path[draw=drawColor,line width= 0.4pt,line join=round,line cap=round,fill=fillColor] (207.70,508.31) circle (  1.16);

\path[draw=drawColor,line width= 0.4pt,line join=round,line cap=round,fill=fillColor] (207.75,508.31) circle (  1.16);

\path[draw=drawColor,line width= 0.4pt,line join=round,line cap=round,fill=fillColor] (207.80,508.31) circle (  1.16);

\path[draw=drawColor,line width= 0.4pt,line join=round,line cap=round,fill=fillColor] (207.85,508.31) circle (  1.16);

\path[draw=drawColor,line width= 0.4pt,line join=round,line cap=round,fill=fillColor] (207.90,508.31) circle (  1.16);

\path[draw=drawColor,line width= 0.4pt,line join=round,line cap=round,fill=fillColor] (207.95,508.31) circle (  1.16);

\path[draw=drawColor,line width= 0.4pt,line join=round,line cap=round,fill=fillColor] (208.00,508.31) circle (  1.16);

\path[draw=drawColor,line width= 0.4pt,line join=round,line cap=round,fill=fillColor] (208.05,508.31) circle (  1.16);

\path[draw=drawColor,line width= 0.4pt,line join=round,line cap=round,fill=fillColor] (208.10,508.31) circle (  1.16);

\path[draw=drawColor,line width= 0.4pt,line join=round,line cap=round,fill=fillColor] (208.15,508.31) circle (  1.16);

\path[draw=drawColor,line width= 0.4pt,line join=round,line cap=round,fill=fillColor] (208.20,508.31) circle (  1.16);

\path[draw=drawColor,line width= 0.4pt,line join=round,line cap=round,fill=fillColor] (208.25,508.31) circle (  1.16);

\path[draw=drawColor,line width= 0.4pt,line join=round,line cap=round,fill=fillColor] (208.29,508.31) circle (  1.16);

\path[draw=drawColor,line width= 0.4pt,line join=round,line cap=round,fill=fillColor] (208.34,508.31) circle (  1.16);

\path[draw=drawColor,line width= 0.4pt,line join=round,line cap=round,fill=fillColor] (208.39,508.31) circle (  1.16);

\path[draw=drawColor,line width= 0.4pt,line join=round,line cap=round,fill=fillColor] (208.44,508.31) circle (  1.16);

\path[draw=drawColor,line width= 0.4pt,line join=round,line cap=round,fill=fillColor] (208.49,508.31) circle (  1.16);

\path[draw=drawColor,line width= 0.4pt,line join=round,line cap=round,fill=fillColor] (208.54,508.31) circle (  1.16);

\path[draw=drawColor,line width= 0.4pt,line join=round,line cap=round,fill=fillColor] (208.59,508.31) circle (  1.16);

\path[draw=drawColor,line width= 0.4pt,line join=round,line cap=round,fill=fillColor] (208.64,508.31) circle (  1.16);

\path[draw=drawColor,line width= 0.4pt,line join=round,line cap=round,fill=fillColor] (208.69,508.31) circle (  1.16);

\path[draw=drawColor,line width= 0.4pt,line join=round,line cap=round,fill=fillColor] (208.74,508.31) circle (  1.16);

\path[draw=drawColor,line width= 0.4pt,line join=round,line cap=round,fill=fillColor] (208.78,508.31) circle (  1.16);

\path[draw=drawColor,line width= 0.4pt,line join=round,line cap=round,fill=fillColor] (208.83,508.31) circle (  1.16);

\path[draw=drawColor,line width= 0.4pt,line join=round,line cap=round,fill=fillColor] (208.88,508.31) circle (  1.16);

\path[draw=drawColor,line width= 0.4pt,line join=round,line cap=round,fill=fillColor] (208.93,508.31) circle (  1.16);

\path[draw=drawColor,line width= 0.4pt,line join=round,line cap=round,fill=fillColor] (208.98,508.31) circle (  1.16);

\path[draw=drawColor,line width= 0.4pt,line join=round,line cap=round,fill=fillColor] (209.03,508.31) circle (  1.16);

\path[draw=drawColor,line width= 0.4pt,line join=round,line cap=round,fill=fillColor] (209.08,508.31) circle (  1.16);

\path[draw=drawColor,line width= 0.4pt,line join=round,line cap=round,fill=fillColor] (209.12,508.31) circle (  1.16);

\path[draw=drawColor,line width= 0.4pt,line join=round,line cap=round,fill=fillColor] (209.17,508.31) circle (  1.16);

\path[draw=drawColor,line width= 0.4pt,line join=round,line cap=round,fill=fillColor] (209.22,508.31) circle (  1.16);

\path[draw=drawColor,line width= 0.4pt,line join=round,line cap=round,fill=fillColor] (209.27,508.31) circle (  1.16);

\path[draw=drawColor,line width= 0.4pt,line join=round,line cap=round,fill=fillColor] (209.32,508.31) circle (  1.16);

\path[draw=drawColor,line width= 0.4pt,line join=round,line cap=round,fill=fillColor] (209.37,508.31) circle (  1.16);

\path[draw=drawColor,line width= 0.4pt,line join=round,line cap=round,fill=fillColor] (209.42,508.31) circle (  1.16);

\path[draw=drawColor,line width= 0.4pt,line join=round,line cap=round,fill=fillColor] (209.46,508.31) circle (  1.16);

\path[draw=drawColor,line width= 0.4pt,line join=round,line cap=round,fill=fillColor] (209.51,508.31) circle (  1.16);

\path[draw=drawColor,line width= 0.4pt,line join=round,line cap=round,fill=fillColor] (209.56,508.31) circle (  1.16);

\path[draw=drawColor,line width= 0.4pt,line join=round,line cap=round,fill=fillColor] (209.61,508.31) circle (  1.16);

\path[draw=drawColor,line width= 0.4pt,line join=round,line cap=round,fill=fillColor] (209.66,508.31) circle (  1.16);

\path[draw=drawColor,line width= 0.4pt,line join=round,line cap=round,fill=fillColor] (209.70,508.31) circle (  1.16);

\path[draw=drawColor,line width= 0.4pt,line join=round,line cap=round,fill=fillColor] (209.75,508.31) circle (  1.16);

\path[draw=drawColor,line width= 0.4pt,line join=round,line cap=round,fill=fillColor] (209.80,508.31) circle (  1.16);

\path[draw=drawColor,line width= 0.4pt,line join=round,line cap=round,fill=fillColor] (209.85,508.31) circle (  1.16);

\path[draw=drawColor,line width= 0.4pt,line join=round,line cap=round,fill=fillColor] (209.90,508.31) circle (  1.16);

\path[draw=drawColor,line width= 0.4pt,line join=round,line cap=round,fill=fillColor] (209.94,508.31) circle (  1.16);

\path[draw=drawColor,line width= 0.4pt,line join=round,line cap=round,fill=fillColor] (209.99,508.31) circle (  1.16);

\path[draw=drawColor,line width= 0.4pt,line join=round,line cap=round,fill=fillColor] (210.04,508.31) circle (  1.16);

\path[draw=drawColor,line width= 0.4pt,line join=round,line cap=round,fill=fillColor] (210.09,508.31) circle (  1.16);

\path[draw=drawColor,line width= 0.4pt,line join=round,line cap=round,fill=fillColor] (210.14,508.31) circle (  1.16);

\path[draw=drawColor,line width= 0.4pt,line join=round,line cap=round,fill=fillColor] (210.18,508.31) circle (  1.16);

\path[draw=drawColor,line width= 0.4pt,line join=round,line cap=round,fill=fillColor] (210.23,508.31) circle (  1.16);

\path[draw=drawColor,line width= 0.4pt,line join=round,line cap=round,fill=fillColor] (210.28,508.31) circle (  1.16);

\path[draw=drawColor,line width= 0.4pt,line join=round,line cap=round,fill=fillColor] (210.33,508.31) circle (  1.16);

\path[draw=drawColor,line width= 0.4pt,line join=round,line cap=round,fill=fillColor] (210.38,508.31) circle (  1.16);

\path[draw=drawColor,line width= 0.4pt,line join=round,line cap=round,fill=fillColor] (210.42,508.31) circle (  1.16);

\path[draw=drawColor,line width= 0.4pt,line join=round,line cap=round,fill=fillColor] (210.47,508.31) circle (  1.16);

\path[draw=drawColor,line width= 0.4pt,line join=round,line cap=round,fill=fillColor] (210.52,508.31) circle (  1.16);

\path[draw=drawColor,line width= 0.4pt,line join=round,line cap=round,fill=fillColor] (210.57,508.31) circle (  1.16);

\path[draw=drawColor,line width= 0.4pt,line join=round,line cap=round,fill=fillColor] (210.61,508.31) circle (  1.16);

\path[draw=drawColor,line width= 0.4pt,line join=round,line cap=round,fill=fillColor] (210.66,508.31) circle (  1.16);

\path[draw=drawColor,line width= 0.4pt,line join=round,line cap=round,fill=fillColor] (210.71,508.31) circle (  1.16);

\path[draw=drawColor,line width= 0.4pt,line join=round,line cap=round,fill=fillColor] (210.76,508.31) circle (  1.16);

\path[draw=drawColor,line width= 0.4pt,line join=round,line cap=round,fill=fillColor] (210.80,508.31) circle (  1.16);

\path[draw=drawColor,line width= 0.4pt,line join=round,line cap=round,fill=fillColor] (210.85,508.31) circle (  1.16);

\path[draw=drawColor,line width= 0.4pt,line join=round,line cap=round,fill=fillColor] (210.90,508.31) circle (  1.16);

\path[draw=drawColor,line width= 0.4pt,line join=round,line cap=round,fill=fillColor] (210.95,508.31) circle (  1.16);

\path[draw=drawColor,line width= 0.4pt,line join=round,line cap=round,fill=fillColor] (210.99,508.31) circle (  1.16);

\path[draw=drawColor,line width= 0.4pt,line join=round,line cap=round,fill=fillColor] (211.04,508.31) circle (  1.16);

\path[draw=drawColor,line width= 0.4pt,line join=round,line cap=round,fill=fillColor] (211.09,508.31) circle (  1.16);

\path[draw=drawColor,line width= 0.4pt,line join=round,line cap=round,fill=fillColor] (211.13,508.31) circle (  1.16);

\path[draw=drawColor,line width= 0.4pt,line join=round,line cap=round,fill=fillColor] (211.18,508.31) circle (  1.16);

\path[draw=drawColor,line width= 0.4pt,line join=round,line cap=round,fill=fillColor] (211.23,508.31) circle (  1.16);

\path[draw=drawColor,line width= 0.4pt,line join=round,line cap=round,fill=fillColor] (211.28,508.31) circle (  1.16);

\path[draw=drawColor,line width= 0.4pt,line join=round,line cap=round,fill=fillColor] (211.32,508.31) circle (  1.16);

\path[draw=drawColor,line width= 0.4pt,line join=round,line cap=round,fill=fillColor] (211.37,508.31) circle (  1.16);

\path[draw=drawColor,line width= 0.4pt,line join=round,line cap=round,fill=fillColor] (211.42,508.31) circle (  1.16);

\path[draw=drawColor,line width= 0.4pt,line join=round,line cap=round,fill=fillColor] (211.46,508.31) circle (  1.16);

\path[draw=drawColor,line width= 0.4pt,line join=round,line cap=round,fill=fillColor] (211.51,508.31) circle (  1.16);

\path[draw=drawColor,line width= 0.4pt,line join=round,line cap=round,fill=fillColor] (211.56,508.31) circle (  1.16);

\path[draw=drawColor,line width= 0.4pt,line join=round,line cap=round,fill=fillColor] (211.60,508.31) circle (  1.16);

\path[draw=drawColor,line width= 0.4pt,line join=round,line cap=round,fill=fillColor] (211.65,508.31) circle (  1.16);

\path[draw=drawColor,line width= 0.4pt,line join=round,line cap=round,fill=fillColor] (211.70,508.31) circle (  1.16);

\path[draw=drawColor,line width= 0.4pt,line join=round,line cap=round,fill=fillColor] (211.75,508.31) circle (  1.16);

\path[draw=drawColor,line width= 0.4pt,line join=round,line cap=round,fill=fillColor] (211.79,508.31) circle (  1.16);

\path[draw=drawColor,line width= 0.4pt,line join=round,line cap=round,fill=fillColor] (211.84,508.31) circle (  1.16);

\path[draw=drawColor,line width= 0.4pt,line join=round,line cap=round,fill=fillColor] (211.89,508.31) circle (  1.16);

\path[draw=drawColor,line width= 0.4pt,line join=round,line cap=round,fill=fillColor] (211.93,508.31) circle (  1.16);

\path[draw=drawColor,line width= 0.4pt,line join=round,line cap=round,fill=fillColor] (211.98,508.31) circle (  1.16);

\path[draw=drawColor,line width= 0.4pt,line join=round,line cap=round,fill=fillColor] (212.03,508.31) circle (  1.16);

\path[draw=drawColor,line width= 0.4pt,line join=round,line cap=round,fill=fillColor] (212.07,508.31) circle (  1.16);

\path[draw=drawColor,line width= 0.4pt,line join=round,line cap=round,fill=fillColor] (212.12,508.31) circle (  1.16);

\path[draw=drawColor,line width= 0.4pt,line join=round,line cap=round,fill=fillColor] (212.16,508.31) circle (  1.16);

\path[draw=drawColor,line width= 0.4pt,line join=round,line cap=round,fill=fillColor] (212.21,508.31) circle (  1.16);

\path[draw=drawColor,line width= 0.4pt,line join=round,line cap=round,fill=fillColor] (212.26,508.31) circle (  1.16);

\path[draw=drawColor,line width= 0.4pt,line join=round,line cap=round,fill=fillColor] (212.30,508.31) circle (  1.16);

\path[draw=drawColor,line width= 0.4pt,line join=round,line cap=round,fill=fillColor] (212.35,508.31) circle (  1.16);

\path[draw=drawColor,line width= 0.4pt,line join=round,line cap=round,fill=fillColor] (212.40,508.31) circle (  1.16);

\path[draw=drawColor,line width= 0.4pt,line join=round,line cap=round,fill=fillColor] (212.44,508.31) circle (  1.16);

\path[draw=drawColor,line width= 0.4pt,line join=round,line cap=round,fill=fillColor] (212.49,508.31) circle (  1.16);

\path[draw=drawColor,line width= 0.4pt,line join=round,line cap=round,fill=fillColor] (212.54,508.31) circle (  1.16);

\path[draw=drawColor,line width= 0.4pt,line join=round,line cap=round,fill=fillColor] (212.58,508.31) circle (  1.16);

\path[draw=drawColor,line width= 0.4pt,line join=round,line cap=round,fill=fillColor] (212.63,508.31) circle (  1.16);

\path[draw=drawColor,line width= 0.4pt,line join=round,line cap=round,fill=fillColor] (212.67,508.31) circle (  1.16);

\path[draw=drawColor,line width= 0.4pt,line join=round,line cap=round,fill=fillColor] (212.72,508.31) circle (  1.16);

\path[draw=drawColor,line width= 0.4pt,line join=round,line cap=round,fill=fillColor] (212.77,508.31) circle (  1.16);

\path[draw=drawColor,line width= 0.4pt,line join=round,line cap=round,fill=fillColor] (212.81,508.31) circle (  1.16);

\path[draw=drawColor,line width= 0.4pt,line join=round,line cap=round,fill=fillColor] (212.86,508.31) circle (  1.16);

\path[draw=drawColor,line width= 0.4pt,line join=round,line cap=round,fill=fillColor] (212.91,508.31) circle (  1.16);

\path[draw=drawColor,line width= 0.4pt,line join=round,line cap=round,fill=fillColor] (212.95,508.31) circle (  1.16);

\path[draw=drawColor,line width= 0.4pt,line join=round,line cap=round,fill=fillColor] (213.00,508.31) circle (  1.16);

\path[draw=drawColor,line width= 0.4pt,line join=round,line cap=round,fill=fillColor] (213.04,508.31) circle (  1.16);

\path[draw=drawColor,line width= 0.4pt,line join=round,line cap=round,fill=fillColor] (213.09,508.31) circle (  1.16);

\path[draw=drawColor,line width= 0.4pt,line join=round,line cap=round,fill=fillColor] (213.14,508.31) circle (  1.16);

\path[draw=drawColor,line width= 0.4pt,line join=round,line cap=round,fill=fillColor] (213.18,508.31) circle (  1.16);

\path[draw=drawColor,line width= 0.4pt,line join=round,line cap=round,fill=fillColor] (213.23,508.31) circle (  1.16);

\path[draw=drawColor,line width= 0.4pt,line join=round,line cap=round,fill=fillColor] (213.27,508.31) circle (  1.16);

\path[draw=drawColor,line width= 0.4pt,line join=round,line cap=round,fill=fillColor] (213.32,508.31) circle (  1.16);

\path[draw=drawColor,line width= 0.4pt,line join=round,line cap=round,fill=fillColor] (213.36,508.31) circle (  1.16);

\path[draw=drawColor,line width= 0.4pt,line join=round,line cap=round,fill=fillColor] (213.41,508.31) circle (  1.16);

\path[draw=drawColor,line width= 0.4pt,line join=round,line cap=round,fill=fillColor] (213.46,508.31) circle (  1.16);

\path[draw=drawColor,line width= 0.4pt,line join=round,line cap=round,fill=fillColor] (213.50,508.31) circle (  1.16);

\path[draw=drawColor,line width= 0.4pt,line join=round,line cap=round,fill=fillColor] (213.55,508.31) circle (  1.16);

\path[draw=drawColor,line width= 0.4pt,line join=round,line cap=round,fill=fillColor] (213.59,508.31) circle (  1.16);

\path[draw=drawColor,line width= 0.4pt,line join=round,line cap=round,fill=fillColor] (213.64,508.31) circle (  1.16);

\path[draw=drawColor,line width= 0.4pt,line join=round,line cap=round,fill=fillColor] (213.68,508.31) circle (  1.16);

\path[draw=drawColor,line width= 0.4pt,line join=round,line cap=round,fill=fillColor] (213.73,508.31) circle (  1.16);

\path[draw=drawColor,line width= 0.4pt,line join=round,line cap=round,fill=fillColor] (213.78,508.31) circle (  1.16);

\path[draw=drawColor,line width= 0.4pt,line join=round,line cap=round,fill=fillColor] (213.82,508.31) circle (  1.16);

\path[draw=drawColor,line width= 0.4pt,line join=round,line cap=round,fill=fillColor] (213.87,508.31) circle (  1.16);

\path[draw=drawColor,line width= 0.4pt,line join=round,line cap=round,fill=fillColor] (213.91,508.31) circle (  1.16);

\path[draw=drawColor,line width= 0.4pt,line join=round,line cap=round,fill=fillColor] (213.96,508.31) circle (  1.16);

\path[draw=drawColor,line width= 0.4pt,line join=round,line cap=round,fill=fillColor] (214.00,508.31) circle (  1.16);

\path[draw=drawColor,line width= 0.4pt,line join=round,line cap=round,fill=fillColor] (214.05,508.31) circle (  1.16);

\path[draw=drawColor,line width= 0.4pt,line join=round,line cap=round,fill=fillColor] (214.09,508.31) circle (  1.16);

\path[draw=drawColor,line width= 0.4pt,line join=round,line cap=round,fill=fillColor] (214.14,508.31) circle (  1.16);

\path[draw=drawColor,line width= 0.4pt,line join=round,line cap=round,fill=fillColor] (214.18,508.31) circle (  1.16);

\path[draw=drawColor,line width= 0.4pt,line join=round,line cap=round,fill=fillColor] (214.23,508.31) circle (  1.16);

\path[draw=drawColor,line width= 0.4pt,line join=round,line cap=round,fill=fillColor] (214.27,508.31) circle (  1.16);

\path[draw=drawColor,line width= 0.4pt,line join=round,line cap=round,fill=fillColor] (214.32,508.31) circle (  1.16);

\path[draw=drawColor,line width= 0.4pt,line join=round,line cap=round,fill=fillColor] (214.36,508.31) circle (  1.16);

\path[draw=drawColor,line width= 0.4pt,line join=round,line cap=round,fill=fillColor] (214.41,508.31) circle (  1.16);

\path[draw=drawColor,line width= 0.4pt,line join=round,line cap=round,fill=fillColor] (214.45,508.31) circle (  1.16);

\path[draw=drawColor,line width= 0.4pt,line join=round,line cap=round,fill=fillColor] (214.50,508.31) circle (  1.16);

\path[draw=drawColor,line width= 0.4pt,line join=round,line cap=round,fill=fillColor] (214.54,508.31) circle (  1.16);

\path[draw=drawColor,line width= 0.4pt,line join=round,line cap=round,fill=fillColor] (214.59,508.31) circle (  1.16);

\path[draw=drawColor,line width= 0.4pt,line join=round,line cap=round,fill=fillColor] (214.63,508.31) circle (  1.16);

\path[draw=drawColor,line width= 0.4pt,line join=round,line cap=round,fill=fillColor] (214.68,508.31) circle (  1.16);

\path[draw=drawColor,line width= 0.4pt,line join=round,line cap=round,fill=fillColor] (214.72,508.31) circle (  1.16);

\path[draw=drawColor,line width= 0.4pt,line join=round,line cap=round,fill=fillColor] (214.77,508.31) circle (  1.16);

\path[draw=drawColor,line width= 0.4pt,line join=round,line cap=round,fill=fillColor] (214.81,508.31) circle (  1.16);

\path[draw=drawColor,line width= 0.4pt,line join=round,line cap=round,fill=fillColor] (214.86,508.31) circle (  1.16);

\path[draw=drawColor,line width= 0.4pt,line join=round,line cap=round,fill=fillColor] (214.90,508.31) circle (  1.16);

\path[draw=drawColor,line width= 0.4pt,line join=round,line cap=round,fill=fillColor] (214.95,508.31) circle (  1.16);

\path[draw=drawColor,line width= 0.4pt,line join=round,line cap=round,fill=fillColor] (214.99,508.31) circle (  1.16);

\path[draw=drawColor,line width= 0.4pt,line join=round,line cap=round,fill=fillColor] (215.04,508.31) circle (  1.16);

\path[draw=drawColor,line width= 0.4pt,line join=round,line cap=round,fill=fillColor] (215.08,508.31) circle (  1.16);

\path[draw=drawColor,line width= 0.4pt,line join=round,line cap=round,fill=fillColor] (215.13,508.31) circle (  1.16);

\path[draw=drawColor,line width= 0.4pt,line join=round,line cap=round,fill=fillColor] (215.17,508.31) circle (  1.16);

\path[draw=drawColor,line width= 0.4pt,line join=round,line cap=round,fill=fillColor] (215.22,508.31) circle (  1.16);

\path[draw=drawColor,line width= 0.4pt,line join=round,line cap=round,fill=fillColor] (215.26,508.31) circle (  1.16);

\path[draw=drawColor,line width= 0.4pt,line join=round,line cap=round,fill=fillColor] (215.31,508.31) circle (  1.16);

\path[draw=drawColor,line width= 0.4pt,line join=round,line cap=round,fill=fillColor] (215.35,508.31) circle (  1.16);

\path[draw=drawColor,line width= 0.4pt,line join=round,line cap=round,fill=fillColor] (215.40,508.31) circle (  1.16);

\path[draw=drawColor,line width= 0.4pt,line join=round,line cap=round,fill=fillColor] (215.44,508.31) circle (  1.16);

\path[draw=drawColor,line width= 0.4pt,line join=round,line cap=round,fill=fillColor] (215.48,508.31) circle (  1.16);

\path[draw=drawColor,line width= 0.4pt,line join=round,line cap=round,fill=fillColor] (215.53,508.31) circle (  1.16);

\path[draw=drawColor,line width= 0.4pt,line join=round,line cap=round,fill=fillColor] (215.57,508.31) circle (  1.16);

\path[draw=drawColor,line width= 0.4pt,line join=round,line cap=round,fill=fillColor] (215.62,508.31) circle (  1.16);

\path[draw=drawColor,line width= 0.4pt,line join=round,line cap=round,fill=fillColor] (215.66,508.31) circle (  1.16);

\path[draw=drawColor,line width= 0.4pt,line join=round,line cap=round,fill=fillColor] (215.71,508.31) circle (  1.16);

\path[draw=drawColor,line width= 0.4pt,line join=round,line cap=round,fill=fillColor] (215.75,508.31) circle (  1.16);

\path[draw=drawColor,line width= 0.4pt,line join=round,line cap=round,fill=fillColor] (215.79,508.31) circle (  1.16);

\path[draw=drawColor,line width= 0.4pt,line join=round,line cap=round,fill=fillColor] (215.84,508.31) circle (  1.16);

\path[draw=drawColor,line width= 0.4pt,line join=round,line cap=round,fill=fillColor] (215.88,508.31) circle (  1.16);

\path[draw=drawColor,line width= 0.4pt,line join=round,line cap=round,fill=fillColor] (215.93,508.31) circle (  1.16);

\path[draw=drawColor,line width= 0.4pt,line join=round,line cap=round,fill=fillColor] (215.97,508.31) circle (  1.16);

\path[draw=drawColor,line width= 0.4pt,line join=round,line cap=round,fill=fillColor] (216.02,508.31) circle (  1.16);

\path[draw=drawColor,line width= 0.4pt,line join=round,line cap=round,fill=fillColor] (216.06,508.31) circle (  1.16);

\path[draw=drawColor,line width= 0.4pt,line join=round,line cap=round,fill=fillColor] (216.10,508.31) circle (  1.16);

\path[draw=drawColor,line width= 0.4pt,line join=round,line cap=round,fill=fillColor] (216.15,508.31) circle (  1.16);

\path[draw=drawColor,line width= 0.4pt,line join=round,line cap=round,fill=fillColor] (216.19,508.31) circle (  1.16);

\path[draw=drawColor,line width= 0.4pt,line join=round,line cap=round,fill=fillColor] (216.24,508.31) circle (  1.16);

\path[draw=drawColor,line width= 0.4pt,line join=round,line cap=round,fill=fillColor] (216.28,508.31) circle (  1.16);

\path[draw=drawColor,line width= 0.4pt,line join=round,line cap=round,fill=fillColor] (216.32,508.31) circle (  1.16);

\path[draw=drawColor,line width= 0.4pt,line join=round,line cap=round,fill=fillColor] (216.37,508.31) circle (  1.16);

\path[draw=drawColor,line width= 0.4pt,line join=round,line cap=round,fill=fillColor] (216.41,508.31) circle (  1.16);

\path[draw=drawColor,line width= 0.4pt,line join=round,line cap=round,fill=fillColor] (216.46,508.31) circle (  1.16);

\path[draw=drawColor,line width= 0.4pt,line join=round,line cap=round,fill=fillColor] (216.50,508.31) circle (  1.16);

\path[draw=drawColor,line width= 0.4pt,line join=round,line cap=round,fill=fillColor] (216.54,508.31) circle (  1.16);

\path[draw=drawColor,line width= 0.4pt,line join=round,line cap=round,fill=fillColor] (216.59,508.31) circle (  1.16);

\path[draw=drawColor,line width= 0.4pt,line join=round,line cap=round,fill=fillColor] (216.63,508.31) circle (  1.16);

\path[draw=drawColor,line width= 0.4pt,line join=round,line cap=round,fill=fillColor] (216.68,508.31) circle (  1.16);

\path[draw=drawColor,line width= 0.4pt,line join=round,line cap=round,fill=fillColor] (216.72,508.31) circle (  1.16);

\path[draw=drawColor,line width= 0.4pt,line join=round,line cap=round,fill=fillColor] (216.76,508.31) circle (  1.16);

\path[draw=drawColor,line width= 0.4pt,line join=round,line cap=round,fill=fillColor] (216.81,508.31) circle (  1.16);

\path[draw=drawColor,line width= 0.4pt,line join=round,line cap=round,fill=fillColor] (216.85,508.31) circle (  1.16);
\definecolor[named]{drawColor}{rgb}{0.00,0.00,0.00}
\definecolor[named]{fillColor}{rgb}{0.00,0.00,0.00}

\path[draw=drawColor,line width= 0.6pt,line join=round,fill=fillColor] ( 74.44,585.15) -- (223.63,585.15);

\node[text=drawColor,anchor=base east,inner sep=0pt, outer sep=0pt, scale=  0.85] at (220.13,584.01) {infeasible solutions};

\path[draw=drawColor,line width= 0.6pt,line join=round,line cap=round] ( 74.44,500.63) rectangle (223.63,592.83);
\end{scope}
\begin{scope}
\path[clip] (  0.00,  0.00) rectangle (505.89,614.29);
\definecolor[named]{drawColor}{rgb}{0.00,0.00,0.00}

\node[text=drawColor,anchor=base east,inner sep=0pt, outer sep=0pt, scale=  0.80] at ( 69.04,505.56) {0.00};

\node[text=drawColor,anchor=base east,inner sep=0pt, outer sep=0pt, scale=  0.80] at ( 69.04,522.11) {0.01};

\node[text=drawColor,anchor=base east,inner sep=0pt, outer sep=0pt, scale=  0.80] at ( 69.04,533.86) {0.05};

\node[text=drawColor,anchor=base east,inner sep=0pt, outer sep=0pt, scale=  0.80] at ( 69.04,550.49) {0.20};

\node[text=drawColor,anchor=base east,inner sep=0pt, outer sep=0pt, scale=  0.80] at ( 69.04,566.54) {0.50};

\node[text=drawColor,anchor=base east,inner sep=0pt, outer sep=0pt, scale=  0.80] at ( 69.04,575.37) {0.75};

\node[text=drawColor,anchor=base east,inner sep=0pt, outer sep=0pt, scale=  0.80] at ( 69.04,582.39) {1.00};
\end{scope}
\begin{scope}
\path[clip] (  0.00,  0.00) rectangle (505.89,614.29);
\definecolor[named]{drawColor}{rgb}{0.00,0.00,0.00}

\path[draw=drawColor,line width= 0.6pt,line join=round] ( 71.44,508.31) --
	( 74.44,508.31);

\path[draw=drawColor,line width= 0.6pt,line join=round] ( 71.44,524.86) --
	( 74.44,524.86);

\path[draw=drawColor,line width= 0.6pt,line join=round] ( 71.44,536.62) --
	( 74.44,536.62);

\path[draw=drawColor,line width= 0.6pt,line join=round] ( 71.44,553.24) --
	( 74.44,553.24);

\path[draw=drawColor,line width= 0.6pt,line join=round] ( 71.44,569.30) --
	( 74.44,569.30);

\path[draw=drawColor,line width= 0.6pt,line join=round] ( 71.44,578.12) --
	( 74.44,578.12);

\path[draw=drawColor,line width= 0.6pt,line join=round] ( 71.44,585.15) --
	( 74.44,585.15);
\end{scope}
\begin{scope}
\path[clip] (  0.00,  0.00) rectangle (505.89,614.29);
\definecolor[named]{drawColor}{rgb}{0.00,0.00,0.00}

\path[draw=drawColor,line width= 0.6pt,line join=round] (150.72,497.63) --
	(150.72,500.63);

\path[draw=drawColor,line width= 0.6pt,line join=round] (172.51,497.63) --
	(172.51,500.63);

\path[draw=drawColor,line width= 0.6pt,line join=round] (187.79,497.63) --
	(187.79,500.63);

\path[draw=drawColor,line width= 0.6pt,line join=round] (199.96,497.63) --
	(199.96,500.63);

\path[draw=drawColor,line width= 0.6pt,line join=round] (210.23,497.63) --
	(210.23,500.63);

\path[draw=drawColor,line width= 0.6pt,line join=round] (219.21,497.63) --
	(219.21,500.63);
\end{scope}
\begin{scope}
\path[clip] (  0.00,  0.00) rectangle (505.89,614.29);
\definecolor[named]{drawColor}{rgb}{0.00,0.00,0.00}

\node[text=drawColor,rotate= 50.00,anchor=base east,inner sep=0pt, outer sep=0pt, scale=  0.80] at (154.94,491.68) {200};

\node[text=drawColor,rotate= 50.00,anchor=base east,inner sep=0pt, outer sep=0pt, scale=  0.80] at (176.73,491.68) {400};

\node[text=drawColor,rotate= 50.00,anchor=base east,inner sep=0pt, outer sep=0pt, scale=  0.80] at (192.01,491.68) {600};

\node[text=drawColor,rotate= 50.00,anchor=base east,inner sep=0pt, outer sep=0pt, scale=  0.80] at (204.18,491.68) {800};

\node[text=drawColor,rotate= 50.00,anchor=base east,inner sep=0pt, outer sep=0pt, scale=  0.80] at (214.45,491.68) {1000};

\node[text=drawColor,rotate= 50.00,anchor=base east,inner sep=0pt, outer sep=0pt, scale=  0.80] at (223.43,491.68) {1200};
\end{scope}
\begin{scope}
\path[clip] (  0.00,  0.00) rectangle (505.89,614.29);
\definecolor[named]{drawColor}{rgb}{0.00,0.00,0.00}

\node[text=drawColor,anchor=base,inner sep=0pt, outer sep=0pt, scale=  0.80] at (149.04,469.12) {\# Instances};
\end{scope}
\begin{scope}
\path[clip] (  0.00,  0.00) rectangle (505.89,614.29);
\definecolor[named]{drawColor}{rgb}{0.00,0.00,0.00}

\node[text=drawColor,rotate= 90.00,anchor=base,inner sep=0pt, outer sep=0pt, scale=  0.80] at ( 44.82,546.73) {1-(Best/Algorithm)};
\end{scope}
\begin{scope}
\path[clip] (  0.00,  0.00) rectangle (505.89,614.29);
\definecolor[named]{drawColor}{rgb}{0.00,0.00,0.00}

\node[text=drawColor,anchor=base,inner sep=0pt, outer sep=0pt, scale=  1.20] at (149.04,600.03) {\ALL};
\end{scope}
\begin{scope}
\path[clip] (271.30,460.72) rectangle (487.53,614.29);
\definecolor[named]{drawColor}{rgb}{1.00,1.00,1.00}
\definecolor[named]{fillColor}{rgb}{1.00,1.00,1.00}

\path[draw=drawColor,line width= 0.6pt,line join=round,line cap=round,fill=fillColor] (271.30,460.72) rectangle (487.53,614.29);
\end{scope}
\begin{scope}
\path[clip] (322.43,494.50) rectangle (481.53,592.83);
\definecolor[named]{fillColor}{rgb}{1.00,1.00,1.00}

\path[fill=fillColor] (322.43,494.50) rectangle (481.53,592.83);
\definecolor[named]{drawColor}{rgb}{0.98,0.98,0.98}

\path[draw=drawColor,line width= 0.6pt,line join=round] (322.43,511.52) --
	(481.53,511.52);

\path[draw=drawColor,line width= 0.6pt,line join=round] (322.43,526.61) --
	(481.53,526.61);

\path[draw=drawColor,line width= 0.6pt,line join=round] (322.43,541.75) --
	(481.53,541.75);

\path[draw=drawColor,line width= 0.6pt,line join=round] (322.43,559.17) --
	(481.53,559.17);

\path[draw=drawColor,line width= 0.6pt,line join=round] (322.43,572.44) --
	(481.53,572.44);

\path[draw=drawColor,line width= 0.6pt,line join=round] (322.43,580.89) --
	(481.53,580.89);

\path[draw=drawColor,line width= 0.6pt,line join=round] (346.57,494.50) --
	(346.57,592.83);

\path[draw=drawColor,line width= 0.6pt,line join=round] (360.73,494.50) --
	(360.73,592.83);

\path[draw=drawColor,line width= 0.6pt,line join=round] (389.03,494.50) --
	(389.03,592.83);

\path[draw=drawColor,line width= 0.6pt,line join=round] (413.11,494.50) --
	(413.11,592.83);

\path[draw=drawColor,line width= 0.6pt,line join=round] (430.95,494.50) --
	(430.95,592.83);

\path[draw=drawColor,line width= 0.6pt,line join=round] (445.52,494.50) --
	(445.52,592.83);

\path[draw=drawColor,line width= 0.6pt,line join=round] (458.03,494.50) --
	(458.03,592.83);

\path[draw=drawColor,line width= 0.6pt,line join=round] (469.08,494.50) --
	(469.08,592.83);
\definecolor[named]{drawColor}{rgb}{0.75,0.75,0.75}

\path[draw=drawColor,line width= 0.6pt,dash pattern=on 1pt off 3pt ,line join=round] (322.43,502.69) --
	(481.53,502.69);

\path[draw=drawColor,line width= 0.6pt,dash pattern=on 1pt off 3pt ,line join=round] (322.43,520.35) --
	(481.53,520.35);

\path[draw=drawColor,line width= 0.6pt,dash pattern=on 1pt off 3pt ,line join=round] (322.43,532.88) --
	(481.53,532.88);

\path[draw=drawColor,line width= 0.6pt,dash pattern=on 1pt off 3pt ,line join=round] (322.43,550.61) --
	(481.53,550.61);

\path[draw=drawColor,line width= 0.6pt,dash pattern=on 1pt off 3pt ,line join=round] (322.43,567.73) --
	(481.53,567.73);

\path[draw=drawColor,line width= 0.6pt,dash pattern=on 1pt off 3pt ,line join=round] (322.43,577.14) --
	(481.53,577.14);

\path[draw=drawColor,line width= 0.6pt,dash pattern=on 1pt off 3pt ,line join=round] (322.43,584.64) --
	(481.53,584.64);

\path[draw=drawColor,line width= 0.6pt,dash pattern=on 1pt off 3pt ,line join=round] (374.88,494.50) --
	(374.88,592.83);

\path[draw=drawColor,line width= 0.6pt,dash pattern=on 1pt off 3pt ,line join=round] (403.19,494.50) --
	(403.19,592.83);

\path[draw=drawColor,line width= 0.6pt,dash pattern=on 1pt off 3pt ,line join=round] (423.04,494.50) --
	(423.04,592.83);

\path[draw=drawColor,line width= 0.6pt,dash pattern=on 1pt off 3pt ,line join=round] (438.85,494.50) --
	(438.85,592.83);

\path[draw=drawColor,line width= 0.6pt,dash pattern=on 1pt off 3pt ,line join=round] (452.20,494.50) --
	(452.20,592.83);

\path[draw=drawColor,line width= 0.6pt,dash pattern=on 1pt off 3pt ,line join=round] (463.87,494.50) --
	(463.87,592.83);

\path[draw=drawColor,line width= 0.6pt,dash pattern=on 1pt off 3pt ,line join=round] (474.30,494.50) --
	(474.30,592.83);
\definecolor[named]{drawColor}{rgb}{0.89,0.10,0.11}
\definecolor[named]{fillColor}{rgb}{0.89,0.10,0.11}

\path[draw=drawColor,line width= 0.4pt,line join=round,line cap=round,fill=fillColor] (329.66,535.38) circle (  1.16);

\path[draw=drawColor,line width= 0.4pt,line join=round,line cap=round,fill=fillColor] (346.22,534.17) circle (  1.16);

\path[draw=drawColor,line width= 0.4pt,line join=round,line cap=round,fill=fillColor] (357.83,533.82) circle (  1.16);

\path[draw=drawColor,line width= 0.4pt,line join=round,line cap=round,fill=fillColor] (367.07,532.21) circle (  1.16);

\path[draw=drawColor,line width= 0.4pt,line join=round,line cap=round,fill=fillColor] (374.88,531.20) circle (  1.16);

\path[draw=drawColor,line width= 0.4pt,line join=round,line cap=round,fill=fillColor] (381.70,530.59) circle (  1.16);

\path[draw=drawColor,line width= 0.4pt,line join=round,line cap=round,fill=fillColor] (387.80,530.55) circle (  1.16);

\path[draw=drawColor,line width= 0.4pt,line join=round,line cap=round,fill=fillColor] (393.35,529.84) circle (  1.16);

\path[draw=drawColor,line width= 0.4pt,line join=round,line cap=round,fill=fillColor] (398.45,529.45) circle (  1.16);

\path[draw=drawColor,line width= 0.4pt,line join=round,line cap=round,fill=fillColor] (403.19,529.29) circle (  1.16);

\path[draw=drawColor,line width= 0.4pt,line join=round,line cap=round,fill=fillColor] (407.61,529.28) circle (  1.16);

\path[draw=drawColor,line width= 0.4pt,line join=round,line cap=round,fill=fillColor] (411.78,528.82) circle (  1.16);

\path[draw=drawColor,line width= 0.4pt,line join=round,line cap=round,fill=fillColor] (415.73,528.80) circle (  1.16);

\path[draw=drawColor,line width= 0.4pt,line join=round,line cap=round,fill=fillColor] (419.47,528.67) circle (  1.16);

\path[draw=drawColor,line width= 0.4pt,line join=round,line cap=round,fill=fillColor] (423.04,528.55) circle (  1.16);

\path[draw=drawColor,line width= 0.4pt,line join=round,line cap=round,fill=fillColor] (426.46,528.32) circle (  1.16);

\path[draw=drawColor,line width= 0.4pt,line join=round,line cap=round,fill=fillColor] (429.73,527.99) circle (  1.16);

\path[draw=drawColor,line width= 0.4pt,line join=round,line cap=round,fill=fillColor] (432.88,527.92) circle (  1.16);

\path[draw=drawColor,line width= 0.4pt,line join=round,line cap=round,fill=fillColor] (435.92,527.75) circle (  1.16);

\path[draw=drawColor,line width= 0.4pt,line join=round,line cap=round,fill=fillColor] (438.85,527.73) circle (  1.16);

\path[draw=drawColor,line width= 0.4pt,line join=round,line cap=round,fill=fillColor] (441.68,527.50) circle (  1.16);

\path[draw=drawColor,line width= 0.4pt,line join=round,line cap=round,fill=fillColor] (444.43,527.28) circle (  1.16);

\path[draw=drawColor,line width= 0.4pt,line join=round,line cap=round,fill=fillColor] (447.09,526.77) circle (  1.16);

\path[draw=drawColor,line width= 0.4pt,line join=round,line cap=round,fill=fillColor] (449.68,526.40) circle (  1.16);

\path[draw=drawColor,line width= 0.4pt,line join=round,line cap=round,fill=fillColor] (452.20,526.16) circle (  1.16);

\path[draw=drawColor,line width= 0.4pt,line join=round,line cap=round,fill=fillColor] (454.65,525.89) circle (  1.16);

\path[draw=drawColor,line width= 0.4pt,line join=round,line cap=round,fill=fillColor] (457.04,524.82) circle (  1.16);

\path[draw=drawColor,line width= 0.4pt,line join=round,line cap=round,fill=fillColor] (459.37,523.24) circle (  1.16);

\path[draw=drawColor,line width= 0.4pt,line join=round,line cap=round,fill=fillColor] (461.64,522.55) circle (  1.16);

\path[draw=drawColor,line width= 0.4pt,line join=round,line cap=round,fill=fillColor] (463.87,521.56) circle (  1.16);

\path[draw=drawColor,line width= 0.4pt,line join=round,line cap=round,fill=fillColor] (466.04,519.74) circle (  1.16);

\path[draw=drawColor,line width= 0.4pt,line join=round,line cap=round,fill=fillColor] (468.17,518.14) circle (  1.16);

\path[draw=drawColor,line width= 0.4pt,line join=round,line cap=round,fill=fillColor] (470.25,518.08) circle (  1.16);

\path[draw=drawColor,line width= 0.4pt,line join=round,line cap=round,fill=fillColor] (472.30,502.69) circle (  1.16);

\path[draw=drawColor,line width= 0.4pt,line join=round,line cap=round,fill=fillColor] (474.30,502.69) circle (  1.16);
\definecolor[named]{drawColor}{rgb}{0.65,0.34,0.16}
\definecolor[named]{fillColor}{rgb}{0.65,0.34,0.16}

\path[draw=drawColor,line width= 0.4pt,line join=round,line cap=round,fill=fillColor] (329.66,520.07) circle (  1.16);

\path[draw=drawColor,line width= 0.4pt,line join=round,line cap=round,fill=fillColor] (346.22,520.06) circle (  1.16);

\path[draw=drawColor,line width= 0.4pt,line join=round,line cap=round,fill=fillColor] (357.83,519.27) circle (  1.16);

\path[draw=drawColor,line width= 0.4pt,line join=round,line cap=round,fill=fillColor] (367.07,518.75) circle (  1.16);

\path[draw=drawColor,line width= 0.4pt,line join=round,line cap=round,fill=fillColor] (374.88,518.26) circle (  1.16);

\path[draw=drawColor,line width= 0.4pt,line join=round,line cap=round,fill=fillColor] (381.70,518.19) circle (  1.16);

\path[draw=drawColor,line width= 0.4pt,line join=round,line cap=round,fill=fillColor] (387.80,518.12) circle (  1.16);

\path[draw=drawColor,line width= 0.4pt,line join=round,line cap=round,fill=fillColor] (393.35,517.99) circle (  1.16);

\path[draw=drawColor,line width= 0.4pt,line join=round,line cap=round,fill=fillColor] (398.45,516.13) circle (  1.16);

\path[draw=drawColor,line width= 0.4pt,line join=round,line cap=round,fill=fillColor] (403.19,515.47) circle (  1.16);

\path[draw=drawColor,line width= 0.4pt,line join=round,line cap=round,fill=fillColor] (407.61,515.33) circle (  1.16);

\path[draw=drawColor,line width= 0.4pt,line join=round,line cap=round,fill=fillColor] (411.78,513.56) circle (  1.16);

\path[draw=drawColor,line width= 0.4pt,line join=round,line cap=round,fill=fillColor] (415.73,513.43) circle (  1.16);

\path[draw=drawColor,line width= 0.4pt,line join=round,line cap=round,fill=fillColor] (419.47,513.25) circle (  1.16);

\path[draw=drawColor,line width= 0.4pt,line join=round,line cap=round,fill=fillColor] (423.04,511.77) circle (  1.16);

\path[draw=drawColor,line width= 0.4pt,line join=round,line cap=round,fill=fillColor] (426.46,511.00) circle (  1.16);

\path[draw=drawColor,line width= 0.4pt,line join=round,line cap=round,fill=fillColor] (429.73,510.38) circle (  1.16);

\path[draw=drawColor,line width= 0.4pt,line join=round,line cap=round,fill=fillColor] (432.88,510.34) circle (  1.16);

\path[draw=drawColor,line width= 0.4pt,line join=round,line cap=round,fill=fillColor] (435.92,509.78) circle (  1.16);

\path[draw=drawColor,line width= 0.4pt,line join=round,line cap=round,fill=fillColor] (438.85,509.36) circle (  1.16);

\path[draw=drawColor,line width= 0.4pt,line join=round,line cap=round,fill=fillColor] (441.68,508.21) circle (  1.16);

\path[draw=drawColor,line width= 0.4pt,line join=round,line cap=round,fill=fillColor] (444.43,506.23) circle (  1.16);

\path[draw=drawColor,line width= 0.4pt,line join=round,line cap=round,fill=fillColor] (447.09,502.69) circle (  1.16);

\path[draw=drawColor,line width= 0.4pt,line join=round,line cap=round,fill=fillColor] (449.68,502.69) circle (  1.16);

\path[draw=drawColor,line width= 0.4pt,line join=round,line cap=round,fill=fillColor] (452.20,502.69) circle (  1.16);

\path[draw=drawColor,line width= 0.4pt,line join=round,line cap=round,fill=fillColor] (454.65,502.69) circle (  1.16);

\path[draw=drawColor,line width= 0.4pt,line join=round,line cap=round,fill=fillColor] (457.04,502.69) circle (  1.16);

\path[draw=drawColor,line width= 0.4pt,line join=round,line cap=round,fill=fillColor] (459.37,502.69) circle (  1.16);

\path[draw=drawColor,line width= 0.4pt,line join=round,line cap=round,fill=fillColor] (461.64,502.69) circle (  1.16);

\path[draw=drawColor,line width= 0.4pt,line join=round,line cap=round,fill=fillColor] (463.87,502.69) circle (  1.16);

\path[draw=drawColor,line width= 0.4pt,line join=round,line cap=round,fill=fillColor] (466.04,502.69) circle (  1.16);

\path[draw=drawColor,line width= 0.4pt,line join=round,line cap=round,fill=fillColor] (468.17,502.69) circle (  1.16);

\path[draw=drawColor,line width= 0.4pt,line join=round,line cap=round,fill=fillColor] (470.25,502.69) circle (  1.16);

\path[draw=drawColor,line width= 0.4pt,line join=round,line cap=round,fill=fillColor] (472.30,502.69) circle (  1.16);

\path[draw=drawColor,line width= 0.4pt,line join=round,line cap=round,fill=fillColor] (474.30,502.69) circle (  1.16);
\definecolor[named]{drawColor}{rgb}{0.22,0.49,0.72}
\definecolor[named]{fillColor}{rgb}{0.22,0.49,0.72}

\path[draw=drawColor,line width= 0.4pt,line join=round,line cap=round,fill=fillColor] (329.66,527.47) circle (  1.16);

\path[draw=drawColor,line width= 0.4pt,line join=round,line cap=round,fill=fillColor] (346.22,527.04) circle (  1.16);

\path[draw=drawColor,line width= 0.4pt,line join=round,line cap=round,fill=fillColor] (357.83,521.11) circle (  1.16);

\path[draw=drawColor,line width= 0.4pt,line join=round,line cap=round,fill=fillColor] (367.07,520.51) circle (  1.16);

\path[draw=drawColor,line width= 0.4pt,line join=round,line cap=round,fill=fillColor] (374.88,519.00) circle (  1.16);

\path[draw=drawColor,line width= 0.4pt,line join=round,line cap=round,fill=fillColor] (381.70,518.49) circle (  1.16);

\path[draw=drawColor,line width= 0.4pt,line join=round,line cap=round,fill=fillColor] (387.80,517.50) circle (  1.16);

\path[draw=drawColor,line width= 0.4pt,line join=round,line cap=round,fill=fillColor] (393.35,516.21) circle (  1.16);

\path[draw=drawColor,line width= 0.4pt,line join=round,line cap=round,fill=fillColor] (398.45,516.21) circle (  1.16);

\path[draw=drawColor,line width= 0.4pt,line join=round,line cap=round,fill=fillColor] (403.19,516.10) circle (  1.16);

\path[draw=drawColor,line width= 0.4pt,line join=round,line cap=round,fill=fillColor] (407.61,515.66) circle (  1.16);

\path[draw=drawColor,line width= 0.4pt,line join=round,line cap=round,fill=fillColor] (411.78,515.01) circle (  1.16);

\path[draw=drawColor,line width= 0.4pt,line join=round,line cap=round,fill=fillColor] (415.73,513.88) circle (  1.16);

\path[draw=drawColor,line width= 0.4pt,line join=round,line cap=round,fill=fillColor] (419.47,513.50) circle (  1.16);

\path[draw=drawColor,line width= 0.4pt,line join=round,line cap=round,fill=fillColor] (423.04,513.09) circle (  1.16);

\path[draw=drawColor,line width= 0.4pt,line join=round,line cap=round,fill=fillColor] (426.46,512.71) circle (  1.16);

\path[draw=drawColor,line width= 0.4pt,line join=round,line cap=round,fill=fillColor] (429.73,512.35) circle (  1.16);

\path[draw=drawColor,line width= 0.4pt,line join=round,line cap=round,fill=fillColor] (432.88,510.34) circle (  1.16);

\path[draw=drawColor,line width= 0.4pt,line join=round,line cap=round,fill=fillColor] (435.92,509.79) circle (  1.16);

\path[draw=drawColor,line width= 0.4pt,line join=round,line cap=round,fill=fillColor] (438.85,509.63) circle (  1.16);

\path[draw=drawColor,line width= 0.4pt,line join=round,line cap=round,fill=fillColor] (441.68,509.56) circle (  1.16);

\path[draw=drawColor,line width= 0.4pt,line join=round,line cap=round,fill=fillColor] (444.43,509.36) circle (  1.16);

\path[draw=drawColor,line width= 0.4pt,line join=round,line cap=round,fill=fillColor] (447.09,507.62) circle (  1.16);

\path[draw=drawColor,line width= 0.4pt,line join=round,line cap=round,fill=fillColor] (449.68,502.69) circle (  1.16);

\path[draw=drawColor,line width= 0.4pt,line join=round,line cap=round,fill=fillColor] (452.20,502.69) circle (  1.16);

\path[draw=drawColor,line width= 0.4pt,line join=round,line cap=round,fill=fillColor] (454.65,502.69) circle (  1.16);

\path[draw=drawColor,line width= 0.4pt,line join=round,line cap=round,fill=fillColor] (457.04,502.69) circle (  1.16);

\path[draw=drawColor,line width= 0.4pt,line join=round,line cap=round,fill=fillColor] (459.37,502.69) circle (  1.16);

\path[draw=drawColor,line width= 0.4pt,line join=round,line cap=round,fill=fillColor] (461.64,502.69) circle (  1.16);

\path[draw=drawColor,line width= 0.4pt,line join=round,line cap=round,fill=fillColor] (463.87,502.69) circle (  1.16);

\path[draw=drawColor,line width= 0.4pt,line join=round,line cap=round,fill=fillColor] (466.04,502.69) circle (  1.16);

\path[draw=drawColor,line width= 0.4pt,line join=round,line cap=round,fill=fillColor] (468.17,502.69) circle (  1.16);

\path[draw=drawColor,line width= 0.4pt,line join=round,line cap=round,fill=fillColor] (470.25,502.69) circle (  1.16);

\path[draw=drawColor,line width= 0.4pt,line join=round,line cap=round,fill=fillColor] (472.30,502.69) circle (  1.16);

\path[draw=drawColor,line width= 0.4pt,line join=round,line cap=round,fill=fillColor] (474.30,502.69) circle (  1.16);
\definecolor[named]{drawColor}{rgb}{0.30,0.69,0.29}
\definecolor[named]{fillColor}{rgb}{0.30,0.69,0.29}

\path[draw=drawColor,line width= 0.4pt,line join=round,line cap=round,fill=fillColor] (329.66,527.53) circle (  1.16);

\path[draw=drawColor,line width= 0.4pt,line join=round,line cap=round,fill=fillColor] (346.22,526.15) circle (  1.16);

\path[draw=drawColor,line width= 0.4pt,line join=round,line cap=round,fill=fillColor] (357.83,524.98) circle (  1.16);

\path[draw=drawColor,line width= 0.4pt,line join=round,line cap=round,fill=fillColor] (367.07,519.24) circle (  1.16);

\path[draw=drawColor,line width= 0.4pt,line join=round,line cap=round,fill=fillColor] (374.88,518.05) circle (  1.16);

\path[draw=drawColor,line width= 0.4pt,line join=round,line cap=round,fill=fillColor] (381.70,517.94) circle (  1.16);

\path[draw=drawColor,line width= 0.4pt,line join=round,line cap=round,fill=fillColor] (387.80,517.40) circle (  1.16);

\path[draw=drawColor,line width= 0.4pt,line join=round,line cap=round,fill=fillColor] (393.35,517.09) circle (  1.16);

\path[draw=drawColor,line width= 0.4pt,line join=round,line cap=round,fill=fillColor] (398.45,516.19) circle (  1.16);

\path[draw=drawColor,line width= 0.4pt,line join=round,line cap=round,fill=fillColor] (403.19,515.45) circle (  1.16);

\path[draw=drawColor,line width= 0.4pt,line join=round,line cap=round,fill=fillColor] (407.61,515.19) circle (  1.16);

\path[draw=drawColor,line width= 0.4pt,line join=round,line cap=round,fill=fillColor] (411.78,514.14) circle (  1.16);

\path[draw=drawColor,line width= 0.4pt,line join=round,line cap=round,fill=fillColor] (415.73,513.71) circle (  1.16);

\path[draw=drawColor,line width= 0.4pt,line join=round,line cap=round,fill=fillColor] (419.47,513.63) circle (  1.16);

\path[draw=drawColor,line width= 0.4pt,line join=round,line cap=round,fill=fillColor] (423.04,513.55) circle (  1.16);

\path[draw=drawColor,line width= 0.4pt,line join=round,line cap=round,fill=fillColor] (426.46,513.24) circle (  1.16);

\path[draw=drawColor,line width= 0.4pt,line join=round,line cap=round,fill=fillColor] (429.73,512.54) circle (  1.16);

\path[draw=drawColor,line width= 0.4pt,line join=round,line cap=round,fill=fillColor] (432.88,512.52) circle (  1.16);

\path[draw=drawColor,line width= 0.4pt,line join=round,line cap=round,fill=fillColor] (435.92,512.23) circle (  1.16);

\path[draw=drawColor,line width= 0.4pt,line join=round,line cap=round,fill=fillColor] (438.85,511.76) circle (  1.16);

\path[draw=drawColor,line width= 0.4pt,line join=round,line cap=round,fill=fillColor] (441.68,511.75) circle (  1.16);

\path[draw=drawColor,line width= 0.4pt,line join=round,line cap=round,fill=fillColor] (444.43,510.34) circle (  1.16);

\path[draw=drawColor,line width= 0.4pt,line join=round,line cap=round,fill=fillColor] (447.09,510.15) circle (  1.16);

\path[draw=drawColor,line width= 0.4pt,line join=round,line cap=round,fill=fillColor] (449.68,509.56) circle (  1.16);

\path[draw=drawColor,line width= 0.4pt,line join=round,line cap=round,fill=fillColor] (452.20,509.36) circle (  1.16);

\path[draw=drawColor,line width= 0.4pt,line join=round,line cap=round,fill=fillColor] (454.65,507.62) circle (  1.16);

\path[draw=drawColor,line width= 0.4pt,line join=round,line cap=round,fill=fillColor] (457.04,502.69) circle (  1.16);

\path[draw=drawColor,line width= 0.4pt,line join=round,line cap=round,fill=fillColor] (459.37,502.69) circle (  1.16);

\path[draw=drawColor,line width= 0.4pt,line join=round,line cap=round,fill=fillColor] (461.64,502.69) circle (  1.16);

\path[draw=drawColor,line width= 0.4pt,line join=round,line cap=round,fill=fillColor] (463.87,502.69) circle (  1.16);

\path[draw=drawColor,line width= 0.4pt,line join=round,line cap=round,fill=fillColor] (466.04,502.69) circle (  1.16);

\path[draw=drawColor,line width= 0.4pt,line join=round,line cap=round,fill=fillColor] (468.17,502.69) circle (  1.16);

\path[draw=drawColor,line width= 0.4pt,line join=round,line cap=round,fill=fillColor] (470.25,502.69) circle (  1.16);

\path[draw=drawColor,line width= 0.4pt,line join=round,line cap=round,fill=fillColor] (472.30,502.69) circle (  1.16);

\path[draw=drawColor,line width= 0.4pt,line join=round,line cap=round,fill=fillColor] (474.30,502.69) circle (  1.16);
\definecolor[named]{drawColor}{rgb}{0.60,0.31,0.64}
\definecolor[named]{fillColor}{rgb}{0.60,0.31,0.64}

\path[draw=drawColor,line width= 0.4pt,line join=round,line cap=round,fill=fillColor] (329.66,525.48) circle (  1.16);

\path[draw=drawColor,line width= 0.4pt,line join=round,line cap=round,fill=fillColor] (346.22,524.03) circle (  1.16);

\path[draw=drawColor,line width= 0.4pt,line join=round,line cap=round,fill=fillColor] (357.83,522.99) circle (  1.16);

\path[draw=drawColor,line width= 0.4pt,line join=round,line cap=round,fill=fillColor] (367.07,522.37) circle (  1.16);

\path[draw=drawColor,line width= 0.4pt,line join=round,line cap=round,fill=fillColor] (374.88,521.37) circle (  1.16);

\path[draw=drawColor,line width= 0.4pt,line join=round,line cap=round,fill=fillColor] (381.70,519.84) circle (  1.16);

\path[draw=drawColor,line width= 0.4pt,line join=round,line cap=round,fill=fillColor] (387.80,517.40) circle (  1.16);

\path[draw=drawColor,line width= 0.4pt,line join=round,line cap=round,fill=fillColor] (393.35,517.13) circle (  1.16);

\path[draw=drawColor,line width= 0.4pt,line join=round,line cap=round,fill=fillColor] (398.45,516.84) circle (  1.16);

\path[draw=drawColor,line width= 0.4pt,line join=round,line cap=round,fill=fillColor] (403.19,516.67) circle (  1.16);

\path[draw=drawColor,line width= 0.4pt,line join=round,line cap=round,fill=fillColor] (407.61,516.27) circle (  1.16);

\path[draw=drawColor,line width= 0.4pt,line join=round,line cap=round,fill=fillColor] (411.78,516.25) circle (  1.16);

\path[draw=drawColor,line width= 0.4pt,line join=round,line cap=round,fill=fillColor] (415.73,516.14) circle (  1.16);

\path[draw=drawColor,line width= 0.4pt,line join=round,line cap=round,fill=fillColor] (419.47,515.59) circle (  1.16);

\path[draw=drawColor,line width= 0.4pt,line join=round,line cap=round,fill=fillColor] (423.04,515.55) circle (  1.16);

\path[draw=drawColor,line width= 0.4pt,line join=round,line cap=round,fill=fillColor] (426.46,515.04) circle (  1.16);

\path[draw=drawColor,line width= 0.4pt,line join=round,line cap=round,fill=fillColor] (429.73,513.49) circle (  1.16);

\path[draw=drawColor,line width= 0.4pt,line join=round,line cap=round,fill=fillColor] (432.88,512.64) circle (  1.16);

\path[draw=drawColor,line width= 0.4pt,line join=round,line cap=round,fill=fillColor] (435.92,512.64) circle (  1.16);

\path[draw=drawColor,line width= 0.4pt,line join=round,line cap=round,fill=fillColor] (438.85,511.15) circle (  1.16);

\path[draw=drawColor,line width= 0.4pt,line join=round,line cap=round,fill=fillColor] (441.68,511.05) circle (  1.16);

\path[draw=drawColor,line width= 0.4pt,line join=round,line cap=round,fill=fillColor] (444.43,508.60) circle (  1.16);

\path[draw=drawColor,line width= 0.4pt,line join=round,line cap=round,fill=fillColor] (447.09,507.14) circle (  1.16);

\path[draw=drawColor,line width= 0.4pt,line join=round,line cap=round,fill=fillColor] (449.68,502.69) circle (  1.16);

\path[draw=drawColor,line width= 0.4pt,line join=round,line cap=round,fill=fillColor] (452.20,502.69) circle (  1.16);

\path[draw=drawColor,line width= 0.4pt,line join=round,line cap=round,fill=fillColor] (454.65,502.69) circle (  1.16);

\path[draw=drawColor,line width= 0.4pt,line join=round,line cap=round,fill=fillColor] (457.04,502.69) circle (  1.16);

\path[draw=drawColor,line width= 0.4pt,line join=round,line cap=round,fill=fillColor] (459.37,502.69) circle (  1.16);

\path[draw=drawColor,line width= 0.4pt,line join=round,line cap=round,fill=fillColor] (461.64,502.69) circle (  1.16);

\path[draw=drawColor,line width= 0.4pt,line join=round,line cap=round,fill=fillColor] (463.87,502.69) circle (  1.16);

\path[draw=drawColor,line width= 0.4pt,line join=round,line cap=round,fill=fillColor] (466.04,502.69) circle (  1.16);

\path[draw=drawColor,line width= 0.4pt,line join=round,line cap=round,fill=fillColor] (468.17,502.69) circle (  1.16);

\path[draw=drawColor,line width= 0.4pt,line join=round,line cap=round,fill=fillColor] (470.25,502.69) circle (  1.16);

\path[draw=drawColor,line width= 0.4pt,line join=round,line cap=round,fill=fillColor] (472.30,502.69) circle (  1.16);

\path[draw=drawColor,line width= 0.4pt,line join=round,line cap=round,fill=fillColor] (474.30,502.69) circle (  1.16);
\definecolor[named]{drawColor}{rgb}{0.00,0.00,0.00}
\definecolor[named]{fillColor}{rgb}{0.00,0.00,0.00}

\path[draw=drawColor,line width= 0.6pt,line join=round,fill=fillColor] (322.43,584.64) -- (481.53,584.64);

\node[text=drawColor,anchor=base east,inner sep=0pt, outer sep=0pt, scale=  0.85] at (478.03,584.01) {infeasible solutions};

\path[draw=drawColor,line width= 0.6pt,line join=round,line cap=round] (322.43,494.50) rectangle (481.53,592.83);
\end{scope}
\begin{scope}
\path[clip] (  0.00,  0.00) rectangle (505.89,614.29);
\definecolor[named]{drawColor}{rgb}{0.00,0.00,0.00}

\node[text=drawColor,anchor=base east,inner sep=0pt, outer sep=0pt, scale=  0.80] at (317.03,499.94) {0.00};

\node[text=drawColor,anchor=base east,inner sep=0pt, outer sep=0pt, scale=  0.80] at (317.03,517.59) {0.01};

\node[text=drawColor,anchor=base east,inner sep=0pt, outer sep=0pt, scale=  0.80] at (317.03,530.13) {0.05};

\node[text=drawColor,anchor=base east,inner sep=0pt, outer sep=0pt, scale=  0.80] at (317.03,547.86) {0.20};

\node[text=drawColor,anchor=base east,inner sep=0pt, outer sep=0pt, scale=  0.80] at (317.03,564.98) {0.50};

\node[text=drawColor,anchor=base east,inner sep=0pt, outer sep=0pt, scale=  0.80] at (317.03,574.39) {0.75};

\node[text=drawColor,anchor=base east,inner sep=0pt, outer sep=0pt, scale=  0.80] at (317.03,581.88) {1.00};
\end{scope}
\begin{scope}
\path[clip] (  0.00,  0.00) rectangle (505.89,614.29);
\definecolor[named]{drawColor}{rgb}{0.00,0.00,0.00}

\path[draw=drawColor,line width= 0.6pt,line join=round] (319.43,502.69) --
	(322.43,502.69);

\path[draw=drawColor,line width= 0.6pt,line join=round] (319.43,520.35) --
	(322.43,520.35);

\path[draw=drawColor,line width= 0.6pt,line join=round] (319.43,532.88) --
	(322.43,532.88);

\path[draw=drawColor,line width= 0.6pt,line join=round] (319.43,550.61) --
	(322.43,550.61);

\path[draw=drawColor,line width= 0.6pt,line join=round] (319.43,567.73) --
	(322.43,567.73);

\path[draw=drawColor,line width= 0.6pt,line join=round] (319.43,577.14) --
	(322.43,577.14);

\path[draw=drawColor,line width= 0.6pt,line join=round] (319.43,584.64) --
	(322.43,584.64);
\end{scope}
\begin{scope}
\path[clip] (  0.00,  0.00) rectangle (505.89,614.29);
\definecolor[named]{drawColor}{rgb}{0.00,0.00,0.00}

\path[draw=drawColor,line width= 0.6pt,line join=round] (374.88,491.50) --
	(374.88,494.50);

\path[draw=drawColor,line width= 0.6pt,line join=round] (403.19,491.50) --
	(403.19,494.50);

\path[draw=drawColor,line width= 0.6pt,line join=round] (423.04,491.50) --
	(423.04,494.50);

\path[draw=drawColor,line width= 0.6pt,line join=round] (438.85,491.50) --
	(438.85,494.50);

\path[draw=drawColor,line width= 0.6pt,line join=round] (452.20,491.50) --
	(452.20,494.50);

\path[draw=drawColor,line width= 0.6pt,line join=round] (463.87,491.50) --
	(463.87,494.50);

\path[draw=drawColor,line width= 0.6pt,line join=round] (474.30,491.50) --
	(474.30,494.50);
\end{scope}
\begin{scope}
\path[clip] (  0.00,  0.00) rectangle (505.89,614.29);
\definecolor[named]{drawColor}{rgb}{0.00,0.00,0.00}

\node[text=drawColor,rotate= 50.00,anchor=base east,inner sep=0pt, outer sep=0pt, scale=  0.80] at (379.10,485.56) {5};

\node[text=drawColor,rotate= 50.00,anchor=base east,inner sep=0pt, outer sep=0pt, scale=  0.80] at (407.41,485.56) {10};

\node[text=drawColor,rotate= 50.00,anchor=base east,inner sep=0pt, outer sep=0pt, scale=  0.80] at (427.26,485.56) {15};

\node[text=drawColor,rotate= 50.00,anchor=base east,inner sep=0pt, outer sep=0pt, scale=  0.80] at (443.07,485.56) {20};

\node[text=drawColor,rotate= 50.00,anchor=base east,inner sep=0pt, outer sep=0pt, scale=  0.80] at (456.42,485.56) {25};

\node[text=drawColor,rotate= 50.00,anchor=base east,inner sep=0pt, outer sep=0pt, scale=  0.80] at (468.09,485.56) {30};

\node[text=drawColor,rotate= 50.00,anchor=base east,inner sep=0pt, outer sep=0pt, scale=  0.80] at (478.52,485.56) {35};
\end{scope}
\begin{scope}
\path[clip] (  0.00,  0.00) rectangle (505.89,614.29);
\definecolor[named]{drawColor}{rgb}{0.00,0.00,0.00}

\node[text=drawColor,anchor=base,inner sep=0pt, outer sep=0pt, scale=  0.80] at (401.98,469.12) {\# Instances};
\end{scope}
\begin{scope}
\path[clip] (  0.00,  0.00) rectangle (505.89,614.29);
\definecolor[named]{drawColor}{rgb}{0.00,0.00,0.00}

\node[text=drawColor,rotate= 90.00,anchor=base,inner sep=0pt, outer sep=0pt, scale=  0.80] at (292.81,543.66) {1-(Best/Algorithm)};
\end{scope}
\begin{scope}
\path[clip] (  0.00,  0.00) rectangle (505.89,614.29);
\definecolor[named]{drawColor}{rgb}{0.00,0.00,0.00}

\node[text=drawColor,anchor=base,inner sep=0pt, outer sep=0pt, scale=  1.20] at (401.98,600.03) {\DAC};
\end{scope}
\begin{scope}
\path[clip] ( 18.36,307.15) rectangle (234.59,460.72);
\definecolor[named]{drawColor}{rgb}{1.00,1.00,1.00}
\definecolor[named]{fillColor}{rgb}{1.00,1.00,1.00}

\path[draw=drawColor,line width= 0.6pt,line join=round,line cap=round,fill=fillColor] ( 18.36,307.15) rectangle (234.59,460.72);
\end{scope}
\begin{scope}
\path[clip] ( 69.49,340.93) rectangle (228.59,439.26);
\definecolor[named]{fillColor}{rgb}{1.00,1.00,1.00}

\path[fill=fillColor] ( 69.49,340.93) rectangle (228.59,439.26);
\definecolor[named]{drawColor}{rgb}{0.98,0.98,0.98}

\path[draw=drawColor,line width= 0.6pt,line join=round] ( 69.49,357.95) --
	(228.59,357.95);

\path[draw=drawColor,line width= 0.6pt,line join=round] ( 69.49,373.04) --
	(228.59,373.04);

\path[draw=drawColor,line width= 0.6pt,line join=round] ( 69.49,388.17) --
	(228.59,388.17);

\path[draw=drawColor,line width= 0.6pt,line join=round] ( 69.49,405.60) --
	(228.59,405.60);

\path[draw=drawColor,line width= 0.6pt,line join=round] ( 69.49,418.86) --
	(228.59,418.86);

\path[draw=drawColor,line width= 0.6pt,line join=round] ( 69.49,427.32) --
	(228.59,427.32);

\path[draw=drawColor,line width= 0.6pt,line join=round] (104.26,340.93) --
	(104.26,439.26);

\path[draw=drawColor,line width= 0.6pt,line join=round] (117.24,340.93) --
	(117.24,439.26);

\path[draw=drawColor,line width= 0.6pt,line join=round] (143.19,340.93) --
	(143.19,439.26);

\path[draw=drawColor,line width= 0.6pt,line join=round] (165.26,340.93) --
	(165.26,439.26);

\path[draw=drawColor,line width= 0.6pt,line join=round] (181.61,340.93) --
	(181.61,439.26);

\path[draw=drawColor,line width= 0.6pt,line join=round] (194.98,340.93) --
	(194.98,439.26);

\path[draw=drawColor,line width= 0.6pt,line join=round] (206.44,340.93) --
	(206.44,439.26);

\path[draw=drawColor,line width= 0.6pt,line join=round] (216.57,340.93) --
	(216.57,439.26);
\definecolor[named]{drawColor}{rgb}{0.75,0.75,0.75}

\path[draw=drawColor,line width= 0.6pt,dash pattern=on 1pt off 3pt ,line join=round] ( 69.49,349.12) --
	(228.59,349.12);

\path[draw=drawColor,line width= 0.6pt,dash pattern=on 1pt off 3pt ,line join=round] ( 69.49,366.77) --
	(228.59,366.77);

\path[draw=drawColor,line width= 0.6pt,dash pattern=on 1pt off 3pt ,line join=round] ( 69.49,379.31) --
	(228.59,379.31);

\path[draw=drawColor,line width= 0.6pt,dash pattern=on 1pt off 3pt ,line join=round] ( 69.49,397.04) --
	(228.59,397.04);

\path[draw=drawColor,line width= 0.6pt,dash pattern=on 1pt off 3pt ,line join=round] ( 69.49,414.16) --
	(228.59,414.16);

\path[draw=drawColor,line width= 0.6pt,dash pattern=on 1pt off 3pt ,line join=round] ( 69.49,423.57) --
	(228.59,423.57);

\path[draw=drawColor,line width= 0.6pt,dash pattern=on 1pt off 3pt ,line join=round] ( 69.49,431.06) --
	(228.59,431.06);

\path[draw=drawColor,line width= 0.6pt,dash pattern=on 1pt off 3pt ,line join=round] (130.21,340.93) --
	(130.21,439.26);

\path[draw=drawColor,line width= 0.6pt,dash pattern=on 1pt off 3pt ,line join=round] (156.16,340.93) --
	(156.16,439.26);

\path[draw=drawColor,line width= 0.6pt,dash pattern=on 1pt off 3pt ,line join=round] (174.37,340.93) --
	(174.37,439.26);

\path[draw=drawColor,line width= 0.6pt,dash pattern=on 1pt off 3pt ,line join=round] (188.86,340.93) --
	(188.86,439.26);

\path[draw=drawColor,line width= 0.6pt,dash pattern=on 1pt off 3pt ,line join=round] (201.09,340.93) --
	(201.09,439.26);

\path[draw=drawColor,line width= 0.6pt,dash pattern=on 1pt off 3pt ,line join=round] (211.79,340.93) --
	(211.79,439.26);

\path[draw=drawColor,line width= 0.6pt,dash pattern=on 1pt off 3pt ,line join=round] (221.36,340.93) --
	(221.36,439.26);
\definecolor[named]{drawColor}{rgb}{0.89,0.10,0.11}
\definecolor[named]{fillColor}{rgb}{0.89,0.10,0.11}

\path[draw=drawColor,line width= 0.4pt,line join=round,line cap=round,fill=fillColor] ( 76.72,378.54) circle (  1.16);

\path[draw=drawColor,line width= 0.4pt,line join=round,line cap=round,fill=fillColor] ( 88.76,377.45) circle (  1.16);

\path[draw=drawColor,line width= 0.4pt,line join=round,line cap=round,fill=fillColor] ( 97.21,377.28) circle (  1.16);

\path[draw=drawColor,line width= 0.4pt,line join=round,line cap=round,fill=fillColor] (103.94,377.19) circle (  1.16);

\path[draw=drawColor,line width= 0.4pt,line join=round,line cap=round,fill=fillColor] (109.62,377.16) circle (  1.16);

\path[draw=drawColor,line width= 0.4pt,line join=round,line cap=round,fill=fillColor] (114.58,377.12) circle (  1.16);

\path[draw=drawColor,line width= 0.4pt,line join=round,line cap=round,fill=fillColor] (119.02,376.83) circle (  1.16);

\path[draw=drawColor,line width= 0.4pt,line join=round,line cap=round,fill=fillColor] (123.06,376.73) circle (  1.16);

\path[draw=drawColor,line width= 0.4pt,line join=round,line cap=round,fill=fillColor] (126.77,376.52) circle (  1.16);

\path[draw=drawColor,line width= 0.4pt,line join=round,line cap=round,fill=fillColor] (130.21,375.92) circle (  1.16);

\path[draw=drawColor,line width= 0.4pt,line join=round,line cap=round,fill=fillColor] (133.44,375.78) circle (  1.16);

\path[draw=drawColor,line width= 0.4pt,line join=round,line cap=round,fill=fillColor] (136.47,375.54) circle (  1.16);

\path[draw=drawColor,line width= 0.4pt,line join=round,line cap=round,fill=fillColor] (139.34,375.48) circle (  1.16);

\path[draw=drawColor,line width= 0.4pt,line join=round,line cap=round,fill=fillColor] (142.06,375.38) circle (  1.16);

\path[draw=drawColor,line width= 0.4pt,line join=round,line cap=round,fill=fillColor] (144.66,375.21) circle (  1.16);

\path[draw=drawColor,line width= 0.4pt,line join=round,line cap=round,fill=fillColor] (147.15,374.98) circle (  1.16);

\path[draw=drawColor,line width= 0.4pt,line join=round,line cap=round,fill=fillColor] (149.53,374.61) circle (  1.16);

\path[draw=drawColor,line width= 0.4pt,line join=round,line cap=round,fill=fillColor] (151.82,374.57) circle (  1.16);

\path[draw=drawColor,line width= 0.4pt,line join=round,line cap=round,fill=fillColor] (154.03,374.41) circle (  1.16);

\path[draw=drawColor,line width= 0.4pt,line join=round,line cap=round,fill=fillColor] (156.16,374.16) circle (  1.16);

\path[draw=drawColor,line width= 0.4pt,line join=round,line cap=round,fill=fillColor] (158.23,373.92) circle (  1.16);

\path[draw=drawColor,line width= 0.4pt,line join=round,line cap=round,fill=fillColor] (160.22,373.88) circle (  1.16);

\path[draw=drawColor,line width= 0.4pt,line join=round,line cap=round,fill=fillColor] (162.16,373.84) circle (  1.16);

\path[draw=drawColor,line width= 0.4pt,line join=round,line cap=round,fill=fillColor] (164.04,373.78) circle (  1.16);

\path[draw=drawColor,line width= 0.4pt,line join=round,line cap=round,fill=fillColor] (165.88,373.69) circle (  1.16);

\path[draw=drawColor,line width= 0.4pt,line join=round,line cap=round,fill=fillColor] (167.66,373.66) circle (  1.16);

\path[draw=drawColor,line width= 0.4pt,line join=round,line cap=round,fill=fillColor] (169.40,373.62) circle (  1.16);

\path[draw=drawColor,line width= 0.4pt,line join=round,line cap=round,fill=fillColor] (171.09,373.60) circle (  1.16);

\path[draw=drawColor,line width= 0.4pt,line join=round,line cap=round,fill=fillColor] (172.75,373.55) circle (  1.16);

\path[draw=drawColor,line width= 0.4pt,line join=round,line cap=round,fill=fillColor] (174.37,373.51) circle (  1.16);

\path[draw=drawColor,line width= 0.4pt,line join=round,line cap=round,fill=fillColor] (175.95,373.41) circle (  1.16);

\path[draw=drawColor,line width= 0.4pt,line join=round,line cap=round,fill=fillColor] (177.50,373.14) circle (  1.16);

\path[draw=drawColor,line width= 0.4pt,line join=round,line cap=round,fill=fillColor] (179.01,373.07) circle (  1.16);

\path[draw=drawColor,line width= 0.4pt,line join=round,line cap=round,fill=fillColor] (180.50,372.88) circle (  1.16);

\path[draw=drawColor,line width= 0.4pt,line join=round,line cap=round,fill=fillColor] (181.96,372.83) circle (  1.16);

\path[draw=drawColor,line width= 0.4pt,line join=round,line cap=round,fill=fillColor] (183.39,372.68) circle (  1.16);

\path[draw=drawColor,line width= 0.4pt,line join=round,line cap=round,fill=fillColor] (184.79,372.46) circle (  1.16);

\path[draw=drawColor,line width= 0.4pt,line join=round,line cap=round,fill=fillColor] (186.17,372.46) circle (  1.16);

\path[draw=drawColor,line width= 0.4pt,line join=round,line cap=round,fill=fillColor] (187.52,372.34) circle (  1.16);

\path[draw=drawColor,line width= 0.4pt,line join=round,line cap=round,fill=fillColor] (188.86,372.33) circle (  1.16);

\path[draw=drawColor,line width= 0.4pt,line join=round,line cap=round,fill=fillColor] (190.17,372.08) circle (  1.16);

\path[draw=drawColor,line width= 0.4pt,line join=round,line cap=round,fill=fillColor] (191.46,371.60) circle (  1.16);

\path[draw=drawColor,line width= 0.4pt,line join=round,line cap=round,fill=fillColor] (192.72,371.46) circle (  1.16);

\path[draw=drawColor,line width= 0.4pt,line join=round,line cap=round,fill=fillColor] (193.97,371.18) circle (  1.16);

\path[draw=drawColor,line width= 0.4pt,line join=round,line cap=round,fill=fillColor] (195.20,371.14) circle (  1.16);

\path[draw=drawColor,line width= 0.4pt,line join=round,line cap=round,fill=fillColor] (196.41,371.12) circle (  1.16);

\path[draw=drawColor,line width= 0.4pt,line join=round,line cap=round,fill=fillColor] (197.61,371.07) circle (  1.16);

\path[draw=drawColor,line width= 0.4pt,line join=round,line cap=round,fill=fillColor] (198.79,370.85) circle (  1.16);

\path[draw=drawColor,line width= 0.4pt,line join=round,line cap=round,fill=fillColor] (199.95,370.82) circle (  1.16);

\path[draw=drawColor,line width= 0.4pt,line join=round,line cap=round,fill=fillColor] (201.09,370.68) circle (  1.16);

\path[draw=drawColor,line width= 0.4pt,line join=round,line cap=round,fill=fillColor] (202.22,370.35) circle (  1.16);

\path[draw=drawColor,line width= 0.4pt,line join=round,line cap=round,fill=fillColor] (203.34,369.94) circle (  1.16);

\path[draw=drawColor,line width= 0.4pt,line join=round,line cap=round,fill=fillColor] (204.44,368.66) circle (  1.16);

\path[draw=drawColor,line width= 0.4pt,line join=round,line cap=round,fill=fillColor] (205.53,368.45) circle (  1.16);

\path[draw=drawColor,line width= 0.4pt,line join=round,line cap=round,fill=fillColor] (206.60,368.35) circle (  1.16);

\path[draw=drawColor,line width= 0.4pt,line join=round,line cap=round,fill=fillColor] (207.67,367.62) circle (  1.16);

\path[draw=drawColor,line width= 0.4pt,line join=round,line cap=round,fill=fillColor] (208.72,366.97) circle (  1.16);

\path[draw=drawColor,line width= 0.4pt,line join=round,line cap=round,fill=fillColor] (209.75,366.79) circle (  1.16);

\path[draw=drawColor,line width= 0.4pt,line join=round,line cap=round,fill=fillColor] (210.78,366.52) circle (  1.16);

\path[draw=drawColor,line width= 0.4pt,line join=round,line cap=round,fill=fillColor] (211.79,365.14) circle (  1.16);

\path[draw=drawColor,line width= 0.4pt,line join=round,line cap=round,fill=fillColor] (212.79,364.78) circle (  1.16);

\path[draw=drawColor,line width= 0.4pt,line join=round,line cap=round,fill=fillColor] (213.78,364.63) circle (  1.16);

\path[draw=drawColor,line width= 0.4pt,line join=round,line cap=round,fill=fillColor] (214.77,364.14) circle (  1.16);

\path[draw=drawColor,line width= 0.4pt,line join=round,line cap=round,fill=fillColor] (215.74,361.21) circle (  1.16);

\path[draw=drawColor,line width= 0.4pt,line join=round,line cap=round,fill=fillColor] (216.70,360.00) circle (  1.16);

\path[draw=drawColor,line width= 0.4pt,line join=round,line cap=round,fill=fillColor] (217.65,358.95) circle (  1.16);

\path[draw=drawColor,line width= 0.4pt,line join=round,line cap=round,fill=fillColor] (218.59,357.77) circle (  1.16);

\path[draw=drawColor,line width= 0.4pt,line join=round,line cap=round,fill=fillColor] (219.52,357.40) circle (  1.16);

\path[draw=drawColor,line width= 0.4pt,line join=round,line cap=round,fill=fillColor] (220.44,355.67) circle (  1.16);

\path[draw=drawColor,line width= 0.4pt,line join=round,line cap=round,fill=fillColor] (221.36,349.12) circle (  1.16);
\definecolor[named]{drawColor}{rgb}{0.65,0.34,0.16}
\definecolor[named]{fillColor}{rgb}{0.65,0.34,0.16}

\path[draw=drawColor,line width= 0.4pt,line join=round,line cap=round,fill=fillColor] ( 76.72,367.39) circle (  1.16);

\path[draw=drawColor,line width= 0.4pt,line join=round,line cap=round,fill=fillColor] ( 88.76,366.55) circle (  1.16);

\path[draw=drawColor,line width= 0.4pt,line join=round,line cap=round,fill=fillColor] ( 97.21,365.34) circle (  1.16);

\path[draw=drawColor,line width= 0.4pt,line join=round,line cap=round,fill=fillColor] (103.94,364.60) circle (  1.16);

\path[draw=drawColor,line width= 0.4pt,line join=round,line cap=round,fill=fillColor] (109.62,364.18) circle (  1.16);

\path[draw=drawColor,line width= 0.4pt,line join=round,line cap=round,fill=fillColor] (114.58,362.98) circle (  1.16);

\path[draw=drawColor,line width= 0.4pt,line join=round,line cap=round,fill=fillColor] (119.02,362.62) circle (  1.16);

\path[draw=drawColor,line width= 0.4pt,line join=round,line cap=round,fill=fillColor] (123.06,362.50) circle (  1.16);

\path[draw=drawColor,line width= 0.4pt,line join=round,line cap=round,fill=fillColor] (126.77,362.33) circle (  1.16);

\path[draw=drawColor,line width= 0.4pt,line join=round,line cap=round,fill=fillColor] (130.21,361.65) circle (  1.16);

\path[draw=drawColor,line width= 0.4pt,line join=round,line cap=round,fill=fillColor] (133.44,361.52) circle (  1.16);

\path[draw=drawColor,line width= 0.4pt,line join=round,line cap=round,fill=fillColor] (136.47,361.51) circle (  1.16);

\path[draw=drawColor,line width= 0.4pt,line join=round,line cap=round,fill=fillColor] (139.34,361.16) circle (  1.16);

\path[draw=drawColor,line width= 0.4pt,line join=round,line cap=round,fill=fillColor] (142.06,360.90) circle (  1.16);

\path[draw=drawColor,line width= 0.4pt,line join=round,line cap=round,fill=fillColor] (144.66,360.86) circle (  1.16);

\path[draw=drawColor,line width= 0.4pt,line join=round,line cap=round,fill=fillColor] (147.15,360.85) circle (  1.16);

\path[draw=drawColor,line width= 0.4pt,line join=round,line cap=round,fill=fillColor] (149.53,360.56) circle (  1.16);

\path[draw=drawColor,line width= 0.4pt,line join=round,line cap=round,fill=fillColor] (151.82,360.24) circle (  1.16);

\path[draw=drawColor,line width= 0.4pt,line join=round,line cap=round,fill=fillColor] (154.03,359.18) circle (  1.16);

\path[draw=drawColor,line width= 0.4pt,line join=round,line cap=round,fill=fillColor] (156.16,359.16) circle (  1.16);

\path[draw=drawColor,line width= 0.4pt,line join=round,line cap=round,fill=fillColor] (158.23,358.71) circle (  1.16);

\path[draw=drawColor,line width= 0.4pt,line join=round,line cap=round,fill=fillColor] (160.22,358.35) circle (  1.16);

\path[draw=drawColor,line width= 0.4pt,line join=round,line cap=round,fill=fillColor] (162.16,357.88) circle (  1.16);

\path[draw=drawColor,line width= 0.4pt,line join=round,line cap=round,fill=fillColor] (164.04,357.44) circle (  1.16);

\path[draw=drawColor,line width= 0.4pt,line join=round,line cap=round,fill=fillColor] (165.88,357.30) circle (  1.16);

\path[draw=drawColor,line width= 0.4pt,line join=round,line cap=round,fill=fillColor] (167.66,356.77) circle (  1.16);

\path[draw=drawColor,line width= 0.4pt,line join=round,line cap=round,fill=fillColor] (169.40,356.68) circle (  1.16);

\path[draw=drawColor,line width= 0.4pt,line join=round,line cap=round,fill=fillColor] (171.09,356.24) circle (  1.16);

\path[draw=drawColor,line width= 0.4pt,line join=round,line cap=round,fill=fillColor] (172.75,355.25) circle (  1.16);

\path[draw=drawColor,line width= 0.4pt,line join=round,line cap=round,fill=fillColor] (174.37,355.18) circle (  1.16);

\path[draw=drawColor,line width= 0.4pt,line join=round,line cap=round,fill=fillColor] (175.95,353.89) circle (  1.16);

\path[draw=drawColor,line width= 0.4pt,line join=round,line cap=round,fill=fillColor] (177.50,353.78) circle (  1.16);

\path[draw=drawColor,line width= 0.4pt,line join=round,line cap=round,fill=fillColor] (179.01,349.12) circle (  1.16);

\path[draw=drawColor,line width= 0.4pt,line join=round,line cap=round,fill=fillColor] (180.50,349.12) circle (  1.16);

\path[draw=drawColor,line width= 0.4pt,line join=round,line cap=round,fill=fillColor] (181.96,349.12) circle (  1.16);

\path[draw=drawColor,line width= 0.4pt,line join=round,line cap=round,fill=fillColor] (183.39,349.12) circle (  1.16);

\path[draw=drawColor,line width= 0.4pt,line join=round,line cap=round,fill=fillColor] (184.79,349.12) circle (  1.16);

\path[draw=drawColor,line width= 0.4pt,line join=round,line cap=round,fill=fillColor] (186.17,349.12) circle (  1.16);

\path[draw=drawColor,line width= 0.4pt,line join=round,line cap=round,fill=fillColor] (187.52,349.12) circle (  1.16);

\path[draw=drawColor,line width= 0.4pt,line join=round,line cap=round,fill=fillColor] (188.86,349.12) circle (  1.16);

\path[draw=drawColor,line width= 0.4pt,line join=round,line cap=round,fill=fillColor] (190.17,349.12) circle (  1.16);

\path[draw=drawColor,line width= 0.4pt,line join=round,line cap=round,fill=fillColor] (191.46,349.12) circle (  1.16);

\path[draw=drawColor,line width= 0.4pt,line join=round,line cap=round,fill=fillColor] (192.72,349.12) circle (  1.16);

\path[draw=drawColor,line width= 0.4pt,line join=round,line cap=round,fill=fillColor] (193.97,349.12) circle (  1.16);

\path[draw=drawColor,line width= 0.4pt,line join=round,line cap=round,fill=fillColor] (195.20,349.12) circle (  1.16);

\path[draw=drawColor,line width= 0.4pt,line join=round,line cap=round,fill=fillColor] (196.41,349.12) circle (  1.16);

\path[draw=drawColor,line width= 0.4pt,line join=round,line cap=round,fill=fillColor] (197.61,349.12) circle (  1.16);

\path[draw=drawColor,line width= 0.4pt,line join=round,line cap=round,fill=fillColor] (198.79,349.12) circle (  1.16);

\path[draw=drawColor,line width= 0.4pt,line join=round,line cap=round,fill=fillColor] (199.95,349.12) circle (  1.16);

\path[draw=drawColor,line width= 0.4pt,line join=round,line cap=round,fill=fillColor] (201.09,349.12) circle (  1.16);

\path[draw=drawColor,line width= 0.4pt,line join=round,line cap=round,fill=fillColor] (202.22,349.12) circle (  1.16);

\path[draw=drawColor,line width= 0.4pt,line join=round,line cap=round,fill=fillColor] (203.34,349.12) circle (  1.16);

\path[draw=drawColor,line width= 0.4pt,line join=round,line cap=round,fill=fillColor] (204.44,349.12) circle (  1.16);

\path[draw=drawColor,line width= 0.4pt,line join=round,line cap=round,fill=fillColor] (205.53,349.12) circle (  1.16);

\path[draw=drawColor,line width= 0.4pt,line join=round,line cap=round,fill=fillColor] (206.60,349.12) circle (  1.16);

\path[draw=drawColor,line width= 0.4pt,line join=round,line cap=round,fill=fillColor] (207.67,349.12) circle (  1.16);

\path[draw=drawColor,line width= 0.4pt,line join=round,line cap=round,fill=fillColor] (208.72,349.12) circle (  1.16);

\path[draw=drawColor,line width= 0.4pt,line join=round,line cap=round,fill=fillColor] (209.75,349.12) circle (  1.16);

\path[draw=drawColor,line width= 0.4pt,line join=round,line cap=round,fill=fillColor] (210.78,349.12) circle (  1.16);

\path[draw=drawColor,line width= 0.4pt,line join=round,line cap=round,fill=fillColor] (211.79,349.12) circle (  1.16);

\path[draw=drawColor,line width= 0.4pt,line join=round,line cap=round,fill=fillColor] (212.79,349.12) circle (  1.16);

\path[draw=drawColor,line width= 0.4pt,line join=round,line cap=round,fill=fillColor] (213.78,349.12) circle (  1.16);

\path[draw=drawColor,line width= 0.4pt,line join=round,line cap=round,fill=fillColor] (214.77,349.12) circle (  1.16);

\path[draw=drawColor,line width= 0.4pt,line join=round,line cap=round,fill=fillColor] (215.74,349.12) circle (  1.16);

\path[draw=drawColor,line width= 0.4pt,line join=round,line cap=round,fill=fillColor] (216.70,349.12) circle (  1.16);

\path[draw=drawColor,line width= 0.4pt,line join=round,line cap=round,fill=fillColor] (217.65,349.12) circle (  1.16);

\path[draw=drawColor,line width= 0.4pt,line join=round,line cap=round,fill=fillColor] (218.59,349.12) circle (  1.16);

\path[draw=drawColor,line width= 0.4pt,line join=round,line cap=round,fill=fillColor] (219.52,349.12) circle (  1.16);

\path[draw=drawColor,line width= 0.4pt,line join=round,line cap=round,fill=fillColor] (220.44,349.12) circle (  1.16);

\path[draw=drawColor,line width= 0.4pt,line join=round,line cap=round,fill=fillColor] (221.36,349.12) circle (  1.16);
\definecolor[named]{drawColor}{rgb}{0.22,0.49,0.72}
\definecolor[named]{fillColor}{rgb}{0.22,0.49,0.72}

\path[draw=drawColor,line width= 0.4pt,line join=round,line cap=round,fill=fillColor] ( 76.72,368.69) circle (  1.16);

\path[draw=drawColor,line width= 0.4pt,line join=round,line cap=round,fill=fillColor] ( 88.76,368.20) circle (  1.16);

\path[draw=drawColor,line width= 0.4pt,line join=round,line cap=round,fill=fillColor] ( 97.21,366.98) circle (  1.16);

\path[draw=drawColor,line width= 0.4pt,line join=round,line cap=round,fill=fillColor] (103.94,366.48) circle (  1.16);

\path[draw=drawColor,line width= 0.4pt,line join=round,line cap=round,fill=fillColor] (109.62,366.33) circle (  1.16);

\path[draw=drawColor,line width= 0.4pt,line join=round,line cap=round,fill=fillColor] (114.58,366.16) circle (  1.16);

\path[draw=drawColor,line width= 0.4pt,line join=round,line cap=round,fill=fillColor] (119.02,365.12) circle (  1.16);

\path[draw=drawColor,line width= 0.4pt,line join=round,line cap=round,fill=fillColor] (123.06,364.87) circle (  1.16);

\path[draw=drawColor,line width= 0.4pt,line join=round,line cap=round,fill=fillColor] (126.77,364.79) circle (  1.16);

\path[draw=drawColor,line width= 0.4pt,line join=round,line cap=round,fill=fillColor] (130.21,364.15) circle (  1.16);

\path[draw=drawColor,line width= 0.4pt,line join=round,line cap=round,fill=fillColor] (133.44,363.65) circle (  1.16);

\path[draw=drawColor,line width= 0.4pt,line join=round,line cap=round,fill=fillColor] (136.47,363.43) circle (  1.16);

\path[draw=drawColor,line width= 0.4pt,line join=round,line cap=round,fill=fillColor] (139.34,363.21) circle (  1.16);

\path[draw=drawColor,line width= 0.4pt,line join=round,line cap=round,fill=fillColor] (142.06,363.20) circle (  1.16);

\path[draw=drawColor,line width= 0.4pt,line join=round,line cap=round,fill=fillColor] (144.66,362.76) circle (  1.16);

\path[draw=drawColor,line width= 0.4pt,line join=round,line cap=round,fill=fillColor] (147.15,362.68) circle (  1.16);

\path[draw=drawColor,line width= 0.4pt,line join=round,line cap=round,fill=fillColor] (149.53,362.64) circle (  1.16);

\path[draw=drawColor,line width= 0.4pt,line join=round,line cap=round,fill=fillColor] (151.82,362.59) circle (  1.16);

\path[draw=drawColor,line width= 0.4pt,line join=round,line cap=round,fill=fillColor] (154.03,362.54) circle (  1.16);

\path[draw=drawColor,line width= 0.4pt,line join=round,line cap=round,fill=fillColor] (156.16,362.50) circle (  1.16);

\path[draw=drawColor,line width= 0.4pt,line join=round,line cap=round,fill=fillColor] (158.23,362.22) circle (  1.16);

\path[draw=drawColor,line width= 0.4pt,line join=round,line cap=round,fill=fillColor] (160.22,362.02) circle (  1.16);

\path[draw=drawColor,line width= 0.4pt,line join=round,line cap=round,fill=fillColor] (162.16,361.77) circle (  1.16);

\path[draw=drawColor,line width= 0.4pt,line join=round,line cap=round,fill=fillColor] (164.04,361.56) circle (  1.16);

\path[draw=drawColor,line width= 0.4pt,line join=round,line cap=round,fill=fillColor] (165.88,361.29) circle (  1.16);

\path[draw=drawColor,line width= 0.4pt,line join=round,line cap=round,fill=fillColor] (167.66,361.23) circle (  1.16);

\path[draw=drawColor,line width= 0.4pt,line join=round,line cap=round,fill=fillColor] (169.40,361.19) circle (  1.16);

\path[draw=drawColor,line width= 0.4pt,line join=round,line cap=round,fill=fillColor] (171.09,361.12) circle (  1.16);

\path[draw=drawColor,line width= 0.4pt,line join=round,line cap=round,fill=fillColor] (172.75,361.12) circle (  1.16);

\path[draw=drawColor,line width= 0.4pt,line join=round,line cap=round,fill=fillColor] (174.37,360.78) circle (  1.16);

\path[draw=drawColor,line width= 0.4pt,line join=round,line cap=round,fill=fillColor] (175.95,360.69) circle (  1.16);

\path[draw=drawColor,line width= 0.4pt,line join=round,line cap=round,fill=fillColor] (177.50,359.55) circle (  1.16);

\path[draw=drawColor,line width= 0.4pt,line join=round,line cap=round,fill=fillColor] (179.01,359.37) circle (  1.16);

\path[draw=drawColor,line width= 0.4pt,line join=round,line cap=round,fill=fillColor] (180.50,359.32) circle (  1.16);

\path[draw=drawColor,line width= 0.4pt,line join=round,line cap=round,fill=fillColor] (181.96,359.07) circle (  1.16);

\path[draw=drawColor,line width= 0.4pt,line join=round,line cap=round,fill=fillColor] (183.39,358.74) circle (  1.16);

\path[draw=drawColor,line width= 0.4pt,line join=round,line cap=round,fill=fillColor] (184.79,358.66) circle (  1.16);

\path[draw=drawColor,line width= 0.4pt,line join=round,line cap=round,fill=fillColor] (186.17,358.31) circle (  1.16);

\path[draw=drawColor,line width= 0.4pt,line join=round,line cap=round,fill=fillColor] (187.52,358.27) circle (  1.16);

\path[draw=drawColor,line width= 0.4pt,line join=round,line cap=round,fill=fillColor] (188.86,358.23) circle (  1.16);

\path[draw=drawColor,line width= 0.4pt,line join=round,line cap=round,fill=fillColor] (190.17,357.85) circle (  1.16);

\path[draw=drawColor,line width= 0.4pt,line join=round,line cap=round,fill=fillColor] (191.46,357.81) circle (  1.16);

\path[draw=drawColor,line width= 0.4pt,line join=round,line cap=round,fill=fillColor] (192.72,356.95) circle (  1.16);

\path[draw=drawColor,line width= 0.4pt,line join=round,line cap=round,fill=fillColor] (193.97,356.65) circle (  1.16);

\path[draw=drawColor,line width= 0.4pt,line join=round,line cap=round,fill=fillColor] (195.20,356.18) circle (  1.16);

\path[draw=drawColor,line width= 0.4pt,line join=round,line cap=round,fill=fillColor] (196.41,356.10) circle (  1.16);

\path[draw=drawColor,line width= 0.4pt,line join=round,line cap=round,fill=fillColor] (197.61,356.07) circle (  1.16);

\path[draw=drawColor,line width= 0.4pt,line join=round,line cap=round,fill=fillColor] (198.79,354.77) circle (  1.16);

\path[draw=drawColor,line width= 0.4pt,line join=round,line cap=round,fill=fillColor] (199.95,349.12) circle (  1.16);

\path[draw=drawColor,line width= 0.4pt,line join=round,line cap=round,fill=fillColor] (201.09,349.12) circle (  1.16);

\path[draw=drawColor,line width= 0.4pt,line join=round,line cap=round,fill=fillColor] (202.22,349.12) circle (  1.16);

\path[draw=drawColor,line width= 0.4pt,line join=round,line cap=round,fill=fillColor] (203.34,349.12) circle (  1.16);

\path[draw=drawColor,line width= 0.4pt,line join=round,line cap=round,fill=fillColor] (204.44,349.12) circle (  1.16);

\path[draw=drawColor,line width= 0.4pt,line join=round,line cap=round,fill=fillColor] (205.53,349.12) circle (  1.16);

\path[draw=drawColor,line width= 0.4pt,line join=round,line cap=round,fill=fillColor] (206.60,349.12) circle (  1.16);

\path[draw=drawColor,line width= 0.4pt,line join=round,line cap=round,fill=fillColor] (207.67,349.12) circle (  1.16);

\path[draw=drawColor,line width= 0.4pt,line join=round,line cap=round,fill=fillColor] (208.72,349.12) circle (  1.16);

\path[draw=drawColor,line width= 0.4pt,line join=round,line cap=round,fill=fillColor] (209.75,349.12) circle (  1.16);

\path[draw=drawColor,line width= 0.4pt,line join=round,line cap=round,fill=fillColor] (210.78,349.12) circle (  1.16);

\path[draw=drawColor,line width= 0.4pt,line join=round,line cap=round,fill=fillColor] (211.79,349.12) circle (  1.16);

\path[draw=drawColor,line width= 0.4pt,line join=round,line cap=round,fill=fillColor] (212.79,349.12) circle (  1.16);

\path[draw=drawColor,line width= 0.4pt,line join=round,line cap=round,fill=fillColor] (213.78,349.12) circle (  1.16);

\path[draw=drawColor,line width= 0.4pt,line join=round,line cap=round,fill=fillColor] (214.77,349.12) circle (  1.16);

\path[draw=drawColor,line width= 0.4pt,line join=round,line cap=round,fill=fillColor] (215.74,349.12) circle (  1.16);

\path[draw=drawColor,line width= 0.4pt,line join=round,line cap=round,fill=fillColor] (216.70,349.12) circle (  1.16);

\path[draw=drawColor,line width= 0.4pt,line join=round,line cap=round,fill=fillColor] (217.65,349.12) circle (  1.16);

\path[draw=drawColor,line width= 0.4pt,line join=round,line cap=round,fill=fillColor] (218.59,349.12) circle (  1.16);

\path[draw=drawColor,line width= 0.4pt,line join=round,line cap=round,fill=fillColor] (219.52,349.12) circle (  1.16);

\path[draw=drawColor,line width= 0.4pt,line join=round,line cap=round,fill=fillColor] (220.44,349.12) circle (  1.16);

\path[draw=drawColor,line width= 0.4pt,line join=round,line cap=round,fill=fillColor] (221.36,349.12) circle (  1.16);
\definecolor[named]{drawColor}{rgb}{0.30,0.69,0.29}
\definecolor[named]{fillColor}{rgb}{0.30,0.69,0.29}

\path[draw=drawColor,line width= 0.4pt,line join=round,line cap=round,fill=fillColor] ( 76.72,368.24) circle (  1.16);

\path[draw=drawColor,line width= 0.4pt,line join=round,line cap=round,fill=fillColor] ( 88.76,366.84) circle (  1.16);

\path[draw=drawColor,line width= 0.4pt,line join=round,line cap=round,fill=fillColor] ( 97.21,366.03) circle (  1.16);

\path[draw=drawColor,line width= 0.4pt,line join=round,line cap=round,fill=fillColor] (103.94,365.94) circle (  1.16);

\path[draw=drawColor,line width= 0.4pt,line join=round,line cap=round,fill=fillColor] (109.62,365.67) circle (  1.16);

\path[draw=drawColor,line width= 0.4pt,line join=round,line cap=round,fill=fillColor] (114.58,365.42) circle (  1.16);

\path[draw=drawColor,line width= 0.4pt,line join=round,line cap=round,fill=fillColor] (119.02,364.84) circle (  1.16);

\path[draw=drawColor,line width= 0.4pt,line join=round,line cap=round,fill=fillColor] (123.06,364.54) circle (  1.16);

\path[draw=drawColor,line width= 0.4pt,line join=round,line cap=round,fill=fillColor] (126.77,364.47) circle (  1.16);

\path[draw=drawColor,line width= 0.4pt,line join=round,line cap=round,fill=fillColor] (130.21,364.17) circle (  1.16);

\path[draw=drawColor,line width= 0.4pt,line join=round,line cap=round,fill=fillColor] (133.44,364.15) circle (  1.16);

\path[draw=drawColor,line width= 0.4pt,line join=round,line cap=round,fill=fillColor] (136.47,363.94) circle (  1.16);

\path[draw=drawColor,line width= 0.4pt,line join=round,line cap=round,fill=fillColor] (139.34,363.86) circle (  1.16);

\path[draw=drawColor,line width= 0.4pt,line join=round,line cap=round,fill=fillColor] (142.06,363.65) circle (  1.16);

\path[draw=drawColor,line width= 0.4pt,line join=round,line cap=round,fill=fillColor] (144.66,363.65) circle (  1.16);

\path[draw=drawColor,line width= 0.4pt,line join=round,line cap=round,fill=fillColor] (147.15,363.42) circle (  1.16);

\path[draw=drawColor,line width= 0.4pt,line join=round,line cap=round,fill=fillColor] (149.53,363.34) circle (  1.16);

\path[draw=drawColor,line width= 0.4pt,line join=round,line cap=round,fill=fillColor] (151.82,363.33) circle (  1.16);

\path[draw=drawColor,line width= 0.4pt,line join=round,line cap=round,fill=fillColor] (154.03,363.20) circle (  1.16);

\path[draw=drawColor,line width= 0.4pt,line join=round,line cap=round,fill=fillColor] (156.16,362.93) circle (  1.16);

\path[draw=drawColor,line width= 0.4pt,line join=round,line cap=round,fill=fillColor] (158.23,362.91) circle (  1.16);

\path[draw=drawColor,line width= 0.4pt,line join=round,line cap=round,fill=fillColor] (160.22,362.70) circle (  1.16);

\path[draw=drawColor,line width= 0.4pt,line join=round,line cap=round,fill=fillColor] (162.16,362.54) circle (  1.16);

\path[draw=drawColor,line width= 0.4pt,line join=round,line cap=round,fill=fillColor] (164.04,362.49) circle (  1.16);

\path[draw=drawColor,line width= 0.4pt,line join=round,line cap=round,fill=fillColor] (165.88,362.16) circle (  1.16);

\path[draw=drawColor,line width= 0.4pt,line join=round,line cap=round,fill=fillColor] (167.66,362.08) circle (  1.16);

\path[draw=drawColor,line width= 0.4pt,line join=round,line cap=round,fill=fillColor] (169.40,361.56) circle (  1.16);

\path[draw=drawColor,line width= 0.4pt,line join=round,line cap=round,fill=fillColor] (171.09,361.56) circle (  1.16);

\path[draw=drawColor,line width= 0.4pt,line join=round,line cap=round,fill=fillColor] (172.75,361.55) circle (  1.16);

\path[draw=drawColor,line width= 0.4pt,line join=round,line cap=round,fill=fillColor] (174.37,361.14) circle (  1.16);

\path[draw=drawColor,line width= 0.4pt,line join=round,line cap=round,fill=fillColor] (175.95,361.12) circle (  1.16);

\path[draw=drawColor,line width= 0.4pt,line join=round,line cap=round,fill=fillColor] (177.50,361.11) circle (  1.16);

\path[draw=drawColor,line width= 0.4pt,line join=round,line cap=round,fill=fillColor] (179.01,360.83) circle (  1.16);

\path[draw=drawColor,line width= 0.4pt,line join=round,line cap=round,fill=fillColor] (180.50,360.76) circle (  1.16);

\path[draw=drawColor,line width= 0.4pt,line join=round,line cap=round,fill=fillColor] (181.96,360.76) circle (  1.16);

\path[draw=drawColor,line width= 0.4pt,line join=round,line cap=round,fill=fillColor] (183.39,360.71) circle (  1.16);

\path[draw=drawColor,line width= 0.4pt,line join=round,line cap=round,fill=fillColor] (184.79,360.63) circle (  1.16);

\path[draw=drawColor,line width= 0.4pt,line join=round,line cap=round,fill=fillColor] (186.17,359.88) circle (  1.16);

\path[draw=drawColor,line width= 0.4pt,line join=round,line cap=round,fill=fillColor] (187.52,359.73) circle (  1.16);

\path[draw=drawColor,line width= 0.4pt,line join=round,line cap=round,fill=fillColor] (188.86,359.58) circle (  1.16);

\path[draw=drawColor,line width= 0.4pt,line join=round,line cap=round,fill=fillColor] (190.17,359.54) circle (  1.16);

\path[draw=drawColor,line width= 0.4pt,line join=round,line cap=round,fill=fillColor] (191.46,359.33) circle (  1.16);

\path[draw=drawColor,line width= 0.4pt,line join=round,line cap=round,fill=fillColor] (192.72,359.33) circle (  1.16);

\path[draw=drawColor,line width= 0.4pt,line join=round,line cap=round,fill=fillColor] (193.97,359.23) circle (  1.16);

\path[draw=drawColor,line width= 0.4pt,line join=round,line cap=round,fill=fillColor] (195.20,358.86) circle (  1.16);

\path[draw=drawColor,line width= 0.4pt,line join=round,line cap=round,fill=fillColor] (196.41,358.66) circle (  1.16);

\path[draw=drawColor,line width= 0.4pt,line join=round,line cap=round,fill=fillColor] (197.61,358.42) circle (  1.16);

\path[draw=drawColor,line width= 0.4pt,line join=round,line cap=round,fill=fillColor] (198.79,358.25) circle (  1.16);

\path[draw=drawColor,line width= 0.4pt,line join=round,line cap=round,fill=fillColor] (199.95,357.81) circle (  1.16);

\path[draw=drawColor,line width= 0.4pt,line join=round,line cap=round,fill=fillColor] (201.09,356.18) circle (  1.16);

\path[draw=drawColor,line width= 0.4pt,line join=round,line cap=round,fill=fillColor] (202.22,355.74) circle (  1.16);

\path[draw=drawColor,line width= 0.4pt,line join=round,line cap=round,fill=fillColor] (203.34,349.12) circle (  1.16);

\path[draw=drawColor,line width= 0.4pt,line join=round,line cap=round,fill=fillColor] (204.44,349.12) circle (  1.16);

\path[draw=drawColor,line width= 0.4pt,line join=round,line cap=round,fill=fillColor] (205.53,349.12) circle (  1.16);

\path[draw=drawColor,line width= 0.4pt,line join=round,line cap=round,fill=fillColor] (206.60,349.12) circle (  1.16);

\path[draw=drawColor,line width= 0.4pt,line join=round,line cap=round,fill=fillColor] (207.67,349.12) circle (  1.16);

\path[draw=drawColor,line width= 0.4pt,line join=round,line cap=round,fill=fillColor] (208.72,349.12) circle (  1.16);

\path[draw=drawColor,line width= 0.4pt,line join=round,line cap=round,fill=fillColor] (209.75,349.12) circle (  1.16);

\path[draw=drawColor,line width= 0.4pt,line join=round,line cap=round,fill=fillColor] (210.78,349.12) circle (  1.16);

\path[draw=drawColor,line width= 0.4pt,line join=round,line cap=round,fill=fillColor] (211.79,349.12) circle (  1.16);

\path[draw=drawColor,line width= 0.4pt,line join=round,line cap=round,fill=fillColor] (212.79,349.12) circle (  1.16);

\path[draw=drawColor,line width= 0.4pt,line join=round,line cap=round,fill=fillColor] (213.78,349.12) circle (  1.16);

\path[draw=drawColor,line width= 0.4pt,line join=round,line cap=round,fill=fillColor] (214.77,349.12) circle (  1.16);

\path[draw=drawColor,line width= 0.4pt,line join=round,line cap=round,fill=fillColor] (215.74,349.12) circle (  1.16);

\path[draw=drawColor,line width= 0.4pt,line join=round,line cap=round,fill=fillColor] (216.70,349.12) circle (  1.16);

\path[draw=drawColor,line width= 0.4pt,line join=round,line cap=round,fill=fillColor] (217.65,349.12) circle (  1.16);

\path[draw=drawColor,line width= 0.4pt,line join=round,line cap=round,fill=fillColor] (218.59,349.12) circle (  1.16);

\path[draw=drawColor,line width= 0.4pt,line join=round,line cap=round,fill=fillColor] (219.52,349.12) circle (  1.16);

\path[draw=drawColor,line width= 0.4pt,line join=round,line cap=round,fill=fillColor] (220.44,349.12) circle (  1.16);

\path[draw=drawColor,line width= 0.4pt,line join=round,line cap=round,fill=fillColor] (221.36,349.12) circle (  1.16);
\definecolor[named]{drawColor}{rgb}{0.60,0.31,0.64}
\definecolor[named]{fillColor}{rgb}{0.60,0.31,0.64}

\path[draw=drawColor,line width= 0.4pt,line join=round,line cap=round,fill=fillColor] ( 76.72,368.77) circle (  1.16);

\path[draw=drawColor,line width= 0.4pt,line join=round,line cap=round,fill=fillColor] ( 88.76,368.30) circle (  1.16);

\path[draw=drawColor,line width= 0.4pt,line join=round,line cap=round,fill=fillColor] ( 97.21,367.58) circle (  1.16);

\path[draw=drawColor,line width= 0.4pt,line join=round,line cap=round,fill=fillColor] (103.94,367.22) circle (  1.16);

\path[draw=drawColor,line width= 0.4pt,line join=round,line cap=round,fill=fillColor] (109.62,366.89) circle (  1.16);

\path[draw=drawColor,line width= 0.4pt,line join=round,line cap=round,fill=fillColor] (114.58,366.65) circle (  1.16);

\path[draw=drawColor,line width= 0.4pt,line join=round,line cap=round,fill=fillColor] (119.02,365.30) circle (  1.16);

\path[draw=drawColor,line width= 0.4pt,line join=round,line cap=round,fill=fillColor] (123.06,365.18) circle (  1.16);

\path[draw=drawColor,line width= 0.4pt,line join=round,line cap=round,fill=fillColor] (126.77,365.18) circle (  1.16);

\path[draw=drawColor,line width= 0.4pt,line join=round,line cap=round,fill=fillColor] (130.21,364.75) circle (  1.16);

\path[draw=drawColor,line width= 0.4pt,line join=round,line cap=round,fill=fillColor] (133.44,364.24) circle (  1.16);

\path[draw=drawColor,line width= 0.4pt,line join=round,line cap=round,fill=fillColor] (136.47,364.15) circle (  1.16);

\path[draw=drawColor,line width= 0.4pt,line join=round,line cap=round,fill=fillColor] (139.34,363.83) circle (  1.16);

\path[draw=drawColor,line width= 0.4pt,line join=round,line cap=round,fill=fillColor] (142.06,363.35) circle (  1.16);

\path[draw=drawColor,line width= 0.4pt,line join=round,line cap=round,fill=fillColor] (144.66,362.60) circle (  1.16);

\path[draw=drawColor,line width= 0.4pt,line join=round,line cap=round,fill=fillColor] (147.15,362.28) circle (  1.16);

\path[draw=drawColor,line width= 0.4pt,line join=round,line cap=round,fill=fillColor] (149.53,362.24) circle (  1.16);

\path[draw=drawColor,line width= 0.4pt,line join=round,line cap=round,fill=fillColor] (151.82,362.21) circle (  1.16);

\path[draw=drawColor,line width= 0.4pt,line join=round,line cap=round,fill=fillColor] (154.03,361.74) circle (  1.16);

\path[draw=drawColor,line width= 0.4pt,line join=round,line cap=round,fill=fillColor] (156.16,361.49) circle (  1.16);

\path[draw=drawColor,line width= 0.4pt,line join=round,line cap=round,fill=fillColor] (158.23,361.12) circle (  1.16);

\path[draw=drawColor,line width= 0.4pt,line join=round,line cap=round,fill=fillColor] (160.22,361.00) circle (  1.16);

\path[draw=drawColor,line width= 0.4pt,line join=round,line cap=round,fill=fillColor] (162.16,360.82) circle (  1.16);

\path[draw=drawColor,line width= 0.4pt,line join=round,line cap=round,fill=fillColor] (164.04,360.35) circle (  1.16);

\path[draw=drawColor,line width= 0.4pt,line join=round,line cap=round,fill=fillColor] (165.88,360.14) circle (  1.16);

\path[draw=drawColor,line width= 0.4pt,line join=round,line cap=round,fill=fillColor] (167.66,359.98) circle (  1.16);

\path[draw=drawColor,line width= 0.4pt,line join=round,line cap=round,fill=fillColor] (169.40,359.94) circle (  1.16);

\path[draw=drawColor,line width= 0.4pt,line join=round,line cap=round,fill=fillColor] (171.09,359.81) circle (  1.16);

\path[draw=drawColor,line width= 0.4pt,line join=round,line cap=round,fill=fillColor] (172.75,359.79) circle (  1.16);

\path[draw=drawColor,line width= 0.4pt,line join=round,line cap=round,fill=fillColor] (174.37,359.31) circle (  1.16);

\path[draw=drawColor,line width= 0.4pt,line join=round,line cap=round,fill=fillColor] (175.95,359.07) circle (  1.16);

\path[draw=drawColor,line width= 0.4pt,line join=round,line cap=round,fill=fillColor] (177.50,358.92) circle (  1.16);

\path[draw=drawColor,line width= 0.4pt,line join=round,line cap=round,fill=fillColor] (179.01,358.85) circle (  1.16);

\path[draw=drawColor,line width= 0.4pt,line join=round,line cap=round,fill=fillColor] (180.50,357.74) circle (  1.16);

\path[draw=drawColor,line width= 0.4pt,line join=round,line cap=round,fill=fillColor] (181.96,357.44) circle (  1.16);

\path[draw=drawColor,line width= 0.4pt,line join=round,line cap=round,fill=fillColor] (183.39,357.09) circle (  1.16);

\path[draw=drawColor,line width= 0.4pt,line join=round,line cap=round,fill=fillColor] (184.79,356.18) circle (  1.16);

\path[draw=drawColor,line width= 0.4pt,line join=round,line cap=round,fill=fillColor] (186.17,355.79) circle (  1.16);

\path[draw=drawColor,line width= 0.4pt,line join=round,line cap=round,fill=fillColor] (187.52,355.58) circle (  1.16);

\path[draw=drawColor,line width= 0.4pt,line join=round,line cap=round,fill=fillColor] (188.86,355.51) circle (  1.16);

\path[draw=drawColor,line width= 0.4pt,line join=round,line cap=round,fill=fillColor] (190.17,354.31) circle (  1.16);

\path[draw=drawColor,line width= 0.4pt,line join=round,line cap=round,fill=fillColor] (191.46,353.92) circle (  1.16);

\path[draw=drawColor,line width= 0.4pt,line join=round,line cap=round,fill=fillColor] (192.72,352.63) circle (  1.16);

\path[draw=drawColor,line width= 0.4pt,line join=round,line cap=round,fill=fillColor] (193.97,349.12) circle (  1.16);

\path[draw=drawColor,line width= 0.4pt,line join=round,line cap=round,fill=fillColor] (195.20,349.12) circle (  1.16);

\path[draw=drawColor,line width= 0.4pt,line join=round,line cap=round,fill=fillColor] (196.41,349.12) circle (  1.16);

\path[draw=drawColor,line width= 0.4pt,line join=round,line cap=round,fill=fillColor] (197.61,349.12) circle (  1.16);

\path[draw=drawColor,line width= 0.4pt,line join=round,line cap=round,fill=fillColor] (198.79,349.12) circle (  1.16);

\path[draw=drawColor,line width= 0.4pt,line join=round,line cap=round,fill=fillColor] (199.95,349.12) circle (  1.16);

\path[draw=drawColor,line width= 0.4pt,line join=round,line cap=round,fill=fillColor] (201.09,349.12) circle (  1.16);

\path[draw=drawColor,line width= 0.4pt,line join=round,line cap=round,fill=fillColor] (202.22,349.12) circle (  1.16);

\path[draw=drawColor,line width= 0.4pt,line join=round,line cap=round,fill=fillColor] (203.34,349.12) circle (  1.16);

\path[draw=drawColor,line width= 0.4pt,line join=round,line cap=round,fill=fillColor] (204.44,349.12) circle (  1.16);

\path[draw=drawColor,line width= 0.4pt,line join=round,line cap=round,fill=fillColor] (205.53,349.12) circle (  1.16);

\path[draw=drawColor,line width= 0.4pt,line join=round,line cap=round,fill=fillColor] (206.60,349.12) circle (  1.16);

\path[draw=drawColor,line width= 0.4pt,line join=round,line cap=round,fill=fillColor] (207.67,349.12) circle (  1.16);

\path[draw=drawColor,line width= 0.4pt,line join=round,line cap=round,fill=fillColor] (208.72,349.12) circle (  1.16);

\path[draw=drawColor,line width= 0.4pt,line join=round,line cap=round,fill=fillColor] (209.75,349.12) circle (  1.16);

\path[draw=drawColor,line width= 0.4pt,line join=round,line cap=round,fill=fillColor] (210.78,349.12) circle (  1.16);

\path[draw=drawColor,line width= 0.4pt,line join=round,line cap=round,fill=fillColor] (211.79,349.12) circle (  1.16);

\path[draw=drawColor,line width= 0.4pt,line join=round,line cap=round,fill=fillColor] (212.79,349.12) circle (  1.16);

\path[draw=drawColor,line width= 0.4pt,line join=round,line cap=round,fill=fillColor] (213.78,349.12) circle (  1.16);

\path[draw=drawColor,line width= 0.4pt,line join=round,line cap=round,fill=fillColor] (214.77,349.12) circle (  1.16);

\path[draw=drawColor,line width= 0.4pt,line join=round,line cap=round,fill=fillColor] (215.74,349.12) circle (  1.16);

\path[draw=drawColor,line width= 0.4pt,line join=round,line cap=round,fill=fillColor] (216.70,349.12) circle (  1.16);

\path[draw=drawColor,line width= 0.4pt,line join=round,line cap=round,fill=fillColor] (217.65,349.12) circle (  1.16);

\path[draw=drawColor,line width= 0.4pt,line join=round,line cap=round,fill=fillColor] (218.59,349.12) circle (  1.16);

\path[draw=drawColor,line width= 0.4pt,line join=round,line cap=round,fill=fillColor] (219.52,349.12) circle (  1.16);

\path[draw=drawColor,line width= 0.4pt,line join=round,line cap=round,fill=fillColor] (220.44,349.12) circle (  1.16);

\path[draw=drawColor,line width= 0.4pt,line join=round,line cap=round,fill=fillColor] (221.36,349.12) circle (  1.16);
\definecolor[named]{drawColor}{rgb}{0.00,0.00,0.00}
\definecolor[named]{fillColor}{rgb}{0.00,0.00,0.00}

\path[draw=drawColor,line width= 0.6pt,line join=round,fill=fillColor] ( 69.49,431.06) -- (228.59,431.06);

\node[text=drawColor,anchor=base east,inner sep=0pt, outer sep=0pt, scale=  0.85] at (225.08,430.44) {infeasible solutions};

\path[draw=drawColor,line width= 0.6pt,line join=round,line cap=round] ( 69.49,340.93) rectangle (228.59,439.26);
\end{scope}
\begin{scope}
\path[clip] (  0.00,  0.00) rectangle (505.89,614.29);
\definecolor[named]{drawColor}{rgb}{0.00,0.00,0.00}

\node[text=drawColor,anchor=base east,inner sep=0pt, outer sep=0pt, scale=  0.80] at ( 64.09,346.37) {0.00};

\node[text=drawColor,anchor=base east,inner sep=0pt, outer sep=0pt, scale=  0.80] at ( 64.09,364.02) {0.01};

\node[text=drawColor,anchor=base east,inner sep=0pt, outer sep=0pt, scale=  0.80] at ( 64.09,376.55) {0.05};

\node[text=drawColor,anchor=base east,inner sep=0pt, outer sep=0pt, scale=  0.80] at ( 64.09,394.29) {0.20};

\node[text=drawColor,anchor=base east,inner sep=0pt, outer sep=0pt, scale=  0.80] at ( 64.09,411.40) {0.50};

\node[text=drawColor,anchor=base east,inner sep=0pt, outer sep=0pt, scale=  0.80] at ( 64.09,420.81) {0.75};

\node[text=drawColor,anchor=base east,inner sep=0pt, outer sep=0pt, scale=  0.80] at ( 64.09,428.31) {1.00};
\end{scope}
\begin{scope}
\path[clip] (  0.00,  0.00) rectangle (505.89,614.29);
\definecolor[named]{drawColor}{rgb}{0.00,0.00,0.00}

\path[draw=drawColor,line width= 0.6pt,line join=round] ( 66.49,349.12) --
	( 69.49,349.12);

\path[draw=drawColor,line width= 0.6pt,line join=round] ( 66.49,366.77) --
	( 69.49,366.77);

\path[draw=drawColor,line width= 0.6pt,line join=round] ( 66.49,379.31) --
	( 69.49,379.31);

\path[draw=drawColor,line width= 0.6pt,line join=round] ( 66.49,397.04) --
	( 69.49,397.04);

\path[draw=drawColor,line width= 0.6pt,line join=round] ( 66.49,414.16) --
	( 69.49,414.16);

\path[draw=drawColor,line width= 0.6pt,line join=round] ( 66.49,423.57) --
	( 69.49,423.57);

\path[draw=drawColor,line width= 0.6pt,line join=round] ( 66.49,431.06) --
	( 69.49,431.06);
\end{scope}
\begin{scope}
\path[clip] (  0.00,  0.00) rectangle (505.89,614.29);
\definecolor[named]{drawColor}{rgb}{0.00,0.00,0.00}

\path[draw=drawColor,line width= 0.6pt,line join=round] (130.21,337.93) --
	(130.21,340.93);

\path[draw=drawColor,line width= 0.6pt,line join=round] (156.16,337.93) --
	(156.16,340.93);

\path[draw=drawColor,line width= 0.6pt,line join=round] (174.37,337.93) --
	(174.37,340.93);

\path[draw=drawColor,line width= 0.6pt,line join=round] (188.86,337.93) --
	(188.86,340.93);

\path[draw=drawColor,line width= 0.6pt,line join=round] (201.09,337.93) --
	(201.09,340.93);

\path[draw=drawColor,line width= 0.6pt,line join=round] (211.79,337.93) --
	(211.79,340.93);

\path[draw=drawColor,line width= 0.6pt,line join=round] (221.36,337.93) --
	(221.36,340.93);
\end{scope}
\begin{scope}
\path[clip] (  0.00,  0.00) rectangle (505.89,614.29);
\definecolor[named]{drawColor}{rgb}{0.00,0.00,0.00}

\node[text=drawColor,rotate= 50.00,anchor=base east,inner sep=0pt, outer sep=0pt, scale=  0.80] at (134.43,331.98) {10};

\node[text=drawColor,rotate= 50.00,anchor=base east,inner sep=0pt, outer sep=0pt, scale=  0.80] at (160.38,331.98) {20};

\node[text=drawColor,rotate= 50.00,anchor=base east,inner sep=0pt, outer sep=0pt, scale=  0.80] at (178.59,331.98) {30};

\node[text=drawColor,rotate= 50.00,anchor=base east,inner sep=0pt, outer sep=0pt, scale=  0.80] at (193.08,331.98) {40};

\node[text=drawColor,rotate= 50.00,anchor=base east,inner sep=0pt, outer sep=0pt, scale=  0.80] at (205.31,331.98) {50};

\node[text=drawColor,rotate= 50.00,anchor=base east,inner sep=0pt, outer sep=0pt, scale=  0.80] at (216.01,331.98) {60};

\node[text=drawColor,rotate= 50.00,anchor=base east,inner sep=0pt, outer sep=0pt, scale=  0.80] at (225.58,331.98) {70};
\end{scope}
\begin{scope}
\path[clip] (  0.00,  0.00) rectangle (505.89,614.29);
\definecolor[named]{drawColor}{rgb}{0.00,0.00,0.00}

\node[text=drawColor,anchor=base,inner sep=0pt, outer sep=0pt, scale=  0.80] at (149.04,315.55) {\# Instances};
\end{scope}
\begin{scope}
\path[clip] (  0.00,  0.00) rectangle (505.89,614.29);
\definecolor[named]{drawColor}{rgb}{0.00,0.00,0.00}

\node[text=drawColor,rotate= 90.00,anchor=base,inner sep=0pt, outer sep=0pt, scale=  0.80] at ( 39.87,390.09) {1-(Best/Algorithm)};
\end{scope}
\begin{scope}
\path[clip] (  0.00,  0.00) rectangle (505.89,614.29);
\definecolor[named]{drawColor}{rgb}{0.00,0.00,0.00}

\node[text=drawColor,anchor=base,inner sep=0pt, outer sep=0pt, scale=  1.20] at (149.04,446.46) {\ISPD};
\end{scope}
\begin{scope}
\path[clip] (273.78,307.15) rectangle (485.05,460.72);
\definecolor[named]{drawColor}{rgb}{1.00,1.00,1.00}
\definecolor[named]{fillColor}{rgb}{1.00,1.00,1.00}

\path[draw=drawColor,line width= 0.6pt,line join=round,line cap=round,fill=fillColor] (273.78,307.15) rectangle (485.05,460.72);
\end{scope}
\begin{scope}
\path[clip] (324.91,343.99) rectangle (479.05,439.26);
\definecolor[named]{fillColor}{rgb}{1.00,1.00,1.00}

\path[fill=fillColor] (324.91,343.99) rectangle (479.05,439.26);
\definecolor[named]{drawColor}{rgb}{0.98,0.98,0.98}

\path[draw=drawColor,line width= 0.6pt,line join=round] (324.91,360.48) --
	(479.05,360.48);

\path[draw=drawColor,line width= 0.6pt,line join=round] (324.91,375.10) --
	(479.05,375.10);

\path[draw=drawColor,line width= 0.6pt,line join=round] (324.91,389.77) --
	(479.05,389.77);

\path[draw=drawColor,line width= 0.6pt,line join=round] (324.91,406.65) --
	(479.05,406.65);

\path[draw=drawColor,line width= 0.6pt,line join=round] (324.91,419.50) --
	(479.05,419.50);

\path[draw=drawColor,line width= 0.6pt,line join=round] (324.91,427.69) --
	(479.05,427.69);

\path[draw=drawColor,line width= 0.6pt,line join=round] (380.85,343.99) --
	(380.85,439.26);

\path[draw=drawColor,line width= 0.6pt,line join=round] (394.42,343.99) --
	(394.42,439.26);

\path[draw=drawColor,line width= 0.6pt,line join=round] (421.56,343.99) --
	(421.56,439.26);

\path[draw=drawColor,line width= 0.6pt,line join=round] (444.65,343.99) --
	(444.65,439.26);

\path[draw=drawColor,line width= 0.6pt,line join=round] (461.75,343.99) --
	(461.75,439.26);
\definecolor[named]{drawColor}{rgb}{0.75,0.75,0.75}

\path[draw=drawColor,line width= 0.6pt,dash pattern=on 1pt off 3pt ,line join=round] (324.91,351.93) --
	(479.05,351.93);

\path[draw=drawColor,line width= 0.6pt,dash pattern=on 1pt off 3pt ,line join=round] (324.91,369.03) --
	(479.05,369.03);

\path[draw=drawColor,line width= 0.6pt,dash pattern=on 1pt off 3pt ,line join=round] (324.91,381.18) --
	(479.05,381.18);

\path[draw=drawColor,line width= 0.6pt,dash pattern=on 1pt off 3pt ,line join=round] (324.91,398.36) --
	(479.05,398.36);

\path[draw=drawColor,line width= 0.6pt,dash pattern=on 1pt off 3pt ,line join=round] (324.91,414.94) --
	(479.05,414.94);

\path[draw=drawColor,line width= 0.6pt,dash pattern=on 1pt off 3pt ,line join=round] (324.91,424.06) --
	(479.05,424.06);

\path[draw=drawColor,line width= 0.6pt,dash pattern=on 1pt off 3pt ,line join=round] (324.91,431.32) --
	(479.05,431.32);

\path[draw=drawColor,line width= 0.6pt,dash pattern=on 1pt off 3pt ,line join=round] (407.99,343.99) --
	(407.99,439.26);

\path[draw=drawColor,line width= 0.6pt,dash pattern=on 1pt off 3pt ,line join=round] (435.13,343.99) --
	(435.13,439.26);

\path[draw=drawColor,line width= 0.6pt,dash pattern=on 1pt off 3pt ,line join=round] (454.17,343.99) --
	(454.17,439.26);

\path[draw=drawColor,line width= 0.6pt,dash pattern=on 1pt off 3pt ,line join=round] (469.33,343.99) --
	(469.33,439.26);
\definecolor[named]{drawColor}{rgb}{0.89,0.10,0.11}
\definecolor[named]{fillColor}{rgb}{0.89,0.10,0.11}

\path[draw=drawColor,line width= 0.4pt,line join=round,line cap=round,fill=fillColor] (331.92,390.33) circle (  1.16);

\path[draw=drawColor,line width= 0.4pt,line join=round,line cap=round,fill=fillColor] (339.28,388.89) circle (  1.16);

\path[draw=drawColor,line width= 0.4pt,line join=round,line cap=round,fill=fillColor] (344.45,387.90) circle (  1.16);

\path[draw=drawColor,line width= 0.4pt,line join=round,line cap=round,fill=fillColor] (348.57,385.28) circle (  1.16);

\path[draw=drawColor,line width= 0.4pt,line join=round,line cap=round,fill=fillColor] (352.04,384.98) circle (  1.16);

\path[draw=drawColor,line width= 0.4pt,line join=round,line cap=round,fill=fillColor] (355.08,383.44) circle (  1.16);

\path[draw=drawColor,line width= 0.4pt,line join=round,line cap=round,fill=fillColor] (357.79,383.39) circle (  1.16);

\path[draw=drawColor,line width= 0.4pt,line join=round,line cap=round,fill=fillColor] (360.26,383.05) circle (  1.16);

\path[draw=drawColor,line width= 0.4pt,line join=round,line cap=round,fill=fillColor] (362.53,382.66) circle (  1.16);

\path[draw=drawColor,line width= 0.4pt,line join=round,line cap=round,fill=fillColor] (364.64,382.59) circle (  1.16);

\path[draw=drawColor,line width= 0.4pt,line join=round,line cap=round,fill=fillColor] (366.61,382.21) circle (  1.16);

\path[draw=drawColor,line width= 0.4pt,line join=round,line cap=round,fill=fillColor] (368.46,382.19) circle (  1.16);

\path[draw=drawColor,line width= 0.4pt,line join=round,line cap=round,fill=fillColor] (370.22,382.01) circle (  1.16);

\path[draw=drawColor,line width= 0.4pt,line join=round,line cap=round,fill=fillColor] (371.89,381.74) circle (  1.16);

\path[draw=drawColor,line width= 0.4pt,line join=round,line cap=round,fill=fillColor] (373.47,381.72) circle (  1.16);

\path[draw=drawColor,line width= 0.4pt,line join=round,line cap=round,fill=fillColor] (374.99,381.62) circle (  1.16);

\path[draw=drawColor,line width= 0.4pt,line join=round,line cap=round,fill=fillColor] (376.45,381.25) circle (  1.16);

\path[draw=drawColor,line width= 0.4pt,line join=round,line cap=round,fill=fillColor] (377.85,380.90) circle (  1.16);

\path[draw=drawColor,line width= 0.4pt,line join=round,line cap=round,fill=fillColor] (379.21,380.43) circle (  1.16);

\path[draw=drawColor,line width= 0.4pt,line join=round,line cap=round,fill=fillColor] (380.51,380.19) circle (  1.16);

\path[draw=drawColor,line width= 0.4pt,line join=round,line cap=round,fill=fillColor] (381.77,379.92) circle (  1.16);

\path[draw=drawColor,line width= 0.4pt,line join=round,line cap=round,fill=fillColor] (382.99,379.88) circle (  1.16);

\path[draw=drawColor,line width= 0.4pt,line join=round,line cap=round,fill=fillColor] (384.18,379.78) circle (  1.16);

\path[draw=drawColor,line width= 0.4pt,line join=round,line cap=round,fill=fillColor] (385.33,379.77) circle (  1.16);

\path[draw=drawColor,line width= 0.4pt,line join=round,line cap=round,fill=fillColor] (386.45,379.48) circle (  1.16);

\path[draw=drawColor,line width= 0.4pt,line join=round,line cap=round,fill=fillColor] (387.54,379.31) circle (  1.16);

\path[draw=drawColor,line width= 0.4pt,line join=round,line cap=round,fill=fillColor] (388.60,379.19) circle (  1.16);

\path[draw=drawColor,line width= 0.4pt,line join=round,line cap=round,fill=fillColor] (389.64,379.14) circle (  1.16);

\path[draw=drawColor,line width= 0.4pt,line join=round,line cap=round,fill=fillColor] (390.65,378.97) circle (  1.16);

\path[draw=drawColor,line width= 0.4pt,line join=round,line cap=round,fill=fillColor] (391.64,378.92) circle (  1.16);

\path[draw=drawColor,line width= 0.4pt,line join=round,line cap=round,fill=fillColor] (392.61,378.47) circle (  1.16);

\path[draw=drawColor,line width= 0.4pt,line join=round,line cap=round,fill=fillColor] (393.56,378.45) circle (  1.16);

\path[draw=drawColor,line width= 0.4pt,line join=round,line cap=round,fill=fillColor] (394.49,378.30) circle (  1.16);

\path[draw=drawColor,line width= 0.4pt,line join=round,line cap=round,fill=fillColor] (395.40,378.19) circle (  1.16);

\path[draw=drawColor,line width= 0.4pt,line join=round,line cap=round,fill=fillColor] (396.29,378.05) circle (  1.16);

\path[draw=drawColor,line width= 0.4pt,line join=round,line cap=round,fill=fillColor] (397.16,377.80) circle (  1.16);

\path[draw=drawColor,line width= 0.4pt,line join=round,line cap=round,fill=fillColor] (398.02,377.42) circle (  1.16);

\path[draw=drawColor,line width= 0.4pt,line join=round,line cap=round,fill=fillColor] (398.86,377.41) circle (  1.16);

\path[draw=drawColor,line width= 0.4pt,line join=round,line cap=round,fill=fillColor] (399.69,377.38) circle (  1.16);

\path[draw=drawColor,line width= 0.4pt,line join=round,line cap=round,fill=fillColor] (400.51,377.32) circle (  1.16);

\path[draw=drawColor,line width= 0.4pt,line join=round,line cap=round,fill=fillColor] (401.31,377.21) circle (  1.16);

\path[draw=drawColor,line width= 0.4pt,line join=round,line cap=round,fill=fillColor] (402.10,377.09) circle (  1.16);

\path[draw=drawColor,line width= 0.4pt,line join=round,line cap=round,fill=fillColor] (402.87,377.08) circle (  1.16);

\path[draw=drawColor,line width= 0.4pt,line join=round,line cap=round,fill=fillColor] (403.64,377.08) circle (  1.16);

\path[draw=drawColor,line width= 0.4pt,line join=round,line cap=round,fill=fillColor] (404.39,377.05) circle (  1.16);

\path[draw=drawColor,line width= 0.4pt,line join=round,line cap=round,fill=fillColor] (405.13,377.03) circle (  1.16);

\path[draw=drawColor,line width= 0.4pt,line join=round,line cap=round,fill=fillColor] (405.86,377.02) circle (  1.16);

\path[draw=drawColor,line width= 0.4pt,line join=round,line cap=round,fill=fillColor] (406.58,376.94) circle (  1.16);

\path[draw=drawColor,line width= 0.4pt,line join=round,line cap=round,fill=fillColor] (407.29,376.70) circle (  1.16);

\path[draw=drawColor,line width= 0.4pt,line join=round,line cap=round,fill=fillColor] (407.99,376.63) circle (  1.16);

\path[draw=drawColor,line width= 0.4pt,line join=round,line cap=round,fill=fillColor] (408.68,376.62) circle (  1.16);

\path[draw=drawColor,line width= 0.4pt,line join=round,line cap=round,fill=fillColor] (409.37,376.56) circle (  1.16);

\path[draw=drawColor,line width= 0.4pt,line join=round,line cap=round,fill=fillColor] (410.04,376.47) circle (  1.16);

\path[draw=drawColor,line width= 0.4pt,line join=round,line cap=round,fill=fillColor] (410.71,376.46) circle (  1.16);

\path[draw=drawColor,line width= 0.4pt,line join=round,line cap=round,fill=fillColor] (411.36,376.38) circle (  1.16);

\path[draw=drawColor,line width= 0.4pt,line join=round,line cap=round,fill=fillColor] (412.01,376.07) circle (  1.16);

\path[draw=drawColor,line width= 0.4pt,line join=round,line cap=round,fill=fillColor] (412.65,375.97) circle (  1.16);

\path[draw=drawColor,line width= 0.4pt,line join=round,line cap=round,fill=fillColor] (413.29,375.79) circle (  1.16);

\path[draw=drawColor,line width= 0.4pt,line join=round,line cap=round,fill=fillColor] (413.92,375.69) circle (  1.16);

\path[draw=drawColor,line width= 0.4pt,line join=round,line cap=round,fill=fillColor] (414.54,375.65) circle (  1.16);

\path[draw=drawColor,line width= 0.4pt,line join=round,line cap=round,fill=fillColor] (415.15,375.59) circle (  1.16);

\path[draw=drawColor,line width= 0.4pt,line join=round,line cap=round,fill=fillColor] (415.76,375.50) circle (  1.16);

\path[draw=drawColor,line width= 0.4pt,line join=round,line cap=round,fill=fillColor] (416.36,375.45) circle (  1.16);

\path[draw=drawColor,line width= 0.4pt,line join=round,line cap=round,fill=fillColor] (416.95,375.40) circle (  1.16);

\path[draw=drawColor,line width= 0.4pt,line join=round,line cap=round,fill=fillColor] (417.54,375.28) circle (  1.16);

\path[draw=drawColor,line width= 0.4pt,line join=round,line cap=round,fill=fillColor] (418.12,375.26) circle (  1.16);

\path[draw=drawColor,line width= 0.4pt,line join=round,line cap=round,fill=fillColor] (418.69,375.06) circle (  1.16);

\path[draw=drawColor,line width= 0.4pt,line join=round,line cap=round,fill=fillColor] (419.26,375.06) circle (  1.16);

\path[draw=drawColor,line width= 0.4pt,line join=round,line cap=round,fill=fillColor] (419.83,374.98) circle (  1.16);

\path[draw=drawColor,line width= 0.4pt,line join=round,line cap=round,fill=fillColor] (420.39,374.84) circle (  1.16);

\path[draw=drawColor,line width= 0.4pt,line join=round,line cap=round,fill=fillColor] (420.94,374.76) circle (  1.16);

\path[draw=drawColor,line width= 0.4pt,line join=round,line cap=round,fill=fillColor] (421.49,374.74) circle (  1.16);

\path[draw=drawColor,line width= 0.4pt,line join=round,line cap=round,fill=fillColor] (422.03,374.71) circle (  1.16);

\path[draw=drawColor,line width= 0.4pt,line join=round,line cap=round,fill=fillColor] (422.57,374.41) circle (  1.16);

\path[draw=drawColor,line width= 0.4pt,line join=round,line cap=round,fill=fillColor] (423.10,374.40) circle (  1.16);

\path[draw=drawColor,line width= 0.4pt,line join=round,line cap=round,fill=fillColor] (423.63,374.27) circle (  1.16);

\path[draw=drawColor,line width= 0.4pt,line join=round,line cap=round,fill=fillColor] (424.16,374.10) circle (  1.16);

\path[draw=drawColor,line width= 0.4pt,line join=round,line cap=round,fill=fillColor] (424.68,374.03) circle (  1.16);

\path[draw=drawColor,line width= 0.4pt,line join=round,line cap=round,fill=fillColor] (425.19,373.95) circle (  1.16);

\path[draw=drawColor,line width= 0.4pt,line join=round,line cap=round,fill=fillColor] (425.70,373.80) circle (  1.16);

\path[draw=drawColor,line width= 0.4pt,line join=round,line cap=round,fill=fillColor] (426.21,373.76) circle (  1.16);

\path[draw=drawColor,line width= 0.4pt,line join=round,line cap=round,fill=fillColor] (426.71,373.67) circle (  1.16);

\path[draw=drawColor,line width= 0.4pt,line join=round,line cap=round,fill=fillColor] (427.21,373.54) circle (  1.16);

\path[draw=drawColor,line width= 0.4pt,line join=round,line cap=round,fill=fillColor] (427.71,373.42) circle (  1.16);

\path[draw=drawColor,line width= 0.4pt,line join=round,line cap=round,fill=fillColor] (428.20,373.16) circle (  1.16);

\path[draw=drawColor,line width= 0.4pt,line join=round,line cap=round,fill=fillColor] (428.68,373.06) circle (  1.16);

\path[draw=drawColor,line width= 0.4pt,line join=round,line cap=round,fill=fillColor] (429.17,372.97) circle (  1.16);

\path[draw=drawColor,line width= 0.4pt,line join=round,line cap=round,fill=fillColor] (429.65,372.96) circle (  1.16);

\path[draw=drawColor,line width= 0.4pt,line join=round,line cap=round,fill=fillColor] (430.12,372.77) circle (  1.16);

\path[draw=drawColor,line width= 0.4pt,line join=round,line cap=round,fill=fillColor] (430.59,372.69) circle (  1.16);

\path[draw=drawColor,line width= 0.4pt,line join=round,line cap=round,fill=fillColor] (431.06,372.69) circle (  1.16);

\path[draw=drawColor,line width= 0.4pt,line join=round,line cap=round,fill=fillColor] (431.53,372.67) circle (  1.16);

\path[draw=drawColor,line width= 0.4pt,line join=round,line cap=round,fill=fillColor] (431.99,372.64) circle (  1.16);

\path[draw=drawColor,line width= 0.4pt,line join=round,line cap=round,fill=fillColor] (432.45,372.63) circle (  1.16);

\path[draw=drawColor,line width= 0.4pt,line join=round,line cap=round,fill=fillColor] (432.90,372.59) circle (  1.16);

\path[draw=drawColor,line width= 0.4pt,line join=round,line cap=round,fill=fillColor] (433.36,372.41) circle (  1.16);

\path[draw=drawColor,line width= 0.4pt,line join=round,line cap=round,fill=fillColor] (433.81,372.28) circle (  1.16);

\path[draw=drawColor,line width= 0.4pt,line join=round,line cap=round,fill=fillColor] (434.25,372.27) circle (  1.16);

\path[draw=drawColor,line width= 0.4pt,line join=round,line cap=round,fill=fillColor] (434.69,372.22) circle (  1.16);

\path[draw=drawColor,line width= 0.4pt,line join=round,line cap=round,fill=fillColor] (435.13,372.07) circle (  1.16);

\path[draw=drawColor,line width= 0.4pt,line join=round,line cap=round,fill=fillColor] (435.57,371.94) circle (  1.16);

\path[draw=drawColor,line width= 0.4pt,line join=round,line cap=round,fill=fillColor] (436.01,371.90) circle (  1.16);

\path[draw=drawColor,line width= 0.4pt,line join=round,line cap=round,fill=fillColor] (436.44,371.77) circle (  1.16);

\path[draw=drawColor,line width= 0.4pt,line join=round,line cap=round,fill=fillColor] (436.87,371.59) circle (  1.16);

\path[draw=drawColor,line width= 0.4pt,line join=round,line cap=round,fill=fillColor] (437.29,371.57) circle (  1.16);

\path[draw=drawColor,line width= 0.4pt,line join=round,line cap=round,fill=fillColor] (437.71,371.57) circle (  1.16);

\path[draw=drawColor,line width= 0.4pt,line join=round,line cap=round,fill=fillColor] (438.14,371.44) circle (  1.16);

\path[draw=drawColor,line width= 0.4pt,line join=round,line cap=round,fill=fillColor] (438.55,371.37) circle (  1.16);

\path[draw=drawColor,line width= 0.4pt,line join=round,line cap=round,fill=fillColor] (438.97,371.36) circle (  1.16);

\path[draw=drawColor,line width= 0.4pt,line join=round,line cap=round,fill=fillColor] (439.38,371.24) circle (  1.16);

\path[draw=drawColor,line width= 0.4pt,line join=round,line cap=round,fill=fillColor] (439.79,371.14) circle (  1.16);

\path[draw=drawColor,line width= 0.4pt,line join=round,line cap=round,fill=fillColor] (440.20,371.06) circle (  1.16);

\path[draw=drawColor,line width= 0.4pt,line join=round,line cap=round,fill=fillColor] (440.60,370.92) circle (  1.16);

\path[draw=drawColor,line width= 0.4pt,line join=round,line cap=round,fill=fillColor] (441.01,370.92) circle (  1.16);

\path[draw=drawColor,line width= 0.4pt,line join=round,line cap=round,fill=fillColor] (441.41,370.83) circle (  1.16);

\path[draw=drawColor,line width= 0.4pt,line join=round,line cap=round,fill=fillColor] (441.81,370.78) circle (  1.16);

\path[draw=drawColor,line width= 0.4pt,line join=round,line cap=round,fill=fillColor] (442.20,370.64) circle (  1.16);

\path[draw=drawColor,line width= 0.4pt,line join=round,line cap=round,fill=fillColor] (442.60,370.49) circle (  1.16);

\path[draw=drawColor,line width= 0.4pt,line join=round,line cap=round,fill=fillColor] (442.99,370.48) circle (  1.16);

\path[draw=drawColor,line width= 0.4pt,line join=round,line cap=round,fill=fillColor] (443.38,370.07) circle (  1.16);

\path[draw=drawColor,line width= 0.4pt,line join=round,line cap=round,fill=fillColor] (443.77,370.05) circle (  1.16);

\path[draw=drawColor,line width= 0.4pt,line join=round,line cap=round,fill=fillColor] (444.15,369.99) circle (  1.16);

\path[draw=drawColor,line width= 0.4pt,line join=round,line cap=round,fill=fillColor] (444.53,369.98) circle (  1.16);

\path[draw=drawColor,line width= 0.4pt,line join=round,line cap=round,fill=fillColor] (444.91,369.90) circle (  1.16);

\path[draw=drawColor,line width= 0.4pt,line join=round,line cap=round,fill=fillColor] (445.29,369.86) circle (  1.16);

\path[draw=drawColor,line width= 0.4pt,line join=round,line cap=round,fill=fillColor] (445.67,369.70) circle (  1.16);

\path[draw=drawColor,line width= 0.4pt,line join=round,line cap=round,fill=fillColor] (446.05,369.59) circle (  1.16);

\path[draw=drawColor,line width= 0.4pt,line join=round,line cap=round,fill=fillColor] (446.42,369.35) circle (  1.16);

\path[draw=drawColor,line width= 0.4pt,line join=round,line cap=round,fill=fillColor] (446.79,369.26) circle (  1.16);

\path[draw=drawColor,line width= 0.4pt,line join=round,line cap=round,fill=fillColor] (447.16,369.08) circle (  1.16);

\path[draw=drawColor,line width= 0.4pt,line join=round,line cap=round,fill=fillColor] (447.53,369.04) circle (  1.16);

\path[draw=drawColor,line width= 0.4pt,line join=round,line cap=round,fill=fillColor] (447.89,369.03) circle (  1.16);

\path[draw=drawColor,line width= 0.4pt,line join=round,line cap=round,fill=fillColor] (448.25,369.02) circle (  1.16);

\path[draw=drawColor,line width= 0.4pt,line join=round,line cap=round,fill=fillColor] (448.62,368.99) circle (  1.16);

\path[draw=drawColor,line width= 0.4pt,line join=round,line cap=round,fill=fillColor] (448.98,368.78) circle (  1.16);

\path[draw=drawColor,line width= 0.4pt,line join=round,line cap=round,fill=fillColor] (449.33,368.70) circle (  1.16);

\path[draw=drawColor,line width= 0.4pt,line join=round,line cap=round,fill=fillColor] (449.69,368.45) circle (  1.16);

\path[draw=drawColor,line width= 0.4pt,line join=round,line cap=round,fill=fillColor] (450.05,368.22) circle (  1.16);

\path[draw=drawColor,line width= 0.4pt,line join=round,line cap=round,fill=fillColor] (450.40,368.13) circle (  1.16);

\path[draw=drawColor,line width= 0.4pt,line join=round,line cap=round,fill=fillColor] (450.75,367.93) circle (  1.16);

\path[draw=drawColor,line width= 0.4pt,line join=round,line cap=round,fill=fillColor] (451.10,367.56) circle (  1.16);

\path[draw=drawColor,line width= 0.4pt,line join=round,line cap=round,fill=fillColor] (451.45,367.55) circle (  1.16);

\path[draw=drawColor,line width= 0.4pt,line join=round,line cap=round,fill=fillColor] (451.79,367.54) circle (  1.16);

\path[draw=drawColor,line width= 0.4pt,line join=round,line cap=round,fill=fillColor] (452.14,367.49) circle (  1.16);

\path[draw=drawColor,line width= 0.4pt,line join=round,line cap=round,fill=fillColor] (452.48,367.42) circle (  1.16);

\path[draw=drawColor,line width= 0.4pt,line join=round,line cap=round,fill=fillColor] (452.82,367.15) circle (  1.16);

\path[draw=drawColor,line width= 0.4pt,line join=round,line cap=round,fill=fillColor] (453.16,367.04) circle (  1.16);

\path[draw=drawColor,line width= 0.4pt,line join=round,line cap=round,fill=fillColor] (453.50,366.87) circle (  1.16);

\path[draw=drawColor,line width= 0.4pt,line join=round,line cap=round,fill=fillColor] (453.84,366.49) circle (  1.16);

\path[draw=drawColor,line width= 0.4pt,line join=round,line cap=round,fill=fillColor] (454.17,366.34) circle (  1.16);

\path[draw=drawColor,line width= 0.4pt,line join=round,line cap=round,fill=fillColor] (454.51,366.23) circle (  1.16);

\path[draw=drawColor,line width= 0.4pt,line join=round,line cap=round,fill=fillColor] (454.84,366.14) circle (  1.16);

\path[draw=drawColor,line width= 0.4pt,line join=round,line cap=round,fill=fillColor] (455.17,366.05) circle (  1.16);

\path[draw=drawColor,line width= 0.4pt,line join=round,line cap=round,fill=fillColor] (455.50,365.65) circle (  1.16);

\path[draw=drawColor,line width= 0.4pt,line join=round,line cap=round,fill=fillColor] (455.83,365.36) circle (  1.16);

\path[draw=drawColor,line width= 0.4pt,line join=round,line cap=round,fill=fillColor] (456.16,365.10) circle (  1.16);

\path[draw=drawColor,line width= 0.4pt,line join=round,line cap=round,fill=fillColor] (456.48,365.00) circle (  1.16);

\path[draw=drawColor,line width= 0.4pt,line join=round,line cap=round,fill=fillColor] (456.80,364.30) circle (  1.16);

\path[draw=drawColor,line width= 0.4pt,line join=round,line cap=round,fill=fillColor] (457.13,364.23) circle (  1.16);

\path[draw=drawColor,line width= 0.4pt,line join=round,line cap=round,fill=fillColor] (457.45,363.93) circle (  1.16);

\path[draw=drawColor,line width= 0.4pt,line join=round,line cap=round,fill=fillColor] (457.77,363.92) circle (  1.16);

\path[draw=drawColor,line width= 0.4pt,line join=round,line cap=round,fill=fillColor] (458.09,363.62) circle (  1.16);

\path[draw=drawColor,line width= 0.4pt,line join=round,line cap=round,fill=fillColor] (458.40,363.25) circle (  1.16);

\path[draw=drawColor,line width= 0.4pt,line join=round,line cap=round,fill=fillColor] (458.72,362.79) circle (  1.16);

\path[draw=drawColor,line width= 0.4pt,line join=round,line cap=round,fill=fillColor] (459.03,362.78) circle (  1.16);

\path[draw=drawColor,line width= 0.4pt,line join=round,line cap=round,fill=fillColor] (459.35,362.54) circle (  1.16);

\path[draw=drawColor,line width= 0.4pt,line join=round,line cap=round,fill=fillColor] (459.66,362.33) circle (  1.16);

\path[draw=drawColor,line width= 0.4pt,line join=round,line cap=round,fill=fillColor] (459.97,362.29) circle (  1.16);

\path[draw=drawColor,line width= 0.4pt,line join=round,line cap=round,fill=fillColor] (460.28,362.24) circle (  1.16);

\path[draw=drawColor,line width= 0.4pt,line join=round,line cap=round,fill=fillColor] (460.59,362.14) circle (  1.16);

\path[draw=drawColor,line width= 0.4pt,line join=round,line cap=round,fill=fillColor] (460.90,361.91) circle (  1.16);

\path[draw=drawColor,line width= 0.4pt,line join=round,line cap=round,fill=fillColor] (461.20,361.85) circle (  1.16);

\path[draw=drawColor,line width= 0.4pt,line join=round,line cap=round,fill=fillColor] (461.51,361.15) circle (  1.16);

\path[draw=drawColor,line width= 0.4pt,line join=round,line cap=round,fill=fillColor] (461.81,360.67) circle (  1.16);

\path[draw=drawColor,line width= 0.4pt,line join=round,line cap=round,fill=fillColor] (462.11,360.60) circle (  1.16);

\path[draw=drawColor,line width= 0.4pt,line join=round,line cap=round,fill=fillColor] (462.42,360.23) circle (  1.16);

\path[draw=drawColor,line width= 0.4pt,line join=round,line cap=round,fill=fillColor] (462.72,359.82) circle (  1.16);

\path[draw=drawColor,line width= 0.4pt,line join=round,line cap=round,fill=fillColor] (463.01,357.06) circle (  1.16);

\path[draw=drawColor,line width= 0.4pt,line join=round,line cap=round,fill=fillColor] (463.31,356.25) circle (  1.16);

\path[draw=drawColor,line width= 0.4pt,line join=round,line cap=round,fill=fillColor] (463.61,351.93) circle (  1.16);

\path[draw=drawColor,line width= 0.4pt,line join=round,line cap=round,fill=fillColor] (463.91,351.93) circle (  1.16);

\path[draw=drawColor,line width= 0.4pt,line join=round,line cap=round,fill=fillColor] (464.20,351.93) circle (  1.16);

\path[draw=drawColor,line width= 0.4pt,line join=round,line cap=round,fill=fillColor] (464.49,351.93) circle (  1.16);

\path[draw=drawColor,line width= 0.4pt,line join=round,line cap=round,fill=fillColor] (464.79,351.93) circle (  1.16);

\path[draw=drawColor,line width= 0.4pt,line join=round,line cap=round,fill=fillColor] (465.08,351.93) circle (  1.16);

\path[draw=drawColor,line width= 0.4pt,line join=round,line cap=round,fill=fillColor] (465.37,351.93) circle (  1.16);

\path[draw=drawColor,line width= 0.4pt,line join=round,line cap=round,fill=fillColor] (465.66,351.93) circle (  1.16);

\path[draw=drawColor,line width= 0.4pt,line join=round,line cap=round,fill=fillColor] (465.95,351.93) circle (  1.16);

\path[draw=drawColor,line width= 0.4pt,line join=round,line cap=round,fill=fillColor] (466.23,351.93) circle (  1.16);

\path[draw=drawColor,line width= 0.4pt,line join=round,line cap=round,fill=fillColor] (466.52,351.93) circle (  1.16);

\path[draw=drawColor,line width= 0.4pt,line join=round,line cap=round,fill=fillColor] (466.81,351.93) circle (  1.16);

\path[draw=drawColor,line width= 0.4pt,line join=round,line cap=round,fill=fillColor] (467.09,351.93) circle (  1.16);

\path[draw=drawColor,line width= 0.4pt,line join=round,line cap=round,fill=fillColor] (467.37,351.93) circle (  1.16);

\path[draw=drawColor,line width= 0.4pt,line join=round,line cap=round,fill=fillColor] (467.66,351.93) circle (  1.16);

\path[draw=drawColor,line width= 0.4pt,line join=round,line cap=round,fill=fillColor] (467.94,351.93) circle (  1.16);

\path[draw=drawColor,line width= 0.4pt,line join=round,line cap=round,fill=fillColor] (468.22,351.93) circle (  1.16);

\path[draw=drawColor,line width= 0.4pt,line join=round,line cap=round,fill=fillColor] (468.50,351.93) circle (  1.16);

\path[draw=drawColor,line width= 0.4pt,line join=round,line cap=round,fill=fillColor] (468.78,351.93) circle (  1.16);

\path[draw=drawColor,line width= 0.4pt,line join=round,line cap=round,fill=fillColor] (469.05,351.93) circle (  1.16);

\path[draw=drawColor,line width= 0.4pt,line join=round,line cap=round,fill=fillColor] (469.33,351.93) circle (  1.16);

\path[draw=drawColor,line width= 0.4pt,line join=round,line cap=round,fill=fillColor] (469.61,351.93) circle (  1.16);

\path[draw=drawColor,line width= 0.4pt,line join=round,line cap=round,fill=fillColor] (469.88,351.93) circle (  1.16);

\path[draw=drawColor,line width= 0.4pt,line join=round,line cap=round,fill=fillColor] (470.15,351.93) circle (  1.16);

\path[draw=drawColor,line width= 0.4pt,line join=round,line cap=round,fill=fillColor] (470.43,351.93) circle (  1.16);

\path[draw=drawColor,line width= 0.4pt,line join=round,line cap=round,fill=fillColor] (470.70,351.93) circle (  1.16);

\path[draw=drawColor,line width= 0.4pt,line join=round,line cap=round,fill=fillColor] (470.97,351.93) circle (  1.16);

\path[draw=drawColor,line width= 0.4pt,line join=round,line cap=round,fill=fillColor] (471.24,351.93) circle (  1.16);

\path[draw=drawColor,line width= 0.4pt,line join=round,line cap=round,fill=fillColor] (471.51,351.93) circle (  1.16);

\path[draw=drawColor,line width= 0.4pt,line join=round,line cap=round,fill=fillColor] (471.78,351.93) circle (  1.16);

\path[draw=drawColor,line width= 0.4pt,line join=round,line cap=round,fill=fillColor] (472.05,351.93) circle (  1.16);
\definecolor[named]{drawColor}{rgb}{0.65,0.34,0.16}
\definecolor[named]{fillColor}{rgb}{0.65,0.34,0.16}

\path[draw=drawColor,line width= 0.4pt,line join=round,line cap=round,fill=fillColor] (331.92,409.07) circle (  1.16);

\path[draw=drawColor,line width= 0.4pt,line join=round,line cap=round,fill=fillColor] (339.28,375.89) circle (  1.16);

\path[draw=drawColor,line width= 0.4pt,line join=round,line cap=round,fill=fillColor] (344.45,373.80) circle (  1.16);

\path[draw=drawColor,line width= 0.4pt,line join=round,line cap=round,fill=fillColor] (348.57,373.65) circle (  1.16);

\path[draw=drawColor,line width= 0.4pt,line join=round,line cap=round,fill=fillColor] (352.04,373.23) circle (  1.16);

\path[draw=drawColor,line width= 0.4pt,line join=round,line cap=round,fill=fillColor] (355.08,372.41) circle (  1.16);

\path[draw=drawColor,line width= 0.4pt,line join=round,line cap=round,fill=fillColor] (357.79,370.68) circle (  1.16);

\path[draw=drawColor,line width= 0.4pt,line join=round,line cap=round,fill=fillColor] (360.26,370.19) circle (  1.16);

\path[draw=drawColor,line width= 0.4pt,line join=round,line cap=round,fill=fillColor] (362.53,369.19) circle (  1.16);

\path[draw=drawColor,line width= 0.4pt,line join=round,line cap=round,fill=fillColor] (364.64,369.05) circle (  1.16);

\path[draw=drawColor,line width= 0.4pt,line join=round,line cap=round,fill=fillColor] (366.61,368.60) circle (  1.16);

\path[draw=drawColor,line width= 0.4pt,line join=round,line cap=round,fill=fillColor] (368.46,368.35) circle (  1.16);

\path[draw=drawColor,line width= 0.4pt,line join=round,line cap=round,fill=fillColor] (370.22,368.20) circle (  1.16);

\path[draw=drawColor,line width= 0.4pt,line join=round,line cap=round,fill=fillColor] (371.89,368.14) circle (  1.16);

\path[draw=drawColor,line width= 0.4pt,line join=round,line cap=round,fill=fillColor] (373.47,367.92) circle (  1.16);

\path[draw=drawColor,line width= 0.4pt,line join=round,line cap=round,fill=fillColor] (374.99,367.89) circle (  1.16);

\path[draw=drawColor,line width= 0.4pt,line join=round,line cap=round,fill=fillColor] (376.45,367.87) circle (  1.16);

\path[draw=drawColor,line width= 0.4pt,line join=round,line cap=round,fill=fillColor] (377.85,367.79) circle (  1.16);

\path[draw=drawColor,line width= 0.4pt,line join=round,line cap=round,fill=fillColor] (379.21,367.65) circle (  1.16);

\path[draw=drawColor,line width= 0.4pt,line join=round,line cap=round,fill=fillColor] (380.51,367.47) circle (  1.16);

\path[draw=drawColor,line width= 0.4pt,line join=round,line cap=round,fill=fillColor] (381.77,367.42) circle (  1.16);

\path[draw=drawColor,line width= 0.4pt,line join=round,line cap=round,fill=fillColor] (382.99,367.10) circle (  1.16);

\path[draw=drawColor,line width= 0.4pt,line join=round,line cap=round,fill=fillColor] (384.18,366.96) circle (  1.16);

\path[draw=drawColor,line width= 0.4pt,line join=round,line cap=round,fill=fillColor] (385.33,366.88) circle (  1.16);

\path[draw=drawColor,line width= 0.4pt,line join=round,line cap=round,fill=fillColor] (386.45,366.71) circle (  1.16);

\path[draw=drawColor,line width= 0.4pt,line join=round,line cap=round,fill=fillColor] (387.54,366.69) circle (  1.16);

\path[draw=drawColor,line width= 0.4pt,line join=round,line cap=round,fill=fillColor] (388.60,366.64) circle (  1.16);

\path[draw=drawColor,line width= 0.4pt,line join=round,line cap=round,fill=fillColor] (389.64,366.59) circle (  1.16);

\path[draw=drawColor,line width= 0.4pt,line join=round,line cap=round,fill=fillColor] (390.65,366.58) circle (  1.16);

\path[draw=drawColor,line width= 0.4pt,line join=round,line cap=round,fill=fillColor] (391.64,366.40) circle (  1.16);

\path[draw=drawColor,line width= 0.4pt,line join=round,line cap=round,fill=fillColor] (392.61,366.27) circle (  1.16);

\path[draw=drawColor,line width= 0.4pt,line join=round,line cap=round,fill=fillColor] (393.56,366.24) circle (  1.16);

\path[draw=drawColor,line width= 0.4pt,line join=round,line cap=round,fill=fillColor] (394.49,366.00) circle (  1.16);

\path[draw=drawColor,line width= 0.4pt,line join=round,line cap=round,fill=fillColor] (395.40,365.92) circle (  1.16);

\path[draw=drawColor,line width= 0.4pt,line join=round,line cap=round,fill=fillColor] (396.29,365.79) circle (  1.16);

\path[draw=drawColor,line width= 0.4pt,line join=round,line cap=round,fill=fillColor] (397.16,365.76) circle (  1.16);

\path[draw=drawColor,line width= 0.4pt,line join=round,line cap=round,fill=fillColor] (398.02,365.74) circle (  1.16);

\path[draw=drawColor,line width= 0.4pt,line join=round,line cap=round,fill=fillColor] (398.86,365.66) circle (  1.16);

\path[draw=drawColor,line width= 0.4pt,line join=round,line cap=round,fill=fillColor] (399.69,365.47) circle (  1.16);

\path[draw=drawColor,line width= 0.4pt,line join=round,line cap=round,fill=fillColor] (400.51,365.46) circle (  1.16);

\path[draw=drawColor,line width= 0.4pt,line join=round,line cap=round,fill=fillColor] (401.31,365.32) circle (  1.16);

\path[draw=drawColor,line width= 0.4pt,line join=round,line cap=round,fill=fillColor] (402.10,365.12) circle (  1.16);

\path[draw=drawColor,line width= 0.4pt,line join=round,line cap=round,fill=fillColor] (402.87,364.98) circle (  1.16);

\path[draw=drawColor,line width= 0.4pt,line join=round,line cap=round,fill=fillColor] (403.64,364.83) circle (  1.16);

\path[draw=drawColor,line width= 0.4pt,line join=round,line cap=round,fill=fillColor] (404.39,364.61) circle (  1.16);

\path[draw=drawColor,line width= 0.4pt,line join=round,line cap=round,fill=fillColor] (405.13,364.48) circle (  1.16);

\path[draw=drawColor,line width= 0.4pt,line join=round,line cap=round,fill=fillColor] (405.86,364.47) circle (  1.16);

\path[draw=drawColor,line width= 0.4pt,line join=round,line cap=round,fill=fillColor] (406.58,364.15) circle (  1.16);

\path[draw=drawColor,line width= 0.4pt,line join=round,line cap=round,fill=fillColor] (407.29,364.07) circle (  1.16);

\path[draw=drawColor,line width= 0.4pt,line join=round,line cap=round,fill=fillColor] (407.99,363.76) circle (  1.16);

\path[draw=drawColor,line width= 0.4pt,line join=round,line cap=round,fill=fillColor] (408.68,363.74) circle (  1.16);

\path[draw=drawColor,line width= 0.4pt,line join=round,line cap=round,fill=fillColor] (409.37,363.72) circle (  1.16);

\path[draw=drawColor,line width= 0.4pt,line join=round,line cap=round,fill=fillColor] (410.04,363.64) circle (  1.16);

\path[draw=drawColor,line width= 0.4pt,line join=round,line cap=round,fill=fillColor] (410.71,363.64) circle (  1.16);

\path[draw=drawColor,line width= 0.4pt,line join=round,line cap=round,fill=fillColor] (411.36,363.36) circle (  1.16);

\path[draw=drawColor,line width= 0.4pt,line join=round,line cap=round,fill=fillColor] (412.01,363.32) circle (  1.16);

\path[draw=drawColor,line width= 0.4pt,line join=round,line cap=round,fill=fillColor] (412.65,363.12) circle (  1.16);

\path[draw=drawColor,line width= 0.4pt,line join=round,line cap=round,fill=fillColor] (413.29,362.83) circle (  1.16);

\path[draw=drawColor,line width= 0.4pt,line join=round,line cap=round,fill=fillColor] (413.92,362.82) circle (  1.16);

\path[draw=drawColor,line width= 0.4pt,line join=round,line cap=round,fill=fillColor] (414.54,362.81) circle (  1.16);

\path[draw=drawColor,line width= 0.4pt,line join=round,line cap=round,fill=fillColor] (415.15,362.51) circle (  1.16);

\path[draw=drawColor,line width= 0.4pt,line join=round,line cap=round,fill=fillColor] (415.76,362.45) circle (  1.16);

\path[draw=drawColor,line width= 0.4pt,line join=round,line cap=round,fill=fillColor] (416.36,362.42) circle (  1.16);

\path[draw=drawColor,line width= 0.4pt,line join=round,line cap=round,fill=fillColor] (416.95,362.35) circle (  1.16);

\path[draw=drawColor,line width= 0.4pt,line join=round,line cap=round,fill=fillColor] (417.54,362.24) circle (  1.16);

\path[draw=drawColor,line width= 0.4pt,line join=round,line cap=round,fill=fillColor] (418.12,362.19) circle (  1.16);

\path[draw=drawColor,line width= 0.4pt,line join=round,line cap=round,fill=fillColor] (418.69,362.12) circle (  1.16);

\path[draw=drawColor,line width= 0.4pt,line join=round,line cap=round,fill=fillColor] (419.26,362.02) circle (  1.16);

\path[draw=drawColor,line width= 0.4pt,line join=round,line cap=round,fill=fillColor] (419.83,361.89) circle (  1.16);

\path[draw=drawColor,line width= 0.4pt,line join=round,line cap=round,fill=fillColor] (420.39,361.85) circle (  1.16);

\path[draw=drawColor,line width= 0.4pt,line join=round,line cap=round,fill=fillColor] (420.94,361.84) circle (  1.16);

\path[draw=drawColor,line width= 0.4pt,line join=round,line cap=round,fill=fillColor] (421.49,361.83) circle (  1.16);

\path[draw=drawColor,line width= 0.4pt,line join=round,line cap=round,fill=fillColor] (422.03,361.83) circle (  1.16);

\path[draw=drawColor,line width= 0.4pt,line join=round,line cap=round,fill=fillColor] (422.57,361.79) circle (  1.16);

\path[draw=drawColor,line width= 0.4pt,line join=round,line cap=round,fill=fillColor] (423.10,361.78) circle (  1.16);

\path[draw=drawColor,line width= 0.4pt,line join=round,line cap=round,fill=fillColor] (423.63,361.76) circle (  1.16);

\path[draw=drawColor,line width= 0.4pt,line join=round,line cap=round,fill=fillColor] (424.16,361.62) circle (  1.16);

\path[draw=drawColor,line width= 0.4pt,line join=round,line cap=round,fill=fillColor] (424.68,361.27) circle (  1.16);

\path[draw=drawColor,line width= 0.4pt,line join=round,line cap=round,fill=fillColor] (425.19,361.05) circle (  1.16);

\path[draw=drawColor,line width= 0.4pt,line join=round,line cap=round,fill=fillColor] (425.70,360.74) circle (  1.16);

\path[draw=drawColor,line width= 0.4pt,line join=round,line cap=round,fill=fillColor] (426.21,360.72) circle (  1.16);

\path[draw=drawColor,line width= 0.4pt,line join=round,line cap=round,fill=fillColor] (426.71,360.70) circle (  1.16);

\path[draw=drawColor,line width= 0.4pt,line join=round,line cap=round,fill=fillColor] (427.21,360.57) circle (  1.16);

\path[draw=drawColor,line width= 0.4pt,line join=round,line cap=round,fill=fillColor] (427.71,360.56) circle (  1.16);

\path[draw=drawColor,line width= 0.4pt,line join=round,line cap=round,fill=fillColor] (428.20,360.49) circle (  1.16);

\path[draw=drawColor,line width= 0.4pt,line join=round,line cap=round,fill=fillColor] (428.68,360.30) circle (  1.16);

\path[draw=drawColor,line width= 0.4pt,line join=round,line cap=round,fill=fillColor] (429.17,360.27) circle (  1.16);

\path[draw=drawColor,line width= 0.4pt,line join=round,line cap=round,fill=fillColor] (429.65,360.21) circle (  1.16);

\path[draw=drawColor,line width= 0.4pt,line join=round,line cap=round,fill=fillColor] (430.12,360.10) circle (  1.16);

\path[draw=drawColor,line width= 0.4pt,line join=round,line cap=round,fill=fillColor] (430.59,360.06) circle (  1.16);

\path[draw=drawColor,line width= 0.4pt,line join=round,line cap=round,fill=fillColor] (431.06,359.79) circle (  1.16);

\path[draw=drawColor,line width= 0.4pt,line join=round,line cap=round,fill=fillColor] (431.53,359.77) circle (  1.16);

\path[draw=drawColor,line width= 0.4pt,line join=round,line cap=round,fill=fillColor] (431.99,359.73) circle (  1.16);

\path[draw=drawColor,line width= 0.4pt,line join=round,line cap=round,fill=fillColor] (432.45,359.60) circle (  1.16);

\path[draw=drawColor,line width= 0.4pt,line join=round,line cap=round,fill=fillColor] (432.90,359.36) circle (  1.16);

\path[draw=drawColor,line width= 0.4pt,line join=round,line cap=round,fill=fillColor] (433.36,359.21) circle (  1.16);

\path[draw=drawColor,line width= 0.4pt,line join=round,line cap=round,fill=fillColor] (433.81,359.08) circle (  1.16);

\path[draw=drawColor,line width= 0.4pt,line join=round,line cap=round,fill=fillColor] (434.25,358.92) circle (  1.16);

\path[draw=drawColor,line width= 0.4pt,line join=round,line cap=round,fill=fillColor] (434.69,358.30) circle (  1.16);

\path[draw=drawColor,line width= 0.4pt,line join=round,line cap=round,fill=fillColor] (435.13,358.25) circle (  1.16);

\path[draw=drawColor,line width= 0.4pt,line join=round,line cap=round,fill=fillColor] (435.57,358.22) circle (  1.16);

\path[draw=drawColor,line width= 0.4pt,line join=round,line cap=round,fill=fillColor] (436.01,358.22) circle (  1.16);

\path[draw=drawColor,line width= 0.4pt,line join=round,line cap=round,fill=fillColor] (436.44,358.12) circle (  1.16);

\path[draw=drawColor,line width= 0.4pt,line join=round,line cap=round,fill=fillColor] (436.87,358.02) circle (  1.16);

\path[draw=drawColor,line width= 0.4pt,line join=round,line cap=round,fill=fillColor] (437.29,358.01) circle (  1.16);

\path[draw=drawColor,line width= 0.4pt,line join=round,line cap=round,fill=fillColor] (437.71,357.59) circle (  1.16);

\path[draw=drawColor,line width= 0.4pt,line join=round,line cap=round,fill=fillColor] (438.14,357.49) circle (  1.16);

\path[draw=drawColor,line width= 0.4pt,line join=round,line cap=round,fill=fillColor] (438.55,357.39) circle (  1.16);

\path[draw=drawColor,line width= 0.4pt,line join=round,line cap=round,fill=fillColor] (438.97,357.32) circle (  1.16);

\path[draw=drawColor,line width= 0.4pt,line join=round,line cap=round,fill=fillColor] (439.38,357.14) circle (  1.16);

\path[draw=drawColor,line width= 0.4pt,line join=round,line cap=round,fill=fillColor] (439.79,357.05) circle (  1.16);

\path[draw=drawColor,line width= 0.4pt,line join=round,line cap=round,fill=fillColor] (440.20,356.36) circle (  1.16);

\path[draw=drawColor,line width= 0.4pt,line join=round,line cap=round,fill=fillColor] (440.60,356.25) circle (  1.16);

\path[draw=drawColor,line width= 0.4pt,line join=round,line cap=round,fill=fillColor] (441.01,355.02) circle (  1.16);

\path[draw=drawColor,line width= 0.4pt,line join=round,line cap=round,fill=fillColor] (441.41,354.60) circle (  1.16);

\path[draw=drawColor,line width= 0.4pt,line join=round,line cap=round,fill=fillColor] (441.81,351.93) circle (  1.16);

\path[draw=drawColor,line width= 0.4pt,line join=round,line cap=round,fill=fillColor] (442.20,351.93) circle (  1.16);

\path[draw=drawColor,line width= 0.4pt,line join=round,line cap=round,fill=fillColor] (442.60,351.93) circle (  1.16);

\path[draw=drawColor,line width= 0.4pt,line join=round,line cap=round,fill=fillColor] (442.99,351.93) circle (  1.16);

\path[draw=drawColor,line width= 0.4pt,line join=round,line cap=round,fill=fillColor] (443.38,351.93) circle (  1.16);

\path[draw=drawColor,line width= 0.4pt,line join=round,line cap=round,fill=fillColor] (443.77,351.93) circle (  1.16);

\path[draw=drawColor,line width= 0.4pt,line join=round,line cap=round,fill=fillColor] (444.15,351.93) circle (  1.16);

\path[draw=drawColor,line width= 0.4pt,line join=round,line cap=round,fill=fillColor] (444.53,351.93) circle (  1.16);

\path[draw=drawColor,line width= 0.4pt,line join=round,line cap=round,fill=fillColor] (444.91,351.93) circle (  1.16);

\path[draw=drawColor,line width= 0.4pt,line join=round,line cap=round,fill=fillColor] (445.29,351.93) circle (  1.16);

\path[draw=drawColor,line width= 0.4pt,line join=round,line cap=round,fill=fillColor] (445.67,351.93) circle (  1.16);

\path[draw=drawColor,line width= 0.4pt,line join=round,line cap=round,fill=fillColor] (446.05,351.93) circle (  1.16);

\path[draw=drawColor,line width= 0.4pt,line join=round,line cap=round,fill=fillColor] (446.42,351.93) circle (  1.16);

\path[draw=drawColor,line width= 0.4pt,line join=round,line cap=round,fill=fillColor] (446.79,351.93) circle (  1.16);

\path[draw=drawColor,line width= 0.4pt,line join=round,line cap=round,fill=fillColor] (447.16,351.93) circle (  1.16);

\path[draw=drawColor,line width= 0.4pt,line join=round,line cap=round,fill=fillColor] (447.53,351.93) circle (  1.16);

\path[draw=drawColor,line width= 0.4pt,line join=round,line cap=round,fill=fillColor] (447.89,351.93) circle (  1.16);

\path[draw=drawColor,line width= 0.4pt,line join=round,line cap=round,fill=fillColor] (448.25,351.93) circle (  1.16);

\path[draw=drawColor,line width= 0.4pt,line join=round,line cap=round,fill=fillColor] (448.62,351.93) circle (  1.16);

\path[draw=drawColor,line width= 0.4pt,line join=round,line cap=round,fill=fillColor] (448.98,351.93) circle (  1.16);

\path[draw=drawColor,line width= 0.4pt,line join=round,line cap=round,fill=fillColor] (449.33,351.93) circle (  1.16);

\path[draw=drawColor,line width= 0.4pt,line join=round,line cap=round,fill=fillColor] (449.69,351.93) circle (  1.16);

\path[draw=drawColor,line width= 0.4pt,line join=round,line cap=round,fill=fillColor] (450.05,351.93) circle (  1.16);

\path[draw=drawColor,line width= 0.4pt,line join=round,line cap=round,fill=fillColor] (450.40,351.93) circle (  1.16);

\path[draw=drawColor,line width= 0.4pt,line join=round,line cap=round,fill=fillColor] (450.75,351.93) circle (  1.16);

\path[draw=drawColor,line width= 0.4pt,line join=round,line cap=round,fill=fillColor] (451.10,351.93) circle (  1.16);

\path[draw=drawColor,line width= 0.4pt,line join=round,line cap=round,fill=fillColor] (451.45,351.93) circle (  1.16);

\path[draw=drawColor,line width= 0.4pt,line join=round,line cap=round,fill=fillColor] (451.79,351.93) circle (  1.16);

\path[draw=drawColor,line width= 0.4pt,line join=round,line cap=round,fill=fillColor] (452.14,351.93) circle (  1.16);

\path[draw=drawColor,line width= 0.4pt,line join=round,line cap=round,fill=fillColor] (452.48,351.93) circle (  1.16);

\path[draw=drawColor,line width= 0.4pt,line join=round,line cap=round,fill=fillColor] (452.82,351.93) circle (  1.16);

\path[draw=drawColor,line width= 0.4pt,line join=round,line cap=round,fill=fillColor] (453.16,351.93) circle (  1.16);

\path[draw=drawColor,line width= 0.4pt,line join=round,line cap=round,fill=fillColor] (453.50,351.93) circle (  1.16);

\path[draw=drawColor,line width= 0.4pt,line join=round,line cap=round,fill=fillColor] (453.84,351.93) circle (  1.16);

\path[draw=drawColor,line width= 0.4pt,line join=round,line cap=round,fill=fillColor] (454.17,351.93) circle (  1.16);

\path[draw=drawColor,line width= 0.4pt,line join=round,line cap=round,fill=fillColor] (454.51,351.93) circle (  1.16);

\path[draw=drawColor,line width= 0.4pt,line join=round,line cap=round,fill=fillColor] (454.84,351.93) circle (  1.16);

\path[draw=drawColor,line width= 0.4pt,line join=round,line cap=round,fill=fillColor] (455.17,351.93) circle (  1.16);

\path[draw=drawColor,line width= 0.4pt,line join=round,line cap=round,fill=fillColor] (455.50,351.93) circle (  1.16);

\path[draw=drawColor,line width= 0.4pt,line join=round,line cap=round,fill=fillColor] (455.83,351.93) circle (  1.16);

\path[draw=drawColor,line width= 0.4pt,line join=round,line cap=round,fill=fillColor] (456.16,351.93) circle (  1.16);

\path[draw=drawColor,line width= 0.4pt,line join=round,line cap=round,fill=fillColor] (456.48,351.93) circle (  1.16);

\path[draw=drawColor,line width= 0.4pt,line join=round,line cap=round,fill=fillColor] (456.80,351.93) circle (  1.16);

\path[draw=drawColor,line width= 0.4pt,line join=round,line cap=round,fill=fillColor] (457.13,351.93) circle (  1.16);

\path[draw=drawColor,line width= 0.4pt,line join=round,line cap=round,fill=fillColor] (457.45,351.93) circle (  1.16);

\path[draw=drawColor,line width= 0.4pt,line join=round,line cap=round,fill=fillColor] (457.77,351.93) circle (  1.16);

\path[draw=drawColor,line width= 0.4pt,line join=round,line cap=round,fill=fillColor] (458.09,351.93) circle (  1.16);

\path[draw=drawColor,line width= 0.4pt,line join=round,line cap=round,fill=fillColor] (458.40,351.93) circle (  1.16);

\path[draw=drawColor,line width= 0.4pt,line join=round,line cap=round,fill=fillColor] (458.72,351.93) circle (  1.16);

\path[draw=drawColor,line width= 0.4pt,line join=round,line cap=round,fill=fillColor] (459.03,351.93) circle (  1.16);

\path[draw=drawColor,line width= 0.4pt,line join=round,line cap=round,fill=fillColor] (459.35,351.93) circle (  1.16);

\path[draw=drawColor,line width= 0.4pt,line join=round,line cap=round,fill=fillColor] (459.66,351.93) circle (  1.16);

\path[draw=drawColor,line width= 0.4pt,line join=round,line cap=round,fill=fillColor] (459.97,351.93) circle (  1.16);

\path[draw=drawColor,line width= 0.4pt,line join=round,line cap=round,fill=fillColor] (460.28,351.93) circle (  1.16);

\path[draw=drawColor,line width= 0.4pt,line join=round,line cap=round,fill=fillColor] (460.59,351.93) circle (  1.16);

\path[draw=drawColor,line width= 0.4pt,line join=round,line cap=round,fill=fillColor] (460.90,351.93) circle (  1.16);

\path[draw=drawColor,line width= 0.4pt,line join=round,line cap=round,fill=fillColor] (461.20,351.93) circle (  1.16);

\path[draw=drawColor,line width= 0.4pt,line join=round,line cap=round,fill=fillColor] (461.51,351.93) circle (  1.16);

\path[draw=drawColor,line width= 0.4pt,line join=round,line cap=round,fill=fillColor] (461.81,351.93) circle (  1.16);

\path[draw=drawColor,line width= 0.4pt,line join=round,line cap=round,fill=fillColor] (462.11,351.93) circle (  1.16);

\path[draw=drawColor,line width= 0.4pt,line join=round,line cap=round,fill=fillColor] (462.42,351.93) circle (  1.16);

\path[draw=drawColor,line width= 0.4pt,line join=round,line cap=round,fill=fillColor] (462.72,351.93) circle (  1.16);

\path[draw=drawColor,line width= 0.4pt,line join=round,line cap=round,fill=fillColor] (463.01,351.93) circle (  1.16);

\path[draw=drawColor,line width= 0.4pt,line join=round,line cap=round,fill=fillColor] (463.31,351.93) circle (  1.16);

\path[draw=drawColor,line width= 0.4pt,line join=round,line cap=round,fill=fillColor] (463.61,351.93) circle (  1.16);

\path[draw=drawColor,line width= 0.4pt,line join=round,line cap=round,fill=fillColor] (463.91,351.93) circle (  1.16);

\path[draw=drawColor,line width= 0.4pt,line join=round,line cap=round,fill=fillColor] (464.20,351.93) circle (  1.16);

\path[draw=drawColor,line width= 0.4pt,line join=round,line cap=round,fill=fillColor] (464.49,351.93) circle (  1.16);

\path[draw=drawColor,line width= 0.4pt,line join=round,line cap=round,fill=fillColor] (464.79,351.93) circle (  1.16);

\path[draw=drawColor,line width= 0.4pt,line join=round,line cap=round,fill=fillColor] (465.08,351.93) circle (  1.16);

\path[draw=drawColor,line width= 0.4pt,line join=round,line cap=round,fill=fillColor] (465.37,351.93) circle (  1.16);

\path[draw=drawColor,line width= 0.4pt,line join=round,line cap=round,fill=fillColor] (465.66,351.93) circle (  1.16);

\path[draw=drawColor,line width= 0.4pt,line join=round,line cap=round,fill=fillColor] (465.95,351.93) circle (  1.16);

\path[draw=drawColor,line width= 0.4pt,line join=round,line cap=round,fill=fillColor] (466.23,351.93) circle (  1.16);

\path[draw=drawColor,line width= 0.4pt,line join=round,line cap=round,fill=fillColor] (466.52,351.93) circle (  1.16);

\path[draw=drawColor,line width= 0.4pt,line join=round,line cap=round,fill=fillColor] (466.81,351.93) circle (  1.16);

\path[draw=drawColor,line width= 0.4pt,line join=round,line cap=round,fill=fillColor] (467.09,351.93) circle (  1.16);

\path[draw=drawColor,line width= 0.4pt,line join=round,line cap=round,fill=fillColor] (467.37,351.93) circle (  1.16);

\path[draw=drawColor,line width= 0.4pt,line join=round,line cap=round,fill=fillColor] (467.66,351.93) circle (  1.16);

\path[draw=drawColor,line width= 0.4pt,line join=round,line cap=round,fill=fillColor] (467.94,351.93) circle (  1.16);

\path[draw=drawColor,line width= 0.4pt,line join=round,line cap=round,fill=fillColor] (468.22,351.93) circle (  1.16);

\path[draw=drawColor,line width= 0.4pt,line join=round,line cap=round,fill=fillColor] (468.50,351.93) circle (  1.16);

\path[draw=drawColor,line width= 0.4pt,line join=round,line cap=round,fill=fillColor] (468.78,351.93) circle (  1.16);

\path[draw=drawColor,line width= 0.4pt,line join=round,line cap=round,fill=fillColor] (469.05,351.93) circle (  1.16);

\path[draw=drawColor,line width= 0.4pt,line join=round,line cap=round,fill=fillColor] (469.33,351.93) circle (  1.16);

\path[draw=drawColor,line width= 0.4pt,line join=round,line cap=round,fill=fillColor] (469.61,351.93) circle (  1.16);

\path[draw=drawColor,line width= 0.4pt,line join=round,line cap=round,fill=fillColor] (469.88,351.93) circle (  1.16);

\path[draw=drawColor,line width= 0.4pt,line join=round,line cap=round,fill=fillColor] (470.15,351.93) circle (  1.16);

\path[draw=drawColor,line width= 0.4pt,line join=round,line cap=round,fill=fillColor] (470.43,351.93) circle (  1.16);

\path[draw=drawColor,line width= 0.4pt,line join=round,line cap=round,fill=fillColor] (470.70,351.93) circle (  1.16);

\path[draw=drawColor,line width= 0.4pt,line join=round,line cap=round,fill=fillColor] (470.97,351.93) circle (  1.16);

\path[draw=drawColor,line width= 0.4pt,line join=round,line cap=round,fill=fillColor] (471.24,351.93) circle (  1.16);

\path[draw=drawColor,line width= 0.4pt,line join=round,line cap=round,fill=fillColor] (471.51,351.93) circle (  1.16);

\path[draw=drawColor,line width= 0.4pt,line join=round,line cap=round,fill=fillColor] (471.78,351.93) circle (  1.16);

\path[draw=drawColor,line width= 0.4pt,line join=round,line cap=round,fill=fillColor] (472.05,351.93) circle (  1.16);
\definecolor[named]{drawColor}{rgb}{0.22,0.49,0.72}
\definecolor[named]{fillColor}{rgb}{0.22,0.49,0.72}

\path[draw=drawColor,line width= 0.4pt,line join=round,line cap=round,fill=fillColor] (331.92,409.07) circle (  1.16);

\path[draw=drawColor,line width= 0.4pt,line join=round,line cap=round,fill=fillColor] (339.28,376.36) circle (  1.16);

\path[draw=drawColor,line width= 0.4pt,line join=round,line cap=round,fill=fillColor] (344.45,374.39) circle (  1.16);

\path[draw=drawColor,line width= 0.4pt,line join=round,line cap=round,fill=fillColor] (348.57,373.71) circle (  1.16);

\path[draw=drawColor,line width= 0.4pt,line join=round,line cap=round,fill=fillColor] (352.04,373.69) circle (  1.16);

\path[draw=drawColor,line width= 0.4pt,line join=round,line cap=round,fill=fillColor] (355.08,373.48) circle (  1.16);

\path[draw=drawColor,line width= 0.4pt,line join=round,line cap=round,fill=fillColor] (357.79,372.40) circle (  1.16);

\path[draw=drawColor,line width= 0.4pt,line join=round,line cap=round,fill=fillColor] (360.26,371.59) circle (  1.16);

\path[draw=drawColor,line width= 0.4pt,line join=round,line cap=round,fill=fillColor] (362.53,371.48) circle (  1.16);

\path[draw=drawColor,line width= 0.4pt,line join=round,line cap=round,fill=fillColor] (364.64,371.34) circle (  1.16);

\path[draw=drawColor,line width= 0.4pt,line join=round,line cap=round,fill=fillColor] (366.61,371.19) circle (  1.16);

\path[draw=drawColor,line width= 0.4pt,line join=round,line cap=round,fill=fillColor] (368.46,370.96) circle (  1.16);

\path[draw=drawColor,line width= 0.4pt,line join=round,line cap=round,fill=fillColor] (370.22,370.50) circle (  1.16);

\path[draw=drawColor,line width= 0.4pt,line join=round,line cap=round,fill=fillColor] (371.89,370.17) circle (  1.16);

\path[draw=drawColor,line width= 0.4pt,line join=round,line cap=round,fill=fillColor] (373.47,370.01) circle (  1.16);

\path[draw=drawColor,line width= 0.4pt,line join=round,line cap=round,fill=fillColor] (374.99,369.71) circle (  1.16);

\path[draw=drawColor,line width= 0.4pt,line join=round,line cap=round,fill=fillColor] (376.45,368.96) circle (  1.16);

\path[draw=drawColor,line width= 0.4pt,line join=round,line cap=round,fill=fillColor] (377.85,368.81) circle (  1.16);

\path[draw=drawColor,line width= 0.4pt,line join=round,line cap=round,fill=fillColor] (379.21,368.66) circle (  1.16);

\path[draw=drawColor,line width= 0.4pt,line join=round,line cap=round,fill=fillColor] (380.51,368.65) circle (  1.16);

\path[draw=drawColor,line width= 0.4pt,line join=round,line cap=round,fill=fillColor] (381.77,368.59) circle (  1.16);

\path[draw=drawColor,line width= 0.4pt,line join=round,line cap=round,fill=fillColor] (382.99,368.42) circle (  1.16);

\path[draw=drawColor,line width= 0.4pt,line join=round,line cap=round,fill=fillColor] (384.18,368.40) circle (  1.16);

\path[draw=drawColor,line width= 0.4pt,line join=round,line cap=round,fill=fillColor] (385.33,368.26) circle (  1.16);

\path[draw=drawColor,line width= 0.4pt,line join=round,line cap=round,fill=fillColor] (386.45,367.42) circle (  1.16);

\path[draw=drawColor,line width= 0.4pt,line join=round,line cap=round,fill=fillColor] (387.54,367.41) circle (  1.16);

\path[draw=drawColor,line width= 0.4pt,line join=round,line cap=round,fill=fillColor] (388.60,367.10) circle (  1.16);

\path[draw=drawColor,line width= 0.4pt,line join=round,line cap=round,fill=fillColor] (389.64,366.88) circle (  1.16);

\path[draw=drawColor,line width= 0.4pt,line join=round,line cap=round,fill=fillColor] (390.65,366.75) circle (  1.16);

\path[draw=drawColor,line width= 0.4pt,line join=round,line cap=round,fill=fillColor] (391.64,366.66) circle (  1.16);

\path[draw=drawColor,line width= 0.4pt,line join=round,line cap=round,fill=fillColor] (392.61,366.28) circle (  1.16);

\path[draw=drawColor,line width= 0.4pt,line join=round,line cap=round,fill=fillColor] (393.56,366.24) circle (  1.16);

\path[draw=drawColor,line width= 0.4pt,line join=round,line cap=round,fill=fillColor] (394.49,366.19) circle (  1.16);

\path[draw=drawColor,line width= 0.4pt,line join=round,line cap=round,fill=fillColor] (395.40,365.91) circle (  1.16);

\path[draw=drawColor,line width= 0.4pt,line join=round,line cap=round,fill=fillColor] (396.29,365.78) circle (  1.16);

\path[draw=drawColor,line width= 0.4pt,line join=round,line cap=round,fill=fillColor] (397.16,365.73) circle (  1.16);

\path[draw=drawColor,line width= 0.4pt,line join=round,line cap=round,fill=fillColor] (398.02,365.61) circle (  1.16);

\path[draw=drawColor,line width= 0.4pt,line join=round,line cap=round,fill=fillColor] (398.86,365.60) circle (  1.16);

\path[draw=drawColor,line width= 0.4pt,line join=round,line cap=round,fill=fillColor] (399.69,365.45) circle (  1.16);

\path[draw=drawColor,line width= 0.4pt,line join=round,line cap=round,fill=fillColor] (400.51,365.32) circle (  1.16);

\path[draw=drawColor,line width= 0.4pt,line join=round,line cap=round,fill=fillColor] (401.31,365.30) circle (  1.16);

\path[draw=drawColor,line width= 0.4pt,line join=round,line cap=round,fill=fillColor] (402.10,365.21) circle (  1.16);

\path[draw=drawColor,line width= 0.4pt,line join=round,line cap=round,fill=fillColor] (402.87,365.05) circle (  1.16);

\path[draw=drawColor,line width= 0.4pt,line join=round,line cap=round,fill=fillColor] (403.64,364.90) circle (  1.16);

\path[draw=drawColor,line width= 0.4pt,line join=round,line cap=round,fill=fillColor] (404.39,364.88) circle (  1.16);

\path[draw=drawColor,line width= 0.4pt,line join=round,line cap=round,fill=fillColor] (405.13,364.84) circle (  1.16);

\path[draw=drawColor,line width= 0.4pt,line join=round,line cap=round,fill=fillColor] (405.86,364.67) circle (  1.16);

\path[draw=drawColor,line width= 0.4pt,line join=round,line cap=round,fill=fillColor] (406.58,364.55) circle (  1.16);

\path[draw=drawColor,line width= 0.4pt,line join=round,line cap=round,fill=fillColor] (407.29,364.48) circle (  1.16);

\path[draw=drawColor,line width= 0.4pt,line join=round,line cap=round,fill=fillColor] (407.99,364.24) circle (  1.16);

\path[draw=drawColor,line width= 0.4pt,line join=round,line cap=round,fill=fillColor] (408.68,364.23) circle (  1.16);

\path[draw=drawColor,line width= 0.4pt,line join=round,line cap=round,fill=fillColor] (409.37,364.14) circle (  1.16);

\path[draw=drawColor,line width= 0.4pt,line join=round,line cap=round,fill=fillColor] (410.04,364.13) circle (  1.16);

\path[draw=drawColor,line width= 0.4pt,line join=round,line cap=round,fill=fillColor] (410.71,364.06) circle (  1.16);

\path[draw=drawColor,line width= 0.4pt,line join=round,line cap=round,fill=fillColor] (411.36,364.00) circle (  1.16);

\path[draw=drawColor,line width= 0.4pt,line join=round,line cap=round,fill=fillColor] (412.01,363.91) circle (  1.16);

\path[draw=drawColor,line width= 0.4pt,line join=round,line cap=round,fill=fillColor] (412.65,363.74) circle (  1.16);

\path[draw=drawColor,line width= 0.4pt,line join=round,line cap=round,fill=fillColor] (413.29,363.73) circle (  1.16);

\path[draw=drawColor,line width= 0.4pt,line join=round,line cap=round,fill=fillColor] (413.92,363.42) circle (  1.16);

\path[draw=drawColor,line width= 0.4pt,line join=round,line cap=round,fill=fillColor] (414.54,363.33) circle (  1.16);

\path[draw=drawColor,line width= 0.4pt,line join=round,line cap=round,fill=fillColor] (415.15,363.28) circle (  1.16);

\path[draw=drawColor,line width= 0.4pt,line join=round,line cap=round,fill=fillColor] (415.76,363.24) circle (  1.16);

\path[draw=drawColor,line width= 0.4pt,line join=round,line cap=round,fill=fillColor] (416.36,363.16) circle (  1.16);

\path[draw=drawColor,line width= 0.4pt,line join=round,line cap=round,fill=fillColor] (416.95,363.14) circle (  1.16);

\path[draw=drawColor,line width= 0.4pt,line join=round,line cap=round,fill=fillColor] (417.54,362.78) circle (  1.16);

\path[draw=drawColor,line width= 0.4pt,line join=round,line cap=round,fill=fillColor] (418.12,362.75) circle (  1.16);

\path[draw=drawColor,line width= 0.4pt,line join=round,line cap=round,fill=fillColor] (418.69,362.72) circle (  1.16);

\path[draw=drawColor,line width= 0.4pt,line join=round,line cap=round,fill=fillColor] (419.26,362.69) circle (  1.16);

\path[draw=drawColor,line width= 0.4pt,line join=round,line cap=round,fill=fillColor] (419.83,362.66) circle (  1.16);

\path[draw=drawColor,line width= 0.4pt,line join=round,line cap=round,fill=fillColor] (420.39,362.51) circle (  1.16);

\path[draw=drawColor,line width= 0.4pt,line join=round,line cap=round,fill=fillColor] (420.94,362.37) circle (  1.16);

\path[draw=drawColor,line width= 0.4pt,line join=round,line cap=round,fill=fillColor] (421.49,362.19) circle (  1.16);

\path[draw=drawColor,line width= 0.4pt,line join=round,line cap=round,fill=fillColor] (422.03,362.18) circle (  1.16);

\path[draw=drawColor,line width= 0.4pt,line join=round,line cap=round,fill=fillColor] (422.57,362.16) circle (  1.16);

\path[draw=drawColor,line width= 0.4pt,line join=round,line cap=round,fill=fillColor] (423.10,362.12) circle (  1.16);

\path[draw=drawColor,line width= 0.4pt,line join=round,line cap=round,fill=fillColor] (423.63,362.01) circle (  1.16);

\path[draw=drawColor,line width= 0.4pt,line join=round,line cap=round,fill=fillColor] (424.16,361.91) circle (  1.16);

\path[draw=drawColor,line width= 0.4pt,line join=round,line cap=round,fill=fillColor] (424.68,361.85) circle (  1.16);

\path[draw=drawColor,line width= 0.4pt,line join=round,line cap=round,fill=fillColor] (425.19,361.85) circle (  1.16);

\path[draw=drawColor,line width= 0.4pt,line join=round,line cap=round,fill=fillColor] (425.70,361.85) circle (  1.16);

\path[draw=drawColor,line width= 0.4pt,line join=round,line cap=round,fill=fillColor] (426.21,361.76) circle (  1.16);

\path[draw=drawColor,line width= 0.4pt,line join=round,line cap=round,fill=fillColor] (426.71,361.75) circle (  1.16);

\path[draw=drawColor,line width= 0.4pt,line join=round,line cap=round,fill=fillColor] (427.21,361.64) circle (  1.16);

\path[draw=drawColor,line width= 0.4pt,line join=round,line cap=round,fill=fillColor] (427.71,361.59) circle (  1.16);

\path[draw=drawColor,line width= 0.4pt,line join=round,line cap=round,fill=fillColor] (428.20,361.59) circle (  1.16);

\path[draw=drawColor,line width= 0.4pt,line join=round,line cap=round,fill=fillColor] (428.68,361.57) circle (  1.16);

\path[draw=drawColor,line width= 0.4pt,line join=round,line cap=round,fill=fillColor] (429.17,361.18) circle (  1.16);

\path[draw=drawColor,line width= 0.4pt,line join=round,line cap=round,fill=fillColor] (429.65,361.05) circle (  1.16);

\path[draw=drawColor,line width= 0.4pt,line join=round,line cap=round,fill=fillColor] (430.12,360.98) circle (  1.16);

\path[draw=drawColor,line width= 0.4pt,line join=round,line cap=round,fill=fillColor] (430.59,360.92) circle (  1.16);

\path[draw=drawColor,line width= 0.4pt,line join=round,line cap=round,fill=fillColor] (431.06,360.76) circle (  1.16);

\path[draw=drawColor,line width= 0.4pt,line join=round,line cap=round,fill=fillColor] (431.53,360.73) circle (  1.16);

\path[draw=drawColor,line width= 0.4pt,line join=round,line cap=round,fill=fillColor] (431.99,360.34) circle (  1.16);

\path[draw=drawColor,line width= 0.4pt,line join=round,line cap=round,fill=fillColor] (432.45,360.21) circle (  1.16);

\path[draw=drawColor,line width= 0.4pt,line join=round,line cap=round,fill=fillColor] (432.90,359.75) circle (  1.16);

\path[draw=drawColor,line width= 0.4pt,line join=round,line cap=round,fill=fillColor] (433.36,359.57) circle (  1.16);

\path[draw=drawColor,line width= 0.4pt,line join=round,line cap=round,fill=fillColor] (433.81,359.25) circle (  1.16);

\path[draw=drawColor,line width= 0.4pt,line join=round,line cap=round,fill=fillColor] (434.25,359.21) circle (  1.16);

\path[draw=drawColor,line width= 0.4pt,line join=round,line cap=round,fill=fillColor] (434.69,359.09) circle (  1.16);

\path[draw=drawColor,line width= 0.4pt,line join=round,line cap=round,fill=fillColor] (435.13,358.59) circle (  1.16);

\path[draw=drawColor,line width= 0.4pt,line join=round,line cap=round,fill=fillColor] (435.57,358.38) circle (  1.16);

\path[draw=drawColor,line width= 0.4pt,line join=round,line cap=round,fill=fillColor] (436.01,358.25) circle (  1.16);

\path[draw=drawColor,line width= 0.4pt,line join=round,line cap=round,fill=fillColor] (436.44,358.11) circle (  1.16);

\path[draw=drawColor,line width= 0.4pt,line join=round,line cap=round,fill=fillColor] (436.87,357.99) circle (  1.16);

\path[draw=drawColor,line width= 0.4pt,line join=round,line cap=round,fill=fillColor] (437.29,357.74) circle (  1.16);

\path[draw=drawColor,line width= 0.4pt,line join=round,line cap=round,fill=fillColor] (437.71,357.49) circle (  1.16);

\path[draw=drawColor,line width= 0.4pt,line join=round,line cap=round,fill=fillColor] (438.14,357.28) circle (  1.16);

\path[draw=drawColor,line width= 0.4pt,line join=round,line cap=round,fill=fillColor] (438.55,357.14) circle (  1.16);

\path[draw=drawColor,line width= 0.4pt,line join=round,line cap=round,fill=fillColor] (438.97,357.06) circle (  1.16);

\path[draw=drawColor,line width= 0.4pt,line join=round,line cap=round,fill=fillColor] (439.38,356.55) circle (  1.16);

\path[draw=drawColor,line width= 0.4pt,line join=round,line cap=round,fill=fillColor] (439.79,356.42) circle (  1.16);

\path[draw=drawColor,line width= 0.4pt,line join=round,line cap=round,fill=fillColor] (440.20,356.38) circle (  1.16);

\path[draw=drawColor,line width= 0.4pt,line join=round,line cap=round,fill=fillColor] (440.60,356.15) circle (  1.16);

\path[draw=drawColor,line width= 0.4pt,line join=round,line cap=round,fill=fillColor] (441.01,355.81) circle (  1.16);

\path[draw=drawColor,line width= 0.4pt,line join=round,line cap=round,fill=fillColor] (441.41,355.74) circle (  1.16);

\path[draw=drawColor,line width= 0.4pt,line join=round,line cap=round,fill=fillColor] (441.81,355.15) circle (  1.16);

\path[draw=drawColor,line width= 0.4pt,line join=round,line cap=round,fill=fillColor] (442.20,355.05) circle (  1.16);

\path[draw=drawColor,line width= 0.4pt,line join=round,line cap=round,fill=fillColor] (442.60,354.67) circle (  1.16);

\path[draw=drawColor,line width= 0.4pt,line join=round,line cap=round,fill=fillColor] (442.99,351.93) circle (  1.16);

\path[draw=drawColor,line width= 0.4pt,line join=round,line cap=round,fill=fillColor] (443.38,351.93) circle (  1.16);

\path[draw=drawColor,line width= 0.4pt,line join=round,line cap=round,fill=fillColor] (443.77,351.93) circle (  1.16);

\path[draw=drawColor,line width= 0.4pt,line join=round,line cap=round,fill=fillColor] (444.15,351.93) circle (  1.16);

\path[draw=drawColor,line width= 0.4pt,line join=round,line cap=round,fill=fillColor] (444.53,351.93) circle (  1.16);

\path[draw=drawColor,line width= 0.4pt,line join=round,line cap=round,fill=fillColor] (444.91,351.93) circle (  1.16);

\path[draw=drawColor,line width= 0.4pt,line join=round,line cap=round,fill=fillColor] (445.29,351.93) circle (  1.16);

\path[draw=drawColor,line width= 0.4pt,line join=round,line cap=round,fill=fillColor] (445.67,351.93) circle (  1.16);

\path[draw=drawColor,line width= 0.4pt,line join=round,line cap=round,fill=fillColor] (446.05,351.93) circle (  1.16);

\path[draw=drawColor,line width= 0.4pt,line join=round,line cap=round,fill=fillColor] (446.42,351.93) circle (  1.16);

\path[draw=drawColor,line width= 0.4pt,line join=round,line cap=round,fill=fillColor] (446.79,351.93) circle (  1.16);

\path[draw=drawColor,line width= 0.4pt,line join=round,line cap=round,fill=fillColor] (447.16,351.93) circle (  1.16);

\path[draw=drawColor,line width= 0.4pt,line join=round,line cap=round,fill=fillColor] (447.53,351.93) circle (  1.16);

\path[draw=drawColor,line width= 0.4pt,line join=round,line cap=round,fill=fillColor] (447.89,351.93) circle (  1.16);

\path[draw=drawColor,line width= 0.4pt,line join=round,line cap=round,fill=fillColor] (448.25,351.93) circle (  1.16);

\path[draw=drawColor,line width= 0.4pt,line join=round,line cap=round,fill=fillColor] (448.62,351.93) circle (  1.16);

\path[draw=drawColor,line width= 0.4pt,line join=round,line cap=round,fill=fillColor] (448.98,351.93) circle (  1.16);

\path[draw=drawColor,line width= 0.4pt,line join=round,line cap=round,fill=fillColor] (449.33,351.93) circle (  1.16);

\path[draw=drawColor,line width= 0.4pt,line join=round,line cap=round,fill=fillColor] (449.69,351.93) circle (  1.16);

\path[draw=drawColor,line width= 0.4pt,line join=round,line cap=round,fill=fillColor] (450.05,351.93) circle (  1.16);

\path[draw=drawColor,line width= 0.4pt,line join=round,line cap=round,fill=fillColor] (450.40,351.93) circle (  1.16);

\path[draw=drawColor,line width= 0.4pt,line join=round,line cap=round,fill=fillColor] (450.75,351.93) circle (  1.16);

\path[draw=drawColor,line width= 0.4pt,line join=round,line cap=round,fill=fillColor] (451.10,351.93) circle (  1.16);

\path[draw=drawColor,line width= 0.4pt,line join=round,line cap=round,fill=fillColor] (451.45,351.93) circle (  1.16);

\path[draw=drawColor,line width= 0.4pt,line join=round,line cap=round,fill=fillColor] (451.79,351.93) circle (  1.16);

\path[draw=drawColor,line width= 0.4pt,line join=round,line cap=round,fill=fillColor] (452.14,351.93) circle (  1.16);

\path[draw=drawColor,line width= 0.4pt,line join=round,line cap=round,fill=fillColor] (452.48,351.93) circle (  1.16);

\path[draw=drawColor,line width= 0.4pt,line join=round,line cap=round,fill=fillColor] (452.82,351.93) circle (  1.16);

\path[draw=drawColor,line width= 0.4pt,line join=round,line cap=round,fill=fillColor] (453.16,351.93) circle (  1.16);

\path[draw=drawColor,line width= 0.4pt,line join=round,line cap=round,fill=fillColor] (453.50,351.93) circle (  1.16);

\path[draw=drawColor,line width= 0.4pt,line join=round,line cap=round,fill=fillColor] (453.84,351.93) circle (  1.16);

\path[draw=drawColor,line width= 0.4pt,line join=round,line cap=round,fill=fillColor] (454.17,351.93) circle (  1.16);

\path[draw=drawColor,line width= 0.4pt,line join=round,line cap=round,fill=fillColor] (454.51,351.93) circle (  1.16);

\path[draw=drawColor,line width= 0.4pt,line join=round,line cap=round,fill=fillColor] (454.84,351.93) circle (  1.16);

\path[draw=drawColor,line width= 0.4pt,line join=round,line cap=round,fill=fillColor] (455.17,351.93) circle (  1.16);

\path[draw=drawColor,line width= 0.4pt,line join=round,line cap=round,fill=fillColor] (455.50,351.93) circle (  1.16);

\path[draw=drawColor,line width= 0.4pt,line join=round,line cap=round,fill=fillColor] (455.83,351.93) circle (  1.16);

\path[draw=drawColor,line width= 0.4pt,line join=round,line cap=round,fill=fillColor] (456.16,351.93) circle (  1.16);

\path[draw=drawColor,line width= 0.4pt,line join=round,line cap=round,fill=fillColor] (456.48,351.93) circle (  1.16);

\path[draw=drawColor,line width= 0.4pt,line join=round,line cap=round,fill=fillColor] (456.80,351.93) circle (  1.16);

\path[draw=drawColor,line width= 0.4pt,line join=round,line cap=round,fill=fillColor] (457.13,351.93) circle (  1.16);

\path[draw=drawColor,line width= 0.4pt,line join=round,line cap=round,fill=fillColor] (457.45,351.93) circle (  1.16);

\path[draw=drawColor,line width= 0.4pt,line join=round,line cap=round,fill=fillColor] (457.77,351.93) circle (  1.16);

\path[draw=drawColor,line width= 0.4pt,line join=round,line cap=round,fill=fillColor] (458.09,351.93) circle (  1.16);

\path[draw=drawColor,line width= 0.4pt,line join=round,line cap=round,fill=fillColor] (458.40,351.93) circle (  1.16);

\path[draw=drawColor,line width= 0.4pt,line join=round,line cap=round,fill=fillColor] (458.72,351.93) circle (  1.16);

\path[draw=drawColor,line width= 0.4pt,line join=round,line cap=round,fill=fillColor] (459.03,351.93) circle (  1.16);

\path[draw=drawColor,line width= 0.4pt,line join=round,line cap=round,fill=fillColor] (459.35,351.93) circle (  1.16);

\path[draw=drawColor,line width= 0.4pt,line join=round,line cap=round,fill=fillColor] (459.66,351.93) circle (  1.16);

\path[draw=drawColor,line width= 0.4pt,line join=round,line cap=round,fill=fillColor] (459.97,351.93) circle (  1.16);

\path[draw=drawColor,line width= 0.4pt,line join=round,line cap=round,fill=fillColor] (460.28,351.93) circle (  1.16);

\path[draw=drawColor,line width= 0.4pt,line join=round,line cap=round,fill=fillColor] (460.59,351.93) circle (  1.16);

\path[draw=drawColor,line width= 0.4pt,line join=round,line cap=round,fill=fillColor] (460.90,351.93) circle (  1.16);

\path[draw=drawColor,line width= 0.4pt,line join=round,line cap=round,fill=fillColor] (461.20,351.93) circle (  1.16);

\path[draw=drawColor,line width= 0.4pt,line join=round,line cap=round,fill=fillColor] (461.51,351.93) circle (  1.16);

\path[draw=drawColor,line width= 0.4pt,line join=round,line cap=round,fill=fillColor] (461.81,351.93) circle (  1.16);

\path[draw=drawColor,line width= 0.4pt,line join=round,line cap=round,fill=fillColor] (462.11,351.93) circle (  1.16);

\path[draw=drawColor,line width= 0.4pt,line join=round,line cap=round,fill=fillColor] (462.42,351.93) circle (  1.16);

\path[draw=drawColor,line width= 0.4pt,line join=round,line cap=round,fill=fillColor] (462.72,351.93) circle (  1.16);

\path[draw=drawColor,line width= 0.4pt,line join=round,line cap=round,fill=fillColor] (463.01,351.93) circle (  1.16);

\path[draw=drawColor,line width= 0.4pt,line join=round,line cap=round,fill=fillColor] (463.31,351.93) circle (  1.16);

\path[draw=drawColor,line width= 0.4pt,line join=round,line cap=round,fill=fillColor] (463.61,351.93) circle (  1.16);

\path[draw=drawColor,line width= 0.4pt,line join=round,line cap=round,fill=fillColor] (463.91,351.93) circle (  1.16);

\path[draw=drawColor,line width= 0.4pt,line join=round,line cap=round,fill=fillColor] (464.20,351.93) circle (  1.16);

\path[draw=drawColor,line width= 0.4pt,line join=round,line cap=round,fill=fillColor] (464.49,351.93) circle (  1.16);

\path[draw=drawColor,line width= 0.4pt,line join=round,line cap=round,fill=fillColor] (464.79,351.93) circle (  1.16);

\path[draw=drawColor,line width= 0.4pt,line join=round,line cap=round,fill=fillColor] (465.08,351.93) circle (  1.16);

\path[draw=drawColor,line width= 0.4pt,line join=round,line cap=round,fill=fillColor] (465.37,351.93) circle (  1.16);

\path[draw=drawColor,line width= 0.4pt,line join=round,line cap=round,fill=fillColor] (465.66,351.93) circle (  1.16);

\path[draw=drawColor,line width= 0.4pt,line join=round,line cap=round,fill=fillColor] (465.95,351.93) circle (  1.16);

\path[draw=drawColor,line width= 0.4pt,line join=round,line cap=round,fill=fillColor] (466.23,351.93) circle (  1.16);

\path[draw=drawColor,line width= 0.4pt,line join=round,line cap=round,fill=fillColor] (466.52,351.93) circle (  1.16);

\path[draw=drawColor,line width= 0.4pt,line join=round,line cap=round,fill=fillColor] (466.81,351.93) circle (  1.16);

\path[draw=drawColor,line width= 0.4pt,line join=round,line cap=round,fill=fillColor] (467.09,351.93) circle (  1.16);

\path[draw=drawColor,line width= 0.4pt,line join=round,line cap=round,fill=fillColor] (467.37,351.93) circle (  1.16);

\path[draw=drawColor,line width= 0.4pt,line join=round,line cap=round,fill=fillColor] (467.66,351.93) circle (  1.16);

\path[draw=drawColor,line width= 0.4pt,line join=round,line cap=round,fill=fillColor] (467.94,351.93) circle (  1.16);

\path[draw=drawColor,line width= 0.4pt,line join=round,line cap=round,fill=fillColor] (468.22,351.93) circle (  1.16);

\path[draw=drawColor,line width= 0.4pt,line join=round,line cap=round,fill=fillColor] (468.50,351.93) circle (  1.16);

\path[draw=drawColor,line width= 0.4pt,line join=round,line cap=round,fill=fillColor] (468.78,351.93) circle (  1.16);

\path[draw=drawColor,line width= 0.4pt,line join=round,line cap=round,fill=fillColor] (469.05,351.93) circle (  1.16);

\path[draw=drawColor,line width= 0.4pt,line join=round,line cap=round,fill=fillColor] (469.33,351.93) circle (  1.16);

\path[draw=drawColor,line width= 0.4pt,line join=round,line cap=round,fill=fillColor] (469.61,351.93) circle (  1.16);

\path[draw=drawColor,line width= 0.4pt,line join=round,line cap=round,fill=fillColor] (469.88,351.93) circle (  1.16);

\path[draw=drawColor,line width= 0.4pt,line join=round,line cap=round,fill=fillColor] (470.15,351.93) circle (  1.16);

\path[draw=drawColor,line width= 0.4pt,line join=round,line cap=round,fill=fillColor] (470.43,351.93) circle (  1.16);

\path[draw=drawColor,line width= 0.4pt,line join=round,line cap=round,fill=fillColor] (470.70,351.93) circle (  1.16);

\path[draw=drawColor,line width= 0.4pt,line join=round,line cap=round,fill=fillColor] (470.97,351.93) circle (  1.16);

\path[draw=drawColor,line width= 0.4pt,line join=round,line cap=round,fill=fillColor] (471.24,351.93) circle (  1.16);

\path[draw=drawColor,line width= 0.4pt,line join=round,line cap=round,fill=fillColor] (471.51,351.93) circle (  1.16);

\path[draw=drawColor,line width= 0.4pt,line join=round,line cap=round,fill=fillColor] (471.78,351.93) circle (  1.16);

\path[draw=drawColor,line width= 0.4pt,line join=round,line cap=round,fill=fillColor] (472.05,351.93) circle (  1.16);
\definecolor[named]{drawColor}{rgb}{0.30,0.69,0.29}
\definecolor[named]{fillColor}{rgb}{0.30,0.69,0.29}

\path[draw=drawColor,line width= 0.4pt,line join=round,line cap=round,fill=fillColor] (331.92,409.07) circle (  1.16);

\path[draw=drawColor,line width= 0.4pt,line join=round,line cap=round,fill=fillColor] (339.28,383.08) circle (  1.16);

\path[draw=drawColor,line width= 0.4pt,line join=round,line cap=round,fill=fillColor] (344.45,381.71) circle (  1.16);

\path[draw=drawColor,line width= 0.4pt,line join=round,line cap=round,fill=fillColor] (348.57,374.61) circle (  1.16);

\path[draw=drawColor,line width= 0.4pt,line join=round,line cap=round,fill=fillColor] (352.04,374.39) circle (  1.16);

\path[draw=drawColor,line width= 0.4pt,line join=round,line cap=round,fill=fillColor] (355.08,373.83) circle (  1.16);

\path[draw=drawColor,line width= 0.4pt,line join=round,line cap=round,fill=fillColor] (357.79,373.53) circle (  1.16);

\path[draw=drawColor,line width= 0.4pt,line join=round,line cap=round,fill=fillColor] (360.26,372.29) circle (  1.16);

\path[draw=drawColor,line width= 0.4pt,line join=round,line cap=round,fill=fillColor] (362.53,371.76) circle (  1.16);

\path[draw=drawColor,line width= 0.4pt,line join=round,line cap=round,fill=fillColor] (364.64,371.72) circle (  1.16);

\path[draw=drawColor,line width= 0.4pt,line join=round,line cap=round,fill=fillColor] (366.61,371.59) circle (  1.16);

\path[draw=drawColor,line width= 0.4pt,line join=round,line cap=round,fill=fillColor] (368.46,371.51) circle (  1.16);

\path[draw=drawColor,line width= 0.4pt,line join=round,line cap=round,fill=fillColor] (370.22,371.38) circle (  1.16);

\path[draw=drawColor,line width= 0.4pt,line join=round,line cap=round,fill=fillColor] (371.89,370.96) circle (  1.16);

\path[draw=drawColor,line width= 0.4pt,line join=round,line cap=round,fill=fillColor] (373.47,369.44) circle (  1.16);

\path[draw=drawColor,line width= 0.4pt,line join=round,line cap=round,fill=fillColor] (374.99,369.21) circle (  1.16);

\path[draw=drawColor,line width= 0.4pt,line join=round,line cap=round,fill=fillColor] (376.45,368.85) circle (  1.16);

\path[draw=drawColor,line width= 0.4pt,line join=round,line cap=round,fill=fillColor] (377.85,368.69) circle (  1.16);

\path[draw=drawColor,line width= 0.4pt,line join=round,line cap=round,fill=fillColor] (379.21,368.42) circle (  1.16);

\path[draw=drawColor,line width= 0.4pt,line join=round,line cap=round,fill=fillColor] (380.51,367.81) circle (  1.16);

\path[draw=drawColor,line width= 0.4pt,line join=round,line cap=round,fill=fillColor] (381.77,367.69) circle (  1.16);

\path[draw=drawColor,line width= 0.4pt,line join=round,line cap=round,fill=fillColor] (382.99,367.68) circle (  1.16);

\path[draw=drawColor,line width= 0.4pt,line join=round,line cap=round,fill=fillColor] (384.18,367.64) circle (  1.16);

\path[draw=drawColor,line width= 0.4pt,line join=round,line cap=round,fill=fillColor] (385.33,367.56) circle (  1.16);

\path[draw=drawColor,line width= 0.4pt,line join=round,line cap=round,fill=fillColor] (386.45,367.42) circle (  1.16);

\path[draw=drawColor,line width= 0.4pt,line join=round,line cap=round,fill=fillColor] (387.54,367.33) circle (  1.16);

\path[draw=drawColor,line width= 0.4pt,line join=round,line cap=round,fill=fillColor] (388.60,367.10) circle (  1.16);

\path[draw=drawColor,line width= 0.4pt,line join=round,line cap=round,fill=fillColor] (389.64,366.59) circle (  1.16);

\path[draw=drawColor,line width= 0.4pt,line join=round,line cap=round,fill=fillColor] (390.65,366.42) circle (  1.16);

\path[draw=drawColor,line width= 0.4pt,line join=round,line cap=round,fill=fillColor] (391.64,366.42) circle (  1.16);

\path[draw=drawColor,line width= 0.4pt,line join=round,line cap=round,fill=fillColor] (392.61,366.39) circle (  1.16);

\path[draw=drawColor,line width= 0.4pt,line join=round,line cap=round,fill=fillColor] (393.56,366.33) circle (  1.16);

\path[draw=drawColor,line width= 0.4pt,line join=round,line cap=round,fill=fillColor] (394.49,366.20) circle (  1.16);

\path[draw=drawColor,line width= 0.4pt,line join=round,line cap=round,fill=fillColor] (395.40,366.09) circle (  1.16);

\path[draw=drawColor,line width= 0.4pt,line join=round,line cap=round,fill=fillColor] (396.29,365.73) circle (  1.16);

\path[draw=drawColor,line width= 0.4pt,line join=round,line cap=round,fill=fillColor] (397.16,365.64) circle (  1.16);

\path[draw=drawColor,line width= 0.4pt,line join=round,line cap=round,fill=fillColor] (398.02,365.61) circle (  1.16);

\path[draw=drawColor,line width= 0.4pt,line join=round,line cap=round,fill=fillColor] (398.86,365.58) circle (  1.16);

\path[draw=drawColor,line width= 0.4pt,line join=round,line cap=round,fill=fillColor] (399.69,365.34) circle (  1.16);

\path[draw=drawColor,line width= 0.4pt,line join=round,line cap=round,fill=fillColor] (400.51,365.23) circle (  1.16);

\path[draw=drawColor,line width= 0.4pt,line join=round,line cap=round,fill=fillColor] (401.31,365.07) circle (  1.16);

\path[draw=drawColor,line width= 0.4pt,line join=round,line cap=round,fill=fillColor] (402.10,365.01) circle (  1.16);

\path[draw=drawColor,line width= 0.4pt,line join=round,line cap=round,fill=fillColor] (402.87,364.90) circle (  1.16);

\path[draw=drawColor,line width= 0.4pt,line join=round,line cap=round,fill=fillColor] (403.64,364.86) circle (  1.16);

\path[draw=drawColor,line width= 0.4pt,line join=round,line cap=round,fill=fillColor] (404.39,364.79) circle (  1.16);

\path[draw=drawColor,line width= 0.4pt,line join=round,line cap=round,fill=fillColor] (405.13,364.76) circle (  1.16);

\path[draw=drawColor,line width= 0.4pt,line join=round,line cap=round,fill=fillColor] (405.86,364.76) circle (  1.16);

\path[draw=drawColor,line width= 0.4pt,line join=round,line cap=round,fill=fillColor] (406.58,364.75) circle (  1.16);

\path[draw=drawColor,line width= 0.4pt,line join=round,line cap=round,fill=fillColor] (407.29,364.74) circle (  1.16);

\path[draw=drawColor,line width= 0.4pt,line join=round,line cap=round,fill=fillColor] (407.99,364.67) circle (  1.16);

\path[draw=drawColor,line width= 0.4pt,line join=round,line cap=round,fill=fillColor] (408.68,364.55) circle (  1.16);

\path[draw=drawColor,line width= 0.4pt,line join=round,line cap=round,fill=fillColor] (409.37,364.16) circle (  1.16);

\path[draw=drawColor,line width= 0.4pt,line join=round,line cap=round,fill=fillColor] (410.04,364.07) circle (  1.16);

\path[draw=drawColor,line width= 0.4pt,line join=round,line cap=round,fill=fillColor] (410.71,363.88) circle (  1.16);

\path[draw=drawColor,line width= 0.4pt,line join=round,line cap=round,fill=fillColor] (411.36,363.74) circle (  1.16);

\path[draw=drawColor,line width= 0.4pt,line join=round,line cap=round,fill=fillColor] (412.01,363.64) circle (  1.16);

\path[draw=drawColor,line width= 0.4pt,line join=round,line cap=round,fill=fillColor] (412.65,363.59) circle (  1.16);

\path[draw=drawColor,line width= 0.4pt,line join=round,line cap=round,fill=fillColor] (413.29,363.52) circle (  1.16);

\path[draw=drawColor,line width= 0.4pt,line join=round,line cap=round,fill=fillColor] (413.92,363.39) circle (  1.16);

\path[draw=drawColor,line width= 0.4pt,line join=round,line cap=round,fill=fillColor] (414.54,363.37) circle (  1.16);

\path[draw=drawColor,line width= 0.4pt,line join=round,line cap=round,fill=fillColor] (415.15,363.33) circle (  1.16);

\path[draw=drawColor,line width= 0.4pt,line join=round,line cap=round,fill=fillColor] (415.76,363.25) circle (  1.16);

\path[draw=drawColor,line width= 0.4pt,line join=round,line cap=round,fill=fillColor] (416.36,362.94) circle (  1.16);

\path[draw=drawColor,line width= 0.4pt,line join=round,line cap=round,fill=fillColor] (416.95,362.83) circle (  1.16);

\path[draw=drawColor,line width= 0.4pt,line join=round,line cap=round,fill=fillColor] (417.54,362.80) circle (  1.16);

\path[draw=drawColor,line width= 0.4pt,line join=round,line cap=round,fill=fillColor] (418.12,362.74) circle (  1.16);

\path[draw=drawColor,line width= 0.4pt,line join=round,line cap=round,fill=fillColor] (418.69,362.72) circle (  1.16);

\path[draw=drawColor,line width= 0.4pt,line join=round,line cap=round,fill=fillColor] (419.26,362.72) circle (  1.16);

\path[draw=drawColor,line width= 0.4pt,line join=round,line cap=round,fill=fillColor] (419.83,362.69) circle (  1.16);

\path[draw=drawColor,line width= 0.4pt,line join=round,line cap=round,fill=fillColor] (420.39,362.63) circle (  1.16);

\path[draw=drawColor,line width= 0.4pt,line join=round,line cap=round,fill=fillColor] (420.94,362.45) circle (  1.16);

\path[draw=drawColor,line width= 0.4pt,line join=round,line cap=round,fill=fillColor] (421.49,362.39) circle (  1.16);

\path[draw=drawColor,line width= 0.4pt,line join=round,line cap=round,fill=fillColor] (422.03,362.34) circle (  1.16);

\path[draw=drawColor,line width= 0.4pt,line join=round,line cap=round,fill=fillColor] (422.57,362.25) circle (  1.16);

\path[draw=drawColor,line width= 0.4pt,line join=round,line cap=round,fill=fillColor] (423.10,362.19) circle (  1.16);

\path[draw=drawColor,line width= 0.4pt,line join=round,line cap=round,fill=fillColor] (423.63,362.16) circle (  1.16);

\path[draw=drawColor,line width= 0.4pt,line join=round,line cap=round,fill=fillColor] (424.16,362.14) circle (  1.16);

\path[draw=drawColor,line width= 0.4pt,line join=round,line cap=round,fill=fillColor] (424.68,362.07) circle (  1.16);

\path[draw=drawColor,line width= 0.4pt,line join=round,line cap=round,fill=fillColor] (425.19,362.05) circle (  1.16);

\path[draw=drawColor,line width= 0.4pt,line join=round,line cap=round,fill=fillColor] (425.70,361.97) circle (  1.16);

\path[draw=drawColor,line width= 0.4pt,line join=round,line cap=round,fill=fillColor] (426.21,361.97) circle (  1.16);

\path[draw=drawColor,line width= 0.4pt,line join=round,line cap=round,fill=fillColor] (426.71,361.92) circle (  1.16);

\path[draw=drawColor,line width= 0.4pt,line join=round,line cap=round,fill=fillColor] (427.21,361.85) circle (  1.16);

\path[draw=drawColor,line width= 0.4pt,line join=round,line cap=round,fill=fillColor] (427.71,361.85) circle (  1.16);

\path[draw=drawColor,line width= 0.4pt,line join=round,line cap=round,fill=fillColor] (428.20,361.78) circle (  1.16);

\path[draw=drawColor,line width= 0.4pt,line join=round,line cap=round,fill=fillColor] (428.68,361.76) circle (  1.16);

\path[draw=drawColor,line width= 0.4pt,line join=round,line cap=round,fill=fillColor] (429.17,361.73) circle (  1.16);

\path[draw=drawColor,line width= 0.4pt,line join=round,line cap=round,fill=fillColor] (429.65,361.36) circle (  1.16);

\path[draw=drawColor,line width= 0.4pt,line join=round,line cap=round,fill=fillColor] (430.12,361.32) circle (  1.16);

\path[draw=drawColor,line width= 0.4pt,line join=round,line cap=round,fill=fillColor] (430.59,361.28) circle (  1.16);

\path[draw=drawColor,line width= 0.4pt,line join=round,line cap=round,fill=fillColor] (431.06,361.26) circle (  1.16);

\path[draw=drawColor,line width= 0.4pt,line join=round,line cap=round,fill=fillColor] (431.53,361.19) circle (  1.16);

\path[draw=drawColor,line width= 0.4pt,line join=round,line cap=round,fill=fillColor] (431.99,361.18) circle (  1.16);

\path[draw=drawColor,line width= 0.4pt,line join=round,line cap=round,fill=fillColor] (432.45,361.06) circle (  1.16);

\path[draw=drawColor,line width= 0.4pt,line join=round,line cap=round,fill=fillColor] (432.90,361.01) circle (  1.16);

\path[draw=drawColor,line width= 0.4pt,line join=round,line cap=round,fill=fillColor] (433.36,360.99) circle (  1.16);

\path[draw=drawColor,line width= 0.4pt,line join=round,line cap=round,fill=fillColor] (433.81,360.98) circle (  1.16);

\path[draw=drawColor,line width= 0.4pt,line join=round,line cap=round,fill=fillColor] (434.25,360.78) circle (  1.16);

\path[draw=drawColor,line width= 0.4pt,line join=round,line cap=round,fill=fillColor] (434.69,360.76) circle (  1.16);

\path[draw=drawColor,line width= 0.4pt,line join=round,line cap=round,fill=fillColor] (435.13,360.53) circle (  1.16);

\path[draw=drawColor,line width= 0.4pt,line join=round,line cap=round,fill=fillColor] (435.57,360.44) circle (  1.16);

\path[draw=drawColor,line width= 0.4pt,line join=round,line cap=round,fill=fillColor] (436.01,360.33) circle (  1.16);

\path[draw=drawColor,line width= 0.4pt,line join=round,line cap=round,fill=fillColor] (436.44,360.26) circle (  1.16);

\path[draw=drawColor,line width= 0.4pt,line join=round,line cap=round,fill=fillColor] (436.87,360.21) circle (  1.16);

\path[draw=drawColor,line width= 0.4pt,line join=round,line cap=round,fill=fillColor] (437.29,360.13) circle (  1.16);

\path[draw=drawColor,line width= 0.4pt,line join=round,line cap=round,fill=fillColor] (437.71,360.08) circle (  1.16);

\path[draw=drawColor,line width= 0.4pt,line join=round,line cap=round,fill=fillColor] (438.14,359.72) circle (  1.16);

\path[draw=drawColor,line width= 0.4pt,line join=round,line cap=round,fill=fillColor] (438.55,359.72) circle (  1.16);

\path[draw=drawColor,line width= 0.4pt,line join=round,line cap=round,fill=fillColor] (438.97,359.21) circle (  1.16);

\path[draw=drawColor,line width= 0.4pt,line join=round,line cap=round,fill=fillColor] (439.38,359.11) circle (  1.16);

\path[draw=drawColor,line width= 0.4pt,line join=round,line cap=round,fill=fillColor] (439.79,359.01) circle (  1.16);

\path[draw=drawColor,line width= 0.4pt,line join=round,line cap=round,fill=fillColor] (440.20,358.84) circle (  1.16);

\path[draw=drawColor,line width= 0.4pt,line join=round,line cap=round,fill=fillColor] (440.60,358.71) circle (  1.16);

\path[draw=drawColor,line width= 0.4pt,line join=round,line cap=round,fill=fillColor] (441.01,358.70) circle (  1.16);

\path[draw=drawColor,line width= 0.4pt,line join=round,line cap=round,fill=fillColor] (441.41,358.63) circle (  1.16);

\path[draw=drawColor,line width= 0.4pt,line join=round,line cap=round,fill=fillColor] (441.81,358.59) circle (  1.16);

\path[draw=drawColor,line width= 0.4pt,line join=round,line cap=round,fill=fillColor] (442.20,358.57) circle (  1.16);

\path[draw=drawColor,line width= 0.4pt,line join=round,line cap=round,fill=fillColor] (442.60,358.46) circle (  1.16);

\path[draw=drawColor,line width= 0.4pt,line join=round,line cap=round,fill=fillColor] (442.99,358.38) circle (  1.16);

\path[draw=drawColor,line width= 0.4pt,line join=round,line cap=round,fill=fillColor] (443.38,358.25) circle (  1.16);

\path[draw=drawColor,line width= 0.4pt,line join=round,line cap=round,fill=fillColor] (443.77,358.18) circle (  1.16);

\path[draw=drawColor,line width= 0.4pt,line join=round,line cap=round,fill=fillColor] (444.15,358.11) circle (  1.16);

\path[draw=drawColor,line width= 0.4pt,line join=round,line cap=round,fill=fillColor] (444.53,357.86) circle (  1.16);

\path[draw=drawColor,line width= 0.4pt,line join=round,line cap=round,fill=fillColor] (444.91,357.49) circle (  1.16);

\path[draw=drawColor,line width= 0.4pt,line join=round,line cap=round,fill=fillColor] (445.29,357.06) circle (  1.16);

\path[draw=drawColor,line width= 0.4pt,line join=round,line cap=round,fill=fillColor] (445.67,356.42) circle (  1.16);

\path[draw=drawColor,line width= 0.4pt,line join=round,line cap=round,fill=fillColor] (446.05,356.15) circle (  1.16);

\path[draw=drawColor,line width= 0.4pt,line join=round,line cap=round,fill=fillColor] (446.42,355.99) circle (  1.16);

\path[draw=drawColor,line width= 0.4pt,line join=round,line cap=round,fill=fillColor] (446.79,355.74) circle (  1.16);

\path[draw=drawColor,line width= 0.4pt,line join=round,line cap=round,fill=fillColor] (447.16,355.70) circle (  1.16);

\path[draw=drawColor,line width= 0.4pt,line join=round,line cap=round,fill=fillColor] (447.53,355.64) circle (  1.16);

\path[draw=drawColor,line width= 0.4pt,line join=round,line cap=round,fill=fillColor] (447.89,351.93) circle (  1.16);

\path[draw=drawColor,line width= 0.4pt,line join=round,line cap=round,fill=fillColor] (448.25,351.93) circle (  1.16);

\path[draw=drawColor,line width= 0.4pt,line join=round,line cap=round,fill=fillColor] (448.62,351.93) circle (  1.16);

\path[draw=drawColor,line width= 0.4pt,line join=round,line cap=round,fill=fillColor] (448.98,351.93) circle (  1.16);

\path[draw=drawColor,line width= 0.4pt,line join=round,line cap=round,fill=fillColor] (449.33,351.93) circle (  1.16);

\path[draw=drawColor,line width= 0.4pt,line join=round,line cap=round,fill=fillColor] (449.69,351.93) circle (  1.16);

\path[draw=drawColor,line width= 0.4pt,line join=round,line cap=round,fill=fillColor] (450.05,351.93) circle (  1.16);

\path[draw=drawColor,line width= 0.4pt,line join=round,line cap=round,fill=fillColor] (450.40,351.93) circle (  1.16);

\path[draw=drawColor,line width= 0.4pt,line join=round,line cap=round,fill=fillColor] (450.75,351.93) circle (  1.16);

\path[draw=drawColor,line width= 0.4pt,line join=round,line cap=round,fill=fillColor] (451.10,351.93) circle (  1.16);

\path[draw=drawColor,line width= 0.4pt,line join=round,line cap=round,fill=fillColor] (451.45,351.93) circle (  1.16);

\path[draw=drawColor,line width= 0.4pt,line join=round,line cap=round,fill=fillColor] (451.79,351.93) circle (  1.16);

\path[draw=drawColor,line width= 0.4pt,line join=round,line cap=round,fill=fillColor] (452.14,351.93) circle (  1.16);

\path[draw=drawColor,line width= 0.4pt,line join=round,line cap=round,fill=fillColor] (452.48,351.93) circle (  1.16);

\path[draw=drawColor,line width= 0.4pt,line join=round,line cap=round,fill=fillColor] (452.82,351.93) circle (  1.16);

\path[draw=drawColor,line width= 0.4pt,line join=round,line cap=round,fill=fillColor] (453.16,351.93) circle (  1.16);

\path[draw=drawColor,line width= 0.4pt,line join=round,line cap=round,fill=fillColor] (453.50,351.93) circle (  1.16);

\path[draw=drawColor,line width= 0.4pt,line join=round,line cap=round,fill=fillColor] (453.84,351.93) circle (  1.16);

\path[draw=drawColor,line width= 0.4pt,line join=round,line cap=round,fill=fillColor] (454.17,351.93) circle (  1.16);

\path[draw=drawColor,line width= 0.4pt,line join=round,line cap=round,fill=fillColor] (454.51,351.93) circle (  1.16);

\path[draw=drawColor,line width= 0.4pt,line join=round,line cap=round,fill=fillColor] (454.84,351.93) circle (  1.16);

\path[draw=drawColor,line width= 0.4pt,line join=round,line cap=round,fill=fillColor] (455.17,351.93) circle (  1.16);

\path[draw=drawColor,line width= 0.4pt,line join=round,line cap=round,fill=fillColor] (455.50,351.93) circle (  1.16);

\path[draw=drawColor,line width= 0.4pt,line join=round,line cap=round,fill=fillColor] (455.83,351.93) circle (  1.16);

\path[draw=drawColor,line width= 0.4pt,line join=round,line cap=round,fill=fillColor] (456.16,351.93) circle (  1.16);

\path[draw=drawColor,line width= 0.4pt,line join=round,line cap=round,fill=fillColor] (456.48,351.93) circle (  1.16);

\path[draw=drawColor,line width= 0.4pt,line join=round,line cap=round,fill=fillColor] (456.80,351.93) circle (  1.16);

\path[draw=drawColor,line width= 0.4pt,line join=round,line cap=round,fill=fillColor] (457.13,351.93) circle (  1.16);

\path[draw=drawColor,line width= 0.4pt,line join=round,line cap=round,fill=fillColor] (457.45,351.93) circle (  1.16);

\path[draw=drawColor,line width= 0.4pt,line join=round,line cap=round,fill=fillColor] (457.77,351.93) circle (  1.16);

\path[draw=drawColor,line width= 0.4pt,line join=round,line cap=round,fill=fillColor] (458.09,351.93) circle (  1.16);

\path[draw=drawColor,line width= 0.4pt,line join=round,line cap=round,fill=fillColor] (458.40,351.93) circle (  1.16);

\path[draw=drawColor,line width= 0.4pt,line join=round,line cap=round,fill=fillColor] (458.72,351.93) circle (  1.16);

\path[draw=drawColor,line width= 0.4pt,line join=round,line cap=round,fill=fillColor] (459.03,351.93) circle (  1.16);

\path[draw=drawColor,line width= 0.4pt,line join=round,line cap=round,fill=fillColor] (459.35,351.93) circle (  1.16);

\path[draw=drawColor,line width= 0.4pt,line join=round,line cap=round,fill=fillColor] (459.66,351.93) circle (  1.16);

\path[draw=drawColor,line width= 0.4pt,line join=round,line cap=round,fill=fillColor] (459.97,351.93) circle (  1.16);

\path[draw=drawColor,line width= 0.4pt,line join=round,line cap=round,fill=fillColor] (460.28,351.93) circle (  1.16);

\path[draw=drawColor,line width= 0.4pt,line join=round,line cap=round,fill=fillColor] (460.59,351.93) circle (  1.16);

\path[draw=drawColor,line width= 0.4pt,line join=round,line cap=round,fill=fillColor] (460.90,351.93) circle (  1.16);

\path[draw=drawColor,line width= 0.4pt,line join=round,line cap=round,fill=fillColor] (461.20,351.93) circle (  1.16);

\path[draw=drawColor,line width= 0.4pt,line join=round,line cap=round,fill=fillColor] (461.51,351.93) circle (  1.16);

\path[draw=drawColor,line width= 0.4pt,line join=round,line cap=round,fill=fillColor] (461.81,351.93) circle (  1.16);

\path[draw=drawColor,line width= 0.4pt,line join=round,line cap=round,fill=fillColor] (462.11,351.93) circle (  1.16);

\path[draw=drawColor,line width= 0.4pt,line join=round,line cap=round,fill=fillColor] (462.42,351.93) circle (  1.16);

\path[draw=drawColor,line width= 0.4pt,line join=round,line cap=round,fill=fillColor] (462.72,351.93) circle (  1.16);

\path[draw=drawColor,line width= 0.4pt,line join=round,line cap=round,fill=fillColor] (463.01,351.93) circle (  1.16);

\path[draw=drawColor,line width= 0.4pt,line join=round,line cap=round,fill=fillColor] (463.31,351.93) circle (  1.16);

\path[draw=drawColor,line width= 0.4pt,line join=round,line cap=round,fill=fillColor] (463.61,351.93) circle (  1.16);

\path[draw=drawColor,line width= 0.4pt,line join=round,line cap=round,fill=fillColor] (463.91,351.93) circle (  1.16);

\path[draw=drawColor,line width= 0.4pt,line join=round,line cap=round,fill=fillColor] (464.20,351.93) circle (  1.16);

\path[draw=drawColor,line width= 0.4pt,line join=round,line cap=round,fill=fillColor] (464.49,351.93) circle (  1.16);

\path[draw=drawColor,line width= 0.4pt,line join=round,line cap=round,fill=fillColor] (464.79,351.93) circle (  1.16);

\path[draw=drawColor,line width= 0.4pt,line join=round,line cap=round,fill=fillColor] (465.08,351.93) circle (  1.16);

\path[draw=drawColor,line width= 0.4pt,line join=round,line cap=round,fill=fillColor] (465.37,351.93) circle (  1.16);

\path[draw=drawColor,line width= 0.4pt,line join=round,line cap=round,fill=fillColor] (465.66,351.93) circle (  1.16);

\path[draw=drawColor,line width= 0.4pt,line join=round,line cap=round,fill=fillColor] (465.95,351.93) circle (  1.16);

\path[draw=drawColor,line width= 0.4pt,line join=round,line cap=round,fill=fillColor] (466.23,351.93) circle (  1.16);

\path[draw=drawColor,line width= 0.4pt,line join=round,line cap=round,fill=fillColor] (466.52,351.93) circle (  1.16);

\path[draw=drawColor,line width= 0.4pt,line join=round,line cap=round,fill=fillColor] (466.81,351.93) circle (  1.16);

\path[draw=drawColor,line width= 0.4pt,line join=round,line cap=round,fill=fillColor] (467.09,351.93) circle (  1.16);

\path[draw=drawColor,line width= 0.4pt,line join=round,line cap=round,fill=fillColor] (467.37,351.93) circle (  1.16);

\path[draw=drawColor,line width= 0.4pt,line join=round,line cap=round,fill=fillColor] (467.66,351.93) circle (  1.16);

\path[draw=drawColor,line width= 0.4pt,line join=round,line cap=round,fill=fillColor] (467.94,351.93) circle (  1.16);

\path[draw=drawColor,line width= 0.4pt,line join=round,line cap=round,fill=fillColor] (468.22,351.93) circle (  1.16);

\path[draw=drawColor,line width= 0.4pt,line join=round,line cap=round,fill=fillColor] (468.50,351.93) circle (  1.16);

\path[draw=drawColor,line width= 0.4pt,line join=round,line cap=round,fill=fillColor] (468.78,351.93) circle (  1.16);

\path[draw=drawColor,line width= 0.4pt,line join=round,line cap=round,fill=fillColor] (469.05,351.93) circle (  1.16);

\path[draw=drawColor,line width= 0.4pt,line join=round,line cap=round,fill=fillColor] (469.33,351.93) circle (  1.16);

\path[draw=drawColor,line width= 0.4pt,line join=round,line cap=round,fill=fillColor] (469.61,351.93) circle (  1.16);

\path[draw=drawColor,line width= 0.4pt,line join=round,line cap=round,fill=fillColor] (469.88,351.93) circle (  1.16);

\path[draw=drawColor,line width= 0.4pt,line join=round,line cap=round,fill=fillColor] (470.15,351.93) circle (  1.16);

\path[draw=drawColor,line width= 0.4pt,line join=round,line cap=round,fill=fillColor] (470.43,351.93) circle (  1.16);

\path[draw=drawColor,line width= 0.4pt,line join=round,line cap=round,fill=fillColor] (470.70,351.93) circle (  1.16);

\path[draw=drawColor,line width= 0.4pt,line join=round,line cap=round,fill=fillColor] (470.97,351.93) circle (  1.16);

\path[draw=drawColor,line width= 0.4pt,line join=round,line cap=round,fill=fillColor] (471.24,351.93) circle (  1.16);

\path[draw=drawColor,line width= 0.4pt,line join=round,line cap=round,fill=fillColor] (471.51,351.93) circle (  1.16);

\path[draw=drawColor,line width= 0.4pt,line join=round,line cap=round,fill=fillColor] (471.78,351.93) circle (  1.16);

\path[draw=drawColor,line width= 0.4pt,line join=round,line cap=round,fill=fillColor] (472.05,351.93) circle (  1.16);
\definecolor[named]{drawColor}{rgb}{0.60,0.31,0.64}
\definecolor[named]{fillColor}{rgb}{0.60,0.31,0.64}

\path[draw=drawColor,line width= 0.4pt,line join=round,line cap=round,fill=fillColor] (331.92,409.07) circle (  1.16);

\path[draw=drawColor,line width= 0.4pt,line join=round,line cap=round,fill=fillColor] (339.28,384.73) circle (  1.16);

\path[draw=drawColor,line width= 0.4pt,line join=round,line cap=round,fill=fillColor] (344.45,376.26) circle (  1.16);

\path[draw=drawColor,line width= 0.4pt,line join=round,line cap=round,fill=fillColor] (348.57,373.69) circle (  1.16);

\path[draw=drawColor,line width= 0.4pt,line join=round,line cap=round,fill=fillColor] (352.04,373.21) circle (  1.16);

\path[draw=drawColor,line width= 0.4pt,line join=round,line cap=round,fill=fillColor] (355.08,372.68) circle (  1.16);

\path[draw=drawColor,line width= 0.4pt,line join=round,line cap=round,fill=fillColor] (357.79,371.73) circle (  1.16);

\path[draw=drawColor,line width= 0.4pt,line join=round,line cap=round,fill=fillColor] (360.26,371.49) circle (  1.16);

\path[draw=drawColor,line width= 0.4pt,line join=round,line cap=round,fill=fillColor] (362.53,371.48) circle (  1.16);

\path[draw=drawColor,line width= 0.4pt,line join=round,line cap=round,fill=fillColor] (364.64,371.48) circle (  1.16);

\path[draw=drawColor,line width= 0.4pt,line join=round,line cap=round,fill=fillColor] (366.61,370.87) circle (  1.16);

\path[draw=drawColor,line width= 0.4pt,line join=round,line cap=round,fill=fillColor] (368.46,370.57) circle (  1.16);

\path[draw=drawColor,line width= 0.4pt,line join=round,line cap=round,fill=fillColor] (370.22,370.37) circle (  1.16);

\path[draw=drawColor,line width= 0.4pt,line join=round,line cap=round,fill=fillColor] (371.89,370.05) circle (  1.16);

\path[draw=drawColor,line width= 0.4pt,line join=round,line cap=round,fill=fillColor] (373.47,369.89) circle (  1.16);

\path[draw=drawColor,line width= 0.4pt,line join=round,line cap=round,fill=fillColor] (374.99,369.88) circle (  1.16);

\path[draw=drawColor,line width= 0.4pt,line join=round,line cap=round,fill=fillColor] (376.45,369.76) circle (  1.16);

\path[draw=drawColor,line width= 0.4pt,line join=round,line cap=round,fill=fillColor] (377.85,369.54) circle (  1.16);

\path[draw=drawColor,line width= 0.4pt,line join=round,line cap=round,fill=fillColor] (379.21,368.89) circle (  1.16);

\path[draw=drawColor,line width= 0.4pt,line join=round,line cap=round,fill=fillColor] (380.51,368.78) circle (  1.16);

\path[draw=drawColor,line width= 0.4pt,line join=round,line cap=round,fill=fillColor] (381.77,368.51) circle (  1.16);

\path[draw=drawColor,line width= 0.4pt,line join=round,line cap=round,fill=fillColor] (382.99,368.30) circle (  1.16);

\path[draw=drawColor,line width= 0.4pt,line join=round,line cap=round,fill=fillColor] (384.18,367.75) circle (  1.16);

\path[draw=drawColor,line width= 0.4pt,line join=round,line cap=round,fill=fillColor] (385.33,367.68) circle (  1.16);

\path[draw=drawColor,line width= 0.4pt,line join=round,line cap=round,fill=fillColor] (386.45,367.67) circle (  1.16);

\path[draw=drawColor,line width= 0.4pt,line join=round,line cap=round,fill=fillColor] (387.54,367.65) circle (  1.16);

\path[draw=drawColor,line width= 0.4pt,line join=round,line cap=round,fill=fillColor] (388.60,367.64) circle (  1.16);

\path[draw=drawColor,line width= 0.4pt,line join=round,line cap=round,fill=fillColor] (389.64,367.59) circle (  1.16);

\path[draw=drawColor,line width= 0.4pt,line join=round,line cap=round,fill=fillColor] (390.65,367.54) circle (  1.16);

\path[draw=drawColor,line width= 0.4pt,line join=round,line cap=round,fill=fillColor] (391.64,367.42) circle (  1.16);

\path[draw=drawColor,line width= 0.4pt,line join=round,line cap=round,fill=fillColor] (392.61,366.98) circle (  1.16);

\path[draw=drawColor,line width= 0.4pt,line join=round,line cap=round,fill=fillColor] (393.56,366.85) circle (  1.16);

\path[draw=drawColor,line width= 0.4pt,line join=round,line cap=round,fill=fillColor] (394.49,366.71) circle (  1.16);

\path[draw=drawColor,line width= 0.4pt,line join=round,line cap=round,fill=fillColor] (395.40,366.68) circle (  1.16);

\path[draw=drawColor,line width= 0.4pt,line join=round,line cap=round,fill=fillColor] (396.29,366.66) circle (  1.16);

\path[draw=drawColor,line width= 0.4pt,line join=round,line cap=round,fill=fillColor] (397.16,366.61) circle (  1.16);

\path[draw=drawColor,line width= 0.4pt,line join=round,line cap=round,fill=fillColor] (398.02,366.46) circle (  1.16);

\path[draw=drawColor,line width= 0.4pt,line join=round,line cap=round,fill=fillColor] (398.86,366.28) circle (  1.16);

\path[draw=drawColor,line width= 0.4pt,line join=round,line cap=round,fill=fillColor] (399.69,366.19) circle (  1.16);

\path[draw=drawColor,line width= 0.4pt,line join=round,line cap=round,fill=fillColor] (400.51,366.08) circle (  1.16);

\path[draw=drawColor,line width= 0.4pt,line join=round,line cap=round,fill=fillColor] (401.31,365.93) circle (  1.16);

\path[draw=drawColor,line width= 0.4pt,line join=round,line cap=round,fill=fillColor] (402.10,365.92) circle (  1.16);

\path[draw=drawColor,line width= 0.4pt,line join=round,line cap=round,fill=fillColor] (402.87,365.78) circle (  1.16);

\path[draw=drawColor,line width= 0.4pt,line join=round,line cap=round,fill=fillColor] (403.64,365.60) circle (  1.16);

\path[draw=drawColor,line width= 0.4pt,line join=round,line cap=round,fill=fillColor] (404.39,365.60) circle (  1.16);

\path[draw=drawColor,line width= 0.4pt,line join=round,line cap=round,fill=fillColor] (405.13,365.47) circle (  1.16);

\path[draw=drawColor,line width= 0.4pt,line join=round,line cap=round,fill=fillColor] (405.86,365.44) circle (  1.16);

\path[draw=drawColor,line width= 0.4pt,line join=round,line cap=round,fill=fillColor] (406.58,365.41) circle (  1.16);

\path[draw=drawColor,line width= 0.4pt,line join=round,line cap=round,fill=fillColor] (407.29,365.36) circle (  1.16);

\path[draw=drawColor,line width= 0.4pt,line join=round,line cap=round,fill=fillColor] (407.99,365.35) circle (  1.16);

\path[draw=drawColor,line width= 0.4pt,line join=round,line cap=round,fill=fillColor] (408.68,365.35) circle (  1.16);

\path[draw=drawColor,line width= 0.4pt,line join=round,line cap=round,fill=fillColor] (409.37,365.33) circle (  1.16);

\path[draw=drawColor,line width= 0.4pt,line join=round,line cap=round,fill=fillColor] (410.04,365.15) circle (  1.16);

\path[draw=drawColor,line width= 0.4pt,line join=round,line cap=round,fill=fillColor] (410.71,365.14) circle (  1.16);

\path[draw=drawColor,line width= 0.4pt,line join=round,line cap=round,fill=fillColor] (411.36,364.86) circle (  1.16);

\path[draw=drawColor,line width= 0.4pt,line join=round,line cap=round,fill=fillColor] (412.01,364.85) circle (  1.16);

\path[draw=drawColor,line width= 0.4pt,line join=round,line cap=round,fill=fillColor] (412.65,364.77) circle (  1.16);

\path[draw=drawColor,line width= 0.4pt,line join=round,line cap=round,fill=fillColor] (413.29,364.39) circle (  1.16);

\path[draw=drawColor,line width= 0.4pt,line join=round,line cap=round,fill=fillColor] (413.92,364.25) circle (  1.16);

\path[draw=drawColor,line width= 0.4pt,line join=round,line cap=round,fill=fillColor] (414.54,364.22) circle (  1.16);

\path[draw=drawColor,line width= 0.4pt,line join=round,line cap=round,fill=fillColor] (415.15,364.06) circle (  1.16);

\path[draw=drawColor,line width= 0.4pt,line join=round,line cap=round,fill=fillColor] (415.76,364.01) circle (  1.16);

\path[draw=drawColor,line width= 0.4pt,line join=round,line cap=round,fill=fillColor] (416.36,363.91) circle (  1.16);

\path[draw=drawColor,line width= 0.4pt,line join=round,line cap=round,fill=fillColor] (416.95,363.60) circle (  1.16);

\path[draw=drawColor,line width= 0.4pt,line join=round,line cap=round,fill=fillColor] (417.54,363.58) circle (  1.16);

\path[draw=drawColor,line width= 0.4pt,line join=round,line cap=round,fill=fillColor] (418.12,363.49) circle (  1.16);

\path[draw=drawColor,line width= 0.4pt,line join=round,line cap=round,fill=fillColor] (418.69,363.47) circle (  1.16);

\path[draw=drawColor,line width= 0.4pt,line join=round,line cap=round,fill=fillColor] (419.26,363.39) circle (  1.16);

\path[draw=drawColor,line width= 0.4pt,line join=round,line cap=round,fill=fillColor] (419.83,363.15) circle (  1.16);

\path[draw=drawColor,line width= 0.4pt,line join=round,line cap=round,fill=fillColor] (420.39,363.13) circle (  1.16);

\path[draw=drawColor,line width= 0.4pt,line join=round,line cap=round,fill=fillColor] (420.94,363.02) circle (  1.16);

\path[draw=drawColor,line width= 0.4pt,line join=round,line cap=round,fill=fillColor] (421.49,362.90) circle (  1.16);

\path[draw=drawColor,line width= 0.4pt,line join=round,line cap=round,fill=fillColor] (422.03,362.72) circle (  1.16);

\path[draw=drawColor,line width= 0.4pt,line join=round,line cap=round,fill=fillColor] (422.57,362.65) circle (  1.16);

\path[draw=drawColor,line width= 0.4pt,line join=round,line cap=round,fill=fillColor] (423.10,362.61) circle (  1.16);

\path[draw=drawColor,line width= 0.4pt,line join=round,line cap=round,fill=fillColor] (423.63,362.53) circle (  1.16);

\path[draw=drawColor,line width= 0.4pt,line join=round,line cap=round,fill=fillColor] (424.16,362.48) circle (  1.16);

\path[draw=drawColor,line width= 0.4pt,line join=round,line cap=round,fill=fillColor] (424.68,362.43) circle (  1.16);

\path[draw=drawColor,line width= 0.4pt,line join=round,line cap=round,fill=fillColor] (425.19,362.43) circle (  1.16);

\path[draw=drawColor,line width= 0.4pt,line join=round,line cap=round,fill=fillColor] (425.70,362.34) circle (  1.16);

\path[draw=drawColor,line width= 0.4pt,line join=round,line cap=round,fill=fillColor] (426.21,362.24) circle (  1.16);

\path[draw=drawColor,line width= 0.4pt,line join=round,line cap=round,fill=fillColor] (426.71,362.22) circle (  1.16);

\path[draw=drawColor,line width= 0.4pt,line join=round,line cap=round,fill=fillColor] (427.21,362.16) circle (  1.16);

\path[draw=drawColor,line width= 0.4pt,line join=round,line cap=round,fill=fillColor] (427.71,362.16) circle (  1.16);

\path[draw=drawColor,line width= 0.4pt,line join=round,line cap=round,fill=fillColor] (428.20,362.14) circle (  1.16);

\path[draw=drawColor,line width= 0.4pt,line join=round,line cap=round,fill=fillColor] (428.68,361.99) circle (  1.16);

\path[draw=drawColor,line width= 0.4pt,line join=round,line cap=round,fill=fillColor] (429.17,361.92) circle (  1.16);

\path[draw=drawColor,line width= 0.4pt,line join=round,line cap=round,fill=fillColor] (429.65,361.91) circle (  1.16);

\path[draw=drawColor,line width= 0.4pt,line join=round,line cap=round,fill=fillColor] (430.12,361.84) circle (  1.16);

\path[draw=drawColor,line width= 0.4pt,line join=round,line cap=round,fill=fillColor] (430.59,361.66) circle (  1.16);

\path[draw=drawColor,line width= 0.4pt,line join=round,line cap=round,fill=fillColor] (431.06,361.61) circle (  1.16);

\path[draw=drawColor,line width= 0.4pt,line join=round,line cap=round,fill=fillColor] (431.53,361.47) circle (  1.16);

\path[draw=drawColor,line width= 0.4pt,line join=round,line cap=round,fill=fillColor] (431.99,361.45) circle (  1.16);

\path[draw=drawColor,line width= 0.4pt,line join=round,line cap=round,fill=fillColor] (432.45,361.45) circle (  1.16);

\path[draw=drawColor,line width= 0.4pt,line join=round,line cap=round,fill=fillColor] (432.90,361.45) circle (  1.16);

\path[draw=drawColor,line width= 0.4pt,line join=round,line cap=round,fill=fillColor] (433.36,361.42) circle (  1.16);

\path[draw=drawColor,line width= 0.4pt,line join=round,line cap=round,fill=fillColor] (433.81,361.34) circle (  1.16);

\path[draw=drawColor,line width= 0.4pt,line join=round,line cap=round,fill=fillColor] (434.25,361.27) circle (  1.16);

\path[draw=drawColor,line width= 0.4pt,line join=round,line cap=round,fill=fillColor] (434.69,361.26) circle (  1.16);

\path[draw=drawColor,line width= 0.4pt,line join=round,line cap=round,fill=fillColor] (435.13,361.20) circle (  1.16);

\path[draw=drawColor,line width= 0.4pt,line join=round,line cap=round,fill=fillColor] (435.57,361.18) circle (  1.16);

\path[draw=drawColor,line width= 0.4pt,line join=round,line cap=round,fill=fillColor] (436.01,361.15) circle (  1.16);

\path[draw=drawColor,line width= 0.4pt,line join=round,line cap=round,fill=fillColor] (436.44,361.13) circle (  1.16);

\path[draw=drawColor,line width= 0.4pt,line join=round,line cap=round,fill=fillColor] (436.87,361.05) circle (  1.16);

\path[draw=drawColor,line width= 0.4pt,line join=round,line cap=round,fill=fillColor] (437.29,360.42) circle (  1.16);

\path[draw=drawColor,line width= 0.4pt,line join=round,line cap=round,fill=fillColor] (437.71,360.33) circle (  1.16);

\path[draw=drawColor,line width= 0.4pt,line join=round,line cap=round,fill=fillColor] (438.14,360.02) circle (  1.16);

\path[draw=drawColor,line width= 0.4pt,line join=round,line cap=round,fill=fillColor] (438.55,359.99) circle (  1.16);

\path[draw=drawColor,line width= 0.4pt,line join=round,line cap=round,fill=fillColor] (438.97,359.90) circle (  1.16);

\path[draw=drawColor,line width= 0.4pt,line join=round,line cap=round,fill=fillColor] (439.38,359.60) circle (  1.16);

\path[draw=drawColor,line width= 0.4pt,line join=round,line cap=round,fill=fillColor] (439.79,359.10) circle (  1.16);

\path[draw=drawColor,line width= 0.4pt,line join=round,line cap=round,fill=fillColor] (440.20,358.95) circle (  1.16);

\path[draw=drawColor,line width= 0.4pt,line join=round,line cap=round,fill=fillColor] (440.60,358.83) circle (  1.16);

\path[draw=drawColor,line width= 0.4pt,line join=round,line cap=round,fill=fillColor] (441.01,358.25) circle (  1.16);

\path[draw=drawColor,line width= 0.4pt,line join=round,line cap=round,fill=fillColor] (441.41,358.22) circle (  1.16);

\path[draw=drawColor,line width= 0.4pt,line join=round,line cap=round,fill=fillColor] (441.81,357.93) circle (  1.16);

\path[draw=drawColor,line width= 0.4pt,line join=round,line cap=round,fill=fillColor] (442.20,357.82) circle (  1.16);

\path[draw=drawColor,line width= 0.4pt,line join=round,line cap=round,fill=fillColor] (442.60,357.79) circle (  1.16);

\path[draw=drawColor,line width= 0.4pt,line join=round,line cap=round,fill=fillColor] (442.99,357.72) circle (  1.16);

\path[draw=drawColor,line width= 0.4pt,line join=round,line cap=round,fill=fillColor] (443.38,357.42) circle (  1.16);

\path[draw=drawColor,line width= 0.4pt,line join=round,line cap=round,fill=fillColor] (443.77,357.33) circle (  1.16);

\path[draw=drawColor,line width= 0.4pt,line join=round,line cap=round,fill=fillColor] (444.15,357.12) circle (  1.16);

\path[draw=drawColor,line width= 0.4pt,line join=round,line cap=round,fill=fillColor] (444.53,357.06) circle (  1.16);

\path[draw=drawColor,line width= 0.4pt,line join=round,line cap=round,fill=fillColor] (444.91,356.34) circle (  1.16);

\path[draw=drawColor,line width= 0.4pt,line join=round,line cap=round,fill=fillColor] (445.29,356.18) circle (  1.16);

\path[draw=drawColor,line width= 0.4pt,line join=round,line cap=round,fill=fillColor] (445.67,355.70) circle (  1.16);

\path[draw=drawColor,line width= 0.4pt,line join=round,line cap=round,fill=fillColor] (446.05,355.60) circle (  1.16);

\path[draw=drawColor,line width= 0.4pt,line join=round,line cap=round,fill=fillColor] (446.42,355.19) circle (  1.16);

\path[draw=drawColor,line width= 0.4pt,line join=round,line cap=round,fill=fillColor] (446.79,351.93) circle (  1.16);

\path[draw=drawColor,line width= 0.4pt,line join=round,line cap=round,fill=fillColor] (447.16,351.93) circle (  1.16);

\path[draw=drawColor,line width= 0.4pt,line join=round,line cap=round,fill=fillColor] (447.53,351.93) circle (  1.16);

\path[draw=drawColor,line width= 0.4pt,line join=round,line cap=round,fill=fillColor] (447.89,351.93) circle (  1.16);

\path[draw=drawColor,line width= 0.4pt,line join=round,line cap=round,fill=fillColor] (448.25,351.93) circle (  1.16);

\path[draw=drawColor,line width= 0.4pt,line join=round,line cap=round,fill=fillColor] (448.62,351.93) circle (  1.16);

\path[draw=drawColor,line width= 0.4pt,line join=round,line cap=round,fill=fillColor] (448.98,351.93) circle (  1.16);

\path[draw=drawColor,line width= 0.4pt,line join=round,line cap=round,fill=fillColor] (449.33,351.93) circle (  1.16);

\path[draw=drawColor,line width= 0.4pt,line join=round,line cap=round,fill=fillColor] (449.69,351.93) circle (  1.16);

\path[draw=drawColor,line width= 0.4pt,line join=round,line cap=round,fill=fillColor] (450.05,351.93) circle (  1.16);

\path[draw=drawColor,line width= 0.4pt,line join=round,line cap=round,fill=fillColor] (450.40,351.93) circle (  1.16);

\path[draw=drawColor,line width= 0.4pt,line join=round,line cap=round,fill=fillColor] (450.75,351.93) circle (  1.16);

\path[draw=drawColor,line width= 0.4pt,line join=round,line cap=round,fill=fillColor] (451.10,351.93) circle (  1.16);

\path[draw=drawColor,line width= 0.4pt,line join=round,line cap=round,fill=fillColor] (451.45,351.93) circle (  1.16);

\path[draw=drawColor,line width= 0.4pt,line join=round,line cap=round,fill=fillColor] (451.79,351.93) circle (  1.16);

\path[draw=drawColor,line width= 0.4pt,line join=round,line cap=round,fill=fillColor] (452.14,351.93) circle (  1.16);

\path[draw=drawColor,line width= 0.4pt,line join=round,line cap=round,fill=fillColor] (452.48,351.93) circle (  1.16);

\path[draw=drawColor,line width= 0.4pt,line join=round,line cap=round,fill=fillColor] (452.82,351.93) circle (  1.16);

\path[draw=drawColor,line width= 0.4pt,line join=round,line cap=round,fill=fillColor] (453.16,351.93) circle (  1.16);

\path[draw=drawColor,line width= 0.4pt,line join=round,line cap=round,fill=fillColor] (453.50,351.93) circle (  1.16);

\path[draw=drawColor,line width= 0.4pt,line join=round,line cap=round,fill=fillColor] (453.84,351.93) circle (  1.16);

\path[draw=drawColor,line width= 0.4pt,line join=round,line cap=round,fill=fillColor] (454.17,351.93) circle (  1.16);

\path[draw=drawColor,line width= 0.4pt,line join=round,line cap=round,fill=fillColor] (454.51,351.93) circle (  1.16);

\path[draw=drawColor,line width= 0.4pt,line join=round,line cap=round,fill=fillColor] (454.84,351.93) circle (  1.16);

\path[draw=drawColor,line width= 0.4pt,line join=round,line cap=round,fill=fillColor] (455.17,351.93) circle (  1.16);

\path[draw=drawColor,line width= 0.4pt,line join=round,line cap=round,fill=fillColor] (455.50,351.93) circle (  1.16);

\path[draw=drawColor,line width= 0.4pt,line join=round,line cap=round,fill=fillColor] (455.83,351.93) circle (  1.16);

\path[draw=drawColor,line width= 0.4pt,line join=round,line cap=round,fill=fillColor] (456.16,351.93) circle (  1.16);

\path[draw=drawColor,line width= 0.4pt,line join=round,line cap=round,fill=fillColor] (456.48,351.93) circle (  1.16);

\path[draw=drawColor,line width= 0.4pt,line join=round,line cap=round,fill=fillColor] (456.80,351.93) circle (  1.16);

\path[draw=drawColor,line width= 0.4pt,line join=round,line cap=round,fill=fillColor] (457.13,351.93) circle (  1.16);

\path[draw=drawColor,line width= 0.4pt,line join=round,line cap=round,fill=fillColor] (457.45,351.93) circle (  1.16);

\path[draw=drawColor,line width= 0.4pt,line join=round,line cap=round,fill=fillColor] (457.77,351.93) circle (  1.16);

\path[draw=drawColor,line width= 0.4pt,line join=round,line cap=round,fill=fillColor] (458.09,351.93) circle (  1.16);

\path[draw=drawColor,line width= 0.4pt,line join=round,line cap=round,fill=fillColor] (458.40,351.93) circle (  1.16);

\path[draw=drawColor,line width= 0.4pt,line join=round,line cap=round,fill=fillColor] (458.72,351.93) circle (  1.16);

\path[draw=drawColor,line width= 0.4pt,line join=round,line cap=round,fill=fillColor] (459.03,351.93) circle (  1.16);

\path[draw=drawColor,line width= 0.4pt,line join=round,line cap=round,fill=fillColor] (459.35,351.93) circle (  1.16);

\path[draw=drawColor,line width= 0.4pt,line join=round,line cap=round,fill=fillColor] (459.66,351.93) circle (  1.16);

\path[draw=drawColor,line width= 0.4pt,line join=round,line cap=round,fill=fillColor] (459.97,351.93) circle (  1.16);

\path[draw=drawColor,line width= 0.4pt,line join=round,line cap=round,fill=fillColor] (460.28,351.93) circle (  1.16);

\path[draw=drawColor,line width= 0.4pt,line join=round,line cap=round,fill=fillColor] (460.59,351.93) circle (  1.16);

\path[draw=drawColor,line width= 0.4pt,line join=round,line cap=round,fill=fillColor] (460.90,351.93) circle (  1.16);

\path[draw=drawColor,line width= 0.4pt,line join=round,line cap=round,fill=fillColor] (461.20,351.93) circle (  1.16);

\path[draw=drawColor,line width= 0.4pt,line join=round,line cap=round,fill=fillColor] (461.51,351.93) circle (  1.16);

\path[draw=drawColor,line width= 0.4pt,line join=round,line cap=round,fill=fillColor] (461.81,351.93) circle (  1.16);

\path[draw=drawColor,line width= 0.4pt,line join=round,line cap=round,fill=fillColor] (462.11,351.93) circle (  1.16);

\path[draw=drawColor,line width= 0.4pt,line join=round,line cap=round,fill=fillColor] (462.42,351.93) circle (  1.16);

\path[draw=drawColor,line width= 0.4pt,line join=round,line cap=round,fill=fillColor] (462.72,351.93) circle (  1.16);

\path[draw=drawColor,line width= 0.4pt,line join=round,line cap=round,fill=fillColor] (463.01,351.93) circle (  1.16);

\path[draw=drawColor,line width= 0.4pt,line join=round,line cap=round,fill=fillColor] (463.31,351.93) circle (  1.16);

\path[draw=drawColor,line width= 0.4pt,line join=round,line cap=round,fill=fillColor] (463.61,351.93) circle (  1.16);

\path[draw=drawColor,line width= 0.4pt,line join=round,line cap=round,fill=fillColor] (463.91,351.93) circle (  1.16);

\path[draw=drawColor,line width= 0.4pt,line join=round,line cap=round,fill=fillColor] (464.20,351.93) circle (  1.16);

\path[draw=drawColor,line width= 0.4pt,line join=round,line cap=round,fill=fillColor] (464.49,351.93) circle (  1.16);

\path[draw=drawColor,line width= 0.4pt,line join=round,line cap=round,fill=fillColor] (464.79,351.93) circle (  1.16);

\path[draw=drawColor,line width= 0.4pt,line join=round,line cap=round,fill=fillColor] (465.08,351.93) circle (  1.16);

\path[draw=drawColor,line width= 0.4pt,line join=round,line cap=round,fill=fillColor] (465.37,351.93) circle (  1.16);

\path[draw=drawColor,line width= 0.4pt,line join=round,line cap=round,fill=fillColor] (465.66,351.93) circle (  1.16);

\path[draw=drawColor,line width= 0.4pt,line join=round,line cap=round,fill=fillColor] (465.95,351.93) circle (  1.16);

\path[draw=drawColor,line width= 0.4pt,line join=round,line cap=round,fill=fillColor] (466.23,351.93) circle (  1.16);

\path[draw=drawColor,line width= 0.4pt,line join=round,line cap=round,fill=fillColor] (466.52,351.93) circle (  1.16);

\path[draw=drawColor,line width= 0.4pt,line join=round,line cap=round,fill=fillColor] (466.81,351.93) circle (  1.16);

\path[draw=drawColor,line width= 0.4pt,line join=round,line cap=round,fill=fillColor] (467.09,351.93) circle (  1.16);

\path[draw=drawColor,line width= 0.4pt,line join=round,line cap=round,fill=fillColor] (467.37,351.93) circle (  1.16);

\path[draw=drawColor,line width= 0.4pt,line join=round,line cap=round,fill=fillColor] (467.66,351.93) circle (  1.16);

\path[draw=drawColor,line width= 0.4pt,line join=round,line cap=round,fill=fillColor] (467.94,351.93) circle (  1.16);

\path[draw=drawColor,line width= 0.4pt,line join=round,line cap=round,fill=fillColor] (468.22,351.93) circle (  1.16);

\path[draw=drawColor,line width= 0.4pt,line join=round,line cap=round,fill=fillColor] (468.50,351.93) circle (  1.16);

\path[draw=drawColor,line width= 0.4pt,line join=round,line cap=round,fill=fillColor] (468.78,351.93) circle (  1.16);

\path[draw=drawColor,line width= 0.4pt,line join=round,line cap=round,fill=fillColor] (469.05,351.93) circle (  1.16);

\path[draw=drawColor,line width= 0.4pt,line join=round,line cap=round,fill=fillColor] (469.33,351.93) circle (  1.16);

\path[draw=drawColor,line width= 0.4pt,line join=round,line cap=round,fill=fillColor] (469.61,351.93) circle (  1.16);

\path[draw=drawColor,line width= 0.4pt,line join=round,line cap=round,fill=fillColor] (469.88,351.93) circle (  1.16);

\path[draw=drawColor,line width= 0.4pt,line join=round,line cap=round,fill=fillColor] (470.15,351.93) circle (  1.16);

\path[draw=drawColor,line width= 0.4pt,line join=round,line cap=round,fill=fillColor] (470.43,351.93) circle (  1.16);

\path[draw=drawColor,line width= 0.4pt,line join=round,line cap=round,fill=fillColor] (470.70,351.93) circle (  1.16);

\path[draw=drawColor,line width= 0.4pt,line join=round,line cap=round,fill=fillColor] (470.97,351.93) circle (  1.16);

\path[draw=drawColor,line width= 0.4pt,line join=round,line cap=round,fill=fillColor] (471.24,351.93) circle (  1.16);

\path[draw=drawColor,line width= 0.4pt,line join=round,line cap=round,fill=fillColor] (471.51,351.93) circle (  1.16);

\path[draw=drawColor,line width= 0.4pt,line join=round,line cap=round,fill=fillColor] (471.78,351.93) circle (  1.16);

\path[draw=drawColor,line width= 0.4pt,line join=round,line cap=round,fill=fillColor] (472.05,351.93) circle (  1.16);
\definecolor[named]{drawColor}{rgb}{0.00,0.00,0.00}
\definecolor[named]{fillColor}{rgb}{0.00,0.00,0.00}

\path[draw=drawColor,line width= 0.6pt,line join=round,fill=fillColor] (324.91,431.32) -- (479.05,431.32);

\node[text=drawColor,anchor=base east,inner sep=0pt, outer sep=0pt, scale=  0.85] at (475.55,430.44) {infeasible solutions};

\path[draw=drawColor,line width= 0.6pt,line join=round,line cap=round] (324.91,343.99) rectangle (479.05,439.26);
\end{scope}
\begin{scope}
\path[clip] (  0.00,  0.00) rectangle (505.89,614.29);
\definecolor[named]{drawColor}{rgb}{0.00,0.00,0.00}

\node[text=drawColor,anchor=base east,inner sep=0pt, outer sep=0pt, scale=  0.80] at (319.51,349.17) {0.00};

\node[text=drawColor,anchor=base east,inner sep=0pt, outer sep=0pt, scale=  0.80] at (319.51,366.28) {0.01};

\node[text=drawColor,anchor=base east,inner sep=0pt, outer sep=0pt, scale=  0.80] at (319.51,378.42) {0.05};

\node[text=drawColor,anchor=base east,inner sep=0pt, outer sep=0pt, scale=  0.80] at (319.51,395.60) {0.20};

\node[text=drawColor,anchor=base east,inner sep=0pt, outer sep=0pt, scale=  0.80] at (319.51,412.18) {0.50};

\node[text=drawColor,anchor=base east,inner sep=0pt, outer sep=0pt, scale=  0.80] at (319.51,421.30) {0.75};

\node[text=drawColor,anchor=base east,inner sep=0pt, outer sep=0pt, scale=  0.80] at (319.51,428.56) {1.00};
\end{scope}
\begin{scope}
\path[clip] (  0.00,  0.00) rectangle (505.89,614.29);
\definecolor[named]{drawColor}{rgb}{0.00,0.00,0.00}

\path[draw=drawColor,line width= 0.6pt,line join=round] (321.91,351.93) --
	(324.91,351.93);

\path[draw=drawColor,line width= 0.6pt,line join=round] (321.91,369.03) --
	(324.91,369.03);

\path[draw=drawColor,line width= 0.6pt,line join=round] (321.91,381.18) --
	(324.91,381.18);

\path[draw=drawColor,line width= 0.6pt,line join=round] (321.91,398.36) --
	(324.91,398.36);

\path[draw=drawColor,line width= 0.6pt,line join=round] (321.91,414.94) --
	(324.91,414.94);

\path[draw=drawColor,line width= 0.6pt,line join=round] (321.91,424.06) --
	(324.91,424.06);

\path[draw=drawColor,line width= 0.6pt,line join=round] (321.91,431.32) --
	(324.91,431.32);
\end{scope}
\begin{scope}
\path[clip] (  0.00,  0.00) rectangle (505.89,614.29);
\definecolor[named]{drawColor}{rgb}{0.00,0.00,0.00}

\path[draw=drawColor,line width= 0.6pt,line join=round] (407.99,340.99) --
	(407.99,343.99);

\path[draw=drawColor,line width= 0.6pt,line join=round] (435.13,340.99) --
	(435.13,343.99);

\path[draw=drawColor,line width= 0.6pt,line join=round] (454.17,340.99) --
	(454.17,343.99);

\path[draw=drawColor,line width= 0.6pt,line join=round] (469.33,340.99) --
	(469.33,343.99);
\end{scope}
\begin{scope}
\path[clip] (  0.00,  0.00) rectangle (505.89,614.29);
\definecolor[named]{drawColor}{rgb}{0.00,0.00,0.00}

\node[text=drawColor,rotate= 50.00,anchor=base east,inner sep=0pt, outer sep=0pt, scale=  0.80] at (412.21,335.05) {50};

\node[text=drawColor,rotate= 50.00,anchor=base east,inner sep=0pt, outer sep=0pt, scale=  0.80] at (439.35,335.05) {100};

\node[text=drawColor,rotate= 50.00,anchor=base east,inner sep=0pt, outer sep=0pt, scale=  0.80] at (458.39,335.05) {150};

\node[text=drawColor,rotate= 50.00,anchor=base east,inner sep=0pt, outer sep=0pt, scale=  0.80] at (473.55,335.05) {200};
\end{scope}
\begin{scope}
\path[clip] (  0.00,  0.00) rectangle (505.89,614.29);
\definecolor[named]{drawColor}{rgb}{0.00,0.00,0.00}

\node[text=drawColor,anchor=base,inner sep=0pt, outer sep=0pt, scale=  0.80] at (401.98,315.55) {\# Instances};
\end{scope}
\begin{scope}
\path[clip] (  0.00,  0.00) rectangle (505.89,614.29);
\definecolor[named]{drawColor}{rgb}{0.00,0.00,0.00}

\node[text=drawColor,rotate= 90.00,anchor=base,inner sep=0pt, outer sep=0pt, scale=  0.80] at (295.29,391.62) {1-(Best/Algorithm)};
\end{scope}
\begin{scope}
\path[clip] (  0.00,  0.00) rectangle (505.89,614.29);
\definecolor[named]{drawColor}{rgb}{0.00,0.00,0.00}

\node[text=drawColor,anchor=base,inner sep=0pt, outer sep=0pt, scale=  1.20] at (401.98,446.46) {\Primal};
\end{scope}
\begin{scope}
\path[clip] ( 20.84,153.57) rectangle (232.11,307.15);
\definecolor[named]{drawColor}{rgb}{1.00,1.00,1.00}
\definecolor[named]{fillColor}{rgb}{1.00,1.00,1.00}

\path[draw=drawColor,line width= 0.6pt,line join=round,line cap=round,fill=fillColor] ( 20.84,153.57) rectangle (232.11,307.15);
\end{scope}
\begin{scope}
\path[clip] ( 71.96,190.42) rectangle (226.11,285.68);
\definecolor[named]{fillColor}{rgb}{1.00,1.00,1.00}

\path[fill=fillColor] ( 71.96,190.42) rectangle (226.11,285.68);
\definecolor[named]{drawColor}{rgb}{0.98,0.98,0.98}

\path[draw=drawColor,line width= 0.6pt,line join=round] ( 71.96,206.91) --
	(226.11,206.91);

\path[draw=drawColor,line width= 0.6pt,line join=round] ( 71.96,221.53) --
	(226.11,221.53);

\path[draw=drawColor,line width= 0.6pt,line join=round] ( 71.96,236.19) --
	(226.11,236.19);

\path[draw=drawColor,line width= 0.6pt,line join=round] ( 71.96,253.07) --
	(226.11,253.07);

\path[draw=drawColor,line width= 0.6pt,line join=round] ( 71.96,265.93) --
	(226.11,265.93);

\path[draw=drawColor,line width= 0.6pt,line join=round] ( 71.96,274.11) --
	(226.11,274.11);

\path[draw=drawColor,line width= 0.6pt,line join=round] (128.38,190.42) --
	(128.38,285.68);

\path[draw=drawColor,line width= 0.6pt,line join=round] (142.08,190.42) --
	(142.08,285.68);

\path[draw=drawColor,line width= 0.6pt,line join=round] (169.49,190.42) --
	(169.49,285.68);

\path[draw=drawColor,line width= 0.6pt,line join=round] (192.80,190.42) --
	(192.80,285.68);

\path[draw=drawColor,line width= 0.6pt,line join=round] (210.07,190.42) --
	(210.07,285.68);
\definecolor[named]{drawColor}{rgb}{0.75,0.75,0.75}

\path[draw=drawColor,line width= 0.6pt,dash pattern=on 1pt off 3pt ,line join=round] ( 71.96,198.35) --
	(226.11,198.35);

\path[draw=drawColor,line width= 0.6pt,dash pattern=on 1pt off 3pt ,line join=round] ( 71.96,215.46) --
	(226.11,215.46);

\path[draw=drawColor,line width= 0.6pt,dash pattern=on 1pt off 3pt ,line join=round] ( 71.96,227.60) --
	(226.11,227.60);

\path[draw=drawColor,line width= 0.6pt,dash pattern=on 1pt off 3pt ,line join=round] ( 71.96,244.78) --
	(226.11,244.78);

\path[draw=drawColor,line width= 0.6pt,dash pattern=on 1pt off 3pt ,line join=round] ( 71.96,261.37) --
	(226.11,261.37);

\path[draw=drawColor,line width= 0.6pt,dash pattern=on 1pt off 3pt ,line join=round] ( 71.96,270.48) --
	(226.11,270.48);

\path[draw=drawColor,line width= 0.6pt,dash pattern=on 1pt off 3pt ,line join=round] ( 71.96,277.74) --
	(226.11,277.74);

\path[draw=drawColor,line width= 0.6pt,dash pattern=on 1pt off 3pt ,line join=round] (155.79,190.42) --
	(155.79,285.68);

\path[draw=drawColor,line width= 0.6pt,dash pattern=on 1pt off 3pt ,line join=round] (183.19,190.42) --
	(183.19,285.68);

\path[draw=drawColor,line width= 0.6pt,dash pattern=on 1pt off 3pt ,line join=round] (202.42,190.42) --
	(202.42,285.68);

\path[draw=drawColor,line width= 0.6pt,dash pattern=on 1pt off 3pt ,line join=round] (217.72,190.42) --
	(217.72,285.68);
\definecolor[named]{drawColor}{rgb}{0.89,0.10,0.11}
\definecolor[named]{fillColor}{rgb}{0.89,0.10,0.11}

\path[draw=drawColor,line width= 0.4pt,line join=round,line cap=round,fill=fillColor] ( 78.97,245.74) circle (  1.16);

\path[draw=drawColor,line width= 0.4pt,line join=round,line cap=round,fill=fillColor] ( 86.41,241.82) circle (  1.16);

\path[draw=drawColor,line width= 0.4pt,line join=round,line cap=round,fill=fillColor] ( 91.63,239.73) circle (  1.16);

\path[draw=drawColor,line width= 0.4pt,line join=round,line cap=round,fill=fillColor] ( 95.78,239.58) circle (  1.16);

\path[draw=drawColor,line width= 0.4pt,line join=round,line cap=round,fill=fillColor] ( 99.29,238.94) circle (  1.16);

\path[draw=drawColor,line width= 0.4pt,line join=round,line cap=round,fill=fillColor] (102.36,238.01) circle (  1.16);

\path[draw=drawColor,line width= 0.4pt,line join=round,line cap=round,fill=fillColor] (105.10,236.85) circle (  1.16);

\path[draw=drawColor,line width= 0.4pt,line join=round,line cap=round,fill=fillColor] (107.59,236.78) circle (  1.16);

\path[draw=drawColor,line width= 0.4pt,line join=round,line cap=round,fill=fillColor] (109.88,236.77) circle (  1.16);

\path[draw=drawColor,line width= 0.4pt,line join=round,line cap=round,fill=fillColor] (112.01,235.87) circle (  1.16);

\path[draw=drawColor,line width= 0.4pt,line join=round,line cap=round,fill=fillColor] (114.00,235.61) circle (  1.16);

\path[draw=drawColor,line width= 0.4pt,line join=round,line cap=round,fill=fillColor] (115.87,235.20) circle (  1.16);

\path[draw=drawColor,line width= 0.4pt,line join=round,line cap=round,fill=fillColor] (117.65,234.32) circle (  1.16);

\path[draw=drawColor,line width= 0.4pt,line join=round,line cap=round,fill=fillColor] (119.33,233.99) circle (  1.16);

\path[draw=drawColor,line width= 0.4pt,line join=round,line cap=round,fill=fillColor] (120.93,233.45) circle (  1.16);

\path[draw=drawColor,line width= 0.4pt,line join=round,line cap=round,fill=fillColor] (122.47,233.24) circle (  1.16);

\path[draw=drawColor,line width= 0.4pt,line join=round,line cap=round,fill=fillColor] (123.94,233.19) circle (  1.16);

\path[draw=drawColor,line width= 0.4pt,line join=round,line cap=round,fill=fillColor] (125.36,233.09) circle (  1.16);

\path[draw=drawColor,line width= 0.4pt,line join=round,line cap=round,fill=fillColor] (126.72,232.78) circle (  1.16);

\path[draw=drawColor,line width= 0.4pt,line join=round,line cap=round,fill=fillColor] (128.04,232.52) circle (  1.16);

\path[draw=drawColor,line width= 0.4pt,line join=round,line cap=round,fill=fillColor] (129.31,232.29) circle (  1.16);

\path[draw=drawColor,line width= 0.4pt,line join=round,line cap=round,fill=fillColor] (130.54,232.08) circle (  1.16);

\path[draw=drawColor,line width= 0.4pt,line join=round,line cap=round,fill=fillColor] (131.74,232.05) circle (  1.16);

\path[draw=drawColor,line width= 0.4pt,line join=round,line cap=round,fill=fillColor] (132.90,231.98) circle (  1.16);

\path[draw=drawColor,line width= 0.4pt,line join=round,line cap=round,fill=fillColor] (134.04,231.79) circle (  1.16);

\path[draw=drawColor,line width= 0.4pt,line join=round,line cap=round,fill=fillColor] (135.14,231.52) circle (  1.16);

\path[draw=drawColor,line width= 0.4pt,line join=round,line cap=round,fill=fillColor] (136.21,231.27) circle (  1.16);

\path[draw=drawColor,line width= 0.4pt,line join=round,line cap=round,fill=fillColor] (137.26,230.96) circle (  1.16);

\path[draw=drawColor,line width= 0.4pt,line join=round,line cap=round,fill=fillColor] (138.28,230.82) circle (  1.16);

\path[draw=drawColor,line width= 0.4pt,line join=round,line cap=round,fill=fillColor] (139.28,230.69) circle (  1.16);

\path[draw=drawColor,line width= 0.4pt,line join=round,line cap=round,fill=fillColor] (140.26,230.40) circle (  1.16);

\path[draw=drawColor,line width= 0.4pt,line join=round,line cap=round,fill=fillColor] (141.21,230.15) circle (  1.16);

\path[draw=drawColor,line width= 0.4pt,line join=round,line cap=round,fill=fillColor] (142.15,229.97) circle (  1.16);

\path[draw=drawColor,line width= 0.4pt,line join=round,line cap=round,fill=fillColor] (143.07,229.78) circle (  1.16);

\path[draw=drawColor,line width= 0.4pt,line join=round,line cap=round,fill=fillColor] (143.97,229.66) circle (  1.16);

\path[draw=drawColor,line width= 0.4pt,line join=round,line cap=round,fill=fillColor] (144.85,229.63) circle (  1.16);

\path[draw=drawColor,line width= 0.4pt,line join=round,line cap=round,fill=fillColor] (145.72,229.48) circle (  1.16);

\path[draw=drawColor,line width= 0.4pt,line join=round,line cap=round,fill=fillColor] (146.57,229.43) circle (  1.16);

\path[draw=drawColor,line width= 0.4pt,line join=round,line cap=round,fill=fillColor] (147.41,229.20) circle (  1.16);

\path[draw=drawColor,line width= 0.4pt,line join=round,line cap=round,fill=fillColor] (148.23,229.13) circle (  1.16);

\path[draw=drawColor,line width= 0.4pt,line join=round,line cap=round,fill=fillColor] (149.04,228.97) circle (  1.16);

\path[draw=drawColor,line width= 0.4pt,line join=round,line cap=round,fill=fillColor] (149.83,228.90) circle (  1.16);

\path[draw=drawColor,line width= 0.4pt,line join=round,line cap=round,fill=fillColor] (150.62,228.89) circle (  1.16);

\path[draw=drawColor,line width= 0.4pt,line join=round,line cap=round,fill=fillColor] (151.39,228.87) circle (  1.16);

\path[draw=drawColor,line width= 0.4pt,line join=round,line cap=round,fill=fillColor] (152.15,228.82) circle (  1.16);

\path[draw=drawColor,line width= 0.4pt,line join=round,line cap=round,fill=fillColor] (152.90,228.73) circle (  1.16);

\path[draw=drawColor,line width= 0.4pt,line join=round,line cap=round,fill=fillColor] (153.63,228.62) circle (  1.16);

\path[draw=drawColor,line width= 0.4pt,line join=round,line cap=round,fill=fillColor] (154.36,228.56) circle (  1.16);

\path[draw=drawColor,line width= 0.4pt,line join=round,line cap=round,fill=fillColor] (155.08,228.54) circle (  1.16);

\path[draw=drawColor,line width= 0.4pt,line join=round,line cap=round,fill=fillColor] (155.79,228.45) circle (  1.16);

\path[draw=drawColor,line width= 0.4pt,line join=round,line cap=round,fill=fillColor] (156.48,228.42) circle (  1.16);

\path[draw=drawColor,line width= 0.4pt,line join=round,line cap=round,fill=fillColor] (157.17,228.18) circle (  1.16);

\path[draw=drawColor,line width= 0.4pt,line join=round,line cap=round,fill=fillColor] (157.85,228.11) circle (  1.16);

\path[draw=drawColor,line width= 0.4pt,line join=round,line cap=round,fill=fillColor] (158.53,228.04) circle (  1.16);

\path[draw=drawColor,line width= 0.4pt,line join=round,line cap=round,fill=fillColor] (159.19,227.89) circle (  1.16);

\path[draw=drawColor,line width= 0.4pt,line join=round,line cap=round,fill=fillColor] (159.85,227.57) circle (  1.16);

\path[draw=drawColor,line width= 0.4pt,line join=round,line cap=round,fill=fillColor] (160.49,227.44) circle (  1.16);

\path[draw=drawColor,line width= 0.4pt,line join=round,line cap=round,fill=fillColor] (161.13,227.28) circle (  1.16);

\path[draw=drawColor,line width= 0.4pt,line join=round,line cap=round,fill=fillColor] (161.77,227.16) circle (  1.16);

\path[draw=drawColor,line width= 0.4pt,line join=round,line cap=round,fill=fillColor] (162.39,227.16) circle (  1.16);

\path[draw=drawColor,line width= 0.4pt,line join=round,line cap=round,fill=fillColor] (163.01,226.87) circle (  1.16);

\path[draw=drawColor,line width= 0.4pt,line join=round,line cap=round,fill=fillColor] (163.62,226.74) circle (  1.16);

\path[draw=drawColor,line width= 0.4pt,line join=round,line cap=round,fill=fillColor] (164.23,226.59) circle (  1.16);

\path[draw=drawColor,line width= 0.4pt,line join=round,line cap=round,fill=fillColor] (164.83,226.44) circle (  1.16);

\path[draw=drawColor,line width= 0.4pt,line join=round,line cap=round,fill=fillColor] (165.42,226.43) circle (  1.16);

\path[draw=drawColor,line width= 0.4pt,line join=round,line cap=round,fill=fillColor] (166.01,226.37) circle (  1.16);

\path[draw=drawColor,line width= 0.4pt,line join=round,line cap=round,fill=fillColor] (166.59,226.00) circle (  1.16);

\path[draw=drawColor,line width= 0.4pt,line join=round,line cap=round,fill=fillColor] (167.17,225.88) circle (  1.16);

\path[draw=drawColor,line width= 0.4pt,line join=round,line cap=round,fill=fillColor] (167.74,225.78) circle (  1.16);

\path[draw=drawColor,line width= 0.4pt,line join=round,line cap=round,fill=fillColor] (168.30,225.77) circle (  1.16);

\path[draw=drawColor,line width= 0.4pt,line join=round,line cap=round,fill=fillColor] (168.86,225.74) circle (  1.16);

\path[draw=drawColor,line width= 0.4pt,line join=round,line cap=round,fill=fillColor] (169.41,225.68) circle (  1.16);

\path[draw=drawColor,line width= 0.4pt,line join=round,line cap=round,fill=fillColor] (169.96,225.47) circle (  1.16);

\path[draw=drawColor,line width= 0.4pt,line join=round,line cap=round,fill=fillColor] (170.51,225.36) circle (  1.16);

\path[draw=drawColor,line width= 0.4pt,line join=round,line cap=round,fill=fillColor] (171.04,224.91) circle (  1.16);

\path[draw=drawColor,line width= 0.4pt,line join=round,line cap=round,fill=fillColor] (171.58,224.89) circle (  1.16);

\path[draw=drawColor,line width= 0.4pt,line join=round,line cap=round,fill=fillColor] (172.11,224.87) circle (  1.16);

\path[draw=drawColor,line width= 0.4pt,line join=round,line cap=round,fill=fillColor] (172.63,224.83) circle (  1.16);

\path[draw=drawColor,line width= 0.4pt,line join=round,line cap=round,fill=fillColor] (173.15,224.61) circle (  1.16);

\path[draw=drawColor,line width= 0.4pt,line join=round,line cap=round,fill=fillColor] (173.67,224.40) circle (  1.16);

\path[draw=drawColor,line width= 0.4pt,line join=round,line cap=round,fill=fillColor] (174.18,224.39) circle (  1.16);

\path[draw=drawColor,line width= 0.4pt,line join=round,line cap=round,fill=fillColor] (174.69,224.37) circle (  1.16);

\path[draw=drawColor,line width= 0.4pt,line join=round,line cap=round,fill=fillColor] (175.19,224.24) circle (  1.16);

\path[draw=drawColor,line width= 0.4pt,line join=round,line cap=round,fill=fillColor] (175.69,224.22) circle (  1.16);

\path[draw=drawColor,line width= 0.4pt,line join=round,line cap=round,fill=fillColor] (176.19,223.81) circle (  1.16);

\path[draw=drawColor,line width= 0.4pt,line join=round,line cap=round,fill=fillColor] (176.68,223.79) circle (  1.16);

\path[draw=drawColor,line width= 0.4pt,line join=round,line cap=round,fill=fillColor] (177.17,223.74) circle (  1.16);

\path[draw=drawColor,line width= 0.4pt,line join=round,line cap=round,fill=fillColor] (177.65,223.73) circle (  1.16);

\path[draw=drawColor,line width= 0.4pt,line join=round,line cap=round,fill=fillColor] (178.13,223.61) circle (  1.16);

\path[draw=drawColor,line width= 0.4pt,line join=round,line cap=round,fill=fillColor] (178.61,223.40) circle (  1.16);

\path[draw=drawColor,line width= 0.4pt,line join=round,line cap=round,fill=fillColor] (179.08,223.36) circle (  1.16);

\path[draw=drawColor,line width= 0.4pt,line join=round,line cap=round,fill=fillColor] (179.55,223.35) circle (  1.16);

\path[draw=drawColor,line width= 0.4pt,line join=round,line cap=round,fill=fillColor] (180.02,223.18) circle (  1.16);

\path[draw=drawColor,line width= 0.4pt,line join=round,line cap=round,fill=fillColor] (180.48,223.15) circle (  1.16);

\path[draw=drawColor,line width= 0.4pt,line join=round,line cap=round,fill=fillColor] (180.94,223.06) circle (  1.16);

\path[draw=drawColor,line width= 0.4pt,line join=round,line cap=round,fill=fillColor] (181.40,223.05) circle (  1.16);

\path[draw=drawColor,line width= 0.4pt,line join=round,line cap=round,fill=fillColor] (181.85,222.98) circle (  1.16);

\path[draw=drawColor,line width= 0.4pt,line join=round,line cap=round,fill=fillColor] (182.30,222.97) circle (  1.16);

\path[draw=drawColor,line width= 0.4pt,line join=round,line cap=round,fill=fillColor] (182.75,222.93) circle (  1.16);

\path[draw=drawColor,line width= 0.4pt,line join=round,line cap=round,fill=fillColor] (183.19,222.72) circle (  1.16);

\path[draw=drawColor,line width= 0.4pt,line join=round,line cap=round,fill=fillColor] (183.63,222.46) circle (  1.16);

\path[draw=drawColor,line width= 0.4pt,line join=round,line cap=round,fill=fillColor] (184.07,222.40) circle (  1.16);

\path[draw=drawColor,line width= 0.4pt,line join=round,line cap=round,fill=fillColor] (184.51,222.03) circle (  1.16);

\path[draw=drawColor,line width= 0.4pt,line join=round,line cap=round,fill=fillColor] (184.94,221.97) circle (  1.16);

\path[draw=drawColor,line width= 0.4pt,line join=round,line cap=round,fill=fillColor] (185.37,221.90) circle (  1.16);

\path[draw=drawColor,line width= 0.4pt,line join=round,line cap=round,fill=fillColor] (185.80,221.87) circle (  1.16);

\path[draw=drawColor,line width= 0.4pt,line join=round,line cap=round,fill=fillColor] (186.22,221.76) circle (  1.16);

\path[draw=drawColor,line width= 0.4pt,line join=round,line cap=round,fill=fillColor] (186.64,221.54) circle (  1.16);

\path[draw=drawColor,line width= 0.4pt,line join=round,line cap=round,fill=fillColor] (187.06,221.24) circle (  1.16);

\path[draw=drawColor,line width= 0.4pt,line join=round,line cap=round,fill=fillColor] (187.48,221.14) circle (  1.16);

\path[draw=drawColor,line width= 0.4pt,line join=round,line cap=round,fill=fillColor] (187.89,221.13) circle (  1.16);

\path[draw=drawColor,line width= 0.4pt,line join=round,line cap=round,fill=fillColor] (188.31,220.83) circle (  1.16);

\path[draw=drawColor,line width= 0.4pt,line join=round,line cap=round,fill=fillColor] (188.72,220.73) circle (  1.16);

\path[draw=drawColor,line width= 0.4pt,line join=round,line cap=round,fill=fillColor] (189.12,220.49) circle (  1.16);

\path[draw=drawColor,line width= 0.4pt,line join=round,line cap=round,fill=fillColor] (189.53,220.48) circle (  1.16);

\path[draw=drawColor,line width= 0.4pt,line join=round,line cap=round,fill=fillColor] (189.93,220.44) circle (  1.16);

\path[draw=drawColor,line width= 0.4pt,line join=round,line cap=round,fill=fillColor] (190.33,220.31) circle (  1.16);

\path[draw=drawColor,line width= 0.4pt,line join=round,line cap=round,fill=fillColor] (190.73,220.20) circle (  1.16);

\path[draw=drawColor,line width= 0.4pt,line join=round,line cap=round,fill=fillColor] (191.12,220.09) circle (  1.16);

\path[draw=drawColor,line width= 0.4pt,line join=round,line cap=round,fill=fillColor] (191.52,219.96) circle (  1.16);

\path[draw=drawColor,line width= 0.4pt,line join=round,line cap=round,fill=fillColor] (191.91,219.95) circle (  1.16);

\path[draw=drawColor,line width= 0.4pt,line join=round,line cap=round,fill=fillColor] (192.30,219.72) circle (  1.16);

\path[draw=drawColor,line width= 0.4pt,line join=round,line cap=round,fill=fillColor] (192.68,219.68) circle (  1.16);

\path[draw=drawColor,line width= 0.4pt,line join=round,line cap=round,fill=fillColor] (193.07,219.50) circle (  1.16);

\path[draw=drawColor,line width= 0.4pt,line join=round,line cap=round,fill=fillColor] (193.45,219.43) circle (  1.16);

\path[draw=drawColor,line width= 0.4pt,line join=round,line cap=round,fill=fillColor] (193.83,219.22) circle (  1.16);

\path[draw=drawColor,line width= 0.4pt,line join=round,line cap=round,fill=fillColor] (194.21,219.18) circle (  1.16);

\path[draw=drawColor,line width= 0.4pt,line join=round,line cap=round,fill=fillColor] (194.59,218.99) circle (  1.16);

\path[draw=drawColor,line width= 0.4pt,line join=round,line cap=round,fill=fillColor] (194.96,218.90) circle (  1.16);

\path[draw=drawColor,line width= 0.4pt,line join=round,line cap=round,fill=fillColor] (195.33,218.85) circle (  1.16);

\path[draw=drawColor,line width= 0.4pt,line join=round,line cap=round,fill=fillColor] (195.70,218.74) circle (  1.16);

\path[draw=drawColor,line width= 0.4pt,line join=round,line cap=round,fill=fillColor] (196.07,218.36) circle (  1.16);

\path[draw=drawColor,line width= 0.4pt,line join=round,line cap=round,fill=fillColor] (196.44,218.29) circle (  1.16);

\path[draw=drawColor,line width= 0.4pt,line join=round,line cap=round,fill=fillColor] (196.80,218.22) circle (  1.16);

\path[draw=drawColor,line width= 0.4pt,line join=round,line cap=round,fill=fillColor] (197.17,218.15) circle (  1.16);

\path[draw=drawColor,line width= 0.4pt,line join=round,line cap=round,fill=fillColor] (197.53,218.10) circle (  1.16);

\path[draw=drawColor,line width= 0.4pt,line join=round,line cap=round,fill=fillColor] (197.89,217.99) circle (  1.16);

\path[draw=drawColor,line width= 0.4pt,line join=round,line cap=round,fill=fillColor] (198.25,217.90) circle (  1.16);

\path[draw=drawColor,line width= 0.4pt,line join=round,line cap=round,fill=fillColor] (198.60,217.85) circle (  1.16);

\path[draw=drawColor,line width= 0.4pt,line join=round,line cap=round,fill=fillColor] (198.96,217.80) circle (  1.16);

\path[draw=drawColor,line width= 0.4pt,line join=round,line cap=round,fill=fillColor] (199.31,217.72) circle (  1.16);

\path[draw=drawColor,line width= 0.4pt,line join=round,line cap=round,fill=fillColor] (199.66,217.63) circle (  1.16);

\path[draw=drawColor,line width= 0.4pt,line join=round,line cap=round,fill=fillColor] (200.01,217.40) circle (  1.16);

\path[draw=drawColor,line width= 0.4pt,line join=round,line cap=round,fill=fillColor] (200.36,217.30) circle (  1.16);

\path[draw=drawColor,line width= 0.4pt,line join=round,line cap=round,fill=fillColor] (200.71,217.22) circle (  1.16);

\path[draw=drawColor,line width= 0.4pt,line join=round,line cap=round,fill=fillColor] (201.05,217.13) circle (  1.16);

\path[draw=drawColor,line width= 0.4pt,line join=round,line cap=round,fill=fillColor] (201.40,216.95) circle (  1.16);

\path[draw=drawColor,line width= 0.4pt,line join=round,line cap=round,fill=fillColor] (201.74,216.87) circle (  1.16);

\path[draw=drawColor,line width= 0.4pt,line join=round,line cap=round,fill=fillColor] (202.08,216.70) circle (  1.16);

\path[draw=drawColor,line width= 0.4pt,line join=round,line cap=round,fill=fillColor] (202.42,216.67) circle (  1.16);

\path[draw=drawColor,line width= 0.4pt,line join=round,line cap=round,fill=fillColor] (202.75,216.51) circle (  1.16);

\path[draw=drawColor,line width= 0.4pt,line join=round,line cap=round,fill=fillColor] (203.09,216.49) circle (  1.16);

\path[draw=drawColor,line width= 0.4pt,line join=round,line cap=round,fill=fillColor] (203.42,216.29) circle (  1.16);

\path[draw=drawColor,line width= 0.4pt,line join=round,line cap=round,fill=fillColor] (203.76,215.89) circle (  1.16);

\path[draw=drawColor,line width= 0.4pt,line join=round,line cap=round,fill=fillColor] (204.09,215.87) circle (  1.16);

\path[draw=drawColor,line width= 0.4pt,line join=round,line cap=round,fill=fillColor] (204.42,215.85) circle (  1.16);

\path[draw=drawColor,line width= 0.4pt,line join=round,line cap=round,fill=fillColor] (204.75,215.84) circle (  1.16);

\path[draw=drawColor,line width= 0.4pt,line join=round,line cap=round,fill=fillColor] (205.07,215.83) circle (  1.16);

\path[draw=drawColor,line width= 0.4pt,line join=round,line cap=round,fill=fillColor] (205.40,215.75) circle (  1.16);

\path[draw=drawColor,line width= 0.4pt,line join=round,line cap=round,fill=fillColor] (205.72,215.65) circle (  1.16);

\path[draw=drawColor,line width= 0.4pt,line join=round,line cap=round,fill=fillColor] (206.05,215.64) circle (  1.16);

\path[draw=drawColor,line width= 0.4pt,line join=round,line cap=round,fill=fillColor] (206.37,215.58) circle (  1.16);

\path[draw=drawColor,line width= 0.4pt,line join=round,line cap=round,fill=fillColor] (206.69,215.36) circle (  1.16);

\path[draw=drawColor,line width= 0.4pt,line join=round,line cap=round,fill=fillColor] (207.01,215.31) circle (  1.16);

\path[draw=drawColor,line width= 0.4pt,line join=round,line cap=round,fill=fillColor] (207.32,215.21) circle (  1.16);

\path[draw=drawColor,line width= 0.4pt,line join=round,line cap=round,fill=fillColor] (207.64,215.21) circle (  1.16);

\path[draw=drawColor,line width= 0.4pt,line join=round,line cap=round,fill=fillColor] (207.96,214.81) circle (  1.16);

\path[draw=drawColor,line width= 0.4pt,line join=round,line cap=round,fill=fillColor] (208.27,214.65) circle (  1.16);

\path[draw=drawColor,line width= 0.4pt,line join=round,line cap=round,fill=fillColor] (208.58,214.56) circle (  1.16);

\path[draw=drawColor,line width= 0.4pt,line join=round,line cap=round,fill=fillColor] (208.89,214.25) circle (  1.16);

\path[draw=drawColor,line width= 0.4pt,line join=round,line cap=round,fill=fillColor] (209.20,214.12) circle (  1.16);

\path[draw=drawColor,line width= 0.4pt,line join=round,line cap=round,fill=fillColor] (209.51,214.07) circle (  1.16);

\path[draw=drawColor,line width= 0.4pt,line join=round,line cap=round,fill=fillColor] (209.82,214.01) circle (  1.16);

\path[draw=drawColor,line width= 0.4pt,line join=round,line cap=round,fill=fillColor] (210.13,213.92) circle (  1.16);

\path[draw=drawColor,line width= 0.4pt,line join=round,line cap=round,fill=fillColor] (210.43,213.69) circle (  1.16);

\path[draw=drawColor,line width= 0.4pt,line join=round,line cap=round,fill=fillColor] (210.74,213.57) circle (  1.16);

\path[draw=drawColor,line width= 0.4pt,line join=round,line cap=round,fill=fillColor] (211.04,213.40) circle (  1.16);

\path[draw=drawColor,line width= 0.4pt,line join=round,line cap=round,fill=fillColor] (211.34,213.31) circle (  1.16);

\path[draw=drawColor,line width= 0.4pt,line join=round,line cap=round,fill=fillColor] (211.64,213.15) circle (  1.16);

\path[draw=drawColor,line width= 0.4pt,line join=round,line cap=round,fill=fillColor] (211.94,212.52) circle (  1.16);

\path[draw=drawColor,line width= 0.4pt,line join=round,line cap=round,fill=fillColor] (212.24,212.36) circle (  1.16);

\path[draw=drawColor,line width= 0.4pt,line join=round,line cap=round,fill=fillColor] (212.54,212.33) circle (  1.16);

\path[draw=drawColor,line width= 0.4pt,line join=round,line cap=round,fill=fillColor] (212.84,212.32) circle (  1.16);

\path[draw=drawColor,line width= 0.4pt,line join=round,line cap=round,fill=fillColor] (213.13,211.70) circle (  1.16);

\path[draw=drawColor,line width= 0.4pt,line join=round,line cap=round,fill=fillColor] (213.43,210.97) circle (  1.16);

\path[draw=drawColor,line width= 0.4pt,line join=round,line cap=round,fill=fillColor] (213.72,210.70) circle (  1.16);

\path[draw=drawColor,line width= 0.4pt,line join=round,line cap=round,fill=fillColor] (214.01,209.54) circle (  1.16);

\path[draw=drawColor,line width= 0.4pt,line join=round,line cap=round,fill=fillColor] (214.30,209.10) circle (  1.16);

\path[draw=drawColor,line width= 0.4pt,line join=round,line cap=round,fill=fillColor] (214.59,209.05) circle (  1.16);

\path[draw=drawColor,line width= 0.4pt,line join=round,line cap=round,fill=fillColor] (214.88,208.82) circle (  1.16);

\path[draw=drawColor,line width= 0.4pt,line join=round,line cap=round,fill=fillColor] (215.17,208.14) circle (  1.16);

\path[draw=drawColor,line width= 0.4pt,line join=round,line cap=round,fill=fillColor] (215.46,207.87) circle (  1.16);

\path[draw=drawColor,line width= 0.4pt,line join=round,line cap=round,fill=fillColor] (215.74,198.35) circle (  1.16);

\path[draw=drawColor,line width= 0.4pt,line join=round,line cap=round,fill=fillColor] (216.03,198.35) circle (  1.16);

\path[draw=drawColor,line width= 0.4pt,line join=round,line cap=round,fill=fillColor] (216.31,198.35) circle (  1.16);

\path[draw=drawColor,line width= 0.4pt,line join=round,line cap=round,fill=fillColor] (216.60,198.35) circle (  1.16);

\path[draw=drawColor,line width= 0.4pt,line join=round,line cap=round,fill=fillColor] (216.88,198.35) circle (  1.16);

\path[draw=drawColor,line width= 0.4pt,line join=round,line cap=round,fill=fillColor] (217.16,198.35) circle (  1.16);

\path[draw=drawColor,line width= 0.4pt,line join=round,line cap=round,fill=fillColor] (217.44,198.35) circle (  1.16);

\path[draw=drawColor,line width= 0.4pt,line join=round,line cap=round,fill=fillColor] (217.72,198.35) circle (  1.16);

\path[draw=drawColor,line width= 0.4pt,line join=round,line cap=round,fill=fillColor] (218.00,198.35) circle (  1.16);

\path[draw=drawColor,line width= 0.4pt,line join=round,line cap=round,fill=fillColor] (218.28,198.35) circle (  1.16);

\path[draw=drawColor,line width= 0.4pt,line join=round,line cap=round,fill=fillColor] (218.55,198.35) circle (  1.16);

\path[draw=drawColor,line width= 0.4pt,line join=round,line cap=round,fill=fillColor] (218.83,198.35) circle (  1.16);

\path[draw=drawColor,line width= 0.4pt,line join=round,line cap=round,fill=fillColor] (219.10,198.35) circle (  1.16);
\definecolor[named]{drawColor}{rgb}{0.65,0.34,0.16}
\definecolor[named]{fillColor}{rgb}{0.65,0.34,0.16}

\path[draw=drawColor,line width= 0.4pt,line join=round,line cap=round,fill=fillColor] ( 78.97,224.23) circle (  1.16);

\path[draw=drawColor,line width= 0.4pt,line join=round,line cap=round,fill=fillColor] ( 86.41,223.38) circle (  1.16);

\path[draw=drawColor,line width= 0.4pt,line join=round,line cap=round,fill=fillColor] ( 91.63,223.20) circle (  1.16);

\path[draw=drawColor,line width= 0.4pt,line join=round,line cap=round,fill=fillColor] ( 95.78,222.16) circle (  1.16);

\path[draw=drawColor,line width= 0.4pt,line join=round,line cap=round,fill=fillColor] ( 99.29,221.17) circle (  1.16);

\path[draw=drawColor,line width= 0.4pt,line join=round,line cap=round,fill=fillColor] (102.36,220.84) circle (  1.16);

\path[draw=drawColor,line width= 0.4pt,line join=round,line cap=round,fill=fillColor] (105.10,220.00) circle (  1.16);

\path[draw=drawColor,line width= 0.4pt,line join=round,line cap=round,fill=fillColor] (107.59,218.14) circle (  1.16);

\path[draw=drawColor,line width= 0.4pt,line join=round,line cap=round,fill=fillColor] (109.88,217.80) circle (  1.16);

\path[draw=drawColor,line width= 0.4pt,line join=round,line cap=round,fill=fillColor] (112.01,217.45) circle (  1.16);

\path[draw=drawColor,line width= 0.4pt,line join=round,line cap=round,fill=fillColor] (114.00,217.20) circle (  1.16);

\path[draw=drawColor,line width= 0.4pt,line join=round,line cap=round,fill=fillColor] (115.87,216.35) circle (  1.16);

\path[draw=drawColor,line width= 0.4pt,line join=round,line cap=round,fill=fillColor] (117.65,216.25) circle (  1.16);

\path[draw=drawColor,line width= 0.4pt,line join=round,line cap=round,fill=fillColor] (119.33,215.99) circle (  1.16);

\path[draw=drawColor,line width= 0.4pt,line join=round,line cap=round,fill=fillColor] (120.93,215.66) circle (  1.16);

\path[draw=drawColor,line width= 0.4pt,line join=round,line cap=round,fill=fillColor] (122.47,215.59) circle (  1.16);

\path[draw=drawColor,line width= 0.4pt,line join=round,line cap=round,fill=fillColor] (123.94,215.33) circle (  1.16);

\path[draw=drawColor,line width= 0.4pt,line join=round,line cap=round,fill=fillColor] (125.36,215.26) circle (  1.16);

\path[draw=drawColor,line width= 0.4pt,line join=round,line cap=round,fill=fillColor] (126.72,215.07) circle (  1.16);

\path[draw=drawColor,line width= 0.4pt,line join=round,line cap=round,fill=fillColor] (128.04,214.93) circle (  1.16);

\path[draw=drawColor,line width= 0.4pt,line join=round,line cap=round,fill=fillColor] (129.31,214.73) circle (  1.16);

\path[draw=drawColor,line width= 0.4pt,line join=round,line cap=round,fill=fillColor] (130.54,214.62) circle (  1.16);

\path[draw=drawColor,line width= 0.4pt,line join=round,line cap=round,fill=fillColor] (131.74,214.44) circle (  1.16);

\path[draw=drawColor,line width= 0.4pt,line join=round,line cap=round,fill=fillColor] (132.90,214.34) circle (  1.16);

\path[draw=drawColor,line width= 0.4pt,line join=round,line cap=round,fill=fillColor] (134.04,213.87) circle (  1.16);

\path[draw=drawColor,line width= 0.4pt,line join=round,line cap=round,fill=fillColor] (135.14,213.84) circle (  1.16);

\path[draw=drawColor,line width= 0.4pt,line join=round,line cap=round,fill=fillColor] (136.21,213.58) circle (  1.16);

\path[draw=drawColor,line width= 0.4pt,line join=round,line cap=round,fill=fillColor] (137.26,213.14) circle (  1.16);

\path[draw=drawColor,line width= 0.4pt,line join=round,line cap=round,fill=fillColor] (138.28,213.08) circle (  1.16);

\path[draw=drawColor,line width= 0.4pt,line join=round,line cap=round,fill=fillColor] (139.28,212.88) circle (  1.16);

\path[draw=drawColor,line width= 0.4pt,line join=round,line cap=round,fill=fillColor] (140.26,212.80) circle (  1.16);

\path[draw=drawColor,line width= 0.4pt,line join=round,line cap=round,fill=fillColor] (141.21,212.70) circle (  1.16);

\path[draw=drawColor,line width= 0.4pt,line join=round,line cap=round,fill=fillColor] (142.15,212.64) circle (  1.16);

\path[draw=drawColor,line width= 0.4pt,line join=round,line cap=round,fill=fillColor] (143.07,212.36) circle (  1.16);

\path[draw=drawColor,line width= 0.4pt,line join=round,line cap=round,fill=fillColor] (143.97,212.31) circle (  1.16);

\path[draw=drawColor,line width= 0.4pt,line join=round,line cap=round,fill=fillColor] (144.85,212.29) circle (  1.16);

\path[draw=drawColor,line width= 0.4pt,line join=round,line cap=round,fill=fillColor] (145.72,212.24) circle (  1.16);

\path[draw=drawColor,line width= 0.4pt,line join=round,line cap=round,fill=fillColor] (146.57,212.07) circle (  1.16);

\path[draw=drawColor,line width= 0.4pt,line join=round,line cap=round,fill=fillColor] (147.41,212.04) circle (  1.16);

\path[draw=drawColor,line width= 0.4pt,line join=round,line cap=round,fill=fillColor] (148.23,211.98) circle (  1.16);

\path[draw=drawColor,line width= 0.4pt,line join=round,line cap=round,fill=fillColor] (149.04,211.98) circle (  1.16);

\path[draw=drawColor,line width= 0.4pt,line join=round,line cap=round,fill=fillColor] (149.83,211.76) circle (  1.16);

\path[draw=drawColor,line width= 0.4pt,line join=round,line cap=round,fill=fillColor] (150.62,211.76) circle (  1.16);

\path[draw=drawColor,line width= 0.4pt,line join=round,line cap=round,fill=fillColor] (151.39,211.68) circle (  1.16);

\path[draw=drawColor,line width= 0.4pt,line join=round,line cap=round,fill=fillColor] (152.15,211.65) circle (  1.16);

\path[draw=drawColor,line width= 0.4pt,line join=round,line cap=round,fill=fillColor] (152.90,211.47) circle (  1.16);

\path[draw=drawColor,line width= 0.4pt,line join=round,line cap=round,fill=fillColor] (153.63,211.46) circle (  1.16);

\path[draw=drawColor,line width= 0.4pt,line join=round,line cap=round,fill=fillColor] (154.36,211.32) circle (  1.16);

\path[draw=drawColor,line width= 0.4pt,line join=round,line cap=round,fill=fillColor] (155.08,211.31) circle (  1.16);

\path[draw=drawColor,line width= 0.4pt,line join=round,line cap=round,fill=fillColor] (155.79,210.81) circle (  1.16);

\path[draw=drawColor,line width= 0.4pt,line join=round,line cap=round,fill=fillColor] (156.48,210.79) circle (  1.16);

\path[draw=drawColor,line width= 0.4pt,line join=round,line cap=round,fill=fillColor] (157.17,210.34) circle (  1.16);

\path[draw=drawColor,line width= 0.4pt,line join=round,line cap=round,fill=fillColor] (157.85,210.11) circle (  1.16);

\path[draw=drawColor,line width= 0.4pt,line join=round,line cap=round,fill=fillColor] (158.53,210.11) circle (  1.16);

\path[draw=drawColor,line width= 0.4pt,line join=round,line cap=round,fill=fillColor] (159.19,210.08) circle (  1.16);

\path[draw=drawColor,line width= 0.4pt,line join=round,line cap=round,fill=fillColor] (159.85,209.94) circle (  1.16);

\path[draw=drawColor,line width= 0.4pt,line join=round,line cap=round,fill=fillColor] (160.49,209.64) circle (  1.16);

\path[draw=drawColor,line width= 0.4pt,line join=round,line cap=round,fill=fillColor] (161.13,209.54) circle (  1.16);

\path[draw=drawColor,line width= 0.4pt,line join=round,line cap=round,fill=fillColor] (161.77,209.51) circle (  1.16);

\path[draw=drawColor,line width= 0.4pt,line join=round,line cap=round,fill=fillColor] (162.39,209.45) circle (  1.16);

\path[draw=drawColor,line width= 0.4pt,line join=round,line cap=round,fill=fillColor] (163.01,209.43) circle (  1.16);

\path[draw=drawColor,line width= 0.4pt,line join=round,line cap=round,fill=fillColor] (163.62,209.27) circle (  1.16);

\path[draw=drawColor,line width= 0.4pt,line join=round,line cap=round,fill=fillColor] (164.23,209.27) circle (  1.16);

\path[draw=drawColor,line width= 0.4pt,line join=round,line cap=round,fill=fillColor] (164.83,209.26) circle (  1.16);

\path[draw=drawColor,line width= 0.4pt,line join=round,line cap=round,fill=fillColor] (165.42,209.19) circle (  1.16);

\path[draw=drawColor,line width= 0.4pt,line join=round,line cap=round,fill=fillColor] (166.01,209.05) circle (  1.16);

\path[draw=drawColor,line width= 0.4pt,line join=round,line cap=round,fill=fillColor] (166.59,208.96) circle (  1.16);

\path[draw=drawColor,line width= 0.4pt,line join=round,line cap=round,fill=fillColor] (167.17,208.96) circle (  1.16);

\path[draw=drawColor,line width= 0.4pt,line join=round,line cap=round,fill=fillColor] (167.74,208.92) circle (  1.16);

\path[draw=drawColor,line width= 0.4pt,line join=round,line cap=round,fill=fillColor] (168.30,208.79) circle (  1.16);

\path[draw=drawColor,line width= 0.4pt,line join=round,line cap=round,fill=fillColor] (168.86,208.66) circle (  1.16);

\path[draw=drawColor,line width= 0.4pt,line join=round,line cap=round,fill=fillColor] (169.41,208.62) circle (  1.16);

\path[draw=drawColor,line width= 0.4pt,line join=round,line cap=round,fill=fillColor] (169.96,208.51) circle (  1.16);

\path[draw=drawColor,line width= 0.4pt,line join=round,line cap=round,fill=fillColor] (170.51,208.39) circle (  1.16);

\path[draw=drawColor,line width= 0.4pt,line join=round,line cap=round,fill=fillColor] (171.04,208.35) circle (  1.16);

\path[draw=drawColor,line width= 0.4pt,line join=round,line cap=round,fill=fillColor] (171.58,208.25) circle (  1.16);

\path[draw=drawColor,line width= 0.4pt,line join=round,line cap=round,fill=fillColor] (172.11,208.22) circle (  1.16);

\path[draw=drawColor,line width= 0.4pt,line join=round,line cap=round,fill=fillColor] (172.63,208.19) circle (  1.16);

\path[draw=drawColor,line width= 0.4pt,line join=round,line cap=round,fill=fillColor] (173.15,208.14) circle (  1.16);

\path[draw=drawColor,line width= 0.4pt,line join=round,line cap=round,fill=fillColor] (173.67,208.00) circle (  1.16);

\path[draw=drawColor,line width= 0.4pt,line join=round,line cap=round,fill=fillColor] (174.18,207.45) circle (  1.16);

\path[draw=drawColor,line width= 0.4pt,line join=round,line cap=round,fill=fillColor] (174.69,207.43) circle (  1.16);

\path[draw=drawColor,line width= 0.4pt,line join=round,line cap=round,fill=fillColor] (175.19,207.28) circle (  1.16);

\path[draw=drawColor,line width= 0.4pt,line join=round,line cap=round,fill=fillColor] (175.69,207.28) circle (  1.16);

\path[draw=drawColor,line width= 0.4pt,line join=round,line cap=round,fill=fillColor] (176.19,207.27) circle (  1.16);

\path[draw=drawColor,line width= 0.4pt,line join=round,line cap=round,fill=fillColor] (176.68,207.15) circle (  1.16);

\path[draw=drawColor,line width= 0.4pt,line join=round,line cap=round,fill=fillColor] (177.17,207.13) circle (  1.16);

\path[draw=drawColor,line width= 0.4pt,line join=round,line cap=round,fill=fillColor] (177.65,207.06) circle (  1.16);

\path[draw=drawColor,line width= 0.4pt,line join=round,line cap=round,fill=fillColor] (178.13,206.98) circle (  1.16);

\path[draw=drawColor,line width= 0.4pt,line join=round,line cap=round,fill=fillColor] (178.61,206.93) circle (  1.16);

\path[draw=drawColor,line width= 0.4pt,line join=round,line cap=round,fill=fillColor] (179.08,206.85) circle (  1.16);

\path[draw=drawColor,line width= 0.4pt,line join=round,line cap=round,fill=fillColor] (179.55,206.82) circle (  1.16);

\path[draw=drawColor,line width= 0.4pt,line join=round,line cap=round,fill=fillColor] (180.02,206.81) circle (  1.16);

\path[draw=drawColor,line width= 0.4pt,line join=round,line cap=round,fill=fillColor] (180.48,206.72) circle (  1.16);

\path[draw=drawColor,line width= 0.4pt,line join=round,line cap=round,fill=fillColor] (180.94,206.66) circle (  1.16);

\path[draw=drawColor,line width= 0.4pt,line join=round,line cap=round,fill=fillColor] (181.40,206.49) circle (  1.16);

\path[draw=drawColor,line width= 0.4pt,line join=round,line cap=round,fill=fillColor] (181.85,206.48) circle (  1.16);

\path[draw=drawColor,line width= 0.4pt,line join=round,line cap=round,fill=fillColor] (182.30,206.42) circle (  1.16);

\path[draw=drawColor,line width= 0.4pt,line join=round,line cap=round,fill=fillColor] (182.75,206.29) circle (  1.16);

\path[draw=drawColor,line width= 0.4pt,line join=round,line cap=round,fill=fillColor] (183.19,205.96) circle (  1.16);

\path[draw=drawColor,line width= 0.4pt,line join=round,line cap=round,fill=fillColor] (183.63,205.95) circle (  1.16);

\path[draw=drawColor,line width= 0.4pt,line join=round,line cap=round,fill=fillColor] (184.07,205.86) circle (  1.16);

\path[draw=drawColor,line width= 0.4pt,line join=round,line cap=round,fill=fillColor] (184.51,205.77) circle (  1.16);

\path[draw=drawColor,line width= 0.4pt,line join=round,line cap=round,fill=fillColor] (184.94,205.68) circle (  1.16);

\path[draw=drawColor,line width= 0.4pt,line join=round,line cap=round,fill=fillColor] (185.37,205.56) circle (  1.16);

\path[draw=drawColor,line width= 0.4pt,line join=round,line cap=round,fill=fillColor] (185.80,205.51) circle (  1.16);

\path[draw=drawColor,line width= 0.4pt,line join=round,line cap=round,fill=fillColor] (186.22,205.18) circle (  1.16);

\path[draw=drawColor,line width= 0.4pt,line join=round,line cap=round,fill=fillColor] (186.64,205.05) circle (  1.16);

\path[draw=drawColor,line width= 0.4pt,line join=round,line cap=round,fill=fillColor] (187.06,204.99) circle (  1.16);

\path[draw=drawColor,line width= 0.4pt,line join=round,line cap=round,fill=fillColor] (187.48,204.93) circle (  1.16);

\path[draw=drawColor,line width= 0.4pt,line join=round,line cap=round,fill=fillColor] (187.89,204.76) circle (  1.16);

\path[draw=drawColor,line width= 0.4pt,line join=round,line cap=round,fill=fillColor] (188.31,204.61) circle (  1.16);

\path[draw=drawColor,line width= 0.4pt,line join=round,line cap=round,fill=fillColor] (188.72,204.43) circle (  1.16);

\path[draw=drawColor,line width= 0.4pt,line join=round,line cap=round,fill=fillColor] (189.12,204.09) circle (  1.16);

\path[draw=drawColor,line width= 0.4pt,line join=round,line cap=round,fill=fillColor] (189.53,204.09) circle (  1.16);

\path[draw=drawColor,line width= 0.4pt,line join=round,line cap=round,fill=fillColor] (189.93,203.73) circle (  1.16);

\path[draw=drawColor,line width= 0.4pt,line join=round,line cap=round,fill=fillColor] (190.33,203.40) circle (  1.16);

\path[draw=drawColor,line width= 0.4pt,line join=round,line cap=round,fill=fillColor] (190.73,202.57) circle (  1.16);

\path[draw=drawColor,line width= 0.4pt,line join=round,line cap=round,fill=fillColor] (191.12,198.35) circle (  1.16);

\path[draw=drawColor,line width= 0.4pt,line join=round,line cap=round,fill=fillColor] (191.52,198.35) circle (  1.16);

\path[draw=drawColor,line width= 0.4pt,line join=round,line cap=round,fill=fillColor] (191.91,198.35) circle (  1.16);

\path[draw=drawColor,line width= 0.4pt,line join=round,line cap=round,fill=fillColor] (192.30,198.35) circle (  1.16);

\path[draw=drawColor,line width= 0.4pt,line join=round,line cap=round,fill=fillColor] (192.68,198.35) circle (  1.16);

\path[draw=drawColor,line width= 0.4pt,line join=round,line cap=round,fill=fillColor] (193.07,198.35) circle (  1.16);

\path[draw=drawColor,line width= 0.4pt,line join=round,line cap=round,fill=fillColor] (193.45,198.35) circle (  1.16);

\path[draw=drawColor,line width= 0.4pt,line join=round,line cap=round,fill=fillColor] (193.83,198.35) circle (  1.16);

\path[draw=drawColor,line width= 0.4pt,line join=round,line cap=round,fill=fillColor] (194.21,198.35) circle (  1.16);

\path[draw=drawColor,line width= 0.4pt,line join=round,line cap=round,fill=fillColor] (194.59,198.35) circle (  1.16);

\path[draw=drawColor,line width= 0.4pt,line join=round,line cap=round,fill=fillColor] (194.96,198.35) circle (  1.16);

\path[draw=drawColor,line width= 0.4pt,line join=round,line cap=round,fill=fillColor] (195.33,198.35) circle (  1.16);

\path[draw=drawColor,line width= 0.4pt,line join=round,line cap=round,fill=fillColor] (195.70,198.35) circle (  1.16);

\path[draw=drawColor,line width= 0.4pt,line join=round,line cap=round,fill=fillColor] (196.07,198.35) circle (  1.16);

\path[draw=drawColor,line width= 0.4pt,line join=round,line cap=round,fill=fillColor] (196.44,198.35) circle (  1.16);

\path[draw=drawColor,line width= 0.4pt,line join=round,line cap=round,fill=fillColor] (196.80,198.35) circle (  1.16);

\path[draw=drawColor,line width= 0.4pt,line join=round,line cap=round,fill=fillColor] (197.17,198.35) circle (  1.16);

\path[draw=drawColor,line width= 0.4pt,line join=round,line cap=round,fill=fillColor] (197.53,198.35) circle (  1.16);

\path[draw=drawColor,line width= 0.4pt,line join=round,line cap=round,fill=fillColor] (197.89,198.35) circle (  1.16);

\path[draw=drawColor,line width= 0.4pt,line join=round,line cap=round,fill=fillColor] (198.25,198.35) circle (  1.16);

\path[draw=drawColor,line width= 0.4pt,line join=round,line cap=round,fill=fillColor] (198.60,198.35) circle (  1.16);

\path[draw=drawColor,line width= 0.4pt,line join=round,line cap=round,fill=fillColor] (198.96,198.35) circle (  1.16);

\path[draw=drawColor,line width= 0.4pt,line join=round,line cap=round,fill=fillColor] (199.31,198.35) circle (  1.16);

\path[draw=drawColor,line width= 0.4pt,line join=round,line cap=round,fill=fillColor] (199.66,198.35) circle (  1.16);

\path[draw=drawColor,line width= 0.4pt,line join=round,line cap=round,fill=fillColor] (200.01,198.35) circle (  1.16);

\path[draw=drawColor,line width= 0.4pt,line join=round,line cap=round,fill=fillColor] (200.36,198.35) circle (  1.16);

\path[draw=drawColor,line width= 0.4pt,line join=round,line cap=round,fill=fillColor] (200.71,198.35) circle (  1.16);

\path[draw=drawColor,line width= 0.4pt,line join=round,line cap=round,fill=fillColor] (201.05,198.35) circle (  1.16);

\path[draw=drawColor,line width= 0.4pt,line join=round,line cap=round,fill=fillColor] (201.40,198.35) circle (  1.16);

\path[draw=drawColor,line width= 0.4pt,line join=round,line cap=round,fill=fillColor] (201.74,198.35) circle (  1.16);

\path[draw=drawColor,line width= 0.4pt,line join=round,line cap=round,fill=fillColor] (202.08,198.35) circle (  1.16);

\path[draw=drawColor,line width= 0.4pt,line join=round,line cap=round,fill=fillColor] (202.42,198.35) circle (  1.16);

\path[draw=drawColor,line width= 0.4pt,line join=round,line cap=round,fill=fillColor] (202.75,198.35) circle (  1.16);

\path[draw=drawColor,line width= 0.4pt,line join=round,line cap=round,fill=fillColor] (203.09,198.35) circle (  1.16);

\path[draw=drawColor,line width= 0.4pt,line join=round,line cap=round,fill=fillColor] (203.42,198.35) circle (  1.16);

\path[draw=drawColor,line width= 0.4pt,line join=round,line cap=round,fill=fillColor] (203.76,198.35) circle (  1.16);

\path[draw=drawColor,line width= 0.4pt,line join=round,line cap=round,fill=fillColor] (204.09,198.35) circle (  1.16);

\path[draw=drawColor,line width= 0.4pt,line join=round,line cap=round,fill=fillColor] (204.42,198.35) circle (  1.16);

\path[draw=drawColor,line width= 0.4pt,line join=round,line cap=round,fill=fillColor] (204.75,198.35) circle (  1.16);

\path[draw=drawColor,line width= 0.4pt,line join=round,line cap=round,fill=fillColor] (205.07,198.35) circle (  1.16);

\path[draw=drawColor,line width= 0.4pt,line join=round,line cap=round,fill=fillColor] (205.40,198.35) circle (  1.16);

\path[draw=drawColor,line width= 0.4pt,line join=round,line cap=round,fill=fillColor] (205.72,198.35) circle (  1.16);

\path[draw=drawColor,line width= 0.4pt,line join=round,line cap=round,fill=fillColor] (206.05,198.35) circle (  1.16);

\path[draw=drawColor,line width= 0.4pt,line join=round,line cap=round,fill=fillColor] (206.37,198.35) circle (  1.16);

\path[draw=drawColor,line width= 0.4pt,line join=round,line cap=round,fill=fillColor] (206.69,198.35) circle (  1.16);

\path[draw=drawColor,line width= 0.4pt,line join=round,line cap=round,fill=fillColor] (207.01,198.35) circle (  1.16);

\path[draw=drawColor,line width= 0.4pt,line join=round,line cap=round,fill=fillColor] (207.32,198.35) circle (  1.16);

\path[draw=drawColor,line width= 0.4pt,line join=round,line cap=round,fill=fillColor] (207.64,198.35) circle (  1.16);

\path[draw=drawColor,line width= 0.4pt,line join=round,line cap=round,fill=fillColor] (207.96,198.35) circle (  1.16);

\path[draw=drawColor,line width= 0.4pt,line join=round,line cap=round,fill=fillColor] (208.27,198.35) circle (  1.16);

\path[draw=drawColor,line width= 0.4pt,line join=round,line cap=round,fill=fillColor] (208.58,198.35) circle (  1.16);

\path[draw=drawColor,line width= 0.4pt,line join=round,line cap=round,fill=fillColor] (208.89,198.35) circle (  1.16);

\path[draw=drawColor,line width= 0.4pt,line join=round,line cap=round,fill=fillColor] (209.20,198.35) circle (  1.16);

\path[draw=drawColor,line width= 0.4pt,line join=round,line cap=round,fill=fillColor] (209.51,198.35) circle (  1.16);

\path[draw=drawColor,line width= 0.4pt,line join=round,line cap=round,fill=fillColor] (209.82,198.35) circle (  1.16);

\path[draw=drawColor,line width= 0.4pt,line join=round,line cap=round,fill=fillColor] (210.13,198.35) circle (  1.16);

\path[draw=drawColor,line width= 0.4pt,line join=round,line cap=round,fill=fillColor] (210.43,198.35) circle (  1.16);

\path[draw=drawColor,line width= 0.4pt,line join=round,line cap=round,fill=fillColor] (210.74,198.35) circle (  1.16);

\path[draw=drawColor,line width= 0.4pt,line join=round,line cap=round,fill=fillColor] (211.04,198.35) circle (  1.16);

\path[draw=drawColor,line width= 0.4pt,line join=round,line cap=round,fill=fillColor] (211.34,198.35) circle (  1.16);

\path[draw=drawColor,line width= 0.4pt,line join=round,line cap=round,fill=fillColor] (211.64,198.35) circle (  1.16);

\path[draw=drawColor,line width= 0.4pt,line join=round,line cap=round,fill=fillColor] (211.94,198.35) circle (  1.16);

\path[draw=drawColor,line width= 0.4pt,line join=round,line cap=round,fill=fillColor] (212.24,198.35) circle (  1.16);

\path[draw=drawColor,line width= 0.4pt,line join=round,line cap=round,fill=fillColor] (212.54,198.35) circle (  1.16);

\path[draw=drawColor,line width= 0.4pt,line join=round,line cap=round,fill=fillColor] (212.84,198.35) circle (  1.16);

\path[draw=drawColor,line width= 0.4pt,line join=round,line cap=round,fill=fillColor] (213.13,198.35) circle (  1.16);

\path[draw=drawColor,line width= 0.4pt,line join=round,line cap=round,fill=fillColor] (213.43,198.35) circle (  1.16);

\path[draw=drawColor,line width= 0.4pt,line join=round,line cap=round,fill=fillColor] (213.72,198.35) circle (  1.16);

\path[draw=drawColor,line width= 0.4pt,line join=round,line cap=round,fill=fillColor] (214.01,198.35) circle (  1.16);

\path[draw=drawColor,line width= 0.4pt,line join=round,line cap=round,fill=fillColor] (214.30,198.35) circle (  1.16);

\path[draw=drawColor,line width= 0.4pt,line join=round,line cap=round,fill=fillColor] (214.59,198.35) circle (  1.16);

\path[draw=drawColor,line width= 0.4pt,line join=round,line cap=round,fill=fillColor] (214.88,198.35) circle (  1.16);

\path[draw=drawColor,line width= 0.4pt,line join=round,line cap=round,fill=fillColor] (215.17,198.35) circle (  1.16);

\path[draw=drawColor,line width= 0.4pt,line join=round,line cap=round,fill=fillColor] (215.46,198.35) circle (  1.16);

\path[draw=drawColor,line width= 0.4pt,line join=round,line cap=round,fill=fillColor] (215.74,198.35) circle (  1.16);

\path[draw=drawColor,line width= 0.4pt,line join=round,line cap=round,fill=fillColor] (216.03,198.35) circle (  1.16);

\path[draw=drawColor,line width= 0.4pt,line join=round,line cap=round,fill=fillColor] (216.31,198.35) circle (  1.16);

\path[draw=drawColor,line width= 0.4pt,line join=round,line cap=round,fill=fillColor] (216.60,198.35) circle (  1.16);

\path[draw=drawColor,line width= 0.4pt,line join=round,line cap=round,fill=fillColor] (216.88,198.35) circle (  1.16);

\path[draw=drawColor,line width= 0.4pt,line join=round,line cap=round,fill=fillColor] (217.16,198.35) circle (  1.16);

\path[draw=drawColor,line width= 0.4pt,line join=round,line cap=round,fill=fillColor] (217.44,198.35) circle (  1.16);

\path[draw=drawColor,line width= 0.4pt,line join=round,line cap=round,fill=fillColor] (217.72,198.35) circle (  1.16);

\path[draw=drawColor,line width= 0.4pt,line join=round,line cap=round,fill=fillColor] (218.00,198.35) circle (  1.16);

\path[draw=drawColor,line width= 0.4pt,line join=round,line cap=round,fill=fillColor] (218.28,198.35) circle (  1.16);

\path[draw=drawColor,line width= 0.4pt,line join=round,line cap=round,fill=fillColor] (218.55,198.35) circle (  1.16);

\path[draw=drawColor,line width= 0.4pt,line join=round,line cap=round,fill=fillColor] (218.83,198.35) circle (  1.16);

\path[draw=drawColor,line width= 0.4pt,line join=round,line cap=round,fill=fillColor] (219.10,198.35) circle (  1.16);
\definecolor[named]{drawColor}{rgb}{0.22,0.49,0.72}
\definecolor[named]{fillColor}{rgb}{0.22,0.49,0.72}

\path[draw=drawColor,line width= 0.4pt,line join=round,line cap=round,fill=fillColor] ( 78.97,223.87) circle (  1.16);

\path[draw=drawColor,line width= 0.4pt,line join=round,line cap=round,fill=fillColor] ( 86.41,222.92) circle (  1.16);

\path[draw=drawColor,line width= 0.4pt,line join=round,line cap=round,fill=fillColor] ( 91.63,222.64) circle (  1.16);

\path[draw=drawColor,line width= 0.4pt,line join=round,line cap=round,fill=fillColor] ( 95.78,220.98) circle (  1.16);

\path[draw=drawColor,line width= 0.4pt,line join=round,line cap=round,fill=fillColor] ( 99.29,220.96) circle (  1.16);

\path[draw=drawColor,line width= 0.4pt,line join=round,line cap=round,fill=fillColor] (102.36,220.82) circle (  1.16);

\path[draw=drawColor,line width= 0.4pt,line join=round,line cap=round,fill=fillColor] (105.10,220.12) circle (  1.16);

\path[draw=drawColor,line width= 0.4pt,line join=round,line cap=round,fill=fillColor] (107.59,219.96) circle (  1.16);

\path[draw=drawColor,line width= 0.4pt,line join=round,line cap=round,fill=fillColor] (109.88,219.81) circle (  1.16);

\path[draw=drawColor,line width= 0.4pt,line join=round,line cap=round,fill=fillColor] (112.01,219.39) circle (  1.16);

\path[draw=drawColor,line width= 0.4pt,line join=round,line cap=round,fill=fillColor] (114.00,218.93) circle (  1.16);

\path[draw=drawColor,line width= 0.4pt,line join=round,line cap=round,fill=fillColor] (115.87,218.79) circle (  1.16);

\path[draw=drawColor,line width= 0.4pt,line join=round,line cap=round,fill=fillColor] (117.65,218.64) circle (  1.16);

\path[draw=drawColor,line width= 0.4pt,line join=round,line cap=round,fill=fillColor] (119.33,218.42) circle (  1.16);

\path[draw=drawColor,line width= 0.4pt,line join=round,line cap=round,fill=fillColor] (120.93,218.32) circle (  1.16);

\path[draw=drawColor,line width= 0.4pt,line join=round,line cap=round,fill=fillColor] (122.47,217.93) circle (  1.16);

\path[draw=drawColor,line width= 0.4pt,line join=round,line cap=round,fill=fillColor] (123.94,217.89) circle (  1.16);

\path[draw=drawColor,line width= 0.4pt,line join=round,line cap=round,fill=fillColor] (125.36,217.88) circle (  1.16);

\path[draw=drawColor,line width= 0.4pt,line join=round,line cap=round,fill=fillColor] (126.72,217.80) circle (  1.16);

\path[draw=drawColor,line width= 0.4pt,line join=round,line cap=round,fill=fillColor] (128.04,217.31) circle (  1.16);

\path[draw=drawColor,line width= 0.4pt,line join=round,line cap=round,fill=fillColor] (129.31,217.11) circle (  1.16);

\path[draw=drawColor,line width= 0.4pt,line join=round,line cap=round,fill=fillColor] (130.54,216.51) circle (  1.16);

\path[draw=drawColor,line width= 0.4pt,line join=round,line cap=round,fill=fillColor] (131.74,216.24) circle (  1.16);

\path[draw=drawColor,line width= 0.4pt,line join=round,line cap=round,fill=fillColor] (132.90,216.10) circle (  1.16);

\path[draw=drawColor,line width= 0.4pt,line join=round,line cap=round,fill=fillColor] (134.04,215.94) circle (  1.16);

\path[draw=drawColor,line width= 0.4pt,line join=round,line cap=round,fill=fillColor] (135.14,215.86) circle (  1.16);

\path[draw=drawColor,line width= 0.4pt,line join=round,line cap=round,fill=fillColor] (136.21,215.83) circle (  1.16);

\path[draw=drawColor,line width= 0.4pt,line join=round,line cap=round,fill=fillColor] (137.26,215.47) circle (  1.16);

\path[draw=drawColor,line width= 0.4pt,line join=round,line cap=round,fill=fillColor] (138.28,215.20) circle (  1.16);

\path[draw=drawColor,line width= 0.4pt,line join=round,line cap=round,fill=fillColor] (139.28,215.14) circle (  1.16);

\path[draw=drawColor,line width= 0.4pt,line join=round,line cap=round,fill=fillColor] (140.26,215.11) circle (  1.16);

\path[draw=drawColor,line width= 0.4pt,line join=round,line cap=round,fill=fillColor] (141.21,215.01) circle (  1.16);

\path[draw=drawColor,line width= 0.4pt,line join=round,line cap=round,fill=fillColor] (142.15,214.83) circle (  1.16);

\path[draw=drawColor,line width= 0.4pt,line join=round,line cap=round,fill=fillColor] (143.07,214.42) circle (  1.16);

\path[draw=drawColor,line width= 0.4pt,line join=round,line cap=round,fill=fillColor] (143.97,214.28) circle (  1.16);

\path[draw=drawColor,line width= 0.4pt,line join=round,line cap=round,fill=fillColor] (144.85,214.23) circle (  1.16);

\path[draw=drawColor,line width= 0.4pt,line join=round,line cap=round,fill=fillColor] (145.72,214.21) circle (  1.16);

\path[draw=drawColor,line width= 0.4pt,line join=round,line cap=round,fill=fillColor] (146.57,214.21) circle (  1.16);

\path[draw=drawColor,line width= 0.4pt,line join=round,line cap=round,fill=fillColor] (147.41,214.15) circle (  1.16);

\path[draw=drawColor,line width= 0.4pt,line join=round,line cap=round,fill=fillColor] (148.23,214.01) circle (  1.16);

\path[draw=drawColor,line width= 0.4pt,line join=round,line cap=round,fill=fillColor] (149.04,214.00) circle (  1.16);

\path[draw=drawColor,line width= 0.4pt,line join=round,line cap=round,fill=fillColor] (149.83,214.00) circle (  1.16);

\path[draw=drawColor,line width= 0.4pt,line join=round,line cap=round,fill=fillColor] (150.62,213.98) circle (  1.16);

\path[draw=drawColor,line width= 0.4pt,line join=round,line cap=round,fill=fillColor] (151.39,213.86) circle (  1.16);

\path[draw=drawColor,line width= 0.4pt,line join=round,line cap=round,fill=fillColor] (152.15,213.84) circle (  1.16);

\path[draw=drawColor,line width= 0.4pt,line join=round,line cap=round,fill=fillColor] (152.90,213.70) circle (  1.16);

\path[draw=drawColor,line width= 0.4pt,line join=round,line cap=round,fill=fillColor] (153.63,213.63) circle (  1.16);

\path[draw=drawColor,line width= 0.4pt,line join=round,line cap=round,fill=fillColor] (154.36,213.31) circle (  1.16);

\path[draw=drawColor,line width= 0.4pt,line join=round,line cap=round,fill=fillColor] (155.08,213.30) circle (  1.16);

\path[draw=drawColor,line width= 0.4pt,line join=round,line cap=round,fill=fillColor] (155.79,213.20) circle (  1.16);

\path[draw=drawColor,line width= 0.4pt,line join=round,line cap=round,fill=fillColor] (156.48,212.89) circle (  1.16);

\path[draw=drawColor,line width= 0.4pt,line join=round,line cap=round,fill=fillColor] (157.17,212.78) circle (  1.16);

\path[draw=drawColor,line width= 0.4pt,line join=round,line cap=round,fill=fillColor] (157.85,212.72) circle (  1.16);

\path[draw=drawColor,line width= 0.4pt,line join=round,line cap=round,fill=fillColor] (158.53,212.62) circle (  1.16);

\path[draw=drawColor,line width= 0.4pt,line join=round,line cap=round,fill=fillColor] (159.19,212.50) circle (  1.16);

\path[draw=drawColor,line width= 0.4pt,line join=round,line cap=round,fill=fillColor] (159.85,212.36) circle (  1.16);

\path[draw=drawColor,line width= 0.4pt,line join=round,line cap=round,fill=fillColor] (160.49,211.96) circle (  1.16);

\path[draw=drawColor,line width= 0.4pt,line join=round,line cap=round,fill=fillColor] (161.13,211.85) circle (  1.16);

\path[draw=drawColor,line width= 0.4pt,line join=round,line cap=round,fill=fillColor] (161.77,211.74) circle (  1.16);

\path[draw=drawColor,line width= 0.4pt,line join=round,line cap=round,fill=fillColor] (162.39,211.65) circle (  1.16);

\path[draw=drawColor,line width= 0.4pt,line join=round,line cap=round,fill=fillColor] (163.01,211.45) circle (  1.16);

\path[draw=drawColor,line width= 0.4pt,line join=round,line cap=round,fill=fillColor] (163.62,211.22) circle (  1.16);

\path[draw=drawColor,line width= 0.4pt,line join=round,line cap=round,fill=fillColor] (164.23,210.89) circle (  1.16);

\path[draw=drawColor,line width= 0.4pt,line join=round,line cap=round,fill=fillColor] (164.83,210.87) circle (  1.16);

\path[draw=drawColor,line width= 0.4pt,line join=round,line cap=round,fill=fillColor] (165.42,210.73) circle (  1.16);

\path[draw=drawColor,line width= 0.4pt,line join=round,line cap=round,fill=fillColor] (166.01,210.67) circle (  1.16);

\path[draw=drawColor,line width= 0.4pt,line join=round,line cap=round,fill=fillColor] (166.59,210.67) circle (  1.16);

\path[draw=drawColor,line width= 0.4pt,line join=round,line cap=round,fill=fillColor] (167.17,210.67) circle (  1.16);

\path[draw=drawColor,line width= 0.4pt,line join=round,line cap=round,fill=fillColor] (167.74,210.29) circle (  1.16);

\path[draw=drawColor,line width= 0.4pt,line join=round,line cap=round,fill=fillColor] (168.30,210.19) circle (  1.16);

\path[draw=drawColor,line width= 0.4pt,line join=round,line cap=round,fill=fillColor] (168.86,210.16) circle (  1.16);

\path[draw=drawColor,line width= 0.4pt,line join=round,line cap=round,fill=fillColor] (169.41,209.98) circle (  1.16);

\path[draw=drawColor,line width= 0.4pt,line join=round,line cap=round,fill=fillColor] (169.96,209.89) circle (  1.16);

\path[draw=drawColor,line width= 0.4pt,line join=round,line cap=round,fill=fillColor] (170.51,209.79) circle (  1.16);

\path[draw=drawColor,line width= 0.4pt,line join=round,line cap=round,fill=fillColor] (171.04,209.75) circle (  1.16);

\path[draw=drawColor,line width= 0.4pt,line join=round,line cap=round,fill=fillColor] (171.58,209.74) circle (  1.16);

\path[draw=drawColor,line width= 0.4pt,line join=round,line cap=round,fill=fillColor] (172.11,209.67) circle (  1.16);

\path[draw=drawColor,line width= 0.4pt,line join=round,line cap=round,fill=fillColor] (172.63,209.64) circle (  1.16);

\path[draw=drawColor,line width= 0.4pt,line join=round,line cap=round,fill=fillColor] (173.15,209.53) circle (  1.16);

\path[draw=drawColor,line width= 0.4pt,line join=round,line cap=round,fill=fillColor] (173.67,209.27) circle (  1.16);

\path[draw=drawColor,line width= 0.4pt,line join=round,line cap=round,fill=fillColor] (174.18,209.27) circle (  1.16);

\path[draw=drawColor,line width= 0.4pt,line join=round,line cap=round,fill=fillColor] (174.69,209.23) circle (  1.16);

\path[draw=drawColor,line width= 0.4pt,line join=round,line cap=round,fill=fillColor] (175.19,209.01) circle (  1.16);

\path[draw=drawColor,line width= 0.4pt,line join=round,line cap=round,fill=fillColor] (175.69,209.00) circle (  1.16);

\path[draw=drawColor,line width= 0.4pt,line join=round,line cap=round,fill=fillColor] (176.19,208.95) circle (  1.16);

\path[draw=drawColor,line width= 0.4pt,line join=round,line cap=round,fill=fillColor] (176.68,208.92) circle (  1.16);

\path[draw=drawColor,line width= 0.4pt,line join=round,line cap=round,fill=fillColor] (177.17,208.82) circle (  1.16);

\path[draw=drawColor,line width= 0.4pt,line join=round,line cap=round,fill=fillColor] (177.65,208.77) circle (  1.16);

\path[draw=drawColor,line width= 0.4pt,line join=round,line cap=round,fill=fillColor] (178.13,208.77) circle (  1.16);

\path[draw=drawColor,line width= 0.4pt,line join=round,line cap=round,fill=fillColor] (178.61,208.39) circle (  1.16);

\path[draw=drawColor,line width= 0.4pt,line join=round,line cap=round,fill=fillColor] (179.08,208.30) circle (  1.16);

\path[draw=drawColor,line width= 0.4pt,line join=round,line cap=round,fill=fillColor] (179.55,208.19) circle (  1.16);

\path[draw=drawColor,line width= 0.4pt,line join=round,line cap=round,fill=fillColor] (180.02,207.38) circle (  1.16);

\path[draw=drawColor,line width= 0.4pt,line join=round,line cap=round,fill=fillColor] (180.48,207.32) circle (  1.16);

\path[draw=drawColor,line width= 0.4pt,line join=round,line cap=round,fill=fillColor] (180.94,207.15) circle (  1.16);

\path[draw=drawColor,line width= 0.4pt,line join=round,line cap=round,fill=fillColor] (181.40,207.13) circle (  1.16);

\path[draw=drawColor,line width= 0.4pt,line join=round,line cap=round,fill=fillColor] (181.85,207.07) circle (  1.16);

\path[draw=drawColor,line width= 0.4pt,line join=round,line cap=round,fill=fillColor] (182.30,207.04) circle (  1.16);

\path[draw=drawColor,line width= 0.4pt,line join=round,line cap=round,fill=fillColor] (182.75,207.00) circle (  1.16);

\path[draw=drawColor,line width= 0.4pt,line join=round,line cap=round,fill=fillColor] (183.19,206.95) circle (  1.16);

\path[draw=drawColor,line width= 0.4pt,line join=round,line cap=round,fill=fillColor] (183.63,206.94) circle (  1.16);

\path[draw=drawColor,line width= 0.4pt,line join=round,line cap=round,fill=fillColor] (184.07,206.91) circle (  1.16);

\path[draw=drawColor,line width= 0.4pt,line join=round,line cap=round,fill=fillColor] (184.51,206.87) circle (  1.16);

\path[draw=drawColor,line width= 0.4pt,line join=round,line cap=round,fill=fillColor] (184.94,206.86) circle (  1.16);

\path[draw=drawColor,line width= 0.4pt,line join=round,line cap=round,fill=fillColor] (185.37,206.85) circle (  1.16);

\path[draw=drawColor,line width= 0.4pt,line join=round,line cap=round,fill=fillColor] (185.80,206.26) circle (  1.16);

\path[draw=drawColor,line width= 0.4pt,line join=round,line cap=round,fill=fillColor] (186.22,206.01) circle (  1.16);

\path[draw=drawColor,line width= 0.4pt,line join=round,line cap=round,fill=fillColor] (186.64,205.96) circle (  1.16);

\path[draw=drawColor,line width= 0.4pt,line join=round,line cap=round,fill=fillColor] (187.06,205.89) circle (  1.16);

\path[draw=drawColor,line width= 0.4pt,line join=round,line cap=round,fill=fillColor] (187.48,205.73) circle (  1.16);

\path[draw=drawColor,line width= 0.4pt,line join=round,line cap=round,fill=fillColor] (187.89,205.63) circle (  1.16);

\path[draw=drawColor,line width= 0.4pt,line join=round,line cap=round,fill=fillColor] (188.31,205.56) circle (  1.16);

\path[draw=drawColor,line width= 0.4pt,line join=round,line cap=round,fill=fillColor] (188.72,205.52) circle (  1.16);

\path[draw=drawColor,line width= 0.4pt,line join=round,line cap=round,fill=fillColor] (189.12,205.37) circle (  1.16);

\path[draw=drawColor,line width= 0.4pt,line join=round,line cap=round,fill=fillColor] (189.53,205.37) circle (  1.16);

\path[draw=drawColor,line width= 0.4pt,line join=round,line cap=round,fill=fillColor] (189.93,205.31) circle (  1.16);

\path[draw=drawColor,line width= 0.4pt,line join=round,line cap=round,fill=fillColor] (190.33,205.05) circle (  1.16);

\path[draw=drawColor,line width= 0.4pt,line join=round,line cap=round,fill=fillColor] (190.73,204.92) circle (  1.16);

\path[draw=drawColor,line width= 0.4pt,line join=round,line cap=round,fill=fillColor] (191.12,204.85) circle (  1.16);

\path[draw=drawColor,line width= 0.4pt,line join=round,line cap=round,fill=fillColor] (191.52,204.30) circle (  1.16);

\path[draw=drawColor,line width= 0.4pt,line join=round,line cap=round,fill=fillColor] (191.91,203.99) circle (  1.16);

\path[draw=drawColor,line width= 0.4pt,line join=round,line cap=round,fill=fillColor] (192.30,203.97) circle (  1.16);

\path[draw=drawColor,line width= 0.4pt,line join=round,line cap=round,fill=fillColor] (192.68,203.78) circle (  1.16);

\path[draw=drawColor,line width= 0.4pt,line join=round,line cap=round,fill=fillColor] (193.07,202.08) circle (  1.16);

\path[draw=drawColor,line width= 0.4pt,line join=round,line cap=round,fill=fillColor] (193.45,201.69) circle (  1.16);

\path[draw=drawColor,line width= 0.4pt,line join=round,line cap=round,fill=fillColor] (193.83,198.35) circle (  1.16);

\path[draw=drawColor,line width= 0.4pt,line join=round,line cap=round,fill=fillColor] (194.21,198.35) circle (  1.16);

\path[draw=drawColor,line width= 0.4pt,line join=round,line cap=round,fill=fillColor] (194.59,198.35) circle (  1.16);

\path[draw=drawColor,line width= 0.4pt,line join=round,line cap=round,fill=fillColor] (194.96,198.35) circle (  1.16);

\path[draw=drawColor,line width= 0.4pt,line join=round,line cap=round,fill=fillColor] (195.33,198.35) circle (  1.16);

\path[draw=drawColor,line width= 0.4pt,line join=round,line cap=round,fill=fillColor] (195.70,198.35) circle (  1.16);

\path[draw=drawColor,line width= 0.4pt,line join=round,line cap=round,fill=fillColor] (196.07,198.35) circle (  1.16);

\path[draw=drawColor,line width= 0.4pt,line join=round,line cap=round,fill=fillColor] (196.44,198.35) circle (  1.16);

\path[draw=drawColor,line width= 0.4pt,line join=round,line cap=round,fill=fillColor] (196.80,198.35) circle (  1.16);

\path[draw=drawColor,line width= 0.4pt,line join=round,line cap=round,fill=fillColor] (197.17,198.35) circle (  1.16);

\path[draw=drawColor,line width= 0.4pt,line join=round,line cap=round,fill=fillColor] (197.53,198.35) circle (  1.16);

\path[draw=drawColor,line width= 0.4pt,line join=round,line cap=round,fill=fillColor] (197.89,198.35) circle (  1.16);

\path[draw=drawColor,line width= 0.4pt,line join=round,line cap=round,fill=fillColor] (198.25,198.35) circle (  1.16);

\path[draw=drawColor,line width= 0.4pt,line join=round,line cap=round,fill=fillColor] (198.60,198.35) circle (  1.16);

\path[draw=drawColor,line width= 0.4pt,line join=round,line cap=round,fill=fillColor] (198.96,198.35) circle (  1.16);

\path[draw=drawColor,line width= 0.4pt,line join=round,line cap=round,fill=fillColor] (199.31,198.35) circle (  1.16);

\path[draw=drawColor,line width= 0.4pt,line join=round,line cap=round,fill=fillColor] (199.66,198.35) circle (  1.16);

\path[draw=drawColor,line width= 0.4pt,line join=round,line cap=round,fill=fillColor] (200.01,198.35) circle (  1.16);

\path[draw=drawColor,line width= 0.4pt,line join=round,line cap=round,fill=fillColor] (200.36,198.35) circle (  1.16);

\path[draw=drawColor,line width= 0.4pt,line join=round,line cap=round,fill=fillColor] (200.71,198.35) circle (  1.16);

\path[draw=drawColor,line width= 0.4pt,line join=round,line cap=round,fill=fillColor] (201.05,198.35) circle (  1.16);

\path[draw=drawColor,line width= 0.4pt,line join=round,line cap=round,fill=fillColor] (201.40,198.35) circle (  1.16);

\path[draw=drawColor,line width= 0.4pt,line join=round,line cap=round,fill=fillColor] (201.74,198.35) circle (  1.16);

\path[draw=drawColor,line width= 0.4pt,line join=round,line cap=round,fill=fillColor] (202.08,198.35) circle (  1.16);

\path[draw=drawColor,line width= 0.4pt,line join=round,line cap=round,fill=fillColor] (202.42,198.35) circle (  1.16);

\path[draw=drawColor,line width= 0.4pt,line join=round,line cap=round,fill=fillColor] (202.75,198.35) circle (  1.16);

\path[draw=drawColor,line width= 0.4pt,line join=round,line cap=round,fill=fillColor] (203.09,198.35) circle (  1.16);

\path[draw=drawColor,line width= 0.4pt,line join=round,line cap=round,fill=fillColor] (203.42,198.35) circle (  1.16);

\path[draw=drawColor,line width= 0.4pt,line join=round,line cap=round,fill=fillColor] (203.76,198.35) circle (  1.16);

\path[draw=drawColor,line width= 0.4pt,line join=round,line cap=round,fill=fillColor] (204.09,198.35) circle (  1.16);

\path[draw=drawColor,line width= 0.4pt,line join=round,line cap=round,fill=fillColor] (204.42,198.35) circle (  1.16);

\path[draw=drawColor,line width= 0.4pt,line join=round,line cap=round,fill=fillColor] (204.75,198.35) circle (  1.16);

\path[draw=drawColor,line width= 0.4pt,line join=round,line cap=round,fill=fillColor] (205.07,198.35) circle (  1.16);

\path[draw=drawColor,line width= 0.4pt,line join=round,line cap=round,fill=fillColor] (205.40,198.35) circle (  1.16);

\path[draw=drawColor,line width= 0.4pt,line join=round,line cap=round,fill=fillColor] (205.72,198.35) circle (  1.16);

\path[draw=drawColor,line width= 0.4pt,line join=round,line cap=round,fill=fillColor] (206.05,198.35) circle (  1.16);

\path[draw=drawColor,line width= 0.4pt,line join=round,line cap=round,fill=fillColor] (206.37,198.35) circle (  1.16);

\path[draw=drawColor,line width= 0.4pt,line join=round,line cap=round,fill=fillColor] (206.69,198.35) circle (  1.16);

\path[draw=drawColor,line width= 0.4pt,line join=round,line cap=round,fill=fillColor] (207.01,198.35) circle (  1.16);

\path[draw=drawColor,line width= 0.4pt,line join=round,line cap=round,fill=fillColor] (207.32,198.35) circle (  1.16);

\path[draw=drawColor,line width= 0.4pt,line join=round,line cap=round,fill=fillColor] (207.64,198.35) circle (  1.16);

\path[draw=drawColor,line width= 0.4pt,line join=round,line cap=round,fill=fillColor] (207.96,198.35) circle (  1.16);

\path[draw=drawColor,line width= 0.4pt,line join=round,line cap=round,fill=fillColor] (208.27,198.35) circle (  1.16);

\path[draw=drawColor,line width= 0.4pt,line join=round,line cap=round,fill=fillColor] (208.58,198.35) circle (  1.16);

\path[draw=drawColor,line width= 0.4pt,line join=round,line cap=round,fill=fillColor] (208.89,198.35) circle (  1.16);

\path[draw=drawColor,line width= 0.4pt,line join=round,line cap=round,fill=fillColor] (209.20,198.35) circle (  1.16);

\path[draw=drawColor,line width= 0.4pt,line join=round,line cap=round,fill=fillColor] (209.51,198.35) circle (  1.16);

\path[draw=drawColor,line width= 0.4pt,line join=round,line cap=round,fill=fillColor] (209.82,198.35) circle (  1.16);

\path[draw=drawColor,line width= 0.4pt,line join=round,line cap=round,fill=fillColor] (210.13,198.35) circle (  1.16);

\path[draw=drawColor,line width= 0.4pt,line join=round,line cap=round,fill=fillColor] (210.43,198.35) circle (  1.16);

\path[draw=drawColor,line width= 0.4pt,line join=round,line cap=round,fill=fillColor] (210.74,198.35) circle (  1.16);

\path[draw=drawColor,line width= 0.4pt,line join=round,line cap=round,fill=fillColor] (211.04,198.35) circle (  1.16);

\path[draw=drawColor,line width= 0.4pt,line join=round,line cap=round,fill=fillColor] (211.34,198.35) circle (  1.16);

\path[draw=drawColor,line width= 0.4pt,line join=round,line cap=round,fill=fillColor] (211.64,198.35) circle (  1.16);

\path[draw=drawColor,line width= 0.4pt,line join=round,line cap=round,fill=fillColor] (211.94,198.35) circle (  1.16);

\path[draw=drawColor,line width= 0.4pt,line join=round,line cap=round,fill=fillColor] (212.24,198.35) circle (  1.16);

\path[draw=drawColor,line width= 0.4pt,line join=round,line cap=round,fill=fillColor] (212.54,198.35) circle (  1.16);

\path[draw=drawColor,line width= 0.4pt,line join=round,line cap=round,fill=fillColor] (212.84,198.35) circle (  1.16);

\path[draw=drawColor,line width= 0.4pt,line join=round,line cap=round,fill=fillColor] (213.13,198.35) circle (  1.16);

\path[draw=drawColor,line width= 0.4pt,line join=round,line cap=round,fill=fillColor] (213.43,198.35) circle (  1.16);

\path[draw=drawColor,line width= 0.4pt,line join=round,line cap=round,fill=fillColor] (213.72,198.35) circle (  1.16);

\path[draw=drawColor,line width= 0.4pt,line join=round,line cap=round,fill=fillColor] (214.01,198.35) circle (  1.16);

\path[draw=drawColor,line width= 0.4pt,line join=round,line cap=round,fill=fillColor] (214.30,198.35) circle (  1.16);

\path[draw=drawColor,line width= 0.4pt,line join=round,line cap=round,fill=fillColor] (214.59,198.35) circle (  1.16);

\path[draw=drawColor,line width= 0.4pt,line join=round,line cap=round,fill=fillColor] (214.88,198.35) circle (  1.16);

\path[draw=drawColor,line width= 0.4pt,line join=round,line cap=round,fill=fillColor] (215.17,198.35) circle (  1.16);

\path[draw=drawColor,line width= 0.4pt,line join=round,line cap=round,fill=fillColor] (215.46,198.35) circle (  1.16);

\path[draw=drawColor,line width= 0.4pt,line join=round,line cap=round,fill=fillColor] (215.74,198.35) circle (  1.16);

\path[draw=drawColor,line width= 0.4pt,line join=round,line cap=round,fill=fillColor] (216.03,198.35) circle (  1.16);

\path[draw=drawColor,line width= 0.4pt,line join=round,line cap=round,fill=fillColor] (216.31,198.35) circle (  1.16);

\path[draw=drawColor,line width= 0.4pt,line join=round,line cap=round,fill=fillColor] (216.60,198.35) circle (  1.16);

\path[draw=drawColor,line width= 0.4pt,line join=round,line cap=round,fill=fillColor] (216.88,198.35) circle (  1.16);

\path[draw=drawColor,line width= 0.4pt,line join=round,line cap=round,fill=fillColor] (217.16,198.35) circle (  1.16);

\path[draw=drawColor,line width= 0.4pt,line join=round,line cap=round,fill=fillColor] (217.44,198.35) circle (  1.16);

\path[draw=drawColor,line width= 0.4pt,line join=round,line cap=round,fill=fillColor] (217.72,198.35) circle (  1.16);

\path[draw=drawColor,line width= 0.4pt,line join=round,line cap=round,fill=fillColor] (218.00,198.35) circle (  1.16);

\path[draw=drawColor,line width= 0.4pt,line join=round,line cap=round,fill=fillColor] (218.28,198.35) circle (  1.16);

\path[draw=drawColor,line width= 0.4pt,line join=round,line cap=round,fill=fillColor] (218.55,198.35) circle (  1.16);

\path[draw=drawColor,line width= 0.4pt,line join=round,line cap=round,fill=fillColor] (218.83,198.35) circle (  1.16);

\path[draw=drawColor,line width= 0.4pt,line join=round,line cap=round,fill=fillColor] (219.10,198.35) circle (  1.16);
\definecolor[named]{drawColor}{rgb}{0.30,0.69,0.29}
\definecolor[named]{fillColor}{rgb}{0.30,0.69,0.29}

\path[draw=drawColor,line width= 0.4pt,line join=round,line cap=round,fill=fillColor] ( 78.97,225.56) circle (  1.16);

\path[draw=drawColor,line width= 0.4pt,line join=round,line cap=round,fill=fillColor] ( 86.41,225.28) circle (  1.16);

\path[draw=drawColor,line width= 0.4pt,line join=round,line cap=round,fill=fillColor] ( 91.63,223.87) circle (  1.16);

\path[draw=drawColor,line width= 0.4pt,line join=round,line cap=round,fill=fillColor] ( 95.78,222.34) circle (  1.16);

\path[draw=drawColor,line width= 0.4pt,line join=round,line cap=round,fill=fillColor] ( 99.29,221.60) circle (  1.16);

\path[draw=drawColor,line width= 0.4pt,line join=round,line cap=round,fill=fillColor] (102.36,221.44) circle (  1.16);

\path[draw=drawColor,line width= 0.4pt,line join=round,line cap=round,fill=fillColor] (105.10,221.36) circle (  1.16);

\path[draw=drawColor,line width= 0.4pt,line join=round,line cap=round,fill=fillColor] (107.59,219.08) circle (  1.16);

\path[draw=drawColor,line width= 0.4pt,line join=round,line cap=round,fill=fillColor] (109.88,219.06) circle (  1.16);

\path[draw=drawColor,line width= 0.4pt,line join=round,line cap=round,fill=fillColor] (112.01,218.42) circle (  1.16);

\path[draw=drawColor,line width= 0.4pt,line join=round,line cap=round,fill=fillColor] (114.00,218.35) circle (  1.16);

\path[draw=drawColor,line width= 0.4pt,line join=round,line cap=round,fill=fillColor] (115.87,218.30) circle (  1.16);

\path[draw=drawColor,line width= 0.4pt,line join=round,line cap=round,fill=fillColor] (117.65,218.18) circle (  1.16);

\path[draw=drawColor,line width= 0.4pt,line join=round,line cap=round,fill=fillColor] (119.33,217.89) circle (  1.16);

\path[draw=drawColor,line width= 0.4pt,line join=round,line cap=round,fill=fillColor] (120.93,217.80) circle (  1.16);

\path[draw=drawColor,line width= 0.4pt,line join=round,line cap=round,fill=fillColor] (122.47,217.74) circle (  1.16);

\path[draw=drawColor,line width= 0.4pt,line join=round,line cap=round,fill=fillColor] (123.94,217.68) circle (  1.16);

\path[draw=drawColor,line width= 0.4pt,line join=round,line cap=round,fill=fillColor] (125.36,217.51) circle (  1.16);

\path[draw=drawColor,line width= 0.4pt,line join=round,line cap=round,fill=fillColor] (126.72,217.31) circle (  1.16);

\path[draw=drawColor,line width= 0.4pt,line join=round,line cap=round,fill=fillColor] (128.04,216.81) circle (  1.16);

\path[draw=drawColor,line width= 0.4pt,line join=round,line cap=round,fill=fillColor] (129.31,216.13) circle (  1.16);

\path[draw=drawColor,line width= 0.4pt,line join=round,line cap=round,fill=fillColor] (130.54,216.12) circle (  1.16);

\path[draw=drawColor,line width= 0.4pt,line join=round,line cap=round,fill=fillColor] (131.74,216.10) circle (  1.16);

\path[draw=drawColor,line width= 0.4pt,line join=round,line cap=round,fill=fillColor] (132.90,216.10) circle (  1.16);

\path[draw=drawColor,line width= 0.4pt,line join=round,line cap=round,fill=fillColor] (134.04,215.88) circle (  1.16);

\path[draw=drawColor,line width= 0.4pt,line join=round,line cap=round,fill=fillColor] (135.14,215.87) circle (  1.16);

\path[draw=drawColor,line width= 0.4pt,line join=round,line cap=round,fill=fillColor] (136.21,215.76) circle (  1.16);

\path[draw=drawColor,line width= 0.4pt,line join=round,line cap=round,fill=fillColor] (137.26,215.75) circle (  1.16);

\path[draw=drawColor,line width= 0.4pt,line join=round,line cap=round,fill=fillColor] (138.28,215.47) circle (  1.16);

\path[draw=drawColor,line width= 0.4pt,line join=round,line cap=round,fill=fillColor] (139.28,215.32) circle (  1.16);

\path[draw=drawColor,line width= 0.4pt,line join=round,line cap=round,fill=fillColor] (140.26,215.19) circle (  1.16);

\path[draw=drawColor,line width= 0.4pt,line join=round,line cap=round,fill=fillColor] (141.21,215.14) circle (  1.16);

\path[draw=drawColor,line width= 0.4pt,line join=round,line cap=round,fill=fillColor] (142.15,214.86) circle (  1.16);

\path[draw=drawColor,line width= 0.4pt,line join=round,line cap=round,fill=fillColor] (143.07,214.45) circle (  1.16);

\path[draw=drawColor,line width= 0.4pt,line join=round,line cap=round,fill=fillColor] (143.97,214.42) circle (  1.16);

\path[draw=drawColor,line width= 0.4pt,line join=round,line cap=round,fill=fillColor] (144.85,214.40) circle (  1.16);

\path[draw=drawColor,line width= 0.4pt,line join=round,line cap=round,fill=fillColor] (145.72,214.39) circle (  1.16);

\path[draw=drawColor,line width= 0.4pt,line join=round,line cap=round,fill=fillColor] (146.57,214.31) circle (  1.16);

\path[draw=drawColor,line width= 0.4pt,line join=round,line cap=round,fill=fillColor] (147.41,214.27) circle (  1.16);

\path[draw=drawColor,line width= 0.4pt,line join=round,line cap=round,fill=fillColor] (148.23,214.23) circle (  1.16);

\path[draw=drawColor,line width= 0.4pt,line join=round,line cap=round,fill=fillColor] (149.04,214.21) circle (  1.16);

\path[draw=drawColor,line width= 0.4pt,line join=round,line cap=round,fill=fillColor] (149.83,214.03) circle (  1.16);

\path[draw=drawColor,line width= 0.4pt,line join=round,line cap=round,fill=fillColor] (150.62,214.00) circle (  1.16);

\path[draw=drawColor,line width= 0.4pt,line join=round,line cap=round,fill=fillColor] (151.39,213.87) circle (  1.16);

\path[draw=drawColor,line width= 0.4pt,line join=round,line cap=round,fill=fillColor] (152.15,213.84) circle (  1.16);

\path[draw=drawColor,line width= 0.4pt,line join=round,line cap=round,fill=fillColor] (152.90,213.72) circle (  1.16);

\path[draw=drawColor,line width= 0.4pt,line join=round,line cap=round,fill=fillColor] (153.63,213.62) circle (  1.16);

\path[draw=drawColor,line width= 0.4pt,line join=round,line cap=round,fill=fillColor] (154.36,213.56) circle (  1.16);

\path[draw=drawColor,line width= 0.4pt,line join=round,line cap=round,fill=fillColor] (155.08,213.30) circle (  1.16);

\path[draw=drawColor,line width= 0.4pt,line join=round,line cap=round,fill=fillColor] (155.79,212.95) circle (  1.16);

\path[draw=drawColor,line width= 0.4pt,line join=round,line cap=round,fill=fillColor] (156.48,212.85) circle (  1.16);

\path[draw=drawColor,line width= 0.4pt,line join=round,line cap=round,fill=fillColor] (157.17,212.78) circle (  1.16);

\path[draw=drawColor,line width= 0.4pt,line join=round,line cap=round,fill=fillColor] (157.85,212.74) circle (  1.16);

\path[draw=drawColor,line width= 0.4pt,line join=round,line cap=round,fill=fillColor] (158.53,212.61) circle (  1.16);

\path[draw=drawColor,line width= 0.4pt,line join=round,line cap=round,fill=fillColor] (159.19,212.55) circle (  1.16);

\path[draw=drawColor,line width= 0.4pt,line join=round,line cap=round,fill=fillColor] (159.85,212.52) circle (  1.16);

\path[draw=drawColor,line width= 0.4pt,line join=round,line cap=round,fill=fillColor] (160.49,212.47) circle (  1.16);

\path[draw=drawColor,line width= 0.4pt,line join=round,line cap=round,fill=fillColor] (161.13,212.47) circle (  1.16);

\path[draw=drawColor,line width= 0.4pt,line join=round,line cap=round,fill=fillColor] (161.77,212.36) circle (  1.16);

\path[draw=drawColor,line width= 0.4pt,line join=round,line cap=round,fill=fillColor] (162.39,212.26) circle (  1.16);

\path[draw=drawColor,line width= 0.4pt,line join=round,line cap=round,fill=fillColor] (163.01,212.23) circle (  1.16);

\path[draw=drawColor,line width= 0.4pt,line join=round,line cap=round,fill=fillColor] (163.62,212.19) circle (  1.16);

\path[draw=drawColor,line width= 0.4pt,line join=round,line cap=round,fill=fillColor] (164.23,212.12) circle (  1.16);

\path[draw=drawColor,line width= 0.4pt,line join=round,line cap=round,fill=fillColor] (164.83,211.79) circle (  1.16);

\path[draw=drawColor,line width= 0.4pt,line join=round,line cap=round,fill=fillColor] (165.42,211.76) circle (  1.16);

\path[draw=drawColor,line width= 0.4pt,line join=round,line cap=round,fill=fillColor] (166.01,211.72) circle (  1.16);

\path[draw=drawColor,line width= 0.4pt,line join=round,line cap=round,fill=fillColor] (166.59,211.70) circle (  1.16);

\path[draw=drawColor,line width= 0.4pt,line join=round,line cap=round,fill=fillColor] (167.17,211.67) circle (  1.16);

\path[draw=drawColor,line width= 0.4pt,line join=round,line cap=round,fill=fillColor] (167.74,211.56) circle (  1.16);

\path[draw=drawColor,line width= 0.4pt,line join=round,line cap=round,fill=fillColor] (168.30,211.56) circle (  1.16);

\path[draw=drawColor,line width= 0.4pt,line join=round,line cap=round,fill=fillColor] (168.86,211.45) circle (  1.16);

\path[draw=drawColor,line width= 0.4pt,line join=round,line cap=round,fill=fillColor] (169.41,211.39) circle (  1.16);

\path[draw=drawColor,line width= 0.4pt,line join=round,line cap=round,fill=fillColor] (169.96,211.36) circle (  1.16);

\path[draw=drawColor,line width= 0.4pt,line join=round,line cap=round,fill=fillColor] (170.51,211.18) circle (  1.16);

\path[draw=drawColor,line width= 0.4pt,line join=round,line cap=round,fill=fillColor] (171.04,210.81) circle (  1.16);

\path[draw=drawColor,line width= 0.4pt,line join=round,line cap=round,fill=fillColor] (171.58,210.63) circle (  1.16);

\path[draw=drawColor,line width= 0.4pt,line join=round,line cap=round,fill=fillColor] (172.11,210.57) circle (  1.16);

\path[draw=drawColor,line width= 0.4pt,line join=round,line cap=round,fill=fillColor] (172.63,210.55) circle (  1.16);

\path[draw=drawColor,line width= 0.4pt,line join=round,line cap=round,fill=fillColor] (173.15,210.31) circle (  1.16);

\path[draw=drawColor,line width= 0.4pt,line join=round,line cap=round,fill=fillColor] (173.67,210.24) circle (  1.16);

\path[draw=drawColor,line width= 0.4pt,line join=round,line cap=round,fill=fillColor] (174.18,210.23) circle (  1.16);

\path[draw=drawColor,line width= 0.4pt,line join=round,line cap=round,fill=fillColor] (174.69,210.20) circle (  1.16);

\path[draw=drawColor,line width= 0.4pt,line join=round,line cap=round,fill=fillColor] (175.19,210.15) circle (  1.16);

\path[draw=drawColor,line width= 0.4pt,line join=round,line cap=round,fill=fillColor] (175.69,210.13) circle (  1.16);

\path[draw=drawColor,line width= 0.4pt,line join=round,line cap=round,fill=fillColor] (176.19,209.98) circle (  1.16);

\path[draw=drawColor,line width= 0.4pt,line join=round,line cap=round,fill=fillColor] (176.68,209.92) circle (  1.16);

\path[draw=drawColor,line width= 0.4pt,line join=round,line cap=round,fill=fillColor] (177.17,209.91) circle (  1.16);

\path[draw=drawColor,line width= 0.4pt,line join=round,line cap=round,fill=fillColor] (177.65,209.84) circle (  1.16);

\path[draw=drawColor,line width= 0.4pt,line join=round,line cap=round,fill=fillColor] (178.13,209.64) circle (  1.16);

\path[draw=drawColor,line width= 0.4pt,line join=round,line cap=round,fill=fillColor] (178.61,209.54) circle (  1.16);

\path[draw=drawColor,line width= 0.4pt,line join=round,line cap=round,fill=fillColor] (179.08,209.45) circle (  1.16);

\path[draw=drawColor,line width= 0.4pt,line join=round,line cap=round,fill=fillColor] (179.55,209.36) circle (  1.16);

\path[draw=drawColor,line width= 0.4pt,line join=round,line cap=round,fill=fillColor] (180.02,209.34) circle (  1.16);

\path[draw=drawColor,line width= 0.4pt,line join=round,line cap=round,fill=fillColor] (180.48,209.34) circle (  1.16);

\path[draw=drawColor,line width= 0.4pt,line join=round,line cap=round,fill=fillColor] (180.94,209.27) circle (  1.16);

\path[draw=drawColor,line width= 0.4pt,line join=round,line cap=round,fill=fillColor] (181.40,209.08) circle (  1.16);

\path[draw=drawColor,line width= 0.4pt,line join=round,line cap=round,fill=fillColor] (181.85,209.08) circle (  1.16);

\path[draw=drawColor,line width= 0.4pt,line join=round,line cap=round,fill=fillColor] (182.30,209.00) circle (  1.16);

\path[draw=drawColor,line width= 0.4pt,line join=round,line cap=round,fill=fillColor] (182.75,208.97) circle (  1.16);

\path[draw=drawColor,line width= 0.4pt,line join=round,line cap=round,fill=fillColor] (183.19,208.92) circle (  1.16);

\path[draw=drawColor,line width= 0.4pt,line join=round,line cap=round,fill=fillColor] (183.63,208.91) circle (  1.16);

\path[draw=drawColor,line width= 0.4pt,line join=round,line cap=round,fill=fillColor] (184.07,208.77) circle (  1.16);

\path[draw=drawColor,line width= 0.4pt,line join=round,line cap=round,fill=fillColor] (184.51,208.72) circle (  1.16);

\path[draw=drawColor,line width= 0.4pt,line join=round,line cap=round,fill=fillColor] (184.94,208.60) circle (  1.16);

\path[draw=drawColor,line width= 0.4pt,line join=round,line cap=round,fill=fillColor] (185.37,208.50) circle (  1.16);

\path[draw=drawColor,line width= 0.4pt,line join=round,line cap=round,fill=fillColor] (185.80,208.49) circle (  1.16);

\path[draw=drawColor,line width= 0.4pt,line join=round,line cap=round,fill=fillColor] (186.22,208.25) circle (  1.16);

\path[draw=drawColor,line width= 0.4pt,line join=round,line cap=round,fill=fillColor] (186.64,208.13) circle (  1.16);

\path[draw=drawColor,line width= 0.4pt,line join=round,line cap=round,fill=fillColor] (187.06,208.02) circle (  1.16);

\path[draw=drawColor,line width= 0.4pt,line join=round,line cap=round,fill=fillColor] (187.48,207.70) circle (  1.16);

\path[draw=drawColor,line width= 0.4pt,line join=round,line cap=round,fill=fillColor] (187.89,207.69) circle (  1.16);

\path[draw=drawColor,line width= 0.4pt,line join=round,line cap=round,fill=fillColor] (188.31,207.67) circle (  1.16);

\path[draw=drawColor,line width= 0.4pt,line join=round,line cap=round,fill=fillColor] (188.72,207.53) circle (  1.16);

\path[draw=drawColor,line width= 0.4pt,line join=round,line cap=round,fill=fillColor] (189.12,207.15) circle (  1.16);

\path[draw=drawColor,line width= 0.4pt,line join=round,line cap=round,fill=fillColor] (189.53,207.07) circle (  1.16);

\path[draw=drawColor,line width= 0.4pt,line join=round,line cap=round,fill=fillColor] (189.93,206.94) circle (  1.16);

\path[draw=drawColor,line width= 0.4pt,line join=round,line cap=round,fill=fillColor] (190.33,206.43) circle (  1.16);

\path[draw=drawColor,line width= 0.4pt,line join=round,line cap=round,fill=fillColor] (190.73,206.12) circle (  1.16);

\path[draw=drawColor,line width= 0.4pt,line join=round,line cap=round,fill=fillColor] (191.12,205.96) circle (  1.16);

\path[draw=drawColor,line width= 0.4pt,line join=round,line cap=round,fill=fillColor] (191.52,205.45) circle (  1.16);

\path[draw=drawColor,line width= 0.4pt,line join=round,line cap=round,fill=fillColor] (191.91,205.10) circle (  1.16);

\path[draw=drawColor,line width= 0.4pt,line join=round,line cap=round,fill=fillColor] (192.30,204.37) circle (  1.16);

\path[draw=drawColor,line width= 0.4pt,line join=round,line cap=round,fill=fillColor] (192.68,204.31) circle (  1.16);

\path[draw=drawColor,line width= 0.4pt,line join=round,line cap=round,fill=fillColor] (193.07,204.30) circle (  1.16);

\path[draw=drawColor,line width= 0.4pt,line join=round,line cap=round,fill=fillColor] (193.45,204.29) circle (  1.16);

\path[draw=drawColor,line width= 0.4pt,line join=round,line cap=round,fill=fillColor] (193.83,204.27) circle (  1.16);

\path[draw=drawColor,line width= 0.4pt,line join=round,line cap=round,fill=fillColor] (194.21,204.24) circle (  1.16);

\path[draw=drawColor,line width= 0.4pt,line join=round,line cap=round,fill=fillColor] (194.59,203.80) circle (  1.16);

\path[draw=drawColor,line width= 0.4pt,line join=round,line cap=round,fill=fillColor] (194.96,203.62) circle (  1.16);

\path[draw=drawColor,line width= 0.4pt,line join=round,line cap=round,fill=fillColor] (195.33,202.62) circle (  1.16);

\path[draw=drawColor,line width= 0.4pt,line join=round,line cap=round,fill=fillColor] (195.70,202.59) circle (  1.16);

\path[draw=drawColor,line width= 0.4pt,line join=round,line cap=round,fill=fillColor] (196.07,198.35) circle (  1.16);

\path[draw=drawColor,line width= 0.4pt,line join=round,line cap=round,fill=fillColor] (196.44,198.35) circle (  1.16);

\path[draw=drawColor,line width= 0.4pt,line join=round,line cap=round,fill=fillColor] (196.80,198.35) circle (  1.16);

\path[draw=drawColor,line width= 0.4pt,line join=round,line cap=round,fill=fillColor] (197.17,198.35) circle (  1.16);

\path[draw=drawColor,line width= 0.4pt,line join=round,line cap=round,fill=fillColor] (197.53,198.35) circle (  1.16);

\path[draw=drawColor,line width= 0.4pt,line join=round,line cap=round,fill=fillColor] (197.89,198.35) circle (  1.16);

\path[draw=drawColor,line width= 0.4pt,line join=round,line cap=round,fill=fillColor] (198.25,198.35) circle (  1.16);

\path[draw=drawColor,line width= 0.4pt,line join=round,line cap=round,fill=fillColor] (198.60,198.35) circle (  1.16);

\path[draw=drawColor,line width= 0.4pt,line join=round,line cap=round,fill=fillColor] (198.96,198.35) circle (  1.16);

\path[draw=drawColor,line width= 0.4pt,line join=round,line cap=round,fill=fillColor] (199.31,198.35) circle (  1.16);

\path[draw=drawColor,line width= 0.4pt,line join=round,line cap=round,fill=fillColor] (199.66,198.35) circle (  1.16);

\path[draw=drawColor,line width= 0.4pt,line join=round,line cap=round,fill=fillColor] (200.01,198.35) circle (  1.16);

\path[draw=drawColor,line width= 0.4pt,line join=round,line cap=round,fill=fillColor] (200.36,198.35) circle (  1.16);

\path[draw=drawColor,line width= 0.4pt,line join=round,line cap=round,fill=fillColor] (200.71,198.35) circle (  1.16);

\path[draw=drawColor,line width= 0.4pt,line join=round,line cap=round,fill=fillColor] (201.05,198.35) circle (  1.16);

\path[draw=drawColor,line width= 0.4pt,line join=round,line cap=round,fill=fillColor] (201.40,198.35) circle (  1.16);

\path[draw=drawColor,line width= 0.4pt,line join=round,line cap=round,fill=fillColor] (201.74,198.35) circle (  1.16);

\path[draw=drawColor,line width= 0.4pt,line join=round,line cap=round,fill=fillColor] (202.08,198.35) circle (  1.16);

\path[draw=drawColor,line width= 0.4pt,line join=round,line cap=round,fill=fillColor] (202.42,198.35) circle (  1.16);

\path[draw=drawColor,line width= 0.4pt,line join=round,line cap=round,fill=fillColor] (202.75,198.35) circle (  1.16);

\path[draw=drawColor,line width= 0.4pt,line join=round,line cap=round,fill=fillColor] (203.09,198.35) circle (  1.16);

\path[draw=drawColor,line width= 0.4pt,line join=round,line cap=round,fill=fillColor] (203.42,198.35) circle (  1.16);

\path[draw=drawColor,line width= 0.4pt,line join=round,line cap=round,fill=fillColor] (203.76,198.35) circle (  1.16);

\path[draw=drawColor,line width= 0.4pt,line join=round,line cap=round,fill=fillColor] (204.09,198.35) circle (  1.16);

\path[draw=drawColor,line width= 0.4pt,line join=round,line cap=round,fill=fillColor] (204.42,198.35) circle (  1.16);

\path[draw=drawColor,line width= 0.4pt,line join=round,line cap=round,fill=fillColor] (204.75,198.35) circle (  1.16);

\path[draw=drawColor,line width= 0.4pt,line join=round,line cap=round,fill=fillColor] (205.07,198.35) circle (  1.16);

\path[draw=drawColor,line width= 0.4pt,line join=round,line cap=round,fill=fillColor] (205.40,198.35) circle (  1.16);

\path[draw=drawColor,line width= 0.4pt,line join=round,line cap=round,fill=fillColor] (205.72,198.35) circle (  1.16);

\path[draw=drawColor,line width= 0.4pt,line join=round,line cap=round,fill=fillColor] (206.05,198.35) circle (  1.16);

\path[draw=drawColor,line width= 0.4pt,line join=round,line cap=round,fill=fillColor] (206.37,198.35) circle (  1.16);

\path[draw=drawColor,line width= 0.4pt,line join=round,line cap=round,fill=fillColor] (206.69,198.35) circle (  1.16);

\path[draw=drawColor,line width= 0.4pt,line join=round,line cap=round,fill=fillColor] (207.01,198.35) circle (  1.16);

\path[draw=drawColor,line width= 0.4pt,line join=round,line cap=round,fill=fillColor] (207.32,198.35) circle (  1.16);

\path[draw=drawColor,line width= 0.4pt,line join=round,line cap=round,fill=fillColor] (207.64,198.35) circle (  1.16);

\path[draw=drawColor,line width= 0.4pt,line join=round,line cap=round,fill=fillColor] (207.96,198.35) circle (  1.16);

\path[draw=drawColor,line width= 0.4pt,line join=round,line cap=round,fill=fillColor] (208.27,198.35) circle (  1.16);

\path[draw=drawColor,line width= 0.4pt,line join=round,line cap=round,fill=fillColor] (208.58,198.35) circle (  1.16);

\path[draw=drawColor,line width= 0.4pt,line join=round,line cap=round,fill=fillColor] (208.89,198.35) circle (  1.16);

\path[draw=drawColor,line width= 0.4pt,line join=round,line cap=round,fill=fillColor] (209.20,198.35) circle (  1.16);

\path[draw=drawColor,line width= 0.4pt,line join=round,line cap=round,fill=fillColor] (209.51,198.35) circle (  1.16);

\path[draw=drawColor,line width= 0.4pt,line join=round,line cap=round,fill=fillColor] (209.82,198.35) circle (  1.16);

\path[draw=drawColor,line width= 0.4pt,line join=round,line cap=round,fill=fillColor] (210.13,198.35) circle (  1.16);

\path[draw=drawColor,line width= 0.4pt,line join=round,line cap=round,fill=fillColor] (210.43,198.35) circle (  1.16);

\path[draw=drawColor,line width= 0.4pt,line join=round,line cap=round,fill=fillColor] (210.74,198.35) circle (  1.16);

\path[draw=drawColor,line width= 0.4pt,line join=round,line cap=round,fill=fillColor] (211.04,198.35) circle (  1.16);

\path[draw=drawColor,line width= 0.4pt,line join=round,line cap=round,fill=fillColor] (211.34,198.35) circle (  1.16);

\path[draw=drawColor,line width= 0.4pt,line join=round,line cap=round,fill=fillColor] (211.64,198.35) circle (  1.16);

\path[draw=drawColor,line width= 0.4pt,line join=round,line cap=round,fill=fillColor] (211.94,198.35) circle (  1.16);

\path[draw=drawColor,line width= 0.4pt,line join=round,line cap=round,fill=fillColor] (212.24,198.35) circle (  1.16);

\path[draw=drawColor,line width= 0.4pt,line join=round,line cap=round,fill=fillColor] (212.54,198.35) circle (  1.16);

\path[draw=drawColor,line width= 0.4pt,line join=round,line cap=round,fill=fillColor] (212.84,198.35) circle (  1.16);

\path[draw=drawColor,line width= 0.4pt,line join=round,line cap=round,fill=fillColor] (213.13,198.35) circle (  1.16);

\path[draw=drawColor,line width= 0.4pt,line join=round,line cap=round,fill=fillColor] (213.43,198.35) circle (  1.16);

\path[draw=drawColor,line width= 0.4pt,line join=round,line cap=round,fill=fillColor] (213.72,198.35) circle (  1.16);

\path[draw=drawColor,line width= 0.4pt,line join=round,line cap=round,fill=fillColor] (214.01,198.35) circle (  1.16);

\path[draw=drawColor,line width= 0.4pt,line join=round,line cap=round,fill=fillColor] (214.30,198.35) circle (  1.16);

\path[draw=drawColor,line width= 0.4pt,line join=round,line cap=round,fill=fillColor] (214.59,198.35) circle (  1.16);

\path[draw=drawColor,line width= 0.4pt,line join=round,line cap=round,fill=fillColor] (214.88,198.35) circle (  1.16);

\path[draw=drawColor,line width= 0.4pt,line join=round,line cap=round,fill=fillColor] (215.17,198.35) circle (  1.16);

\path[draw=drawColor,line width= 0.4pt,line join=round,line cap=round,fill=fillColor] (215.46,198.35) circle (  1.16);

\path[draw=drawColor,line width= 0.4pt,line join=round,line cap=round,fill=fillColor] (215.74,198.35) circle (  1.16);

\path[draw=drawColor,line width= 0.4pt,line join=round,line cap=round,fill=fillColor] (216.03,198.35) circle (  1.16);

\path[draw=drawColor,line width= 0.4pt,line join=round,line cap=round,fill=fillColor] (216.31,198.35) circle (  1.16);

\path[draw=drawColor,line width= 0.4pt,line join=round,line cap=round,fill=fillColor] (216.60,198.35) circle (  1.16);

\path[draw=drawColor,line width= 0.4pt,line join=round,line cap=round,fill=fillColor] (216.88,198.35) circle (  1.16);

\path[draw=drawColor,line width= 0.4pt,line join=round,line cap=round,fill=fillColor] (217.16,198.35) circle (  1.16);

\path[draw=drawColor,line width= 0.4pt,line join=round,line cap=round,fill=fillColor] (217.44,198.35) circle (  1.16);

\path[draw=drawColor,line width= 0.4pt,line join=round,line cap=round,fill=fillColor] (217.72,198.35) circle (  1.16);

\path[draw=drawColor,line width= 0.4pt,line join=round,line cap=round,fill=fillColor] (218.00,198.35) circle (  1.16);

\path[draw=drawColor,line width= 0.4pt,line join=round,line cap=round,fill=fillColor] (218.28,198.35) circle (  1.16);

\path[draw=drawColor,line width= 0.4pt,line join=round,line cap=round,fill=fillColor] (218.55,198.35) circle (  1.16);

\path[draw=drawColor,line width= 0.4pt,line join=round,line cap=round,fill=fillColor] (218.83,198.35) circle (  1.16);

\path[draw=drawColor,line width= 0.4pt,line join=round,line cap=round,fill=fillColor] (219.10,198.35) circle (  1.16);
\definecolor[named]{drawColor}{rgb}{0.60,0.31,0.64}
\definecolor[named]{fillColor}{rgb}{0.60,0.31,0.64}

\path[draw=drawColor,line width= 0.4pt,line join=round,line cap=round,fill=fillColor] ( 78.97,225.18) circle (  1.16);

\path[draw=drawColor,line width= 0.4pt,line join=round,line cap=round,fill=fillColor] ( 86.41,223.87) circle (  1.16);

\path[draw=drawColor,line width= 0.4pt,line join=round,line cap=round,fill=fillColor] ( 91.63,221.50) circle (  1.16);

\path[draw=drawColor,line width= 0.4pt,line join=round,line cap=round,fill=fillColor] ( 95.78,221.24) circle (  1.16);

\path[draw=drawColor,line width= 0.4pt,line join=round,line cap=round,fill=fillColor] ( 99.29,220.76) circle (  1.16);

\path[draw=drawColor,line width= 0.4pt,line join=round,line cap=round,fill=fillColor] (102.36,220.26) circle (  1.16);

\path[draw=drawColor,line width= 0.4pt,line join=round,line cap=round,fill=fillColor] (105.10,220.22) circle (  1.16);

\path[draw=drawColor,line width= 0.4pt,line join=round,line cap=round,fill=fillColor] (107.59,220.16) circle (  1.16);

\path[draw=drawColor,line width= 0.4pt,line join=round,line cap=round,fill=fillColor] (109.88,220.12) circle (  1.16);

\path[draw=drawColor,line width= 0.4pt,line join=round,line cap=round,fill=fillColor] (112.01,219.99) circle (  1.16);

\path[draw=drawColor,line width= 0.4pt,line join=round,line cap=round,fill=fillColor] (114.00,219.08) circle (  1.16);

\path[draw=drawColor,line width= 0.4pt,line join=round,line cap=round,fill=fillColor] (115.87,219.04) circle (  1.16);

\path[draw=drawColor,line width= 0.4pt,line join=round,line cap=round,fill=fillColor] (117.65,218.50) circle (  1.16);

\path[draw=drawColor,line width= 0.4pt,line join=round,line cap=round,fill=fillColor] (119.33,218.08) circle (  1.16);

\path[draw=drawColor,line width= 0.4pt,line join=round,line cap=round,fill=fillColor] (120.93,217.85) circle (  1.16);

\path[draw=drawColor,line width= 0.4pt,line join=round,line cap=round,fill=fillColor] (122.47,217.80) circle (  1.16);

\path[draw=drawColor,line width= 0.4pt,line join=round,line cap=round,fill=fillColor] (123.94,217.56) circle (  1.16);

\path[draw=drawColor,line width= 0.4pt,line join=round,line cap=round,fill=fillColor] (125.36,216.91) circle (  1.16);

\path[draw=drawColor,line width= 0.4pt,line join=round,line cap=round,fill=fillColor] (126.72,216.84) circle (  1.16);

\path[draw=drawColor,line width= 0.4pt,line join=round,line cap=round,fill=fillColor] (128.04,216.65) circle (  1.16);

\path[draw=drawColor,line width= 0.4pt,line join=round,line cap=round,fill=fillColor] (129.31,216.40) circle (  1.16);

\path[draw=drawColor,line width= 0.4pt,line join=round,line cap=round,fill=fillColor] (130.54,216.39) circle (  1.16);

\path[draw=drawColor,line width= 0.4pt,line join=round,line cap=round,fill=fillColor] (131.74,216.12) circle (  1.16);

\path[draw=drawColor,line width= 0.4pt,line join=round,line cap=round,fill=fillColor] (132.90,215.85) circle (  1.16);

\path[draw=drawColor,line width= 0.4pt,line join=round,line cap=round,fill=fillColor] (134.04,215.23) circle (  1.16);

\path[draw=drawColor,line width= 0.4pt,line join=round,line cap=round,fill=fillColor] (135.14,215.21) circle (  1.16);

\path[draw=drawColor,line width= 0.4pt,line join=round,line cap=round,fill=fillColor] (136.21,215.17) circle (  1.16);

\path[draw=drawColor,line width= 0.4pt,line join=round,line cap=round,fill=fillColor] (137.26,214.56) circle (  1.16);

\path[draw=drawColor,line width= 0.4pt,line join=round,line cap=round,fill=fillColor] (138.28,214.56) circle (  1.16);

\path[draw=drawColor,line width= 0.4pt,line join=round,line cap=round,fill=fillColor] (139.28,214.31) circle (  1.16);

\path[draw=drawColor,line width= 0.4pt,line join=round,line cap=round,fill=fillColor] (140.26,214.23) circle (  1.16);

\path[draw=drawColor,line width= 0.4pt,line join=round,line cap=round,fill=fillColor] (141.21,214.21) circle (  1.16);

\path[draw=drawColor,line width= 0.4pt,line join=round,line cap=round,fill=fillColor] (142.15,214.10) circle (  1.16);

\path[draw=drawColor,line width= 0.4pt,line join=round,line cap=round,fill=fillColor] (143.07,214.01) circle (  1.16);

\path[draw=drawColor,line width= 0.4pt,line join=round,line cap=round,fill=fillColor] (143.97,213.96) circle (  1.16);

\path[draw=drawColor,line width= 0.4pt,line join=round,line cap=round,fill=fillColor] (144.85,213.89) circle (  1.16);

\path[draw=drawColor,line width= 0.4pt,line join=round,line cap=round,fill=fillColor] (145.72,213.87) circle (  1.16);

\path[draw=drawColor,line width= 0.4pt,line join=round,line cap=round,fill=fillColor] (146.57,213.83) circle (  1.16);

\path[draw=drawColor,line width= 0.4pt,line join=round,line cap=round,fill=fillColor] (147.41,213.68) circle (  1.16);

\path[draw=drawColor,line width= 0.4pt,line join=round,line cap=round,fill=fillColor] (148.23,213.48) circle (  1.16);

\path[draw=drawColor,line width= 0.4pt,line join=round,line cap=round,fill=fillColor] (149.04,213.48) circle (  1.16);

\path[draw=drawColor,line width= 0.4pt,line join=round,line cap=round,fill=fillColor] (149.83,213.42) circle (  1.16);

\path[draw=drawColor,line width= 0.4pt,line join=round,line cap=round,fill=fillColor] (150.62,213.41) circle (  1.16);

\path[draw=drawColor,line width= 0.4pt,line join=round,line cap=round,fill=fillColor] (151.39,213.37) circle (  1.16);

\path[draw=drawColor,line width= 0.4pt,line join=round,line cap=round,fill=fillColor] (152.15,213.30) circle (  1.16);

\path[draw=drawColor,line width= 0.4pt,line join=round,line cap=round,fill=fillColor] (152.90,213.15) circle (  1.16);

\path[draw=drawColor,line width= 0.4pt,line join=round,line cap=round,fill=fillColor] (153.63,213.14) circle (  1.16);

\path[draw=drawColor,line width= 0.4pt,line join=round,line cap=round,fill=fillColor] (154.36,213.08) circle (  1.16);

\path[draw=drawColor,line width= 0.4pt,line join=round,line cap=round,fill=fillColor] (155.08,213.01) circle (  1.16);

\path[draw=drawColor,line width= 0.4pt,line join=round,line cap=round,fill=fillColor] (155.79,212.83) circle (  1.16);

\path[draw=drawColor,line width= 0.4pt,line join=round,line cap=round,fill=fillColor] (156.48,212.82) circle (  1.16);

\path[draw=drawColor,line width= 0.4pt,line join=round,line cap=round,fill=fillColor] (157.17,212.81) circle (  1.16);

\path[draw=drawColor,line width= 0.4pt,line join=round,line cap=round,fill=fillColor] (157.85,212.78) circle (  1.16);

\path[draw=drawColor,line width= 0.4pt,line join=round,line cap=round,fill=fillColor] (158.53,212.70) circle (  1.16);

\path[draw=drawColor,line width= 0.4pt,line join=round,line cap=round,fill=fillColor] (159.19,212.67) circle (  1.16);

\path[draw=drawColor,line width= 0.4pt,line join=round,line cap=round,fill=fillColor] (159.85,212.56) circle (  1.16);

\path[draw=drawColor,line width= 0.4pt,line join=round,line cap=round,fill=fillColor] (160.49,212.52) circle (  1.16);

\path[draw=drawColor,line width= 0.4pt,line join=round,line cap=round,fill=fillColor] (161.13,212.44) circle (  1.16);

\path[draw=drawColor,line width= 0.4pt,line join=round,line cap=round,fill=fillColor] (161.77,212.39) circle (  1.16);

\path[draw=drawColor,line width= 0.4pt,line join=round,line cap=round,fill=fillColor] (162.39,212.36) circle (  1.16);

\path[draw=drawColor,line width= 0.4pt,line join=round,line cap=round,fill=fillColor] (163.01,212.29) circle (  1.16);

\path[draw=drawColor,line width= 0.4pt,line join=round,line cap=round,fill=fillColor] (163.62,212.17) circle (  1.16);

\path[draw=drawColor,line width= 0.4pt,line join=round,line cap=round,fill=fillColor] (164.23,211.85) circle (  1.16);

\path[draw=drawColor,line width= 0.4pt,line join=round,line cap=round,fill=fillColor] (164.83,211.84) circle (  1.16);

\path[draw=drawColor,line width= 0.4pt,line join=round,line cap=round,fill=fillColor] (165.42,211.58) circle (  1.16);

\path[draw=drawColor,line width= 0.4pt,line join=round,line cap=round,fill=fillColor] (166.01,211.38) circle (  1.16);

\path[draw=drawColor,line width= 0.4pt,line join=round,line cap=round,fill=fillColor] (166.59,211.29) circle (  1.16);

\path[draw=drawColor,line width= 0.4pt,line join=round,line cap=round,fill=fillColor] (167.17,211.27) circle (  1.16);

\path[draw=drawColor,line width= 0.4pt,line join=round,line cap=round,fill=fillColor] (167.74,211.12) circle (  1.16);

\path[draw=drawColor,line width= 0.4pt,line join=round,line cap=round,fill=fillColor] (168.30,211.02) circle (  1.16);

\path[draw=drawColor,line width= 0.4pt,line join=round,line cap=round,fill=fillColor] (168.86,210.99) circle (  1.16);

\path[draw=drawColor,line width= 0.4pt,line join=round,line cap=round,fill=fillColor] (169.41,210.94) circle (  1.16);

\path[draw=drawColor,line width= 0.4pt,line join=round,line cap=round,fill=fillColor] (169.96,210.83) circle (  1.16);

\path[draw=drawColor,line width= 0.4pt,line join=round,line cap=round,fill=fillColor] (170.51,210.49) circle (  1.16);

\path[draw=drawColor,line width= 0.4pt,line join=round,line cap=round,fill=fillColor] (171.04,210.31) circle (  1.16);

\path[draw=drawColor,line width= 0.4pt,line join=round,line cap=round,fill=fillColor] (171.58,210.16) circle (  1.16);

\path[draw=drawColor,line width= 0.4pt,line join=round,line cap=round,fill=fillColor] (172.11,210.09) circle (  1.16);

\path[draw=drawColor,line width= 0.4pt,line join=round,line cap=round,fill=fillColor] (172.63,209.98) circle (  1.16);

\path[draw=drawColor,line width= 0.4pt,line join=round,line cap=round,fill=fillColor] (173.15,209.91) circle (  1.16);

\path[draw=drawColor,line width= 0.4pt,line join=round,line cap=round,fill=fillColor] (173.67,209.58) circle (  1.16);

\path[draw=drawColor,line width= 0.4pt,line join=round,line cap=round,fill=fillColor] (174.18,209.27) circle (  1.16);

\path[draw=drawColor,line width= 0.4pt,line join=round,line cap=round,fill=fillColor] (174.69,209.23) circle (  1.16);

\path[draw=drawColor,line width= 0.4pt,line join=round,line cap=round,fill=fillColor] (175.19,209.20) circle (  1.16);

\path[draw=drawColor,line width= 0.4pt,line join=round,line cap=round,fill=fillColor] (175.69,209.16) circle (  1.16);

\path[draw=drawColor,line width= 0.4pt,line join=round,line cap=round,fill=fillColor] (176.19,209.12) circle (  1.16);

\path[draw=drawColor,line width= 0.4pt,line join=round,line cap=round,fill=fillColor] (176.68,209.06) circle (  1.16);

\path[draw=drawColor,line width= 0.4pt,line join=round,line cap=round,fill=fillColor] (177.17,208.92) circle (  1.16);

\path[draw=drawColor,line width= 0.4pt,line join=round,line cap=round,fill=fillColor] (177.65,208.88) circle (  1.16);

\path[draw=drawColor,line width= 0.4pt,line join=round,line cap=round,fill=fillColor] (178.13,208.83) circle (  1.16);

\path[draw=drawColor,line width= 0.4pt,line join=round,line cap=round,fill=fillColor] (178.61,208.82) circle (  1.16);

\path[draw=drawColor,line width= 0.4pt,line join=round,line cap=round,fill=fillColor] (179.08,208.59) circle (  1.16);

\path[draw=drawColor,line width= 0.4pt,line join=round,line cap=round,fill=fillColor] (179.55,208.48) circle (  1.16);

\path[draw=drawColor,line width= 0.4pt,line join=round,line cap=round,fill=fillColor] (180.02,208.36) circle (  1.16);

\path[draw=drawColor,line width= 0.4pt,line join=round,line cap=round,fill=fillColor] (180.48,208.14) circle (  1.16);

\path[draw=drawColor,line width= 0.4pt,line join=round,line cap=round,fill=fillColor] (180.94,208.03) circle (  1.16);

\path[draw=drawColor,line width= 0.4pt,line join=round,line cap=round,fill=fillColor] (181.40,207.83) circle (  1.16);

\path[draw=drawColor,line width= 0.4pt,line join=round,line cap=round,fill=fillColor] (181.85,207.74) circle (  1.16);

\path[draw=drawColor,line width= 0.4pt,line join=round,line cap=round,fill=fillColor] (182.30,207.62) circle (  1.16);

\path[draw=drawColor,line width= 0.4pt,line join=round,line cap=round,fill=fillColor] (182.75,207.27) circle (  1.16);

\path[draw=drawColor,line width= 0.4pt,line join=round,line cap=round,fill=fillColor] (183.19,207.19) circle (  1.16);

\path[draw=drawColor,line width= 0.4pt,line join=round,line cap=round,fill=fillColor] (183.63,207.13) circle (  1.16);

\path[draw=drawColor,line width= 0.4pt,line join=round,line cap=round,fill=fillColor] (184.07,207.00) circle (  1.16);

\path[draw=drawColor,line width= 0.4pt,line join=round,line cap=round,fill=fillColor] (184.51,206.35) circle (  1.16);

\path[draw=drawColor,line width= 0.4pt,line join=round,line cap=round,fill=fillColor] (184.94,205.95) circle (  1.16);

\path[draw=drawColor,line width= 0.4pt,line join=round,line cap=round,fill=fillColor] (185.37,205.85) circle (  1.16);

\path[draw=drawColor,line width= 0.4pt,line join=round,line cap=round,fill=fillColor] (185.80,205.71) circle (  1.16);

\path[draw=drawColor,line width= 0.4pt,line join=round,line cap=round,fill=fillColor] (186.22,205.56) circle (  1.16);

\path[draw=drawColor,line width= 0.4pt,line join=round,line cap=round,fill=fillColor] (186.64,205.56) circle (  1.16);

\path[draw=drawColor,line width= 0.4pt,line join=round,line cap=round,fill=fillColor] (187.06,205.51) circle (  1.16);

\path[draw=drawColor,line width= 0.4pt,line join=round,line cap=round,fill=fillColor] (187.48,205.35) circle (  1.16);

\path[draw=drawColor,line width= 0.4pt,line join=round,line cap=round,fill=fillColor] (187.89,205.29) circle (  1.16);

\path[draw=drawColor,line width= 0.4pt,line join=round,line cap=round,fill=fillColor] (188.31,205.28) circle (  1.16);

\path[draw=drawColor,line width= 0.4pt,line join=round,line cap=round,fill=fillColor] (188.72,205.11) circle (  1.16);

\path[draw=drawColor,line width= 0.4pt,line join=round,line cap=round,fill=fillColor] (189.12,205.04) circle (  1.16);

\path[draw=drawColor,line width= 0.4pt,line join=round,line cap=round,fill=fillColor] (189.53,204.81) circle (  1.16);

\path[draw=drawColor,line width= 0.4pt,line join=round,line cap=round,fill=fillColor] (189.93,204.59) circle (  1.16);

\path[draw=drawColor,line width= 0.4pt,line join=round,line cap=round,fill=fillColor] (190.33,204.52) circle (  1.16);

\path[draw=drawColor,line width= 0.4pt,line join=round,line cap=round,fill=fillColor] (190.73,204.24) circle (  1.16);

\path[draw=drawColor,line width= 0.4pt,line join=round,line cap=round,fill=fillColor] (191.12,202.65) circle (  1.16);

\path[draw=drawColor,line width= 0.4pt,line join=round,line cap=round,fill=fillColor] (191.52,198.35) circle (  1.16);

\path[draw=drawColor,line width= 0.4pt,line join=round,line cap=round,fill=fillColor] (191.91,198.35) circle (  1.16);

\path[draw=drawColor,line width= 0.4pt,line join=round,line cap=round,fill=fillColor] (192.30,198.35) circle (  1.16);

\path[draw=drawColor,line width= 0.4pt,line join=round,line cap=round,fill=fillColor] (192.68,198.35) circle (  1.16);

\path[draw=drawColor,line width= 0.4pt,line join=round,line cap=round,fill=fillColor] (193.07,198.35) circle (  1.16);

\path[draw=drawColor,line width= 0.4pt,line join=round,line cap=round,fill=fillColor] (193.45,198.35) circle (  1.16);

\path[draw=drawColor,line width= 0.4pt,line join=round,line cap=round,fill=fillColor] (193.83,198.35) circle (  1.16);

\path[draw=drawColor,line width= 0.4pt,line join=round,line cap=round,fill=fillColor] (194.21,198.35) circle (  1.16);

\path[draw=drawColor,line width= 0.4pt,line join=round,line cap=round,fill=fillColor] (194.59,198.35) circle (  1.16);

\path[draw=drawColor,line width= 0.4pt,line join=round,line cap=round,fill=fillColor] (194.96,198.35) circle (  1.16);

\path[draw=drawColor,line width= 0.4pt,line join=round,line cap=round,fill=fillColor] (195.33,198.35) circle (  1.16);

\path[draw=drawColor,line width= 0.4pt,line join=round,line cap=round,fill=fillColor] (195.70,198.35) circle (  1.16);

\path[draw=drawColor,line width= 0.4pt,line join=round,line cap=round,fill=fillColor] (196.07,198.35) circle (  1.16);

\path[draw=drawColor,line width= 0.4pt,line join=round,line cap=round,fill=fillColor] (196.44,198.35) circle (  1.16);

\path[draw=drawColor,line width= 0.4pt,line join=round,line cap=round,fill=fillColor] (196.80,198.35) circle (  1.16);

\path[draw=drawColor,line width= 0.4pt,line join=round,line cap=round,fill=fillColor] (197.17,198.35) circle (  1.16);

\path[draw=drawColor,line width= 0.4pt,line join=round,line cap=round,fill=fillColor] (197.53,198.35) circle (  1.16);

\path[draw=drawColor,line width= 0.4pt,line join=round,line cap=round,fill=fillColor] (197.89,198.35) circle (  1.16);

\path[draw=drawColor,line width= 0.4pt,line join=round,line cap=round,fill=fillColor] (198.25,198.35) circle (  1.16);

\path[draw=drawColor,line width= 0.4pt,line join=round,line cap=round,fill=fillColor] (198.60,198.35) circle (  1.16);

\path[draw=drawColor,line width= 0.4pt,line join=round,line cap=round,fill=fillColor] (198.96,198.35) circle (  1.16);

\path[draw=drawColor,line width= 0.4pt,line join=round,line cap=round,fill=fillColor] (199.31,198.35) circle (  1.16);

\path[draw=drawColor,line width= 0.4pt,line join=round,line cap=round,fill=fillColor] (199.66,198.35) circle (  1.16);

\path[draw=drawColor,line width= 0.4pt,line join=round,line cap=round,fill=fillColor] (200.01,198.35) circle (  1.16);

\path[draw=drawColor,line width= 0.4pt,line join=round,line cap=round,fill=fillColor] (200.36,198.35) circle (  1.16);

\path[draw=drawColor,line width= 0.4pt,line join=round,line cap=round,fill=fillColor] (200.71,198.35) circle (  1.16);

\path[draw=drawColor,line width= 0.4pt,line join=round,line cap=round,fill=fillColor] (201.05,198.35) circle (  1.16);

\path[draw=drawColor,line width= 0.4pt,line join=round,line cap=round,fill=fillColor] (201.40,198.35) circle (  1.16);

\path[draw=drawColor,line width= 0.4pt,line join=round,line cap=round,fill=fillColor] (201.74,198.35) circle (  1.16);

\path[draw=drawColor,line width= 0.4pt,line join=round,line cap=round,fill=fillColor] (202.08,198.35) circle (  1.16);

\path[draw=drawColor,line width= 0.4pt,line join=round,line cap=round,fill=fillColor] (202.42,198.35) circle (  1.16);

\path[draw=drawColor,line width= 0.4pt,line join=round,line cap=round,fill=fillColor] (202.75,198.35) circle (  1.16);

\path[draw=drawColor,line width= 0.4pt,line join=round,line cap=round,fill=fillColor] (203.09,198.35) circle (  1.16);

\path[draw=drawColor,line width= 0.4pt,line join=round,line cap=round,fill=fillColor] (203.42,198.35) circle (  1.16);

\path[draw=drawColor,line width= 0.4pt,line join=round,line cap=round,fill=fillColor] (203.76,198.35) circle (  1.16);

\path[draw=drawColor,line width= 0.4pt,line join=round,line cap=round,fill=fillColor] (204.09,198.35) circle (  1.16);

\path[draw=drawColor,line width= 0.4pt,line join=round,line cap=round,fill=fillColor] (204.42,198.35) circle (  1.16);

\path[draw=drawColor,line width= 0.4pt,line join=round,line cap=round,fill=fillColor] (204.75,198.35) circle (  1.16);

\path[draw=drawColor,line width= 0.4pt,line join=round,line cap=round,fill=fillColor] (205.07,198.35) circle (  1.16);

\path[draw=drawColor,line width= 0.4pt,line join=round,line cap=round,fill=fillColor] (205.40,198.35) circle (  1.16);

\path[draw=drawColor,line width= 0.4pt,line join=round,line cap=round,fill=fillColor] (205.72,198.35) circle (  1.16);

\path[draw=drawColor,line width= 0.4pt,line join=round,line cap=round,fill=fillColor] (206.05,198.35) circle (  1.16);

\path[draw=drawColor,line width= 0.4pt,line join=round,line cap=round,fill=fillColor] (206.37,198.35) circle (  1.16);

\path[draw=drawColor,line width= 0.4pt,line join=round,line cap=round,fill=fillColor] (206.69,198.35) circle (  1.16);

\path[draw=drawColor,line width= 0.4pt,line join=round,line cap=round,fill=fillColor] (207.01,198.35) circle (  1.16);

\path[draw=drawColor,line width= 0.4pt,line join=round,line cap=round,fill=fillColor] (207.32,198.35) circle (  1.16);

\path[draw=drawColor,line width= 0.4pt,line join=round,line cap=round,fill=fillColor] (207.64,198.35) circle (  1.16);

\path[draw=drawColor,line width= 0.4pt,line join=round,line cap=round,fill=fillColor] (207.96,198.35) circle (  1.16);

\path[draw=drawColor,line width= 0.4pt,line join=round,line cap=round,fill=fillColor] (208.27,198.35) circle (  1.16);

\path[draw=drawColor,line width= 0.4pt,line join=round,line cap=round,fill=fillColor] (208.58,198.35) circle (  1.16);

\path[draw=drawColor,line width= 0.4pt,line join=round,line cap=round,fill=fillColor] (208.89,198.35) circle (  1.16);

\path[draw=drawColor,line width= 0.4pt,line join=round,line cap=round,fill=fillColor] (209.20,198.35) circle (  1.16);

\path[draw=drawColor,line width= 0.4pt,line join=round,line cap=round,fill=fillColor] (209.51,198.35) circle (  1.16);

\path[draw=drawColor,line width= 0.4pt,line join=round,line cap=round,fill=fillColor] (209.82,198.35) circle (  1.16);

\path[draw=drawColor,line width= 0.4pt,line join=round,line cap=round,fill=fillColor] (210.13,198.35) circle (  1.16);

\path[draw=drawColor,line width= 0.4pt,line join=round,line cap=round,fill=fillColor] (210.43,198.35) circle (  1.16);

\path[draw=drawColor,line width= 0.4pt,line join=round,line cap=round,fill=fillColor] (210.74,198.35) circle (  1.16);

\path[draw=drawColor,line width= 0.4pt,line join=round,line cap=round,fill=fillColor] (211.04,198.35) circle (  1.16);

\path[draw=drawColor,line width= 0.4pt,line join=round,line cap=round,fill=fillColor] (211.34,198.35) circle (  1.16);

\path[draw=drawColor,line width= 0.4pt,line join=round,line cap=round,fill=fillColor] (211.64,198.35) circle (  1.16);

\path[draw=drawColor,line width= 0.4pt,line join=round,line cap=round,fill=fillColor] (211.94,198.35) circle (  1.16);

\path[draw=drawColor,line width= 0.4pt,line join=round,line cap=round,fill=fillColor] (212.24,198.35) circle (  1.16);

\path[draw=drawColor,line width= 0.4pt,line join=round,line cap=round,fill=fillColor] (212.54,198.35) circle (  1.16);

\path[draw=drawColor,line width= 0.4pt,line join=round,line cap=round,fill=fillColor] (212.84,198.35) circle (  1.16);

\path[draw=drawColor,line width= 0.4pt,line join=round,line cap=round,fill=fillColor] (213.13,198.35) circle (  1.16);

\path[draw=drawColor,line width= 0.4pt,line join=round,line cap=round,fill=fillColor] (213.43,198.35) circle (  1.16);

\path[draw=drawColor,line width= 0.4pt,line join=round,line cap=round,fill=fillColor] (213.72,198.35) circle (  1.16);

\path[draw=drawColor,line width= 0.4pt,line join=round,line cap=round,fill=fillColor] (214.01,198.35) circle (  1.16);

\path[draw=drawColor,line width= 0.4pt,line join=round,line cap=round,fill=fillColor] (214.30,198.35) circle (  1.16);

\path[draw=drawColor,line width= 0.4pt,line join=round,line cap=round,fill=fillColor] (214.59,198.35) circle (  1.16);

\path[draw=drawColor,line width= 0.4pt,line join=round,line cap=round,fill=fillColor] (214.88,198.35) circle (  1.16);

\path[draw=drawColor,line width= 0.4pt,line join=round,line cap=round,fill=fillColor] (215.17,198.35) circle (  1.16);

\path[draw=drawColor,line width= 0.4pt,line join=round,line cap=round,fill=fillColor] (215.46,198.35) circle (  1.16);

\path[draw=drawColor,line width= 0.4pt,line join=round,line cap=round,fill=fillColor] (215.74,198.35) circle (  1.16);

\path[draw=drawColor,line width= 0.4pt,line join=round,line cap=round,fill=fillColor] (216.03,198.35) circle (  1.16);

\path[draw=drawColor,line width= 0.4pt,line join=round,line cap=round,fill=fillColor] (216.31,198.35) circle (  1.16);

\path[draw=drawColor,line width= 0.4pt,line join=round,line cap=round,fill=fillColor] (216.60,198.35) circle (  1.16);

\path[draw=drawColor,line width= 0.4pt,line join=round,line cap=round,fill=fillColor] (216.88,198.35) circle (  1.16);

\path[draw=drawColor,line width= 0.4pt,line join=round,line cap=round,fill=fillColor] (217.16,198.35) circle (  1.16);

\path[draw=drawColor,line width= 0.4pt,line join=round,line cap=round,fill=fillColor] (217.44,198.35) circle (  1.16);

\path[draw=drawColor,line width= 0.4pt,line join=round,line cap=round,fill=fillColor] (217.72,198.35) circle (  1.16);

\path[draw=drawColor,line width= 0.4pt,line join=round,line cap=round,fill=fillColor] (218.00,198.35) circle (  1.16);

\path[draw=drawColor,line width= 0.4pt,line join=round,line cap=round,fill=fillColor] (218.28,198.35) circle (  1.16);

\path[draw=drawColor,line width= 0.4pt,line join=round,line cap=round,fill=fillColor] (218.55,198.35) circle (  1.16);

\path[draw=drawColor,line width= 0.4pt,line join=round,line cap=round,fill=fillColor] (218.83,198.35) circle (  1.16);

\path[draw=drawColor,line width= 0.4pt,line join=round,line cap=round,fill=fillColor] (219.10,198.35) circle (  1.16);
\definecolor[named]{drawColor}{rgb}{0.00,0.00,0.00}
\definecolor[named]{fillColor}{rgb}{0.00,0.00,0.00}

\path[draw=drawColor,line width= 0.6pt,line join=round,fill=fillColor] ( 71.96,277.74) -- (226.11,277.74);

\node[text=drawColor,anchor=base east,inner sep=0pt, outer sep=0pt, scale=  0.85] at (222.61,276.86) {infeasible solutions};

\path[draw=drawColor,line width= 0.6pt,line join=round,line cap=round] ( 71.96,190.42) rectangle (226.11,285.68);
\end{scope}
\begin{scope}
\path[clip] (  0.00,  0.00) rectangle (505.89,614.29);
\definecolor[named]{drawColor}{rgb}{0.00,0.00,0.00}

\node[text=drawColor,anchor=base east,inner sep=0pt, outer sep=0pt, scale=  0.80] at ( 66.56,195.60) {0.00};

\node[text=drawColor,anchor=base east,inner sep=0pt, outer sep=0pt, scale=  0.80] at ( 66.56,212.70) {0.01};

\node[text=drawColor,anchor=base east,inner sep=0pt, outer sep=0pt, scale=  0.80] at ( 66.56,224.85) {0.05};

\node[text=drawColor,anchor=base east,inner sep=0pt, outer sep=0pt, scale=  0.80] at ( 66.56,242.03) {0.20};

\node[text=drawColor,anchor=base east,inner sep=0pt, outer sep=0pt, scale=  0.80] at ( 66.56,258.61) {0.50};

\node[text=drawColor,anchor=base east,inner sep=0pt, outer sep=0pt, scale=  0.80] at ( 66.56,267.73) {0.75};

\node[text=drawColor,anchor=base east,inner sep=0pt, outer sep=0pt, scale=  0.80] at ( 66.56,274.99) {1.00};
\end{scope}
\begin{scope}
\path[clip] (  0.00,  0.00) rectangle (505.89,614.29);
\definecolor[named]{drawColor}{rgb}{0.00,0.00,0.00}

\path[draw=drawColor,line width= 0.6pt,line join=round] ( 68.96,198.35) --
	( 71.96,198.35);

\path[draw=drawColor,line width= 0.6pt,line join=round] ( 68.96,215.46) --
	( 71.96,215.46);

\path[draw=drawColor,line width= 0.6pt,line join=round] ( 68.96,227.60) --
	( 71.96,227.60);

\path[draw=drawColor,line width= 0.6pt,line join=round] ( 68.96,244.78) --
	( 71.96,244.78);

\path[draw=drawColor,line width= 0.6pt,line join=round] ( 68.96,261.37) --
	( 71.96,261.37);

\path[draw=drawColor,line width= 0.6pt,line join=round] ( 68.96,270.48) --
	( 71.96,270.48);

\path[draw=drawColor,line width= 0.6pt,line join=round] ( 68.96,277.74) --
	( 71.96,277.74);
\end{scope}
\begin{scope}
\path[clip] (  0.00,  0.00) rectangle (505.89,614.29);
\definecolor[named]{drawColor}{rgb}{0.00,0.00,0.00}

\path[draw=drawColor,line width= 0.6pt,line join=round] (155.79,187.42) --
	(155.79,190.42);

\path[draw=drawColor,line width= 0.6pt,line join=round] (183.19,187.42) --
	(183.19,190.42);

\path[draw=drawColor,line width= 0.6pt,line join=round] (202.42,187.42) --
	(202.42,190.42);

\path[draw=drawColor,line width= 0.6pt,line join=round] (217.72,187.42) --
	(217.72,190.42);
\end{scope}
\begin{scope}
\path[clip] (  0.00,  0.00) rectangle (505.89,614.29);
\definecolor[named]{drawColor}{rgb}{0.00,0.00,0.00}

\node[text=drawColor,rotate= 50.00,anchor=base east,inner sep=0pt, outer sep=0pt, scale=  0.80] at (160.01,181.47) {50};

\node[text=drawColor,rotate= 50.00,anchor=base east,inner sep=0pt, outer sep=0pt, scale=  0.80] at (187.41,181.47) {100};

\node[text=drawColor,rotate= 50.00,anchor=base east,inner sep=0pt, outer sep=0pt, scale=  0.80] at (206.64,181.47) {150};

\node[text=drawColor,rotate= 50.00,anchor=base east,inner sep=0pt, outer sep=0pt, scale=  0.80] at (221.94,181.47) {200};
\end{scope}
\begin{scope}
\path[clip] (  0.00,  0.00) rectangle (505.89,614.29);
\definecolor[named]{drawColor}{rgb}{0.00,0.00,0.00}

\node[text=drawColor,anchor=base,inner sep=0pt, outer sep=0pt, scale=  0.80] at (149.04,161.97) {\# Instances};
\end{scope}
\begin{scope}
\path[clip] (  0.00,  0.00) rectangle (505.89,614.29);
\definecolor[named]{drawColor}{rgb}{0.00,0.00,0.00}

\node[text=drawColor,rotate= 90.00,anchor=base,inner sep=0pt, outer sep=0pt, scale=  0.80] at ( 42.35,238.05) {1-(Best/Algorithm)};
\end{scope}
\begin{scope}
\path[clip] (  0.00,  0.00) rectangle (505.89,614.29);
\definecolor[named]{drawColor}{rgb}{0.00,0.00,0.00}

\node[text=drawColor,anchor=base,inner sep=0pt, outer sep=0pt, scale=  1.20] at (149.04,292.88) {\Dual};
\end{scope}
\begin{scope}
\path[clip] (273.78,153.57) rectangle (485.05,307.15);
\definecolor[named]{drawColor}{rgb}{1.00,1.00,1.00}
\definecolor[named]{fillColor}{rgb}{1.00,1.00,1.00}

\path[draw=drawColor,line width= 0.6pt,line join=round,line cap=round,fill=fillColor] (273.78,153.57) rectangle (485.05,307.15);
\end{scope}
\begin{scope}
\path[clip] (324.91,190.42) rectangle (479.05,285.68);
\definecolor[named]{fillColor}{rgb}{1.00,1.00,1.00}

\path[fill=fillColor] (324.91,190.42) rectangle (479.05,285.68);
\definecolor[named]{drawColor}{rgb}{0.98,0.98,0.98}

\path[draw=drawColor,line width= 0.6pt,line join=round] (324.91,206.91) --
	(479.05,206.91);

\path[draw=drawColor,line width= 0.6pt,line join=round] (324.91,221.53) --
	(479.05,221.53);

\path[draw=drawColor,line width= 0.6pt,line join=round] (324.91,236.19) --
	(479.05,236.19);

\path[draw=drawColor,line width= 0.6pt,line join=round] (324.91,253.07) --
	(479.05,253.07);

\path[draw=drawColor,line width= 0.6pt,line join=round] (324.91,265.93) --
	(479.05,265.93);

\path[draw=drawColor,line width= 0.6pt,line join=round] (324.91,274.11) --
	(479.05,274.11);

\path[draw=drawColor,line width= 0.6pt,line join=round] (380.85,190.42) --
	(380.85,285.68);

\path[draw=drawColor,line width= 0.6pt,line join=round] (394.42,190.42) --
	(394.42,285.68);

\path[draw=drawColor,line width= 0.6pt,line join=round] (421.56,190.42) --
	(421.56,285.68);

\path[draw=drawColor,line width= 0.6pt,line join=round] (444.65,190.42) --
	(444.65,285.68);

\path[draw=drawColor,line width= 0.6pt,line join=round] (461.75,190.42) --
	(461.75,285.68);
\definecolor[named]{drawColor}{rgb}{0.75,0.75,0.75}

\path[draw=drawColor,line width= 0.6pt,dash pattern=on 1pt off 3pt ,line join=round] (324.91,198.35) --
	(479.05,198.35);

\path[draw=drawColor,line width= 0.6pt,dash pattern=on 1pt off 3pt ,line join=round] (324.91,215.46) --
	(479.05,215.46);

\path[draw=drawColor,line width= 0.6pt,dash pattern=on 1pt off 3pt ,line join=round] (324.91,227.60) --
	(479.05,227.60);

\path[draw=drawColor,line width= 0.6pt,dash pattern=on 1pt off 3pt ,line join=round] (324.91,244.78) --
	(479.05,244.78);

\path[draw=drawColor,line width= 0.6pt,dash pattern=on 1pt off 3pt ,line join=round] (324.91,261.37) --
	(479.05,261.37);

\path[draw=drawColor,line width= 0.6pt,dash pattern=on 1pt off 3pt ,line join=round] (324.91,270.48) --
	(479.05,270.48);

\path[draw=drawColor,line width= 0.6pt,dash pattern=on 1pt off 3pt ,line join=round] (324.91,277.74) --
	(479.05,277.74);

\path[draw=drawColor,line width= 0.6pt,dash pattern=on 1pt off 3pt ,line join=round] (407.99,190.42) --
	(407.99,285.68);

\path[draw=drawColor,line width= 0.6pt,dash pattern=on 1pt off 3pt ,line join=round] (435.13,190.42) --
	(435.13,285.68);

\path[draw=drawColor,line width= 0.6pt,dash pattern=on 1pt off 3pt ,line join=round] (454.17,190.42) --
	(454.17,285.68);

\path[draw=drawColor,line width= 0.6pt,dash pattern=on 1pt off 3pt ,line join=round] (469.33,190.42) --
	(469.33,285.68);
\definecolor[named]{drawColor}{rgb}{0.89,0.10,0.11}
\definecolor[named]{fillColor}{rgb}{0.89,0.10,0.11}

\path[draw=drawColor,line width= 0.4pt,line join=round,line cap=round,fill=fillColor] (331.92,246.94) circle (  1.16);

\path[draw=drawColor,line width= 0.4pt,line join=round,line cap=round,fill=fillColor] (339.28,246.46) circle (  1.16);

\path[draw=drawColor,line width= 0.4pt,line join=round,line cap=round,fill=fillColor] (344.45,238.22) circle (  1.16);

\path[draw=drawColor,line width= 0.4pt,line join=round,line cap=round,fill=fillColor] (348.57,237.65) circle (  1.16);

\path[draw=drawColor,line width= 0.4pt,line join=round,line cap=round,fill=fillColor] (352.04,235.20) circle (  1.16);

\path[draw=drawColor,line width= 0.4pt,line join=round,line cap=round,fill=fillColor] (355.08,234.61) circle (  1.16);

\path[draw=drawColor,line width= 0.4pt,line join=round,line cap=round,fill=fillColor] (357.79,234.29) circle (  1.16);

\path[draw=drawColor,line width= 0.4pt,line join=round,line cap=round,fill=fillColor] (360.26,232.58) circle (  1.16);

\path[draw=drawColor,line width= 0.4pt,line join=round,line cap=round,fill=fillColor] (362.53,231.85) circle (  1.16);

\path[draw=drawColor,line width= 0.4pt,line join=round,line cap=round,fill=fillColor] (364.64,231.75) circle (  1.16);

\path[draw=drawColor,line width= 0.4pt,line join=round,line cap=round,fill=fillColor] (366.61,231.53) circle (  1.16);

\path[draw=drawColor,line width= 0.4pt,line join=round,line cap=round,fill=fillColor] (368.46,231.23) circle (  1.16);

\path[draw=drawColor,line width= 0.4pt,line join=round,line cap=round,fill=fillColor] (370.22,230.86) circle (  1.16);

\path[draw=drawColor,line width= 0.4pt,line join=round,line cap=round,fill=fillColor] (371.89,230.84) circle (  1.16);

\path[draw=drawColor,line width= 0.4pt,line join=round,line cap=round,fill=fillColor] (373.47,230.71) circle (  1.16);

\path[draw=drawColor,line width= 0.4pt,line join=round,line cap=round,fill=fillColor] (374.99,230.64) circle (  1.16);

\path[draw=drawColor,line width= 0.4pt,line join=round,line cap=round,fill=fillColor] (376.45,230.07) circle (  1.16);

\path[draw=drawColor,line width= 0.4pt,line join=round,line cap=round,fill=fillColor] (377.85,230.05) circle (  1.16);

\path[draw=drawColor,line width= 0.4pt,line join=round,line cap=round,fill=fillColor] (379.21,230.00) circle (  1.16);

\path[draw=drawColor,line width= 0.4pt,line join=round,line cap=round,fill=fillColor] (380.51,229.98) circle (  1.16);

\path[draw=drawColor,line width= 0.4pt,line join=round,line cap=round,fill=fillColor] (381.77,229.78) circle (  1.16);

\path[draw=drawColor,line width= 0.4pt,line join=round,line cap=round,fill=fillColor] (382.99,229.58) circle (  1.16);

\path[draw=drawColor,line width= 0.4pt,line join=round,line cap=round,fill=fillColor] (384.18,229.58) circle (  1.16);

\path[draw=drawColor,line width= 0.4pt,line join=round,line cap=round,fill=fillColor] (385.33,229.47) circle (  1.16);

\path[draw=drawColor,line width= 0.4pt,line join=round,line cap=round,fill=fillColor] (386.45,229.40) circle (  1.16);

\path[draw=drawColor,line width= 0.4pt,line join=round,line cap=round,fill=fillColor] (387.54,229.36) circle (  1.16);

\path[draw=drawColor,line width= 0.4pt,line join=round,line cap=round,fill=fillColor] (388.60,228.81) circle (  1.16);

\path[draw=drawColor,line width= 0.4pt,line join=round,line cap=round,fill=fillColor] (389.64,228.76) circle (  1.16);

\path[draw=drawColor,line width= 0.4pt,line join=round,line cap=round,fill=fillColor] (390.65,228.76) circle (  1.16);

\path[draw=drawColor,line width= 0.4pt,line join=round,line cap=round,fill=fillColor] (391.64,228.75) circle (  1.16);

\path[draw=drawColor,line width= 0.4pt,line join=round,line cap=round,fill=fillColor] (392.61,228.67) circle (  1.16);

\path[draw=drawColor,line width= 0.4pt,line join=round,line cap=round,fill=fillColor] (393.56,228.56) circle (  1.16);

\path[draw=drawColor,line width= 0.4pt,line join=round,line cap=round,fill=fillColor] (394.49,228.54) circle (  1.16);

\path[draw=drawColor,line width= 0.4pt,line join=round,line cap=round,fill=fillColor] (395.40,228.32) circle (  1.16);

\path[draw=drawColor,line width= 0.4pt,line join=round,line cap=round,fill=fillColor] (396.29,228.23) circle (  1.16);

\path[draw=drawColor,line width= 0.4pt,line join=round,line cap=round,fill=fillColor] (397.16,228.05) circle (  1.16);

\path[draw=drawColor,line width= 0.4pt,line join=round,line cap=round,fill=fillColor] (398.02,227.88) circle (  1.16);

\path[draw=drawColor,line width= 0.4pt,line join=round,line cap=round,fill=fillColor] (398.86,227.58) circle (  1.16);

\path[draw=drawColor,line width= 0.4pt,line join=round,line cap=round,fill=fillColor] (399.69,227.51) circle (  1.16);

\path[draw=drawColor,line width= 0.4pt,line join=round,line cap=round,fill=fillColor] (400.51,227.47) circle (  1.16);

\path[draw=drawColor,line width= 0.4pt,line join=round,line cap=round,fill=fillColor] (401.31,227.35) circle (  1.16);

\path[draw=drawColor,line width= 0.4pt,line join=round,line cap=round,fill=fillColor] (402.10,227.30) circle (  1.16);

\path[draw=drawColor,line width= 0.4pt,line join=round,line cap=round,fill=fillColor] (402.87,227.27) circle (  1.16);

\path[draw=drawColor,line width= 0.4pt,line join=round,line cap=round,fill=fillColor] (403.64,227.13) circle (  1.16);

\path[draw=drawColor,line width= 0.4pt,line join=round,line cap=round,fill=fillColor] (404.39,227.06) circle (  1.16);

\path[draw=drawColor,line width= 0.4pt,line join=round,line cap=round,fill=fillColor] (405.13,226.94) circle (  1.16);

\path[draw=drawColor,line width= 0.4pt,line join=round,line cap=round,fill=fillColor] (405.86,226.89) circle (  1.16);

\path[draw=drawColor,line width= 0.4pt,line join=round,line cap=round,fill=fillColor] (406.58,226.89) circle (  1.16);

\path[draw=drawColor,line width= 0.4pt,line join=round,line cap=round,fill=fillColor] (407.29,226.74) circle (  1.16);

\path[draw=drawColor,line width= 0.4pt,line join=round,line cap=round,fill=fillColor] (407.99,226.67) circle (  1.16);

\path[draw=drawColor,line width= 0.4pt,line join=round,line cap=round,fill=fillColor] (408.68,226.60) circle (  1.16);

\path[draw=drawColor,line width= 0.4pt,line join=round,line cap=round,fill=fillColor] (409.37,226.17) circle (  1.16);

\path[draw=drawColor,line width= 0.4pt,line join=round,line cap=round,fill=fillColor] (410.04,226.14) circle (  1.16);

\path[draw=drawColor,line width= 0.4pt,line join=round,line cap=round,fill=fillColor] (410.71,226.11) circle (  1.16);

\path[draw=drawColor,line width= 0.4pt,line join=round,line cap=round,fill=fillColor] (411.36,226.02) circle (  1.16);

\path[draw=drawColor,line width= 0.4pt,line join=round,line cap=round,fill=fillColor] (412.01,225.96) circle (  1.16);

\path[draw=drawColor,line width= 0.4pt,line join=round,line cap=round,fill=fillColor] (412.65,225.80) circle (  1.16);

\path[draw=drawColor,line width= 0.4pt,line join=round,line cap=round,fill=fillColor] (413.29,225.69) circle (  1.16);

\path[draw=drawColor,line width= 0.4pt,line join=round,line cap=round,fill=fillColor] (413.92,225.57) circle (  1.16);

\path[draw=drawColor,line width= 0.4pt,line join=round,line cap=round,fill=fillColor] (414.54,225.56) circle (  1.16);

\path[draw=drawColor,line width= 0.4pt,line join=round,line cap=round,fill=fillColor] (415.15,225.54) circle (  1.16);

\path[draw=drawColor,line width= 0.4pt,line join=round,line cap=round,fill=fillColor] (415.76,225.51) circle (  1.16);

\path[draw=drawColor,line width= 0.4pt,line join=round,line cap=round,fill=fillColor] (416.36,225.44) circle (  1.16);

\path[draw=drawColor,line width= 0.4pt,line join=round,line cap=round,fill=fillColor] (416.95,225.27) circle (  1.16);

\path[draw=drawColor,line width= 0.4pt,line join=round,line cap=round,fill=fillColor] (417.54,225.16) circle (  1.16);

\path[draw=drawColor,line width= 0.4pt,line join=round,line cap=round,fill=fillColor] (418.12,225.09) circle (  1.16);

\path[draw=drawColor,line width= 0.4pt,line join=round,line cap=round,fill=fillColor] (418.69,225.04) circle (  1.16);

\path[draw=drawColor,line width= 0.4pt,line join=round,line cap=round,fill=fillColor] (419.26,224.85) circle (  1.16);

\path[draw=drawColor,line width= 0.4pt,line join=round,line cap=round,fill=fillColor] (419.83,224.75) circle (  1.16);

\path[draw=drawColor,line width= 0.4pt,line join=round,line cap=round,fill=fillColor] (420.39,224.69) circle (  1.16);

\path[draw=drawColor,line width= 0.4pt,line join=round,line cap=round,fill=fillColor] (420.94,224.67) circle (  1.16);

\path[draw=drawColor,line width= 0.4pt,line join=round,line cap=round,fill=fillColor] (421.49,224.59) circle (  1.16);

\path[draw=drawColor,line width= 0.4pt,line join=round,line cap=round,fill=fillColor] (422.03,224.00) circle (  1.16);

\path[draw=drawColor,line width= 0.4pt,line join=round,line cap=round,fill=fillColor] (422.57,223.98) circle (  1.16);

\path[draw=drawColor,line width= 0.4pt,line join=round,line cap=round,fill=fillColor] (423.10,223.81) circle (  1.16);

\path[draw=drawColor,line width= 0.4pt,line join=round,line cap=round,fill=fillColor] (423.63,223.47) circle (  1.16);

\path[draw=drawColor,line width= 0.4pt,line join=round,line cap=round,fill=fillColor] (424.16,223.35) circle (  1.16);

\path[draw=drawColor,line width= 0.4pt,line join=round,line cap=round,fill=fillColor] (424.68,223.04) circle (  1.16);

\path[draw=drawColor,line width= 0.4pt,line join=round,line cap=round,fill=fillColor] (425.19,223.04) circle (  1.16);

\path[draw=drawColor,line width= 0.4pt,line join=round,line cap=round,fill=fillColor] (425.70,222.81) circle (  1.16);

\path[draw=drawColor,line width= 0.4pt,line join=round,line cap=round,fill=fillColor] (426.21,222.66) circle (  1.16);

\path[draw=drawColor,line width= 0.4pt,line join=round,line cap=round,fill=fillColor] (426.71,222.64) circle (  1.16);

\path[draw=drawColor,line width= 0.4pt,line join=round,line cap=round,fill=fillColor] (427.21,222.57) circle (  1.16);

\path[draw=drawColor,line width= 0.4pt,line join=round,line cap=round,fill=fillColor] (427.71,222.55) circle (  1.16);

\path[draw=drawColor,line width= 0.4pt,line join=round,line cap=round,fill=fillColor] (428.20,222.26) circle (  1.16);

\path[draw=drawColor,line width= 0.4pt,line join=round,line cap=round,fill=fillColor] (428.68,222.18) circle (  1.16);

\path[draw=drawColor,line width= 0.4pt,line join=round,line cap=round,fill=fillColor] (429.17,222.13) circle (  1.16);

\path[draw=drawColor,line width= 0.4pt,line join=round,line cap=round,fill=fillColor] (429.65,222.10) circle (  1.16);

\path[draw=drawColor,line width= 0.4pt,line join=round,line cap=round,fill=fillColor] (430.12,221.70) circle (  1.16);

\path[draw=drawColor,line width= 0.4pt,line join=round,line cap=round,fill=fillColor] (430.59,221.69) circle (  1.16);

\path[draw=drawColor,line width= 0.4pt,line join=round,line cap=round,fill=fillColor] (431.06,221.62) circle (  1.16);

\path[draw=drawColor,line width= 0.4pt,line join=round,line cap=round,fill=fillColor] (431.53,221.61) circle (  1.16);

\path[draw=drawColor,line width= 0.4pt,line join=round,line cap=round,fill=fillColor] (431.99,221.51) circle (  1.16);

\path[draw=drawColor,line width= 0.4pt,line join=round,line cap=round,fill=fillColor] (432.45,221.42) circle (  1.16);

\path[draw=drawColor,line width= 0.4pt,line join=round,line cap=round,fill=fillColor] (432.90,221.32) circle (  1.16);

\path[draw=drawColor,line width= 0.4pt,line join=round,line cap=round,fill=fillColor] (433.36,221.27) circle (  1.16);

\path[draw=drawColor,line width= 0.4pt,line join=round,line cap=round,fill=fillColor] (433.81,221.27) circle (  1.16);

\path[draw=drawColor,line width= 0.4pt,line join=round,line cap=round,fill=fillColor] (434.25,221.04) circle (  1.16);

\path[draw=drawColor,line width= 0.4pt,line join=round,line cap=round,fill=fillColor] (434.69,220.92) circle (  1.16);

\path[draw=drawColor,line width= 0.4pt,line join=round,line cap=round,fill=fillColor] (435.13,220.88) circle (  1.16);

\path[draw=drawColor,line width= 0.4pt,line join=round,line cap=round,fill=fillColor] (435.57,220.73) circle (  1.16);

\path[draw=drawColor,line width= 0.4pt,line join=round,line cap=round,fill=fillColor] (436.01,220.44) circle (  1.16);

\path[draw=drawColor,line width= 0.4pt,line join=round,line cap=round,fill=fillColor] (436.44,220.38) circle (  1.16);

\path[draw=drawColor,line width= 0.4pt,line join=round,line cap=round,fill=fillColor] (436.87,220.37) circle (  1.16);

\path[draw=drawColor,line width= 0.4pt,line join=round,line cap=round,fill=fillColor] (437.29,220.25) circle (  1.16);

\path[draw=drawColor,line width= 0.4pt,line join=round,line cap=round,fill=fillColor] (437.71,220.23) circle (  1.16);

\path[draw=drawColor,line width= 0.4pt,line join=round,line cap=round,fill=fillColor] (438.14,220.20) circle (  1.16);

\path[draw=drawColor,line width= 0.4pt,line join=round,line cap=round,fill=fillColor] (438.55,219.90) circle (  1.16);

\path[draw=drawColor,line width= 0.4pt,line join=round,line cap=round,fill=fillColor] (438.97,219.61) circle (  1.16);

\path[draw=drawColor,line width= 0.4pt,line join=round,line cap=round,fill=fillColor] (439.38,219.38) circle (  1.16);

\path[draw=drawColor,line width= 0.4pt,line join=round,line cap=round,fill=fillColor] (439.79,219.32) circle (  1.16);

\path[draw=drawColor,line width= 0.4pt,line join=round,line cap=round,fill=fillColor] (440.20,219.32) circle (  1.16);

\path[draw=drawColor,line width= 0.4pt,line join=round,line cap=round,fill=fillColor] (440.60,219.24) circle (  1.16);

\path[draw=drawColor,line width= 0.4pt,line join=round,line cap=round,fill=fillColor] (441.01,219.14) circle (  1.16);

\path[draw=drawColor,line width= 0.4pt,line join=round,line cap=round,fill=fillColor] (441.41,219.12) circle (  1.16);

\path[draw=drawColor,line width= 0.4pt,line join=round,line cap=round,fill=fillColor] (441.81,219.02) circle (  1.16);

\path[draw=drawColor,line width= 0.4pt,line join=round,line cap=round,fill=fillColor] (442.20,218.93) circle (  1.16);

\path[draw=drawColor,line width= 0.4pt,line join=round,line cap=round,fill=fillColor] (442.60,218.87) circle (  1.16);

\path[draw=drawColor,line width= 0.4pt,line join=round,line cap=round,fill=fillColor] (442.99,218.64) circle (  1.16);

\path[draw=drawColor,line width= 0.4pt,line join=round,line cap=round,fill=fillColor] (443.38,218.49) circle (  1.16);

\path[draw=drawColor,line width= 0.4pt,line join=round,line cap=round,fill=fillColor] (443.77,218.44) circle (  1.16);

\path[draw=drawColor,line width= 0.4pt,line join=round,line cap=round,fill=fillColor] (444.15,218.30) circle (  1.16);

\path[draw=drawColor,line width= 0.4pt,line join=round,line cap=round,fill=fillColor] (444.53,218.23) circle (  1.16);

\path[draw=drawColor,line width= 0.4pt,line join=round,line cap=round,fill=fillColor] (444.91,218.15) circle (  1.16);

\path[draw=drawColor,line width= 0.4pt,line join=round,line cap=round,fill=fillColor] (445.29,218.11) circle (  1.16);

\path[draw=drawColor,line width= 0.4pt,line join=round,line cap=round,fill=fillColor] (445.67,218.11) circle (  1.16);

\path[draw=drawColor,line width= 0.4pt,line join=round,line cap=round,fill=fillColor] (446.05,218.09) circle (  1.16);

\path[draw=drawColor,line width= 0.4pt,line join=round,line cap=round,fill=fillColor] (446.42,218.06) circle (  1.16);

\path[draw=drawColor,line width= 0.4pt,line join=round,line cap=round,fill=fillColor] (446.79,218.06) circle (  1.16);

\path[draw=drawColor,line width= 0.4pt,line join=round,line cap=round,fill=fillColor] (447.16,217.96) circle (  1.16);

\path[draw=drawColor,line width= 0.4pt,line join=round,line cap=round,fill=fillColor] (447.53,217.93) circle (  1.16);

\path[draw=drawColor,line width= 0.4pt,line join=round,line cap=round,fill=fillColor] (447.89,217.66) circle (  1.16);

\path[draw=drawColor,line width= 0.4pt,line join=round,line cap=round,fill=fillColor] (448.25,217.60) circle (  1.16);

\path[draw=drawColor,line width= 0.4pt,line join=round,line cap=round,fill=fillColor] (448.62,217.49) circle (  1.16);

\path[draw=drawColor,line width= 0.4pt,line join=round,line cap=round,fill=fillColor] (448.98,217.13) circle (  1.16);

\path[draw=drawColor,line width= 0.4pt,line join=round,line cap=round,fill=fillColor] (449.33,217.06) circle (  1.16);

\path[draw=drawColor,line width= 0.4pt,line join=round,line cap=round,fill=fillColor] (449.69,216.78) circle (  1.16);

\path[draw=drawColor,line width= 0.4pt,line join=round,line cap=round,fill=fillColor] (450.05,216.66) circle (  1.16);

\path[draw=drawColor,line width= 0.4pt,line join=round,line cap=round,fill=fillColor] (450.40,216.65) circle (  1.16);

\path[draw=drawColor,line width= 0.4pt,line join=round,line cap=round,fill=fillColor] (450.75,216.63) circle (  1.16);

\path[draw=drawColor,line width= 0.4pt,line join=round,line cap=round,fill=fillColor] (451.10,216.61) circle (  1.16);

\path[draw=drawColor,line width= 0.4pt,line join=round,line cap=round,fill=fillColor] (451.45,216.60) circle (  1.16);

\path[draw=drawColor,line width= 0.4pt,line join=round,line cap=round,fill=fillColor] (451.79,216.50) circle (  1.16);

\path[draw=drawColor,line width= 0.4pt,line join=round,line cap=round,fill=fillColor] (452.14,216.46) circle (  1.16);

\path[draw=drawColor,line width= 0.4pt,line join=round,line cap=round,fill=fillColor] (452.48,216.35) circle (  1.16);

\path[draw=drawColor,line width= 0.4pt,line join=round,line cap=round,fill=fillColor] (452.82,216.29) circle (  1.16);

\path[draw=drawColor,line width= 0.4pt,line join=round,line cap=round,fill=fillColor] (453.16,216.14) circle (  1.16);

\path[draw=drawColor,line width= 0.4pt,line join=round,line cap=round,fill=fillColor] (453.50,216.11) circle (  1.16);

\path[draw=drawColor,line width= 0.4pt,line join=round,line cap=round,fill=fillColor] (453.84,215.84) circle (  1.16);

\path[draw=drawColor,line width= 0.4pt,line join=round,line cap=round,fill=fillColor] (454.17,215.35) circle (  1.16);

\path[draw=drawColor,line width= 0.4pt,line join=round,line cap=round,fill=fillColor] (454.51,215.10) circle (  1.16);

\path[draw=drawColor,line width= 0.4pt,line join=round,line cap=round,fill=fillColor] (454.84,214.92) circle (  1.16);

\path[draw=drawColor,line width= 0.4pt,line join=round,line cap=round,fill=fillColor] (455.17,214.90) circle (  1.16);

\path[draw=drawColor,line width= 0.4pt,line join=round,line cap=round,fill=fillColor] (455.50,214.77) circle (  1.16);

\path[draw=drawColor,line width= 0.4pt,line join=round,line cap=round,fill=fillColor] (455.83,214.54) circle (  1.16);

\path[draw=drawColor,line width= 0.4pt,line join=round,line cap=round,fill=fillColor] (456.16,213.76) circle (  1.16);

\path[draw=drawColor,line width= 0.4pt,line join=round,line cap=round,fill=fillColor] (456.48,213.67) circle (  1.16);

\path[draw=drawColor,line width= 0.4pt,line join=round,line cap=round,fill=fillColor] (456.80,213.67) circle (  1.16);

\path[draw=drawColor,line width= 0.4pt,line join=round,line cap=round,fill=fillColor] (457.13,213.66) circle (  1.16);

\path[draw=drawColor,line width= 0.4pt,line join=round,line cap=round,fill=fillColor] (457.45,213.61) circle (  1.16);

\path[draw=drawColor,line width= 0.4pt,line join=round,line cap=round,fill=fillColor] (457.77,213.58) circle (  1.16);

\path[draw=drawColor,line width= 0.4pt,line join=round,line cap=round,fill=fillColor] (458.09,213.39) circle (  1.16);

\path[draw=drawColor,line width= 0.4pt,line join=round,line cap=round,fill=fillColor] (458.40,213.09) circle (  1.16);

\path[draw=drawColor,line width= 0.4pt,line join=round,line cap=round,fill=fillColor] (458.72,212.69) circle (  1.16);

\path[draw=drawColor,line width= 0.4pt,line join=round,line cap=round,fill=fillColor] (459.03,212.46) circle (  1.16);

\path[draw=drawColor,line width= 0.4pt,line join=round,line cap=round,fill=fillColor] (459.35,212.29) circle (  1.16);

\path[draw=drawColor,line width= 0.4pt,line join=round,line cap=round,fill=fillColor] (459.66,212.10) circle (  1.16);

\path[draw=drawColor,line width= 0.4pt,line join=round,line cap=round,fill=fillColor] (459.97,211.94) circle (  1.16);

\path[draw=drawColor,line width= 0.4pt,line join=round,line cap=round,fill=fillColor] (460.28,211.77) circle (  1.16);

\path[draw=drawColor,line width= 0.4pt,line join=round,line cap=round,fill=fillColor] (460.59,211.05) circle (  1.16);

\path[draw=drawColor,line width= 0.4pt,line join=round,line cap=round,fill=fillColor] (460.90,211.02) circle (  1.16);

\path[draw=drawColor,line width= 0.4pt,line join=round,line cap=round,fill=fillColor] (461.20,210.71) circle (  1.16);

\path[draw=drawColor,line width= 0.4pt,line join=round,line cap=round,fill=fillColor] (461.51,210.44) circle (  1.16);

\path[draw=drawColor,line width= 0.4pt,line join=round,line cap=round,fill=fillColor] (461.81,210.25) circle (  1.16);

\path[draw=drawColor,line width= 0.4pt,line join=round,line cap=round,fill=fillColor] (462.11,209.43) circle (  1.16);

\path[draw=drawColor,line width= 0.4pt,line join=round,line cap=round,fill=fillColor] (462.42,209.20) circle (  1.16);

\path[draw=drawColor,line width= 0.4pt,line join=round,line cap=round,fill=fillColor] (462.72,208.93) circle (  1.16);

\path[draw=drawColor,line width= 0.4pt,line join=round,line cap=round,fill=fillColor] (463.01,207.78) circle (  1.16);

\path[draw=drawColor,line width= 0.4pt,line join=round,line cap=round,fill=fillColor] (463.31,207.61) circle (  1.16);

\path[draw=drawColor,line width= 0.4pt,line join=round,line cap=round,fill=fillColor] (463.61,207.46) circle (  1.16);

\path[draw=drawColor,line width= 0.4pt,line join=round,line cap=round,fill=fillColor] (463.91,207.02) circle (  1.16);

\path[draw=drawColor,line width= 0.4pt,line join=round,line cap=round,fill=fillColor] (464.20,206.22) circle (  1.16);

\path[draw=drawColor,line width= 0.4pt,line join=round,line cap=round,fill=fillColor] (464.49,206.13) circle (  1.16);

\path[draw=drawColor,line width= 0.4pt,line join=round,line cap=round,fill=fillColor] (464.79,205.13) circle (  1.16);

\path[draw=drawColor,line width= 0.4pt,line join=round,line cap=round,fill=fillColor] (465.08,198.35) circle (  1.16);

\path[draw=drawColor,line width= 0.4pt,line join=round,line cap=round,fill=fillColor] (465.37,198.35) circle (  1.16);

\path[draw=drawColor,line width= 0.4pt,line join=round,line cap=round,fill=fillColor] (465.66,198.35) circle (  1.16);

\path[draw=drawColor,line width= 0.4pt,line join=round,line cap=round,fill=fillColor] (465.95,198.35) circle (  1.16);

\path[draw=drawColor,line width= 0.4pt,line join=round,line cap=round,fill=fillColor] (466.23,198.35) circle (  1.16);

\path[draw=drawColor,line width= 0.4pt,line join=round,line cap=round,fill=fillColor] (466.52,198.35) circle (  1.16);

\path[draw=drawColor,line width= 0.4pt,line join=round,line cap=round,fill=fillColor] (466.81,198.35) circle (  1.16);

\path[draw=drawColor,line width= 0.4pt,line join=round,line cap=round,fill=fillColor] (467.09,198.35) circle (  1.16);

\path[draw=drawColor,line width= 0.4pt,line join=round,line cap=round,fill=fillColor] (467.37,198.35) circle (  1.16);

\path[draw=drawColor,line width= 0.4pt,line join=round,line cap=round,fill=fillColor] (467.66,198.35) circle (  1.16);

\path[draw=drawColor,line width= 0.4pt,line join=round,line cap=round,fill=fillColor] (467.94,198.35) circle (  1.16);

\path[draw=drawColor,line width= 0.4pt,line join=round,line cap=round,fill=fillColor] (468.22,198.35) circle (  1.16);

\path[draw=drawColor,line width= 0.4pt,line join=round,line cap=round,fill=fillColor] (468.50,198.35) circle (  1.16);

\path[draw=drawColor,line width= 0.4pt,line join=round,line cap=round,fill=fillColor] (468.78,198.35) circle (  1.16);

\path[draw=drawColor,line width= 0.4pt,line join=round,line cap=round,fill=fillColor] (469.05,198.35) circle (  1.16);

\path[draw=drawColor,line width= 0.4pt,line join=round,line cap=round,fill=fillColor] (469.33,198.35) circle (  1.16);

\path[draw=drawColor,line width= 0.4pt,line join=round,line cap=round,fill=fillColor] (469.61,198.35) circle (  1.16);

\path[draw=drawColor,line width= 0.4pt,line join=round,line cap=round,fill=fillColor] (469.88,198.35) circle (  1.16);

\path[draw=drawColor,line width= 0.4pt,line join=round,line cap=round,fill=fillColor] (470.15,198.35) circle (  1.16);

\path[draw=drawColor,line width= 0.4pt,line join=round,line cap=round,fill=fillColor] (470.43,198.35) circle (  1.16);

\path[draw=drawColor,line width= 0.4pt,line join=round,line cap=round,fill=fillColor] (470.70,198.35) circle (  1.16);

\path[draw=drawColor,line width= 0.4pt,line join=round,line cap=round,fill=fillColor] (470.97,198.35) circle (  1.16);

\path[draw=drawColor,line width= 0.4pt,line join=round,line cap=round,fill=fillColor] (471.24,198.35) circle (  1.16);

\path[draw=drawColor,line width= 0.4pt,line join=round,line cap=round,fill=fillColor] (471.51,198.35) circle (  1.16);

\path[draw=drawColor,line width= 0.4pt,line join=round,line cap=round,fill=fillColor] (471.78,198.35) circle (  1.16);

\path[draw=drawColor,line width= 0.4pt,line join=round,line cap=round,fill=fillColor] (472.05,198.35) circle (  1.16);
\definecolor[named]{drawColor}{rgb}{0.65,0.34,0.16}
\definecolor[named]{fillColor}{rgb}{0.65,0.34,0.16}

\path[draw=drawColor,line width= 0.4pt,line join=round,line cap=round,fill=fillColor] (331.92,225.40) circle (  1.16);

\path[draw=drawColor,line width= 0.4pt,line join=round,line cap=round,fill=fillColor] (339.28,223.54) circle (  1.16);

\path[draw=drawColor,line width= 0.4pt,line join=round,line cap=round,fill=fillColor] (344.45,222.51) circle (  1.16);

\path[draw=drawColor,line width= 0.4pt,line join=round,line cap=round,fill=fillColor] (348.57,221.13) circle (  1.16);

\path[draw=drawColor,line width= 0.4pt,line join=round,line cap=round,fill=fillColor] (352.04,217.39) circle (  1.16);

\path[draw=drawColor,line width= 0.4pt,line join=round,line cap=round,fill=fillColor] (355.08,217.00) circle (  1.16);

\path[draw=drawColor,line width= 0.4pt,line join=round,line cap=round,fill=fillColor] (357.79,216.23) circle (  1.16);

\path[draw=drawColor,line width= 0.4pt,line join=round,line cap=round,fill=fillColor] (360.26,216.19) circle (  1.16);

\path[draw=drawColor,line width= 0.4pt,line join=round,line cap=round,fill=fillColor] (362.53,215.89) circle (  1.16);

\path[draw=drawColor,line width= 0.4pt,line join=round,line cap=round,fill=fillColor] (364.64,215.54) circle (  1.16);

\path[draw=drawColor,line width= 0.4pt,line join=round,line cap=round,fill=fillColor] (366.61,215.37) circle (  1.16);

\path[draw=drawColor,line width= 0.4pt,line join=round,line cap=round,fill=fillColor] (368.46,215.35) circle (  1.16);

\path[draw=drawColor,line width= 0.4pt,line join=round,line cap=round,fill=fillColor] (370.22,215.21) circle (  1.16);

\path[draw=drawColor,line width= 0.4pt,line join=round,line cap=round,fill=fillColor] (371.89,215.19) circle (  1.16);

\path[draw=drawColor,line width= 0.4pt,line join=round,line cap=round,fill=fillColor] (373.47,214.91) circle (  1.16);

\path[draw=drawColor,line width= 0.4pt,line join=round,line cap=round,fill=fillColor] (374.99,214.49) circle (  1.16);

\path[draw=drawColor,line width= 0.4pt,line join=round,line cap=round,fill=fillColor] (376.45,214.48) circle (  1.16);

\path[draw=drawColor,line width= 0.4pt,line join=round,line cap=round,fill=fillColor] (377.85,213.83) circle (  1.16);

\path[draw=drawColor,line width= 0.4pt,line join=round,line cap=round,fill=fillColor] (379.21,213.82) circle (  1.16);

\path[draw=drawColor,line width= 0.4pt,line join=round,line cap=round,fill=fillColor] (380.51,213.79) circle (  1.16);

\path[draw=drawColor,line width= 0.4pt,line join=round,line cap=round,fill=fillColor] (381.77,213.74) circle (  1.16);

\path[draw=drawColor,line width= 0.4pt,line join=round,line cap=round,fill=fillColor] (382.99,213.71) circle (  1.16);

\path[draw=drawColor,line width= 0.4pt,line join=round,line cap=round,fill=fillColor] (384.18,213.63) circle (  1.16);

\path[draw=drawColor,line width= 0.4pt,line join=round,line cap=round,fill=fillColor] (385.33,213.61) circle (  1.16);

\path[draw=drawColor,line width= 0.4pt,line join=round,line cap=round,fill=fillColor] (386.45,213.50) circle (  1.16);

\path[draw=drawColor,line width= 0.4pt,line join=round,line cap=round,fill=fillColor] (387.54,213.47) circle (  1.16);

\path[draw=drawColor,line width= 0.4pt,line join=round,line cap=round,fill=fillColor] (388.60,213.35) circle (  1.16);

\path[draw=drawColor,line width= 0.4pt,line join=round,line cap=round,fill=fillColor] (389.64,212.77) circle (  1.16);

\path[draw=drawColor,line width= 0.4pt,line join=round,line cap=round,fill=fillColor] (390.65,212.74) circle (  1.16);

\path[draw=drawColor,line width= 0.4pt,line join=round,line cap=round,fill=fillColor] (391.64,212.28) circle (  1.16);

\path[draw=drawColor,line width= 0.4pt,line join=round,line cap=round,fill=fillColor] (392.61,212.24) circle (  1.16);

\path[draw=drawColor,line width= 0.4pt,line join=round,line cap=round,fill=fillColor] (393.56,212.20) circle (  1.16);

\path[draw=drawColor,line width= 0.4pt,line join=round,line cap=round,fill=fillColor] (394.49,212.20) circle (  1.16);

\path[draw=drawColor,line width= 0.4pt,line join=round,line cap=round,fill=fillColor] (395.40,211.89) circle (  1.16);

\path[draw=drawColor,line width= 0.4pt,line join=round,line cap=round,fill=fillColor] (396.29,211.73) circle (  1.16);

\path[draw=drawColor,line width= 0.4pt,line join=round,line cap=round,fill=fillColor] (397.16,211.70) circle (  1.16);

\path[draw=drawColor,line width= 0.4pt,line join=round,line cap=round,fill=fillColor] (398.02,211.43) circle (  1.16);

\path[draw=drawColor,line width= 0.4pt,line join=round,line cap=round,fill=fillColor] (398.86,211.42) circle (  1.16);

\path[draw=drawColor,line width= 0.4pt,line join=round,line cap=round,fill=fillColor] (399.69,211.33) circle (  1.16);

\path[draw=drawColor,line width= 0.4pt,line join=round,line cap=round,fill=fillColor] (400.51,211.08) circle (  1.16);

\path[draw=drawColor,line width= 0.4pt,line join=round,line cap=round,fill=fillColor] (401.31,210.99) circle (  1.16);

\path[draw=drawColor,line width= 0.4pt,line join=round,line cap=round,fill=fillColor] (402.10,210.95) circle (  1.16);

\path[draw=drawColor,line width= 0.4pt,line join=round,line cap=round,fill=fillColor] (402.87,210.72) circle (  1.16);

\path[draw=drawColor,line width= 0.4pt,line join=round,line cap=round,fill=fillColor] (403.64,210.70) circle (  1.16);

\path[draw=drawColor,line width= 0.4pt,line join=round,line cap=round,fill=fillColor] (404.39,210.60) circle (  1.16);

\path[draw=drawColor,line width= 0.4pt,line join=round,line cap=round,fill=fillColor] (405.13,210.40) circle (  1.16);

\path[draw=drawColor,line width= 0.4pt,line join=round,line cap=round,fill=fillColor] (405.86,210.36) circle (  1.16);

\path[draw=drawColor,line width= 0.4pt,line join=round,line cap=round,fill=fillColor] (406.58,210.25) circle (  1.16);

\path[draw=drawColor,line width= 0.4pt,line join=round,line cap=round,fill=fillColor] (407.29,210.05) circle (  1.16);

\path[draw=drawColor,line width= 0.4pt,line join=round,line cap=round,fill=fillColor] (407.99,209.86) circle (  1.16);

\path[draw=drawColor,line width= 0.4pt,line join=round,line cap=round,fill=fillColor] (408.68,209.84) circle (  1.16);

\path[draw=drawColor,line width= 0.4pt,line join=round,line cap=round,fill=fillColor] (409.37,209.83) circle (  1.16);

\path[draw=drawColor,line width= 0.4pt,line join=round,line cap=round,fill=fillColor] (410.04,209.55) circle (  1.16);

\path[draw=drawColor,line width= 0.4pt,line join=round,line cap=round,fill=fillColor] (410.71,209.51) circle (  1.16);

\path[draw=drawColor,line width= 0.4pt,line join=round,line cap=round,fill=fillColor] (411.36,209.48) circle (  1.16);

\path[draw=drawColor,line width= 0.4pt,line join=round,line cap=round,fill=fillColor] (412.01,209.43) circle (  1.16);

\path[draw=drawColor,line width= 0.4pt,line join=round,line cap=round,fill=fillColor] (412.65,209.40) circle (  1.16);

\path[draw=drawColor,line width= 0.4pt,line join=round,line cap=round,fill=fillColor] (413.29,209.33) circle (  1.16);

\path[draw=drawColor,line width= 0.4pt,line join=round,line cap=round,fill=fillColor] (413.92,209.25) circle (  1.16);

\path[draw=drawColor,line width= 0.4pt,line join=round,line cap=round,fill=fillColor] (414.54,209.25) circle (  1.16);

\path[draw=drawColor,line width= 0.4pt,line join=round,line cap=round,fill=fillColor] (415.15,209.22) circle (  1.16);

\path[draw=drawColor,line width= 0.4pt,line join=round,line cap=round,fill=fillColor] (415.76,209.21) circle (  1.16);

\path[draw=drawColor,line width= 0.4pt,line join=round,line cap=round,fill=fillColor] (416.36,209.18) circle (  1.16);

\path[draw=drawColor,line width= 0.4pt,line join=round,line cap=round,fill=fillColor] (416.95,209.06) circle (  1.16);

\path[draw=drawColor,line width= 0.4pt,line join=round,line cap=round,fill=fillColor] (417.54,209.00) circle (  1.16);

\path[draw=drawColor,line width= 0.4pt,line join=round,line cap=round,fill=fillColor] (418.12,208.89) circle (  1.16);

\path[draw=drawColor,line width= 0.4pt,line join=round,line cap=round,fill=fillColor] (418.69,208.84) circle (  1.16);

\path[draw=drawColor,line width= 0.4pt,line join=round,line cap=round,fill=fillColor] (419.26,208.79) circle (  1.16);

\path[draw=drawColor,line width= 0.4pt,line join=round,line cap=round,fill=fillColor] (419.83,208.69) circle (  1.16);

\path[draw=drawColor,line width= 0.4pt,line join=round,line cap=round,fill=fillColor] (420.39,208.57) circle (  1.16);

\path[draw=drawColor,line width= 0.4pt,line join=round,line cap=round,fill=fillColor] (420.94,208.46) circle (  1.16);

\path[draw=drawColor,line width= 0.4pt,line join=round,line cap=round,fill=fillColor] (421.49,208.39) circle (  1.16);

\path[draw=drawColor,line width= 0.4pt,line join=round,line cap=round,fill=fillColor] (422.03,208.33) circle (  1.16);

\path[draw=drawColor,line width= 0.4pt,line join=round,line cap=round,fill=fillColor] (422.57,208.30) circle (  1.16);

\path[draw=drawColor,line width= 0.4pt,line join=round,line cap=round,fill=fillColor] (423.10,208.08) circle (  1.16);

\path[draw=drawColor,line width= 0.4pt,line join=round,line cap=round,fill=fillColor] (423.63,207.97) circle (  1.16);

\path[draw=drawColor,line width= 0.4pt,line join=round,line cap=round,fill=fillColor] (424.16,207.77) circle (  1.16);

\path[draw=drawColor,line width= 0.4pt,line join=round,line cap=round,fill=fillColor] (424.68,207.76) circle (  1.16);

\path[draw=drawColor,line width= 0.4pt,line join=round,line cap=round,fill=fillColor] (425.19,207.73) circle (  1.16);

\path[draw=drawColor,line width= 0.4pt,line join=round,line cap=round,fill=fillColor] (425.70,207.46) circle (  1.16);

\path[draw=drawColor,line width= 0.4pt,line join=round,line cap=round,fill=fillColor] (426.21,207.22) circle (  1.16);

\path[draw=drawColor,line width= 0.4pt,line join=round,line cap=round,fill=fillColor] (426.71,207.21) circle (  1.16);

\path[draw=drawColor,line width= 0.4pt,line join=round,line cap=round,fill=fillColor] (427.21,207.06) circle (  1.16);

\path[draw=drawColor,line width= 0.4pt,line join=round,line cap=round,fill=fillColor] (427.71,207.04) circle (  1.16);

\path[draw=drawColor,line width= 0.4pt,line join=round,line cap=round,fill=fillColor] (428.20,207.04) circle (  1.16);

\path[draw=drawColor,line width= 0.4pt,line join=round,line cap=round,fill=fillColor] (428.68,207.03) circle (  1.16);

\path[draw=drawColor,line width= 0.4pt,line join=round,line cap=round,fill=fillColor] (429.17,206.93) circle (  1.16);

\path[draw=drawColor,line width= 0.4pt,line join=round,line cap=round,fill=fillColor] (429.65,206.89) circle (  1.16);

\path[draw=drawColor,line width= 0.4pt,line join=round,line cap=round,fill=fillColor] (430.12,206.85) circle (  1.16);

\path[draw=drawColor,line width= 0.4pt,line join=round,line cap=round,fill=fillColor] (430.59,206.77) circle (  1.16);

\path[draw=drawColor,line width= 0.4pt,line join=round,line cap=round,fill=fillColor] (431.06,206.73) circle (  1.16);

\path[draw=drawColor,line width= 0.4pt,line join=round,line cap=round,fill=fillColor] (431.53,206.73) circle (  1.16);

\path[draw=drawColor,line width= 0.4pt,line join=round,line cap=round,fill=fillColor] (431.99,206.19) circle (  1.16);

\path[draw=drawColor,line width= 0.4pt,line join=round,line cap=round,fill=fillColor] (432.45,206.03) circle (  1.16);

\path[draw=drawColor,line width= 0.4pt,line join=round,line cap=round,fill=fillColor] (432.90,205.40) circle (  1.16);

\path[draw=drawColor,line width= 0.4pt,line join=round,line cap=round,fill=fillColor] (433.36,205.39) circle (  1.16);

\path[draw=drawColor,line width= 0.4pt,line join=round,line cap=round,fill=fillColor] (433.81,205.38) circle (  1.16);

\path[draw=drawColor,line width= 0.4pt,line join=round,line cap=round,fill=fillColor] (434.25,205.33) circle (  1.16);

\path[draw=drawColor,line width= 0.4pt,line join=round,line cap=round,fill=fillColor] (434.69,205.06) circle (  1.16);

\path[draw=drawColor,line width= 0.4pt,line join=round,line cap=round,fill=fillColor] (435.13,204.98) circle (  1.16);

\path[draw=drawColor,line width= 0.4pt,line join=round,line cap=round,fill=fillColor] (435.57,204.92) circle (  1.16);

\path[draw=drawColor,line width= 0.4pt,line join=round,line cap=round,fill=fillColor] (436.01,204.89) circle (  1.16);

\path[draw=drawColor,line width= 0.4pt,line join=round,line cap=round,fill=fillColor] (436.44,204.81) circle (  1.16);

\path[draw=drawColor,line width= 0.4pt,line join=round,line cap=round,fill=fillColor] (436.87,204.71) circle (  1.16);

\path[draw=drawColor,line width= 0.4pt,line join=round,line cap=round,fill=fillColor] (437.29,204.43) circle (  1.16);

\path[draw=drawColor,line width= 0.4pt,line join=round,line cap=round,fill=fillColor] (437.71,204.42) circle (  1.16);

\path[draw=drawColor,line width= 0.4pt,line join=round,line cap=round,fill=fillColor] (438.14,204.12) circle (  1.16);

\path[draw=drawColor,line width= 0.4pt,line join=round,line cap=round,fill=fillColor] (438.55,204.10) circle (  1.16);

\path[draw=drawColor,line width= 0.4pt,line join=round,line cap=round,fill=fillColor] (438.97,204.06) circle (  1.16);

\path[draw=drawColor,line width= 0.4pt,line join=round,line cap=round,fill=fillColor] (439.38,203.80) circle (  1.16);

\path[draw=drawColor,line width= 0.4pt,line join=round,line cap=round,fill=fillColor] (439.79,203.61) circle (  1.16);

\path[draw=drawColor,line width= 0.4pt,line join=round,line cap=round,fill=fillColor] (440.20,203.33) circle (  1.16);

\path[draw=drawColor,line width= 0.4pt,line join=round,line cap=round,fill=fillColor] (440.60,202.96) circle (  1.16);

\path[draw=drawColor,line width= 0.4pt,line join=round,line cap=round,fill=fillColor] (441.01,202.81) circle (  1.16);

\path[draw=drawColor,line width= 0.4pt,line join=round,line cap=round,fill=fillColor] (441.41,202.37) circle (  1.16);

\path[draw=drawColor,line width= 0.4pt,line join=round,line cap=round,fill=fillColor] (441.81,201.93) circle (  1.16);

\path[draw=drawColor,line width= 0.4pt,line join=round,line cap=round,fill=fillColor] (442.20,201.71) circle (  1.16);

\path[draw=drawColor,line width= 0.4pt,line join=round,line cap=round,fill=fillColor] (442.60,201.07) circle (  1.16);

\path[draw=drawColor,line width= 0.4pt,line join=round,line cap=round,fill=fillColor] (442.99,200.71) circle (  1.16);

\path[draw=drawColor,line width= 0.4pt,line join=round,line cap=round,fill=fillColor] (443.38,198.35) circle (  1.16);

\path[draw=drawColor,line width= 0.4pt,line join=round,line cap=round,fill=fillColor] (443.77,198.35) circle (  1.16);

\path[draw=drawColor,line width= 0.4pt,line join=round,line cap=round,fill=fillColor] (444.15,198.35) circle (  1.16);

\path[draw=drawColor,line width= 0.4pt,line join=round,line cap=round,fill=fillColor] (444.53,198.35) circle (  1.16);

\path[draw=drawColor,line width= 0.4pt,line join=round,line cap=round,fill=fillColor] (444.91,198.35) circle (  1.16);

\path[draw=drawColor,line width= 0.4pt,line join=round,line cap=round,fill=fillColor] (445.29,198.35) circle (  1.16);

\path[draw=drawColor,line width= 0.4pt,line join=round,line cap=round,fill=fillColor] (445.67,198.35) circle (  1.16);

\path[draw=drawColor,line width= 0.4pt,line join=round,line cap=round,fill=fillColor] (446.05,198.35) circle (  1.16);

\path[draw=drawColor,line width= 0.4pt,line join=round,line cap=round,fill=fillColor] (446.42,198.35) circle (  1.16);

\path[draw=drawColor,line width= 0.4pt,line join=round,line cap=round,fill=fillColor] (446.79,198.35) circle (  1.16);

\path[draw=drawColor,line width= 0.4pt,line join=round,line cap=round,fill=fillColor] (447.16,198.35) circle (  1.16);

\path[draw=drawColor,line width= 0.4pt,line join=round,line cap=round,fill=fillColor] (447.53,198.35) circle (  1.16);

\path[draw=drawColor,line width= 0.4pt,line join=round,line cap=round,fill=fillColor] (447.89,198.35) circle (  1.16);

\path[draw=drawColor,line width= 0.4pt,line join=round,line cap=round,fill=fillColor] (448.25,198.35) circle (  1.16);

\path[draw=drawColor,line width= 0.4pt,line join=round,line cap=round,fill=fillColor] (448.62,198.35) circle (  1.16);

\path[draw=drawColor,line width= 0.4pt,line join=round,line cap=round,fill=fillColor] (448.98,198.35) circle (  1.16);

\path[draw=drawColor,line width= 0.4pt,line join=round,line cap=round,fill=fillColor] (449.33,198.35) circle (  1.16);

\path[draw=drawColor,line width= 0.4pt,line join=round,line cap=round,fill=fillColor] (449.69,198.35) circle (  1.16);

\path[draw=drawColor,line width= 0.4pt,line join=round,line cap=round,fill=fillColor] (450.05,198.35) circle (  1.16);

\path[draw=drawColor,line width= 0.4pt,line join=round,line cap=round,fill=fillColor] (450.40,198.35) circle (  1.16);

\path[draw=drawColor,line width= 0.4pt,line join=round,line cap=round,fill=fillColor] (450.75,198.35) circle (  1.16);

\path[draw=drawColor,line width= 0.4pt,line join=round,line cap=round,fill=fillColor] (451.10,198.35) circle (  1.16);

\path[draw=drawColor,line width= 0.4pt,line join=round,line cap=round,fill=fillColor] (451.45,198.35) circle (  1.16);

\path[draw=drawColor,line width= 0.4pt,line join=round,line cap=round,fill=fillColor] (451.79,198.35) circle (  1.16);

\path[draw=drawColor,line width= 0.4pt,line join=round,line cap=round,fill=fillColor] (452.14,198.35) circle (  1.16);

\path[draw=drawColor,line width= 0.4pt,line join=round,line cap=round,fill=fillColor] (452.48,198.35) circle (  1.16);

\path[draw=drawColor,line width= 0.4pt,line join=round,line cap=round,fill=fillColor] (452.82,198.35) circle (  1.16);

\path[draw=drawColor,line width= 0.4pt,line join=round,line cap=round,fill=fillColor] (453.16,198.35) circle (  1.16);

\path[draw=drawColor,line width= 0.4pt,line join=round,line cap=round,fill=fillColor] (453.50,198.35) circle (  1.16);

\path[draw=drawColor,line width= 0.4pt,line join=round,line cap=round,fill=fillColor] (453.84,198.35) circle (  1.16);

\path[draw=drawColor,line width= 0.4pt,line join=round,line cap=round,fill=fillColor] (454.17,198.35) circle (  1.16);

\path[draw=drawColor,line width= 0.4pt,line join=round,line cap=round,fill=fillColor] (454.51,198.35) circle (  1.16);

\path[draw=drawColor,line width= 0.4pt,line join=round,line cap=round,fill=fillColor] (454.84,198.35) circle (  1.16);

\path[draw=drawColor,line width= 0.4pt,line join=round,line cap=round,fill=fillColor] (455.17,198.35) circle (  1.16);

\path[draw=drawColor,line width= 0.4pt,line join=round,line cap=round,fill=fillColor] (455.50,198.35) circle (  1.16);

\path[draw=drawColor,line width= 0.4pt,line join=round,line cap=round,fill=fillColor] (455.83,198.35) circle (  1.16);

\path[draw=drawColor,line width= 0.4pt,line join=round,line cap=round,fill=fillColor] (456.16,198.35) circle (  1.16);

\path[draw=drawColor,line width= 0.4pt,line join=round,line cap=round,fill=fillColor] (456.48,198.35) circle (  1.16);

\path[draw=drawColor,line width= 0.4pt,line join=round,line cap=round,fill=fillColor] (456.80,198.35) circle (  1.16);

\path[draw=drawColor,line width= 0.4pt,line join=round,line cap=round,fill=fillColor] (457.13,198.35) circle (  1.16);

\path[draw=drawColor,line width= 0.4pt,line join=round,line cap=round,fill=fillColor] (457.45,198.35) circle (  1.16);

\path[draw=drawColor,line width= 0.4pt,line join=round,line cap=round,fill=fillColor] (457.77,198.35) circle (  1.16);

\path[draw=drawColor,line width= 0.4pt,line join=round,line cap=round,fill=fillColor] (458.09,198.35) circle (  1.16);

\path[draw=drawColor,line width= 0.4pt,line join=round,line cap=round,fill=fillColor] (458.40,198.35) circle (  1.16);

\path[draw=drawColor,line width= 0.4pt,line join=round,line cap=round,fill=fillColor] (458.72,198.35) circle (  1.16);

\path[draw=drawColor,line width= 0.4pt,line join=round,line cap=round,fill=fillColor] (459.03,198.35) circle (  1.16);

\path[draw=drawColor,line width= 0.4pt,line join=round,line cap=round,fill=fillColor] (459.35,198.35) circle (  1.16);

\path[draw=drawColor,line width= 0.4pt,line join=round,line cap=round,fill=fillColor] (459.66,198.35) circle (  1.16);

\path[draw=drawColor,line width= 0.4pt,line join=round,line cap=round,fill=fillColor] (459.97,198.35) circle (  1.16);

\path[draw=drawColor,line width= 0.4pt,line join=round,line cap=round,fill=fillColor] (460.28,198.35) circle (  1.16);

\path[draw=drawColor,line width= 0.4pt,line join=round,line cap=round,fill=fillColor] (460.59,198.35) circle (  1.16);

\path[draw=drawColor,line width= 0.4pt,line join=round,line cap=round,fill=fillColor] (460.90,198.35) circle (  1.16);

\path[draw=drawColor,line width= 0.4pt,line join=round,line cap=round,fill=fillColor] (461.20,198.35) circle (  1.16);

\path[draw=drawColor,line width= 0.4pt,line join=round,line cap=round,fill=fillColor] (461.51,198.35) circle (  1.16);

\path[draw=drawColor,line width= 0.4pt,line join=round,line cap=round,fill=fillColor] (461.81,198.35) circle (  1.16);

\path[draw=drawColor,line width= 0.4pt,line join=round,line cap=round,fill=fillColor] (462.11,198.35) circle (  1.16);

\path[draw=drawColor,line width= 0.4pt,line join=round,line cap=round,fill=fillColor] (462.42,198.35) circle (  1.16);

\path[draw=drawColor,line width= 0.4pt,line join=round,line cap=round,fill=fillColor] (462.72,198.35) circle (  1.16);

\path[draw=drawColor,line width= 0.4pt,line join=round,line cap=round,fill=fillColor] (463.01,198.35) circle (  1.16);

\path[draw=drawColor,line width= 0.4pt,line join=round,line cap=round,fill=fillColor] (463.31,198.35) circle (  1.16);

\path[draw=drawColor,line width= 0.4pt,line join=round,line cap=round,fill=fillColor] (463.61,198.35) circle (  1.16);

\path[draw=drawColor,line width= 0.4pt,line join=round,line cap=round,fill=fillColor] (463.91,198.35) circle (  1.16);

\path[draw=drawColor,line width= 0.4pt,line join=round,line cap=round,fill=fillColor] (464.20,198.35) circle (  1.16);

\path[draw=drawColor,line width= 0.4pt,line join=round,line cap=round,fill=fillColor] (464.49,198.35) circle (  1.16);

\path[draw=drawColor,line width= 0.4pt,line join=round,line cap=round,fill=fillColor] (464.79,198.35) circle (  1.16);

\path[draw=drawColor,line width= 0.4pt,line join=round,line cap=round,fill=fillColor] (465.08,198.35) circle (  1.16);

\path[draw=drawColor,line width= 0.4pt,line join=round,line cap=round,fill=fillColor] (465.37,198.35) circle (  1.16);

\path[draw=drawColor,line width= 0.4pt,line join=round,line cap=round,fill=fillColor] (465.66,198.35) circle (  1.16);

\path[draw=drawColor,line width= 0.4pt,line join=round,line cap=round,fill=fillColor] (465.95,198.35) circle (  1.16);

\path[draw=drawColor,line width= 0.4pt,line join=round,line cap=round,fill=fillColor] (466.23,198.35) circle (  1.16);

\path[draw=drawColor,line width= 0.4pt,line join=round,line cap=round,fill=fillColor] (466.52,198.35) circle (  1.16);

\path[draw=drawColor,line width= 0.4pt,line join=round,line cap=round,fill=fillColor] (466.81,198.35) circle (  1.16);

\path[draw=drawColor,line width= 0.4pt,line join=round,line cap=round,fill=fillColor] (467.09,198.35) circle (  1.16);

\path[draw=drawColor,line width= 0.4pt,line join=round,line cap=round,fill=fillColor] (467.37,198.35) circle (  1.16);

\path[draw=drawColor,line width= 0.4pt,line join=round,line cap=round,fill=fillColor] (467.66,198.35) circle (  1.16);

\path[draw=drawColor,line width= 0.4pt,line join=round,line cap=round,fill=fillColor] (467.94,198.35) circle (  1.16);

\path[draw=drawColor,line width= 0.4pt,line join=round,line cap=round,fill=fillColor] (468.22,198.35) circle (  1.16);

\path[draw=drawColor,line width= 0.4pt,line join=round,line cap=round,fill=fillColor] (468.50,198.35) circle (  1.16);

\path[draw=drawColor,line width= 0.4pt,line join=round,line cap=round,fill=fillColor] (468.78,198.35) circle (  1.16);

\path[draw=drawColor,line width= 0.4pt,line join=round,line cap=round,fill=fillColor] (469.05,198.35) circle (  1.16);

\path[draw=drawColor,line width= 0.4pt,line join=round,line cap=round,fill=fillColor] (469.33,198.35) circle (  1.16);

\path[draw=drawColor,line width= 0.4pt,line join=round,line cap=round,fill=fillColor] (469.61,198.35) circle (  1.16);

\path[draw=drawColor,line width= 0.4pt,line join=round,line cap=round,fill=fillColor] (469.88,198.35) circle (  1.16);

\path[draw=drawColor,line width= 0.4pt,line join=round,line cap=round,fill=fillColor] (470.15,198.35) circle (  1.16);

\path[draw=drawColor,line width= 0.4pt,line join=round,line cap=round,fill=fillColor] (470.43,198.35) circle (  1.16);

\path[draw=drawColor,line width= 0.4pt,line join=round,line cap=round,fill=fillColor] (470.70,198.35) circle (  1.16);

\path[draw=drawColor,line width= 0.4pt,line join=round,line cap=round,fill=fillColor] (470.97,198.35) circle (  1.16);

\path[draw=drawColor,line width= 0.4pt,line join=round,line cap=round,fill=fillColor] (471.24,198.35) circle (  1.16);

\path[draw=drawColor,line width= 0.4pt,line join=round,line cap=round,fill=fillColor] (471.51,198.35) circle (  1.16);

\path[draw=drawColor,line width= 0.4pt,line join=round,line cap=round,fill=fillColor] (471.78,198.35) circle (  1.16);

\path[draw=drawColor,line width= 0.4pt,line join=round,line cap=round,fill=fillColor] (472.05,198.35) circle (  1.16);
\definecolor[named]{drawColor}{rgb}{0.22,0.49,0.72}
\definecolor[named]{fillColor}{rgb}{0.22,0.49,0.72}

\path[draw=drawColor,line width= 0.4pt,line join=round,line cap=round,fill=fillColor] (331.92,226.85) circle (  1.16);

\path[draw=drawColor,line width= 0.4pt,line join=round,line cap=round,fill=fillColor] (339.28,223.54) circle (  1.16);

\path[draw=drawColor,line width= 0.4pt,line join=round,line cap=round,fill=fillColor] (344.45,222.17) circle (  1.16);

\path[draw=drawColor,line width= 0.4pt,line join=round,line cap=round,fill=fillColor] (348.57,222.13) circle (  1.16);

\path[draw=drawColor,line width= 0.4pt,line join=round,line cap=round,fill=fillColor] (352.04,221.82) circle (  1.16);

\path[draw=drawColor,line width= 0.4pt,line join=round,line cap=round,fill=fillColor] (355.08,220.20) circle (  1.16);

\path[draw=drawColor,line width= 0.4pt,line join=round,line cap=round,fill=fillColor] (357.79,219.93) circle (  1.16);

\path[draw=drawColor,line width= 0.4pt,line join=round,line cap=round,fill=fillColor] (360.26,218.14) circle (  1.16);

\path[draw=drawColor,line width= 0.4pt,line join=round,line cap=round,fill=fillColor] (362.53,217.66) circle (  1.16);

\path[draw=drawColor,line width= 0.4pt,line join=round,line cap=round,fill=fillColor] (364.64,216.97) circle (  1.16);

\path[draw=drawColor,line width= 0.4pt,line join=round,line cap=round,fill=fillColor] (366.61,216.59) circle (  1.16);

\path[draw=drawColor,line width= 0.4pt,line join=round,line cap=round,fill=fillColor] (368.46,216.28) circle (  1.16);

\path[draw=drawColor,line width= 0.4pt,line join=round,line cap=round,fill=fillColor] (370.22,216.16) circle (  1.16);

\path[draw=drawColor,line width= 0.4pt,line join=round,line cap=round,fill=fillColor] (371.89,216.03) circle (  1.16);

\path[draw=drawColor,line width= 0.4pt,line join=round,line cap=round,fill=fillColor] (373.47,215.79) circle (  1.16);

\path[draw=drawColor,line width= 0.4pt,line join=round,line cap=round,fill=fillColor] (374.99,215.30) circle (  1.16);

\path[draw=drawColor,line width= 0.4pt,line join=round,line cap=round,fill=fillColor] (376.45,215.02) circle (  1.16);

\path[draw=drawColor,line width= 0.4pt,line join=round,line cap=round,fill=fillColor] (377.85,214.46) circle (  1.16);

\path[draw=drawColor,line width= 0.4pt,line join=round,line cap=round,fill=fillColor] (379.21,214.37) circle (  1.16);

\path[draw=drawColor,line width= 0.4pt,line join=round,line cap=round,fill=fillColor] (380.51,214.33) circle (  1.16);

\path[draw=drawColor,line width= 0.4pt,line join=round,line cap=round,fill=fillColor] (381.77,214.33) circle (  1.16);

\path[draw=drawColor,line width= 0.4pt,line join=round,line cap=round,fill=fillColor] (382.99,213.67) circle (  1.16);

\path[draw=drawColor,line width= 0.4pt,line join=round,line cap=round,fill=fillColor] (384.18,213.66) circle (  1.16);

\path[draw=drawColor,line width= 0.4pt,line join=round,line cap=round,fill=fillColor] (385.33,213.59) circle (  1.16);

\path[draw=drawColor,line width= 0.4pt,line join=round,line cap=round,fill=fillColor] (386.45,213.56) circle (  1.16);

\path[draw=drawColor,line width= 0.4pt,line join=round,line cap=round,fill=fillColor] (387.54,213.37) circle (  1.16);

\path[draw=drawColor,line width= 0.4pt,line join=round,line cap=round,fill=fillColor] (388.60,213.28) circle (  1.16);

\path[draw=drawColor,line width= 0.4pt,line join=round,line cap=round,fill=fillColor] (389.64,213.16) circle (  1.16);

\path[draw=drawColor,line width= 0.4pt,line join=round,line cap=round,fill=fillColor] (390.65,213.13) circle (  1.16);

\path[draw=drawColor,line width= 0.4pt,line join=round,line cap=round,fill=fillColor] (391.64,213.10) circle (  1.16);

\path[draw=drawColor,line width= 0.4pt,line join=round,line cap=round,fill=fillColor] (392.61,212.94) circle (  1.16);

\path[draw=drawColor,line width= 0.4pt,line join=round,line cap=round,fill=fillColor] (393.56,212.77) circle (  1.16);

\path[draw=drawColor,line width= 0.4pt,line join=round,line cap=round,fill=fillColor] (394.49,212.71) circle (  1.16);

\path[draw=drawColor,line width= 0.4pt,line join=round,line cap=round,fill=fillColor] (395.40,212.70) circle (  1.16);

\path[draw=drawColor,line width= 0.4pt,line join=round,line cap=round,fill=fillColor] (396.29,212.69) circle (  1.16);

\path[draw=drawColor,line width= 0.4pt,line join=round,line cap=round,fill=fillColor] (397.16,212.33) circle (  1.16);

\path[draw=drawColor,line width= 0.4pt,line join=round,line cap=round,fill=fillColor] (398.02,212.24) circle (  1.16);

\path[draw=drawColor,line width= 0.4pt,line join=round,line cap=round,fill=fillColor] (398.86,212.20) circle (  1.16);

\path[draw=drawColor,line width= 0.4pt,line join=round,line cap=round,fill=fillColor] (399.69,212.08) circle (  1.16);

\path[draw=drawColor,line width= 0.4pt,line join=round,line cap=round,fill=fillColor] (400.51,211.97) circle (  1.16);

\path[draw=drawColor,line width= 0.4pt,line join=round,line cap=round,fill=fillColor] (401.31,211.88) circle (  1.16);

\path[draw=drawColor,line width= 0.4pt,line join=round,line cap=round,fill=fillColor] (402.10,211.87) circle (  1.16);

\path[draw=drawColor,line width= 0.4pt,line join=round,line cap=round,fill=fillColor] (402.87,211.83) circle (  1.16);

\path[draw=drawColor,line width= 0.4pt,line join=round,line cap=round,fill=fillColor] (403.64,211.70) circle (  1.16);

\path[draw=drawColor,line width= 0.4pt,line join=round,line cap=round,fill=fillColor] (404.39,211.61) circle (  1.16);

\path[draw=drawColor,line width= 0.4pt,line join=round,line cap=round,fill=fillColor] (405.13,211.53) circle (  1.16);

\path[draw=drawColor,line width= 0.4pt,line join=round,line cap=round,fill=fillColor] (405.86,211.51) circle (  1.16);

\path[draw=drawColor,line width= 0.4pt,line join=round,line cap=round,fill=fillColor] (406.58,211.29) circle (  1.16);

\path[draw=drawColor,line width= 0.4pt,line join=round,line cap=round,fill=fillColor] (407.29,211.06) circle (  1.16);

\path[draw=drawColor,line width= 0.4pt,line join=round,line cap=round,fill=fillColor] (407.99,210.94) circle (  1.16);

\path[draw=drawColor,line width= 0.4pt,line join=round,line cap=round,fill=fillColor] (408.68,210.91) circle (  1.16);

\path[draw=drawColor,line width= 0.4pt,line join=round,line cap=round,fill=fillColor] (409.37,210.89) circle (  1.16);

\path[draw=drawColor,line width= 0.4pt,line join=round,line cap=round,fill=fillColor] (410.04,210.83) circle (  1.16);

\path[draw=drawColor,line width= 0.4pt,line join=round,line cap=round,fill=fillColor] (410.71,210.75) circle (  1.16);

\path[draw=drawColor,line width= 0.4pt,line join=round,line cap=round,fill=fillColor] (411.36,210.74) circle (  1.16);

\path[draw=drawColor,line width= 0.4pt,line join=round,line cap=round,fill=fillColor] (412.01,210.72) circle (  1.16);

\path[draw=drawColor,line width= 0.4pt,line join=round,line cap=round,fill=fillColor] (412.65,210.70) circle (  1.16);

\path[draw=drawColor,line width= 0.4pt,line join=round,line cap=round,fill=fillColor] (413.29,210.45) circle (  1.16);

\path[draw=drawColor,line width= 0.4pt,line join=round,line cap=round,fill=fillColor] (413.92,210.32) circle (  1.16);

\path[draw=drawColor,line width= 0.4pt,line join=round,line cap=round,fill=fillColor] (414.54,210.11) circle (  1.16);

\path[draw=drawColor,line width= 0.4pt,line join=round,line cap=round,fill=fillColor] (415.15,209.93) circle (  1.16);

\path[draw=drawColor,line width= 0.4pt,line join=round,line cap=round,fill=fillColor] (415.76,209.65) circle (  1.16);

\path[draw=drawColor,line width= 0.4pt,line join=round,line cap=round,fill=fillColor] (416.36,209.62) circle (  1.16);

\path[draw=drawColor,line width= 0.4pt,line join=round,line cap=round,fill=fillColor] (416.95,209.58) circle (  1.16);

\path[draw=drawColor,line width= 0.4pt,line join=round,line cap=round,fill=fillColor] (417.54,209.54) circle (  1.16);

\path[draw=drawColor,line width= 0.4pt,line join=round,line cap=round,fill=fillColor] (418.12,209.43) circle (  1.16);

\path[draw=drawColor,line width= 0.4pt,line join=round,line cap=round,fill=fillColor] (418.69,209.38) circle (  1.16);

\path[draw=drawColor,line width= 0.4pt,line join=round,line cap=round,fill=fillColor] (419.26,209.07) circle (  1.16);

\path[draw=drawColor,line width= 0.4pt,line join=round,line cap=round,fill=fillColor] (419.83,208.98) circle (  1.16);

\path[draw=drawColor,line width= 0.4pt,line join=round,line cap=round,fill=fillColor] (420.39,208.97) circle (  1.16);

\path[draw=drawColor,line width= 0.4pt,line join=round,line cap=round,fill=fillColor] (420.94,208.76) circle (  1.16);

\path[draw=drawColor,line width= 0.4pt,line join=round,line cap=round,fill=fillColor] (421.49,208.73) circle (  1.16);

\path[draw=drawColor,line width= 0.4pt,line join=round,line cap=round,fill=fillColor] (422.03,208.70) circle (  1.16);

\path[draw=drawColor,line width= 0.4pt,line join=round,line cap=round,fill=fillColor] (422.57,208.69) circle (  1.16);

\path[draw=drawColor,line width= 0.4pt,line join=round,line cap=round,fill=fillColor] (423.10,208.39) circle (  1.16);

\path[draw=drawColor,line width= 0.4pt,line join=round,line cap=round,fill=fillColor] (423.63,208.35) circle (  1.16);

\path[draw=drawColor,line width= 0.4pt,line join=round,line cap=round,fill=fillColor] (424.16,208.25) circle (  1.16);

\path[draw=drawColor,line width= 0.4pt,line join=round,line cap=round,fill=fillColor] (424.68,208.18) circle (  1.16);

\path[draw=drawColor,line width= 0.4pt,line join=round,line cap=round,fill=fillColor] (425.19,208.02) circle (  1.16);

\path[draw=drawColor,line width= 0.4pt,line join=round,line cap=round,fill=fillColor] (425.70,208.01) circle (  1.16);

\path[draw=drawColor,line width= 0.4pt,line join=round,line cap=round,fill=fillColor] (426.21,207.96) circle (  1.16);

\path[draw=drawColor,line width= 0.4pt,line join=round,line cap=round,fill=fillColor] (426.71,207.87) circle (  1.16);

\path[draw=drawColor,line width= 0.4pt,line join=round,line cap=round,fill=fillColor] (427.21,207.53) circle (  1.16);

\path[draw=drawColor,line width= 0.4pt,line join=round,line cap=round,fill=fillColor] (427.71,207.47) circle (  1.16);

\path[draw=drawColor,line width= 0.4pt,line join=round,line cap=round,fill=fillColor] (428.20,207.46) circle (  1.16);

\path[draw=drawColor,line width= 0.4pt,line join=round,line cap=round,fill=fillColor] (428.68,207.11) circle (  1.16);

\path[draw=drawColor,line width= 0.4pt,line join=round,line cap=round,fill=fillColor] (429.17,207.06) circle (  1.16);

\path[draw=drawColor,line width= 0.4pt,line join=round,line cap=round,fill=fillColor] (429.65,207.03) circle (  1.16);

\path[draw=drawColor,line width= 0.4pt,line join=round,line cap=round,fill=fillColor] (430.12,206.81) circle (  1.16);

\path[draw=drawColor,line width= 0.4pt,line join=round,line cap=round,fill=fillColor] (430.59,206.80) circle (  1.16);

\path[draw=drawColor,line width= 0.4pt,line join=round,line cap=round,fill=fillColor] (431.06,206.79) circle (  1.16);

\path[draw=drawColor,line width= 0.4pt,line join=round,line cap=round,fill=fillColor] (431.53,206.75) circle (  1.16);

\path[draw=drawColor,line width= 0.4pt,line join=round,line cap=round,fill=fillColor] (431.99,206.73) circle (  1.16);

\path[draw=drawColor,line width= 0.4pt,line join=round,line cap=round,fill=fillColor] (432.45,206.72) circle (  1.16);

\path[draw=drawColor,line width= 0.4pt,line join=round,line cap=round,fill=fillColor] (432.90,206.56) circle (  1.16);

\path[draw=drawColor,line width= 0.4pt,line join=round,line cap=round,fill=fillColor] (433.36,206.41) circle (  1.16);

\path[draw=drawColor,line width= 0.4pt,line join=round,line cap=round,fill=fillColor] (433.81,206.40) circle (  1.16);

\path[draw=drawColor,line width= 0.4pt,line join=round,line cap=round,fill=fillColor] (434.25,206.20) circle (  1.16);

\path[draw=drawColor,line width= 0.4pt,line join=round,line cap=round,fill=fillColor] (434.69,206.20) circle (  1.16);

\path[draw=drawColor,line width= 0.4pt,line join=round,line cap=round,fill=fillColor] (435.13,206.19) circle (  1.16);

\path[draw=drawColor,line width= 0.4pt,line join=round,line cap=round,fill=fillColor] (435.57,205.85) circle (  1.16);

\path[draw=drawColor,line width= 0.4pt,line join=round,line cap=round,fill=fillColor] (436.01,205.62) circle (  1.16);

\path[draw=drawColor,line width= 0.4pt,line join=round,line cap=round,fill=fillColor] (436.44,205.42) circle (  1.16);

\path[draw=drawColor,line width= 0.4pt,line join=round,line cap=round,fill=fillColor] (436.87,205.21) circle (  1.16);

\path[draw=drawColor,line width= 0.4pt,line join=round,line cap=round,fill=fillColor] (437.29,205.12) circle (  1.16);

\path[draw=drawColor,line width= 0.4pt,line join=round,line cap=round,fill=fillColor] (437.71,205.08) circle (  1.16);

\path[draw=drawColor,line width= 0.4pt,line join=round,line cap=round,fill=fillColor] (438.14,204.92) circle (  1.16);

\path[draw=drawColor,line width= 0.4pt,line join=round,line cap=round,fill=fillColor] (438.55,204.74) circle (  1.16);

\path[draw=drawColor,line width= 0.4pt,line join=round,line cap=round,fill=fillColor] (438.97,204.48) circle (  1.16);

\path[draw=drawColor,line width= 0.4pt,line join=round,line cap=round,fill=fillColor] (439.38,204.28) circle (  1.16);

\path[draw=drawColor,line width= 0.4pt,line join=round,line cap=round,fill=fillColor] (439.79,203.45) circle (  1.16);

\path[draw=drawColor,line width= 0.4pt,line join=round,line cap=round,fill=fillColor] (440.20,203.06) circle (  1.16);

\path[draw=drawColor,line width= 0.4pt,line join=round,line cap=round,fill=fillColor] (440.60,202.74) circle (  1.16);

\path[draw=drawColor,line width= 0.4pt,line join=round,line cap=round,fill=fillColor] (441.01,202.58) circle (  1.16);

\path[draw=drawColor,line width= 0.4pt,line join=round,line cap=round,fill=fillColor] (441.41,198.35) circle (  1.16);

\path[draw=drawColor,line width= 0.4pt,line join=round,line cap=round,fill=fillColor] (441.81,198.35) circle (  1.16);

\path[draw=drawColor,line width= 0.4pt,line join=round,line cap=round,fill=fillColor] (442.20,198.35) circle (  1.16);

\path[draw=drawColor,line width= 0.4pt,line join=round,line cap=round,fill=fillColor] (442.60,198.35) circle (  1.16);

\path[draw=drawColor,line width= 0.4pt,line join=round,line cap=round,fill=fillColor] (442.99,198.35) circle (  1.16);

\path[draw=drawColor,line width= 0.4pt,line join=round,line cap=round,fill=fillColor] (443.38,198.35) circle (  1.16);

\path[draw=drawColor,line width= 0.4pt,line join=round,line cap=round,fill=fillColor] (443.77,198.35) circle (  1.16);

\path[draw=drawColor,line width= 0.4pt,line join=round,line cap=round,fill=fillColor] (444.15,198.35) circle (  1.16);

\path[draw=drawColor,line width= 0.4pt,line join=round,line cap=round,fill=fillColor] (444.53,198.35) circle (  1.16);

\path[draw=drawColor,line width= 0.4pt,line join=round,line cap=round,fill=fillColor] (444.91,198.35) circle (  1.16);

\path[draw=drawColor,line width= 0.4pt,line join=round,line cap=round,fill=fillColor] (445.29,198.35) circle (  1.16);

\path[draw=drawColor,line width= 0.4pt,line join=round,line cap=round,fill=fillColor] (445.67,198.35) circle (  1.16);

\path[draw=drawColor,line width= 0.4pt,line join=round,line cap=round,fill=fillColor] (446.05,198.35) circle (  1.16);

\path[draw=drawColor,line width= 0.4pt,line join=round,line cap=round,fill=fillColor] (446.42,198.35) circle (  1.16);

\path[draw=drawColor,line width= 0.4pt,line join=round,line cap=round,fill=fillColor] (446.79,198.35) circle (  1.16);

\path[draw=drawColor,line width= 0.4pt,line join=round,line cap=round,fill=fillColor] (447.16,198.35) circle (  1.16);

\path[draw=drawColor,line width= 0.4pt,line join=round,line cap=round,fill=fillColor] (447.53,198.35) circle (  1.16);

\path[draw=drawColor,line width= 0.4pt,line join=round,line cap=round,fill=fillColor] (447.89,198.35) circle (  1.16);

\path[draw=drawColor,line width= 0.4pt,line join=round,line cap=round,fill=fillColor] (448.25,198.35) circle (  1.16);

\path[draw=drawColor,line width= 0.4pt,line join=round,line cap=round,fill=fillColor] (448.62,198.35) circle (  1.16);

\path[draw=drawColor,line width= 0.4pt,line join=round,line cap=round,fill=fillColor] (448.98,198.35) circle (  1.16);

\path[draw=drawColor,line width= 0.4pt,line join=round,line cap=round,fill=fillColor] (449.33,198.35) circle (  1.16);

\path[draw=drawColor,line width= 0.4pt,line join=round,line cap=round,fill=fillColor] (449.69,198.35) circle (  1.16);

\path[draw=drawColor,line width= 0.4pt,line join=round,line cap=round,fill=fillColor] (450.05,198.35) circle (  1.16);

\path[draw=drawColor,line width= 0.4pt,line join=round,line cap=round,fill=fillColor] (450.40,198.35) circle (  1.16);

\path[draw=drawColor,line width= 0.4pt,line join=round,line cap=round,fill=fillColor] (450.75,198.35) circle (  1.16);

\path[draw=drawColor,line width= 0.4pt,line join=round,line cap=round,fill=fillColor] (451.10,198.35) circle (  1.16);

\path[draw=drawColor,line width= 0.4pt,line join=round,line cap=round,fill=fillColor] (451.45,198.35) circle (  1.16);

\path[draw=drawColor,line width= 0.4pt,line join=round,line cap=round,fill=fillColor] (451.79,198.35) circle (  1.16);

\path[draw=drawColor,line width= 0.4pt,line join=round,line cap=round,fill=fillColor] (452.14,198.35) circle (  1.16);

\path[draw=drawColor,line width= 0.4pt,line join=round,line cap=round,fill=fillColor] (452.48,198.35) circle (  1.16);

\path[draw=drawColor,line width= 0.4pt,line join=round,line cap=round,fill=fillColor] (452.82,198.35) circle (  1.16);

\path[draw=drawColor,line width= 0.4pt,line join=round,line cap=round,fill=fillColor] (453.16,198.35) circle (  1.16);

\path[draw=drawColor,line width= 0.4pt,line join=round,line cap=round,fill=fillColor] (453.50,198.35) circle (  1.16);

\path[draw=drawColor,line width= 0.4pt,line join=round,line cap=round,fill=fillColor] (453.84,198.35) circle (  1.16);

\path[draw=drawColor,line width= 0.4pt,line join=round,line cap=round,fill=fillColor] (454.17,198.35) circle (  1.16);

\path[draw=drawColor,line width= 0.4pt,line join=round,line cap=round,fill=fillColor] (454.51,198.35) circle (  1.16);

\path[draw=drawColor,line width= 0.4pt,line join=round,line cap=round,fill=fillColor] (454.84,198.35) circle (  1.16);

\path[draw=drawColor,line width= 0.4pt,line join=round,line cap=round,fill=fillColor] (455.17,198.35) circle (  1.16);

\path[draw=drawColor,line width= 0.4pt,line join=round,line cap=round,fill=fillColor] (455.50,198.35) circle (  1.16);

\path[draw=drawColor,line width= 0.4pt,line join=round,line cap=round,fill=fillColor] (455.83,198.35) circle (  1.16);

\path[draw=drawColor,line width= 0.4pt,line join=round,line cap=round,fill=fillColor] (456.16,198.35) circle (  1.16);

\path[draw=drawColor,line width= 0.4pt,line join=round,line cap=round,fill=fillColor] (456.48,198.35) circle (  1.16);

\path[draw=drawColor,line width= 0.4pt,line join=round,line cap=round,fill=fillColor] (456.80,198.35) circle (  1.16);

\path[draw=drawColor,line width= 0.4pt,line join=round,line cap=round,fill=fillColor] (457.13,198.35) circle (  1.16);

\path[draw=drawColor,line width= 0.4pt,line join=round,line cap=round,fill=fillColor] (457.45,198.35) circle (  1.16);

\path[draw=drawColor,line width= 0.4pt,line join=round,line cap=round,fill=fillColor] (457.77,198.35) circle (  1.16);

\path[draw=drawColor,line width= 0.4pt,line join=round,line cap=round,fill=fillColor] (458.09,198.35) circle (  1.16);

\path[draw=drawColor,line width= 0.4pt,line join=round,line cap=round,fill=fillColor] (458.40,198.35) circle (  1.16);

\path[draw=drawColor,line width= 0.4pt,line join=round,line cap=round,fill=fillColor] (458.72,198.35) circle (  1.16);

\path[draw=drawColor,line width= 0.4pt,line join=round,line cap=round,fill=fillColor] (459.03,198.35) circle (  1.16);

\path[draw=drawColor,line width= 0.4pt,line join=round,line cap=round,fill=fillColor] (459.35,198.35) circle (  1.16);

\path[draw=drawColor,line width= 0.4pt,line join=round,line cap=round,fill=fillColor] (459.66,198.35) circle (  1.16);

\path[draw=drawColor,line width= 0.4pt,line join=round,line cap=round,fill=fillColor] (459.97,198.35) circle (  1.16);

\path[draw=drawColor,line width= 0.4pt,line join=round,line cap=round,fill=fillColor] (460.28,198.35) circle (  1.16);

\path[draw=drawColor,line width= 0.4pt,line join=round,line cap=round,fill=fillColor] (460.59,198.35) circle (  1.16);

\path[draw=drawColor,line width= 0.4pt,line join=round,line cap=round,fill=fillColor] (460.90,198.35) circle (  1.16);

\path[draw=drawColor,line width= 0.4pt,line join=round,line cap=round,fill=fillColor] (461.20,198.35) circle (  1.16);

\path[draw=drawColor,line width= 0.4pt,line join=round,line cap=round,fill=fillColor] (461.51,198.35) circle (  1.16);

\path[draw=drawColor,line width= 0.4pt,line join=round,line cap=round,fill=fillColor] (461.81,198.35) circle (  1.16);

\path[draw=drawColor,line width= 0.4pt,line join=round,line cap=round,fill=fillColor] (462.11,198.35) circle (  1.16);

\path[draw=drawColor,line width= 0.4pt,line join=round,line cap=round,fill=fillColor] (462.42,198.35) circle (  1.16);

\path[draw=drawColor,line width= 0.4pt,line join=round,line cap=round,fill=fillColor] (462.72,198.35) circle (  1.16);

\path[draw=drawColor,line width= 0.4pt,line join=round,line cap=round,fill=fillColor] (463.01,198.35) circle (  1.16);

\path[draw=drawColor,line width= 0.4pt,line join=round,line cap=round,fill=fillColor] (463.31,198.35) circle (  1.16);

\path[draw=drawColor,line width= 0.4pt,line join=round,line cap=round,fill=fillColor] (463.61,198.35) circle (  1.16);

\path[draw=drawColor,line width= 0.4pt,line join=round,line cap=round,fill=fillColor] (463.91,198.35) circle (  1.16);

\path[draw=drawColor,line width= 0.4pt,line join=round,line cap=round,fill=fillColor] (464.20,198.35) circle (  1.16);

\path[draw=drawColor,line width= 0.4pt,line join=round,line cap=round,fill=fillColor] (464.49,198.35) circle (  1.16);

\path[draw=drawColor,line width= 0.4pt,line join=round,line cap=round,fill=fillColor] (464.79,198.35) circle (  1.16);

\path[draw=drawColor,line width= 0.4pt,line join=round,line cap=round,fill=fillColor] (465.08,198.35) circle (  1.16);

\path[draw=drawColor,line width= 0.4pt,line join=round,line cap=round,fill=fillColor] (465.37,198.35) circle (  1.16);

\path[draw=drawColor,line width= 0.4pt,line join=round,line cap=round,fill=fillColor] (465.66,198.35) circle (  1.16);

\path[draw=drawColor,line width= 0.4pt,line join=round,line cap=round,fill=fillColor] (465.95,198.35) circle (  1.16);

\path[draw=drawColor,line width= 0.4pt,line join=round,line cap=round,fill=fillColor] (466.23,198.35) circle (  1.16);

\path[draw=drawColor,line width= 0.4pt,line join=round,line cap=round,fill=fillColor] (466.52,198.35) circle (  1.16);

\path[draw=drawColor,line width= 0.4pt,line join=round,line cap=round,fill=fillColor] (466.81,198.35) circle (  1.16);

\path[draw=drawColor,line width= 0.4pt,line join=round,line cap=round,fill=fillColor] (467.09,198.35) circle (  1.16);

\path[draw=drawColor,line width= 0.4pt,line join=round,line cap=round,fill=fillColor] (467.37,198.35) circle (  1.16);

\path[draw=drawColor,line width= 0.4pt,line join=round,line cap=round,fill=fillColor] (467.66,198.35) circle (  1.16);

\path[draw=drawColor,line width= 0.4pt,line join=round,line cap=round,fill=fillColor] (467.94,198.35) circle (  1.16);

\path[draw=drawColor,line width= 0.4pt,line join=round,line cap=round,fill=fillColor] (468.22,198.35) circle (  1.16);

\path[draw=drawColor,line width= 0.4pt,line join=round,line cap=round,fill=fillColor] (468.50,198.35) circle (  1.16);

\path[draw=drawColor,line width= 0.4pt,line join=round,line cap=round,fill=fillColor] (468.78,198.35) circle (  1.16);

\path[draw=drawColor,line width= 0.4pt,line join=round,line cap=round,fill=fillColor] (469.05,198.35) circle (  1.16);

\path[draw=drawColor,line width= 0.4pt,line join=round,line cap=round,fill=fillColor] (469.33,198.35) circle (  1.16);

\path[draw=drawColor,line width= 0.4pt,line join=round,line cap=round,fill=fillColor] (469.61,198.35) circle (  1.16);

\path[draw=drawColor,line width= 0.4pt,line join=round,line cap=round,fill=fillColor] (469.88,198.35) circle (  1.16);

\path[draw=drawColor,line width= 0.4pt,line join=round,line cap=round,fill=fillColor] (470.15,198.35) circle (  1.16);

\path[draw=drawColor,line width= 0.4pt,line join=round,line cap=round,fill=fillColor] (470.43,198.35) circle (  1.16);

\path[draw=drawColor,line width= 0.4pt,line join=round,line cap=round,fill=fillColor] (470.70,198.35) circle (  1.16);

\path[draw=drawColor,line width= 0.4pt,line join=round,line cap=round,fill=fillColor] (470.97,198.35) circle (  1.16);

\path[draw=drawColor,line width= 0.4pt,line join=round,line cap=round,fill=fillColor] (471.24,198.35) circle (  1.16);

\path[draw=drawColor,line width= 0.4pt,line join=round,line cap=round,fill=fillColor] (471.51,198.35) circle (  1.16);

\path[draw=drawColor,line width= 0.4pt,line join=round,line cap=round,fill=fillColor] (471.78,198.35) circle (  1.16);

\path[draw=drawColor,line width= 0.4pt,line join=round,line cap=round,fill=fillColor] (472.05,198.35) circle (  1.16);
\definecolor[named]{drawColor}{rgb}{0.30,0.69,0.29}
\definecolor[named]{fillColor}{rgb}{0.30,0.69,0.29}

\path[draw=drawColor,line width= 0.4pt,line join=round,line cap=round,fill=fillColor] (331.92,226.85) circle (  1.16);

\path[draw=drawColor,line width= 0.4pt,line join=round,line cap=round,fill=fillColor] (339.28,223.54) circle (  1.16);

\path[draw=drawColor,line width= 0.4pt,line join=round,line cap=round,fill=fillColor] (344.45,222.11) circle (  1.16);

\path[draw=drawColor,line width= 0.4pt,line join=round,line cap=round,fill=fillColor] (348.57,221.13) circle (  1.16);

\path[draw=drawColor,line width= 0.4pt,line join=round,line cap=round,fill=fillColor] (352.04,221.06) circle (  1.16);

\path[draw=drawColor,line width= 0.4pt,line join=round,line cap=round,fill=fillColor] (355.08,219.93) circle (  1.16);

\path[draw=drawColor,line width= 0.4pt,line join=round,line cap=round,fill=fillColor] (357.79,219.49) circle (  1.16);

\path[draw=drawColor,line width= 0.4pt,line join=round,line cap=round,fill=fillColor] (360.26,217.59) circle (  1.16);

\path[draw=drawColor,line width= 0.4pt,line join=round,line cap=round,fill=fillColor] (362.53,216.32) circle (  1.16);

\path[draw=drawColor,line width= 0.4pt,line join=round,line cap=round,fill=fillColor] (364.64,216.16) circle (  1.16);

\path[draw=drawColor,line width= 0.4pt,line join=round,line cap=round,fill=fillColor] (366.61,215.96) circle (  1.16);

\path[draw=drawColor,line width= 0.4pt,line join=round,line cap=round,fill=fillColor] (368.46,215.94) circle (  1.16);

\path[draw=drawColor,line width= 0.4pt,line join=round,line cap=round,fill=fillColor] (370.22,215.80) circle (  1.16);

\path[draw=drawColor,line width= 0.4pt,line join=round,line cap=round,fill=fillColor] (371.89,215.68) circle (  1.16);

\path[draw=drawColor,line width= 0.4pt,line join=round,line cap=round,fill=fillColor] (373.47,215.25) circle (  1.16);

\path[draw=drawColor,line width= 0.4pt,line join=round,line cap=round,fill=fillColor] (374.99,214.75) circle (  1.16);

\path[draw=drawColor,line width= 0.4pt,line join=round,line cap=round,fill=fillColor] (376.45,214.55) circle (  1.16);

\path[draw=drawColor,line width= 0.4pt,line join=round,line cap=round,fill=fillColor] (377.85,214.37) circle (  1.16);

\path[draw=drawColor,line width= 0.4pt,line join=round,line cap=round,fill=fillColor] (379.21,214.07) circle (  1.16);

\path[draw=drawColor,line width= 0.4pt,line join=round,line cap=round,fill=fillColor] (380.51,213.91) circle (  1.16);

\path[draw=drawColor,line width= 0.4pt,line join=round,line cap=round,fill=fillColor] (381.77,213.53) circle (  1.16);

\path[draw=drawColor,line width= 0.4pt,line join=round,line cap=round,fill=fillColor] (382.99,213.39) circle (  1.16);

\path[draw=drawColor,line width= 0.4pt,line join=round,line cap=round,fill=fillColor] (384.18,213.25) circle (  1.16);

\path[draw=drawColor,line width= 0.4pt,line join=round,line cap=round,fill=fillColor] (385.33,212.87) circle (  1.16);

\path[draw=drawColor,line width= 0.4pt,line join=round,line cap=round,fill=fillColor] (386.45,212.85) circle (  1.16);

\path[draw=drawColor,line width= 0.4pt,line join=round,line cap=round,fill=fillColor] (387.54,212.77) circle (  1.16);

\path[draw=drawColor,line width= 0.4pt,line join=round,line cap=round,fill=fillColor] (388.60,212.55) circle (  1.16);

\path[draw=drawColor,line width= 0.4pt,line join=round,line cap=round,fill=fillColor] (389.64,212.28) circle (  1.16);

\path[draw=drawColor,line width= 0.4pt,line join=round,line cap=round,fill=fillColor] (390.65,211.98) circle (  1.16);

\path[draw=drawColor,line width= 0.4pt,line join=round,line cap=round,fill=fillColor] (391.64,211.97) circle (  1.16);

\path[draw=drawColor,line width= 0.4pt,line join=round,line cap=round,fill=fillColor] (392.61,211.97) circle (  1.16);

\path[draw=drawColor,line width= 0.4pt,line join=round,line cap=round,fill=fillColor] (393.56,211.86) circle (  1.16);

\path[draw=drawColor,line width= 0.4pt,line join=round,line cap=round,fill=fillColor] (394.49,211.73) circle (  1.16);

\path[draw=drawColor,line width= 0.4pt,line join=round,line cap=round,fill=fillColor] (395.40,211.63) circle (  1.16);

\path[draw=drawColor,line width= 0.4pt,line join=round,line cap=round,fill=fillColor] (396.29,211.59) circle (  1.16);

\path[draw=drawColor,line width= 0.4pt,line join=round,line cap=round,fill=fillColor] (397.16,211.49) circle (  1.16);

\path[draw=drawColor,line width= 0.4pt,line join=round,line cap=round,fill=fillColor] (398.02,211.46) circle (  1.16);

\path[draw=drawColor,line width= 0.4pt,line join=round,line cap=round,fill=fillColor] (398.86,211.31) circle (  1.16);

\path[draw=drawColor,line width= 0.4pt,line join=round,line cap=round,fill=fillColor] (399.69,210.75) circle (  1.16);

\path[draw=drawColor,line width= 0.4pt,line join=round,line cap=round,fill=fillColor] (400.51,210.72) circle (  1.16);

\path[draw=drawColor,line width= 0.4pt,line join=round,line cap=round,fill=fillColor] (401.31,210.70) circle (  1.16);

\path[draw=drawColor,line width= 0.4pt,line join=round,line cap=round,fill=fillColor] (402.10,210.63) circle (  1.16);

\path[draw=drawColor,line width= 0.4pt,line join=round,line cap=round,fill=fillColor] (402.87,210.63) circle (  1.16);

\path[draw=drawColor,line width= 0.4pt,line join=round,line cap=round,fill=fillColor] (403.64,210.59) circle (  1.16);

\path[draw=drawColor,line width= 0.4pt,line join=round,line cap=round,fill=fillColor] (404.39,210.58) circle (  1.16);

\path[draw=drawColor,line width= 0.4pt,line join=round,line cap=round,fill=fillColor] (405.13,210.56) circle (  1.16);

\path[draw=drawColor,line width= 0.4pt,line join=round,line cap=round,fill=fillColor] (405.86,210.52) circle (  1.16);

\path[draw=drawColor,line width= 0.4pt,line join=round,line cap=round,fill=fillColor] (406.58,210.48) circle (  1.16);

\path[draw=drawColor,line width= 0.4pt,line join=round,line cap=round,fill=fillColor] (407.29,210.48) circle (  1.16);

\path[draw=drawColor,line width= 0.4pt,line join=round,line cap=round,fill=fillColor] (407.99,210.29) circle (  1.16);

\path[draw=drawColor,line width= 0.4pt,line join=round,line cap=round,fill=fillColor] (408.68,210.17) circle (  1.16);

\path[draw=drawColor,line width= 0.4pt,line join=round,line cap=round,fill=fillColor] (409.37,209.98) circle (  1.16);

\path[draw=drawColor,line width= 0.4pt,line join=round,line cap=round,fill=fillColor] (410.04,209.87) circle (  1.16);

\path[draw=drawColor,line width= 0.4pt,line join=round,line cap=round,fill=fillColor] (410.71,209.64) circle (  1.16);

\path[draw=drawColor,line width= 0.4pt,line join=round,line cap=round,fill=fillColor] (411.36,209.61) circle (  1.16);

\path[draw=drawColor,line width= 0.4pt,line join=round,line cap=round,fill=fillColor] (412.01,209.60) circle (  1.16);

\path[draw=drawColor,line width= 0.4pt,line join=round,line cap=round,fill=fillColor] (412.65,209.59) circle (  1.16);

\path[draw=drawColor,line width= 0.4pt,line join=round,line cap=round,fill=fillColor] (413.29,209.50) circle (  1.16);

\path[draw=drawColor,line width= 0.4pt,line join=round,line cap=round,fill=fillColor] (413.92,209.41) circle (  1.16);

\path[draw=drawColor,line width= 0.4pt,line join=round,line cap=round,fill=fillColor] (414.54,209.41) circle (  1.16);

\path[draw=drawColor,line width= 0.4pt,line join=round,line cap=round,fill=fillColor] (415.15,209.34) circle (  1.16);

\path[draw=drawColor,line width= 0.4pt,line join=round,line cap=round,fill=fillColor] (415.76,209.16) circle (  1.16);

\path[draw=drawColor,line width= 0.4pt,line join=round,line cap=round,fill=fillColor] (416.36,209.07) circle (  1.16);

\path[draw=drawColor,line width= 0.4pt,line join=round,line cap=round,fill=fillColor] (416.95,209.05) circle (  1.16);

\path[draw=drawColor,line width= 0.4pt,line join=round,line cap=round,fill=fillColor] (417.54,209.04) circle (  1.16);

\path[draw=drawColor,line width= 0.4pt,line join=round,line cap=round,fill=fillColor] (418.12,208.75) circle (  1.16);

\path[draw=drawColor,line width= 0.4pt,line join=round,line cap=round,fill=fillColor] (418.69,208.64) circle (  1.16);

\path[draw=drawColor,line width= 0.4pt,line join=round,line cap=round,fill=fillColor] (419.26,208.35) circle (  1.16);

\path[draw=drawColor,line width= 0.4pt,line join=round,line cap=round,fill=fillColor] (419.83,208.25) circle (  1.16);

\path[draw=drawColor,line width= 0.4pt,line join=round,line cap=round,fill=fillColor] (420.39,208.20) circle (  1.16);

\path[draw=drawColor,line width= 0.4pt,line join=round,line cap=round,fill=fillColor] (420.94,208.07) circle (  1.16);

\path[draw=drawColor,line width= 0.4pt,line join=round,line cap=round,fill=fillColor] (421.49,208.05) circle (  1.16);

\path[draw=drawColor,line width= 0.4pt,line join=round,line cap=round,fill=fillColor] (422.03,208.00) circle (  1.16);

\path[draw=drawColor,line width= 0.4pt,line join=round,line cap=round,fill=fillColor] (422.57,207.96) circle (  1.16);

\path[draw=drawColor,line width= 0.4pt,line join=round,line cap=round,fill=fillColor] (423.10,207.87) circle (  1.16);

\path[draw=drawColor,line width= 0.4pt,line join=round,line cap=round,fill=fillColor] (423.63,207.75) circle (  1.16);

\path[draw=drawColor,line width= 0.4pt,line join=round,line cap=round,fill=fillColor] (424.16,207.70) circle (  1.16);

\path[draw=drawColor,line width= 0.4pt,line join=round,line cap=round,fill=fillColor] (424.68,207.64) circle (  1.16);

\path[draw=drawColor,line width= 0.4pt,line join=round,line cap=round,fill=fillColor] (425.19,207.60) circle (  1.16);

\path[draw=drawColor,line width= 0.4pt,line join=round,line cap=round,fill=fillColor] (425.70,207.55) circle (  1.16);

\path[draw=drawColor,line width= 0.4pt,line join=round,line cap=round,fill=fillColor] (426.21,207.55) circle (  1.16);

\path[draw=drawColor,line width= 0.4pt,line join=round,line cap=round,fill=fillColor] (426.71,207.54) circle (  1.16);

\path[draw=drawColor,line width= 0.4pt,line join=round,line cap=round,fill=fillColor] (427.21,207.46) circle (  1.16);

\path[draw=drawColor,line width= 0.4pt,line join=round,line cap=round,fill=fillColor] (427.71,207.46) circle (  1.16);

\path[draw=drawColor,line width= 0.4pt,line join=round,line cap=round,fill=fillColor] (428.20,207.45) circle (  1.16);

\path[draw=drawColor,line width= 0.4pt,line join=round,line cap=round,fill=fillColor] (428.68,207.30) circle (  1.16);

\path[draw=drawColor,line width= 0.4pt,line join=round,line cap=round,fill=fillColor] (429.17,207.24) circle (  1.16);

\path[draw=drawColor,line width= 0.4pt,line join=round,line cap=round,fill=fillColor] (429.65,207.19) circle (  1.16);

\path[draw=drawColor,line width= 0.4pt,line join=round,line cap=round,fill=fillColor] (430.12,207.17) circle (  1.16);

\path[draw=drawColor,line width= 0.4pt,line join=round,line cap=round,fill=fillColor] (430.59,207.16) circle (  1.16);

\path[draw=drawColor,line width= 0.4pt,line join=round,line cap=round,fill=fillColor] (431.06,207.11) circle (  1.16);

\path[draw=drawColor,line width= 0.4pt,line join=round,line cap=round,fill=fillColor] (431.53,207.11) circle (  1.16);

\path[draw=drawColor,line width= 0.4pt,line join=round,line cap=round,fill=fillColor] (431.99,207.06) circle (  1.16);

\path[draw=drawColor,line width= 0.4pt,line join=round,line cap=round,fill=fillColor] (432.45,207.01) circle (  1.16);

\path[draw=drawColor,line width= 0.4pt,line join=round,line cap=round,fill=fillColor] (432.90,207.00) circle (  1.16);

\path[draw=drawColor,line width= 0.4pt,line join=round,line cap=round,fill=fillColor] (433.36,206.99) circle (  1.16);

\path[draw=drawColor,line width= 0.4pt,line join=round,line cap=round,fill=fillColor] (433.81,206.99) circle (  1.16);

\path[draw=drawColor,line width= 0.4pt,line join=round,line cap=round,fill=fillColor] (434.25,206.93) circle (  1.16);

\path[draw=drawColor,line width= 0.4pt,line join=round,line cap=round,fill=fillColor] (434.69,206.93) circle (  1.16);

\path[draw=drawColor,line width= 0.4pt,line join=round,line cap=round,fill=fillColor] (435.13,206.40) circle (  1.16);

\path[draw=drawColor,line width= 0.4pt,line join=round,line cap=round,fill=fillColor] (435.57,206.31) circle (  1.16);

\path[draw=drawColor,line width= 0.4pt,line join=round,line cap=round,fill=fillColor] (436.01,206.19) circle (  1.16);

\path[draw=drawColor,line width= 0.4pt,line join=round,line cap=round,fill=fillColor] (436.44,206.15) circle (  1.16);

\path[draw=drawColor,line width= 0.4pt,line join=round,line cap=round,fill=fillColor] (436.87,206.12) circle (  1.16);

\path[draw=drawColor,line width= 0.4pt,line join=round,line cap=round,fill=fillColor] (437.29,205.99) circle (  1.16);

\path[draw=drawColor,line width= 0.4pt,line join=round,line cap=round,fill=fillColor] (437.71,205.85) circle (  1.16);

\path[draw=drawColor,line width= 0.4pt,line join=round,line cap=round,fill=fillColor] (438.14,205.58) circle (  1.16);

\path[draw=drawColor,line width= 0.4pt,line join=round,line cap=round,fill=fillColor] (438.55,205.34) circle (  1.16);

\path[draw=drawColor,line width= 0.4pt,line join=round,line cap=round,fill=fillColor] (438.97,205.16) circle (  1.16);

\path[draw=drawColor,line width= 0.4pt,line join=round,line cap=round,fill=fillColor] (439.38,205.12) circle (  1.16);

\path[draw=drawColor,line width= 0.4pt,line join=round,line cap=round,fill=fillColor] (439.79,205.08) circle (  1.16);

\path[draw=drawColor,line width= 0.4pt,line join=round,line cap=round,fill=fillColor] (440.20,204.92) circle (  1.16);

\path[draw=drawColor,line width= 0.4pt,line join=round,line cap=round,fill=fillColor] (440.60,204.91) circle (  1.16);

\path[draw=drawColor,line width= 0.4pt,line join=round,line cap=round,fill=fillColor] (441.01,204.62) circle (  1.16);

\path[draw=drawColor,line width= 0.4pt,line join=round,line cap=round,fill=fillColor] (441.41,204.37) circle (  1.16);

\path[draw=drawColor,line width= 0.4pt,line join=round,line cap=round,fill=fillColor] (441.81,204.23) circle (  1.16);

\path[draw=drawColor,line width= 0.4pt,line join=round,line cap=round,fill=fillColor] (442.20,203.80) circle (  1.16);

\path[draw=drawColor,line width= 0.4pt,line join=round,line cap=round,fill=fillColor] (442.60,203.79) circle (  1.16);

\path[draw=drawColor,line width= 0.4pt,line join=round,line cap=round,fill=fillColor] (442.99,203.73) circle (  1.16);

\path[draw=drawColor,line width= 0.4pt,line join=round,line cap=round,fill=fillColor] (443.38,203.71) circle (  1.16);

\path[draw=drawColor,line width= 0.4pt,line join=round,line cap=round,fill=fillColor] (443.77,203.55) circle (  1.16);

\path[draw=drawColor,line width= 0.4pt,line join=round,line cap=round,fill=fillColor] (444.15,203.52) circle (  1.16);

\path[draw=drawColor,line width= 0.4pt,line join=round,line cap=round,fill=fillColor] (444.53,203.45) circle (  1.16);

\path[draw=drawColor,line width= 0.4pt,line join=round,line cap=round,fill=fillColor] (444.91,202.98) circle (  1.16);

\path[draw=drawColor,line width= 0.4pt,line join=round,line cap=round,fill=fillColor] (445.29,202.86) circle (  1.16);

\path[draw=drawColor,line width= 0.4pt,line join=round,line cap=round,fill=fillColor] (445.67,202.74) circle (  1.16);

\path[draw=drawColor,line width= 0.4pt,line join=round,line cap=round,fill=fillColor] (446.05,201.80) circle (  1.16);

\path[draw=drawColor,line width= 0.4pt,line join=round,line cap=round,fill=fillColor] (446.42,201.40) circle (  1.16);

\path[draw=drawColor,line width= 0.4pt,line join=round,line cap=round,fill=fillColor] (446.79,198.35) circle (  1.16);

\path[draw=drawColor,line width= 0.4pt,line join=round,line cap=round,fill=fillColor] (447.16,198.35) circle (  1.16);

\path[draw=drawColor,line width= 0.4pt,line join=round,line cap=round,fill=fillColor] (447.53,198.35) circle (  1.16);

\path[draw=drawColor,line width= 0.4pt,line join=round,line cap=round,fill=fillColor] (447.89,198.35) circle (  1.16);

\path[draw=drawColor,line width= 0.4pt,line join=round,line cap=round,fill=fillColor] (448.25,198.35) circle (  1.16);

\path[draw=drawColor,line width= 0.4pt,line join=round,line cap=round,fill=fillColor] (448.62,198.35) circle (  1.16);

\path[draw=drawColor,line width= 0.4pt,line join=round,line cap=round,fill=fillColor] (448.98,198.35) circle (  1.16);

\path[draw=drawColor,line width= 0.4pt,line join=round,line cap=round,fill=fillColor] (449.33,198.35) circle (  1.16);

\path[draw=drawColor,line width= 0.4pt,line join=round,line cap=round,fill=fillColor] (449.69,198.35) circle (  1.16);

\path[draw=drawColor,line width= 0.4pt,line join=round,line cap=round,fill=fillColor] (450.05,198.35) circle (  1.16);

\path[draw=drawColor,line width= 0.4pt,line join=round,line cap=round,fill=fillColor] (450.40,198.35) circle (  1.16);

\path[draw=drawColor,line width= 0.4pt,line join=round,line cap=round,fill=fillColor] (450.75,198.35) circle (  1.16);

\path[draw=drawColor,line width= 0.4pt,line join=round,line cap=round,fill=fillColor] (451.10,198.35) circle (  1.16);

\path[draw=drawColor,line width= 0.4pt,line join=round,line cap=round,fill=fillColor] (451.45,198.35) circle (  1.16);

\path[draw=drawColor,line width= 0.4pt,line join=round,line cap=round,fill=fillColor] (451.79,198.35) circle (  1.16);

\path[draw=drawColor,line width= 0.4pt,line join=round,line cap=round,fill=fillColor] (452.14,198.35) circle (  1.16);

\path[draw=drawColor,line width= 0.4pt,line join=round,line cap=round,fill=fillColor] (452.48,198.35) circle (  1.16);

\path[draw=drawColor,line width= 0.4pt,line join=round,line cap=round,fill=fillColor] (452.82,198.35) circle (  1.16);

\path[draw=drawColor,line width= 0.4pt,line join=round,line cap=round,fill=fillColor] (453.16,198.35) circle (  1.16);

\path[draw=drawColor,line width= 0.4pt,line join=round,line cap=round,fill=fillColor] (453.50,198.35) circle (  1.16);

\path[draw=drawColor,line width= 0.4pt,line join=round,line cap=round,fill=fillColor] (453.84,198.35) circle (  1.16);

\path[draw=drawColor,line width= 0.4pt,line join=round,line cap=round,fill=fillColor] (454.17,198.35) circle (  1.16);

\path[draw=drawColor,line width= 0.4pt,line join=round,line cap=round,fill=fillColor] (454.51,198.35) circle (  1.16);

\path[draw=drawColor,line width= 0.4pt,line join=round,line cap=round,fill=fillColor] (454.84,198.35) circle (  1.16);

\path[draw=drawColor,line width= 0.4pt,line join=round,line cap=round,fill=fillColor] (455.17,198.35) circle (  1.16);

\path[draw=drawColor,line width= 0.4pt,line join=round,line cap=round,fill=fillColor] (455.50,198.35) circle (  1.16);

\path[draw=drawColor,line width= 0.4pt,line join=round,line cap=round,fill=fillColor] (455.83,198.35) circle (  1.16);

\path[draw=drawColor,line width= 0.4pt,line join=round,line cap=round,fill=fillColor] (456.16,198.35) circle (  1.16);

\path[draw=drawColor,line width= 0.4pt,line join=round,line cap=round,fill=fillColor] (456.48,198.35) circle (  1.16);

\path[draw=drawColor,line width= 0.4pt,line join=round,line cap=round,fill=fillColor] (456.80,198.35) circle (  1.16);

\path[draw=drawColor,line width= 0.4pt,line join=round,line cap=round,fill=fillColor] (457.13,198.35) circle (  1.16);

\path[draw=drawColor,line width= 0.4pt,line join=round,line cap=round,fill=fillColor] (457.45,198.35) circle (  1.16);

\path[draw=drawColor,line width= 0.4pt,line join=round,line cap=round,fill=fillColor] (457.77,198.35) circle (  1.16);

\path[draw=drawColor,line width= 0.4pt,line join=round,line cap=round,fill=fillColor] (458.09,198.35) circle (  1.16);

\path[draw=drawColor,line width= 0.4pt,line join=round,line cap=round,fill=fillColor] (458.40,198.35) circle (  1.16);

\path[draw=drawColor,line width= 0.4pt,line join=round,line cap=round,fill=fillColor] (458.72,198.35) circle (  1.16);

\path[draw=drawColor,line width= 0.4pt,line join=round,line cap=round,fill=fillColor] (459.03,198.35) circle (  1.16);

\path[draw=drawColor,line width= 0.4pt,line join=round,line cap=round,fill=fillColor] (459.35,198.35) circle (  1.16);

\path[draw=drawColor,line width= 0.4pt,line join=round,line cap=round,fill=fillColor] (459.66,198.35) circle (  1.16);

\path[draw=drawColor,line width= 0.4pt,line join=round,line cap=round,fill=fillColor] (459.97,198.35) circle (  1.16);

\path[draw=drawColor,line width= 0.4pt,line join=round,line cap=round,fill=fillColor] (460.28,198.35) circle (  1.16);

\path[draw=drawColor,line width= 0.4pt,line join=round,line cap=round,fill=fillColor] (460.59,198.35) circle (  1.16);

\path[draw=drawColor,line width= 0.4pt,line join=round,line cap=round,fill=fillColor] (460.90,198.35) circle (  1.16);

\path[draw=drawColor,line width= 0.4pt,line join=round,line cap=round,fill=fillColor] (461.20,198.35) circle (  1.16);

\path[draw=drawColor,line width= 0.4pt,line join=round,line cap=round,fill=fillColor] (461.51,198.35) circle (  1.16);

\path[draw=drawColor,line width= 0.4pt,line join=round,line cap=round,fill=fillColor] (461.81,198.35) circle (  1.16);

\path[draw=drawColor,line width= 0.4pt,line join=round,line cap=round,fill=fillColor] (462.11,198.35) circle (  1.16);

\path[draw=drawColor,line width= 0.4pt,line join=round,line cap=round,fill=fillColor] (462.42,198.35) circle (  1.16);

\path[draw=drawColor,line width= 0.4pt,line join=round,line cap=round,fill=fillColor] (462.72,198.35) circle (  1.16);

\path[draw=drawColor,line width= 0.4pt,line join=round,line cap=round,fill=fillColor] (463.01,198.35) circle (  1.16);

\path[draw=drawColor,line width= 0.4pt,line join=round,line cap=round,fill=fillColor] (463.31,198.35) circle (  1.16);

\path[draw=drawColor,line width= 0.4pt,line join=round,line cap=round,fill=fillColor] (463.61,198.35) circle (  1.16);

\path[draw=drawColor,line width= 0.4pt,line join=round,line cap=round,fill=fillColor] (463.91,198.35) circle (  1.16);

\path[draw=drawColor,line width= 0.4pt,line join=round,line cap=round,fill=fillColor] (464.20,198.35) circle (  1.16);

\path[draw=drawColor,line width= 0.4pt,line join=round,line cap=round,fill=fillColor] (464.49,198.35) circle (  1.16);

\path[draw=drawColor,line width= 0.4pt,line join=round,line cap=round,fill=fillColor] (464.79,198.35) circle (  1.16);

\path[draw=drawColor,line width= 0.4pt,line join=round,line cap=round,fill=fillColor] (465.08,198.35) circle (  1.16);

\path[draw=drawColor,line width= 0.4pt,line join=round,line cap=round,fill=fillColor] (465.37,198.35) circle (  1.16);

\path[draw=drawColor,line width= 0.4pt,line join=round,line cap=round,fill=fillColor] (465.66,198.35) circle (  1.16);

\path[draw=drawColor,line width= 0.4pt,line join=round,line cap=round,fill=fillColor] (465.95,198.35) circle (  1.16);

\path[draw=drawColor,line width= 0.4pt,line join=round,line cap=round,fill=fillColor] (466.23,198.35) circle (  1.16);

\path[draw=drawColor,line width= 0.4pt,line join=round,line cap=round,fill=fillColor] (466.52,198.35) circle (  1.16);

\path[draw=drawColor,line width= 0.4pt,line join=round,line cap=round,fill=fillColor] (466.81,198.35) circle (  1.16);

\path[draw=drawColor,line width= 0.4pt,line join=round,line cap=round,fill=fillColor] (467.09,198.35) circle (  1.16);

\path[draw=drawColor,line width= 0.4pt,line join=round,line cap=round,fill=fillColor] (467.37,198.35) circle (  1.16);

\path[draw=drawColor,line width= 0.4pt,line join=round,line cap=round,fill=fillColor] (467.66,198.35) circle (  1.16);

\path[draw=drawColor,line width= 0.4pt,line join=round,line cap=round,fill=fillColor] (467.94,198.35) circle (  1.16);

\path[draw=drawColor,line width= 0.4pt,line join=round,line cap=round,fill=fillColor] (468.22,198.35) circle (  1.16);

\path[draw=drawColor,line width= 0.4pt,line join=round,line cap=round,fill=fillColor] (468.50,198.35) circle (  1.16);

\path[draw=drawColor,line width= 0.4pt,line join=round,line cap=round,fill=fillColor] (468.78,198.35) circle (  1.16);

\path[draw=drawColor,line width= 0.4pt,line join=round,line cap=round,fill=fillColor] (469.05,198.35) circle (  1.16);

\path[draw=drawColor,line width= 0.4pt,line join=round,line cap=round,fill=fillColor] (469.33,198.35) circle (  1.16);

\path[draw=drawColor,line width= 0.4pt,line join=round,line cap=round,fill=fillColor] (469.61,198.35) circle (  1.16);

\path[draw=drawColor,line width= 0.4pt,line join=round,line cap=round,fill=fillColor] (469.88,198.35) circle (  1.16);

\path[draw=drawColor,line width= 0.4pt,line join=round,line cap=round,fill=fillColor] (470.15,198.35) circle (  1.16);

\path[draw=drawColor,line width= 0.4pt,line join=round,line cap=round,fill=fillColor] (470.43,198.35) circle (  1.16);

\path[draw=drawColor,line width= 0.4pt,line join=round,line cap=round,fill=fillColor] (470.70,198.35) circle (  1.16);

\path[draw=drawColor,line width= 0.4pt,line join=round,line cap=round,fill=fillColor] (470.97,198.35) circle (  1.16);

\path[draw=drawColor,line width= 0.4pt,line join=round,line cap=round,fill=fillColor] (471.24,198.35) circle (  1.16);

\path[draw=drawColor,line width= 0.4pt,line join=round,line cap=round,fill=fillColor] (471.51,198.35) circle (  1.16);

\path[draw=drawColor,line width= 0.4pt,line join=round,line cap=round,fill=fillColor] (471.78,198.35) circle (  1.16);

\path[draw=drawColor,line width= 0.4pt,line join=round,line cap=round,fill=fillColor] (472.05,198.35) circle (  1.16);
\definecolor[named]{drawColor}{rgb}{0.60,0.31,0.64}
\definecolor[named]{fillColor}{rgb}{0.60,0.31,0.64}

\path[draw=drawColor,line width= 0.4pt,line join=round,line cap=round,fill=fillColor] (331.92,223.54) circle (  1.16);

\path[draw=drawColor,line width= 0.4pt,line join=round,line cap=round,fill=fillColor] (339.28,221.38) circle (  1.16);

\path[draw=drawColor,line width= 0.4pt,line join=round,line cap=round,fill=fillColor] (344.45,221.13) circle (  1.16);

\path[draw=drawColor,line width= 0.4pt,line join=round,line cap=round,fill=fillColor] (348.57,220.93) circle (  1.16);

\path[draw=drawColor,line width= 0.4pt,line join=round,line cap=round,fill=fillColor] (352.04,220.44) circle (  1.16);

\path[draw=drawColor,line width= 0.4pt,line join=round,line cap=round,fill=fillColor] (355.08,219.33) circle (  1.16);

\path[draw=drawColor,line width= 0.4pt,line join=round,line cap=round,fill=fillColor] (357.79,218.93) circle (  1.16);

\path[draw=drawColor,line width= 0.4pt,line join=round,line cap=round,fill=fillColor] (360.26,218.81) circle (  1.16);

\path[draw=drawColor,line width= 0.4pt,line join=round,line cap=round,fill=fillColor] (362.53,218.28) circle (  1.16);

\path[draw=drawColor,line width= 0.4pt,line join=round,line cap=round,fill=fillColor] (364.64,217.07) circle (  1.16);

\path[draw=drawColor,line width= 0.4pt,line join=round,line cap=round,fill=fillColor] (366.61,216.94) circle (  1.16);

\path[draw=drawColor,line width= 0.4pt,line join=round,line cap=round,fill=fillColor] (368.46,216.51) circle (  1.16);

\path[draw=drawColor,line width= 0.4pt,line join=round,line cap=round,fill=fillColor] (370.22,216.44) circle (  1.16);

\path[draw=drawColor,line width= 0.4pt,line join=round,line cap=round,fill=fillColor] (371.89,215.91) circle (  1.16);

\path[draw=drawColor,line width= 0.4pt,line join=round,line cap=round,fill=fillColor] (373.47,215.86) circle (  1.16);

\path[draw=drawColor,line width= 0.4pt,line join=round,line cap=round,fill=fillColor] (374.99,215.83) circle (  1.16);

\path[draw=drawColor,line width= 0.4pt,line join=round,line cap=round,fill=fillColor] (376.45,215.52) circle (  1.16);

\path[draw=drawColor,line width= 0.4pt,line join=round,line cap=round,fill=fillColor] (377.85,215.28) circle (  1.16);

\path[draw=drawColor,line width= 0.4pt,line join=round,line cap=round,fill=fillColor] (379.21,215.02) circle (  1.16);

\path[draw=drawColor,line width= 0.4pt,line join=round,line cap=round,fill=fillColor] (380.51,214.87) circle (  1.16);

\path[draw=drawColor,line width= 0.4pt,line join=round,line cap=round,fill=fillColor] (381.77,214.66) circle (  1.16);

\path[draw=drawColor,line width= 0.4pt,line join=round,line cap=round,fill=fillColor] (382.99,214.35) circle (  1.16);

\path[draw=drawColor,line width= 0.4pt,line join=round,line cap=round,fill=fillColor] (384.18,214.00) circle (  1.16);

\path[draw=drawColor,line width= 0.4pt,line join=round,line cap=round,fill=fillColor] (385.33,213.75) circle (  1.16);

\path[draw=drawColor,line width= 0.4pt,line join=round,line cap=round,fill=fillColor] (386.45,213.47) circle (  1.16);

\path[draw=drawColor,line width= 0.4pt,line join=round,line cap=round,fill=fillColor] (387.54,213.44) circle (  1.16);

\path[draw=drawColor,line width= 0.4pt,line join=round,line cap=round,fill=fillColor] (388.60,213.20) circle (  1.16);

\path[draw=drawColor,line width= 0.4pt,line join=round,line cap=round,fill=fillColor] (389.64,212.94) circle (  1.16);

\path[draw=drawColor,line width= 0.4pt,line join=round,line cap=round,fill=fillColor] (390.65,212.91) circle (  1.16);

\path[draw=drawColor,line width= 0.4pt,line join=round,line cap=round,fill=fillColor] (391.64,212.81) circle (  1.16);

\path[draw=drawColor,line width= 0.4pt,line join=round,line cap=round,fill=fillColor] (392.61,212.78) circle (  1.16);

\path[draw=drawColor,line width= 0.4pt,line join=round,line cap=round,fill=fillColor] (393.56,212.70) circle (  1.16);

\path[draw=drawColor,line width= 0.4pt,line join=round,line cap=round,fill=fillColor] (394.49,212.60) circle (  1.16);

\path[draw=drawColor,line width= 0.4pt,line join=round,line cap=round,fill=fillColor] (395.40,212.37) circle (  1.16);

\path[draw=drawColor,line width= 0.4pt,line join=round,line cap=round,fill=fillColor] (396.29,212.28) circle (  1.16);

\path[draw=drawColor,line width= 0.4pt,line join=round,line cap=round,fill=fillColor] (397.16,212.20) circle (  1.16);

\path[draw=drawColor,line width= 0.4pt,line join=round,line cap=round,fill=fillColor] (398.02,212.17) circle (  1.16);

\path[draw=drawColor,line width= 0.4pt,line join=round,line cap=round,fill=fillColor] (398.86,212.07) circle (  1.16);

\path[draw=drawColor,line width= 0.4pt,line join=round,line cap=round,fill=fillColor] (399.69,212.00) circle (  1.16);

\path[draw=drawColor,line width= 0.4pt,line join=round,line cap=round,fill=fillColor] (400.51,211.94) circle (  1.16);

\path[draw=drawColor,line width= 0.4pt,line join=round,line cap=round,fill=fillColor] (401.31,211.86) circle (  1.16);

\path[draw=drawColor,line width= 0.4pt,line join=round,line cap=round,fill=fillColor] (402.10,211.67) circle (  1.16);

\path[draw=drawColor,line width= 0.4pt,line join=round,line cap=round,fill=fillColor] (402.87,211.60) circle (  1.16);

\path[draw=drawColor,line width= 0.4pt,line join=round,line cap=round,fill=fillColor] (403.64,211.54) circle (  1.16);

\path[draw=drawColor,line width= 0.4pt,line join=round,line cap=round,fill=fillColor] (404.39,211.39) circle (  1.16);

\path[draw=drawColor,line width= 0.4pt,line join=round,line cap=round,fill=fillColor] (405.13,211.39) circle (  1.16);

\path[draw=drawColor,line width= 0.4pt,line join=round,line cap=round,fill=fillColor] (405.86,211.29) circle (  1.16);

\path[draw=drawColor,line width= 0.4pt,line join=round,line cap=round,fill=fillColor] (406.58,211.06) circle (  1.16);

\path[draw=drawColor,line width= 0.4pt,line join=round,line cap=round,fill=fillColor] (407.29,210.92) circle (  1.16);

\path[draw=drawColor,line width= 0.4pt,line join=round,line cap=round,fill=fillColor] (407.99,210.83) circle (  1.16);

\path[draw=drawColor,line width= 0.4pt,line join=round,line cap=round,fill=fillColor] (408.68,210.80) circle (  1.16);

\path[draw=drawColor,line width= 0.4pt,line join=round,line cap=round,fill=fillColor] (409.37,210.69) circle (  1.16);

\path[draw=drawColor,line width= 0.4pt,line join=round,line cap=round,fill=fillColor] (410.04,210.54) circle (  1.16);

\path[draw=drawColor,line width= 0.4pt,line join=round,line cap=round,fill=fillColor] (410.71,210.53) circle (  1.16);

\path[draw=drawColor,line width= 0.4pt,line join=round,line cap=round,fill=fillColor] (411.36,210.52) circle (  1.16);

\path[draw=drawColor,line width= 0.4pt,line join=round,line cap=round,fill=fillColor] (412.01,210.27) circle (  1.16);

\path[draw=drawColor,line width= 0.4pt,line join=round,line cap=round,fill=fillColor] (412.65,210.23) circle (  1.16);

\path[draw=drawColor,line width= 0.4pt,line join=round,line cap=round,fill=fillColor] (413.29,210.17) circle (  1.16);

\path[draw=drawColor,line width= 0.4pt,line join=round,line cap=round,fill=fillColor] (413.92,210.07) circle (  1.16);

\path[draw=drawColor,line width= 0.4pt,line join=round,line cap=round,fill=fillColor] (414.54,209.92) circle (  1.16);

\path[draw=drawColor,line width= 0.4pt,line join=round,line cap=round,fill=fillColor] (415.15,209.85) circle (  1.16);

\path[draw=drawColor,line width= 0.4pt,line join=round,line cap=round,fill=fillColor] (415.76,209.78) circle (  1.16);

\path[draw=drawColor,line width= 0.4pt,line join=round,line cap=round,fill=fillColor] (416.36,209.66) circle (  1.16);

\path[draw=drawColor,line width= 0.4pt,line join=round,line cap=round,fill=fillColor] (416.95,209.62) circle (  1.16);

\path[draw=drawColor,line width= 0.4pt,line join=round,line cap=round,fill=fillColor] (417.54,209.50) circle (  1.16);

\path[draw=drawColor,line width= 0.4pt,line join=round,line cap=round,fill=fillColor] (418.12,209.49) circle (  1.16);

\path[draw=drawColor,line width= 0.4pt,line join=round,line cap=round,fill=fillColor] (418.69,209.41) circle (  1.16);

\path[draw=drawColor,line width= 0.4pt,line join=round,line cap=round,fill=fillColor] (419.26,209.32) circle (  1.16);

\path[draw=drawColor,line width= 0.4pt,line join=round,line cap=round,fill=fillColor] (419.83,209.21) circle (  1.16);

\path[draw=drawColor,line width= 0.4pt,line join=round,line cap=round,fill=fillColor] (420.39,209.10) circle (  1.16);

\path[draw=drawColor,line width= 0.4pt,line join=round,line cap=round,fill=fillColor] (420.94,208.84) circle (  1.16);

\path[draw=drawColor,line width= 0.4pt,line join=round,line cap=round,fill=fillColor] (421.49,208.79) circle (  1.16);

\path[draw=drawColor,line width= 0.4pt,line join=round,line cap=round,fill=fillColor] (422.03,208.79) circle (  1.16);

\path[draw=drawColor,line width= 0.4pt,line join=round,line cap=round,fill=fillColor] (422.57,208.76) circle (  1.16);

\path[draw=drawColor,line width= 0.4pt,line join=round,line cap=round,fill=fillColor] (423.10,208.65) circle (  1.16);

\path[draw=drawColor,line width= 0.4pt,line join=round,line cap=round,fill=fillColor] (423.63,208.64) circle (  1.16);

\path[draw=drawColor,line width= 0.4pt,line join=round,line cap=round,fill=fillColor] (424.16,208.51) circle (  1.16);

\path[draw=drawColor,line width= 0.4pt,line join=round,line cap=round,fill=fillColor] (424.68,208.50) circle (  1.16);

\path[draw=drawColor,line width= 0.4pt,line join=round,line cap=round,fill=fillColor] (425.19,208.48) circle (  1.16);

\path[draw=drawColor,line width= 0.4pt,line join=round,line cap=round,fill=fillColor] (425.70,208.32) circle (  1.16);

\path[draw=drawColor,line width= 0.4pt,line join=round,line cap=round,fill=fillColor] (426.21,208.27) circle (  1.16);

\path[draw=drawColor,line width= 0.4pt,line join=round,line cap=round,fill=fillColor] (426.71,208.21) circle (  1.16);

\path[draw=drawColor,line width= 0.4pt,line join=round,line cap=round,fill=fillColor] (427.21,208.17) circle (  1.16);

\path[draw=drawColor,line width= 0.4pt,line join=round,line cap=round,fill=fillColor] (427.71,208.14) circle (  1.16);

\path[draw=drawColor,line width= 0.4pt,line join=round,line cap=round,fill=fillColor] (428.20,208.05) circle (  1.16);

\path[draw=drawColor,line width= 0.4pt,line join=round,line cap=round,fill=fillColor] (428.68,207.97) circle (  1.16);

\path[draw=drawColor,line width= 0.4pt,line join=round,line cap=round,fill=fillColor] (429.17,207.92) circle (  1.16);

\path[draw=drawColor,line width= 0.4pt,line join=round,line cap=round,fill=fillColor] (429.65,207.77) circle (  1.16);

\path[draw=drawColor,line width= 0.4pt,line join=round,line cap=round,fill=fillColor] (430.12,207.74) circle (  1.16);

\path[draw=drawColor,line width= 0.4pt,line join=round,line cap=round,fill=fillColor] (430.59,207.64) circle (  1.16);

\path[draw=drawColor,line width= 0.4pt,line join=round,line cap=round,fill=fillColor] (431.06,207.50) circle (  1.16);

\path[draw=drawColor,line width= 0.4pt,line join=round,line cap=round,fill=fillColor] (431.53,207.50) circle (  1.16);

\path[draw=drawColor,line width= 0.4pt,line join=round,line cap=round,fill=fillColor] (431.99,207.47) circle (  1.16);

\path[draw=drawColor,line width= 0.4pt,line join=round,line cap=round,fill=fillColor] (432.45,207.39) circle (  1.16);

\path[draw=drawColor,line width= 0.4pt,line join=round,line cap=round,fill=fillColor] (432.90,207.31) circle (  1.16);

\path[draw=drawColor,line width= 0.4pt,line join=round,line cap=round,fill=fillColor] (433.36,207.26) circle (  1.16);

\path[draw=drawColor,line width= 0.4pt,line join=round,line cap=round,fill=fillColor] (433.81,207.01) circle (  1.16);

\path[draw=drawColor,line width= 0.4pt,line join=round,line cap=round,fill=fillColor] (434.25,207.00) circle (  1.16);

\path[draw=drawColor,line width= 0.4pt,line join=round,line cap=round,fill=fillColor] (434.69,206.96) circle (  1.16);

\path[draw=drawColor,line width= 0.4pt,line join=round,line cap=round,fill=fillColor] (435.13,206.80) circle (  1.16);

\path[draw=drawColor,line width= 0.4pt,line join=round,line cap=round,fill=fillColor] (435.57,206.77) circle (  1.16);

\path[draw=drawColor,line width= 0.4pt,line join=round,line cap=round,fill=fillColor] (436.01,206.77) circle (  1.16);

\path[draw=drawColor,line width= 0.4pt,line join=round,line cap=round,fill=fillColor] (436.44,206.56) circle (  1.16);

\path[draw=drawColor,line width= 0.4pt,line join=round,line cap=round,fill=fillColor] (436.87,206.48) circle (  1.16);

\path[draw=drawColor,line width= 0.4pt,line join=round,line cap=round,fill=fillColor] (437.29,206.40) circle (  1.16);

\path[draw=drawColor,line width= 0.4pt,line join=round,line cap=round,fill=fillColor] (437.71,206.39) circle (  1.16);

\path[draw=drawColor,line width= 0.4pt,line join=round,line cap=round,fill=fillColor] (438.14,206.12) circle (  1.16);

\path[draw=drawColor,line width= 0.4pt,line join=round,line cap=round,fill=fillColor] (438.55,206.12) circle (  1.16);

\path[draw=drawColor,line width= 0.4pt,line join=round,line cap=round,fill=fillColor] (438.97,206.09) circle (  1.16);

\path[draw=drawColor,line width= 0.4pt,line join=round,line cap=round,fill=fillColor] (439.38,205.89) circle (  1.16);

\path[draw=drawColor,line width= 0.4pt,line join=round,line cap=round,fill=fillColor] (439.79,205.85) circle (  1.16);

\path[draw=drawColor,line width= 0.4pt,line join=round,line cap=round,fill=fillColor] (440.20,205.83) circle (  1.16);

\path[draw=drawColor,line width= 0.4pt,line join=round,line cap=round,fill=fillColor] (440.60,205.57) circle (  1.16);

\path[draw=drawColor,line width= 0.4pt,line join=round,line cap=round,fill=fillColor] (441.01,205.48) circle (  1.16);

\path[draw=drawColor,line width= 0.4pt,line join=round,line cap=round,fill=fillColor] (441.41,205.40) circle (  1.16);

\path[draw=drawColor,line width= 0.4pt,line join=round,line cap=round,fill=fillColor] (441.81,205.16) circle (  1.16);

\path[draw=drawColor,line width= 0.4pt,line join=round,line cap=round,fill=fillColor] (442.20,205.13) circle (  1.16);

\path[draw=drawColor,line width= 0.4pt,line join=round,line cap=round,fill=fillColor] (442.60,205.13) circle (  1.16);

\path[draw=drawColor,line width= 0.4pt,line join=round,line cap=round,fill=fillColor] (442.99,204.94) circle (  1.16);

\path[draw=drawColor,line width= 0.4pt,line join=round,line cap=round,fill=fillColor] (443.38,204.89) circle (  1.16);

\path[draw=drawColor,line width= 0.4pt,line join=round,line cap=round,fill=fillColor] (443.77,204.53) circle (  1.16);

\path[draw=drawColor,line width= 0.4pt,line join=round,line cap=round,fill=fillColor] (444.15,204.36) circle (  1.16);

\path[draw=drawColor,line width= 0.4pt,line join=round,line cap=round,fill=fillColor] (444.53,204.16) circle (  1.16);

\path[draw=drawColor,line width= 0.4pt,line join=round,line cap=round,fill=fillColor] (444.91,204.12) circle (  1.16);

\path[draw=drawColor,line width= 0.4pt,line join=round,line cap=round,fill=fillColor] (445.29,203.95) circle (  1.16);

\path[draw=drawColor,line width= 0.4pt,line join=round,line cap=round,fill=fillColor] (445.67,203.89) circle (  1.16);

\path[draw=drawColor,line width= 0.4pt,line join=round,line cap=round,fill=fillColor] (446.05,203.57) circle (  1.16);

\path[draw=drawColor,line width= 0.4pt,line join=round,line cap=round,fill=fillColor] (446.42,203.45) circle (  1.16);

\path[draw=drawColor,line width= 0.4pt,line join=round,line cap=round,fill=fillColor] (446.79,203.15) circle (  1.16);

\path[draw=drawColor,line width= 0.4pt,line join=round,line cap=round,fill=fillColor] (447.16,202.86) circle (  1.16);

\path[draw=drawColor,line width= 0.4pt,line join=round,line cap=round,fill=fillColor] (447.53,202.30) circle (  1.16);

\path[draw=drawColor,line width= 0.4pt,line join=round,line cap=round,fill=fillColor] (447.89,202.26) circle (  1.16);

\path[draw=drawColor,line width= 0.4pt,line join=round,line cap=round,fill=fillColor] (448.25,201.40) circle (  1.16);

\path[draw=drawColor,line width= 0.4pt,line join=round,line cap=round,fill=fillColor] (448.62,198.35) circle (  1.16);

\path[draw=drawColor,line width= 0.4pt,line join=round,line cap=round,fill=fillColor] (448.98,198.35) circle (  1.16);

\path[draw=drawColor,line width= 0.4pt,line join=round,line cap=round,fill=fillColor] (449.33,198.35) circle (  1.16);

\path[draw=drawColor,line width= 0.4pt,line join=round,line cap=round,fill=fillColor] (449.69,198.35) circle (  1.16);

\path[draw=drawColor,line width= 0.4pt,line join=round,line cap=round,fill=fillColor] (450.05,198.35) circle (  1.16);

\path[draw=drawColor,line width= 0.4pt,line join=round,line cap=round,fill=fillColor] (450.40,198.35) circle (  1.16);

\path[draw=drawColor,line width= 0.4pt,line join=round,line cap=round,fill=fillColor] (450.75,198.35) circle (  1.16);

\path[draw=drawColor,line width= 0.4pt,line join=round,line cap=round,fill=fillColor] (451.10,198.35) circle (  1.16);

\path[draw=drawColor,line width= 0.4pt,line join=round,line cap=round,fill=fillColor] (451.45,198.35) circle (  1.16);

\path[draw=drawColor,line width= 0.4pt,line join=round,line cap=round,fill=fillColor] (451.79,198.35) circle (  1.16);

\path[draw=drawColor,line width= 0.4pt,line join=round,line cap=round,fill=fillColor] (452.14,198.35) circle (  1.16);

\path[draw=drawColor,line width= 0.4pt,line join=round,line cap=round,fill=fillColor] (452.48,198.35) circle (  1.16);

\path[draw=drawColor,line width= 0.4pt,line join=round,line cap=round,fill=fillColor] (452.82,198.35) circle (  1.16);

\path[draw=drawColor,line width= 0.4pt,line join=round,line cap=round,fill=fillColor] (453.16,198.35) circle (  1.16);

\path[draw=drawColor,line width= 0.4pt,line join=round,line cap=round,fill=fillColor] (453.50,198.35) circle (  1.16);

\path[draw=drawColor,line width= 0.4pt,line join=round,line cap=round,fill=fillColor] (453.84,198.35) circle (  1.16);

\path[draw=drawColor,line width= 0.4pt,line join=round,line cap=round,fill=fillColor] (454.17,198.35) circle (  1.16);

\path[draw=drawColor,line width= 0.4pt,line join=round,line cap=round,fill=fillColor] (454.51,198.35) circle (  1.16);

\path[draw=drawColor,line width= 0.4pt,line join=round,line cap=round,fill=fillColor] (454.84,198.35) circle (  1.16);

\path[draw=drawColor,line width= 0.4pt,line join=round,line cap=round,fill=fillColor] (455.17,198.35) circle (  1.16);

\path[draw=drawColor,line width= 0.4pt,line join=round,line cap=round,fill=fillColor] (455.50,198.35) circle (  1.16);

\path[draw=drawColor,line width= 0.4pt,line join=round,line cap=round,fill=fillColor] (455.83,198.35) circle (  1.16);

\path[draw=drawColor,line width= 0.4pt,line join=round,line cap=round,fill=fillColor] (456.16,198.35) circle (  1.16);

\path[draw=drawColor,line width= 0.4pt,line join=round,line cap=round,fill=fillColor] (456.48,198.35) circle (  1.16);

\path[draw=drawColor,line width= 0.4pt,line join=round,line cap=round,fill=fillColor] (456.80,198.35) circle (  1.16);

\path[draw=drawColor,line width= 0.4pt,line join=round,line cap=round,fill=fillColor] (457.13,198.35) circle (  1.16);

\path[draw=drawColor,line width= 0.4pt,line join=round,line cap=round,fill=fillColor] (457.45,198.35) circle (  1.16);

\path[draw=drawColor,line width= 0.4pt,line join=round,line cap=round,fill=fillColor] (457.77,198.35) circle (  1.16);

\path[draw=drawColor,line width= 0.4pt,line join=round,line cap=round,fill=fillColor] (458.09,198.35) circle (  1.16);

\path[draw=drawColor,line width= 0.4pt,line join=round,line cap=round,fill=fillColor] (458.40,198.35) circle (  1.16);

\path[draw=drawColor,line width= 0.4pt,line join=round,line cap=round,fill=fillColor] (458.72,198.35) circle (  1.16);

\path[draw=drawColor,line width= 0.4pt,line join=round,line cap=round,fill=fillColor] (459.03,198.35) circle (  1.16);

\path[draw=drawColor,line width= 0.4pt,line join=round,line cap=round,fill=fillColor] (459.35,198.35) circle (  1.16);

\path[draw=drawColor,line width= 0.4pt,line join=round,line cap=round,fill=fillColor] (459.66,198.35) circle (  1.16);

\path[draw=drawColor,line width= 0.4pt,line join=round,line cap=round,fill=fillColor] (459.97,198.35) circle (  1.16);

\path[draw=drawColor,line width= 0.4pt,line join=round,line cap=round,fill=fillColor] (460.28,198.35) circle (  1.16);

\path[draw=drawColor,line width= 0.4pt,line join=round,line cap=round,fill=fillColor] (460.59,198.35) circle (  1.16);

\path[draw=drawColor,line width= 0.4pt,line join=round,line cap=round,fill=fillColor] (460.90,198.35) circle (  1.16);

\path[draw=drawColor,line width= 0.4pt,line join=round,line cap=round,fill=fillColor] (461.20,198.35) circle (  1.16);

\path[draw=drawColor,line width= 0.4pt,line join=round,line cap=round,fill=fillColor] (461.51,198.35) circle (  1.16);

\path[draw=drawColor,line width= 0.4pt,line join=round,line cap=round,fill=fillColor] (461.81,198.35) circle (  1.16);

\path[draw=drawColor,line width= 0.4pt,line join=round,line cap=round,fill=fillColor] (462.11,198.35) circle (  1.16);

\path[draw=drawColor,line width= 0.4pt,line join=round,line cap=round,fill=fillColor] (462.42,198.35) circle (  1.16);

\path[draw=drawColor,line width= 0.4pt,line join=round,line cap=round,fill=fillColor] (462.72,198.35) circle (  1.16);

\path[draw=drawColor,line width= 0.4pt,line join=round,line cap=round,fill=fillColor] (463.01,198.35) circle (  1.16);

\path[draw=drawColor,line width= 0.4pt,line join=round,line cap=round,fill=fillColor] (463.31,198.35) circle (  1.16);

\path[draw=drawColor,line width= 0.4pt,line join=round,line cap=round,fill=fillColor] (463.61,198.35) circle (  1.16);

\path[draw=drawColor,line width= 0.4pt,line join=round,line cap=round,fill=fillColor] (463.91,198.35) circle (  1.16);

\path[draw=drawColor,line width= 0.4pt,line join=round,line cap=round,fill=fillColor] (464.20,198.35) circle (  1.16);

\path[draw=drawColor,line width= 0.4pt,line join=round,line cap=round,fill=fillColor] (464.49,198.35) circle (  1.16);

\path[draw=drawColor,line width= 0.4pt,line join=round,line cap=round,fill=fillColor] (464.79,198.35) circle (  1.16);

\path[draw=drawColor,line width= 0.4pt,line join=round,line cap=round,fill=fillColor] (465.08,198.35) circle (  1.16);

\path[draw=drawColor,line width= 0.4pt,line join=round,line cap=round,fill=fillColor] (465.37,198.35) circle (  1.16);

\path[draw=drawColor,line width= 0.4pt,line join=round,line cap=round,fill=fillColor] (465.66,198.35) circle (  1.16);

\path[draw=drawColor,line width= 0.4pt,line join=round,line cap=round,fill=fillColor] (465.95,198.35) circle (  1.16);

\path[draw=drawColor,line width= 0.4pt,line join=round,line cap=round,fill=fillColor] (466.23,198.35) circle (  1.16);

\path[draw=drawColor,line width= 0.4pt,line join=round,line cap=round,fill=fillColor] (466.52,198.35) circle (  1.16);

\path[draw=drawColor,line width= 0.4pt,line join=round,line cap=round,fill=fillColor] (466.81,198.35) circle (  1.16);

\path[draw=drawColor,line width= 0.4pt,line join=round,line cap=round,fill=fillColor] (467.09,198.35) circle (  1.16);

\path[draw=drawColor,line width= 0.4pt,line join=round,line cap=round,fill=fillColor] (467.37,198.35) circle (  1.16);

\path[draw=drawColor,line width= 0.4pt,line join=round,line cap=round,fill=fillColor] (467.66,198.35) circle (  1.16);

\path[draw=drawColor,line width= 0.4pt,line join=round,line cap=round,fill=fillColor] (467.94,198.35) circle (  1.16);

\path[draw=drawColor,line width= 0.4pt,line join=round,line cap=round,fill=fillColor] (468.22,198.35) circle (  1.16);

\path[draw=drawColor,line width= 0.4pt,line join=round,line cap=round,fill=fillColor] (468.50,198.35) circle (  1.16);

\path[draw=drawColor,line width= 0.4pt,line join=round,line cap=round,fill=fillColor] (468.78,198.35) circle (  1.16);

\path[draw=drawColor,line width= 0.4pt,line join=round,line cap=round,fill=fillColor] (469.05,198.35) circle (  1.16);

\path[draw=drawColor,line width= 0.4pt,line join=round,line cap=round,fill=fillColor] (469.33,198.35) circle (  1.16);

\path[draw=drawColor,line width= 0.4pt,line join=round,line cap=round,fill=fillColor] (469.61,198.35) circle (  1.16);

\path[draw=drawColor,line width= 0.4pt,line join=round,line cap=round,fill=fillColor] (469.88,198.35) circle (  1.16);

\path[draw=drawColor,line width= 0.4pt,line join=round,line cap=round,fill=fillColor] (470.15,198.35) circle (  1.16);

\path[draw=drawColor,line width= 0.4pt,line join=round,line cap=round,fill=fillColor] (470.43,198.35) circle (  1.16);

\path[draw=drawColor,line width= 0.4pt,line join=round,line cap=round,fill=fillColor] (470.70,198.35) circle (  1.16);

\path[draw=drawColor,line width= 0.4pt,line join=round,line cap=round,fill=fillColor] (470.97,198.35) circle (  1.16);

\path[draw=drawColor,line width= 0.4pt,line join=round,line cap=round,fill=fillColor] (471.24,198.35) circle (  1.16);

\path[draw=drawColor,line width= 0.4pt,line join=round,line cap=round,fill=fillColor] (471.51,198.35) circle (  1.16);

\path[draw=drawColor,line width= 0.4pt,line join=round,line cap=round,fill=fillColor] (471.78,198.35) circle (  1.16);

\path[draw=drawColor,line width= 0.4pt,line join=round,line cap=round,fill=fillColor] (472.05,198.35) circle (  1.16);
\definecolor[named]{drawColor}{rgb}{0.00,0.00,0.00}
\definecolor[named]{fillColor}{rgb}{0.00,0.00,0.00}

\path[draw=drawColor,line width= 0.6pt,line join=round,fill=fillColor] (324.91,277.74) -- (479.05,277.74);

\node[text=drawColor,anchor=base east,inner sep=0pt, outer sep=0pt, scale=  0.85] at (475.55,276.86) {infeasible solutions};

\path[draw=drawColor,line width= 0.6pt,line join=round,line cap=round] (324.91,190.42) rectangle (479.05,285.68);
\end{scope}
\begin{scope}
\path[clip] (  0.00,  0.00) rectangle (505.89,614.29);
\definecolor[named]{drawColor}{rgb}{0.00,0.00,0.00}

\node[text=drawColor,anchor=base east,inner sep=0pt, outer sep=0pt, scale=  0.80] at (319.51,195.60) {0.00};

\node[text=drawColor,anchor=base east,inner sep=0pt, outer sep=0pt, scale=  0.80] at (319.51,212.70) {0.01};

\node[text=drawColor,anchor=base east,inner sep=0pt, outer sep=0pt, scale=  0.80] at (319.51,224.85) {0.05};

\node[text=drawColor,anchor=base east,inner sep=0pt, outer sep=0pt, scale=  0.80] at (319.51,242.03) {0.20};

\node[text=drawColor,anchor=base east,inner sep=0pt, outer sep=0pt, scale=  0.80] at (319.51,258.61) {0.50};

\node[text=drawColor,anchor=base east,inner sep=0pt, outer sep=0pt, scale=  0.80] at (319.51,267.73) {0.75};

\node[text=drawColor,anchor=base east,inner sep=0pt, outer sep=0pt, scale=  0.80] at (319.51,274.99) {1.00};
\end{scope}
\begin{scope}
\path[clip] (  0.00,  0.00) rectangle (505.89,614.29);
\definecolor[named]{drawColor}{rgb}{0.00,0.00,0.00}

\path[draw=drawColor,line width= 0.6pt,line join=round] (321.91,198.35) --
	(324.91,198.35);

\path[draw=drawColor,line width= 0.6pt,line join=round] (321.91,215.46) --
	(324.91,215.46);

\path[draw=drawColor,line width= 0.6pt,line join=round] (321.91,227.60) --
	(324.91,227.60);

\path[draw=drawColor,line width= 0.6pt,line join=round] (321.91,244.78) --
	(324.91,244.78);

\path[draw=drawColor,line width= 0.6pt,line join=round] (321.91,261.37) --
	(324.91,261.37);

\path[draw=drawColor,line width= 0.6pt,line join=round] (321.91,270.48) --
	(324.91,270.48);

\path[draw=drawColor,line width= 0.6pt,line join=round] (321.91,277.74) --
	(324.91,277.74);
\end{scope}
\begin{scope}
\path[clip] (  0.00,  0.00) rectangle (505.89,614.29);
\definecolor[named]{drawColor}{rgb}{0.00,0.00,0.00}

\path[draw=drawColor,line width= 0.6pt,line join=round] (407.99,187.42) --
	(407.99,190.42);

\path[draw=drawColor,line width= 0.6pt,line join=round] (435.13,187.42) --
	(435.13,190.42);

\path[draw=drawColor,line width= 0.6pt,line join=round] (454.17,187.42) --
	(454.17,190.42);

\path[draw=drawColor,line width= 0.6pt,line join=round] (469.33,187.42) --
	(469.33,190.42);
\end{scope}
\begin{scope}
\path[clip] (  0.00,  0.00) rectangle (505.89,614.29);
\definecolor[named]{drawColor}{rgb}{0.00,0.00,0.00}

\node[text=drawColor,rotate= 50.00,anchor=base east,inner sep=0pt, outer sep=0pt, scale=  0.80] at (412.21,181.47) {50};

\node[text=drawColor,rotate= 50.00,anchor=base east,inner sep=0pt, outer sep=0pt, scale=  0.80] at (439.35,181.47) {100};

\node[text=drawColor,rotate= 50.00,anchor=base east,inner sep=0pt, outer sep=0pt, scale=  0.80] at (458.39,181.47) {150};

\node[text=drawColor,rotate= 50.00,anchor=base east,inner sep=0pt, outer sep=0pt, scale=  0.80] at (473.55,181.47) {200};
\end{scope}
\begin{scope}
\path[clip] (  0.00,  0.00) rectangle (505.89,614.29);
\definecolor[named]{drawColor}{rgb}{0.00,0.00,0.00}

\node[text=drawColor,anchor=base,inner sep=0pt, outer sep=0pt, scale=  0.80] at (401.98,161.97) {\# Instances};
\end{scope}
\begin{scope}
\path[clip] (  0.00,  0.00) rectangle (505.89,614.29);
\definecolor[named]{drawColor}{rgb}{0.00,0.00,0.00}

\node[text=drawColor,rotate= 90.00,anchor=base,inner sep=0pt, outer sep=0pt, scale=  0.80] at (295.29,238.05) {1-(Best/Algorithm)};
\end{scope}
\begin{scope}
\path[clip] (  0.00,  0.00) rectangle (505.89,614.29);
\definecolor[named]{drawColor}{rgb}{0.00,0.00,0.00}

\node[text=drawColor,anchor=base,inner sep=0pt, outer sep=0pt, scale=  1.20] at (401.98,292.88) {\Literal};
\end{scope}
\begin{scope}
\path[clip] ( 20.84,  0.00) rectangle (232.11,153.57);
\definecolor[named]{drawColor}{rgb}{1.00,1.00,1.00}
\definecolor[named]{fillColor}{rgb}{1.00,1.00,1.00}

\path[draw=drawColor,line width= 0.6pt,line join=round,line cap=round,fill=fillColor] ( 20.84,  0.00) rectangle (232.11,153.57);
\end{scope}
\begin{scope}
\path[clip] ( 71.96, 36.84) rectangle (226.11,132.11);
\definecolor[named]{fillColor}{rgb}{1.00,1.00,1.00}

\path[fill=fillColor] ( 71.96, 36.84) rectangle (226.11,132.11);
\definecolor[named]{drawColor}{rgb}{0.98,0.98,0.98}

\path[draw=drawColor,line width= 0.6pt,line join=round] ( 71.96, 53.33) --
	(226.11, 53.33);

\path[draw=drawColor,line width= 0.6pt,line join=round] ( 71.96, 67.96) --
	(226.11, 67.96);

\path[draw=drawColor,line width= 0.6pt,line join=round] ( 71.96, 82.62) --
	(226.11, 82.62);

\path[draw=drawColor,line width= 0.6pt,line join=round] ( 71.96, 99.50) --
	(226.11, 99.50);

\path[draw=drawColor,line width= 0.6pt,line join=round] ( 71.96,112.35) --
	(226.11,112.35);

\path[draw=drawColor,line width= 0.6pt,line join=round] ( 71.96,120.54) --
	(226.11,120.54);

\path[draw=drawColor,line width= 0.6pt,line join=round] (116.43, 36.84) --
	(116.43,132.11);

\path[draw=drawColor,line width= 0.6pt,line join=round] (126.81, 36.84) --
	(126.81,132.11);

\path[draw=drawColor,line width= 0.6pt,line join=round] (147.59, 36.84) --
	(147.59,132.11);

\path[draw=drawColor,line width= 0.6pt,line join=round] (165.26, 36.84) --
	(165.26,132.11);

\path[draw=drawColor,line width= 0.6pt,line join=round] (178.35, 36.84) --
	(178.35,132.11);

\path[draw=drawColor,line width= 0.6pt,line join=round] (189.05, 36.84) --
	(189.05,132.11);

\path[draw=drawColor,line width= 0.6pt,line join=round] (198.23, 36.84) --
	(198.23,132.11);

\path[draw=drawColor,line width= 0.6pt,line join=round] (206.34, 36.84) --
	(206.34,132.11);

\path[draw=drawColor,line width= 0.6pt,line join=round] (213.65, 36.84) --
	(213.65,132.11);

\path[draw=drawColor,line width= 0.6pt,line join=round] (220.33, 36.84) --
	(220.33,132.11);
\definecolor[named]{drawColor}{rgb}{0.75,0.75,0.75}

\path[draw=drawColor,line width= 0.6pt,dash pattern=on 1pt off 3pt ,line join=round] ( 71.96, 44.78) --
	(226.11, 44.78);

\path[draw=drawColor,line width= 0.6pt,dash pattern=on 1pt off 3pt ,line join=round] ( 71.96, 61.88) --
	(226.11, 61.88);

\path[draw=drawColor,line width= 0.6pt,dash pattern=on 1pt off 3pt ,line join=round] ( 71.96, 74.03) --
	(226.11, 74.03);

\path[draw=drawColor,line width= 0.6pt,dash pattern=on 1pt off 3pt ,line join=round] ( 71.96, 91.21) --
	(226.11, 91.21);

\path[draw=drawColor,line width= 0.6pt,dash pattern=on 1pt off 3pt ,line join=round] ( 71.96,107.79) --
	(226.11,107.79);

\path[draw=drawColor,line width= 0.6pt,dash pattern=on 1pt off 3pt ,line join=round] ( 71.96,116.91) --
	(226.11,116.91);

\path[draw=drawColor,line width= 0.6pt,dash pattern=on 1pt off 3pt ,line join=round] ( 71.96,124.17) --
	(226.11,124.17);

\path[draw=drawColor,line width= 0.6pt,dash pattern=on 1pt off 3pt ,line join=round] (137.20, 36.84) --
	(137.20,132.11);

\path[draw=drawColor,line width= 0.6pt,dash pattern=on 1pt off 3pt ,line join=round] (157.98, 36.84) --
	(157.98,132.11);

\path[draw=drawColor,line width= 0.6pt,dash pattern=on 1pt off 3pt ,line join=round] (172.55, 36.84) --
	(172.55,132.11);

\path[draw=drawColor,line width= 0.6pt,dash pattern=on 1pt off 3pt ,line join=round] (184.15, 36.84) --
	(184.15,132.11);

\path[draw=drawColor,line width= 0.6pt,dash pattern=on 1pt off 3pt ,line join=round] (193.95, 36.84) --
	(193.95,132.11);

\path[draw=drawColor,line width= 0.6pt,dash pattern=on 1pt off 3pt ,line join=round] (202.51, 36.84) --
	(202.51,132.11);

\path[draw=drawColor,line width= 0.6pt,dash pattern=on 1pt off 3pt ,line join=round] (210.17, 36.84) --
	(210.17,132.11);

\path[draw=drawColor,line width= 0.6pt,dash pattern=on 1pt off 3pt ,line join=round] (217.13, 36.84) --
	(217.13,132.11);

\path[draw=drawColor,line width= 0.6pt,dash pattern=on 1pt off 3pt ,line join=round] (223.53, 36.84) --
	(223.53,132.11);
\definecolor[named]{drawColor}{rgb}{0.89,0.10,0.11}
\definecolor[named]{fillColor}{rgb}{0.89,0.10,0.11}

\path[draw=drawColor,line width= 0.4pt,line join=round,line cap=round,fill=fillColor] ( 78.97, 82.68) circle (  1.16);

\path[draw=drawColor,line width= 0.4pt,line join=round,line cap=round,fill=fillColor] ( 84.61, 82.09) circle (  1.16);

\path[draw=drawColor,line width= 0.4pt,line join=round,line cap=round,fill=fillColor] ( 88.57, 81.84) circle (  1.16);

\path[draw=drawColor,line width= 0.4pt,line join=round,line cap=round,fill=fillColor] ( 91.71, 79.87) circle (  1.16);

\path[draw=drawColor,line width= 0.4pt,line join=round,line cap=round,fill=fillColor] ( 94.37, 79.59) circle (  1.16);

\path[draw=drawColor,line width= 0.4pt,line join=round,line cap=round,fill=fillColor] ( 96.70, 79.14) circle (  1.16);

\path[draw=drawColor,line width= 0.4pt,line join=round,line cap=round,fill=fillColor] ( 98.78, 78.70) circle (  1.16);

\path[draw=drawColor,line width= 0.4pt,line join=round,line cap=round,fill=fillColor] (100.67, 78.46) circle (  1.16);

\path[draw=drawColor,line width= 0.4pt,line join=round,line cap=round,fill=fillColor] (102.40, 77.63) circle (  1.16);

\path[draw=drawColor,line width= 0.4pt,line join=round,line cap=round,fill=fillColor] (104.02, 77.62) circle (  1.16);

\path[draw=drawColor,line width= 0.4pt,line join=round,line cap=round,fill=fillColor] (105.53, 76.15) circle (  1.16);

\path[draw=drawColor,line width= 0.4pt,line join=round,line cap=round,fill=fillColor] (106.95, 76.10) circle (  1.16);

\path[draw=drawColor,line width= 0.4pt,line join=round,line cap=round,fill=fillColor] (108.29, 75.90) circle (  1.16);

\path[draw=drawColor,line width= 0.4pt,line join=round,line cap=round,fill=fillColor] (109.56, 75.89) circle (  1.16);

\path[draw=drawColor,line width= 0.4pt,line join=round,line cap=round,fill=fillColor] (110.78, 75.74) circle (  1.16);

\path[draw=drawColor,line width= 0.4pt,line join=round,line cap=round,fill=fillColor] (111.94, 75.55) circle (  1.16);

\path[draw=drawColor,line width= 0.4pt,line join=round,line cap=round,fill=fillColor] (113.06, 75.54) circle (  1.16);

\path[draw=drawColor,line width= 0.4pt,line join=round,line cap=round,fill=fillColor] (114.13, 75.49) circle (  1.16);

\path[draw=drawColor,line width= 0.4pt,line join=round,line cap=round,fill=fillColor] (115.17, 75.36) circle (  1.16);

\path[draw=drawColor,line width= 0.4pt,line join=round,line cap=round,fill=fillColor] (116.17, 75.34) circle (  1.16);

\path[draw=drawColor,line width= 0.4pt,line join=round,line cap=round,fill=fillColor] (117.13, 75.30) circle (  1.16);

\path[draw=drawColor,line width= 0.4pt,line join=round,line cap=round,fill=fillColor] (118.07, 75.16) circle (  1.16);

\path[draw=drawColor,line width= 0.4pt,line join=round,line cap=round,fill=fillColor] (118.97, 75.02) circle (  1.16);

\path[draw=drawColor,line width= 0.4pt,line join=round,line cap=round,fill=fillColor] (119.86, 74.91) circle (  1.16);

\path[draw=drawColor,line width= 0.4pt,line join=round,line cap=round,fill=fillColor] (120.71, 74.91) circle (  1.16);

\path[draw=drawColor,line width= 0.4pt,line join=round,line cap=round,fill=fillColor] (121.55, 74.33) circle (  1.16);

\path[draw=drawColor,line width= 0.4pt,line join=round,line cap=round,fill=fillColor] (122.36, 74.33) circle (  1.16);

\path[draw=drawColor,line width= 0.4pt,line join=round,line cap=round,fill=fillColor] (123.16, 74.33) circle (  1.16);

\path[draw=drawColor,line width= 0.4pt,line join=round,line cap=round,fill=fillColor] (123.93, 74.24) circle (  1.16);

\path[draw=drawColor,line width= 0.4pt,line join=round,line cap=round,fill=fillColor] (124.69, 73.58) circle (  1.16);

\path[draw=drawColor,line width= 0.4pt,line join=round,line cap=round,fill=fillColor] (125.43, 73.35) circle (  1.16);

\path[draw=drawColor,line width= 0.4pt,line join=round,line cap=round,fill=fillColor] (126.15, 73.09) circle (  1.16);

\path[draw=drawColor,line width= 0.4pt,line join=round,line cap=round,fill=fillColor] (126.86, 72.80) circle (  1.16);

\path[draw=drawColor,line width= 0.4pt,line join=round,line cap=round,fill=fillColor] (127.56, 72.74) circle (  1.16);

\path[draw=drawColor,line width= 0.4pt,line join=round,line cap=round,fill=fillColor] (128.24, 72.73) circle (  1.16);

\path[draw=drawColor,line width= 0.4pt,line join=round,line cap=round,fill=fillColor] (128.91, 72.70) circle (  1.16);

\path[draw=drawColor,line width= 0.4pt,line join=round,line cap=round,fill=fillColor] (129.57, 72.70) circle (  1.16);

\path[draw=drawColor,line width= 0.4pt,line join=round,line cap=round,fill=fillColor] (130.21, 72.51) circle (  1.16);

\path[draw=drawColor,line width= 0.4pt,line join=round,line cap=round,fill=fillColor] (130.85, 72.20) circle (  1.16);

\path[draw=drawColor,line width= 0.4pt,line join=round,line cap=round,fill=fillColor] (131.47, 72.19) circle (  1.16);

\path[draw=drawColor,line width= 0.4pt,line join=round,line cap=round,fill=fillColor] (132.09, 71.95) circle (  1.16);

\path[draw=drawColor,line width= 0.4pt,line join=round,line cap=round,fill=fillColor] (132.69, 71.66) circle (  1.16);

\path[draw=drawColor,line width= 0.4pt,line join=round,line cap=round,fill=fillColor] (133.28, 71.43) circle (  1.16);

\path[draw=drawColor,line width= 0.4pt,line join=round,line cap=round,fill=fillColor] (133.87, 71.15) circle (  1.16);

\path[draw=drawColor,line width= 0.4pt,line join=round,line cap=round,fill=fillColor] (134.44, 71.06) circle (  1.16);

\path[draw=drawColor,line width= 0.4pt,line join=round,line cap=round,fill=fillColor] (135.01, 71.01) circle (  1.16);

\path[draw=drawColor,line width= 0.4pt,line join=round,line cap=round,fill=fillColor] (135.57, 70.82) circle (  1.16);

\path[draw=drawColor,line width= 0.4pt,line join=round,line cap=round,fill=fillColor] (136.12, 70.81) circle (  1.16);

\path[draw=drawColor,line width= 0.4pt,line join=round,line cap=round,fill=fillColor] (136.67, 70.74) circle (  1.16);

\path[draw=drawColor,line width= 0.4pt,line join=round,line cap=round,fill=fillColor] (137.20, 70.71) circle (  1.16);

\path[draw=drawColor,line width= 0.4pt,line join=round,line cap=round,fill=fillColor] (137.73, 70.69) circle (  1.16);

\path[draw=drawColor,line width= 0.4pt,line join=round,line cap=round,fill=fillColor] (138.25, 70.60) circle (  1.16);

\path[draw=drawColor,line width= 0.4pt,line join=round,line cap=round,fill=fillColor] (138.77, 70.39) circle (  1.16);

\path[draw=drawColor,line width= 0.4pt,line join=round,line cap=round,fill=fillColor] (139.28, 70.32) circle (  1.16);

\path[draw=drawColor,line width= 0.4pt,line join=round,line cap=round,fill=fillColor] (139.78, 70.31) circle (  1.16);

\path[draw=drawColor,line width= 0.4pt,line join=round,line cap=round,fill=fillColor] (140.28, 70.25) circle (  1.16);

\path[draw=drawColor,line width= 0.4pt,line join=round,line cap=round,fill=fillColor] (140.77, 70.24) circle (  1.16);

\path[draw=drawColor,line width= 0.4pt,line join=round,line cap=round,fill=fillColor] (141.26, 70.21) circle (  1.16);

\path[draw=drawColor,line width= 0.4pt,line join=round,line cap=round,fill=fillColor] (141.74, 70.21) circle (  1.16);

\path[draw=drawColor,line width= 0.4pt,line join=round,line cap=round,fill=fillColor] (142.21, 70.11) circle (  1.16);

\path[draw=drawColor,line width= 0.4pt,line join=round,line cap=round,fill=fillColor] (142.68, 70.07) circle (  1.16);

\path[draw=drawColor,line width= 0.4pt,line join=round,line cap=round,fill=fillColor] (143.14, 69.95) circle (  1.16);

\path[draw=drawColor,line width= 0.4pt,line join=round,line cap=round,fill=fillColor] (143.60, 69.93) circle (  1.16);

\path[draw=drawColor,line width= 0.4pt,line join=round,line cap=round,fill=fillColor] (144.06, 69.89) circle (  1.16);

\path[draw=drawColor,line width= 0.4pt,line join=round,line cap=round,fill=fillColor] (144.51, 69.63) circle (  1.16);

\path[draw=drawColor,line width= 0.4pt,line join=round,line cap=round,fill=fillColor] (144.95, 69.58) circle (  1.16);

\path[draw=drawColor,line width= 0.4pt,line join=round,line cap=round,fill=fillColor] (145.39, 69.53) circle (  1.16);

\path[draw=drawColor,line width= 0.4pt,line join=round,line cap=round,fill=fillColor] (145.83, 69.49) circle (  1.16);

\path[draw=drawColor,line width= 0.4pt,line join=round,line cap=round,fill=fillColor] (146.26, 69.31) circle (  1.16);

\path[draw=drawColor,line width= 0.4pt,line join=round,line cap=round,fill=fillColor] (146.69, 69.25) circle (  1.16);

\path[draw=drawColor,line width= 0.4pt,line join=round,line cap=round,fill=fillColor] (147.11, 69.17) circle (  1.16);

\path[draw=drawColor,line width= 0.4pt,line join=round,line cap=round,fill=fillColor] (147.53, 69.16) circle (  1.16);

\path[draw=drawColor,line width= 0.4pt,line join=round,line cap=round,fill=fillColor] (147.95, 69.15) circle (  1.16);

\path[draw=drawColor,line width= 0.4pt,line join=round,line cap=round,fill=fillColor] (148.36, 69.14) circle (  1.16);

\path[draw=drawColor,line width= 0.4pt,line join=round,line cap=round,fill=fillColor] (148.77, 69.12) circle (  1.16);

\path[draw=drawColor,line width= 0.4pt,line join=round,line cap=round,fill=fillColor] (149.17, 69.11) circle (  1.16);

\path[draw=drawColor,line width= 0.4pt,line join=round,line cap=round,fill=fillColor] (149.57, 69.11) circle (  1.16);

\path[draw=drawColor,line width= 0.4pt,line join=round,line cap=round,fill=fillColor] (149.97, 68.94) circle (  1.16);

\path[draw=drawColor,line width= 0.4pt,line join=round,line cap=round,fill=fillColor] (150.37, 68.85) circle (  1.16);

\path[draw=drawColor,line width= 0.4pt,line join=round,line cap=round,fill=fillColor] (150.76, 68.82) circle (  1.16);

\path[draw=drawColor,line width= 0.4pt,line join=round,line cap=round,fill=fillColor] (151.15, 68.79) circle (  1.16);

\path[draw=drawColor,line width= 0.4pt,line join=round,line cap=round,fill=fillColor] (151.53, 68.78) circle (  1.16);

\path[draw=drawColor,line width= 0.4pt,line join=round,line cap=round,fill=fillColor] (151.91, 68.69) circle (  1.16);

\path[draw=drawColor,line width= 0.4pt,line join=round,line cap=round,fill=fillColor] (152.29, 68.68) circle (  1.16);

\path[draw=drawColor,line width= 0.4pt,line join=round,line cap=round,fill=fillColor] (152.67, 68.65) circle (  1.16);

\path[draw=drawColor,line width= 0.4pt,line join=round,line cap=round,fill=fillColor] (153.04, 68.65) circle (  1.16);

\path[draw=drawColor,line width= 0.4pt,line join=round,line cap=round,fill=fillColor] (153.41, 68.64) circle (  1.16);

\path[draw=drawColor,line width= 0.4pt,line join=round,line cap=round,fill=fillColor] (153.78, 68.56) circle (  1.16);

\path[draw=drawColor,line width= 0.4pt,line join=round,line cap=round,fill=fillColor] (154.14, 68.34) circle (  1.16);

\path[draw=drawColor,line width= 0.4pt,line join=round,line cap=round,fill=fillColor] (154.50, 68.32) circle (  1.16);

\path[draw=drawColor,line width= 0.4pt,line join=round,line cap=round,fill=fillColor] (154.86, 68.30) circle (  1.16);

\path[draw=drawColor,line width= 0.4pt,line join=round,line cap=round,fill=fillColor] (155.22, 68.28) circle (  1.16);

\path[draw=drawColor,line width= 0.4pt,line join=round,line cap=round,fill=fillColor] (155.57, 68.24) circle (  1.16);

\path[draw=drawColor,line width= 0.4pt,line join=round,line cap=round,fill=fillColor] (155.92, 68.11) circle (  1.16);

\path[draw=drawColor,line width= 0.4pt,line join=round,line cap=round,fill=fillColor] (156.27, 68.08) circle (  1.16);

\path[draw=drawColor,line width= 0.4pt,line join=round,line cap=round,fill=fillColor] (156.62, 68.08) circle (  1.16);

\path[draw=drawColor,line width= 0.4pt,line join=round,line cap=round,fill=fillColor] (156.96, 67.95) circle (  1.16);

\path[draw=drawColor,line width= 0.4pt,line join=round,line cap=round,fill=fillColor] (157.30, 67.88) circle (  1.16);

\path[draw=drawColor,line width= 0.4pt,line join=round,line cap=round,fill=fillColor] (157.64, 67.74) circle (  1.16);

\path[draw=drawColor,line width= 0.4pt,line join=round,line cap=round,fill=fillColor] (157.98, 67.73) circle (  1.16);

\path[draw=drawColor,line width= 0.4pt,line join=round,line cap=round,fill=fillColor] (158.31, 67.70) circle (  1.16);

\path[draw=drawColor,line width= 0.4pt,line join=round,line cap=round,fill=fillColor] (158.64, 67.56) circle (  1.16);

\path[draw=drawColor,line width= 0.4pt,line join=round,line cap=round,fill=fillColor] (158.97, 67.55) circle (  1.16);

\path[draw=drawColor,line width= 0.4pt,line join=round,line cap=round,fill=fillColor] (159.30, 67.49) circle (  1.16);

\path[draw=drawColor,line width= 0.4pt,line join=round,line cap=round,fill=fillColor] (159.63, 67.40) circle (  1.16);

\path[draw=drawColor,line width= 0.4pt,line join=round,line cap=round,fill=fillColor] (159.95, 67.33) circle (  1.16);

\path[draw=drawColor,line width= 0.4pt,line join=round,line cap=round,fill=fillColor] (160.27, 67.27) circle (  1.16);

\path[draw=drawColor,line width= 0.4pt,line join=round,line cap=round,fill=fillColor] (160.59, 67.26) circle (  1.16);

\path[draw=drawColor,line width= 0.4pt,line join=round,line cap=round,fill=fillColor] (160.91, 67.17) circle (  1.16);

\path[draw=drawColor,line width= 0.4pt,line join=round,line cap=round,fill=fillColor] (161.23, 66.99) circle (  1.16);

\path[draw=drawColor,line width= 0.4pt,line join=round,line cap=round,fill=fillColor] (161.54, 66.95) circle (  1.16);

\path[draw=drawColor,line width= 0.4pt,line join=round,line cap=round,fill=fillColor] (161.85, 66.94) circle (  1.16);

\path[draw=drawColor,line width= 0.4pt,line join=round,line cap=round,fill=fillColor] (162.16, 66.90) circle (  1.16);

\path[draw=drawColor,line width= 0.4pt,line join=round,line cap=round,fill=fillColor] (162.47, 66.89) circle (  1.16);

\path[draw=drawColor,line width= 0.4pt,line join=round,line cap=round,fill=fillColor] (162.78, 66.85) circle (  1.16);

\path[draw=drawColor,line width= 0.4pt,line join=round,line cap=round,fill=fillColor] (163.08, 66.84) circle (  1.16);

\path[draw=drawColor,line width= 0.4pt,line join=round,line cap=round,fill=fillColor] (163.39, 66.81) circle (  1.16);

\path[draw=drawColor,line width= 0.4pt,line join=round,line cap=round,fill=fillColor] (163.69, 66.81) circle (  1.16);

\path[draw=drawColor,line width= 0.4pt,line join=round,line cap=round,fill=fillColor] (163.99, 66.80) circle (  1.16);

\path[draw=drawColor,line width= 0.4pt,line join=round,line cap=round,fill=fillColor] (164.29, 66.77) circle (  1.16);

\path[draw=drawColor,line width= 0.4pt,line join=round,line cap=round,fill=fillColor] (164.58, 66.75) circle (  1.16);

\path[draw=drawColor,line width= 0.4pt,line join=round,line cap=round,fill=fillColor] (164.88, 66.70) circle (  1.16);

\path[draw=drawColor,line width= 0.4pt,line join=round,line cap=round,fill=fillColor] (165.17, 66.67) circle (  1.16);

\path[draw=drawColor,line width= 0.4pt,line join=round,line cap=round,fill=fillColor] (165.46, 66.66) circle (  1.16);

\path[draw=drawColor,line width= 0.4pt,line join=round,line cap=round,fill=fillColor] (165.75, 66.65) circle (  1.16);

\path[draw=drawColor,line width= 0.4pt,line join=round,line cap=round,fill=fillColor] (166.04, 66.63) circle (  1.16);

\path[draw=drawColor,line width= 0.4pt,line join=round,line cap=round,fill=fillColor] (166.33, 66.54) circle (  1.16);

\path[draw=drawColor,line width= 0.4pt,line join=round,line cap=round,fill=fillColor] (166.61, 66.48) circle (  1.16);

\path[draw=drawColor,line width= 0.4pt,line join=round,line cap=round,fill=fillColor] (166.90, 66.47) circle (  1.16);

\path[draw=drawColor,line width= 0.4pt,line join=round,line cap=round,fill=fillColor] (167.18, 66.47) circle (  1.16);

\path[draw=drawColor,line width= 0.4pt,line join=round,line cap=round,fill=fillColor] (167.46, 66.46) circle (  1.16);

\path[draw=drawColor,line width= 0.4pt,line join=round,line cap=round,fill=fillColor] (167.74, 66.44) circle (  1.16);

\path[draw=drawColor,line width= 0.4pt,line join=round,line cap=round,fill=fillColor] (168.02, 66.42) circle (  1.16);

\path[draw=drawColor,line width= 0.4pt,line join=round,line cap=round,fill=fillColor] (168.30, 66.23) circle (  1.16);

\path[draw=drawColor,line width= 0.4pt,line join=round,line cap=round,fill=fillColor] (168.57, 66.20) circle (  1.16);

\path[draw=drawColor,line width= 0.4pt,line join=round,line cap=round,fill=fillColor] (168.85, 66.14) circle (  1.16);

\path[draw=drawColor,line width= 0.4pt,line join=round,line cap=round,fill=fillColor] (169.12, 66.08) circle (  1.16);

\path[draw=drawColor,line width= 0.4pt,line join=round,line cap=round,fill=fillColor] (169.39, 66.03) circle (  1.16);

\path[draw=drawColor,line width= 0.4pt,line join=round,line cap=round,fill=fillColor] (169.66, 65.85) circle (  1.16);

\path[draw=drawColor,line width= 0.4pt,line join=round,line cap=round,fill=fillColor] (169.93, 65.84) circle (  1.16);

\path[draw=drawColor,line width= 0.4pt,line join=round,line cap=round,fill=fillColor] (170.20, 65.78) circle (  1.16);

\path[draw=drawColor,line width= 0.4pt,line join=round,line cap=round,fill=fillColor] (170.46, 65.63) circle (  1.16);

\path[draw=drawColor,line width= 0.4pt,line join=round,line cap=round,fill=fillColor] (170.73, 65.61) circle (  1.16);

\path[draw=drawColor,line width= 0.4pt,line join=round,line cap=round,fill=fillColor] (170.99, 65.56) circle (  1.16);

\path[draw=drawColor,line width= 0.4pt,line join=round,line cap=round,fill=fillColor] (171.25, 65.53) circle (  1.16);

\path[draw=drawColor,line width= 0.4pt,line join=round,line cap=round,fill=fillColor] (171.52, 65.48) circle (  1.16);

\path[draw=drawColor,line width= 0.4pt,line join=round,line cap=round,fill=fillColor] (171.78, 65.45) circle (  1.16);

\path[draw=drawColor,line width= 0.4pt,line join=round,line cap=round,fill=fillColor] (172.04, 65.42) circle (  1.16);

\path[draw=drawColor,line width= 0.4pt,line join=round,line cap=round,fill=fillColor] (172.29, 65.39) circle (  1.16);

\path[draw=drawColor,line width= 0.4pt,line join=round,line cap=round,fill=fillColor] (172.55, 65.38) circle (  1.16);

\path[draw=drawColor,line width= 0.4pt,line join=round,line cap=round,fill=fillColor] (172.81, 65.31) circle (  1.16);

\path[draw=drawColor,line width= 0.4pt,line join=round,line cap=round,fill=fillColor] (173.06, 65.25) circle (  1.16);

\path[draw=drawColor,line width= 0.4pt,line join=round,line cap=round,fill=fillColor] (173.31, 65.22) circle (  1.16);

\path[draw=drawColor,line width= 0.4pt,line join=round,line cap=round,fill=fillColor] (173.57, 65.22) circle (  1.16);

\path[draw=drawColor,line width= 0.4pt,line join=round,line cap=round,fill=fillColor] (173.82, 65.14) circle (  1.16);

\path[draw=drawColor,line width= 0.4pt,line join=round,line cap=round,fill=fillColor] (174.07, 65.13) circle (  1.16);

\path[draw=drawColor,line width= 0.4pt,line join=round,line cap=round,fill=fillColor] (174.32, 65.09) circle (  1.16);

\path[draw=drawColor,line width= 0.4pt,line join=round,line cap=round,fill=fillColor] (174.56, 65.08) circle (  1.16);

\path[draw=drawColor,line width= 0.4pt,line join=round,line cap=round,fill=fillColor] (174.81, 65.07) circle (  1.16);

\path[draw=drawColor,line width= 0.4pt,line join=round,line cap=round,fill=fillColor] (175.06, 65.04) circle (  1.16);

\path[draw=drawColor,line width= 0.4pt,line join=round,line cap=round,fill=fillColor] (175.30, 64.96) circle (  1.16);

\path[draw=drawColor,line width= 0.4pt,line join=round,line cap=round,fill=fillColor] (175.55, 64.95) circle (  1.16);

\path[draw=drawColor,line width= 0.4pt,line join=round,line cap=round,fill=fillColor] (175.79, 64.84) circle (  1.16);

\path[draw=drawColor,line width= 0.4pt,line join=round,line cap=round,fill=fillColor] (176.03, 64.79) circle (  1.16);

\path[draw=drawColor,line width= 0.4pt,line join=round,line cap=round,fill=fillColor] (176.27, 64.75) circle (  1.16);

\path[draw=drawColor,line width= 0.4pt,line join=round,line cap=round,fill=fillColor] (176.51, 64.64) circle (  1.16);

\path[draw=drawColor,line width= 0.4pt,line join=round,line cap=round,fill=fillColor] (176.75, 64.63) circle (  1.16);

\path[draw=drawColor,line width= 0.4pt,line join=round,line cap=round,fill=fillColor] (176.99, 64.63) circle (  1.16);

\path[draw=drawColor,line width= 0.4pt,line join=round,line cap=round,fill=fillColor] (177.22, 64.57) circle (  1.16);

\path[draw=drawColor,line width= 0.4pt,line join=round,line cap=round,fill=fillColor] (177.46, 64.56) circle (  1.16);

\path[draw=drawColor,line width= 0.4pt,line join=round,line cap=round,fill=fillColor] (177.70, 64.45) circle (  1.16);

\path[draw=drawColor,line width= 0.4pt,line join=round,line cap=round,fill=fillColor] (177.93, 64.44) circle (  1.16);

\path[draw=drawColor,line width= 0.4pt,line join=round,line cap=round,fill=fillColor] (178.16, 64.41) circle (  1.16);

\path[draw=drawColor,line width= 0.4pt,line join=round,line cap=round,fill=fillColor] (178.40, 64.40) circle (  1.16);

\path[draw=drawColor,line width= 0.4pt,line join=round,line cap=round,fill=fillColor] (178.63, 64.39) circle (  1.16);

\path[draw=drawColor,line width= 0.4pt,line join=round,line cap=round,fill=fillColor] (178.86, 64.35) circle (  1.16);

\path[draw=drawColor,line width= 0.4pt,line join=round,line cap=round,fill=fillColor] (179.09, 64.30) circle (  1.16);

\path[draw=drawColor,line width= 0.4pt,line join=round,line cap=round,fill=fillColor] (179.32, 64.26) circle (  1.16);

\path[draw=drawColor,line width= 0.4pt,line join=round,line cap=round,fill=fillColor] (179.55, 64.17) circle (  1.16);

\path[draw=drawColor,line width= 0.4pt,line join=round,line cap=round,fill=fillColor] (179.77, 64.13) circle (  1.16);

\path[draw=drawColor,line width= 0.4pt,line join=round,line cap=round,fill=fillColor] (180.00, 64.10) circle (  1.16);

\path[draw=drawColor,line width= 0.4pt,line join=round,line cap=round,fill=fillColor] (180.22, 63.98) circle (  1.16);

\path[draw=drawColor,line width= 0.4pt,line join=round,line cap=round,fill=fillColor] (180.45, 63.88) circle (  1.16);

\path[draw=drawColor,line width= 0.4pt,line join=round,line cap=round,fill=fillColor] (180.67, 63.85) circle (  1.16);

\path[draw=drawColor,line width= 0.4pt,line join=round,line cap=round,fill=fillColor] (180.90, 63.81) circle (  1.16);

\path[draw=drawColor,line width= 0.4pt,line join=round,line cap=round,fill=fillColor] (181.12, 63.81) circle (  1.16);

\path[draw=drawColor,line width= 0.4pt,line join=round,line cap=round,fill=fillColor] (181.34, 63.77) circle (  1.16);

\path[draw=drawColor,line width= 0.4pt,line join=round,line cap=round,fill=fillColor] (181.56, 63.76) circle (  1.16);

\path[draw=drawColor,line width= 0.4pt,line join=round,line cap=round,fill=fillColor] (181.78, 63.75) circle (  1.16);

\path[draw=drawColor,line width= 0.4pt,line join=round,line cap=round,fill=fillColor] (182.00, 63.73) circle (  1.16);

\path[draw=drawColor,line width= 0.4pt,line join=round,line cap=round,fill=fillColor] (182.22, 63.69) circle (  1.16);

\path[draw=drawColor,line width= 0.4pt,line join=round,line cap=round,fill=fillColor] (182.44, 63.67) circle (  1.16);

\path[draw=drawColor,line width= 0.4pt,line join=round,line cap=round,fill=fillColor] (182.65, 63.58) circle (  1.16);

\path[draw=drawColor,line width= 0.4pt,line join=round,line cap=round,fill=fillColor] (182.87, 63.57) circle (  1.16);

\path[draw=drawColor,line width= 0.4pt,line join=round,line cap=round,fill=fillColor] (183.09, 63.57) circle (  1.16);

\path[draw=drawColor,line width= 0.4pt,line join=round,line cap=round,fill=fillColor] (183.30, 63.49) circle (  1.16);

\path[draw=drawColor,line width= 0.4pt,line join=round,line cap=round,fill=fillColor] (183.51, 63.29) circle (  1.16);

\path[draw=drawColor,line width= 0.4pt,line join=round,line cap=round,fill=fillColor] (183.73, 63.25) circle (  1.16);

\path[draw=drawColor,line width= 0.4pt,line join=round,line cap=round,fill=fillColor] (183.94, 63.21) circle (  1.16);

\path[draw=drawColor,line width= 0.4pt,line join=round,line cap=round,fill=fillColor] (184.15, 63.19) circle (  1.16);

\path[draw=drawColor,line width= 0.4pt,line join=round,line cap=round,fill=fillColor] (184.36, 63.19) circle (  1.16);

\path[draw=drawColor,line width= 0.4pt,line join=round,line cap=round,fill=fillColor] (184.57, 63.15) circle (  1.16);

\path[draw=drawColor,line width= 0.4pt,line join=round,line cap=round,fill=fillColor] (184.78, 63.13) circle (  1.16);

\path[draw=drawColor,line width= 0.4pt,line join=round,line cap=round,fill=fillColor] (184.99, 63.11) circle (  1.16);

\path[draw=drawColor,line width= 0.4pt,line join=round,line cap=round,fill=fillColor] (185.20, 63.02) circle (  1.16);

\path[draw=drawColor,line width= 0.4pt,line join=round,line cap=round,fill=fillColor] (185.41, 62.96) circle (  1.16);

\path[draw=drawColor,line width= 0.4pt,line join=round,line cap=round,fill=fillColor] (185.61, 62.91) circle (  1.16);

\path[draw=drawColor,line width= 0.4pt,line join=round,line cap=round,fill=fillColor] (185.82, 62.85) circle (  1.16);

\path[draw=drawColor,line width= 0.4pt,line join=round,line cap=round,fill=fillColor] (186.03, 62.81) circle (  1.16);

\path[draw=drawColor,line width= 0.4pt,line join=round,line cap=round,fill=fillColor] (186.23, 62.71) circle (  1.16);

\path[draw=drawColor,line width= 0.4pt,line join=round,line cap=round,fill=fillColor] (186.44, 62.68) circle (  1.16);

\path[draw=drawColor,line width= 0.4pt,line join=round,line cap=round,fill=fillColor] (186.64, 62.65) circle (  1.16);

\path[draw=drawColor,line width= 0.4pt,line join=round,line cap=round,fill=fillColor] (186.84, 62.65) circle (  1.16);

\path[draw=drawColor,line width= 0.4pt,line join=round,line cap=round,fill=fillColor] (187.05, 62.59) circle (  1.16);

\path[draw=drawColor,line width= 0.4pt,line join=round,line cap=round,fill=fillColor] (187.25, 62.54) circle (  1.16);

\path[draw=drawColor,line width= 0.4pt,line join=round,line cap=round,fill=fillColor] (187.45, 62.45) circle (  1.16);

\path[draw=drawColor,line width= 0.4pt,line join=round,line cap=round,fill=fillColor] (187.65, 62.38) circle (  1.16);

\path[draw=drawColor,line width= 0.4pt,line join=round,line cap=round,fill=fillColor] (187.85, 62.38) circle (  1.16);

\path[draw=drawColor,line width= 0.4pt,line join=round,line cap=round,fill=fillColor] (188.05, 62.37) circle (  1.16);

\path[draw=drawColor,line width= 0.4pt,line join=round,line cap=round,fill=fillColor] (188.25, 62.36) circle (  1.16);

\path[draw=drawColor,line width= 0.4pt,line join=round,line cap=round,fill=fillColor] (188.45, 62.36) circle (  1.16);

\path[draw=drawColor,line width= 0.4pt,line join=round,line cap=round,fill=fillColor] (188.64, 62.35) circle (  1.16);

\path[draw=drawColor,line width= 0.4pt,line join=round,line cap=round,fill=fillColor] (188.84, 62.33) circle (  1.16);

\path[draw=drawColor,line width= 0.4pt,line join=round,line cap=round,fill=fillColor] (189.04, 62.33) circle (  1.16);

\path[draw=drawColor,line width= 0.4pt,line join=round,line cap=round,fill=fillColor] (189.23, 62.26) circle (  1.16);

\path[draw=drawColor,line width= 0.4pt,line join=round,line cap=round,fill=fillColor] (189.43, 62.19) circle (  1.16);

\path[draw=drawColor,line width= 0.4pt,line join=round,line cap=round,fill=fillColor] (189.62, 62.13) circle (  1.16);

\path[draw=drawColor,line width= 0.4pt,line join=round,line cap=round,fill=fillColor] (189.82, 62.07) circle (  1.16);

\path[draw=drawColor,line width= 0.4pt,line join=round,line cap=round,fill=fillColor] (190.01, 62.00) circle (  1.16);

\path[draw=drawColor,line width= 0.4pt,line join=round,line cap=round,fill=fillColor] (190.20, 61.81) circle (  1.16);

\path[draw=drawColor,line width= 0.4pt,line join=round,line cap=round,fill=fillColor] (190.39, 61.81) circle (  1.16);

\path[draw=drawColor,line width= 0.4pt,line join=round,line cap=round,fill=fillColor] (190.59, 61.68) circle (  1.16);

\path[draw=drawColor,line width= 0.4pt,line join=round,line cap=round,fill=fillColor] (190.78, 61.65) circle (  1.16);

\path[draw=drawColor,line width= 0.4pt,line join=round,line cap=round,fill=fillColor] (190.97, 61.63) circle (  1.16);

\path[draw=drawColor,line width= 0.4pt,line join=round,line cap=round,fill=fillColor] (191.16, 61.63) circle (  1.16);

\path[draw=drawColor,line width= 0.4pt,line join=round,line cap=round,fill=fillColor] (191.35, 61.63) circle (  1.16);

\path[draw=drawColor,line width= 0.4pt,line join=round,line cap=round,fill=fillColor] (191.54, 61.63) circle (  1.16);

\path[draw=drawColor,line width= 0.4pt,line join=round,line cap=round,fill=fillColor] (191.73, 61.50) circle (  1.16);

\path[draw=drawColor,line width= 0.4pt,line join=round,line cap=round,fill=fillColor] (191.91, 61.45) circle (  1.16);

\path[draw=drawColor,line width= 0.4pt,line join=round,line cap=round,fill=fillColor] (192.10, 61.40) circle (  1.16);

\path[draw=drawColor,line width= 0.4pt,line join=round,line cap=round,fill=fillColor] (192.29, 61.39) circle (  1.16);

\path[draw=drawColor,line width= 0.4pt,line join=round,line cap=round,fill=fillColor] (192.47, 61.29) circle (  1.16);

\path[draw=drawColor,line width= 0.4pt,line join=round,line cap=round,fill=fillColor] (192.66, 61.26) circle (  1.16);

\path[draw=drawColor,line width= 0.4pt,line join=round,line cap=round,fill=fillColor] (192.85, 61.24) circle (  1.16);

\path[draw=drawColor,line width= 0.4pt,line join=round,line cap=round,fill=fillColor] (193.03, 61.24) circle (  1.16);

\path[draw=drawColor,line width= 0.4pt,line join=round,line cap=round,fill=fillColor] (193.22, 61.06) circle (  1.16);

\path[draw=drawColor,line width= 0.4pt,line join=round,line cap=round,fill=fillColor] (193.40, 61.05) circle (  1.16);

\path[draw=drawColor,line width= 0.4pt,line join=round,line cap=round,fill=fillColor] (193.58, 61.02) circle (  1.16);

\path[draw=drawColor,line width= 0.4pt,line join=round,line cap=round,fill=fillColor] (193.77, 61.01) circle (  1.16);

\path[draw=drawColor,line width= 0.4pt,line join=round,line cap=round,fill=fillColor] (193.95, 60.96) circle (  1.16);

\path[draw=drawColor,line width= 0.4pt,line join=round,line cap=round,fill=fillColor] (194.13, 60.94) circle (  1.16);

\path[draw=drawColor,line width= 0.4pt,line join=round,line cap=round,fill=fillColor] (194.31, 60.84) circle (  1.16);

\path[draw=drawColor,line width= 0.4pt,line join=round,line cap=round,fill=fillColor] (194.49, 60.82) circle (  1.16);

\path[draw=drawColor,line width= 0.4pt,line join=round,line cap=round,fill=fillColor] (194.67, 60.63) circle (  1.16);

\path[draw=drawColor,line width= 0.4pt,line join=round,line cap=round,fill=fillColor] (194.85, 60.48) circle (  1.16);

\path[draw=drawColor,line width= 0.4pt,line join=round,line cap=round,fill=fillColor] (195.03, 60.43) circle (  1.16);

\path[draw=drawColor,line width= 0.4pt,line join=round,line cap=round,fill=fillColor] (195.21, 60.41) circle (  1.16);

\path[draw=drawColor,line width= 0.4pt,line join=round,line cap=round,fill=fillColor] (195.39, 60.37) circle (  1.16);

\path[draw=drawColor,line width= 0.4pt,line join=round,line cap=round,fill=fillColor] (195.57, 60.33) circle (  1.16);

\path[draw=drawColor,line width= 0.4pt,line join=round,line cap=round,fill=fillColor] (195.75, 60.29) circle (  1.16);

\path[draw=drawColor,line width= 0.4pt,line join=round,line cap=round,fill=fillColor] (195.92, 60.29) circle (  1.16);

\path[draw=drawColor,line width= 0.4pt,line join=round,line cap=round,fill=fillColor] (196.10, 60.14) circle (  1.16);

\path[draw=drawColor,line width= 0.4pt,line join=round,line cap=round,fill=fillColor] (196.28, 60.10) circle (  1.16);

\path[draw=drawColor,line width= 0.4pt,line join=round,line cap=round,fill=fillColor] (196.45, 60.06) circle (  1.16);

\path[draw=drawColor,line width= 0.4pt,line join=round,line cap=round,fill=fillColor] (196.63, 59.98) circle (  1.16);

\path[draw=drawColor,line width= 0.4pt,line join=round,line cap=round,fill=fillColor] (196.80, 59.97) circle (  1.16);

\path[draw=drawColor,line width= 0.4pt,line join=round,line cap=round,fill=fillColor] (196.98, 59.95) circle (  1.16);

\path[draw=drawColor,line width= 0.4pt,line join=round,line cap=round,fill=fillColor] (197.15, 59.91) circle (  1.16);

\path[draw=drawColor,line width= 0.4pt,line join=round,line cap=round,fill=fillColor] (197.33, 59.86) circle (  1.16);

\path[draw=drawColor,line width= 0.4pt,line join=round,line cap=round,fill=fillColor] (197.50, 59.68) circle (  1.16);

\path[draw=drawColor,line width= 0.4pt,line join=round,line cap=round,fill=fillColor] (197.67, 59.52) circle (  1.16);

\path[draw=drawColor,line width= 0.4pt,line join=round,line cap=round,fill=fillColor] (197.85, 59.47) circle (  1.16);

\path[draw=drawColor,line width= 0.4pt,line join=round,line cap=round,fill=fillColor] (198.02, 59.44) circle (  1.16);

\path[draw=drawColor,line width= 0.4pt,line join=round,line cap=round,fill=fillColor] (198.19, 59.40) circle (  1.16);

\path[draw=drawColor,line width= 0.4pt,line join=round,line cap=round,fill=fillColor] (198.36, 59.38) circle (  1.16);

\path[draw=drawColor,line width= 0.4pt,line join=round,line cap=round,fill=fillColor] (198.53, 59.32) circle (  1.16);

\path[draw=drawColor,line width= 0.4pt,line join=round,line cap=round,fill=fillColor] (198.70, 59.19) circle (  1.16);

\path[draw=drawColor,line width= 0.4pt,line join=round,line cap=round,fill=fillColor] (198.87, 59.01) circle (  1.16);

\path[draw=drawColor,line width= 0.4pt,line join=round,line cap=round,fill=fillColor] (199.04, 58.96) circle (  1.16);

\path[draw=drawColor,line width= 0.4pt,line join=round,line cap=round,fill=fillColor] (199.21, 58.95) circle (  1.16);

\path[draw=drawColor,line width= 0.4pt,line join=round,line cap=round,fill=fillColor] (199.38, 58.94) circle (  1.16);

\path[draw=drawColor,line width= 0.4pt,line join=round,line cap=round,fill=fillColor] (199.55, 58.84) circle (  1.16);

\path[draw=drawColor,line width= 0.4pt,line join=round,line cap=round,fill=fillColor] (199.72, 58.82) circle (  1.16);

\path[draw=drawColor,line width= 0.4pt,line join=round,line cap=round,fill=fillColor] (199.88, 58.67) circle (  1.16);

\path[draw=drawColor,line width= 0.4pt,line join=round,line cap=round,fill=fillColor] (200.05, 58.61) circle (  1.16);

\path[draw=drawColor,line width= 0.4pt,line join=round,line cap=round,fill=fillColor] (200.22, 58.49) circle (  1.16);

\path[draw=drawColor,line width= 0.4pt,line join=round,line cap=round,fill=fillColor] (200.38, 58.27) circle (  1.16);

\path[draw=drawColor,line width= 0.4pt,line join=round,line cap=round,fill=fillColor] (200.55, 58.18) circle (  1.16);

\path[draw=drawColor,line width= 0.4pt,line join=round,line cap=round,fill=fillColor] (200.72, 58.07) circle (  1.16);

\path[draw=drawColor,line width= 0.4pt,line join=round,line cap=round,fill=fillColor] (200.88, 57.98) circle (  1.16);

\path[draw=drawColor,line width= 0.4pt,line join=round,line cap=round,fill=fillColor] (201.05, 57.90) circle (  1.16);

\path[draw=drawColor,line width= 0.4pt,line join=round,line cap=round,fill=fillColor] (201.21, 57.80) circle (  1.16);

\path[draw=drawColor,line width= 0.4pt,line join=round,line cap=round,fill=fillColor] (201.37, 57.78) circle (  1.16);

\path[draw=drawColor,line width= 0.4pt,line join=round,line cap=round,fill=fillColor] (201.54, 57.72) circle (  1.16);

\path[draw=drawColor,line width= 0.4pt,line join=round,line cap=round,fill=fillColor] (201.70, 57.64) circle (  1.16);

\path[draw=drawColor,line width= 0.4pt,line join=round,line cap=round,fill=fillColor] (201.86, 57.63) circle (  1.16);

\path[draw=drawColor,line width= 0.4pt,line join=round,line cap=round,fill=fillColor] (202.03, 57.60) circle (  1.16);

\path[draw=drawColor,line width= 0.4pt,line join=round,line cap=round,fill=fillColor] (202.19, 57.55) circle (  1.16);

\path[draw=drawColor,line width= 0.4pt,line join=round,line cap=round,fill=fillColor] (202.35, 57.39) circle (  1.16);

\path[draw=drawColor,line width= 0.4pt,line join=round,line cap=round,fill=fillColor] (202.51, 57.13) circle (  1.16);

\path[draw=drawColor,line width= 0.4pt,line join=round,line cap=round,fill=fillColor] (202.67, 57.00) circle (  1.16);

\path[draw=drawColor,line width= 0.4pt,line join=round,line cap=round,fill=fillColor] (202.83, 56.70) circle (  1.16);

\path[draw=drawColor,line width= 0.4pt,line join=round,line cap=round,fill=fillColor] (202.99, 56.66) circle (  1.16);

\path[draw=drawColor,line width= 0.4pt,line join=round,line cap=round,fill=fillColor] (203.16, 56.60) circle (  1.16);

\path[draw=drawColor,line width= 0.4pt,line join=round,line cap=round,fill=fillColor] (203.31, 56.49) circle (  1.16);

\path[draw=drawColor,line width= 0.4pt,line join=round,line cap=round,fill=fillColor] (203.47, 56.37) circle (  1.16);

\path[draw=drawColor,line width= 0.4pt,line join=round,line cap=round,fill=fillColor] (203.63, 56.35) circle (  1.16);

\path[draw=drawColor,line width= 0.4pt,line join=round,line cap=round,fill=fillColor] (203.79, 56.16) circle (  1.16);

\path[draw=drawColor,line width= 0.4pt,line join=round,line cap=round,fill=fillColor] (203.95, 55.71) circle (  1.16);

\path[draw=drawColor,line width= 0.4pt,line join=round,line cap=round,fill=fillColor] (204.11, 55.62) circle (  1.16);

\path[draw=drawColor,line width= 0.4pt,line join=round,line cap=round,fill=fillColor] (204.27, 55.50) circle (  1.16);

\path[draw=drawColor,line width= 0.4pt,line join=round,line cap=round,fill=fillColor] (204.42, 55.25) circle (  1.16);

\path[draw=drawColor,line width= 0.4pt,line join=round,line cap=round,fill=fillColor] (204.58, 55.11) circle (  1.16);

\path[draw=drawColor,line width= 0.4pt,line join=round,line cap=round,fill=fillColor] (204.74, 54.96) circle (  1.16);

\path[draw=drawColor,line width= 0.4pt,line join=round,line cap=round,fill=fillColor] (204.89, 53.36) circle (  1.16);

\path[draw=drawColor,line width= 0.4pt,line join=round,line cap=round,fill=fillColor] (205.05, 52.84) circle (  1.16);

\path[draw=drawColor,line width= 0.4pt,line join=round,line cap=round,fill=fillColor] (205.21, 52.83) circle (  1.16);

\path[draw=drawColor,line width= 0.4pt,line join=round,line cap=round,fill=fillColor] (205.36, 52.57) circle (  1.16);

\path[draw=drawColor,line width= 0.4pt,line join=round,line cap=round,fill=fillColor] (205.52, 52.33) circle (  1.16);

\path[draw=drawColor,line width= 0.4pt,line join=round,line cap=round,fill=fillColor] (205.67, 52.17) circle (  1.16);

\path[draw=drawColor,line width= 0.4pt,line join=round,line cap=round,fill=fillColor] (205.83, 51.66) circle (  1.16);

\path[draw=drawColor,line width= 0.4pt,line join=round,line cap=round,fill=fillColor] (205.98, 51.32) circle (  1.16);

\path[draw=drawColor,line width= 0.4pt,line join=round,line cap=round,fill=fillColor] (206.13, 50.91) circle (  1.16);

\path[draw=drawColor,line width= 0.4pt,line join=round,line cap=round,fill=fillColor] (206.29, 49.67) circle (  1.16);

\path[draw=drawColor,line width= 0.4pt,line join=round,line cap=round,fill=fillColor] (206.44, 49.41) circle (  1.16);

\path[draw=drawColor,line width= 0.4pt,line join=round,line cap=round,fill=fillColor] (206.59, 44.78) circle (  1.16);

\path[draw=drawColor,line width= 0.4pt,line join=round,line cap=round,fill=fillColor] (206.74, 44.78) circle (  1.16);

\path[draw=drawColor,line width= 0.4pt,line join=round,line cap=round,fill=fillColor] (206.90, 44.78) circle (  1.16);

\path[draw=drawColor,line width= 0.4pt,line join=round,line cap=round,fill=fillColor] (207.05, 44.78) circle (  1.16);

\path[draw=drawColor,line width= 0.4pt,line join=round,line cap=round,fill=fillColor] (207.20, 44.78) circle (  1.16);

\path[draw=drawColor,line width= 0.4pt,line join=round,line cap=round,fill=fillColor] (207.35, 44.78) circle (  1.16);

\path[draw=drawColor,line width= 0.4pt,line join=round,line cap=round,fill=fillColor] (207.50, 44.78) circle (  1.16);

\path[draw=drawColor,line width= 0.4pt,line join=round,line cap=round,fill=fillColor] (207.65, 44.78) circle (  1.16);

\path[draw=drawColor,line width= 0.4pt,line join=round,line cap=round,fill=fillColor] (207.80, 44.78) circle (  1.16);

\path[draw=drawColor,line width= 0.4pt,line join=round,line cap=round,fill=fillColor] (207.95, 44.78) circle (  1.16);

\path[draw=drawColor,line width= 0.4pt,line join=round,line cap=round,fill=fillColor] (208.10, 44.78) circle (  1.16);

\path[draw=drawColor,line width= 0.4pt,line join=round,line cap=round,fill=fillColor] (208.25, 44.78) circle (  1.16);

\path[draw=drawColor,line width= 0.4pt,line join=round,line cap=round,fill=fillColor] (208.40, 44.78) circle (  1.16);

\path[draw=drawColor,line width= 0.4pt,line join=round,line cap=round,fill=fillColor] (208.55, 44.78) circle (  1.16);

\path[draw=drawColor,line width= 0.4pt,line join=round,line cap=round,fill=fillColor] (208.70, 44.78) circle (  1.16);

\path[draw=drawColor,line width= 0.4pt,line join=round,line cap=round,fill=fillColor] (208.85, 44.78) circle (  1.16);

\path[draw=drawColor,line width= 0.4pt,line join=round,line cap=round,fill=fillColor] (209.00, 44.78) circle (  1.16);

\path[draw=drawColor,line width= 0.4pt,line join=round,line cap=round,fill=fillColor] (209.14, 44.78) circle (  1.16);

\path[draw=drawColor,line width= 0.4pt,line join=round,line cap=round,fill=fillColor] (209.29, 44.78) circle (  1.16);

\path[draw=drawColor,line width= 0.4pt,line join=round,line cap=round,fill=fillColor] (209.44, 44.78) circle (  1.16);

\path[draw=drawColor,line width= 0.4pt,line join=round,line cap=round,fill=fillColor] (209.59, 44.78) circle (  1.16);

\path[draw=drawColor,line width= 0.4pt,line join=round,line cap=round,fill=fillColor] (209.73, 44.78) circle (  1.16);

\path[draw=drawColor,line width= 0.4pt,line join=round,line cap=round,fill=fillColor] (209.88, 44.78) circle (  1.16);

\path[draw=drawColor,line width= 0.4pt,line join=round,line cap=round,fill=fillColor] (210.02, 44.78) circle (  1.16);

\path[draw=drawColor,line width= 0.4pt,line join=round,line cap=round,fill=fillColor] (210.17, 44.78) circle (  1.16);

\path[draw=drawColor,line width= 0.4pt,line join=round,line cap=round,fill=fillColor] (210.32, 44.78) circle (  1.16);

\path[draw=drawColor,line width= 0.4pt,line join=round,line cap=round,fill=fillColor] (210.46, 44.78) circle (  1.16);

\path[draw=drawColor,line width= 0.4pt,line join=round,line cap=round,fill=fillColor] (210.61, 44.78) circle (  1.16);

\path[draw=drawColor,line width= 0.4pt,line join=round,line cap=round,fill=fillColor] (210.75, 44.78) circle (  1.16);

\path[draw=drawColor,line width= 0.4pt,line join=round,line cap=round,fill=fillColor] (210.89, 44.78) circle (  1.16);

\path[draw=drawColor,line width= 0.4pt,line join=round,line cap=round,fill=fillColor] (211.04, 44.78) circle (  1.16);

\path[draw=drawColor,line width= 0.4pt,line join=round,line cap=round,fill=fillColor] (211.18, 44.78) circle (  1.16);

\path[draw=drawColor,line width= 0.4pt,line join=round,line cap=round,fill=fillColor] (211.33, 44.78) circle (  1.16);

\path[draw=drawColor,line width= 0.4pt,line join=round,line cap=round,fill=fillColor] (211.47, 44.78) circle (  1.16);

\path[draw=drawColor,line width= 0.4pt,line join=round,line cap=round,fill=fillColor] (211.61, 44.78) circle (  1.16);

\path[draw=drawColor,line width= 0.4pt,line join=round,line cap=round,fill=fillColor] (211.76, 44.78) circle (  1.16);

\path[draw=drawColor,line width= 0.4pt,line join=round,line cap=round,fill=fillColor] (211.90, 44.78) circle (  1.16);

\path[draw=drawColor,line width= 0.4pt,line join=round,line cap=round,fill=fillColor] (212.04, 44.78) circle (  1.16);

\path[draw=drawColor,line width= 0.4pt,line join=round,line cap=round,fill=fillColor] (212.18, 44.78) circle (  1.16);

\path[draw=drawColor,line width= 0.4pt,line join=round,line cap=round,fill=fillColor] (212.32, 44.78) circle (  1.16);

\path[draw=drawColor,line width= 0.4pt,line join=round,line cap=round,fill=fillColor] (212.47, 44.78) circle (  1.16);

\path[draw=drawColor,line width= 0.4pt,line join=round,line cap=round,fill=fillColor] (212.61, 44.78) circle (  1.16);

\path[draw=drawColor,line width= 0.4pt,line join=round,line cap=round,fill=fillColor] (212.75, 44.78) circle (  1.16);

\path[draw=drawColor,line width= 0.4pt,line join=round,line cap=round,fill=fillColor] (212.89, 44.78) circle (  1.16);

\path[draw=drawColor,line width= 0.4pt,line join=round,line cap=round,fill=fillColor] (213.03, 44.78) circle (  1.16);

\path[draw=drawColor,line width= 0.4pt,line join=round,line cap=round,fill=fillColor] (213.17, 44.78) circle (  1.16);

\path[draw=drawColor,line width= 0.4pt,line join=round,line cap=round,fill=fillColor] (213.31, 44.78) circle (  1.16);

\path[draw=drawColor,line width= 0.4pt,line join=round,line cap=round,fill=fillColor] (213.45, 44.78) circle (  1.16);

\path[draw=drawColor,line width= 0.4pt,line join=round,line cap=round,fill=fillColor] (213.59, 44.78) circle (  1.16);

\path[draw=drawColor,line width= 0.4pt,line join=round,line cap=round,fill=fillColor] (213.73, 44.78) circle (  1.16);

\path[draw=drawColor,line width= 0.4pt,line join=round,line cap=round,fill=fillColor] (213.87, 44.78) circle (  1.16);

\path[draw=drawColor,line width= 0.4pt,line join=round,line cap=round,fill=fillColor] (214.00, 44.78) circle (  1.16);

\path[draw=drawColor,line width= 0.4pt,line join=round,line cap=round,fill=fillColor] (214.14, 44.78) circle (  1.16);

\path[draw=drawColor,line width= 0.4pt,line join=round,line cap=round,fill=fillColor] (214.28, 44.78) circle (  1.16);

\path[draw=drawColor,line width= 0.4pt,line join=round,line cap=round,fill=fillColor] (214.42, 44.78) circle (  1.16);

\path[draw=drawColor,line width= 0.4pt,line join=round,line cap=round,fill=fillColor] (214.56, 44.78) circle (  1.16);

\path[draw=drawColor,line width= 0.4pt,line join=round,line cap=round,fill=fillColor] (214.69, 44.78) circle (  1.16);

\path[draw=drawColor,line width= 0.4pt,line join=round,line cap=round,fill=fillColor] (214.83, 44.78) circle (  1.16);

\path[draw=drawColor,line width= 0.4pt,line join=round,line cap=round,fill=fillColor] (214.97, 44.78) circle (  1.16);

\path[draw=drawColor,line width= 0.4pt,line join=round,line cap=round,fill=fillColor] (215.11, 44.78) circle (  1.16);

\path[draw=drawColor,line width= 0.4pt,line join=round,line cap=round,fill=fillColor] (215.24, 44.78) circle (  1.16);

\path[draw=drawColor,line width= 0.4pt,line join=round,line cap=round,fill=fillColor] (215.38, 44.78) circle (  1.16);

\path[draw=drawColor,line width= 0.4pt,line join=round,line cap=round,fill=fillColor] (215.51, 44.78) circle (  1.16);

\path[draw=drawColor,line width= 0.4pt,line join=round,line cap=round,fill=fillColor] (215.65, 44.78) circle (  1.16);

\path[draw=drawColor,line width= 0.4pt,line join=round,line cap=round,fill=fillColor] (215.79, 44.78) circle (  1.16);

\path[draw=drawColor,line width= 0.4pt,line join=round,line cap=round,fill=fillColor] (215.92, 44.78) circle (  1.16);

\path[draw=drawColor,line width= 0.4pt,line join=round,line cap=round,fill=fillColor] (216.06, 44.78) circle (  1.16);

\path[draw=drawColor,line width= 0.4pt,line join=round,line cap=round,fill=fillColor] (216.19, 44.78) circle (  1.16);

\path[draw=drawColor,line width= 0.4pt,line join=round,line cap=round,fill=fillColor] (216.33, 44.78) circle (  1.16);

\path[draw=drawColor,line width= 0.4pt,line join=round,line cap=round,fill=fillColor] (216.46, 44.78) circle (  1.16);

\path[draw=drawColor,line width= 0.4pt,line join=round,line cap=round,fill=fillColor] (216.59, 44.78) circle (  1.16);

\path[draw=drawColor,line width= 0.4pt,line join=round,line cap=round,fill=fillColor] (216.73, 44.78) circle (  1.16);

\path[draw=drawColor,line width= 0.4pt,line join=round,line cap=round,fill=fillColor] (216.86, 44.78) circle (  1.16);

\path[draw=drawColor,line width= 0.4pt,line join=round,line cap=round,fill=fillColor] (217.00, 44.78) circle (  1.16);

\path[draw=drawColor,line width= 0.4pt,line join=round,line cap=round,fill=fillColor] (217.13, 44.78) circle (  1.16);

\path[draw=drawColor,line width= 0.4pt,line join=round,line cap=round,fill=fillColor] (217.26, 44.78) circle (  1.16);

\path[draw=drawColor,line width= 0.4pt,line join=round,line cap=round,fill=fillColor] (217.40, 44.78) circle (  1.16);

\path[draw=drawColor,line width= 0.4pt,line join=round,line cap=round,fill=fillColor] (217.53, 44.78) circle (  1.16);

\path[draw=drawColor,line width= 0.4pt,line join=round,line cap=round,fill=fillColor] (217.66, 44.78) circle (  1.16);

\path[draw=drawColor,line width= 0.4pt,line join=round,line cap=round,fill=fillColor] (217.79, 44.78) circle (  1.16);

\path[draw=drawColor,line width= 0.4pt,line join=round,line cap=round,fill=fillColor] (217.92, 44.78) circle (  1.16);

\path[draw=drawColor,line width= 0.4pt,line join=round,line cap=round,fill=fillColor] (218.06, 44.78) circle (  1.16);

\path[draw=drawColor,line width= 0.4pt,line join=round,line cap=round,fill=fillColor] (218.19, 44.78) circle (  1.16);

\path[draw=drawColor,line width= 0.4pt,line join=round,line cap=round,fill=fillColor] (218.32, 44.78) circle (  1.16);

\path[draw=drawColor,line width= 0.4pt,line join=round,line cap=round,fill=fillColor] (218.45, 44.78) circle (  1.16);

\path[draw=drawColor,line width= 0.4pt,line join=round,line cap=round,fill=fillColor] (218.58, 44.78) circle (  1.16);

\path[draw=drawColor,line width= 0.4pt,line join=round,line cap=round,fill=fillColor] (218.71, 44.78) circle (  1.16);

\path[draw=drawColor,line width= 0.4pt,line join=round,line cap=round,fill=fillColor] (218.84, 44.78) circle (  1.16);

\path[draw=drawColor,line width= 0.4pt,line join=round,line cap=round,fill=fillColor] (218.97, 44.78) circle (  1.16);

\path[draw=drawColor,line width= 0.4pt,line join=round,line cap=round,fill=fillColor] (219.10, 44.78) circle (  1.16);
\definecolor[named]{drawColor}{rgb}{0.65,0.34,0.16}
\definecolor[named]{fillColor}{rgb}{0.65,0.34,0.16}

\path[draw=drawColor,line width= 0.4pt,line join=round,line cap=round,fill=fillColor] ( 78.97, 84.15) circle (  1.16);

\path[draw=drawColor,line width= 0.4pt,line join=round,line cap=round,fill=fillColor] ( 84.61, 72.13) circle (  1.16);

\path[draw=drawColor,line width= 0.4pt,line join=round,line cap=round,fill=fillColor] ( 88.57, 69.19) circle (  1.16);

\path[draw=drawColor,line width= 0.4pt,line join=round,line cap=round,fill=fillColor] ( 91.71, 67.77) circle (  1.16);

\path[draw=drawColor,line width= 0.4pt,line join=round,line cap=round,fill=fillColor] ( 94.37, 67.05) circle (  1.16);

\path[draw=drawColor,line width= 0.4pt,line join=round,line cap=round,fill=fillColor] ( 96.70, 66.62) circle (  1.16);

\path[draw=drawColor,line width= 0.4pt,line join=round,line cap=round,fill=fillColor] ( 98.78, 66.18) circle (  1.16);

\path[draw=drawColor,line width= 0.4pt,line join=round,line cap=round,fill=fillColor] (100.67, 66.17) circle (  1.16);

\path[draw=drawColor,line width= 0.4pt,line join=round,line cap=round,fill=fillColor] (102.40, 65.94) circle (  1.16);

\path[draw=drawColor,line width= 0.4pt,line join=round,line cap=round,fill=fillColor] (104.02, 65.49) circle (  1.16);

\path[draw=drawColor,line width= 0.4pt,line join=round,line cap=round,fill=fillColor] (105.53, 64.38) circle (  1.16);

\path[draw=drawColor,line width= 0.4pt,line join=round,line cap=round,fill=fillColor] (106.95, 63.56) circle (  1.16);

\path[draw=drawColor,line width= 0.4pt,line join=round,line cap=round,fill=fillColor] (108.29, 63.18) circle (  1.16);

\path[draw=drawColor,line width= 0.4pt,line join=round,line cap=round,fill=fillColor] (109.56, 63.01) circle (  1.16);

\path[draw=drawColor,line width= 0.4pt,line join=round,line cap=round,fill=fillColor] (110.78, 62.98) circle (  1.16);

\path[draw=drawColor,line width= 0.4pt,line join=round,line cap=round,fill=fillColor] (111.94, 62.90) circle (  1.16);

\path[draw=drawColor,line width= 0.4pt,line join=round,line cap=round,fill=fillColor] (113.06, 62.84) circle (  1.16);

\path[draw=drawColor,line width= 0.4pt,line join=round,line cap=round,fill=fillColor] (114.13, 62.79) circle (  1.16);

\path[draw=drawColor,line width= 0.4pt,line join=round,line cap=round,fill=fillColor] (115.17, 62.21) circle (  1.16);

\path[draw=drawColor,line width= 0.4pt,line join=round,line cap=round,fill=fillColor] (116.17, 62.08) circle (  1.16);

\path[draw=drawColor,line width= 0.4pt,line join=round,line cap=round,fill=fillColor] (117.13, 62.01) circle (  1.16);

\path[draw=drawColor,line width= 0.4pt,line join=round,line cap=round,fill=fillColor] (118.07, 61.90) circle (  1.16);

\path[draw=drawColor,line width= 0.4pt,line join=round,line cap=round,fill=fillColor] (118.97, 61.18) circle (  1.16);

\path[draw=drawColor,line width= 0.4pt,line join=round,line cap=round,fill=fillColor] (119.86, 61.04) circle (  1.16);

\path[draw=drawColor,line width= 0.4pt,line join=round,line cap=round,fill=fillColor] (120.71, 60.95) circle (  1.16);

\path[draw=drawColor,line width= 0.4pt,line join=round,line cap=round,fill=fillColor] (121.55, 60.90) circle (  1.16);

\path[draw=drawColor,line width= 0.4pt,line join=round,line cap=round,fill=fillColor] (122.36, 60.84) circle (  1.16);

\path[draw=drawColor,line width= 0.4pt,line join=round,line cap=round,fill=fillColor] (123.16, 60.51) circle (  1.16);

\path[draw=drawColor,line width= 0.4pt,line join=round,line cap=round,fill=fillColor] (123.93, 60.28) circle (  1.16);

\path[draw=drawColor,line width= 0.4pt,line join=round,line cap=round,fill=fillColor] (124.69, 60.19) circle (  1.16);

\path[draw=drawColor,line width= 0.4pt,line join=round,line cap=round,fill=fillColor] (125.43, 59.90) circle (  1.16);

\path[draw=drawColor,line width= 0.4pt,line join=round,line cap=round,fill=fillColor] (126.15, 59.83) circle (  1.16);

\path[draw=drawColor,line width= 0.4pt,line join=round,line cap=round,fill=fillColor] (126.86, 59.71) circle (  1.16);

\path[draw=drawColor,line width= 0.4pt,line join=round,line cap=round,fill=fillColor] (127.56, 59.67) circle (  1.16);

\path[draw=drawColor,line width= 0.4pt,line join=round,line cap=round,fill=fillColor] (128.24, 59.54) circle (  1.16);

\path[draw=drawColor,line width= 0.4pt,line join=round,line cap=round,fill=fillColor] (128.91, 59.39) circle (  1.16);

\path[draw=drawColor,line width= 0.4pt,line join=round,line cap=round,fill=fillColor] (129.57, 59.30) circle (  1.16);

\path[draw=drawColor,line width= 0.4pt,line join=round,line cap=round,fill=fillColor] (130.21, 59.30) circle (  1.16);

\path[draw=drawColor,line width= 0.4pt,line join=round,line cap=round,fill=fillColor] (130.85, 59.18) circle (  1.16);

\path[draw=drawColor,line width= 0.4pt,line join=round,line cap=round,fill=fillColor] (131.47, 59.10) circle (  1.16);

\path[draw=drawColor,line width= 0.4pt,line join=round,line cap=round,fill=fillColor] (132.09, 59.04) circle (  1.16);

\path[draw=drawColor,line width= 0.4pt,line join=round,line cap=round,fill=fillColor] (132.69, 58.92) circle (  1.16);

\path[draw=drawColor,line width= 0.4pt,line join=round,line cap=round,fill=fillColor] (133.28, 58.65) circle (  1.16);

\path[draw=drawColor,line width= 0.4pt,line join=round,line cap=round,fill=fillColor] (133.87, 58.61) circle (  1.16);

\path[draw=drawColor,line width= 0.4pt,line join=round,line cap=round,fill=fillColor] (134.44, 58.47) circle (  1.16);

\path[draw=drawColor,line width= 0.4pt,line join=round,line cap=round,fill=fillColor] (135.01, 58.46) circle (  1.16);

\path[draw=drawColor,line width= 0.4pt,line join=round,line cap=round,fill=fillColor] (135.57, 58.43) circle (  1.16);

\path[draw=drawColor,line width= 0.4pt,line join=round,line cap=round,fill=fillColor] (136.12, 58.36) circle (  1.16);

\path[draw=drawColor,line width= 0.4pt,line join=round,line cap=round,fill=fillColor] (136.67, 58.31) circle (  1.16);

\path[draw=drawColor,line width= 0.4pt,line join=round,line cap=round,fill=fillColor] (137.20, 58.22) circle (  1.16);

\path[draw=drawColor,line width= 0.4pt,line join=round,line cap=round,fill=fillColor] (137.73, 58.21) circle (  1.16);

\path[draw=drawColor,line width= 0.4pt,line join=round,line cap=round,fill=fillColor] (138.25, 58.12) circle (  1.16);

\path[draw=drawColor,line width= 0.4pt,line join=round,line cap=round,fill=fillColor] (138.77, 58.00) circle (  1.16);

\path[draw=drawColor,line width= 0.4pt,line join=round,line cap=round,fill=fillColor] (139.28, 57.85) circle (  1.16);

\path[draw=drawColor,line width= 0.4pt,line join=round,line cap=round,fill=fillColor] (139.78, 57.69) circle (  1.16);

\path[draw=drawColor,line width= 0.4pt,line join=round,line cap=round,fill=fillColor] (140.28, 57.68) circle (  1.16);

\path[draw=drawColor,line width= 0.4pt,line join=round,line cap=round,fill=fillColor] (140.77, 57.67) circle (  1.16);

\path[draw=drawColor,line width= 0.4pt,line join=round,line cap=round,fill=fillColor] (141.26, 57.66) circle (  1.16);

\path[draw=drawColor,line width= 0.4pt,line join=round,line cap=round,fill=fillColor] (141.74, 57.52) circle (  1.16);

\path[draw=drawColor,line width= 0.4pt,line join=round,line cap=round,fill=fillColor] (142.21, 57.48) circle (  1.16);

\path[draw=drawColor,line width= 0.4pt,line join=round,line cap=round,fill=fillColor] (142.68, 57.45) circle (  1.16);

\path[draw=drawColor,line width= 0.4pt,line join=round,line cap=round,fill=fillColor] (143.14, 57.44) circle (  1.16);

\path[draw=drawColor,line width= 0.4pt,line join=round,line cap=round,fill=fillColor] (143.60, 57.43) circle (  1.16);

\path[draw=drawColor,line width= 0.4pt,line join=round,line cap=round,fill=fillColor] (144.06, 57.43) circle (  1.16);

\path[draw=drawColor,line width= 0.4pt,line join=round,line cap=round,fill=fillColor] (144.51, 57.36) circle (  1.16);

\path[draw=drawColor,line width= 0.4pt,line join=round,line cap=round,fill=fillColor] (144.95, 57.32) circle (  1.16);

\path[draw=drawColor,line width= 0.4pt,line join=round,line cap=round,fill=fillColor] (145.39, 57.28) circle (  1.16);

\path[draw=drawColor,line width= 0.4pt,line join=round,line cap=round,fill=fillColor] (145.83, 57.24) circle (  1.16);

\path[draw=drawColor,line width= 0.4pt,line join=round,line cap=round,fill=fillColor] (146.26, 57.23) circle (  1.16);

\path[draw=drawColor,line width= 0.4pt,line join=round,line cap=round,fill=fillColor] (146.69, 57.21) circle (  1.16);

\path[draw=drawColor,line width= 0.4pt,line join=round,line cap=round,fill=fillColor] (147.11, 57.09) circle (  1.16);

\path[draw=drawColor,line width= 0.4pt,line join=round,line cap=round,fill=fillColor] (147.53, 56.96) circle (  1.16);

\path[draw=drawColor,line width= 0.4pt,line join=round,line cap=round,fill=fillColor] (147.95, 56.93) circle (  1.16);

\path[draw=drawColor,line width= 0.4pt,line join=round,line cap=round,fill=fillColor] (148.36, 56.90) circle (  1.16);

\path[draw=drawColor,line width= 0.4pt,line join=round,line cap=round,fill=fillColor] (148.77, 56.90) circle (  1.16);

\path[draw=drawColor,line width= 0.4pt,line join=round,line cap=round,fill=fillColor] (149.17, 56.88) circle (  1.16);

\path[draw=drawColor,line width= 0.4pt,line join=round,line cap=round,fill=fillColor] (149.57, 56.87) circle (  1.16);

\path[draw=drawColor,line width= 0.4pt,line join=round,line cap=round,fill=fillColor] (149.97, 56.81) circle (  1.16);

\path[draw=drawColor,line width= 0.4pt,line join=round,line cap=round,fill=fillColor] (150.37, 56.80) circle (  1.16);

\path[draw=drawColor,line width= 0.4pt,line join=round,line cap=round,fill=fillColor] (150.76, 56.77) circle (  1.16);

\path[draw=drawColor,line width= 0.4pt,line join=round,line cap=round,fill=fillColor] (151.15, 56.67) circle (  1.16);

\path[draw=drawColor,line width= 0.4pt,line join=round,line cap=round,fill=fillColor] (151.53, 56.61) circle (  1.16);

\path[draw=drawColor,line width= 0.4pt,line join=round,line cap=round,fill=fillColor] (151.91, 56.53) circle (  1.16);

\path[draw=drawColor,line width= 0.4pt,line join=round,line cap=round,fill=fillColor] (152.29, 56.52) circle (  1.16);

\path[draw=drawColor,line width= 0.4pt,line join=round,line cap=round,fill=fillColor] (152.67, 56.49) circle (  1.16);

\path[draw=drawColor,line width= 0.4pt,line join=round,line cap=round,fill=fillColor] (153.04, 56.40) circle (  1.16);

\path[draw=drawColor,line width= 0.4pt,line join=round,line cap=round,fill=fillColor] (153.41, 56.40) circle (  1.16);

\path[draw=drawColor,line width= 0.4pt,line join=round,line cap=round,fill=fillColor] (153.78, 56.37) circle (  1.16);

\path[draw=drawColor,line width= 0.4pt,line join=round,line cap=round,fill=fillColor] (154.14, 56.26) circle (  1.16);

\path[draw=drawColor,line width= 0.4pt,line join=round,line cap=round,fill=fillColor] (154.50, 56.09) circle (  1.16);

\path[draw=drawColor,line width= 0.4pt,line join=round,line cap=round,fill=fillColor] (154.86, 56.04) circle (  1.16);

\path[draw=drawColor,line width= 0.4pt,line join=round,line cap=round,fill=fillColor] (155.22, 56.03) circle (  1.16);

\path[draw=drawColor,line width= 0.4pt,line join=round,line cap=round,fill=fillColor] (155.57, 56.00) circle (  1.16);

\path[draw=drawColor,line width= 0.4pt,line join=round,line cap=round,fill=fillColor] (155.92, 55.98) circle (  1.16);

\path[draw=drawColor,line width= 0.4pt,line join=round,line cap=round,fill=fillColor] (156.27, 55.98) circle (  1.16);

\path[draw=drawColor,line width= 0.4pt,line join=round,line cap=round,fill=fillColor] (156.62, 55.93) circle (  1.16);

\path[draw=drawColor,line width= 0.4pt,line join=round,line cap=round,fill=fillColor] (156.96, 55.93) circle (  1.16);

\path[draw=drawColor,line width= 0.4pt,line join=round,line cap=round,fill=fillColor] (157.30, 55.92) circle (  1.16);

\path[draw=drawColor,line width= 0.4pt,line join=round,line cap=round,fill=fillColor] (157.64, 55.90) circle (  1.16);

\path[draw=drawColor,line width= 0.4pt,line join=round,line cap=round,fill=fillColor] (157.98, 55.82) circle (  1.16);

\path[draw=drawColor,line width= 0.4pt,line join=round,line cap=round,fill=fillColor] (158.31, 55.78) circle (  1.16);

\path[draw=drawColor,line width= 0.4pt,line join=round,line cap=round,fill=fillColor] (158.64, 55.76) circle (  1.16);

\path[draw=drawColor,line width= 0.4pt,line join=round,line cap=round,fill=fillColor] (158.97, 55.62) circle (  1.16);

\path[draw=drawColor,line width= 0.4pt,line join=round,line cap=round,fill=fillColor] (159.30, 55.59) circle (  1.16);

\path[draw=drawColor,line width= 0.4pt,line join=round,line cap=round,fill=fillColor] (159.63, 55.54) circle (  1.16);

\path[draw=drawColor,line width= 0.4pt,line join=round,line cap=round,fill=fillColor] (159.95, 55.52) circle (  1.16);

\path[draw=drawColor,line width= 0.4pt,line join=round,line cap=round,fill=fillColor] (160.27, 55.45) circle (  1.16);

\path[draw=drawColor,line width= 0.4pt,line join=round,line cap=round,fill=fillColor] (160.59, 55.43) circle (  1.16);

\path[draw=drawColor,line width= 0.4pt,line join=round,line cap=round,fill=fillColor] (160.91, 55.41) circle (  1.16);

\path[draw=drawColor,line width= 0.4pt,line join=round,line cap=round,fill=fillColor] (161.23, 55.22) circle (  1.16);

\path[draw=drawColor,line width= 0.4pt,line join=round,line cap=round,fill=fillColor] (161.54, 55.20) circle (  1.16);

\path[draw=drawColor,line width= 0.4pt,line join=round,line cap=round,fill=fillColor] (161.85, 55.10) circle (  1.16);

\path[draw=drawColor,line width= 0.4pt,line join=round,line cap=round,fill=fillColor] (162.16, 55.07) circle (  1.16);

\path[draw=drawColor,line width= 0.4pt,line join=round,line cap=round,fill=fillColor] (162.47, 55.02) circle (  1.16);

\path[draw=drawColor,line width= 0.4pt,line join=round,line cap=round,fill=fillColor] (162.78, 55.00) circle (  1.16);

\path[draw=drawColor,line width= 0.4pt,line join=round,line cap=round,fill=fillColor] (163.08, 54.94) circle (  1.16);

\path[draw=drawColor,line width= 0.4pt,line join=round,line cap=round,fill=fillColor] (163.39, 54.93) circle (  1.16);

\path[draw=drawColor,line width= 0.4pt,line join=round,line cap=round,fill=fillColor] (163.69, 54.92) circle (  1.16);

\path[draw=drawColor,line width= 0.4pt,line join=round,line cap=round,fill=fillColor] (163.99, 54.86) circle (  1.16);

\path[draw=drawColor,line width= 0.4pt,line join=round,line cap=round,fill=fillColor] (164.29, 54.80) circle (  1.16);

\path[draw=drawColor,line width= 0.4pt,line join=round,line cap=round,fill=fillColor] (164.58, 54.77) circle (  1.16);

\path[draw=drawColor,line width= 0.4pt,line join=round,line cap=round,fill=fillColor] (164.88, 54.67) circle (  1.16);

\path[draw=drawColor,line width= 0.4pt,line join=round,line cap=round,fill=fillColor] (165.17, 54.64) circle (  1.16);

\path[draw=drawColor,line width= 0.4pt,line join=round,line cap=round,fill=fillColor] (165.46, 54.58) circle (  1.16);

\path[draw=drawColor,line width= 0.4pt,line join=round,line cap=round,fill=fillColor] (165.75, 54.51) circle (  1.16);

\path[draw=drawColor,line width= 0.4pt,line join=round,line cap=round,fill=fillColor] (166.04, 54.16) circle (  1.16);

\path[draw=drawColor,line width= 0.4pt,line join=round,line cap=round,fill=fillColor] (166.33, 54.12) circle (  1.16);

\path[draw=drawColor,line width= 0.4pt,line join=round,line cap=round,fill=fillColor] (166.61, 54.07) circle (  1.16);

\path[draw=drawColor,line width= 0.4pt,line join=round,line cap=round,fill=fillColor] (166.90, 54.06) circle (  1.16);

\path[draw=drawColor,line width= 0.4pt,line join=round,line cap=round,fill=fillColor] (167.18, 53.97) circle (  1.16);

\path[draw=drawColor,line width= 0.4pt,line join=round,line cap=round,fill=fillColor] (167.46, 53.94) circle (  1.16);

\path[draw=drawColor,line width= 0.4pt,line join=round,line cap=round,fill=fillColor] (167.74, 53.89) circle (  1.16);

\path[draw=drawColor,line width= 0.4pt,line join=round,line cap=round,fill=fillColor] (168.02, 53.74) circle (  1.16);

\path[draw=drawColor,line width= 0.4pt,line join=round,line cap=round,fill=fillColor] (168.30, 53.57) circle (  1.16);

\path[draw=drawColor,line width= 0.4pt,line join=round,line cap=round,fill=fillColor] (168.57, 53.54) circle (  1.16);

\path[draw=drawColor,line width= 0.4pt,line join=round,line cap=round,fill=fillColor] (168.85, 53.42) circle (  1.16);

\path[draw=drawColor,line width= 0.4pt,line join=round,line cap=round,fill=fillColor] (169.12, 53.40) circle (  1.16);

\path[draw=drawColor,line width= 0.4pt,line join=round,line cap=round,fill=fillColor] (169.39, 53.36) circle (  1.16);

\path[draw=drawColor,line width= 0.4pt,line join=round,line cap=round,fill=fillColor] (169.66, 53.07) circle (  1.16);

\path[draw=drawColor,line width= 0.4pt,line join=round,line cap=round,fill=fillColor] (169.93, 53.04) circle (  1.16);

\path[draw=drawColor,line width= 0.4pt,line join=round,line cap=round,fill=fillColor] (170.20, 52.80) circle (  1.16);

\path[draw=drawColor,line width= 0.4pt,line join=round,line cap=round,fill=fillColor] (170.46, 52.69) circle (  1.16);

\path[draw=drawColor,line width= 0.4pt,line join=round,line cap=round,fill=fillColor] (170.73, 52.65) circle (  1.16);

\path[draw=drawColor,line width= 0.4pt,line join=round,line cap=round,fill=fillColor] (170.99, 52.55) circle (  1.16);

\path[draw=drawColor,line width= 0.4pt,line join=round,line cap=round,fill=fillColor] (171.25, 52.40) circle (  1.16);

\path[draw=drawColor,line width= 0.4pt,line join=round,line cap=round,fill=fillColor] (171.52, 52.33) circle (  1.16);

\path[draw=drawColor,line width= 0.4pt,line join=round,line cap=round,fill=fillColor] (171.78, 52.30) circle (  1.16);

\path[draw=drawColor,line width= 0.4pt,line join=round,line cap=round,fill=fillColor] (172.04, 52.25) circle (  1.16);

\path[draw=drawColor,line width= 0.4pt,line join=round,line cap=round,fill=fillColor] (172.29, 52.24) circle (  1.16);

\path[draw=drawColor,line width= 0.4pt,line join=round,line cap=round,fill=fillColor] (172.55, 52.21) circle (  1.16);

\path[draw=drawColor,line width= 0.4pt,line join=round,line cap=round,fill=fillColor] (172.81, 52.16) circle (  1.16);

\path[draw=drawColor,line width= 0.4pt,line join=round,line cap=round,fill=fillColor] (173.06, 52.09) circle (  1.16);

\path[draw=drawColor,line width= 0.4pt,line join=round,line cap=round,fill=fillColor] (173.31, 52.06) circle (  1.16);

\path[draw=drawColor,line width= 0.4pt,line join=round,line cap=round,fill=fillColor] (173.57, 51.99) circle (  1.16);

\path[draw=drawColor,line width= 0.4pt,line join=round,line cap=round,fill=fillColor] (173.82, 51.84) circle (  1.16);

\path[draw=drawColor,line width= 0.4pt,line join=round,line cap=round,fill=fillColor] (174.07, 51.61) circle (  1.16);

\path[draw=drawColor,line width= 0.4pt,line join=round,line cap=round,fill=fillColor] (174.32, 51.58) circle (  1.16);

\path[draw=drawColor,line width= 0.4pt,line join=round,line cap=round,fill=fillColor] (174.56, 51.48) circle (  1.16);

\path[draw=drawColor,line width= 0.4pt,line join=round,line cap=round,fill=fillColor] (174.81, 51.46) circle (  1.16);

\path[draw=drawColor,line width= 0.4pt,line join=round,line cap=round,fill=fillColor] (175.06, 51.45) circle (  1.16);

\path[draw=drawColor,line width= 0.4pt,line join=round,line cap=round,fill=fillColor] (175.30, 51.44) circle (  1.16);

\path[draw=drawColor,line width= 0.4pt,line join=round,line cap=round,fill=fillColor] (175.55, 51.35) circle (  1.16);

\path[draw=drawColor,line width= 0.4pt,line join=round,line cap=round,fill=fillColor] (175.79, 51.26) circle (  1.16);

\path[draw=drawColor,line width= 0.4pt,line join=round,line cap=round,fill=fillColor] (176.03, 51.21) circle (  1.16);

\path[draw=drawColor,line width= 0.4pt,line join=round,line cap=round,fill=fillColor] (176.27, 50.71) circle (  1.16);

\path[draw=drawColor,line width= 0.4pt,line join=round,line cap=round,fill=fillColor] (176.51, 50.66) circle (  1.16);

\path[draw=drawColor,line width= 0.4pt,line join=round,line cap=round,fill=fillColor] (176.75, 50.62) circle (  1.16);

\path[draw=drawColor,line width= 0.4pt,line join=round,line cap=round,fill=fillColor] (176.99, 50.60) circle (  1.16);

\path[draw=drawColor,line width= 0.4pt,line join=round,line cap=round,fill=fillColor] (177.22, 50.59) circle (  1.16);

\path[draw=drawColor,line width= 0.4pt,line join=round,line cap=round,fill=fillColor] (177.46, 50.56) circle (  1.16);

\path[draw=drawColor,line width= 0.4pt,line join=round,line cap=round,fill=fillColor] (177.70, 50.54) circle (  1.16);

\path[draw=drawColor,line width= 0.4pt,line join=round,line cap=round,fill=fillColor] (177.93, 50.49) circle (  1.16);

\path[draw=drawColor,line width= 0.4pt,line join=round,line cap=round,fill=fillColor] (178.16, 50.45) circle (  1.16);

\path[draw=drawColor,line width= 0.4pt,line join=round,line cap=round,fill=fillColor] (178.40, 50.32) circle (  1.16);

\path[draw=drawColor,line width= 0.4pt,line join=round,line cap=round,fill=fillColor] (178.63, 50.28) circle (  1.16);

\path[draw=drawColor,line width= 0.4pt,line join=round,line cap=round,fill=fillColor] (178.86, 50.27) circle (  1.16);

\path[draw=drawColor,line width= 0.4pt,line join=round,line cap=round,fill=fillColor] (179.09, 50.24) circle (  1.16);

\path[draw=drawColor,line width= 0.4pt,line join=round,line cap=round,fill=fillColor] (179.32, 50.01) circle (  1.16);

\path[draw=drawColor,line width= 0.4pt,line join=round,line cap=round,fill=fillColor] (179.55, 50.00) circle (  1.16);

\path[draw=drawColor,line width= 0.4pt,line join=round,line cap=round,fill=fillColor] (179.77, 49.81) circle (  1.16);

\path[draw=drawColor,line width= 0.4pt,line join=round,line cap=round,fill=fillColor] (180.00, 49.68) circle (  1.16);

\path[draw=drawColor,line width= 0.4pt,line join=round,line cap=round,fill=fillColor] (180.22, 49.46) circle (  1.16);

\path[draw=drawColor,line width= 0.4pt,line join=round,line cap=round,fill=fillColor] (180.45, 49.46) circle (  1.16);

\path[draw=drawColor,line width= 0.4pt,line join=round,line cap=round,fill=fillColor] (180.67, 49.07) circle (  1.16);

\path[draw=drawColor,line width= 0.4pt,line join=round,line cap=round,fill=fillColor] (180.90, 48.90) circle (  1.16);

\path[draw=drawColor,line width= 0.4pt,line join=round,line cap=round,fill=fillColor] (181.12, 48.53) circle (  1.16);

\path[draw=drawColor,line width= 0.4pt,line join=round,line cap=round,fill=fillColor] (181.34, 44.78) circle (  1.16);

\path[draw=drawColor,line width= 0.4pt,line join=round,line cap=round,fill=fillColor] (181.56, 44.78) circle (  1.16);

\path[draw=drawColor,line width= 0.4pt,line join=round,line cap=round,fill=fillColor] (181.78, 44.78) circle (  1.16);

\path[draw=drawColor,line width= 0.4pt,line join=round,line cap=round,fill=fillColor] (182.00, 44.78) circle (  1.16);

\path[draw=drawColor,line width= 0.4pt,line join=round,line cap=round,fill=fillColor] (182.22, 44.78) circle (  1.16);

\path[draw=drawColor,line width= 0.4pt,line join=round,line cap=round,fill=fillColor] (182.44, 44.78) circle (  1.16);

\path[draw=drawColor,line width= 0.4pt,line join=round,line cap=round,fill=fillColor] (182.65, 44.78) circle (  1.16);

\path[draw=drawColor,line width= 0.4pt,line join=round,line cap=round,fill=fillColor] (182.87, 44.78) circle (  1.16);

\path[draw=drawColor,line width= 0.4pt,line join=round,line cap=round,fill=fillColor] (183.09, 44.78) circle (  1.16);

\path[draw=drawColor,line width= 0.4pt,line join=round,line cap=round,fill=fillColor] (183.30, 44.78) circle (  1.16);

\path[draw=drawColor,line width= 0.4pt,line join=round,line cap=round,fill=fillColor] (183.51, 44.78) circle (  1.16);

\path[draw=drawColor,line width= 0.4pt,line join=round,line cap=round,fill=fillColor] (183.73, 44.78) circle (  1.16);

\path[draw=drawColor,line width= 0.4pt,line join=round,line cap=round,fill=fillColor] (183.94, 44.78) circle (  1.16);

\path[draw=drawColor,line width= 0.4pt,line join=round,line cap=round,fill=fillColor] (184.15, 44.78) circle (  1.16);

\path[draw=drawColor,line width= 0.4pt,line join=round,line cap=round,fill=fillColor] (184.36, 44.78) circle (  1.16);

\path[draw=drawColor,line width= 0.4pt,line join=round,line cap=round,fill=fillColor] (184.57, 44.78) circle (  1.16);

\path[draw=drawColor,line width= 0.4pt,line join=round,line cap=round,fill=fillColor] (184.78, 44.78) circle (  1.16);

\path[draw=drawColor,line width= 0.4pt,line join=round,line cap=round,fill=fillColor] (184.99, 44.78) circle (  1.16);

\path[draw=drawColor,line width= 0.4pt,line join=round,line cap=round,fill=fillColor] (185.20, 44.78) circle (  1.16);

\path[draw=drawColor,line width= 0.4pt,line join=round,line cap=round,fill=fillColor] (185.41, 44.78) circle (  1.16);

\path[draw=drawColor,line width= 0.4pt,line join=round,line cap=round,fill=fillColor] (185.61, 44.78) circle (  1.16);

\path[draw=drawColor,line width= 0.4pt,line join=round,line cap=round,fill=fillColor] (185.82, 44.78) circle (  1.16);

\path[draw=drawColor,line width= 0.4pt,line join=round,line cap=round,fill=fillColor] (186.03, 44.78) circle (  1.16);

\path[draw=drawColor,line width= 0.4pt,line join=round,line cap=round,fill=fillColor] (186.23, 44.78) circle (  1.16);

\path[draw=drawColor,line width= 0.4pt,line join=round,line cap=round,fill=fillColor] (186.44, 44.78) circle (  1.16);

\path[draw=drawColor,line width= 0.4pt,line join=round,line cap=round,fill=fillColor] (186.64, 44.78) circle (  1.16);

\path[draw=drawColor,line width= 0.4pt,line join=round,line cap=round,fill=fillColor] (186.84, 44.78) circle (  1.16);

\path[draw=drawColor,line width= 0.4pt,line join=round,line cap=round,fill=fillColor] (187.05, 44.78) circle (  1.16);

\path[draw=drawColor,line width= 0.4pt,line join=round,line cap=round,fill=fillColor] (187.25, 44.78) circle (  1.16);

\path[draw=drawColor,line width= 0.4pt,line join=round,line cap=round,fill=fillColor] (187.45, 44.78) circle (  1.16);

\path[draw=drawColor,line width= 0.4pt,line join=round,line cap=round,fill=fillColor] (187.65, 44.78) circle (  1.16);

\path[draw=drawColor,line width= 0.4pt,line join=round,line cap=round,fill=fillColor] (187.85, 44.78) circle (  1.16);

\path[draw=drawColor,line width= 0.4pt,line join=round,line cap=round,fill=fillColor] (188.05, 44.78) circle (  1.16);

\path[draw=drawColor,line width= 0.4pt,line join=round,line cap=round,fill=fillColor] (188.25, 44.78) circle (  1.16);

\path[draw=drawColor,line width= 0.4pt,line join=round,line cap=round,fill=fillColor] (188.45, 44.78) circle (  1.16);

\path[draw=drawColor,line width= 0.4pt,line join=round,line cap=round,fill=fillColor] (188.64, 44.78) circle (  1.16);

\path[draw=drawColor,line width= 0.4pt,line join=round,line cap=round,fill=fillColor] (188.84, 44.78) circle (  1.16);

\path[draw=drawColor,line width= 0.4pt,line join=round,line cap=round,fill=fillColor] (189.04, 44.78) circle (  1.16);

\path[draw=drawColor,line width= 0.4pt,line join=round,line cap=round,fill=fillColor] (189.23, 44.78) circle (  1.16);

\path[draw=drawColor,line width= 0.4pt,line join=round,line cap=round,fill=fillColor] (189.43, 44.78) circle (  1.16);

\path[draw=drawColor,line width= 0.4pt,line join=round,line cap=round,fill=fillColor] (189.62, 44.78) circle (  1.16);

\path[draw=drawColor,line width= 0.4pt,line join=round,line cap=round,fill=fillColor] (189.82, 44.78) circle (  1.16);

\path[draw=drawColor,line width= 0.4pt,line join=round,line cap=round,fill=fillColor] (190.01, 44.78) circle (  1.16);

\path[draw=drawColor,line width= 0.4pt,line join=round,line cap=round,fill=fillColor] (190.20, 44.78) circle (  1.16);

\path[draw=drawColor,line width= 0.4pt,line join=round,line cap=round,fill=fillColor] (190.39, 44.78) circle (  1.16);

\path[draw=drawColor,line width= 0.4pt,line join=round,line cap=round,fill=fillColor] (190.59, 44.78) circle (  1.16);

\path[draw=drawColor,line width= 0.4pt,line join=round,line cap=round,fill=fillColor] (190.78, 44.78) circle (  1.16);

\path[draw=drawColor,line width= 0.4pt,line join=round,line cap=round,fill=fillColor] (190.97, 44.78) circle (  1.16);

\path[draw=drawColor,line width= 0.4pt,line join=round,line cap=round,fill=fillColor] (191.16, 44.78) circle (  1.16);

\path[draw=drawColor,line width= 0.4pt,line join=round,line cap=round,fill=fillColor] (191.35, 44.78) circle (  1.16);

\path[draw=drawColor,line width= 0.4pt,line join=round,line cap=round,fill=fillColor] (191.54, 44.78) circle (  1.16);

\path[draw=drawColor,line width= 0.4pt,line join=round,line cap=round,fill=fillColor] (191.73, 44.78) circle (  1.16);

\path[draw=drawColor,line width= 0.4pt,line join=round,line cap=round,fill=fillColor] (191.91, 44.78) circle (  1.16);

\path[draw=drawColor,line width= 0.4pt,line join=round,line cap=round,fill=fillColor] (192.10, 44.78) circle (  1.16);

\path[draw=drawColor,line width= 0.4pt,line join=round,line cap=round,fill=fillColor] (192.29, 44.78) circle (  1.16);

\path[draw=drawColor,line width= 0.4pt,line join=round,line cap=round,fill=fillColor] (192.47, 44.78) circle (  1.16);

\path[draw=drawColor,line width= 0.4pt,line join=round,line cap=round,fill=fillColor] (192.66, 44.78) circle (  1.16);

\path[draw=drawColor,line width= 0.4pt,line join=round,line cap=round,fill=fillColor] (192.85, 44.78) circle (  1.16);

\path[draw=drawColor,line width= 0.4pt,line join=round,line cap=round,fill=fillColor] (193.03, 44.78) circle (  1.16);

\path[draw=drawColor,line width= 0.4pt,line join=round,line cap=round,fill=fillColor] (193.22, 44.78) circle (  1.16);

\path[draw=drawColor,line width= 0.4pt,line join=round,line cap=round,fill=fillColor] (193.40, 44.78) circle (  1.16);

\path[draw=drawColor,line width= 0.4pt,line join=round,line cap=round,fill=fillColor] (193.58, 44.78) circle (  1.16);

\path[draw=drawColor,line width= 0.4pt,line join=round,line cap=round,fill=fillColor] (193.77, 44.78) circle (  1.16);

\path[draw=drawColor,line width= 0.4pt,line join=round,line cap=round,fill=fillColor] (193.95, 44.78) circle (  1.16);

\path[draw=drawColor,line width= 0.4pt,line join=round,line cap=round,fill=fillColor] (194.13, 44.78) circle (  1.16);

\path[draw=drawColor,line width= 0.4pt,line join=round,line cap=round,fill=fillColor] (194.31, 44.78) circle (  1.16);

\path[draw=drawColor,line width= 0.4pt,line join=round,line cap=round,fill=fillColor] (194.49, 44.78) circle (  1.16);

\path[draw=drawColor,line width= 0.4pt,line join=round,line cap=round,fill=fillColor] (194.67, 44.78) circle (  1.16);

\path[draw=drawColor,line width= 0.4pt,line join=round,line cap=round,fill=fillColor] (194.85, 44.78) circle (  1.16);

\path[draw=drawColor,line width= 0.4pt,line join=round,line cap=round,fill=fillColor] (195.03, 44.78) circle (  1.16);

\path[draw=drawColor,line width= 0.4pt,line join=round,line cap=round,fill=fillColor] (195.21, 44.78) circle (  1.16);

\path[draw=drawColor,line width= 0.4pt,line join=round,line cap=round,fill=fillColor] (195.39, 44.78) circle (  1.16);

\path[draw=drawColor,line width= 0.4pt,line join=round,line cap=round,fill=fillColor] (195.57, 44.78) circle (  1.16);

\path[draw=drawColor,line width= 0.4pt,line join=round,line cap=round,fill=fillColor] (195.75, 44.78) circle (  1.16);

\path[draw=drawColor,line width= 0.4pt,line join=round,line cap=round,fill=fillColor] (195.92, 44.78) circle (  1.16);

\path[draw=drawColor,line width= 0.4pt,line join=round,line cap=round,fill=fillColor] (196.10, 44.78) circle (  1.16);

\path[draw=drawColor,line width= 0.4pt,line join=round,line cap=round,fill=fillColor] (196.28, 44.78) circle (  1.16);

\path[draw=drawColor,line width= 0.4pt,line join=round,line cap=round,fill=fillColor] (196.45, 44.78) circle (  1.16);

\path[draw=drawColor,line width= 0.4pt,line join=round,line cap=round,fill=fillColor] (196.63, 44.78) circle (  1.16);

\path[draw=drawColor,line width= 0.4pt,line join=round,line cap=round,fill=fillColor] (196.80, 44.78) circle (  1.16);

\path[draw=drawColor,line width= 0.4pt,line join=round,line cap=round,fill=fillColor] (196.98, 44.78) circle (  1.16);

\path[draw=drawColor,line width= 0.4pt,line join=round,line cap=round,fill=fillColor] (197.15, 44.78) circle (  1.16);

\path[draw=drawColor,line width= 0.4pt,line join=round,line cap=round,fill=fillColor] (197.33, 44.78) circle (  1.16);

\path[draw=drawColor,line width= 0.4pt,line join=round,line cap=round,fill=fillColor] (197.50, 44.78) circle (  1.16);

\path[draw=drawColor,line width= 0.4pt,line join=round,line cap=round,fill=fillColor] (197.67, 44.78) circle (  1.16);

\path[draw=drawColor,line width= 0.4pt,line join=round,line cap=round,fill=fillColor] (197.85, 44.78) circle (  1.16);

\path[draw=drawColor,line width= 0.4pt,line join=round,line cap=round,fill=fillColor] (198.02, 44.78) circle (  1.16);

\path[draw=drawColor,line width= 0.4pt,line join=round,line cap=round,fill=fillColor] (198.19, 44.78) circle (  1.16);

\path[draw=drawColor,line width= 0.4pt,line join=round,line cap=round,fill=fillColor] (198.36, 44.78) circle (  1.16);

\path[draw=drawColor,line width= 0.4pt,line join=round,line cap=round,fill=fillColor] (198.53, 44.78) circle (  1.16);

\path[draw=drawColor,line width= 0.4pt,line join=round,line cap=round,fill=fillColor] (198.70, 44.78) circle (  1.16);

\path[draw=drawColor,line width= 0.4pt,line join=round,line cap=round,fill=fillColor] (198.87, 44.78) circle (  1.16);

\path[draw=drawColor,line width= 0.4pt,line join=round,line cap=round,fill=fillColor] (199.04, 44.78) circle (  1.16);

\path[draw=drawColor,line width= 0.4pt,line join=round,line cap=round,fill=fillColor] (199.21, 44.78) circle (  1.16);

\path[draw=drawColor,line width= 0.4pt,line join=round,line cap=round,fill=fillColor] (199.38, 44.78) circle (  1.16);

\path[draw=drawColor,line width= 0.4pt,line join=round,line cap=round,fill=fillColor] (199.55, 44.78) circle (  1.16);

\path[draw=drawColor,line width= 0.4pt,line join=round,line cap=round,fill=fillColor] (199.72, 44.78) circle (  1.16);

\path[draw=drawColor,line width= 0.4pt,line join=round,line cap=round,fill=fillColor] (199.88, 44.78) circle (  1.16);

\path[draw=drawColor,line width= 0.4pt,line join=round,line cap=round,fill=fillColor] (200.05, 44.78) circle (  1.16);

\path[draw=drawColor,line width= 0.4pt,line join=round,line cap=round,fill=fillColor] (200.22, 44.78) circle (  1.16);

\path[draw=drawColor,line width= 0.4pt,line join=round,line cap=round,fill=fillColor] (200.38, 44.78) circle (  1.16);

\path[draw=drawColor,line width= 0.4pt,line join=round,line cap=round,fill=fillColor] (200.55, 44.78) circle (  1.16);

\path[draw=drawColor,line width= 0.4pt,line join=round,line cap=round,fill=fillColor] (200.72, 44.78) circle (  1.16);

\path[draw=drawColor,line width= 0.4pt,line join=round,line cap=round,fill=fillColor] (200.88, 44.78) circle (  1.16);

\path[draw=drawColor,line width= 0.4pt,line join=round,line cap=round,fill=fillColor] (201.05, 44.78) circle (  1.16);

\path[draw=drawColor,line width= 0.4pt,line join=round,line cap=round,fill=fillColor] (201.21, 44.78) circle (  1.16);

\path[draw=drawColor,line width= 0.4pt,line join=round,line cap=round,fill=fillColor] (201.37, 44.78) circle (  1.16);

\path[draw=drawColor,line width= 0.4pt,line join=round,line cap=round,fill=fillColor] (201.54, 44.78) circle (  1.16);

\path[draw=drawColor,line width= 0.4pt,line join=round,line cap=round,fill=fillColor] (201.70, 44.78) circle (  1.16);

\path[draw=drawColor,line width= 0.4pt,line join=round,line cap=round,fill=fillColor] (201.86, 44.78) circle (  1.16);

\path[draw=drawColor,line width= 0.4pt,line join=round,line cap=round,fill=fillColor] (202.03, 44.78) circle (  1.16);

\path[draw=drawColor,line width= 0.4pt,line join=round,line cap=round,fill=fillColor] (202.19, 44.78) circle (  1.16);

\path[draw=drawColor,line width= 0.4pt,line join=round,line cap=round,fill=fillColor] (202.35, 44.78) circle (  1.16);

\path[draw=drawColor,line width= 0.4pt,line join=round,line cap=round,fill=fillColor] (202.51, 44.78) circle (  1.16);

\path[draw=drawColor,line width= 0.4pt,line join=round,line cap=round,fill=fillColor] (202.67, 44.78) circle (  1.16);

\path[draw=drawColor,line width= 0.4pt,line join=round,line cap=round,fill=fillColor] (202.83, 44.78) circle (  1.16);

\path[draw=drawColor,line width= 0.4pt,line join=round,line cap=round,fill=fillColor] (202.99, 44.78) circle (  1.16);

\path[draw=drawColor,line width= 0.4pt,line join=round,line cap=round,fill=fillColor] (203.16, 44.78) circle (  1.16);

\path[draw=drawColor,line width= 0.4pt,line join=round,line cap=round,fill=fillColor] (203.31, 44.78) circle (  1.16);

\path[draw=drawColor,line width= 0.4pt,line join=round,line cap=round,fill=fillColor] (203.47, 44.78) circle (  1.16);

\path[draw=drawColor,line width= 0.4pt,line join=round,line cap=round,fill=fillColor] (203.63, 44.78) circle (  1.16);

\path[draw=drawColor,line width= 0.4pt,line join=round,line cap=round,fill=fillColor] (203.79, 44.78) circle (  1.16);

\path[draw=drawColor,line width= 0.4pt,line join=round,line cap=round,fill=fillColor] (203.95, 44.78) circle (  1.16);

\path[draw=drawColor,line width= 0.4pt,line join=round,line cap=round,fill=fillColor] (204.11, 44.78) circle (  1.16);

\path[draw=drawColor,line width= 0.4pt,line join=round,line cap=round,fill=fillColor] (204.27, 44.78) circle (  1.16);

\path[draw=drawColor,line width= 0.4pt,line join=round,line cap=round,fill=fillColor] (204.42, 44.78) circle (  1.16);

\path[draw=drawColor,line width= 0.4pt,line join=round,line cap=round,fill=fillColor] (204.58, 44.78) circle (  1.16);

\path[draw=drawColor,line width= 0.4pt,line join=round,line cap=round,fill=fillColor] (204.74, 44.78) circle (  1.16);

\path[draw=drawColor,line width= 0.4pt,line join=round,line cap=round,fill=fillColor] (204.89, 44.78) circle (  1.16);

\path[draw=drawColor,line width= 0.4pt,line join=round,line cap=round,fill=fillColor] (205.05, 44.78) circle (  1.16);

\path[draw=drawColor,line width= 0.4pt,line join=round,line cap=round,fill=fillColor] (205.21, 44.78) circle (  1.16);

\path[draw=drawColor,line width= 0.4pt,line join=round,line cap=round,fill=fillColor] (205.36, 44.78) circle (  1.16);

\path[draw=drawColor,line width= 0.4pt,line join=round,line cap=round,fill=fillColor] (205.52, 44.78) circle (  1.16);

\path[draw=drawColor,line width= 0.4pt,line join=round,line cap=round,fill=fillColor] (205.67, 44.78) circle (  1.16);

\path[draw=drawColor,line width= 0.4pt,line join=round,line cap=round,fill=fillColor] (205.83, 44.78) circle (  1.16);

\path[draw=drawColor,line width= 0.4pt,line join=round,line cap=round,fill=fillColor] (205.98, 44.78) circle (  1.16);

\path[draw=drawColor,line width= 0.4pt,line join=round,line cap=round,fill=fillColor] (206.13, 44.78) circle (  1.16);

\path[draw=drawColor,line width= 0.4pt,line join=round,line cap=round,fill=fillColor] (206.29, 44.78) circle (  1.16);

\path[draw=drawColor,line width= 0.4pt,line join=round,line cap=round,fill=fillColor] (206.44, 44.78) circle (  1.16);

\path[draw=drawColor,line width= 0.4pt,line join=round,line cap=round,fill=fillColor] (206.59, 44.78) circle (  1.16);

\path[draw=drawColor,line width= 0.4pt,line join=round,line cap=round,fill=fillColor] (206.74, 44.78) circle (  1.16);

\path[draw=drawColor,line width= 0.4pt,line join=round,line cap=round,fill=fillColor] (206.90, 44.78) circle (  1.16);

\path[draw=drawColor,line width= 0.4pt,line join=round,line cap=round,fill=fillColor] (207.05, 44.78) circle (  1.16);

\path[draw=drawColor,line width= 0.4pt,line join=round,line cap=round,fill=fillColor] (207.20, 44.78) circle (  1.16);

\path[draw=drawColor,line width= 0.4pt,line join=round,line cap=round,fill=fillColor] (207.35, 44.78) circle (  1.16);

\path[draw=drawColor,line width= 0.4pt,line join=round,line cap=round,fill=fillColor] (207.50, 44.78) circle (  1.16);

\path[draw=drawColor,line width= 0.4pt,line join=round,line cap=round,fill=fillColor] (207.65, 44.78) circle (  1.16);

\path[draw=drawColor,line width= 0.4pt,line join=round,line cap=round,fill=fillColor] (207.80, 44.78) circle (  1.16);

\path[draw=drawColor,line width= 0.4pt,line join=round,line cap=round,fill=fillColor] (207.95, 44.78) circle (  1.16);

\path[draw=drawColor,line width= 0.4pt,line join=round,line cap=round,fill=fillColor] (208.10, 44.78) circle (  1.16);

\path[draw=drawColor,line width= 0.4pt,line join=round,line cap=round,fill=fillColor] (208.25, 44.78) circle (  1.16);

\path[draw=drawColor,line width= 0.4pt,line join=round,line cap=round,fill=fillColor] (208.40, 44.78) circle (  1.16);

\path[draw=drawColor,line width= 0.4pt,line join=round,line cap=round,fill=fillColor] (208.55, 44.78) circle (  1.16);

\path[draw=drawColor,line width= 0.4pt,line join=round,line cap=round,fill=fillColor] (208.70, 44.78) circle (  1.16);

\path[draw=drawColor,line width= 0.4pt,line join=round,line cap=round,fill=fillColor] (208.85, 44.78) circle (  1.16);

\path[draw=drawColor,line width= 0.4pt,line join=round,line cap=round,fill=fillColor] (209.00, 44.78) circle (  1.16);

\path[draw=drawColor,line width= 0.4pt,line join=round,line cap=round,fill=fillColor] (209.14, 44.78) circle (  1.16);

\path[draw=drawColor,line width= 0.4pt,line join=round,line cap=round,fill=fillColor] (209.29, 44.78) circle (  1.16);

\path[draw=drawColor,line width= 0.4pt,line join=round,line cap=round,fill=fillColor] (209.44, 44.78) circle (  1.16);

\path[draw=drawColor,line width= 0.4pt,line join=round,line cap=round,fill=fillColor] (209.59, 44.78) circle (  1.16);

\path[draw=drawColor,line width= 0.4pt,line join=round,line cap=round,fill=fillColor] (209.73, 44.78) circle (  1.16);

\path[draw=drawColor,line width= 0.4pt,line join=round,line cap=round,fill=fillColor] (209.88, 44.78) circle (  1.16);

\path[draw=drawColor,line width= 0.4pt,line join=round,line cap=round,fill=fillColor] (210.02, 44.78) circle (  1.16);

\path[draw=drawColor,line width= 0.4pt,line join=round,line cap=round,fill=fillColor] (210.17, 44.78) circle (  1.16);

\path[draw=drawColor,line width= 0.4pt,line join=round,line cap=round,fill=fillColor] (210.32, 44.78) circle (  1.16);

\path[draw=drawColor,line width= 0.4pt,line join=round,line cap=round,fill=fillColor] (210.46, 44.78) circle (  1.16);

\path[draw=drawColor,line width= 0.4pt,line join=round,line cap=round,fill=fillColor] (210.61, 44.78) circle (  1.16);

\path[draw=drawColor,line width= 0.4pt,line join=round,line cap=round,fill=fillColor] (210.75, 44.78) circle (  1.16);

\path[draw=drawColor,line width= 0.4pt,line join=round,line cap=round,fill=fillColor] (210.89, 44.78) circle (  1.16);

\path[draw=drawColor,line width= 0.4pt,line join=round,line cap=round,fill=fillColor] (211.04, 44.78) circle (  1.16);

\path[draw=drawColor,line width= 0.4pt,line join=round,line cap=round,fill=fillColor] (211.18, 44.78) circle (  1.16);

\path[draw=drawColor,line width= 0.4pt,line join=round,line cap=round,fill=fillColor] (211.33, 44.78) circle (  1.16);

\path[draw=drawColor,line width= 0.4pt,line join=round,line cap=round,fill=fillColor] (211.47, 44.78) circle (  1.16);

\path[draw=drawColor,line width= 0.4pt,line join=round,line cap=round,fill=fillColor] (211.61, 44.78) circle (  1.16);

\path[draw=drawColor,line width= 0.4pt,line join=round,line cap=round,fill=fillColor] (211.76, 44.78) circle (  1.16);

\path[draw=drawColor,line width= 0.4pt,line join=round,line cap=round,fill=fillColor] (211.90, 44.78) circle (  1.16);

\path[draw=drawColor,line width= 0.4pt,line join=round,line cap=round,fill=fillColor] (212.04, 44.78) circle (  1.16);

\path[draw=drawColor,line width= 0.4pt,line join=round,line cap=round,fill=fillColor] (212.18, 44.78) circle (  1.16);

\path[draw=drawColor,line width= 0.4pt,line join=round,line cap=round,fill=fillColor] (212.32, 44.78) circle (  1.16);

\path[draw=drawColor,line width= 0.4pt,line join=round,line cap=round,fill=fillColor] (212.47, 44.78) circle (  1.16);

\path[draw=drawColor,line width= 0.4pt,line join=round,line cap=round,fill=fillColor] (212.61, 44.78) circle (  1.16);

\path[draw=drawColor,line width= 0.4pt,line join=round,line cap=round,fill=fillColor] (212.75, 44.78) circle (  1.16);

\path[draw=drawColor,line width= 0.4pt,line join=round,line cap=round,fill=fillColor] (212.89, 44.78) circle (  1.16);

\path[draw=drawColor,line width= 0.4pt,line join=round,line cap=round,fill=fillColor] (213.03, 44.78) circle (  1.16);

\path[draw=drawColor,line width= 0.4pt,line join=round,line cap=round,fill=fillColor] (213.17, 44.78) circle (  1.16);

\path[draw=drawColor,line width= 0.4pt,line join=round,line cap=round,fill=fillColor] (213.31, 44.78) circle (  1.16);

\path[draw=drawColor,line width= 0.4pt,line join=round,line cap=round,fill=fillColor] (213.45, 44.78) circle (  1.16);

\path[draw=drawColor,line width= 0.4pt,line join=round,line cap=round,fill=fillColor] (213.59, 44.78) circle (  1.16);

\path[draw=drawColor,line width= 0.4pt,line join=round,line cap=round,fill=fillColor] (213.73, 44.78) circle (  1.16);

\path[draw=drawColor,line width= 0.4pt,line join=round,line cap=round,fill=fillColor] (213.87, 44.78) circle (  1.16);

\path[draw=drawColor,line width= 0.4pt,line join=round,line cap=round,fill=fillColor] (214.00, 44.78) circle (  1.16);

\path[draw=drawColor,line width= 0.4pt,line join=round,line cap=round,fill=fillColor] (214.14, 44.78) circle (  1.16);

\path[draw=drawColor,line width= 0.4pt,line join=round,line cap=round,fill=fillColor] (214.28, 44.78) circle (  1.16);

\path[draw=drawColor,line width= 0.4pt,line join=round,line cap=round,fill=fillColor] (214.42, 44.78) circle (  1.16);

\path[draw=drawColor,line width= 0.4pt,line join=round,line cap=round,fill=fillColor] (214.56, 44.78) circle (  1.16);

\path[draw=drawColor,line width= 0.4pt,line join=round,line cap=round,fill=fillColor] (214.69, 44.78) circle (  1.16);

\path[draw=drawColor,line width= 0.4pt,line join=round,line cap=round,fill=fillColor] (214.83, 44.78) circle (  1.16);

\path[draw=drawColor,line width= 0.4pt,line join=round,line cap=round,fill=fillColor] (214.97, 44.78) circle (  1.16);

\path[draw=drawColor,line width= 0.4pt,line join=round,line cap=round,fill=fillColor] (215.11, 44.78) circle (  1.16);

\path[draw=drawColor,line width= 0.4pt,line join=round,line cap=round,fill=fillColor] (215.24, 44.78) circle (  1.16);

\path[draw=drawColor,line width= 0.4pt,line join=round,line cap=round,fill=fillColor] (215.38, 44.78) circle (  1.16);

\path[draw=drawColor,line width= 0.4pt,line join=round,line cap=round,fill=fillColor] (215.51, 44.78) circle (  1.16);

\path[draw=drawColor,line width= 0.4pt,line join=round,line cap=round,fill=fillColor] (215.65, 44.78) circle (  1.16);

\path[draw=drawColor,line width= 0.4pt,line join=round,line cap=round,fill=fillColor] (215.79, 44.78) circle (  1.16);

\path[draw=drawColor,line width= 0.4pt,line join=round,line cap=round,fill=fillColor] (215.92, 44.78) circle (  1.16);

\path[draw=drawColor,line width= 0.4pt,line join=round,line cap=round,fill=fillColor] (216.06, 44.78) circle (  1.16);

\path[draw=drawColor,line width= 0.4pt,line join=round,line cap=round,fill=fillColor] (216.19, 44.78) circle (  1.16);

\path[draw=drawColor,line width= 0.4pt,line join=round,line cap=round,fill=fillColor] (216.33, 44.78) circle (  1.16);

\path[draw=drawColor,line width= 0.4pt,line join=round,line cap=round,fill=fillColor] (216.46, 44.78) circle (  1.16);

\path[draw=drawColor,line width= 0.4pt,line join=round,line cap=round,fill=fillColor] (216.59, 44.78) circle (  1.16);

\path[draw=drawColor,line width= 0.4pt,line join=round,line cap=round,fill=fillColor] (216.73, 44.78) circle (  1.16);

\path[draw=drawColor,line width= 0.4pt,line join=round,line cap=round,fill=fillColor] (216.86, 44.78) circle (  1.16);

\path[draw=drawColor,line width= 0.4pt,line join=round,line cap=round,fill=fillColor] (217.00, 44.78) circle (  1.16);

\path[draw=drawColor,line width= 0.4pt,line join=round,line cap=round,fill=fillColor] (217.13, 44.78) circle (  1.16);

\path[draw=drawColor,line width= 0.4pt,line join=round,line cap=round,fill=fillColor] (217.26, 44.78) circle (  1.16);

\path[draw=drawColor,line width= 0.4pt,line join=round,line cap=round,fill=fillColor] (217.40, 44.78) circle (  1.16);

\path[draw=drawColor,line width= 0.4pt,line join=round,line cap=round,fill=fillColor] (217.53, 44.78) circle (  1.16);

\path[draw=drawColor,line width= 0.4pt,line join=round,line cap=round,fill=fillColor] (217.66, 44.78) circle (  1.16);

\path[draw=drawColor,line width= 0.4pt,line join=round,line cap=round,fill=fillColor] (217.79, 44.78) circle (  1.16);

\path[draw=drawColor,line width= 0.4pt,line join=round,line cap=round,fill=fillColor] (217.92, 44.78) circle (  1.16);

\path[draw=drawColor,line width= 0.4pt,line join=round,line cap=round,fill=fillColor] (218.06, 44.78) circle (  1.16);

\path[draw=drawColor,line width= 0.4pt,line join=round,line cap=round,fill=fillColor] (218.19, 44.78) circle (  1.16);

\path[draw=drawColor,line width= 0.4pt,line join=round,line cap=round,fill=fillColor] (218.32, 44.78) circle (  1.16);

\path[draw=drawColor,line width= 0.4pt,line join=round,line cap=round,fill=fillColor] (218.45, 44.78) circle (  1.16);

\path[draw=drawColor,line width= 0.4pt,line join=round,line cap=round,fill=fillColor] (218.58, 44.78) circle (  1.16);

\path[draw=drawColor,line width= 0.4pt,line join=round,line cap=round,fill=fillColor] (218.71, 44.78) circle (  1.16);

\path[draw=drawColor,line width= 0.4pt,line join=round,line cap=round,fill=fillColor] (218.84, 44.78) circle (  1.16);

\path[draw=drawColor,line width= 0.4pt,line join=round,line cap=round,fill=fillColor] (218.97, 44.78) circle (  1.16);

\path[draw=drawColor,line width= 0.4pt,line join=round,line cap=round,fill=fillColor] (219.10, 44.78) circle (  1.16);
\definecolor[named]{drawColor}{rgb}{0.22,0.49,0.72}
\definecolor[named]{fillColor}{rgb}{0.22,0.49,0.72}

\path[draw=drawColor,line width= 0.4pt,line join=round,line cap=round,fill=fillColor] ( 78.97, 84.15) circle (  1.16);

\path[draw=drawColor,line width= 0.4pt,line join=round,line cap=round,fill=fillColor] ( 84.61, 72.78) circle (  1.16);

\path[draw=drawColor,line width= 0.4pt,line join=round,line cap=round,fill=fillColor] ( 88.57, 72.17) circle (  1.16);

\path[draw=drawColor,line width= 0.4pt,line join=round,line cap=round,fill=fillColor] ( 91.71, 69.71) circle (  1.16);

\path[draw=drawColor,line width= 0.4pt,line join=round,line cap=round,fill=fillColor] ( 94.37, 68.77) circle (  1.16);

\path[draw=drawColor,line width= 0.4pt,line join=round,line cap=round,fill=fillColor] ( 96.70, 68.48) circle (  1.16);

\path[draw=drawColor,line width= 0.4pt,line join=round,line cap=round,fill=fillColor] ( 98.78, 66.43) circle (  1.16);

\path[draw=drawColor,line width= 0.4pt,line join=round,line cap=round,fill=fillColor] (100.67, 66.17) circle (  1.16);

\path[draw=drawColor,line width= 0.4pt,line join=round,line cap=round,fill=fillColor] (102.40, 65.84) circle (  1.16);

\path[draw=drawColor,line width= 0.4pt,line join=round,line cap=round,fill=fillColor] (104.02, 65.50) circle (  1.16);

\path[draw=drawColor,line width= 0.4pt,line join=round,line cap=round,fill=fillColor] (105.53, 65.38) circle (  1.16);

\path[draw=drawColor,line width= 0.4pt,line join=round,line cap=round,fill=fillColor] (106.95, 65.22) circle (  1.16);

\path[draw=drawColor,line width= 0.4pt,line join=round,line cap=round,fill=fillColor] (108.29, 65.08) circle (  1.16);

\path[draw=drawColor,line width= 0.4pt,line join=round,line cap=round,fill=fillColor] (109.56, 65.06) circle (  1.16);

\path[draw=drawColor,line width= 0.4pt,line join=round,line cap=round,fill=fillColor] (110.78, 64.58) circle (  1.16);

\path[draw=drawColor,line width= 0.4pt,line join=round,line cap=round,fill=fillColor] (111.94, 64.55) circle (  1.16);

\path[draw=drawColor,line width= 0.4pt,line join=round,line cap=round,fill=fillColor] (113.06, 63.96) circle (  1.16);

\path[draw=drawColor,line width= 0.4pt,line join=round,line cap=round,fill=fillColor] (114.13, 63.60) circle (  1.16);

\path[draw=drawColor,line width= 0.4pt,line join=round,line cap=round,fill=fillColor] (115.17, 63.51) circle (  1.16);

\path[draw=drawColor,line width= 0.4pt,line join=round,line cap=round,fill=fillColor] (116.17, 63.49) circle (  1.16);

\path[draw=drawColor,line width= 0.4pt,line join=round,line cap=round,fill=fillColor] (117.13, 63.42) circle (  1.16);

\path[draw=drawColor,line width= 0.4pt,line join=round,line cap=round,fill=fillColor] (118.07, 63.32) circle (  1.16);

\path[draw=drawColor,line width= 0.4pt,line join=round,line cap=round,fill=fillColor] (118.97, 63.03) circle (  1.16);

\path[draw=drawColor,line width= 0.4pt,line join=round,line cap=round,fill=fillColor] (119.86, 62.94) circle (  1.16);

\path[draw=drawColor,line width= 0.4pt,line join=round,line cap=round,fill=fillColor] (120.71, 62.84) circle (  1.16);

\path[draw=drawColor,line width= 0.4pt,line join=round,line cap=round,fill=fillColor] (121.55, 62.51) circle (  1.16);

\path[draw=drawColor,line width= 0.4pt,line join=round,line cap=round,fill=fillColor] (122.36, 62.37) circle (  1.16);

\path[draw=drawColor,line width= 0.4pt,line join=round,line cap=round,fill=fillColor] (123.16, 62.32) circle (  1.16);

\path[draw=drawColor,line width= 0.4pt,line join=round,line cap=round,fill=fillColor] (123.93, 62.31) circle (  1.16);

\path[draw=drawColor,line width= 0.4pt,line join=round,line cap=round,fill=fillColor] (124.69, 62.21) circle (  1.16);

\path[draw=drawColor,line width= 0.4pt,line join=round,line cap=round,fill=fillColor] (125.43, 62.11) circle (  1.16);

\path[draw=drawColor,line width= 0.4pt,line join=round,line cap=round,fill=fillColor] (126.15, 62.08) circle (  1.16);

\path[draw=drawColor,line width= 0.4pt,line join=round,line cap=round,fill=fillColor] (126.86, 62.03) circle (  1.16);

\path[draw=drawColor,line width= 0.4pt,line join=round,line cap=round,fill=fillColor] (127.56, 61.82) circle (  1.16);

\path[draw=drawColor,line width= 0.4pt,line join=round,line cap=round,fill=fillColor] (128.24, 61.81) circle (  1.16);

\path[draw=drawColor,line width= 0.4pt,line join=round,line cap=round,fill=fillColor] (128.91, 61.64) circle (  1.16);

\path[draw=drawColor,line width= 0.4pt,line join=round,line cap=round,fill=fillColor] (129.57, 61.63) circle (  1.16);

\path[draw=drawColor,line width= 0.4pt,line join=round,line cap=round,fill=fillColor] (130.21, 61.19) circle (  1.16);

\path[draw=drawColor,line width= 0.4pt,line join=round,line cap=round,fill=fillColor] (130.85, 61.11) circle (  1.16);

\path[draw=drawColor,line width= 0.4pt,line join=round,line cap=round,fill=fillColor] (131.47, 61.03) circle (  1.16);

\path[draw=drawColor,line width= 0.4pt,line join=round,line cap=round,fill=fillColor] (132.09, 60.94) circle (  1.16);

\path[draw=drawColor,line width= 0.4pt,line join=round,line cap=round,fill=fillColor] (132.69, 60.73) circle (  1.16);

\path[draw=drawColor,line width= 0.4pt,line join=round,line cap=round,fill=fillColor] (133.28, 60.70) circle (  1.16);

\path[draw=drawColor,line width= 0.4pt,line join=round,line cap=round,fill=fillColor] (133.87, 60.57) circle (  1.16);

\path[draw=drawColor,line width= 0.4pt,line join=round,line cap=round,fill=fillColor] (134.44, 60.46) circle (  1.16);

\path[draw=drawColor,line width= 0.4pt,line join=round,line cap=round,fill=fillColor] (135.01, 60.23) circle (  1.16);

\path[draw=drawColor,line width= 0.4pt,line join=round,line cap=round,fill=fillColor] (135.57, 60.23) circle (  1.16);

\path[draw=drawColor,line width= 0.4pt,line join=round,line cap=round,fill=fillColor] (136.12, 59.94) circle (  1.16);

\path[draw=drawColor,line width= 0.4pt,line join=round,line cap=round,fill=fillColor] (136.67, 59.90) circle (  1.16);

\path[draw=drawColor,line width= 0.4pt,line join=round,line cap=round,fill=fillColor] (137.20, 59.88) circle (  1.16);

\path[draw=drawColor,line width= 0.4pt,line join=round,line cap=round,fill=fillColor] (137.73, 59.74) circle (  1.16);

\path[draw=drawColor,line width= 0.4pt,line join=round,line cap=round,fill=fillColor] (138.25, 59.42) circle (  1.16);

\path[draw=drawColor,line width= 0.4pt,line join=round,line cap=round,fill=fillColor] (138.77, 59.30) circle (  1.16);

\path[draw=drawColor,line width= 0.4pt,line join=round,line cap=round,fill=fillColor] (139.28, 59.19) circle (  1.16);

\path[draw=drawColor,line width= 0.4pt,line join=round,line cap=round,fill=fillColor] (139.78, 58.97) circle (  1.16);

\path[draw=drawColor,line width= 0.4pt,line join=round,line cap=round,fill=fillColor] (140.28, 58.96) circle (  1.16);

\path[draw=drawColor,line width= 0.4pt,line join=round,line cap=round,fill=fillColor] (140.77, 58.95) circle (  1.16);

\path[draw=drawColor,line width= 0.4pt,line join=round,line cap=round,fill=fillColor] (141.26, 58.86) circle (  1.16);

\path[draw=drawColor,line width= 0.4pt,line join=round,line cap=round,fill=fillColor] (141.74, 58.73) circle (  1.16);

\path[draw=drawColor,line width= 0.4pt,line join=round,line cap=round,fill=fillColor] (142.21, 58.68) circle (  1.16);

\path[draw=drawColor,line width= 0.4pt,line join=round,line cap=round,fill=fillColor] (142.68, 58.55) circle (  1.16);

\path[draw=drawColor,line width= 0.4pt,line join=round,line cap=round,fill=fillColor] (143.14, 58.55) circle (  1.16);

\path[draw=drawColor,line width= 0.4pt,line join=round,line cap=round,fill=fillColor] (143.60, 58.46) circle (  1.16);

\path[draw=drawColor,line width= 0.4pt,line join=round,line cap=round,fill=fillColor] (144.06, 58.46) circle (  1.16);

\path[draw=drawColor,line width= 0.4pt,line join=round,line cap=round,fill=fillColor] (144.51, 58.42) circle (  1.16);

\path[draw=drawColor,line width= 0.4pt,line join=round,line cap=round,fill=fillColor] (144.95, 58.36) circle (  1.16);

\path[draw=drawColor,line width= 0.4pt,line join=round,line cap=round,fill=fillColor] (145.39, 58.22) circle (  1.16);

\path[draw=drawColor,line width= 0.4pt,line join=round,line cap=round,fill=fillColor] (145.83, 58.21) circle (  1.16);

\path[draw=drawColor,line width= 0.4pt,line join=round,line cap=round,fill=fillColor] (146.26, 58.10) circle (  1.16);

\path[draw=drawColor,line width= 0.4pt,line join=round,line cap=round,fill=fillColor] (146.69, 58.08) circle (  1.16);

\path[draw=drawColor,line width= 0.4pt,line join=round,line cap=round,fill=fillColor] (147.11, 58.04) circle (  1.16);

\path[draw=drawColor,line width= 0.4pt,line join=round,line cap=round,fill=fillColor] (147.53, 57.93) circle (  1.16);

\path[draw=drawColor,line width= 0.4pt,line join=round,line cap=round,fill=fillColor] (147.95, 57.92) circle (  1.16);

\path[draw=drawColor,line width= 0.4pt,line join=round,line cap=round,fill=fillColor] (148.36, 57.91) circle (  1.16);

\path[draw=drawColor,line width= 0.4pt,line join=round,line cap=round,fill=fillColor] (148.77, 57.90) circle (  1.16);

\path[draw=drawColor,line width= 0.4pt,line join=round,line cap=round,fill=fillColor] (149.17, 57.89) circle (  1.16);

\path[draw=drawColor,line width= 0.4pt,line join=round,line cap=round,fill=fillColor] (149.57, 57.87) circle (  1.16);

\path[draw=drawColor,line width= 0.4pt,line join=round,line cap=round,fill=fillColor] (149.97, 57.85) circle (  1.16);

\path[draw=drawColor,line width= 0.4pt,line join=round,line cap=round,fill=fillColor] (150.37, 57.79) circle (  1.16);

\path[draw=drawColor,line width= 0.4pt,line join=round,line cap=round,fill=fillColor] (150.76, 57.76) circle (  1.16);

\path[draw=drawColor,line width= 0.4pt,line join=round,line cap=round,fill=fillColor] (151.15, 57.75) circle (  1.16);

\path[draw=drawColor,line width= 0.4pt,line join=round,line cap=round,fill=fillColor] (151.53, 57.75) circle (  1.16);

\path[draw=drawColor,line width= 0.4pt,line join=round,line cap=round,fill=fillColor] (151.91, 57.70) circle (  1.16);

\path[draw=drawColor,line width= 0.4pt,line join=round,line cap=round,fill=fillColor] (152.29, 57.59) circle (  1.16);

\path[draw=drawColor,line width= 0.4pt,line join=round,line cap=round,fill=fillColor] (152.67, 57.56) circle (  1.16);

\path[draw=drawColor,line width= 0.4pt,line join=round,line cap=round,fill=fillColor] (153.04, 57.50) circle (  1.16);

\path[draw=drawColor,line width= 0.4pt,line join=round,line cap=round,fill=fillColor] (153.41, 57.46) circle (  1.16);

\path[draw=drawColor,line width= 0.4pt,line join=round,line cap=round,fill=fillColor] (153.78, 57.40) circle (  1.16);

\path[draw=drawColor,line width= 0.4pt,line join=round,line cap=round,fill=fillColor] (154.14, 57.40) circle (  1.16);

\path[draw=drawColor,line width= 0.4pt,line join=round,line cap=round,fill=fillColor] (154.50, 57.40) circle (  1.16);

\path[draw=drawColor,line width= 0.4pt,line join=round,line cap=round,fill=fillColor] (154.86, 57.36) circle (  1.16);

\path[draw=drawColor,line width= 0.4pt,line join=round,line cap=round,fill=fillColor] (155.22, 57.30) circle (  1.16);

\path[draw=drawColor,line width= 0.4pt,line join=round,line cap=round,fill=fillColor] (155.57, 57.30) circle (  1.16);

\path[draw=drawColor,line width= 0.4pt,line join=round,line cap=round,fill=fillColor] (155.92, 57.19) circle (  1.16);

\path[draw=drawColor,line width= 0.4pt,line join=round,line cap=round,fill=fillColor] (156.27, 57.10) circle (  1.16);

\path[draw=drawColor,line width= 0.4pt,line join=round,line cap=round,fill=fillColor] (156.62, 57.04) circle (  1.16);

\path[draw=drawColor,line width= 0.4pt,line join=round,line cap=round,fill=fillColor] (156.96, 56.87) circle (  1.16);

\path[draw=drawColor,line width= 0.4pt,line join=round,line cap=round,fill=fillColor] (157.30, 56.83) circle (  1.16);

\path[draw=drawColor,line width= 0.4pt,line join=round,line cap=round,fill=fillColor] (157.64, 56.77) circle (  1.16);

\path[draw=drawColor,line width= 0.4pt,line join=round,line cap=round,fill=fillColor] (157.98, 56.77) circle (  1.16);

\path[draw=drawColor,line width= 0.4pt,line join=round,line cap=round,fill=fillColor] (158.31, 56.74) circle (  1.16);

\path[draw=drawColor,line width= 0.4pt,line join=round,line cap=round,fill=fillColor] (158.64, 56.68) circle (  1.16);

\path[draw=drawColor,line width= 0.4pt,line join=round,line cap=round,fill=fillColor] (158.97, 56.66) circle (  1.16);

\path[draw=drawColor,line width= 0.4pt,line join=round,line cap=round,fill=fillColor] (159.30, 56.57) circle (  1.16);

\path[draw=drawColor,line width= 0.4pt,line join=round,line cap=round,fill=fillColor] (159.63, 56.55) circle (  1.16);

\path[draw=drawColor,line width= 0.4pt,line join=round,line cap=round,fill=fillColor] (159.95, 56.48) circle (  1.16);

\path[draw=drawColor,line width= 0.4pt,line join=round,line cap=round,fill=fillColor] (160.27, 56.48) circle (  1.16);

\path[draw=drawColor,line width= 0.4pt,line join=round,line cap=round,fill=fillColor] (160.59, 56.45) circle (  1.16);

\path[draw=drawColor,line width= 0.4pt,line join=round,line cap=round,fill=fillColor] (160.91, 56.44) circle (  1.16);

\path[draw=drawColor,line width= 0.4pt,line join=round,line cap=round,fill=fillColor] (161.23, 56.40) circle (  1.16);

\path[draw=drawColor,line width= 0.4pt,line join=round,line cap=round,fill=fillColor] (161.54, 56.29) circle (  1.16);

\path[draw=drawColor,line width= 0.4pt,line join=round,line cap=round,fill=fillColor] (161.85, 56.27) circle (  1.16);

\path[draw=drawColor,line width= 0.4pt,line join=round,line cap=round,fill=fillColor] (162.16, 56.27) circle (  1.16);

\path[draw=drawColor,line width= 0.4pt,line join=round,line cap=round,fill=fillColor] (162.47, 56.24) circle (  1.16);

\path[draw=drawColor,line width= 0.4pt,line join=round,line cap=round,fill=fillColor] (162.78, 56.13) circle (  1.16);

\path[draw=drawColor,line width= 0.4pt,line join=round,line cap=round,fill=fillColor] (163.08, 56.10) circle (  1.16);

\path[draw=drawColor,line width= 0.4pt,line join=round,line cap=round,fill=fillColor] (163.39, 56.08) circle (  1.16);

\path[draw=drawColor,line width= 0.4pt,line join=round,line cap=round,fill=fillColor] (163.69, 56.06) circle (  1.16);

\path[draw=drawColor,line width= 0.4pt,line join=round,line cap=round,fill=fillColor] (163.99, 56.04) circle (  1.16);

\path[draw=drawColor,line width= 0.4pt,line join=round,line cap=round,fill=fillColor] (164.29, 56.01) circle (  1.16);

\path[draw=drawColor,line width= 0.4pt,line join=round,line cap=round,fill=fillColor] (164.58, 55.93) circle (  1.16);

\path[draw=drawColor,line width= 0.4pt,line join=round,line cap=round,fill=fillColor] (164.88, 55.92) circle (  1.16);

\path[draw=drawColor,line width= 0.4pt,line join=round,line cap=round,fill=fillColor] (165.17, 55.90) circle (  1.16);

\path[draw=drawColor,line width= 0.4pt,line join=round,line cap=round,fill=fillColor] (165.46, 55.86) circle (  1.16);

\path[draw=drawColor,line width= 0.4pt,line join=round,line cap=round,fill=fillColor] (165.75, 55.64) circle (  1.16);

\path[draw=drawColor,line width= 0.4pt,line join=round,line cap=round,fill=fillColor] (166.04, 55.61) circle (  1.16);

\path[draw=drawColor,line width= 0.4pt,line join=round,line cap=round,fill=fillColor] (166.33, 55.53) circle (  1.16);

\path[draw=drawColor,line width= 0.4pt,line join=round,line cap=round,fill=fillColor] (166.61, 55.44) circle (  1.16);

\path[draw=drawColor,line width= 0.4pt,line join=round,line cap=round,fill=fillColor] (166.90, 55.43) circle (  1.16);

\path[draw=drawColor,line width= 0.4pt,line join=round,line cap=round,fill=fillColor] (167.18, 55.38) circle (  1.16);

\path[draw=drawColor,line width= 0.4pt,line join=round,line cap=round,fill=fillColor] (167.46, 55.34) circle (  1.16);

\path[draw=drawColor,line width= 0.4pt,line join=round,line cap=round,fill=fillColor] (167.74, 55.33) circle (  1.16);

\path[draw=drawColor,line width= 0.4pt,line join=round,line cap=round,fill=fillColor] (168.02, 55.32) circle (  1.16);

\path[draw=drawColor,line width= 0.4pt,line join=round,line cap=round,fill=fillColor] (168.30, 55.29) circle (  1.16);

\path[draw=drawColor,line width= 0.4pt,line join=round,line cap=round,fill=fillColor] (168.57, 55.26) circle (  1.16);

\path[draw=drawColor,line width= 0.4pt,line join=round,line cap=round,fill=fillColor] (168.85, 55.20) circle (  1.16);

\path[draw=drawColor,line width= 0.4pt,line join=round,line cap=round,fill=fillColor] (169.12, 55.20) circle (  1.16);

\path[draw=drawColor,line width= 0.4pt,line join=round,line cap=round,fill=fillColor] (169.39, 55.12) circle (  1.16);

\path[draw=drawColor,line width= 0.4pt,line join=round,line cap=round,fill=fillColor] (169.66, 55.11) circle (  1.16);

\path[draw=drawColor,line width= 0.4pt,line join=round,line cap=round,fill=fillColor] (169.93, 55.09) circle (  1.16);

\path[draw=drawColor,line width= 0.4pt,line join=round,line cap=round,fill=fillColor] (170.20, 55.06) circle (  1.16);

\path[draw=drawColor,line width= 0.4pt,line join=round,line cap=round,fill=fillColor] (170.46, 55.02) circle (  1.16);

\path[draw=drawColor,line width= 0.4pt,line join=round,line cap=round,fill=fillColor] (170.73, 55.00) circle (  1.16);

\path[draw=drawColor,line width= 0.4pt,line join=round,line cap=round,fill=fillColor] (170.99, 54.85) circle (  1.16);

\path[draw=drawColor,line width= 0.4pt,line join=round,line cap=round,fill=fillColor] (171.25, 54.84) circle (  1.16);

\path[draw=drawColor,line width= 0.4pt,line join=round,line cap=round,fill=fillColor] (171.52, 54.79) circle (  1.16);

\path[draw=drawColor,line width= 0.4pt,line join=round,line cap=round,fill=fillColor] (171.78, 54.78) circle (  1.16);

\path[draw=drawColor,line width= 0.4pt,line join=round,line cap=round,fill=fillColor] (172.04, 54.77) circle (  1.16);

\path[draw=drawColor,line width= 0.4pt,line join=round,line cap=round,fill=fillColor] (172.29, 54.76) circle (  1.16);

\path[draw=drawColor,line width= 0.4pt,line join=round,line cap=round,fill=fillColor] (172.55, 54.65) circle (  1.16);

\path[draw=drawColor,line width= 0.4pt,line join=round,line cap=round,fill=fillColor] (172.81, 54.62) circle (  1.16);

\path[draw=drawColor,line width= 0.4pt,line join=round,line cap=round,fill=fillColor] (173.06, 54.52) circle (  1.16);

\path[draw=drawColor,line width= 0.4pt,line join=round,line cap=round,fill=fillColor] (173.31, 54.45) circle (  1.16);

\path[draw=drawColor,line width= 0.4pt,line join=round,line cap=round,fill=fillColor] (173.57, 54.44) circle (  1.16);

\path[draw=drawColor,line width= 0.4pt,line join=round,line cap=round,fill=fillColor] (173.82, 54.36) circle (  1.16);

\path[draw=drawColor,line width= 0.4pt,line join=round,line cap=round,fill=fillColor] (174.07, 54.29) circle (  1.16);

\path[draw=drawColor,line width= 0.4pt,line join=round,line cap=round,fill=fillColor] (174.32, 54.28) circle (  1.16);

\path[draw=drawColor,line width= 0.4pt,line join=round,line cap=round,fill=fillColor] (174.56, 54.24) circle (  1.16);

\path[draw=drawColor,line width= 0.4pt,line join=round,line cap=round,fill=fillColor] (174.81, 54.20) circle (  1.16);

\path[draw=drawColor,line width= 0.4pt,line join=round,line cap=round,fill=fillColor] (175.06, 54.17) circle (  1.16);

\path[draw=drawColor,line width= 0.4pt,line join=round,line cap=round,fill=fillColor] (175.30, 54.14) circle (  1.16);

\path[draw=drawColor,line width= 0.4pt,line join=round,line cap=round,fill=fillColor] (175.55, 54.13) circle (  1.16);

\path[draw=drawColor,line width= 0.4pt,line join=round,line cap=round,fill=fillColor] (175.79, 54.11) circle (  1.16);

\path[draw=drawColor,line width= 0.4pt,line join=round,line cap=round,fill=fillColor] (176.03, 54.05) circle (  1.16);

\path[draw=drawColor,line width= 0.4pt,line join=round,line cap=round,fill=fillColor] (176.27, 53.95) circle (  1.16);

\path[draw=drawColor,line width= 0.4pt,line join=round,line cap=round,fill=fillColor] (176.51, 53.94) circle (  1.16);

\path[draw=drawColor,line width= 0.4pt,line join=round,line cap=round,fill=fillColor] (176.75, 53.90) circle (  1.16);

\path[draw=drawColor,line width= 0.4pt,line join=round,line cap=round,fill=fillColor] (176.99, 53.89) circle (  1.16);

\path[draw=drawColor,line width= 0.4pt,line join=round,line cap=round,fill=fillColor] (177.22, 53.80) circle (  1.16);

\path[draw=drawColor,line width= 0.4pt,line join=round,line cap=round,fill=fillColor] (177.46, 53.75) circle (  1.16);

\path[draw=drawColor,line width= 0.4pt,line join=round,line cap=round,fill=fillColor] (177.70, 53.75) circle (  1.16);

\path[draw=drawColor,line width= 0.4pt,line join=round,line cap=round,fill=fillColor] (177.93, 53.60) circle (  1.16);

\path[draw=drawColor,line width= 0.4pt,line join=round,line cap=round,fill=fillColor] (178.16, 53.56) circle (  1.16);

\path[draw=drawColor,line width= 0.4pt,line join=round,line cap=round,fill=fillColor] (178.40, 53.53) circle (  1.16);

\path[draw=drawColor,line width= 0.4pt,line join=round,line cap=round,fill=fillColor] (178.63, 53.38) circle (  1.16);

\path[draw=drawColor,line width= 0.4pt,line join=round,line cap=round,fill=fillColor] (178.86, 53.36) circle (  1.16);

\path[draw=drawColor,line width= 0.4pt,line join=round,line cap=round,fill=fillColor] (179.09, 53.32) circle (  1.16);

\path[draw=drawColor,line width= 0.4pt,line join=round,line cap=round,fill=fillColor] (179.32, 53.31) circle (  1.16);

\path[draw=drawColor,line width= 0.4pt,line join=round,line cap=round,fill=fillColor] (179.55, 53.26) circle (  1.16);

\path[draw=drawColor,line width= 0.4pt,line join=round,line cap=round,fill=fillColor] (179.77, 53.20) circle (  1.16);

\path[draw=drawColor,line width= 0.4pt,line join=round,line cap=round,fill=fillColor] (180.00, 53.17) circle (  1.16);

\path[draw=drawColor,line width= 0.4pt,line join=round,line cap=round,fill=fillColor] (180.22, 53.11) circle (  1.16);

\path[draw=drawColor,line width= 0.4pt,line join=round,line cap=round,fill=fillColor] (180.45, 53.04) circle (  1.16);

\path[draw=drawColor,line width= 0.4pt,line join=round,line cap=round,fill=fillColor] (180.67, 52.98) circle (  1.16);

\path[draw=drawColor,line width= 0.4pt,line join=round,line cap=round,fill=fillColor] (180.90, 52.85) circle (  1.16);

\path[draw=drawColor,line width= 0.4pt,line join=round,line cap=round,fill=fillColor] (181.12, 52.73) circle (  1.16);

\path[draw=drawColor,line width= 0.4pt,line join=round,line cap=round,fill=fillColor] (181.34, 52.66) circle (  1.16);

\path[draw=drawColor,line width= 0.4pt,line join=round,line cap=round,fill=fillColor] (181.56, 52.63) circle (  1.16);

\path[draw=drawColor,line width= 0.4pt,line join=round,line cap=round,fill=fillColor] (181.78, 52.40) circle (  1.16);

\path[draw=drawColor,line width= 0.4pt,line join=round,line cap=round,fill=fillColor] (182.00, 52.40) circle (  1.16);

\path[draw=drawColor,line width= 0.4pt,line join=round,line cap=round,fill=fillColor] (182.22, 52.37) circle (  1.16);

\path[draw=drawColor,line width= 0.4pt,line join=round,line cap=round,fill=fillColor] (182.44, 52.30) circle (  1.16);

\path[draw=drawColor,line width= 0.4pt,line join=round,line cap=round,fill=fillColor] (182.65, 52.23) circle (  1.16);

\path[draw=drawColor,line width= 0.4pt,line join=round,line cap=round,fill=fillColor] (182.87, 52.10) circle (  1.16);

\path[draw=drawColor,line width= 0.4pt,line join=round,line cap=round,fill=fillColor] (183.09, 52.07) circle (  1.16);

\path[draw=drawColor,line width= 0.4pt,line join=round,line cap=round,fill=fillColor] (183.30, 51.92) circle (  1.16);

\path[draw=drawColor,line width= 0.4pt,line join=round,line cap=round,fill=fillColor] (183.51, 51.87) circle (  1.16);

\path[draw=drawColor,line width= 0.4pt,line join=round,line cap=round,fill=fillColor] (183.73, 51.84) circle (  1.16);

\path[draw=drawColor,line width= 0.4pt,line join=round,line cap=round,fill=fillColor] (183.94, 51.75) circle (  1.16);

\path[draw=drawColor,line width= 0.4pt,line join=round,line cap=round,fill=fillColor] (184.15, 51.55) circle (  1.16);

\path[draw=drawColor,line width= 0.4pt,line join=round,line cap=round,fill=fillColor] (184.36, 51.39) circle (  1.16);

\path[draw=drawColor,line width= 0.4pt,line join=round,line cap=round,fill=fillColor] (184.57, 51.39) circle (  1.16);

\path[draw=drawColor,line width= 0.4pt,line join=round,line cap=round,fill=fillColor] (184.78, 51.26) circle (  1.16);

\path[draw=drawColor,line width= 0.4pt,line join=round,line cap=round,fill=fillColor] (184.99, 51.11) circle (  1.16);

\path[draw=drawColor,line width= 0.4pt,line join=round,line cap=round,fill=fillColor] (185.20, 50.86) circle (  1.16);

\path[draw=drawColor,line width= 0.4pt,line join=round,line cap=round,fill=fillColor] (185.41, 50.63) circle (  1.16);

\path[draw=drawColor,line width= 0.4pt,line join=round,line cap=round,fill=fillColor] (185.61, 50.60) circle (  1.16);

\path[draw=drawColor,line width= 0.4pt,line join=round,line cap=round,fill=fillColor] (185.82, 50.56) circle (  1.16);

\path[draw=drawColor,line width= 0.4pt,line join=round,line cap=round,fill=fillColor] (186.03, 50.53) circle (  1.16);

\path[draw=drawColor,line width= 0.4pt,line join=round,line cap=round,fill=fillColor] (186.23, 50.32) circle (  1.16);

\path[draw=drawColor,line width= 0.4pt,line join=round,line cap=round,fill=fillColor] (186.44, 50.02) circle (  1.16);

\path[draw=drawColor,line width= 0.4pt,line join=round,line cap=round,fill=fillColor] (186.64, 49.81) circle (  1.16);

\path[draw=drawColor,line width= 0.4pt,line join=round,line cap=round,fill=fillColor] (186.84, 49.78) circle (  1.16);

\path[draw=drawColor,line width= 0.4pt,line join=round,line cap=round,fill=fillColor] (187.05, 49.72) circle (  1.16);

\path[draw=drawColor,line width= 0.4pt,line join=round,line cap=round,fill=fillColor] (187.25, 49.57) circle (  1.16);

\path[draw=drawColor,line width= 0.4pt,line join=round,line cap=round,fill=fillColor] (187.45, 49.07) circle (  1.16);

\path[draw=drawColor,line width= 0.4pt,line join=round,line cap=round,fill=fillColor] (187.65, 49.05) circle (  1.16);

\path[draw=drawColor,line width= 0.4pt,line join=round,line cap=round,fill=fillColor] (187.85, 48.87) circle (  1.16);

\path[draw=drawColor,line width= 0.4pt,line join=round,line cap=round,fill=fillColor] (188.05, 48.72) circle (  1.16);

\path[draw=drawColor,line width= 0.4pt,line join=round,line cap=round,fill=fillColor] (188.25, 48.64) circle (  1.16);

\path[draw=drawColor,line width= 0.4pt,line join=round,line cap=round,fill=fillColor] (188.45, 48.40) circle (  1.16);

\path[draw=drawColor,line width= 0.4pt,line join=round,line cap=round,fill=fillColor] (188.64, 48.26) circle (  1.16);

\path[draw=drawColor,line width= 0.4pt,line join=round,line cap=round,fill=fillColor] (188.84, 47.87) circle (  1.16);

\path[draw=drawColor,line width= 0.4pt,line join=round,line cap=round,fill=fillColor] (189.04, 47.58) circle (  1.16);

\path[draw=drawColor,line width= 0.4pt,line join=round,line cap=round,fill=fillColor] (189.23, 44.78) circle (  1.16);

\path[draw=drawColor,line width= 0.4pt,line join=round,line cap=round,fill=fillColor] (189.43, 44.78) circle (  1.16);

\path[draw=drawColor,line width= 0.4pt,line join=round,line cap=round,fill=fillColor] (189.62, 44.78) circle (  1.16);

\path[draw=drawColor,line width= 0.4pt,line join=round,line cap=round,fill=fillColor] (189.82, 44.78) circle (  1.16);

\path[draw=drawColor,line width= 0.4pt,line join=round,line cap=round,fill=fillColor] (190.01, 44.78) circle (  1.16);

\path[draw=drawColor,line width= 0.4pt,line join=round,line cap=round,fill=fillColor] (190.20, 44.78) circle (  1.16);

\path[draw=drawColor,line width= 0.4pt,line join=round,line cap=round,fill=fillColor] (190.39, 44.78) circle (  1.16);

\path[draw=drawColor,line width= 0.4pt,line join=round,line cap=round,fill=fillColor] (190.59, 44.78) circle (  1.16);

\path[draw=drawColor,line width= 0.4pt,line join=round,line cap=round,fill=fillColor] (190.78, 44.78) circle (  1.16);

\path[draw=drawColor,line width= 0.4pt,line join=round,line cap=round,fill=fillColor] (190.97, 44.78) circle (  1.16);

\path[draw=drawColor,line width= 0.4pt,line join=round,line cap=round,fill=fillColor] (191.16, 44.78) circle (  1.16);

\path[draw=drawColor,line width= 0.4pt,line join=round,line cap=round,fill=fillColor] (191.35, 44.78) circle (  1.16);

\path[draw=drawColor,line width= 0.4pt,line join=round,line cap=round,fill=fillColor] (191.54, 44.78) circle (  1.16);

\path[draw=drawColor,line width= 0.4pt,line join=round,line cap=round,fill=fillColor] (191.73, 44.78) circle (  1.16);

\path[draw=drawColor,line width= 0.4pt,line join=round,line cap=round,fill=fillColor] (191.91, 44.78) circle (  1.16);

\path[draw=drawColor,line width= 0.4pt,line join=round,line cap=round,fill=fillColor] (192.10, 44.78) circle (  1.16);

\path[draw=drawColor,line width= 0.4pt,line join=round,line cap=round,fill=fillColor] (192.29, 44.78) circle (  1.16);

\path[draw=drawColor,line width= 0.4pt,line join=round,line cap=round,fill=fillColor] (192.47, 44.78) circle (  1.16);

\path[draw=drawColor,line width= 0.4pt,line join=round,line cap=round,fill=fillColor] (192.66, 44.78) circle (  1.16);

\path[draw=drawColor,line width= 0.4pt,line join=round,line cap=round,fill=fillColor] (192.85, 44.78) circle (  1.16);

\path[draw=drawColor,line width= 0.4pt,line join=round,line cap=round,fill=fillColor] (193.03, 44.78) circle (  1.16);

\path[draw=drawColor,line width= 0.4pt,line join=round,line cap=round,fill=fillColor] (193.22, 44.78) circle (  1.16);

\path[draw=drawColor,line width= 0.4pt,line join=round,line cap=round,fill=fillColor] (193.40, 44.78) circle (  1.16);

\path[draw=drawColor,line width= 0.4pt,line join=round,line cap=round,fill=fillColor] (193.58, 44.78) circle (  1.16);

\path[draw=drawColor,line width= 0.4pt,line join=round,line cap=round,fill=fillColor] (193.77, 44.78) circle (  1.16);

\path[draw=drawColor,line width= 0.4pt,line join=round,line cap=round,fill=fillColor] (193.95, 44.78) circle (  1.16);

\path[draw=drawColor,line width= 0.4pt,line join=round,line cap=round,fill=fillColor] (194.13, 44.78) circle (  1.16);

\path[draw=drawColor,line width= 0.4pt,line join=round,line cap=round,fill=fillColor] (194.31, 44.78) circle (  1.16);

\path[draw=drawColor,line width= 0.4pt,line join=round,line cap=round,fill=fillColor] (194.49, 44.78) circle (  1.16);

\path[draw=drawColor,line width= 0.4pt,line join=round,line cap=round,fill=fillColor] (194.67, 44.78) circle (  1.16);

\path[draw=drawColor,line width= 0.4pt,line join=round,line cap=round,fill=fillColor] (194.85, 44.78) circle (  1.16);

\path[draw=drawColor,line width= 0.4pt,line join=round,line cap=round,fill=fillColor] (195.03, 44.78) circle (  1.16);

\path[draw=drawColor,line width= 0.4pt,line join=round,line cap=round,fill=fillColor] (195.21, 44.78) circle (  1.16);

\path[draw=drawColor,line width= 0.4pt,line join=round,line cap=round,fill=fillColor] (195.39, 44.78) circle (  1.16);

\path[draw=drawColor,line width= 0.4pt,line join=round,line cap=round,fill=fillColor] (195.57, 44.78) circle (  1.16);

\path[draw=drawColor,line width= 0.4pt,line join=round,line cap=round,fill=fillColor] (195.75, 44.78) circle (  1.16);

\path[draw=drawColor,line width= 0.4pt,line join=round,line cap=round,fill=fillColor] (195.92, 44.78) circle (  1.16);

\path[draw=drawColor,line width= 0.4pt,line join=round,line cap=round,fill=fillColor] (196.10, 44.78) circle (  1.16);

\path[draw=drawColor,line width= 0.4pt,line join=round,line cap=round,fill=fillColor] (196.28, 44.78) circle (  1.16);

\path[draw=drawColor,line width= 0.4pt,line join=round,line cap=round,fill=fillColor] (196.45, 44.78) circle (  1.16);

\path[draw=drawColor,line width= 0.4pt,line join=round,line cap=round,fill=fillColor] (196.63, 44.78) circle (  1.16);

\path[draw=drawColor,line width= 0.4pt,line join=round,line cap=round,fill=fillColor] (196.80, 44.78) circle (  1.16);

\path[draw=drawColor,line width= 0.4pt,line join=round,line cap=round,fill=fillColor] (196.98, 44.78) circle (  1.16);

\path[draw=drawColor,line width= 0.4pt,line join=round,line cap=round,fill=fillColor] (197.15, 44.78) circle (  1.16);

\path[draw=drawColor,line width= 0.4pt,line join=round,line cap=round,fill=fillColor] (197.33, 44.78) circle (  1.16);

\path[draw=drawColor,line width= 0.4pt,line join=round,line cap=round,fill=fillColor] (197.50, 44.78) circle (  1.16);

\path[draw=drawColor,line width= 0.4pt,line join=round,line cap=round,fill=fillColor] (197.67, 44.78) circle (  1.16);

\path[draw=drawColor,line width= 0.4pt,line join=round,line cap=round,fill=fillColor] (197.85, 44.78) circle (  1.16);

\path[draw=drawColor,line width= 0.4pt,line join=round,line cap=round,fill=fillColor] (198.02, 44.78) circle (  1.16);

\path[draw=drawColor,line width= 0.4pt,line join=round,line cap=round,fill=fillColor] (198.19, 44.78) circle (  1.16);

\path[draw=drawColor,line width= 0.4pt,line join=round,line cap=round,fill=fillColor] (198.36, 44.78) circle (  1.16);

\path[draw=drawColor,line width= 0.4pt,line join=round,line cap=round,fill=fillColor] (198.53, 44.78) circle (  1.16);

\path[draw=drawColor,line width= 0.4pt,line join=round,line cap=round,fill=fillColor] (198.70, 44.78) circle (  1.16);

\path[draw=drawColor,line width= 0.4pt,line join=round,line cap=round,fill=fillColor] (198.87, 44.78) circle (  1.16);

\path[draw=drawColor,line width= 0.4pt,line join=round,line cap=round,fill=fillColor] (199.04, 44.78) circle (  1.16);

\path[draw=drawColor,line width= 0.4pt,line join=round,line cap=round,fill=fillColor] (199.21, 44.78) circle (  1.16);

\path[draw=drawColor,line width= 0.4pt,line join=round,line cap=round,fill=fillColor] (199.38, 44.78) circle (  1.16);

\path[draw=drawColor,line width= 0.4pt,line join=round,line cap=round,fill=fillColor] (199.55, 44.78) circle (  1.16);

\path[draw=drawColor,line width= 0.4pt,line join=round,line cap=round,fill=fillColor] (199.72, 44.78) circle (  1.16);

\path[draw=drawColor,line width= 0.4pt,line join=round,line cap=round,fill=fillColor] (199.88, 44.78) circle (  1.16);

\path[draw=drawColor,line width= 0.4pt,line join=round,line cap=round,fill=fillColor] (200.05, 44.78) circle (  1.16);

\path[draw=drawColor,line width= 0.4pt,line join=round,line cap=round,fill=fillColor] (200.22, 44.78) circle (  1.16);

\path[draw=drawColor,line width= 0.4pt,line join=round,line cap=round,fill=fillColor] (200.38, 44.78) circle (  1.16);

\path[draw=drawColor,line width= 0.4pt,line join=round,line cap=round,fill=fillColor] (200.55, 44.78) circle (  1.16);

\path[draw=drawColor,line width= 0.4pt,line join=round,line cap=round,fill=fillColor] (200.72, 44.78) circle (  1.16);

\path[draw=drawColor,line width= 0.4pt,line join=round,line cap=round,fill=fillColor] (200.88, 44.78) circle (  1.16);

\path[draw=drawColor,line width= 0.4pt,line join=round,line cap=round,fill=fillColor] (201.05, 44.78) circle (  1.16);

\path[draw=drawColor,line width= 0.4pt,line join=round,line cap=round,fill=fillColor] (201.21, 44.78) circle (  1.16);

\path[draw=drawColor,line width= 0.4pt,line join=round,line cap=round,fill=fillColor] (201.37, 44.78) circle (  1.16);

\path[draw=drawColor,line width= 0.4pt,line join=round,line cap=round,fill=fillColor] (201.54, 44.78) circle (  1.16);

\path[draw=drawColor,line width= 0.4pt,line join=round,line cap=round,fill=fillColor] (201.70, 44.78) circle (  1.16);

\path[draw=drawColor,line width= 0.4pt,line join=round,line cap=round,fill=fillColor] (201.86, 44.78) circle (  1.16);

\path[draw=drawColor,line width= 0.4pt,line join=round,line cap=round,fill=fillColor] (202.03, 44.78) circle (  1.16);

\path[draw=drawColor,line width= 0.4pt,line join=round,line cap=round,fill=fillColor] (202.19, 44.78) circle (  1.16);

\path[draw=drawColor,line width= 0.4pt,line join=round,line cap=round,fill=fillColor] (202.35, 44.78) circle (  1.16);

\path[draw=drawColor,line width= 0.4pt,line join=round,line cap=round,fill=fillColor] (202.51, 44.78) circle (  1.16);

\path[draw=drawColor,line width= 0.4pt,line join=round,line cap=round,fill=fillColor] (202.67, 44.78) circle (  1.16);

\path[draw=drawColor,line width= 0.4pt,line join=round,line cap=round,fill=fillColor] (202.83, 44.78) circle (  1.16);

\path[draw=drawColor,line width= 0.4pt,line join=round,line cap=round,fill=fillColor] (202.99, 44.78) circle (  1.16);

\path[draw=drawColor,line width= 0.4pt,line join=round,line cap=round,fill=fillColor] (203.16, 44.78) circle (  1.16);

\path[draw=drawColor,line width= 0.4pt,line join=round,line cap=round,fill=fillColor] (203.31, 44.78) circle (  1.16);

\path[draw=drawColor,line width= 0.4pt,line join=round,line cap=round,fill=fillColor] (203.47, 44.78) circle (  1.16);

\path[draw=drawColor,line width= 0.4pt,line join=round,line cap=round,fill=fillColor] (203.63, 44.78) circle (  1.16);

\path[draw=drawColor,line width= 0.4pt,line join=round,line cap=round,fill=fillColor] (203.79, 44.78) circle (  1.16);

\path[draw=drawColor,line width= 0.4pt,line join=round,line cap=round,fill=fillColor] (203.95, 44.78) circle (  1.16);

\path[draw=drawColor,line width= 0.4pt,line join=round,line cap=round,fill=fillColor] (204.11, 44.78) circle (  1.16);

\path[draw=drawColor,line width= 0.4pt,line join=round,line cap=round,fill=fillColor] (204.27, 44.78) circle (  1.16);

\path[draw=drawColor,line width= 0.4pt,line join=round,line cap=round,fill=fillColor] (204.42, 44.78) circle (  1.16);

\path[draw=drawColor,line width= 0.4pt,line join=round,line cap=round,fill=fillColor] (204.58, 44.78) circle (  1.16);

\path[draw=drawColor,line width= 0.4pt,line join=round,line cap=round,fill=fillColor] (204.74, 44.78) circle (  1.16);

\path[draw=drawColor,line width= 0.4pt,line join=round,line cap=round,fill=fillColor] (204.89, 44.78) circle (  1.16);

\path[draw=drawColor,line width= 0.4pt,line join=round,line cap=round,fill=fillColor] (205.05, 44.78) circle (  1.16);

\path[draw=drawColor,line width= 0.4pt,line join=round,line cap=round,fill=fillColor] (205.21, 44.78) circle (  1.16);

\path[draw=drawColor,line width= 0.4pt,line join=round,line cap=round,fill=fillColor] (205.36, 44.78) circle (  1.16);

\path[draw=drawColor,line width= 0.4pt,line join=round,line cap=round,fill=fillColor] (205.52, 44.78) circle (  1.16);

\path[draw=drawColor,line width= 0.4pt,line join=round,line cap=round,fill=fillColor] (205.67, 44.78) circle (  1.16);

\path[draw=drawColor,line width= 0.4pt,line join=round,line cap=round,fill=fillColor] (205.83, 44.78) circle (  1.16);

\path[draw=drawColor,line width= 0.4pt,line join=round,line cap=round,fill=fillColor] (205.98, 44.78) circle (  1.16);

\path[draw=drawColor,line width= 0.4pt,line join=round,line cap=round,fill=fillColor] (206.13, 44.78) circle (  1.16);

\path[draw=drawColor,line width= 0.4pt,line join=round,line cap=round,fill=fillColor] (206.29, 44.78) circle (  1.16);

\path[draw=drawColor,line width= 0.4pt,line join=round,line cap=round,fill=fillColor] (206.44, 44.78) circle (  1.16);

\path[draw=drawColor,line width= 0.4pt,line join=round,line cap=round,fill=fillColor] (206.59, 44.78) circle (  1.16);

\path[draw=drawColor,line width= 0.4pt,line join=round,line cap=round,fill=fillColor] (206.74, 44.78) circle (  1.16);

\path[draw=drawColor,line width= 0.4pt,line join=round,line cap=round,fill=fillColor] (206.90, 44.78) circle (  1.16);

\path[draw=drawColor,line width= 0.4pt,line join=round,line cap=round,fill=fillColor] (207.05, 44.78) circle (  1.16);

\path[draw=drawColor,line width= 0.4pt,line join=round,line cap=round,fill=fillColor] (207.20, 44.78) circle (  1.16);

\path[draw=drawColor,line width= 0.4pt,line join=round,line cap=round,fill=fillColor] (207.35, 44.78) circle (  1.16);

\path[draw=drawColor,line width= 0.4pt,line join=round,line cap=round,fill=fillColor] (207.50, 44.78) circle (  1.16);

\path[draw=drawColor,line width= 0.4pt,line join=round,line cap=round,fill=fillColor] (207.65, 44.78) circle (  1.16);

\path[draw=drawColor,line width= 0.4pt,line join=round,line cap=round,fill=fillColor] (207.80, 44.78) circle (  1.16);

\path[draw=drawColor,line width= 0.4pt,line join=round,line cap=round,fill=fillColor] (207.95, 44.78) circle (  1.16);

\path[draw=drawColor,line width= 0.4pt,line join=round,line cap=round,fill=fillColor] (208.10, 44.78) circle (  1.16);

\path[draw=drawColor,line width= 0.4pt,line join=round,line cap=round,fill=fillColor] (208.25, 44.78) circle (  1.16);

\path[draw=drawColor,line width= 0.4pt,line join=round,line cap=round,fill=fillColor] (208.40, 44.78) circle (  1.16);

\path[draw=drawColor,line width= 0.4pt,line join=round,line cap=round,fill=fillColor] (208.55, 44.78) circle (  1.16);

\path[draw=drawColor,line width= 0.4pt,line join=round,line cap=round,fill=fillColor] (208.70, 44.78) circle (  1.16);

\path[draw=drawColor,line width= 0.4pt,line join=round,line cap=round,fill=fillColor] (208.85, 44.78) circle (  1.16);

\path[draw=drawColor,line width= 0.4pt,line join=round,line cap=round,fill=fillColor] (209.00, 44.78) circle (  1.16);

\path[draw=drawColor,line width= 0.4pt,line join=round,line cap=round,fill=fillColor] (209.14, 44.78) circle (  1.16);

\path[draw=drawColor,line width= 0.4pt,line join=round,line cap=round,fill=fillColor] (209.29, 44.78) circle (  1.16);

\path[draw=drawColor,line width= 0.4pt,line join=round,line cap=round,fill=fillColor] (209.44, 44.78) circle (  1.16);

\path[draw=drawColor,line width= 0.4pt,line join=round,line cap=round,fill=fillColor] (209.59, 44.78) circle (  1.16);

\path[draw=drawColor,line width= 0.4pt,line join=round,line cap=round,fill=fillColor] (209.73, 44.78) circle (  1.16);

\path[draw=drawColor,line width= 0.4pt,line join=round,line cap=round,fill=fillColor] (209.88, 44.78) circle (  1.16);

\path[draw=drawColor,line width= 0.4pt,line join=round,line cap=round,fill=fillColor] (210.02, 44.78) circle (  1.16);

\path[draw=drawColor,line width= 0.4pt,line join=round,line cap=round,fill=fillColor] (210.17, 44.78) circle (  1.16);

\path[draw=drawColor,line width= 0.4pt,line join=round,line cap=round,fill=fillColor] (210.32, 44.78) circle (  1.16);

\path[draw=drawColor,line width= 0.4pt,line join=round,line cap=round,fill=fillColor] (210.46, 44.78) circle (  1.16);

\path[draw=drawColor,line width= 0.4pt,line join=round,line cap=round,fill=fillColor] (210.61, 44.78) circle (  1.16);

\path[draw=drawColor,line width= 0.4pt,line join=round,line cap=round,fill=fillColor] (210.75, 44.78) circle (  1.16);

\path[draw=drawColor,line width= 0.4pt,line join=round,line cap=round,fill=fillColor] (210.89, 44.78) circle (  1.16);

\path[draw=drawColor,line width= 0.4pt,line join=round,line cap=round,fill=fillColor] (211.04, 44.78) circle (  1.16);

\path[draw=drawColor,line width= 0.4pt,line join=round,line cap=round,fill=fillColor] (211.18, 44.78) circle (  1.16);

\path[draw=drawColor,line width= 0.4pt,line join=round,line cap=round,fill=fillColor] (211.33, 44.78) circle (  1.16);

\path[draw=drawColor,line width= 0.4pt,line join=round,line cap=round,fill=fillColor] (211.47, 44.78) circle (  1.16);

\path[draw=drawColor,line width= 0.4pt,line join=round,line cap=round,fill=fillColor] (211.61, 44.78) circle (  1.16);

\path[draw=drawColor,line width= 0.4pt,line join=round,line cap=round,fill=fillColor] (211.76, 44.78) circle (  1.16);

\path[draw=drawColor,line width= 0.4pt,line join=round,line cap=round,fill=fillColor] (211.90, 44.78) circle (  1.16);

\path[draw=drawColor,line width= 0.4pt,line join=round,line cap=round,fill=fillColor] (212.04, 44.78) circle (  1.16);

\path[draw=drawColor,line width= 0.4pt,line join=round,line cap=round,fill=fillColor] (212.18, 44.78) circle (  1.16);

\path[draw=drawColor,line width= 0.4pt,line join=round,line cap=round,fill=fillColor] (212.32, 44.78) circle (  1.16);

\path[draw=drawColor,line width= 0.4pt,line join=round,line cap=round,fill=fillColor] (212.47, 44.78) circle (  1.16);

\path[draw=drawColor,line width= 0.4pt,line join=round,line cap=round,fill=fillColor] (212.61, 44.78) circle (  1.16);

\path[draw=drawColor,line width= 0.4pt,line join=round,line cap=round,fill=fillColor] (212.75, 44.78) circle (  1.16);

\path[draw=drawColor,line width= 0.4pt,line join=round,line cap=round,fill=fillColor] (212.89, 44.78) circle (  1.16);

\path[draw=drawColor,line width= 0.4pt,line join=round,line cap=round,fill=fillColor] (213.03, 44.78) circle (  1.16);

\path[draw=drawColor,line width= 0.4pt,line join=round,line cap=round,fill=fillColor] (213.17, 44.78) circle (  1.16);

\path[draw=drawColor,line width= 0.4pt,line join=round,line cap=round,fill=fillColor] (213.31, 44.78) circle (  1.16);

\path[draw=drawColor,line width= 0.4pt,line join=round,line cap=round,fill=fillColor] (213.45, 44.78) circle (  1.16);

\path[draw=drawColor,line width= 0.4pt,line join=round,line cap=round,fill=fillColor] (213.59, 44.78) circle (  1.16);

\path[draw=drawColor,line width= 0.4pt,line join=round,line cap=round,fill=fillColor] (213.73, 44.78) circle (  1.16);

\path[draw=drawColor,line width= 0.4pt,line join=round,line cap=round,fill=fillColor] (213.87, 44.78) circle (  1.16);

\path[draw=drawColor,line width= 0.4pt,line join=round,line cap=round,fill=fillColor] (214.00, 44.78) circle (  1.16);

\path[draw=drawColor,line width= 0.4pt,line join=round,line cap=round,fill=fillColor] (214.14, 44.78) circle (  1.16);

\path[draw=drawColor,line width= 0.4pt,line join=round,line cap=round,fill=fillColor] (214.28, 44.78) circle (  1.16);

\path[draw=drawColor,line width= 0.4pt,line join=round,line cap=round,fill=fillColor] (214.42, 44.78) circle (  1.16);

\path[draw=drawColor,line width= 0.4pt,line join=round,line cap=round,fill=fillColor] (214.56, 44.78) circle (  1.16);

\path[draw=drawColor,line width= 0.4pt,line join=round,line cap=round,fill=fillColor] (214.69, 44.78) circle (  1.16);

\path[draw=drawColor,line width= 0.4pt,line join=round,line cap=round,fill=fillColor] (214.83, 44.78) circle (  1.16);

\path[draw=drawColor,line width= 0.4pt,line join=round,line cap=round,fill=fillColor] (214.97, 44.78) circle (  1.16);

\path[draw=drawColor,line width= 0.4pt,line join=round,line cap=round,fill=fillColor] (215.11, 44.78) circle (  1.16);

\path[draw=drawColor,line width= 0.4pt,line join=round,line cap=round,fill=fillColor] (215.24, 44.78) circle (  1.16);

\path[draw=drawColor,line width= 0.4pt,line join=round,line cap=round,fill=fillColor] (215.38, 44.78) circle (  1.16);

\path[draw=drawColor,line width= 0.4pt,line join=round,line cap=round,fill=fillColor] (215.51, 44.78) circle (  1.16);

\path[draw=drawColor,line width= 0.4pt,line join=round,line cap=round,fill=fillColor] (215.65, 44.78) circle (  1.16);

\path[draw=drawColor,line width= 0.4pt,line join=round,line cap=round,fill=fillColor] (215.79, 44.78) circle (  1.16);

\path[draw=drawColor,line width= 0.4pt,line join=round,line cap=round,fill=fillColor] (215.92, 44.78) circle (  1.16);

\path[draw=drawColor,line width= 0.4pt,line join=round,line cap=round,fill=fillColor] (216.06, 44.78) circle (  1.16);

\path[draw=drawColor,line width= 0.4pt,line join=round,line cap=round,fill=fillColor] (216.19, 44.78) circle (  1.16);

\path[draw=drawColor,line width= 0.4pt,line join=round,line cap=round,fill=fillColor] (216.33, 44.78) circle (  1.16);

\path[draw=drawColor,line width= 0.4pt,line join=round,line cap=round,fill=fillColor] (216.46, 44.78) circle (  1.16);

\path[draw=drawColor,line width= 0.4pt,line join=round,line cap=round,fill=fillColor] (216.59, 44.78) circle (  1.16);

\path[draw=drawColor,line width= 0.4pt,line join=round,line cap=round,fill=fillColor] (216.73, 44.78) circle (  1.16);

\path[draw=drawColor,line width= 0.4pt,line join=round,line cap=round,fill=fillColor] (216.86, 44.78) circle (  1.16);

\path[draw=drawColor,line width= 0.4pt,line join=round,line cap=round,fill=fillColor] (217.00, 44.78) circle (  1.16);

\path[draw=drawColor,line width= 0.4pt,line join=round,line cap=round,fill=fillColor] (217.13, 44.78) circle (  1.16);

\path[draw=drawColor,line width= 0.4pt,line join=round,line cap=round,fill=fillColor] (217.26, 44.78) circle (  1.16);

\path[draw=drawColor,line width= 0.4pt,line join=round,line cap=round,fill=fillColor] (217.40, 44.78) circle (  1.16);

\path[draw=drawColor,line width= 0.4pt,line join=round,line cap=round,fill=fillColor] (217.53, 44.78) circle (  1.16);

\path[draw=drawColor,line width= 0.4pt,line join=round,line cap=round,fill=fillColor] (217.66, 44.78) circle (  1.16);

\path[draw=drawColor,line width= 0.4pt,line join=round,line cap=round,fill=fillColor] (217.79, 44.78) circle (  1.16);

\path[draw=drawColor,line width= 0.4pt,line join=round,line cap=round,fill=fillColor] (217.92, 44.78) circle (  1.16);

\path[draw=drawColor,line width= 0.4pt,line join=round,line cap=round,fill=fillColor] (218.06, 44.78) circle (  1.16);

\path[draw=drawColor,line width= 0.4pt,line join=round,line cap=round,fill=fillColor] (218.19, 44.78) circle (  1.16);

\path[draw=drawColor,line width= 0.4pt,line join=round,line cap=round,fill=fillColor] (218.32, 44.78) circle (  1.16);

\path[draw=drawColor,line width= 0.4pt,line join=round,line cap=round,fill=fillColor] (218.45, 44.78) circle (  1.16);

\path[draw=drawColor,line width= 0.4pt,line join=round,line cap=round,fill=fillColor] (218.58, 44.78) circle (  1.16);

\path[draw=drawColor,line width= 0.4pt,line join=round,line cap=round,fill=fillColor] (218.71, 44.78) circle (  1.16);

\path[draw=drawColor,line width= 0.4pt,line join=round,line cap=round,fill=fillColor] (218.84, 44.78) circle (  1.16);

\path[draw=drawColor,line width= 0.4pt,line join=round,line cap=round,fill=fillColor] (218.97, 44.78) circle (  1.16);

\path[draw=drawColor,line width= 0.4pt,line join=round,line cap=round,fill=fillColor] (219.10, 44.78) circle (  1.16);
\definecolor[named]{drawColor}{rgb}{0.30,0.69,0.29}
\definecolor[named]{fillColor}{rgb}{0.30,0.69,0.29}

\path[draw=drawColor,line width= 0.4pt,line join=round,line cap=round,fill=fillColor] ( 78.97, 84.15) circle (  1.16);

\path[draw=drawColor,line width= 0.4pt,line join=round,line cap=round,fill=fillColor] ( 84.61, 70.47) circle (  1.16);

\path[draw=drawColor,line width= 0.4pt,line join=round,line cap=round,fill=fillColor] ( 88.57, 69.71) circle (  1.16);

\path[draw=drawColor,line width= 0.4pt,line join=round,line cap=round,fill=fillColor] ( 91.71, 68.77) circle (  1.16);

\path[draw=drawColor,line width= 0.4pt,line join=round,line cap=round,fill=fillColor] ( 94.37, 68.48) circle (  1.16);

\path[draw=drawColor,line width= 0.4pt,line join=round,line cap=round,fill=fillColor] ( 96.70, 67.34) circle (  1.16);

\path[draw=drawColor,line width= 0.4pt,line join=round,line cap=round,fill=fillColor] ( 98.78, 66.45) circle (  1.16);

\path[draw=drawColor,line width= 0.4pt,line join=round,line cap=round,fill=fillColor] (100.67, 66.43) circle (  1.16);

\path[draw=drawColor,line width= 0.4pt,line join=round,line cap=round,fill=fillColor] (102.40, 66.17) circle (  1.16);

\path[draw=drawColor,line width= 0.4pt,line join=round,line cap=round,fill=fillColor] (104.02, 66.08) circle (  1.16);

\path[draw=drawColor,line width= 0.4pt,line join=round,line cap=round,fill=fillColor] (105.53, 65.80) circle (  1.16);

\path[draw=drawColor,line width= 0.4pt,line join=round,line cap=round,fill=fillColor] (106.95, 65.63) circle (  1.16);

\path[draw=drawColor,line width= 0.4pt,line join=round,line cap=round,fill=fillColor] (108.29, 64.79) circle (  1.16);

\path[draw=drawColor,line width= 0.4pt,line join=round,line cap=round,fill=fillColor] (109.56, 64.13) circle (  1.16);

\path[draw=drawColor,line width= 0.4pt,line join=round,line cap=round,fill=fillColor] (110.78, 63.96) circle (  1.16);

\path[draw=drawColor,line width= 0.4pt,line join=round,line cap=round,fill=fillColor] (111.94, 63.49) circle (  1.16);

\path[draw=drawColor,line width= 0.4pt,line join=round,line cap=round,fill=fillColor] (113.06, 63.03) circle (  1.16);

\path[draw=drawColor,line width= 0.4pt,line join=round,line cap=round,fill=fillColor] (114.13, 62.88) circle (  1.16);

\path[draw=drawColor,line width= 0.4pt,line join=round,line cap=round,fill=fillColor] (115.17, 62.84) circle (  1.16);

\path[draw=drawColor,line width= 0.4pt,line join=round,line cap=round,fill=fillColor] (116.17, 62.51) circle (  1.16);

\path[draw=drawColor,line width= 0.4pt,line join=round,line cap=round,fill=fillColor] (117.13, 62.37) circle (  1.16);

\path[draw=drawColor,line width= 0.4pt,line join=round,line cap=round,fill=fillColor] (118.07, 62.35) circle (  1.16);

\path[draw=drawColor,line width= 0.4pt,line join=round,line cap=round,fill=fillColor] (118.97, 62.08) circle (  1.16);

\path[draw=drawColor,line width= 0.4pt,line join=round,line cap=round,fill=fillColor] (119.86, 61.94) circle (  1.16);

\path[draw=drawColor,line width= 0.4pt,line join=round,line cap=round,fill=fillColor] (120.71, 61.81) circle (  1.16);

\path[draw=drawColor,line width= 0.4pt,line join=round,line cap=round,fill=fillColor] (121.55, 61.64) circle (  1.16);

\path[draw=drawColor,line width= 0.4pt,line join=round,line cap=round,fill=fillColor] (122.36, 61.28) circle (  1.16);

\path[draw=drawColor,line width= 0.4pt,line join=round,line cap=round,fill=fillColor] (123.16, 61.25) circle (  1.16);

\path[draw=drawColor,line width= 0.4pt,line join=round,line cap=round,fill=fillColor] (123.93, 61.04) circle (  1.16);

\path[draw=drawColor,line width= 0.4pt,line join=round,line cap=round,fill=fillColor] (124.69, 61.03) circle (  1.16);

\path[draw=drawColor,line width= 0.4pt,line join=round,line cap=round,fill=fillColor] (125.43, 60.86) circle (  1.16);

\path[draw=drawColor,line width= 0.4pt,line join=round,line cap=round,fill=fillColor] (126.15, 60.86) circle (  1.16);

\path[draw=drawColor,line width= 0.4pt,line join=round,line cap=round,fill=fillColor] (126.86, 60.85) circle (  1.16);

\path[draw=drawColor,line width= 0.4pt,line join=round,line cap=round,fill=fillColor] (127.56, 60.73) circle (  1.16);

\path[draw=drawColor,line width= 0.4pt,line join=round,line cap=round,fill=fillColor] (128.24, 60.61) circle (  1.16);

\path[draw=drawColor,line width= 0.4pt,line join=round,line cap=round,fill=fillColor] (128.91, 60.54) circle (  1.16);

\path[draw=drawColor,line width= 0.4pt,line join=round,line cap=round,fill=fillColor] (129.57, 60.49) circle (  1.16);

\path[draw=drawColor,line width= 0.4pt,line join=round,line cap=round,fill=fillColor] (130.21, 60.46) circle (  1.16);

\path[draw=drawColor,line width= 0.4pt,line join=round,line cap=round,fill=fillColor] (130.85, 60.43) circle (  1.16);

\path[draw=drawColor,line width= 0.4pt,line join=round,line cap=round,fill=fillColor] (131.47, 60.34) circle (  1.16);

\path[draw=drawColor,line width= 0.4pt,line join=round,line cap=round,fill=fillColor] (132.09, 60.08) circle (  1.16);

\path[draw=drawColor,line width= 0.4pt,line join=round,line cap=round,fill=fillColor] (132.69, 60.04) circle (  1.16);

\path[draw=drawColor,line width= 0.4pt,line join=round,line cap=round,fill=fillColor] (133.28, 59.94) circle (  1.16);

\path[draw=drawColor,line width= 0.4pt,line join=round,line cap=round,fill=fillColor] (133.87, 59.93) circle (  1.16);

\path[draw=drawColor,line width= 0.4pt,line join=round,line cap=round,fill=fillColor] (134.44, 59.90) circle (  1.16);

\path[draw=drawColor,line width= 0.4pt,line join=round,line cap=round,fill=fillColor] (135.01, 59.63) circle (  1.16);

\path[draw=drawColor,line width= 0.4pt,line join=round,line cap=round,fill=fillColor] (135.57, 59.58) circle (  1.16);

\path[draw=drawColor,line width= 0.4pt,line join=round,line cap=round,fill=fillColor] (136.12, 59.52) circle (  1.16);

\path[draw=drawColor,line width= 0.4pt,line join=round,line cap=round,fill=fillColor] (136.67, 59.51) circle (  1.16);

\path[draw=drawColor,line width= 0.4pt,line join=round,line cap=round,fill=fillColor] (137.20, 59.47) circle (  1.16);

\path[draw=drawColor,line width= 0.4pt,line join=round,line cap=round,fill=fillColor] (137.73, 59.45) circle (  1.16);

\path[draw=drawColor,line width= 0.4pt,line join=round,line cap=round,fill=fillColor] (138.25, 59.40) circle (  1.16);

\path[draw=drawColor,line width= 0.4pt,line join=round,line cap=round,fill=fillColor] (138.77, 59.39) circle (  1.16);

\path[draw=drawColor,line width= 0.4pt,line join=round,line cap=round,fill=fillColor] (139.28, 59.27) circle (  1.16);

\path[draw=drawColor,line width= 0.4pt,line join=round,line cap=round,fill=fillColor] (139.78, 59.26) circle (  1.16);

\path[draw=drawColor,line width= 0.4pt,line join=round,line cap=round,fill=fillColor] (140.28, 59.25) circle (  1.16);

\path[draw=drawColor,line width= 0.4pt,line join=round,line cap=round,fill=fillColor] (140.77, 59.19) circle (  1.16);

\path[draw=drawColor,line width= 0.4pt,line join=round,line cap=round,fill=fillColor] (141.26, 59.02) circle (  1.16);

\path[draw=drawColor,line width= 0.4pt,line join=round,line cap=round,fill=fillColor] (141.74, 58.95) circle (  1.16);

\path[draw=drawColor,line width= 0.4pt,line join=round,line cap=round,fill=fillColor] (142.21, 58.95) circle (  1.16);

\path[draw=drawColor,line width= 0.4pt,line join=round,line cap=round,fill=fillColor] (142.68, 58.89) circle (  1.16);

\path[draw=drawColor,line width= 0.4pt,line join=round,line cap=round,fill=fillColor] (143.14, 58.86) circle (  1.16);

\path[draw=drawColor,line width= 0.4pt,line join=round,line cap=round,fill=fillColor] (143.60, 58.74) circle (  1.16);

\path[draw=drawColor,line width= 0.4pt,line join=round,line cap=round,fill=fillColor] (144.06, 58.73) circle (  1.16);

\path[draw=drawColor,line width= 0.4pt,line join=round,line cap=round,fill=fillColor] (144.51, 58.68) circle (  1.16);

\path[draw=drawColor,line width= 0.4pt,line join=round,line cap=round,fill=fillColor] (144.95, 58.55) circle (  1.16);

\path[draw=drawColor,line width= 0.4pt,line join=round,line cap=round,fill=fillColor] (145.39, 58.49) circle (  1.16);

\path[draw=drawColor,line width= 0.4pt,line join=round,line cap=round,fill=fillColor] (145.83, 58.46) circle (  1.16);

\path[draw=drawColor,line width= 0.4pt,line join=round,line cap=round,fill=fillColor] (146.26, 58.42) circle (  1.16);

\path[draw=drawColor,line width= 0.4pt,line join=round,line cap=round,fill=fillColor] (146.69, 58.42) circle (  1.16);

\path[draw=drawColor,line width= 0.4pt,line join=round,line cap=round,fill=fillColor] (147.11, 58.36) circle (  1.16);

\path[draw=drawColor,line width= 0.4pt,line join=round,line cap=round,fill=fillColor] (147.53, 58.30) circle (  1.16);

\path[draw=drawColor,line width= 0.4pt,line join=round,line cap=round,fill=fillColor] (147.95, 58.23) circle (  1.16);

\path[draw=drawColor,line width= 0.4pt,line join=round,line cap=round,fill=fillColor] (148.36, 58.22) circle (  1.16);

\path[draw=drawColor,line width= 0.4pt,line join=round,line cap=round,fill=fillColor] (148.77, 58.16) circle (  1.16);

\path[draw=drawColor,line width= 0.4pt,line join=round,line cap=round,fill=fillColor] (149.17, 58.09) circle (  1.16);

\path[draw=drawColor,line width= 0.4pt,line join=round,line cap=round,fill=fillColor] (149.57, 57.81) circle (  1.16);

\path[draw=drawColor,line width= 0.4pt,line join=round,line cap=round,fill=fillColor] (149.97, 57.79) circle (  1.16);

\path[draw=drawColor,line width= 0.4pt,line join=round,line cap=round,fill=fillColor] (150.37, 57.76) circle (  1.16);

\path[draw=drawColor,line width= 0.4pt,line join=round,line cap=round,fill=fillColor] (150.76, 57.76) circle (  1.16);

\path[draw=drawColor,line width= 0.4pt,line join=round,line cap=round,fill=fillColor] (151.15, 57.70) circle (  1.16);

\path[draw=drawColor,line width= 0.4pt,line join=round,line cap=round,fill=fillColor] (151.53, 57.63) circle (  1.16);

\path[draw=drawColor,line width= 0.4pt,line join=round,line cap=round,fill=fillColor] (151.91, 57.58) circle (  1.16);

\path[draw=drawColor,line width= 0.4pt,line join=round,line cap=round,fill=fillColor] (152.29, 57.56) circle (  1.16);

\path[draw=drawColor,line width= 0.4pt,line join=round,line cap=round,fill=fillColor] (152.67, 57.52) circle (  1.16);

\path[draw=drawColor,line width= 0.4pt,line join=round,line cap=round,fill=fillColor] (153.04, 57.38) circle (  1.16);

\path[draw=drawColor,line width= 0.4pt,line join=round,line cap=round,fill=fillColor] (153.41, 57.36) circle (  1.16);

\path[draw=drawColor,line width= 0.4pt,line join=round,line cap=round,fill=fillColor] (153.78, 57.34) circle (  1.16);

\path[draw=drawColor,line width= 0.4pt,line join=round,line cap=round,fill=fillColor] (154.14, 57.30) circle (  1.16);

\path[draw=drawColor,line width= 0.4pt,line join=round,line cap=round,fill=fillColor] (154.50, 57.30) circle (  1.16);

\path[draw=drawColor,line width= 0.4pt,line join=round,line cap=round,fill=fillColor] (154.86, 57.25) circle (  1.16);

\path[draw=drawColor,line width= 0.4pt,line join=round,line cap=round,fill=fillColor] (155.22, 57.22) circle (  1.16);

\path[draw=drawColor,line width= 0.4pt,line join=round,line cap=round,fill=fillColor] (155.57, 57.20) circle (  1.16);

\path[draw=drawColor,line width= 0.4pt,line join=round,line cap=round,fill=fillColor] (155.92, 57.18) circle (  1.16);

\path[draw=drawColor,line width= 0.4pt,line join=round,line cap=round,fill=fillColor] (156.27, 57.10) circle (  1.16);

\path[draw=drawColor,line width= 0.4pt,line join=round,line cap=round,fill=fillColor] (156.62, 56.96) circle (  1.16);

\path[draw=drawColor,line width= 0.4pt,line join=round,line cap=round,fill=fillColor] (156.96, 56.92) circle (  1.16);

\path[draw=drawColor,line width= 0.4pt,line join=round,line cap=round,fill=fillColor] (157.30, 56.87) circle (  1.16);

\path[draw=drawColor,line width= 0.4pt,line join=round,line cap=round,fill=fillColor] (157.64, 56.85) circle (  1.16);

\path[draw=drawColor,line width= 0.4pt,line join=round,line cap=round,fill=fillColor] (157.98, 56.77) circle (  1.16);

\path[draw=drawColor,line width= 0.4pt,line join=round,line cap=round,fill=fillColor] (158.31, 56.73) circle (  1.16);

\path[draw=drawColor,line width= 0.4pt,line join=round,line cap=round,fill=fillColor] (158.64, 56.70) circle (  1.16);

\path[draw=drawColor,line width= 0.4pt,line join=round,line cap=round,fill=fillColor] (158.97, 56.68) circle (  1.16);

\path[draw=drawColor,line width= 0.4pt,line join=round,line cap=round,fill=fillColor] (159.30, 56.59) circle (  1.16);

\path[draw=drawColor,line width= 0.4pt,line join=round,line cap=round,fill=fillColor] (159.63, 56.58) circle (  1.16);

\path[draw=drawColor,line width= 0.4pt,line join=round,line cap=round,fill=fillColor] (159.95, 56.55) circle (  1.16);

\path[draw=drawColor,line width= 0.4pt,line join=round,line cap=round,fill=fillColor] (160.27, 56.37) circle (  1.16);

\path[draw=drawColor,line width= 0.4pt,line join=round,line cap=round,fill=fillColor] (160.59, 56.36) circle (  1.16);

\path[draw=drawColor,line width= 0.4pt,line join=round,line cap=round,fill=fillColor] (160.91, 56.29) circle (  1.16);

\path[draw=drawColor,line width= 0.4pt,line join=round,line cap=round,fill=fillColor] (161.23, 56.26) circle (  1.16);

\path[draw=drawColor,line width= 0.4pt,line join=round,line cap=round,fill=fillColor] (161.54, 56.13) circle (  1.16);

\path[draw=drawColor,line width= 0.4pt,line join=round,line cap=round,fill=fillColor] (161.85, 56.06) circle (  1.16);

\path[draw=drawColor,line width= 0.4pt,line join=round,line cap=round,fill=fillColor] (162.16, 56.01) circle (  1.16);

\path[draw=drawColor,line width= 0.4pt,line join=round,line cap=round,fill=fillColor] (162.47, 55.94) circle (  1.16);

\path[draw=drawColor,line width= 0.4pt,line join=round,line cap=round,fill=fillColor] (162.78, 55.92) circle (  1.16);

\path[draw=drawColor,line width= 0.4pt,line join=round,line cap=round,fill=fillColor] (163.08, 55.89) circle (  1.16);

\path[draw=drawColor,line width= 0.4pt,line join=round,line cap=round,fill=fillColor] (163.39, 55.85) circle (  1.16);

\path[draw=drawColor,line width= 0.4pt,line join=round,line cap=round,fill=fillColor] (163.69, 55.76) circle (  1.16);

\path[draw=drawColor,line width= 0.4pt,line join=round,line cap=round,fill=fillColor] (163.99, 55.76) circle (  1.16);

\path[draw=drawColor,line width= 0.4pt,line join=round,line cap=round,fill=fillColor] (164.29, 55.73) circle (  1.16);

\path[draw=drawColor,line width= 0.4pt,line join=round,line cap=round,fill=fillColor] (164.58, 55.64) circle (  1.16);

\path[draw=drawColor,line width= 0.4pt,line join=round,line cap=round,fill=fillColor] (164.88, 55.60) circle (  1.16);

\path[draw=drawColor,line width= 0.4pt,line join=round,line cap=round,fill=fillColor] (165.17, 55.59) circle (  1.16);

\path[draw=drawColor,line width= 0.4pt,line join=round,line cap=round,fill=fillColor] (165.46, 55.56) circle (  1.16);

\path[draw=drawColor,line width= 0.4pt,line join=round,line cap=round,fill=fillColor] (165.75, 55.48) circle (  1.16);

\path[draw=drawColor,line width= 0.4pt,line join=round,line cap=round,fill=fillColor] (166.04, 55.43) circle (  1.16);

\path[draw=drawColor,line width= 0.4pt,line join=round,line cap=round,fill=fillColor] (166.33, 55.32) circle (  1.16);

\path[draw=drawColor,line width= 0.4pt,line join=round,line cap=round,fill=fillColor] (166.61, 55.29) circle (  1.16);

\path[draw=drawColor,line width= 0.4pt,line join=round,line cap=round,fill=fillColor] (166.90, 55.28) circle (  1.16);

\path[draw=drawColor,line width= 0.4pt,line join=round,line cap=round,fill=fillColor] (167.18, 55.27) circle (  1.16);

\path[draw=drawColor,line width= 0.4pt,line join=round,line cap=round,fill=fillColor] (167.46, 55.27) circle (  1.16);

\path[draw=drawColor,line width= 0.4pt,line join=round,line cap=round,fill=fillColor] (167.74, 55.24) circle (  1.16);

\path[draw=drawColor,line width= 0.4pt,line join=round,line cap=round,fill=fillColor] (168.02, 55.22) circle (  1.16);

\path[draw=drawColor,line width= 0.4pt,line join=round,line cap=round,fill=fillColor] (168.30, 55.20) circle (  1.16);

\path[draw=drawColor,line width= 0.4pt,line join=round,line cap=round,fill=fillColor] (168.57, 55.13) circle (  1.16);

\path[draw=drawColor,line width= 0.4pt,line join=round,line cap=round,fill=fillColor] (168.85, 55.11) circle (  1.16);

\path[draw=drawColor,line width= 0.4pt,line join=round,line cap=round,fill=fillColor] (169.12, 55.01) circle (  1.16);

\path[draw=drawColor,line width= 0.4pt,line join=round,line cap=round,fill=fillColor] (169.39, 55.00) circle (  1.16);

\path[draw=drawColor,line width= 0.4pt,line join=round,line cap=round,fill=fillColor] (169.66, 54.84) circle (  1.16);

\path[draw=drawColor,line width= 0.4pt,line join=round,line cap=round,fill=fillColor] (169.93, 54.84) circle (  1.16);

\path[draw=drawColor,line width= 0.4pt,line join=round,line cap=round,fill=fillColor] (170.20, 54.83) circle (  1.16);

\path[draw=drawColor,line width= 0.4pt,line join=round,line cap=round,fill=fillColor] (170.46, 54.76) circle (  1.16);

\path[draw=drawColor,line width= 0.4pt,line join=round,line cap=round,fill=fillColor] (170.73, 54.75) circle (  1.16);

\path[draw=drawColor,line width= 0.4pt,line join=round,line cap=round,fill=fillColor] (170.99, 54.70) circle (  1.16);

\path[draw=drawColor,line width= 0.4pt,line join=round,line cap=round,fill=fillColor] (171.25, 54.70) circle (  1.16);

\path[draw=drawColor,line width= 0.4pt,line join=round,line cap=round,fill=fillColor] (171.52, 54.66) circle (  1.16);

\path[draw=drawColor,line width= 0.4pt,line join=round,line cap=round,fill=fillColor] (171.78, 54.44) circle (  1.16);

\path[draw=drawColor,line width= 0.4pt,line join=round,line cap=round,fill=fillColor] (172.04, 54.42) circle (  1.16);

\path[draw=drawColor,line width= 0.4pt,line join=round,line cap=round,fill=fillColor] (172.29, 54.36) circle (  1.16);

\path[draw=drawColor,line width= 0.4pt,line join=round,line cap=round,fill=fillColor] (172.55, 54.35) circle (  1.16);

\path[draw=drawColor,line width= 0.4pt,line join=round,line cap=round,fill=fillColor] (172.81, 54.32) circle (  1.16);

\path[draw=drawColor,line width= 0.4pt,line join=round,line cap=round,fill=fillColor] (173.06, 54.24) circle (  1.16);

\path[draw=drawColor,line width= 0.4pt,line join=round,line cap=round,fill=fillColor] (173.31, 54.13) circle (  1.16);

\path[draw=drawColor,line width= 0.4pt,line join=round,line cap=round,fill=fillColor] (173.57, 54.07) circle (  1.16);

\path[draw=drawColor,line width= 0.4pt,line join=round,line cap=round,fill=fillColor] (173.82, 54.03) circle (  1.16);

\path[draw=drawColor,line width= 0.4pt,line join=round,line cap=round,fill=fillColor] (174.07, 54.02) circle (  1.16);

\path[draw=drawColor,line width= 0.4pt,line join=round,line cap=round,fill=fillColor] (174.32, 53.94) circle (  1.16);

\path[draw=drawColor,line width= 0.4pt,line join=round,line cap=round,fill=fillColor] (174.56, 53.91) circle (  1.16);

\path[draw=drawColor,line width= 0.4pt,line join=round,line cap=round,fill=fillColor] (174.81, 53.90) circle (  1.16);

\path[draw=drawColor,line width= 0.4pt,line join=round,line cap=round,fill=fillColor] (175.06, 53.75) circle (  1.16);

\path[draw=drawColor,line width= 0.4pt,line join=round,line cap=round,fill=fillColor] (175.30, 53.74) circle (  1.16);

\path[draw=drawColor,line width= 0.4pt,line join=round,line cap=round,fill=fillColor] (175.55, 53.66) circle (  1.16);

\path[draw=drawColor,line width= 0.4pt,line join=round,line cap=round,fill=fillColor] (175.79, 53.62) circle (  1.16);

\path[draw=drawColor,line width= 0.4pt,line join=round,line cap=round,fill=fillColor] (176.03, 53.56) circle (  1.16);

\path[draw=drawColor,line width= 0.4pt,line join=round,line cap=round,fill=fillColor] (176.27, 53.55) circle (  1.16);

\path[draw=drawColor,line width= 0.4pt,line join=round,line cap=round,fill=fillColor] (176.51, 53.54) circle (  1.16);

\path[draw=drawColor,line width= 0.4pt,line join=round,line cap=round,fill=fillColor] (176.75, 53.54) circle (  1.16);

\path[draw=drawColor,line width= 0.4pt,line join=round,line cap=round,fill=fillColor] (176.99, 53.53) circle (  1.16);

\path[draw=drawColor,line width= 0.4pt,line join=round,line cap=round,fill=fillColor] (177.22, 53.50) circle (  1.16);

\path[draw=drawColor,line width= 0.4pt,line join=round,line cap=round,fill=fillColor] (177.46, 53.41) circle (  1.16);

\path[draw=drawColor,line width= 0.4pt,line join=round,line cap=round,fill=fillColor] (177.70, 53.38) circle (  1.16);

\path[draw=drawColor,line width= 0.4pt,line join=round,line cap=round,fill=fillColor] (177.93, 53.36) circle (  1.16);

\path[draw=drawColor,line width= 0.4pt,line join=round,line cap=round,fill=fillColor] (178.16, 53.31) circle (  1.16);

\path[draw=drawColor,line width= 0.4pt,line join=round,line cap=round,fill=fillColor] (178.40, 53.21) circle (  1.16);

\path[draw=drawColor,line width= 0.4pt,line join=round,line cap=round,fill=fillColor] (178.63, 53.20) circle (  1.16);

\path[draw=drawColor,line width= 0.4pt,line join=round,line cap=round,fill=fillColor] (178.86, 53.20) circle (  1.16);

\path[draw=drawColor,line width= 0.4pt,line join=round,line cap=round,fill=fillColor] (179.09, 53.17) circle (  1.16);

\path[draw=drawColor,line width= 0.4pt,line join=round,line cap=round,fill=fillColor] (179.32, 53.16) circle (  1.16);

\path[draw=drawColor,line width= 0.4pt,line join=round,line cap=round,fill=fillColor] (179.55, 53.03) circle (  1.16);

\path[draw=drawColor,line width= 0.4pt,line join=round,line cap=round,fill=fillColor] (179.77, 52.98) circle (  1.16);

\path[draw=drawColor,line width= 0.4pt,line join=round,line cap=round,fill=fillColor] (180.00, 52.97) circle (  1.16);

\path[draw=drawColor,line width= 0.4pt,line join=round,line cap=round,fill=fillColor] (180.22, 52.97) circle (  1.16);

\path[draw=drawColor,line width= 0.4pt,line join=round,line cap=round,fill=fillColor] (180.45, 52.90) circle (  1.16);

\path[draw=drawColor,line width= 0.4pt,line join=round,line cap=round,fill=fillColor] (180.67, 52.80) circle (  1.16);

\path[draw=drawColor,line width= 0.4pt,line join=round,line cap=round,fill=fillColor] (180.90, 52.73) circle (  1.16);

\path[draw=drawColor,line width= 0.4pt,line join=round,line cap=round,fill=fillColor] (181.12, 52.53) circle (  1.16);

\path[draw=drawColor,line width= 0.4pt,line join=round,line cap=round,fill=fillColor] (181.34, 52.49) circle (  1.16);

\path[draw=drawColor,line width= 0.4pt,line join=round,line cap=round,fill=fillColor] (181.56, 52.46) circle (  1.16);

\path[draw=drawColor,line width= 0.4pt,line join=round,line cap=round,fill=fillColor] (181.78, 52.30) circle (  1.16);

\path[draw=drawColor,line width= 0.4pt,line join=round,line cap=round,fill=fillColor] (182.00, 52.05) circle (  1.16);

\path[draw=drawColor,line width= 0.4pt,line join=round,line cap=round,fill=fillColor] (182.22, 52.02) circle (  1.16);

\path[draw=drawColor,line width= 0.4pt,line join=round,line cap=round,fill=fillColor] (182.44, 51.97) circle (  1.16);

\path[draw=drawColor,line width= 0.4pt,line join=round,line cap=round,fill=fillColor] (182.65, 51.90) circle (  1.16);

\path[draw=drawColor,line width= 0.4pt,line join=round,line cap=round,fill=fillColor] (182.87, 51.84) circle (  1.16);

\path[draw=drawColor,line width= 0.4pt,line join=round,line cap=round,fill=fillColor] (183.09, 51.68) circle (  1.16);

\path[draw=drawColor,line width= 0.4pt,line join=round,line cap=round,fill=fillColor] (183.30, 51.55) circle (  1.16);

\path[draw=drawColor,line width= 0.4pt,line join=round,line cap=round,fill=fillColor] (183.51, 51.38) circle (  1.16);

\path[draw=drawColor,line width= 0.4pt,line join=round,line cap=round,fill=fillColor] (183.73, 51.26) circle (  1.16);

\path[draw=drawColor,line width= 0.4pt,line join=round,line cap=round,fill=fillColor] (183.94, 51.11) circle (  1.16);

\path[draw=drawColor,line width= 0.4pt,line join=round,line cap=round,fill=fillColor] (184.15, 50.91) circle (  1.16);

\path[draw=drawColor,line width= 0.4pt,line join=round,line cap=round,fill=fillColor] (184.36, 50.87) circle (  1.16);

\path[draw=drawColor,line width= 0.4pt,line join=round,line cap=round,fill=fillColor] (184.57, 50.84) circle (  1.16);

\path[draw=drawColor,line width= 0.4pt,line join=round,line cap=round,fill=fillColor] (184.78, 50.63) circle (  1.16);

\path[draw=drawColor,line width= 0.4pt,line join=round,line cap=round,fill=fillColor] (184.99, 50.32) circle (  1.16);

\path[draw=drawColor,line width= 0.4pt,line join=round,line cap=round,fill=fillColor] (185.20, 50.17) circle (  1.16);

\path[draw=drawColor,line width= 0.4pt,line join=round,line cap=round,fill=fillColor] (185.41, 50.02) circle (  1.16);

\path[draw=drawColor,line width= 0.4pt,line join=round,line cap=round,fill=fillColor] (185.61, 49.86) circle (  1.16);

\path[draw=drawColor,line width= 0.4pt,line join=round,line cap=round,fill=fillColor] (185.82, 49.85) circle (  1.16);

\path[draw=drawColor,line width= 0.4pt,line join=round,line cap=round,fill=fillColor] (186.03, 49.81) circle (  1.16);

\path[draw=drawColor,line width= 0.4pt,line join=round,line cap=round,fill=fillColor] (186.23, 49.57) circle (  1.16);

\path[draw=drawColor,line width= 0.4pt,line join=round,line cap=round,fill=fillColor] (186.44, 49.29) circle (  1.16);

\path[draw=drawColor,line width= 0.4pt,line join=round,line cap=round,fill=fillColor] (186.64, 49.07) circle (  1.16);

\path[draw=drawColor,line width= 0.4pt,line join=round,line cap=round,fill=fillColor] (186.84, 48.87) circle (  1.16);

\path[draw=drawColor,line width= 0.4pt,line join=round,line cap=round,fill=fillColor] (187.05, 48.83) circle (  1.16);

\path[draw=drawColor,line width= 0.4pt,line join=round,line cap=round,fill=fillColor] (187.25, 48.24) circle (  1.16);

\path[draw=drawColor,line width= 0.4pt,line join=round,line cap=round,fill=fillColor] (187.45, 47.88) circle (  1.16);

\path[draw=drawColor,line width= 0.4pt,line join=round,line cap=round,fill=fillColor] (187.65, 46.53) circle (  1.16);

\path[draw=drawColor,line width= 0.4pt,line join=round,line cap=round,fill=fillColor] (187.85, 44.78) circle (  1.16);

\path[draw=drawColor,line width= 0.4pt,line join=round,line cap=round,fill=fillColor] (188.05, 44.78) circle (  1.16);

\path[draw=drawColor,line width= 0.4pt,line join=round,line cap=round,fill=fillColor] (188.25, 44.78) circle (  1.16);

\path[draw=drawColor,line width= 0.4pt,line join=round,line cap=round,fill=fillColor] (188.45, 44.78) circle (  1.16);

\path[draw=drawColor,line width= 0.4pt,line join=round,line cap=round,fill=fillColor] (188.64, 44.78) circle (  1.16);

\path[draw=drawColor,line width= 0.4pt,line join=round,line cap=round,fill=fillColor] (188.84, 44.78) circle (  1.16);

\path[draw=drawColor,line width= 0.4pt,line join=round,line cap=round,fill=fillColor] (189.04, 44.78) circle (  1.16);

\path[draw=drawColor,line width= 0.4pt,line join=round,line cap=round,fill=fillColor] (189.23, 44.78) circle (  1.16);

\path[draw=drawColor,line width= 0.4pt,line join=round,line cap=round,fill=fillColor] (189.43, 44.78) circle (  1.16);

\path[draw=drawColor,line width= 0.4pt,line join=round,line cap=round,fill=fillColor] (189.62, 44.78) circle (  1.16);

\path[draw=drawColor,line width= 0.4pt,line join=round,line cap=round,fill=fillColor] (189.82, 44.78) circle (  1.16);

\path[draw=drawColor,line width= 0.4pt,line join=round,line cap=round,fill=fillColor] (190.01, 44.78) circle (  1.16);

\path[draw=drawColor,line width= 0.4pt,line join=round,line cap=round,fill=fillColor] (190.20, 44.78) circle (  1.16);

\path[draw=drawColor,line width= 0.4pt,line join=round,line cap=round,fill=fillColor] (190.39, 44.78) circle (  1.16);

\path[draw=drawColor,line width= 0.4pt,line join=round,line cap=round,fill=fillColor] (190.59, 44.78) circle (  1.16);

\path[draw=drawColor,line width= 0.4pt,line join=round,line cap=round,fill=fillColor] (190.78, 44.78) circle (  1.16);

\path[draw=drawColor,line width= 0.4pt,line join=round,line cap=round,fill=fillColor] (190.97, 44.78) circle (  1.16);

\path[draw=drawColor,line width= 0.4pt,line join=round,line cap=round,fill=fillColor] (191.16, 44.78) circle (  1.16);

\path[draw=drawColor,line width= 0.4pt,line join=round,line cap=round,fill=fillColor] (191.35, 44.78) circle (  1.16);

\path[draw=drawColor,line width= 0.4pt,line join=round,line cap=round,fill=fillColor] (191.54, 44.78) circle (  1.16);

\path[draw=drawColor,line width= 0.4pt,line join=round,line cap=round,fill=fillColor] (191.73, 44.78) circle (  1.16);

\path[draw=drawColor,line width= 0.4pt,line join=round,line cap=round,fill=fillColor] (191.91, 44.78) circle (  1.16);

\path[draw=drawColor,line width= 0.4pt,line join=round,line cap=round,fill=fillColor] (192.10, 44.78) circle (  1.16);

\path[draw=drawColor,line width= 0.4pt,line join=round,line cap=round,fill=fillColor] (192.29, 44.78) circle (  1.16);

\path[draw=drawColor,line width= 0.4pt,line join=round,line cap=round,fill=fillColor] (192.47, 44.78) circle (  1.16);

\path[draw=drawColor,line width= 0.4pt,line join=round,line cap=round,fill=fillColor] (192.66, 44.78) circle (  1.16);

\path[draw=drawColor,line width= 0.4pt,line join=round,line cap=round,fill=fillColor] (192.85, 44.78) circle (  1.16);

\path[draw=drawColor,line width= 0.4pt,line join=round,line cap=round,fill=fillColor] (193.03, 44.78) circle (  1.16);

\path[draw=drawColor,line width= 0.4pt,line join=round,line cap=round,fill=fillColor] (193.22, 44.78) circle (  1.16);

\path[draw=drawColor,line width= 0.4pt,line join=round,line cap=round,fill=fillColor] (193.40, 44.78) circle (  1.16);

\path[draw=drawColor,line width= 0.4pt,line join=round,line cap=round,fill=fillColor] (193.58, 44.78) circle (  1.16);

\path[draw=drawColor,line width= 0.4pt,line join=round,line cap=round,fill=fillColor] (193.77, 44.78) circle (  1.16);

\path[draw=drawColor,line width= 0.4pt,line join=round,line cap=round,fill=fillColor] (193.95, 44.78) circle (  1.16);

\path[draw=drawColor,line width= 0.4pt,line join=round,line cap=round,fill=fillColor] (194.13, 44.78) circle (  1.16);

\path[draw=drawColor,line width= 0.4pt,line join=round,line cap=round,fill=fillColor] (194.31, 44.78) circle (  1.16);

\path[draw=drawColor,line width= 0.4pt,line join=round,line cap=round,fill=fillColor] (194.49, 44.78) circle (  1.16);

\path[draw=drawColor,line width= 0.4pt,line join=round,line cap=round,fill=fillColor] (194.67, 44.78) circle (  1.16);

\path[draw=drawColor,line width= 0.4pt,line join=round,line cap=round,fill=fillColor] (194.85, 44.78) circle (  1.16);

\path[draw=drawColor,line width= 0.4pt,line join=round,line cap=round,fill=fillColor] (195.03, 44.78) circle (  1.16);

\path[draw=drawColor,line width= 0.4pt,line join=round,line cap=round,fill=fillColor] (195.21, 44.78) circle (  1.16);

\path[draw=drawColor,line width= 0.4pt,line join=round,line cap=round,fill=fillColor] (195.39, 44.78) circle (  1.16);

\path[draw=drawColor,line width= 0.4pt,line join=round,line cap=round,fill=fillColor] (195.57, 44.78) circle (  1.16);

\path[draw=drawColor,line width= 0.4pt,line join=round,line cap=round,fill=fillColor] (195.75, 44.78) circle (  1.16);

\path[draw=drawColor,line width= 0.4pt,line join=round,line cap=round,fill=fillColor] (195.92, 44.78) circle (  1.16);

\path[draw=drawColor,line width= 0.4pt,line join=round,line cap=round,fill=fillColor] (196.10, 44.78) circle (  1.16);

\path[draw=drawColor,line width= 0.4pt,line join=round,line cap=round,fill=fillColor] (196.28, 44.78) circle (  1.16);

\path[draw=drawColor,line width= 0.4pt,line join=round,line cap=round,fill=fillColor] (196.45, 44.78) circle (  1.16);

\path[draw=drawColor,line width= 0.4pt,line join=round,line cap=round,fill=fillColor] (196.63, 44.78) circle (  1.16);

\path[draw=drawColor,line width= 0.4pt,line join=round,line cap=round,fill=fillColor] (196.80, 44.78) circle (  1.16);

\path[draw=drawColor,line width= 0.4pt,line join=round,line cap=round,fill=fillColor] (196.98, 44.78) circle (  1.16);

\path[draw=drawColor,line width= 0.4pt,line join=round,line cap=round,fill=fillColor] (197.15, 44.78) circle (  1.16);

\path[draw=drawColor,line width= 0.4pt,line join=round,line cap=round,fill=fillColor] (197.33, 44.78) circle (  1.16);

\path[draw=drawColor,line width= 0.4pt,line join=round,line cap=round,fill=fillColor] (197.50, 44.78) circle (  1.16);

\path[draw=drawColor,line width= 0.4pt,line join=round,line cap=round,fill=fillColor] (197.67, 44.78) circle (  1.16);

\path[draw=drawColor,line width= 0.4pt,line join=round,line cap=round,fill=fillColor] (197.85, 44.78) circle (  1.16);

\path[draw=drawColor,line width= 0.4pt,line join=round,line cap=round,fill=fillColor] (198.02, 44.78) circle (  1.16);

\path[draw=drawColor,line width= 0.4pt,line join=round,line cap=round,fill=fillColor] (198.19, 44.78) circle (  1.16);

\path[draw=drawColor,line width= 0.4pt,line join=round,line cap=round,fill=fillColor] (198.36, 44.78) circle (  1.16);

\path[draw=drawColor,line width= 0.4pt,line join=round,line cap=round,fill=fillColor] (198.53, 44.78) circle (  1.16);

\path[draw=drawColor,line width= 0.4pt,line join=round,line cap=round,fill=fillColor] (198.70, 44.78) circle (  1.16);

\path[draw=drawColor,line width= 0.4pt,line join=round,line cap=round,fill=fillColor] (198.87, 44.78) circle (  1.16);

\path[draw=drawColor,line width= 0.4pt,line join=round,line cap=round,fill=fillColor] (199.04, 44.78) circle (  1.16);

\path[draw=drawColor,line width= 0.4pt,line join=round,line cap=round,fill=fillColor] (199.21, 44.78) circle (  1.16);

\path[draw=drawColor,line width= 0.4pt,line join=round,line cap=round,fill=fillColor] (199.38, 44.78) circle (  1.16);

\path[draw=drawColor,line width= 0.4pt,line join=round,line cap=round,fill=fillColor] (199.55, 44.78) circle (  1.16);

\path[draw=drawColor,line width= 0.4pt,line join=round,line cap=round,fill=fillColor] (199.72, 44.78) circle (  1.16);

\path[draw=drawColor,line width= 0.4pt,line join=round,line cap=round,fill=fillColor] (199.88, 44.78) circle (  1.16);

\path[draw=drawColor,line width= 0.4pt,line join=round,line cap=round,fill=fillColor] (200.05, 44.78) circle (  1.16);

\path[draw=drawColor,line width= 0.4pt,line join=round,line cap=round,fill=fillColor] (200.22, 44.78) circle (  1.16);

\path[draw=drawColor,line width= 0.4pt,line join=round,line cap=round,fill=fillColor] (200.38, 44.78) circle (  1.16);

\path[draw=drawColor,line width= 0.4pt,line join=round,line cap=round,fill=fillColor] (200.55, 44.78) circle (  1.16);

\path[draw=drawColor,line width= 0.4pt,line join=round,line cap=round,fill=fillColor] (200.72, 44.78) circle (  1.16);

\path[draw=drawColor,line width= 0.4pt,line join=round,line cap=round,fill=fillColor] (200.88, 44.78) circle (  1.16);

\path[draw=drawColor,line width= 0.4pt,line join=round,line cap=round,fill=fillColor] (201.05, 44.78) circle (  1.16);

\path[draw=drawColor,line width= 0.4pt,line join=round,line cap=round,fill=fillColor] (201.21, 44.78) circle (  1.16);

\path[draw=drawColor,line width= 0.4pt,line join=round,line cap=round,fill=fillColor] (201.37, 44.78) circle (  1.16);

\path[draw=drawColor,line width= 0.4pt,line join=round,line cap=round,fill=fillColor] (201.54, 44.78) circle (  1.16);

\path[draw=drawColor,line width= 0.4pt,line join=round,line cap=round,fill=fillColor] (201.70, 44.78) circle (  1.16);

\path[draw=drawColor,line width= 0.4pt,line join=round,line cap=round,fill=fillColor] (201.86, 44.78) circle (  1.16);

\path[draw=drawColor,line width= 0.4pt,line join=round,line cap=round,fill=fillColor] (202.03, 44.78) circle (  1.16);

\path[draw=drawColor,line width= 0.4pt,line join=round,line cap=round,fill=fillColor] (202.19, 44.78) circle (  1.16);

\path[draw=drawColor,line width= 0.4pt,line join=round,line cap=round,fill=fillColor] (202.35, 44.78) circle (  1.16);

\path[draw=drawColor,line width= 0.4pt,line join=round,line cap=round,fill=fillColor] (202.51, 44.78) circle (  1.16);

\path[draw=drawColor,line width= 0.4pt,line join=round,line cap=round,fill=fillColor] (202.67, 44.78) circle (  1.16);

\path[draw=drawColor,line width= 0.4pt,line join=round,line cap=round,fill=fillColor] (202.83, 44.78) circle (  1.16);

\path[draw=drawColor,line width= 0.4pt,line join=round,line cap=round,fill=fillColor] (202.99, 44.78) circle (  1.16);

\path[draw=drawColor,line width= 0.4pt,line join=round,line cap=round,fill=fillColor] (203.16, 44.78) circle (  1.16);

\path[draw=drawColor,line width= 0.4pt,line join=round,line cap=round,fill=fillColor] (203.31, 44.78) circle (  1.16);

\path[draw=drawColor,line width= 0.4pt,line join=round,line cap=round,fill=fillColor] (203.47, 44.78) circle (  1.16);

\path[draw=drawColor,line width= 0.4pt,line join=round,line cap=round,fill=fillColor] (203.63, 44.78) circle (  1.16);

\path[draw=drawColor,line width= 0.4pt,line join=round,line cap=round,fill=fillColor] (203.79, 44.78) circle (  1.16);

\path[draw=drawColor,line width= 0.4pt,line join=round,line cap=round,fill=fillColor] (203.95, 44.78) circle (  1.16);

\path[draw=drawColor,line width= 0.4pt,line join=round,line cap=round,fill=fillColor] (204.11, 44.78) circle (  1.16);

\path[draw=drawColor,line width= 0.4pt,line join=round,line cap=round,fill=fillColor] (204.27, 44.78) circle (  1.16);

\path[draw=drawColor,line width= 0.4pt,line join=round,line cap=round,fill=fillColor] (204.42, 44.78) circle (  1.16);

\path[draw=drawColor,line width= 0.4pt,line join=round,line cap=round,fill=fillColor] (204.58, 44.78) circle (  1.16);

\path[draw=drawColor,line width= 0.4pt,line join=round,line cap=round,fill=fillColor] (204.74, 44.78) circle (  1.16);

\path[draw=drawColor,line width= 0.4pt,line join=round,line cap=round,fill=fillColor] (204.89, 44.78) circle (  1.16);

\path[draw=drawColor,line width= 0.4pt,line join=round,line cap=round,fill=fillColor] (205.05, 44.78) circle (  1.16);

\path[draw=drawColor,line width= 0.4pt,line join=round,line cap=round,fill=fillColor] (205.21, 44.78) circle (  1.16);

\path[draw=drawColor,line width= 0.4pt,line join=round,line cap=round,fill=fillColor] (205.36, 44.78) circle (  1.16);

\path[draw=drawColor,line width= 0.4pt,line join=round,line cap=round,fill=fillColor] (205.52, 44.78) circle (  1.16);

\path[draw=drawColor,line width= 0.4pt,line join=round,line cap=round,fill=fillColor] (205.67, 44.78) circle (  1.16);

\path[draw=drawColor,line width= 0.4pt,line join=round,line cap=round,fill=fillColor] (205.83, 44.78) circle (  1.16);

\path[draw=drawColor,line width= 0.4pt,line join=round,line cap=round,fill=fillColor] (205.98, 44.78) circle (  1.16);

\path[draw=drawColor,line width= 0.4pt,line join=round,line cap=round,fill=fillColor] (206.13, 44.78) circle (  1.16);

\path[draw=drawColor,line width= 0.4pt,line join=round,line cap=round,fill=fillColor] (206.29, 44.78) circle (  1.16);

\path[draw=drawColor,line width= 0.4pt,line join=round,line cap=round,fill=fillColor] (206.44, 44.78) circle (  1.16);

\path[draw=drawColor,line width= 0.4pt,line join=round,line cap=round,fill=fillColor] (206.59, 44.78) circle (  1.16);

\path[draw=drawColor,line width= 0.4pt,line join=round,line cap=round,fill=fillColor] (206.74, 44.78) circle (  1.16);

\path[draw=drawColor,line width= 0.4pt,line join=round,line cap=round,fill=fillColor] (206.90, 44.78) circle (  1.16);

\path[draw=drawColor,line width= 0.4pt,line join=round,line cap=round,fill=fillColor] (207.05, 44.78) circle (  1.16);

\path[draw=drawColor,line width= 0.4pt,line join=round,line cap=round,fill=fillColor] (207.20, 44.78) circle (  1.16);

\path[draw=drawColor,line width= 0.4pt,line join=round,line cap=round,fill=fillColor] (207.35, 44.78) circle (  1.16);

\path[draw=drawColor,line width= 0.4pt,line join=round,line cap=round,fill=fillColor] (207.50, 44.78) circle (  1.16);

\path[draw=drawColor,line width= 0.4pt,line join=round,line cap=round,fill=fillColor] (207.65, 44.78) circle (  1.16);

\path[draw=drawColor,line width= 0.4pt,line join=round,line cap=round,fill=fillColor] (207.80, 44.78) circle (  1.16);

\path[draw=drawColor,line width= 0.4pt,line join=round,line cap=round,fill=fillColor] (207.95, 44.78) circle (  1.16);

\path[draw=drawColor,line width= 0.4pt,line join=round,line cap=round,fill=fillColor] (208.10, 44.78) circle (  1.16);

\path[draw=drawColor,line width= 0.4pt,line join=round,line cap=round,fill=fillColor] (208.25, 44.78) circle (  1.16);

\path[draw=drawColor,line width= 0.4pt,line join=round,line cap=round,fill=fillColor] (208.40, 44.78) circle (  1.16);

\path[draw=drawColor,line width= 0.4pt,line join=round,line cap=round,fill=fillColor] (208.55, 44.78) circle (  1.16);

\path[draw=drawColor,line width= 0.4pt,line join=round,line cap=round,fill=fillColor] (208.70, 44.78) circle (  1.16);

\path[draw=drawColor,line width= 0.4pt,line join=round,line cap=round,fill=fillColor] (208.85, 44.78) circle (  1.16);

\path[draw=drawColor,line width= 0.4pt,line join=round,line cap=round,fill=fillColor] (209.00, 44.78) circle (  1.16);

\path[draw=drawColor,line width= 0.4pt,line join=round,line cap=round,fill=fillColor] (209.14, 44.78) circle (  1.16);

\path[draw=drawColor,line width= 0.4pt,line join=round,line cap=round,fill=fillColor] (209.29, 44.78) circle (  1.16);

\path[draw=drawColor,line width= 0.4pt,line join=round,line cap=round,fill=fillColor] (209.44, 44.78) circle (  1.16);

\path[draw=drawColor,line width= 0.4pt,line join=round,line cap=round,fill=fillColor] (209.59, 44.78) circle (  1.16);

\path[draw=drawColor,line width= 0.4pt,line join=round,line cap=round,fill=fillColor] (209.73, 44.78) circle (  1.16);

\path[draw=drawColor,line width= 0.4pt,line join=round,line cap=round,fill=fillColor] (209.88, 44.78) circle (  1.16);

\path[draw=drawColor,line width= 0.4pt,line join=round,line cap=round,fill=fillColor] (210.02, 44.78) circle (  1.16);

\path[draw=drawColor,line width= 0.4pt,line join=round,line cap=round,fill=fillColor] (210.17, 44.78) circle (  1.16);

\path[draw=drawColor,line width= 0.4pt,line join=round,line cap=round,fill=fillColor] (210.32, 44.78) circle (  1.16);

\path[draw=drawColor,line width= 0.4pt,line join=round,line cap=round,fill=fillColor] (210.46, 44.78) circle (  1.16);

\path[draw=drawColor,line width= 0.4pt,line join=round,line cap=round,fill=fillColor] (210.61, 44.78) circle (  1.16);

\path[draw=drawColor,line width= 0.4pt,line join=round,line cap=round,fill=fillColor] (210.75, 44.78) circle (  1.16);

\path[draw=drawColor,line width= 0.4pt,line join=round,line cap=round,fill=fillColor] (210.89, 44.78) circle (  1.16);

\path[draw=drawColor,line width= 0.4pt,line join=round,line cap=round,fill=fillColor] (211.04, 44.78) circle (  1.16);

\path[draw=drawColor,line width= 0.4pt,line join=round,line cap=round,fill=fillColor] (211.18, 44.78) circle (  1.16);

\path[draw=drawColor,line width= 0.4pt,line join=round,line cap=round,fill=fillColor] (211.33, 44.78) circle (  1.16);

\path[draw=drawColor,line width= 0.4pt,line join=round,line cap=round,fill=fillColor] (211.47, 44.78) circle (  1.16);

\path[draw=drawColor,line width= 0.4pt,line join=round,line cap=round,fill=fillColor] (211.61, 44.78) circle (  1.16);

\path[draw=drawColor,line width= 0.4pt,line join=round,line cap=round,fill=fillColor] (211.76, 44.78) circle (  1.16);

\path[draw=drawColor,line width= 0.4pt,line join=round,line cap=round,fill=fillColor] (211.90, 44.78) circle (  1.16);

\path[draw=drawColor,line width= 0.4pt,line join=round,line cap=round,fill=fillColor] (212.04, 44.78) circle (  1.16);

\path[draw=drawColor,line width= 0.4pt,line join=round,line cap=round,fill=fillColor] (212.18, 44.78) circle (  1.16);

\path[draw=drawColor,line width= 0.4pt,line join=round,line cap=round,fill=fillColor] (212.32, 44.78) circle (  1.16);

\path[draw=drawColor,line width= 0.4pt,line join=round,line cap=round,fill=fillColor] (212.47, 44.78) circle (  1.16);

\path[draw=drawColor,line width= 0.4pt,line join=round,line cap=round,fill=fillColor] (212.61, 44.78) circle (  1.16);

\path[draw=drawColor,line width= 0.4pt,line join=round,line cap=round,fill=fillColor] (212.75, 44.78) circle (  1.16);

\path[draw=drawColor,line width= 0.4pt,line join=round,line cap=round,fill=fillColor] (212.89, 44.78) circle (  1.16);

\path[draw=drawColor,line width= 0.4pt,line join=round,line cap=round,fill=fillColor] (213.03, 44.78) circle (  1.16);

\path[draw=drawColor,line width= 0.4pt,line join=round,line cap=round,fill=fillColor] (213.17, 44.78) circle (  1.16);

\path[draw=drawColor,line width= 0.4pt,line join=round,line cap=round,fill=fillColor] (213.31, 44.78) circle (  1.16);

\path[draw=drawColor,line width= 0.4pt,line join=round,line cap=round,fill=fillColor] (213.45, 44.78) circle (  1.16);

\path[draw=drawColor,line width= 0.4pt,line join=round,line cap=round,fill=fillColor] (213.59, 44.78) circle (  1.16);

\path[draw=drawColor,line width= 0.4pt,line join=round,line cap=round,fill=fillColor] (213.73, 44.78) circle (  1.16);

\path[draw=drawColor,line width= 0.4pt,line join=round,line cap=round,fill=fillColor] (213.87, 44.78) circle (  1.16);

\path[draw=drawColor,line width= 0.4pt,line join=round,line cap=round,fill=fillColor] (214.00, 44.78) circle (  1.16);

\path[draw=drawColor,line width= 0.4pt,line join=round,line cap=round,fill=fillColor] (214.14, 44.78) circle (  1.16);

\path[draw=drawColor,line width= 0.4pt,line join=round,line cap=round,fill=fillColor] (214.28, 44.78) circle (  1.16);

\path[draw=drawColor,line width= 0.4pt,line join=round,line cap=round,fill=fillColor] (214.42, 44.78) circle (  1.16);

\path[draw=drawColor,line width= 0.4pt,line join=round,line cap=round,fill=fillColor] (214.56, 44.78) circle (  1.16);

\path[draw=drawColor,line width= 0.4pt,line join=round,line cap=round,fill=fillColor] (214.69, 44.78) circle (  1.16);

\path[draw=drawColor,line width= 0.4pt,line join=round,line cap=round,fill=fillColor] (214.83, 44.78) circle (  1.16);

\path[draw=drawColor,line width= 0.4pt,line join=round,line cap=round,fill=fillColor] (214.97, 44.78) circle (  1.16);

\path[draw=drawColor,line width= 0.4pt,line join=round,line cap=round,fill=fillColor] (215.11, 44.78) circle (  1.16);

\path[draw=drawColor,line width= 0.4pt,line join=round,line cap=round,fill=fillColor] (215.24, 44.78) circle (  1.16);

\path[draw=drawColor,line width= 0.4pt,line join=round,line cap=round,fill=fillColor] (215.38, 44.78) circle (  1.16);

\path[draw=drawColor,line width= 0.4pt,line join=round,line cap=round,fill=fillColor] (215.51, 44.78) circle (  1.16);

\path[draw=drawColor,line width= 0.4pt,line join=round,line cap=round,fill=fillColor] (215.65, 44.78) circle (  1.16);

\path[draw=drawColor,line width= 0.4pt,line join=round,line cap=round,fill=fillColor] (215.79, 44.78) circle (  1.16);

\path[draw=drawColor,line width= 0.4pt,line join=round,line cap=round,fill=fillColor] (215.92, 44.78) circle (  1.16);

\path[draw=drawColor,line width= 0.4pt,line join=round,line cap=round,fill=fillColor] (216.06, 44.78) circle (  1.16);

\path[draw=drawColor,line width= 0.4pt,line join=round,line cap=round,fill=fillColor] (216.19, 44.78) circle (  1.16);

\path[draw=drawColor,line width= 0.4pt,line join=round,line cap=round,fill=fillColor] (216.33, 44.78) circle (  1.16);

\path[draw=drawColor,line width= 0.4pt,line join=round,line cap=round,fill=fillColor] (216.46, 44.78) circle (  1.16);

\path[draw=drawColor,line width= 0.4pt,line join=round,line cap=round,fill=fillColor] (216.59, 44.78) circle (  1.16);

\path[draw=drawColor,line width= 0.4pt,line join=round,line cap=round,fill=fillColor] (216.73, 44.78) circle (  1.16);

\path[draw=drawColor,line width= 0.4pt,line join=round,line cap=round,fill=fillColor] (216.86, 44.78) circle (  1.16);

\path[draw=drawColor,line width= 0.4pt,line join=round,line cap=round,fill=fillColor] (217.00, 44.78) circle (  1.16);

\path[draw=drawColor,line width= 0.4pt,line join=round,line cap=round,fill=fillColor] (217.13, 44.78) circle (  1.16);

\path[draw=drawColor,line width= 0.4pt,line join=round,line cap=round,fill=fillColor] (217.26, 44.78) circle (  1.16);

\path[draw=drawColor,line width= 0.4pt,line join=round,line cap=round,fill=fillColor] (217.40, 44.78) circle (  1.16);

\path[draw=drawColor,line width= 0.4pt,line join=round,line cap=round,fill=fillColor] (217.53, 44.78) circle (  1.16);

\path[draw=drawColor,line width= 0.4pt,line join=round,line cap=round,fill=fillColor] (217.66, 44.78) circle (  1.16);

\path[draw=drawColor,line width= 0.4pt,line join=round,line cap=round,fill=fillColor] (217.79, 44.78) circle (  1.16);

\path[draw=drawColor,line width= 0.4pt,line join=round,line cap=round,fill=fillColor] (217.92, 44.78) circle (  1.16);

\path[draw=drawColor,line width= 0.4pt,line join=round,line cap=round,fill=fillColor] (218.06, 44.78) circle (  1.16);

\path[draw=drawColor,line width= 0.4pt,line join=round,line cap=round,fill=fillColor] (218.19, 44.78) circle (  1.16);

\path[draw=drawColor,line width= 0.4pt,line join=round,line cap=round,fill=fillColor] (218.32, 44.78) circle (  1.16);

\path[draw=drawColor,line width= 0.4pt,line join=round,line cap=round,fill=fillColor] (218.45, 44.78) circle (  1.16);

\path[draw=drawColor,line width= 0.4pt,line join=round,line cap=round,fill=fillColor] (218.58, 44.78) circle (  1.16);

\path[draw=drawColor,line width= 0.4pt,line join=round,line cap=round,fill=fillColor] (218.71, 44.78) circle (  1.16);

\path[draw=drawColor,line width= 0.4pt,line join=round,line cap=round,fill=fillColor] (218.84, 44.78) circle (  1.16);

\path[draw=drawColor,line width= 0.4pt,line join=round,line cap=round,fill=fillColor] (218.97, 44.78) circle (  1.16);

\path[draw=drawColor,line width= 0.4pt,line join=round,line cap=round,fill=fillColor] (219.10, 44.78) circle (  1.16);
\definecolor[named]{drawColor}{rgb}{0.60,0.31,0.64}
\definecolor[named]{fillColor}{rgb}{0.60,0.31,0.64}

\path[draw=drawColor,line width= 0.4pt,line join=round,line cap=round,fill=fillColor] ( 78.97, 84.15) circle (  1.16);

\path[draw=drawColor,line width= 0.4pt,line join=round,line cap=round,fill=fillColor] ( 84.61, 70.47) circle (  1.16);

\path[draw=drawColor,line width= 0.4pt,line join=round,line cap=round,fill=fillColor] ( 88.57, 70.31) circle (  1.16);

\path[draw=drawColor,line width= 0.4pt,line join=round,line cap=round,fill=fillColor] ( 91.71, 67.99) circle (  1.16);

\path[draw=drawColor,line width= 0.4pt,line join=round,line cap=round,fill=fillColor] ( 94.37, 67.65) circle (  1.16);

\path[draw=drawColor,line width= 0.4pt,line join=round,line cap=round,fill=fillColor] ( 96.70, 67.05) circle (  1.16);

\path[draw=drawColor,line width= 0.4pt,line join=round,line cap=round,fill=fillColor] ( 98.78, 65.83) circle (  1.16);

\path[draw=drawColor,line width= 0.4pt,line join=round,line cap=round,fill=fillColor] (100.67, 65.57) circle (  1.16);

\path[draw=drawColor,line width= 0.4pt,line join=round,line cap=round,fill=fillColor] (102.40, 64.71) circle (  1.16);

\path[draw=drawColor,line width= 0.4pt,line join=round,line cap=round,fill=fillColor] (104.02, 64.54) circle (  1.16);

\path[draw=drawColor,line width= 0.4pt,line join=round,line cap=round,fill=fillColor] (105.53, 63.97) circle (  1.16);

\path[draw=drawColor,line width= 0.4pt,line join=round,line cap=round,fill=fillColor] (106.95, 63.94) circle (  1.16);

\path[draw=drawColor,line width= 0.4pt,line join=round,line cap=round,fill=fillColor] (108.29, 63.92) circle (  1.16);

\path[draw=drawColor,line width= 0.4pt,line join=round,line cap=round,fill=fillColor] (109.56, 63.88) circle (  1.16);

\path[draw=drawColor,line width= 0.4pt,line join=round,line cap=round,fill=fillColor] (110.78, 63.86) circle (  1.16);

\path[draw=drawColor,line width= 0.4pt,line join=round,line cap=round,fill=fillColor] (111.94, 63.65) circle (  1.16);

\path[draw=drawColor,line width= 0.4pt,line join=round,line cap=round,fill=fillColor] (113.06, 63.56) circle (  1.16);

\path[draw=drawColor,line width= 0.4pt,line join=round,line cap=round,fill=fillColor] (114.13, 63.55) circle (  1.16);

\path[draw=drawColor,line width= 0.4pt,line join=round,line cap=round,fill=fillColor] (115.17, 63.17) circle (  1.16);

\path[draw=drawColor,line width= 0.4pt,line join=round,line cap=round,fill=fillColor] (116.17, 62.99) circle (  1.16);

\path[draw=drawColor,line width= 0.4pt,line join=round,line cap=round,fill=fillColor] (117.13, 62.95) circle (  1.16);

\path[draw=drawColor,line width= 0.4pt,line join=round,line cap=round,fill=fillColor] (118.07, 62.90) circle (  1.16);

\path[draw=drawColor,line width= 0.4pt,line join=round,line cap=round,fill=fillColor] (118.97, 62.88) circle (  1.16);

\path[draw=drawColor,line width= 0.4pt,line join=round,line cap=round,fill=fillColor] (119.86, 62.39) circle (  1.16);

\path[draw=drawColor,line width= 0.4pt,line join=round,line cap=round,fill=fillColor] (120.71, 62.18) circle (  1.16);

\path[draw=drawColor,line width= 0.4pt,line join=round,line cap=round,fill=fillColor] (121.55, 62.08) circle (  1.16);

\path[draw=drawColor,line width= 0.4pt,line join=round,line cap=round,fill=fillColor] (122.36, 62.01) circle (  1.16);

\path[draw=drawColor,line width= 0.4pt,line join=round,line cap=round,fill=fillColor] (123.16, 61.75) circle (  1.16);

\path[draw=drawColor,line width= 0.4pt,line join=round,line cap=round,fill=fillColor] (123.93, 61.74) circle (  1.16);

\path[draw=drawColor,line width= 0.4pt,line join=round,line cap=round,fill=fillColor] (124.69, 61.64) circle (  1.16);

\path[draw=drawColor,line width= 0.4pt,line join=round,line cap=round,fill=fillColor] (125.43, 61.52) circle (  1.16);

\path[draw=drawColor,line width= 0.4pt,line join=round,line cap=round,fill=fillColor] (126.15, 61.49) circle (  1.16);

\path[draw=drawColor,line width= 0.4pt,line join=round,line cap=round,fill=fillColor] (126.86, 61.20) circle (  1.16);

\path[draw=drawColor,line width= 0.4pt,line join=round,line cap=round,fill=fillColor] (127.56, 61.00) circle (  1.16);

\path[draw=drawColor,line width= 0.4pt,line join=round,line cap=round,fill=fillColor] (128.24, 60.95) circle (  1.16);

\path[draw=drawColor,line width= 0.4pt,line join=round,line cap=round,fill=fillColor] (128.91, 60.83) circle (  1.16);

\path[draw=drawColor,line width= 0.4pt,line join=round,line cap=round,fill=fillColor] (129.57, 60.80) circle (  1.16);

\path[draw=drawColor,line width= 0.4pt,line join=round,line cap=round,fill=fillColor] (130.21, 60.71) circle (  1.16);

\path[draw=drawColor,line width= 0.4pt,line join=round,line cap=round,fill=fillColor] (130.85, 60.66) circle (  1.16);

\path[draw=drawColor,line width= 0.4pt,line join=round,line cap=round,fill=fillColor] (131.47, 60.66) circle (  1.16);

\path[draw=drawColor,line width= 0.4pt,line join=round,line cap=round,fill=fillColor] (132.09, 60.65) circle (  1.16);

\path[draw=drawColor,line width= 0.4pt,line join=round,line cap=round,fill=fillColor] (132.69, 60.61) circle (  1.16);

\path[draw=drawColor,line width= 0.4pt,line join=round,line cap=round,fill=fillColor] (133.28, 60.27) circle (  1.16);

\path[draw=drawColor,line width= 0.4pt,line join=round,line cap=round,fill=fillColor] (133.87, 60.26) circle (  1.16);

\path[draw=drawColor,line width= 0.4pt,line join=round,line cap=round,fill=fillColor] (134.44, 60.21) circle (  1.16);

\path[draw=drawColor,line width= 0.4pt,line join=round,line cap=round,fill=fillColor] (135.01, 60.15) circle (  1.16);

\path[draw=drawColor,line width= 0.4pt,line join=round,line cap=round,fill=fillColor] (135.57, 60.00) circle (  1.16);

\path[draw=drawColor,line width= 0.4pt,line join=round,line cap=round,fill=fillColor] (136.12, 59.94) circle (  1.16);

\path[draw=drawColor,line width= 0.4pt,line join=round,line cap=round,fill=fillColor] (136.67, 59.84) circle (  1.16);

\path[draw=drawColor,line width= 0.4pt,line join=round,line cap=round,fill=fillColor] (137.20, 59.80) circle (  1.16);

\path[draw=drawColor,line width= 0.4pt,line join=round,line cap=round,fill=fillColor] (137.73, 59.79) circle (  1.16);

\path[draw=drawColor,line width= 0.4pt,line join=round,line cap=round,fill=fillColor] (138.25, 59.63) circle (  1.16);

\path[draw=drawColor,line width= 0.4pt,line join=round,line cap=round,fill=fillColor] (138.77, 59.45) circle (  1.16);

\path[draw=drawColor,line width= 0.4pt,line join=round,line cap=round,fill=fillColor] (139.28, 59.40) circle (  1.16);

\path[draw=drawColor,line width= 0.4pt,line join=round,line cap=round,fill=fillColor] (139.78, 59.39) circle (  1.16);

\path[draw=drawColor,line width= 0.4pt,line join=round,line cap=round,fill=fillColor] (140.28, 59.25) circle (  1.16);

\path[draw=drawColor,line width= 0.4pt,line join=round,line cap=round,fill=fillColor] (140.77, 59.07) circle (  1.16);

\path[draw=drawColor,line width= 0.4pt,line join=round,line cap=round,fill=fillColor] (141.26, 59.04) circle (  1.16);

\path[draw=drawColor,line width= 0.4pt,line join=round,line cap=round,fill=fillColor] (141.74, 59.03) circle (  1.16);

\path[draw=drawColor,line width= 0.4pt,line join=round,line cap=round,fill=fillColor] (142.21, 59.00) circle (  1.16);

\path[draw=drawColor,line width= 0.4pt,line join=round,line cap=round,fill=fillColor] (142.68, 58.96) circle (  1.16);

\path[draw=drawColor,line width= 0.4pt,line join=round,line cap=round,fill=fillColor] (143.14, 58.87) circle (  1.16);

\path[draw=drawColor,line width= 0.4pt,line join=round,line cap=round,fill=fillColor] (143.60, 58.82) circle (  1.16);

\path[draw=drawColor,line width= 0.4pt,line join=round,line cap=round,fill=fillColor] (144.06, 58.79) circle (  1.16);

\path[draw=drawColor,line width= 0.4pt,line join=round,line cap=round,fill=fillColor] (144.51, 58.61) circle (  1.16);

\path[draw=drawColor,line width= 0.4pt,line join=round,line cap=round,fill=fillColor] (144.95, 58.22) circle (  1.16);

\path[draw=drawColor,line width= 0.4pt,line join=round,line cap=round,fill=fillColor] (145.39, 58.11) circle (  1.16);

\path[draw=drawColor,line width= 0.4pt,line join=round,line cap=round,fill=fillColor] (145.83, 57.92) circle (  1.16);

\path[draw=drawColor,line width= 0.4pt,line join=round,line cap=round,fill=fillColor] (146.26, 57.76) circle (  1.16);

\path[draw=drawColor,line width= 0.4pt,line join=round,line cap=round,fill=fillColor] (146.69, 57.76) circle (  1.16);

\path[draw=drawColor,line width= 0.4pt,line join=round,line cap=round,fill=fillColor] (147.11, 57.72) circle (  1.16);

\path[draw=drawColor,line width= 0.4pt,line join=round,line cap=round,fill=fillColor] (147.53, 57.63) circle (  1.16);

\path[draw=drawColor,line width= 0.4pt,line join=round,line cap=round,fill=fillColor] (147.95, 57.60) circle (  1.16);

\path[draw=drawColor,line width= 0.4pt,line join=round,line cap=round,fill=fillColor] (148.36, 57.57) circle (  1.16);

\path[draw=drawColor,line width= 0.4pt,line join=round,line cap=round,fill=fillColor] (148.77, 57.52) circle (  1.16);

\path[draw=drawColor,line width= 0.4pt,line join=round,line cap=round,fill=fillColor] (149.17, 57.47) circle (  1.16);

\path[draw=drawColor,line width= 0.4pt,line join=round,line cap=round,fill=fillColor] (149.57, 57.46) circle (  1.16);

\path[draw=drawColor,line width= 0.4pt,line join=round,line cap=round,fill=fillColor] (149.97, 57.36) circle (  1.16);

\path[draw=drawColor,line width= 0.4pt,line join=round,line cap=round,fill=fillColor] (150.37, 57.16) circle (  1.16);

\path[draw=drawColor,line width= 0.4pt,line join=round,line cap=round,fill=fillColor] (150.76, 57.16) circle (  1.16);

\path[draw=drawColor,line width= 0.4pt,line join=round,line cap=round,fill=fillColor] (151.15, 57.13) circle (  1.16);

\path[draw=drawColor,line width= 0.4pt,line join=round,line cap=round,fill=fillColor] (151.53, 57.10) circle (  1.16);

\path[draw=drawColor,line width= 0.4pt,line join=round,line cap=round,fill=fillColor] (151.91, 57.09) circle (  1.16);

\path[draw=drawColor,line width= 0.4pt,line join=round,line cap=round,fill=fillColor] (152.29, 57.08) circle (  1.16);

\path[draw=drawColor,line width= 0.4pt,line join=round,line cap=round,fill=fillColor] (152.67, 56.99) circle (  1.16);

\path[draw=drawColor,line width= 0.4pt,line join=round,line cap=round,fill=fillColor] (153.04, 56.99) circle (  1.16);

\path[draw=drawColor,line width= 0.4pt,line join=round,line cap=round,fill=fillColor] (153.41, 56.84) circle (  1.16);

\path[draw=drawColor,line width= 0.4pt,line join=round,line cap=round,fill=fillColor] (153.78, 56.82) circle (  1.16);

\path[draw=drawColor,line width= 0.4pt,line join=round,line cap=round,fill=fillColor] (154.14, 56.76) circle (  1.16);

\path[draw=drawColor,line width= 0.4pt,line join=round,line cap=round,fill=fillColor] (154.50, 56.73) circle (  1.16);

\path[draw=drawColor,line width= 0.4pt,line join=round,line cap=round,fill=fillColor] (154.86, 56.70) circle (  1.16);

\path[draw=drawColor,line width= 0.4pt,line join=round,line cap=round,fill=fillColor] (155.22, 56.69) circle (  1.16);

\path[draw=drawColor,line width= 0.4pt,line join=round,line cap=round,fill=fillColor] (155.57, 56.64) circle (  1.16);

\path[draw=drawColor,line width= 0.4pt,line join=round,line cap=round,fill=fillColor] (155.92, 56.63) circle (  1.16);

\path[draw=drawColor,line width= 0.4pt,line join=round,line cap=round,fill=fillColor] (156.27, 56.63) circle (  1.16);

\path[draw=drawColor,line width= 0.4pt,line join=round,line cap=round,fill=fillColor] (156.62, 56.58) circle (  1.16);

\path[draw=drawColor,line width= 0.4pt,line join=round,line cap=round,fill=fillColor] (156.96, 56.58) circle (  1.16);

\path[draw=drawColor,line width= 0.4pt,line join=round,line cap=round,fill=fillColor] (157.30, 56.55) circle (  1.16);

\path[draw=drawColor,line width= 0.4pt,line join=round,line cap=round,fill=fillColor] (157.64, 56.52) circle (  1.16);

\path[draw=drawColor,line width= 0.4pt,line join=round,line cap=round,fill=fillColor] (157.98, 56.52) circle (  1.16);

\path[draw=drawColor,line width= 0.4pt,line join=round,line cap=round,fill=fillColor] (158.31, 56.45) circle (  1.16);

\path[draw=drawColor,line width= 0.4pt,line join=round,line cap=round,fill=fillColor] (158.64, 56.43) circle (  1.16);

\path[draw=drawColor,line width= 0.4pt,line join=round,line cap=round,fill=fillColor] (158.97, 56.40) circle (  1.16);

\path[draw=drawColor,line width= 0.4pt,line join=round,line cap=round,fill=fillColor] (159.30, 56.38) circle (  1.16);

\path[draw=drawColor,line width= 0.4pt,line join=round,line cap=round,fill=fillColor] (159.63, 56.38) circle (  1.16);

\path[draw=drawColor,line width= 0.4pt,line join=round,line cap=round,fill=fillColor] (159.95, 56.37) circle (  1.16);

\path[draw=drawColor,line width= 0.4pt,line join=round,line cap=round,fill=fillColor] (160.27, 56.33) circle (  1.16);

\path[draw=drawColor,line width= 0.4pt,line join=round,line cap=round,fill=fillColor] (160.59, 56.21) circle (  1.16);

\path[draw=drawColor,line width= 0.4pt,line join=round,line cap=round,fill=fillColor] (160.91, 56.20) circle (  1.16);

\path[draw=drawColor,line width= 0.4pt,line join=round,line cap=round,fill=fillColor] (161.23, 56.20) circle (  1.16);

\path[draw=drawColor,line width= 0.4pt,line join=round,line cap=round,fill=fillColor] (161.54, 55.99) circle (  1.16);

\path[draw=drawColor,line width= 0.4pt,line join=round,line cap=round,fill=fillColor] (161.85, 55.95) circle (  1.16);

\path[draw=drawColor,line width= 0.4pt,line join=round,line cap=round,fill=fillColor] (162.16, 55.94) circle (  1.16);

\path[draw=drawColor,line width= 0.4pt,line join=round,line cap=round,fill=fillColor] (162.47, 55.92) circle (  1.16);

\path[draw=drawColor,line width= 0.4pt,line join=round,line cap=round,fill=fillColor] (162.78, 55.85) circle (  1.16);

\path[draw=drawColor,line width= 0.4pt,line join=round,line cap=round,fill=fillColor] (163.08, 55.84) circle (  1.16);

\path[draw=drawColor,line width= 0.4pt,line join=round,line cap=round,fill=fillColor] (163.39, 55.82) circle (  1.16);

\path[draw=drawColor,line width= 0.4pt,line join=round,line cap=round,fill=fillColor] (163.69, 55.77) circle (  1.16);

\path[draw=drawColor,line width= 0.4pt,line join=round,line cap=round,fill=fillColor] (163.99, 55.77) circle (  1.16);

\path[draw=drawColor,line width= 0.4pt,line join=round,line cap=round,fill=fillColor] (164.29, 55.74) circle (  1.16);

\path[draw=drawColor,line width= 0.4pt,line join=round,line cap=round,fill=fillColor] (164.58, 55.65) circle (  1.16);

\path[draw=drawColor,line width= 0.4pt,line join=round,line cap=round,fill=fillColor] (164.88, 55.56) circle (  1.16);

\path[draw=drawColor,line width= 0.4pt,line join=round,line cap=round,fill=fillColor] (165.17, 55.51) circle (  1.16);

\path[draw=drawColor,line width= 0.4pt,line join=round,line cap=round,fill=fillColor] (165.46, 55.50) circle (  1.16);

\path[draw=drawColor,line width= 0.4pt,line join=round,line cap=round,fill=fillColor] (165.75, 55.49) circle (  1.16);

\path[draw=drawColor,line width= 0.4pt,line join=round,line cap=round,fill=fillColor] (166.04, 55.49) circle (  1.16);

\path[draw=drawColor,line width= 0.4pt,line join=round,line cap=round,fill=fillColor] (166.33, 55.45) circle (  1.16);

\path[draw=drawColor,line width= 0.4pt,line join=round,line cap=round,fill=fillColor] (166.61, 55.32) circle (  1.16);

\path[draw=drawColor,line width= 0.4pt,line join=round,line cap=round,fill=fillColor] (166.90, 55.29) circle (  1.16);

\path[draw=drawColor,line width= 0.4pt,line join=round,line cap=round,fill=fillColor] (167.18, 55.27) circle (  1.16);

\path[draw=drawColor,line width= 0.4pt,line join=round,line cap=round,fill=fillColor] (167.46, 55.22) circle (  1.16);

\path[draw=drawColor,line width= 0.4pt,line join=round,line cap=round,fill=fillColor] (167.74, 55.19) circle (  1.16);

\path[draw=drawColor,line width= 0.4pt,line join=round,line cap=round,fill=fillColor] (168.02, 55.18) circle (  1.16);

\path[draw=drawColor,line width= 0.4pt,line join=round,line cap=round,fill=fillColor] (168.30, 55.16) circle (  1.16);

\path[draw=drawColor,line width= 0.4pt,line join=round,line cap=round,fill=fillColor] (168.57, 55.07) circle (  1.16);

\path[draw=drawColor,line width= 0.4pt,line join=round,line cap=round,fill=fillColor] (168.85, 54.92) circle (  1.16);

\path[draw=drawColor,line width= 0.4pt,line join=round,line cap=round,fill=fillColor] (169.12, 54.83) circle (  1.16);

\path[draw=drawColor,line width= 0.4pt,line join=round,line cap=round,fill=fillColor] (169.39, 54.80) circle (  1.16);

\path[draw=drawColor,line width= 0.4pt,line join=round,line cap=round,fill=fillColor] (169.66, 54.78) circle (  1.16);

\path[draw=drawColor,line width= 0.4pt,line join=round,line cap=round,fill=fillColor] (169.93, 54.71) circle (  1.16);

\path[draw=drawColor,line width= 0.4pt,line join=round,line cap=round,fill=fillColor] (170.20, 54.62) circle (  1.16);

\path[draw=drawColor,line width= 0.4pt,line join=round,line cap=round,fill=fillColor] (170.46, 54.52) circle (  1.16);

\path[draw=drawColor,line width= 0.4pt,line join=round,line cap=round,fill=fillColor] (170.73, 54.49) circle (  1.16);

\path[draw=drawColor,line width= 0.4pt,line join=round,line cap=round,fill=fillColor] (170.99, 54.43) circle (  1.16);

\path[draw=drawColor,line width= 0.4pt,line join=round,line cap=round,fill=fillColor] (171.25, 54.42) circle (  1.16);

\path[draw=drawColor,line width= 0.4pt,line join=round,line cap=round,fill=fillColor] (171.52, 54.29) circle (  1.16);

\path[draw=drawColor,line width= 0.4pt,line join=round,line cap=round,fill=fillColor] (171.78, 54.23) circle (  1.16);

\path[draw=drawColor,line width= 0.4pt,line join=round,line cap=round,fill=fillColor] (172.04, 54.19) circle (  1.16);

\path[draw=drawColor,line width= 0.4pt,line join=round,line cap=round,fill=fillColor] (172.29, 54.14) circle (  1.16);

\path[draw=drawColor,line width= 0.4pt,line join=round,line cap=round,fill=fillColor] (172.55, 54.10) circle (  1.16);

\path[draw=drawColor,line width= 0.4pt,line join=round,line cap=round,fill=fillColor] (172.81, 53.96) circle (  1.16);

\path[draw=drawColor,line width= 0.4pt,line join=round,line cap=round,fill=fillColor] (173.06, 53.95) circle (  1.16);

\path[draw=drawColor,line width= 0.4pt,line join=round,line cap=round,fill=fillColor] (173.31, 53.90) circle (  1.16);

\path[draw=drawColor,line width= 0.4pt,line join=round,line cap=round,fill=fillColor] (173.57, 53.88) circle (  1.16);

\path[draw=drawColor,line width= 0.4pt,line join=round,line cap=round,fill=fillColor] (173.82, 53.42) circle (  1.16);

\path[draw=drawColor,line width= 0.4pt,line join=round,line cap=round,fill=fillColor] (174.07, 53.40) circle (  1.16);

\path[draw=drawColor,line width= 0.4pt,line join=round,line cap=round,fill=fillColor] (174.32, 53.38) circle (  1.16);

\path[draw=drawColor,line width= 0.4pt,line join=round,line cap=round,fill=fillColor] (174.56, 53.31) circle (  1.16);

\path[draw=drawColor,line width= 0.4pt,line join=round,line cap=round,fill=fillColor] (174.81, 53.25) circle (  1.16);

\path[draw=drawColor,line width= 0.4pt,line join=round,line cap=round,fill=fillColor] (175.06, 53.13) circle (  1.16);

\path[draw=drawColor,line width= 0.4pt,line join=round,line cap=round,fill=fillColor] (175.30, 53.10) circle (  1.16);

\path[draw=drawColor,line width= 0.4pt,line join=round,line cap=round,fill=fillColor] (175.55, 53.05) circle (  1.16);

\path[draw=drawColor,line width= 0.4pt,line join=round,line cap=round,fill=fillColor] (175.79, 53.04) circle (  1.16);

\path[draw=drawColor,line width= 0.4pt,line join=round,line cap=round,fill=fillColor] (176.03, 52.98) circle (  1.16);

\path[draw=drawColor,line width= 0.4pt,line join=round,line cap=round,fill=fillColor] (176.27, 52.89) circle (  1.16);

\path[draw=drawColor,line width= 0.4pt,line join=round,line cap=round,fill=fillColor] (176.51, 52.74) circle (  1.16);

\path[draw=drawColor,line width= 0.4pt,line join=round,line cap=round,fill=fillColor] (176.75, 52.50) circle (  1.16);

\path[draw=drawColor,line width= 0.4pt,line join=round,line cap=round,fill=fillColor] (176.99, 52.46) circle (  1.16);

\path[draw=drawColor,line width= 0.4pt,line join=round,line cap=round,fill=fillColor] (177.22, 52.40) circle (  1.16);

\path[draw=drawColor,line width= 0.4pt,line join=round,line cap=round,fill=fillColor] (177.46, 52.36) circle (  1.16);

\path[draw=drawColor,line width= 0.4pt,line join=round,line cap=round,fill=fillColor] (177.70, 52.30) circle (  1.16);

\path[draw=drawColor,line width= 0.4pt,line join=round,line cap=round,fill=fillColor] (177.93, 52.20) circle (  1.16);

\path[draw=drawColor,line width= 0.4pt,line join=round,line cap=round,fill=fillColor] (178.16, 52.18) circle (  1.16);

\path[draw=drawColor,line width= 0.4pt,line join=round,line cap=round,fill=fillColor] (178.40, 52.10) circle (  1.16);

\path[draw=drawColor,line width= 0.4pt,line join=round,line cap=round,fill=fillColor] (178.63, 52.06) circle (  1.16);

\path[draw=drawColor,line width= 0.4pt,line join=round,line cap=round,fill=fillColor] (178.86, 51.95) circle (  1.16);

\path[draw=drawColor,line width= 0.4pt,line join=round,line cap=round,fill=fillColor] (179.09, 51.87) circle (  1.16);

\path[draw=drawColor,line width= 0.4pt,line join=round,line cap=round,fill=fillColor] (179.32, 51.87) circle (  1.16);

\path[draw=drawColor,line width= 0.4pt,line join=round,line cap=round,fill=fillColor] (179.55, 51.86) circle (  1.16);

\path[draw=drawColor,line width= 0.4pt,line join=round,line cap=round,fill=fillColor] (179.77, 51.86) circle (  1.16);

\path[draw=drawColor,line width= 0.4pt,line join=round,line cap=round,fill=fillColor] (180.00, 51.74) circle (  1.16);

\path[draw=drawColor,line width= 0.4pt,line join=round,line cap=round,fill=fillColor] (180.22, 51.74) circle (  1.16);

\path[draw=drawColor,line width= 0.4pt,line join=round,line cap=round,fill=fillColor] (180.45, 51.36) circle (  1.16);

\path[draw=drawColor,line width= 0.4pt,line join=round,line cap=round,fill=fillColor] (180.67, 51.29) circle (  1.16);

\path[draw=drawColor,line width= 0.4pt,line join=round,line cap=round,fill=fillColor] (180.90, 50.89) circle (  1.16);

\path[draw=drawColor,line width= 0.4pt,line join=round,line cap=round,fill=fillColor] (181.12, 50.74) circle (  1.16);

\path[draw=drawColor,line width= 0.4pt,line join=round,line cap=round,fill=fillColor] (181.34, 50.67) circle (  1.16);

\path[draw=drawColor,line width= 0.4pt,line join=round,line cap=round,fill=fillColor] (181.56, 50.55) circle (  1.16);

\path[draw=drawColor,line width= 0.4pt,line join=round,line cap=round,fill=fillColor] (181.78, 50.15) circle (  1.16);

\path[draw=drawColor,line width= 0.4pt,line join=round,line cap=round,fill=fillColor] (182.00, 50.06) circle (  1.16);

\path[draw=drawColor,line width= 0.4pt,line join=round,line cap=round,fill=fillColor] (182.22, 49.82) circle (  1.16);

\path[draw=drawColor,line width= 0.4pt,line join=round,line cap=round,fill=fillColor] (182.44, 49.80) circle (  1.16);

\path[draw=drawColor,line width= 0.4pt,line join=round,line cap=round,fill=fillColor] (182.65, 49.40) circle (  1.16);

\path[draw=drawColor,line width= 0.4pt,line join=round,line cap=round,fill=fillColor] (182.87, 49.05) circle (  1.16);

\path[draw=drawColor,line width= 0.4pt,line join=round,line cap=round,fill=fillColor] (183.09, 48.34) circle (  1.16);

\path[draw=drawColor,line width= 0.4pt,line join=round,line cap=round,fill=fillColor] (183.30, 48.04) circle (  1.16);

\path[draw=drawColor,line width= 0.4pt,line join=round,line cap=round,fill=fillColor] (183.51, 47.38) circle (  1.16);

\path[draw=drawColor,line width= 0.4pt,line join=round,line cap=round,fill=fillColor] (183.73, 44.78) circle (  1.16);

\path[draw=drawColor,line width= 0.4pt,line join=round,line cap=round,fill=fillColor] (183.94, 44.78) circle (  1.16);

\path[draw=drawColor,line width= 0.4pt,line join=round,line cap=round,fill=fillColor] (184.15, 44.78) circle (  1.16);

\path[draw=drawColor,line width= 0.4pt,line join=round,line cap=round,fill=fillColor] (184.36, 44.78) circle (  1.16);

\path[draw=drawColor,line width= 0.4pt,line join=round,line cap=round,fill=fillColor] (184.57, 44.78) circle (  1.16);

\path[draw=drawColor,line width= 0.4pt,line join=round,line cap=round,fill=fillColor] (184.78, 44.78) circle (  1.16);

\path[draw=drawColor,line width= 0.4pt,line join=round,line cap=round,fill=fillColor] (184.99, 44.78) circle (  1.16);

\path[draw=drawColor,line width= 0.4pt,line join=round,line cap=round,fill=fillColor] (185.20, 44.78) circle (  1.16);

\path[draw=drawColor,line width= 0.4pt,line join=round,line cap=round,fill=fillColor] (185.41, 44.78) circle (  1.16);

\path[draw=drawColor,line width= 0.4pt,line join=round,line cap=round,fill=fillColor] (185.61, 44.78) circle (  1.16);

\path[draw=drawColor,line width= 0.4pt,line join=round,line cap=round,fill=fillColor] (185.82, 44.78) circle (  1.16);

\path[draw=drawColor,line width= 0.4pt,line join=round,line cap=round,fill=fillColor] (186.03, 44.78) circle (  1.16);

\path[draw=drawColor,line width= 0.4pt,line join=round,line cap=round,fill=fillColor] (186.23, 44.78) circle (  1.16);

\path[draw=drawColor,line width= 0.4pt,line join=round,line cap=round,fill=fillColor] (186.44, 44.78) circle (  1.16);

\path[draw=drawColor,line width= 0.4pt,line join=round,line cap=round,fill=fillColor] (186.64, 44.78) circle (  1.16);

\path[draw=drawColor,line width= 0.4pt,line join=round,line cap=round,fill=fillColor] (186.84, 44.78) circle (  1.16);

\path[draw=drawColor,line width= 0.4pt,line join=round,line cap=round,fill=fillColor] (187.05, 44.78) circle (  1.16);

\path[draw=drawColor,line width= 0.4pt,line join=round,line cap=round,fill=fillColor] (187.25, 44.78) circle (  1.16);

\path[draw=drawColor,line width= 0.4pt,line join=round,line cap=round,fill=fillColor] (187.45, 44.78) circle (  1.16);

\path[draw=drawColor,line width= 0.4pt,line join=round,line cap=round,fill=fillColor] (187.65, 44.78) circle (  1.16);

\path[draw=drawColor,line width= 0.4pt,line join=round,line cap=round,fill=fillColor] (187.85, 44.78) circle (  1.16);

\path[draw=drawColor,line width= 0.4pt,line join=round,line cap=round,fill=fillColor] (188.05, 44.78) circle (  1.16);

\path[draw=drawColor,line width= 0.4pt,line join=round,line cap=round,fill=fillColor] (188.25, 44.78) circle (  1.16);

\path[draw=drawColor,line width= 0.4pt,line join=round,line cap=round,fill=fillColor] (188.45, 44.78) circle (  1.16);

\path[draw=drawColor,line width= 0.4pt,line join=round,line cap=round,fill=fillColor] (188.64, 44.78) circle (  1.16);

\path[draw=drawColor,line width= 0.4pt,line join=round,line cap=round,fill=fillColor] (188.84, 44.78) circle (  1.16);

\path[draw=drawColor,line width= 0.4pt,line join=round,line cap=round,fill=fillColor] (189.04, 44.78) circle (  1.16);

\path[draw=drawColor,line width= 0.4pt,line join=round,line cap=round,fill=fillColor] (189.23, 44.78) circle (  1.16);

\path[draw=drawColor,line width= 0.4pt,line join=round,line cap=round,fill=fillColor] (189.43, 44.78) circle (  1.16);

\path[draw=drawColor,line width= 0.4pt,line join=round,line cap=round,fill=fillColor] (189.62, 44.78) circle (  1.16);

\path[draw=drawColor,line width= 0.4pt,line join=round,line cap=round,fill=fillColor] (189.82, 44.78) circle (  1.16);

\path[draw=drawColor,line width= 0.4pt,line join=round,line cap=round,fill=fillColor] (190.01, 44.78) circle (  1.16);

\path[draw=drawColor,line width= 0.4pt,line join=round,line cap=round,fill=fillColor] (190.20, 44.78) circle (  1.16);

\path[draw=drawColor,line width= 0.4pt,line join=round,line cap=round,fill=fillColor] (190.39, 44.78) circle (  1.16);

\path[draw=drawColor,line width= 0.4pt,line join=round,line cap=round,fill=fillColor] (190.59, 44.78) circle (  1.16);

\path[draw=drawColor,line width= 0.4pt,line join=round,line cap=round,fill=fillColor] (190.78, 44.78) circle (  1.16);

\path[draw=drawColor,line width= 0.4pt,line join=round,line cap=round,fill=fillColor] (190.97, 44.78) circle (  1.16);

\path[draw=drawColor,line width= 0.4pt,line join=round,line cap=round,fill=fillColor] (191.16, 44.78) circle (  1.16);

\path[draw=drawColor,line width= 0.4pt,line join=round,line cap=round,fill=fillColor] (191.35, 44.78) circle (  1.16);

\path[draw=drawColor,line width= 0.4pt,line join=round,line cap=round,fill=fillColor] (191.54, 44.78) circle (  1.16);

\path[draw=drawColor,line width= 0.4pt,line join=round,line cap=round,fill=fillColor] (191.73, 44.78) circle (  1.16);

\path[draw=drawColor,line width= 0.4pt,line join=round,line cap=round,fill=fillColor] (191.91, 44.78) circle (  1.16);

\path[draw=drawColor,line width= 0.4pt,line join=round,line cap=round,fill=fillColor] (192.10, 44.78) circle (  1.16);

\path[draw=drawColor,line width= 0.4pt,line join=round,line cap=round,fill=fillColor] (192.29, 44.78) circle (  1.16);

\path[draw=drawColor,line width= 0.4pt,line join=round,line cap=round,fill=fillColor] (192.47, 44.78) circle (  1.16);

\path[draw=drawColor,line width= 0.4pt,line join=round,line cap=round,fill=fillColor] (192.66, 44.78) circle (  1.16);

\path[draw=drawColor,line width= 0.4pt,line join=round,line cap=round,fill=fillColor] (192.85, 44.78) circle (  1.16);

\path[draw=drawColor,line width= 0.4pt,line join=round,line cap=round,fill=fillColor] (193.03, 44.78) circle (  1.16);

\path[draw=drawColor,line width= 0.4pt,line join=round,line cap=round,fill=fillColor] (193.22, 44.78) circle (  1.16);

\path[draw=drawColor,line width= 0.4pt,line join=round,line cap=round,fill=fillColor] (193.40, 44.78) circle (  1.16);

\path[draw=drawColor,line width= 0.4pt,line join=round,line cap=round,fill=fillColor] (193.58, 44.78) circle (  1.16);

\path[draw=drawColor,line width= 0.4pt,line join=round,line cap=round,fill=fillColor] (193.77, 44.78) circle (  1.16);

\path[draw=drawColor,line width= 0.4pt,line join=round,line cap=round,fill=fillColor] (193.95, 44.78) circle (  1.16);

\path[draw=drawColor,line width= 0.4pt,line join=round,line cap=round,fill=fillColor] (194.13, 44.78) circle (  1.16);

\path[draw=drawColor,line width= 0.4pt,line join=round,line cap=round,fill=fillColor] (194.31, 44.78) circle (  1.16);

\path[draw=drawColor,line width= 0.4pt,line join=round,line cap=round,fill=fillColor] (194.49, 44.78) circle (  1.16);

\path[draw=drawColor,line width= 0.4pt,line join=round,line cap=round,fill=fillColor] (194.67, 44.78) circle (  1.16);

\path[draw=drawColor,line width= 0.4pt,line join=round,line cap=round,fill=fillColor] (194.85, 44.78) circle (  1.16);

\path[draw=drawColor,line width= 0.4pt,line join=round,line cap=round,fill=fillColor] (195.03, 44.78) circle (  1.16);

\path[draw=drawColor,line width= 0.4pt,line join=round,line cap=round,fill=fillColor] (195.21, 44.78) circle (  1.16);

\path[draw=drawColor,line width= 0.4pt,line join=round,line cap=round,fill=fillColor] (195.39, 44.78) circle (  1.16);

\path[draw=drawColor,line width= 0.4pt,line join=round,line cap=round,fill=fillColor] (195.57, 44.78) circle (  1.16);

\path[draw=drawColor,line width= 0.4pt,line join=round,line cap=round,fill=fillColor] (195.75, 44.78) circle (  1.16);

\path[draw=drawColor,line width= 0.4pt,line join=round,line cap=round,fill=fillColor] (195.92, 44.78) circle (  1.16);

\path[draw=drawColor,line width= 0.4pt,line join=round,line cap=round,fill=fillColor] (196.10, 44.78) circle (  1.16);

\path[draw=drawColor,line width= 0.4pt,line join=round,line cap=round,fill=fillColor] (196.28, 44.78) circle (  1.16);

\path[draw=drawColor,line width= 0.4pt,line join=round,line cap=round,fill=fillColor] (196.45, 44.78) circle (  1.16);

\path[draw=drawColor,line width= 0.4pt,line join=round,line cap=round,fill=fillColor] (196.63, 44.78) circle (  1.16);

\path[draw=drawColor,line width= 0.4pt,line join=round,line cap=round,fill=fillColor] (196.80, 44.78) circle (  1.16);

\path[draw=drawColor,line width= 0.4pt,line join=round,line cap=round,fill=fillColor] (196.98, 44.78) circle (  1.16);

\path[draw=drawColor,line width= 0.4pt,line join=round,line cap=round,fill=fillColor] (197.15, 44.78) circle (  1.16);

\path[draw=drawColor,line width= 0.4pt,line join=round,line cap=round,fill=fillColor] (197.33, 44.78) circle (  1.16);

\path[draw=drawColor,line width= 0.4pt,line join=round,line cap=round,fill=fillColor] (197.50, 44.78) circle (  1.16);

\path[draw=drawColor,line width= 0.4pt,line join=round,line cap=round,fill=fillColor] (197.67, 44.78) circle (  1.16);

\path[draw=drawColor,line width= 0.4pt,line join=round,line cap=round,fill=fillColor] (197.85, 44.78) circle (  1.16);

\path[draw=drawColor,line width= 0.4pt,line join=round,line cap=round,fill=fillColor] (198.02, 44.78) circle (  1.16);

\path[draw=drawColor,line width= 0.4pt,line join=round,line cap=round,fill=fillColor] (198.19, 44.78) circle (  1.16);

\path[draw=drawColor,line width= 0.4pt,line join=round,line cap=round,fill=fillColor] (198.36, 44.78) circle (  1.16);

\path[draw=drawColor,line width= 0.4pt,line join=round,line cap=round,fill=fillColor] (198.53, 44.78) circle (  1.16);

\path[draw=drawColor,line width= 0.4pt,line join=round,line cap=round,fill=fillColor] (198.70, 44.78) circle (  1.16);

\path[draw=drawColor,line width= 0.4pt,line join=round,line cap=round,fill=fillColor] (198.87, 44.78) circle (  1.16);

\path[draw=drawColor,line width= 0.4pt,line join=round,line cap=round,fill=fillColor] (199.04, 44.78) circle (  1.16);

\path[draw=drawColor,line width= 0.4pt,line join=round,line cap=round,fill=fillColor] (199.21, 44.78) circle (  1.16);

\path[draw=drawColor,line width= 0.4pt,line join=round,line cap=round,fill=fillColor] (199.38, 44.78) circle (  1.16);

\path[draw=drawColor,line width= 0.4pt,line join=round,line cap=round,fill=fillColor] (199.55, 44.78) circle (  1.16);

\path[draw=drawColor,line width= 0.4pt,line join=round,line cap=round,fill=fillColor] (199.72, 44.78) circle (  1.16);

\path[draw=drawColor,line width= 0.4pt,line join=round,line cap=round,fill=fillColor] (199.88, 44.78) circle (  1.16);

\path[draw=drawColor,line width= 0.4pt,line join=round,line cap=round,fill=fillColor] (200.05, 44.78) circle (  1.16);

\path[draw=drawColor,line width= 0.4pt,line join=round,line cap=round,fill=fillColor] (200.22, 44.78) circle (  1.16);

\path[draw=drawColor,line width= 0.4pt,line join=round,line cap=round,fill=fillColor] (200.38, 44.78) circle (  1.16);

\path[draw=drawColor,line width= 0.4pt,line join=round,line cap=round,fill=fillColor] (200.55, 44.78) circle (  1.16);

\path[draw=drawColor,line width= 0.4pt,line join=round,line cap=round,fill=fillColor] (200.72, 44.78) circle (  1.16);

\path[draw=drawColor,line width= 0.4pt,line join=round,line cap=round,fill=fillColor] (200.88, 44.78) circle (  1.16);

\path[draw=drawColor,line width= 0.4pt,line join=round,line cap=round,fill=fillColor] (201.05, 44.78) circle (  1.16);

\path[draw=drawColor,line width= 0.4pt,line join=round,line cap=round,fill=fillColor] (201.21, 44.78) circle (  1.16);

\path[draw=drawColor,line width= 0.4pt,line join=round,line cap=round,fill=fillColor] (201.37, 44.78) circle (  1.16);

\path[draw=drawColor,line width= 0.4pt,line join=round,line cap=round,fill=fillColor] (201.54, 44.78) circle (  1.16);

\path[draw=drawColor,line width= 0.4pt,line join=round,line cap=round,fill=fillColor] (201.70, 44.78) circle (  1.16);

\path[draw=drawColor,line width= 0.4pt,line join=round,line cap=round,fill=fillColor] (201.86, 44.78) circle (  1.16);

\path[draw=drawColor,line width= 0.4pt,line join=round,line cap=round,fill=fillColor] (202.03, 44.78) circle (  1.16);

\path[draw=drawColor,line width= 0.4pt,line join=round,line cap=round,fill=fillColor] (202.19, 44.78) circle (  1.16);

\path[draw=drawColor,line width= 0.4pt,line join=round,line cap=round,fill=fillColor] (202.35, 44.78) circle (  1.16);

\path[draw=drawColor,line width= 0.4pt,line join=round,line cap=round,fill=fillColor] (202.51, 44.78) circle (  1.16);

\path[draw=drawColor,line width= 0.4pt,line join=round,line cap=round,fill=fillColor] (202.67, 44.78) circle (  1.16);

\path[draw=drawColor,line width= 0.4pt,line join=round,line cap=round,fill=fillColor] (202.83, 44.78) circle (  1.16);

\path[draw=drawColor,line width= 0.4pt,line join=round,line cap=round,fill=fillColor] (202.99, 44.78) circle (  1.16);

\path[draw=drawColor,line width= 0.4pt,line join=round,line cap=round,fill=fillColor] (203.16, 44.78) circle (  1.16);

\path[draw=drawColor,line width= 0.4pt,line join=round,line cap=round,fill=fillColor] (203.31, 44.78) circle (  1.16);

\path[draw=drawColor,line width= 0.4pt,line join=round,line cap=round,fill=fillColor] (203.47, 44.78) circle (  1.16);

\path[draw=drawColor,line width= 0.4pt,line join=round,line cap=round,fill=fillColor] (203.63, 44.78) circle (  1.16);

\path[draw=drawColor,line width= 0.4pt,line join=round,line cap=round,fill=fillColor] (203.79, 44.78) circle (  1.16);

\path[draw=drawColor,line width= 0.4pt,line join=round,line cap=round,fill=fillColor] (203.95, 44.78) circle (  1.16);

\path[draw=drawColor,line width= 0.4pt,line join=round,line cap=round,fill=fillColor] (204.11, 44.78) circle (  1.16);

\path[draw=drawColor,line width= 0.4pt,line join=round,line cap=round,fill=fillColor] (204.27, 44.78) circle (  1.16);

\path[draw=drawColor,line width= 0.4pt,line join=round,line cap=round,fill=fillColor] (204.42, 44.78) circle (  1.16);

\path[draw=drawColor,line width= 0.4pt,line join=round,line cap=round,fill=fillColor] (204.58, 44.78) circle (  1.16);

\path[draw=drawColor,line width= 0.4pt,line join=round,line cap=round,fill=fillColor] (204.74, 44.78) circle (  1.16);

\path[draw=drawColor,line width= 0.4pt,line join=round,line cap=round,fill=fillColor] (204.89, 44.78) circle (  1.16);

\path[draw=drawColor,line width= 0.4pt,line join=round,line cap=round,fill=fillColor] (205.05, 44.78) circle (  1.16);

\path[draw=drawColor,line width= 0.4pt,line join=round,line cap=round,fill=fillColor] (205.21, 44.78) circle (  1.16);

\path[draw=drawColor,line width= 0.4pt,line join=round,line cap=round,fill=fillColor] (205.36, 44.78) circle (  1.16);

\path[draw=drawColor,line width= 0.4pt,line join=round,line cap=round,fill=fillColor] (205.52, 44.78) circle (  1.16);

\path[draw=drawColor,line width= 0.4pt,line join=round,line cap=round,fill=fillColor] (205.67, 44.78) circle (  1.16);

\path[draw=drawColor,line width= 0.4pt,line join=round,line cap=round,fill=fillColor] (205.83, 44.78) circle (  1.16);

\path[draw=drawColor,line width= 0.4pt,line join=round,line cap=round,fill=fillColor] (205.98, 44.78) circle (  1.16);

\path[draw=drawColor,line width= 0.4pt,line join=round,line cap=round,fill=fillColor] (206.13, 44.78) circle (  1.16);

\path[draw=drawColor,line width= 0.4pt,line join=round,line cap=round,fill=fillColor] (206.29, 44.78) circle (  1.16);

\path[draw=drawColor,line width= 0.4pt,line join=round,line cap=round,fill=fillColor] (206.44, 44.78) circle (  1.16);

\path[draw=drawColor,line width= 0.4pt,line join=round,line cap=round,fill=fillColor] (206.59, 44.78) circle (  1.16);

\path[draw=drawColor,line width= 0.4pt,line join=round,line cap=round,fill=fillColor] (206.74, 44.78) circle (  1.16);

\path[draw=drawColor,line width= 0.4pt,line join=round,line cap=round,fill=fillColor] (206.90, 44.78) circle (  1.16);

\path[draw=drawColor,line width= 0.4pt,line join=round,line cap=round,fill=fillColor] (207.05, 44.78) circle (  1.16);

\path[draw=drawColor,line width= 0.4pt,line join=round,line cap=round,fill=fillColor] (207.20, 44.78) circle (  1.16);

\path[draw=drawColor,line width= 0.4pt,line join=round,line cap=round,fill=fillColor] (207.35, 44.78) circle (  1.16);

\path[draw=drawColor,line width= 0.4pt,line join=round,line cap=round,fill=fillColor] (207.50, 44.78) circle (  1.16);

\path[draw=drawColor,line width= 0.4pt,line join=round,line cap=round,fill=fillColor] (207.65, 44.78) circle (  1.16);

\path[draw=drawColor,line width= 0.4pt,line join=round,line cap=round,fill=fillColor] (207.80, 44.78) circle (  1.16);

\path[draw=drawColor,line width= 0.4pt,line join=round,line cap=round,fill=fillColor] (207.95, 44.78) circle (  1.16);

\path[draw=drawColor,line width= 0.4pt,line join=round,line cap=round,fill=fillColor] (208.10, 44.78) circle (  1.16);

\path[draw=drawColor,line width= 0.4pt,line join=round,line cap=round,fill=fillColor] (208.25, 44.78) circle (  1.16);

\path[draw=drawColor,line width= 0.4pt,line join=round,line cap=round,fill=fillColor] (208.40, 44.78) circle (  1.16);

\path[draw=drawColor,line width= 0.4pt,line join=round,line cap=round,fill=fillColor] (208.55, 44.78) circle (  1.16);

\path[draw=drawColor,line width= 0.4pt,line join=round,line cap=round,fill=fillColor] (208.70, 44.78) circle (  1.16);

\path[draw=drawColor,line width= 0.4pt,line join=round,line cap=round,fill=fillColor] (208.85, 44.78) circle (  1.16);

\path[draw=drawColor,line width= 0.4pt,line join=round,line cap=round,fill=fillColor] (209.00, 44.78) circle (  1.16);

\path[draw=drawColor,line width= 0.4pt,line join=round,line cap=round,fill=fillColor] (209.14, 44.78) circle (  1.16);

\path[draw=drawColor,line width= 0.4pt,line join=round,line cap=round,fill=fillColor] (209.29, 44.78) circle (  1.16);

\path[draw=drawColor,line width= 0.4pt,line join=round,line cap=round,fill=fillColor] (209.44, 44.78) circle (  1.16);

\path[draw=drawColor,line width= 0.4pt,line join=round,line cap=round,fill=fillColor] (209.59, 44.78) circle (  1.16);

\path[draw=drawColor,line width= 0.4pt,line join=round,line cap=round,fill=fillColor] (209.73, 44.78) circle (  1.16);

\path[draw=drawColor,line width= 0.4pt,line join=round,line cap=round,fill=fillColor] (209.88, 44.78) circle (  1.16);

\path[draw=drawColor,line width= 0.4pt,line join=round,line cap=round,fill=fillColor] (210.02, 44.78) circle (  1.16);

\path[draw=drawColor,line width= 0.4pt,line join=round,line cap=round,fill=fillColor] (210.17, 44.78) circle (  1.16);

\path[draw=drawColor,line width= 0.4pt,line join=round,line cap=round,fill=fillColor] (210.32, 44.78) circle (  1.16);

\path[draw=drawColor,line width= 0.4pt,line join=round,line cap=round,fill=fillColor] (210.46, 44.78) circle (  1.16);

\path[draw=drawColor,line width= 0.4pt,line join=round,line cap=round,fill=fillColor] (210.61, 44.78) circle (  1.16);

\path[draw=drawColor,line width= 0.4pt,line join=round,line cap=round,fill=fillColor] (210.75, 44.78) circle (  1.16);

\path[draw=drawColor,line width= 0.4pt,line join=round,line cap=round,fill=fillColor] (210.89, 44.78) circle (  1.16);

\path[draw=drawColor,line width= 0.4pt,line join=round,line cap=round,fill=fillColor] (211.04, 44.78) circle (  1.16);

\path[draw=drawColor,line width= 0.4pt,line join=round,line cap=round,fill=fillColor] (211.18, 44.78) circle (  1.16);

\path[draw=drawColor,line width= 0.4pt,line join=round,line cap=round,fill=fillColor] (211.33, 44.78) circle (  1.16);

\path[draw=drawColor,line width= 0.4pt,line join=round,line cap=round,fill=fillColor] (211.47, 44.78) circle (  1.16);

\path[draw=drawColor,line width= 0.4pt,line join=round,line cap=round,fill=fillColor] (211.61, 44.78) circle (  1.16);

\path[draw=drawColor,line width= 0.4pt,line join=round,line cap=round,fill=fillColor] (211.76, 44.78) circle (  1.16);

\path[draw=drawColor,line width= 0.4pt,line join=round,line cap=round,fill=fillColor] (211.90, 44.78) circle (  1.16);

\path[draw=drawColor,line width= 0.4pt,line join=round,line cap=round,fill=fillColor] (212.04, 44.78) circle (  1.16);

\path[draw=drawColor,line width= 0.4pt,line join=round,line cap=round,fill=fillColor] (212.18, 44.78) circle (  1.16);

\path[draw=drawColor,line width= 0.4pt,line join=round,line cap=round,fill=fillColor] (212.32, 44.78) circle (  1.16);

\path[draw=drawColor,line width= 0.4pt,line join=round,line cap=round,fill=fillColor] (212.47, 44.78) circle (  1.16);

\path[draw=drawColor,line width= 0.4pt,line join=round,line cap=round,fill=fillColor] (212.61, 44.78) circle (  1.16);

\path[draw=drawColor,line width= 0.4pt,line join=round,line cap=round,fill=fillColor] (212.75, 44.78) circle (  1.16);

\path[draw=drawColor,line width= 0.4pt,line join=round,line cap=round,fill=fillColor] (212.89, 44.78) circle (  1.16);

\path[draw=drawColor,line width= 0.4pt,line join=round,line cap=round,fill=fillColor] (213.03, 44.78) circle (  1.16);

\path[draw=drawColor,line width= 0.4pt,line join=round,line cap=round,fill=fillColor] (213.17, 44.78) circle (  1.16);

\path[draw=drawColor,line width= 0.4pt,line join=round,line cap=round,fill=fillColor] (213.31, 44.78) circle (  1.16);

\path[draw=drawColor,line width= 0.4pt,line join=round,line cap=round,fill=fillColor] (213.45, 44.78) circle (  1.16);

\path[draw=drawColor,line width= 0.4pt,line join=round,line cap=round,fill=fillColor] (213.59, 44.78) circle (  1.16);

\path[draw=drawColor,line width= 0.4pt,line join=round,line cap=round,fill=fillColor] (213.73, 44.78) circle (  1.16);

\path[draw=drawColor,line width= 0.4pt,line join=round,line cap=round,fill=fillColor] (213.87, 44.78) circle (  1.16);

\path[draw=drawColor,line width= 0.4pt,line join=round,line cap=round,fill=fillColor] (214.00, 44.78) circle (  1.16);

\path[draw=drawColor,line width= 0.4pt,line join=round,line cap=round,fill=fillColor] (214.14, 44.78) circle (  1.16);

\path[draw=drawColor,line width= 0.4pt,line join=round,line cap=round,fill=fillColor] (214.28, 44.78) circle (  1.16);

\path[draw=drawColor,line width= 0.4pt,line join=round,line cap=round,fill=fillColor] (214.42, 44.78) circle (  1.16);

\path[draw=drawColor,line width= 0.4pt,line join=round,line cap=round,fill=fillColor] (214.56, 44.78) circle (  1.16);

\path[draw=drawColor,line width= 0.4pt,line join=round,line cap=round,fill=fillColor] (214.69, 44.78) circle (  1.16);

\path[draw=drawColor,line width= 0.4pt,line join=round,line cap=round,fill=fillColor] (214.83, 44.78) circle (  1.16);

\path[draw=drawColor,line width= 0.4pt,line join=round,line cap=round,fill=fillColor] (214.97, 44.78) circle (  1.16);

\path[draw=drawColor,line width= 0.4pt,line join=round,line cap=round,fill=fillColor] (215.11, 44.78) circle (  1.16);

\path[draw=drawColor,line width= 0.4pt,line join=round,line cap=round,fill=fillColor] (215.24, 44.78) circle (  1.16);

\path[draw=drawColor,line width= 0.4pt,line join=round,line cap=round,fill=fillColor] (215.38, 44.78) circle (  1.16);

\path[draw=drawColor,line width= 0.4pt,line join=round,line cap=round,fill=fillColor] (215.51, 44.78) circle (  1.16);

\path[draw=drawColor,line width= 0.4pt,line join=round,line cap=round,fill=fillColor] (215.65, 44.78) circle (  1.16);

\path[draw=drawColor,line width= 0.4pt,line join=round,line cap=round,fill=fillColor] (215.79, 44.78) circle (  1.16);

\path[draw=drawColor,line width= 0.4pt,line join=round,line cap=round,fill=fillColor] (215.92, 44.78) circle (  1.16);

\path[draw=drawColor,line width= 0.4pt,line join=round,line cap=round,fill=fillColor] (216.06, 44.78) circle (  1.16);

\path[draw=drawColor,line width= 0.4pt,line join=round,line cap=round,fill=fillColor] (216.19, 44.78) circle (  1.16);

\path[draw=drawColor,line width= 0.4pt,line join=round,line cap=round,fill=fillColor] (216.33, 44.78) circle (  1.16);

\path[draw=drawColor,line width= 0.4pt,line join=round,line cap=round,fill=fillColor] (216.46, 44.78) circle (  1.16);

\path[draw=drawColor,line width= 0.4pt,line join=round,line cap=round,fill=fillColor] (216.59, 44.78) circle (  1.16);

\path[draw=drawColor,line width= 0.4pt,line join=round,line cap=round,fill=fillColor] (216.73, 44.78) circle (  1.16);

\path[draw=drawColor,line width= 0.4pt,line join=round,line cap=round,fill=fillColor] (216.86, 44.78) circle (  1.16);

\path[draw=drawColor,line width= 0.4pt,line join=round,line cap=round,fill=fillColor] (217.00, 44.78) circle (  1.16);

\path[draw=drawColor,line width= 0.4pt,line join=round,line cap=round,fill=fillColor] (217.13, 44.78) circle (  1.16);

\path[draw=drawColor,line width= 0.4pt,line join=round,line cap=round,fill=fillColor] (217.26, 44.78) circle (  1.16);

\path[draw=drawColor,line width= 0.4pt,line join=round,line cap=round,fill=fillColor] (217.40, 44.78) circle (  1.16);

\path[draw=drawColor,line width= 0.4pt,line join=round,line cap=round,fill=fillColor] (217.53, 44.78) circle (  1.16);

\path[draw=drawColor,line width= 0.4pt,line join=round,line cap=round,fill=fillColor] (217.66, 44.78) circle (  1.16);

\path[draw=drawColor,line width= 0.4pt,line join=round,line cap=round,fill=fillColor] (217.79, 44.78) circle (  1.16);

\path[draw=drawColor,line width= 0.4pt,line join=round,line cap=round,fill=fillColor] (217.92, 44.78) circle (  1.16);

\path[draw=drawColor,line width= 0.4pt,line join=round,line cap=round,fill=fillColor] (218.06, 44.78) circle (  1.16);

\path[draw=drawColor,line width= 0.4pt,line join=round,line cap=round,fill=fillColor] (218.19, 44.78) circle (  1.16);

\path[draw=drawColor,line width= 0.4pt,line join=round,line cap=round,fill=fillColor] (218.32, 44.78) circle (  1.16);

\path[draw=drawColor,line width= 0.4pt,line join=round,line cap=round,fill=fillColor] (218.45, 44.78) circle (  1.16);

\path[draw=drawColor,line width= 0.4pt,line join=round,line cap=round,fill=fillColor] (218.58, 44.78) circle (  1.16);

\path[draw=drawColor,line width= 0.4pt,line join=round,line cap=round,fill=fillColor] (218.71, 44.78) circle (  1.16);

\path[draw=drawColor,line width= 0.4pt,line join=round,line cap=round,fill=fillColor] (218.84, 44.78) circle (  1.16);

\path[draw=drawColor,line width= 0.4pt,line join=round,line cap=round,fill=fillColor] (218.97, 44.78) circle (  1.16);

\path[draw=drawColor,line width= 0.4pt,line join=round,line cap=round,fill=fillColor] (219.10, 44.78) circle (  1.16);
\definecolor[named]{drawColor}{rgb}{0.00,0.00,0.00}
\definecolor[named]{fillColor}{rgb}{0.00,0.00,0.00}

\path[draw=drawColor,line width= 0.6pt,line join=round,fill=fillColor] ( 71.96,124.17) -- (226.11,124.17);

\node[text=drawColor,anchor=base east,inner sep=0pt, outer sep=0pt, scale=  0.85] at (222.61,123.29) {infeasible solutions};

\path[draw=drawColor,line width= 0.6pt,line join=round,line cap=round] ( 71.96, 36.84) rectangle (226.11,132.11);
\end{scope}
\begin{scope}
\path[clip] (  0.00,  0.00) rectangle (505.89,614.29);
\definecolor[named]{drawColor}{rgb}{0.00,0.00,0.00}

\node[text=drawColor,anchor=base east,inner sep=0pt, outer sep=0pt, scale=  0.80] at ( 66.56, 42.03) {0.00};

\node[text=drawColor,anchor=base east,inner sep=0pt, outer sep=0pt, scale=  0.80] at ( 66.56, 59.13) {0.01};

\node[text=drawColor,anchor=base east,inner sep=0pt, outer sep=0pt, scale=  0.80] at ( 66.56, 71.27) {0.05};

\node[text=drawColor,anchor=base east,inner sep=0pt, outer sep=0pt, scale=  0.80] at ( 66.56, 88.45) {0.20};

\node[text=drawColor,anchor=base east,inner sep=0pt, outer sep=0pt, scale=  0.80] at ( 66.56,105.04) {0.50};

\node[text=drawColor,anchor=base east,inner sep=0pt, outer sep=0pt, scale=  0.80] at ( 66.56,114.16) {0.75};

\node[text=drawColor,anchor=base east,inner sep=0pt, outer sep=0pt, scale=  0.80] at ( 66.56,121.42) {1.00};
\end{scope}
\begin{scope}
\path[clip] (  0.00,  0.00) rectangle (505.89,614.29);
\definecolor[named]{drawColor}{rgb}{0.00,0.00,0.00}

\path[draw=drawColor,line width= 0.6pt,line join=round] ( 68.96, 44.78) --
	( 71.96, 44.78);

\path[draw=drawColor,line width= 0.6pt,line join=round] ( 68.96, 61.88) --
	( 71.96, 61.88);

\path[draw=drawColor,line width= 0.6pt,line join=round] ( 68.96, 74.03) --
	( 71.96, 74.03);

\path[draw=drawColor,line width= 0.6pt,line join=round] ( 68.96, 91.21) --
	( 71.96, 91.21);

\path[draw=drawColor,line width= 0.6pt,line join=round] ( 68.96,107.79) --
	( 71.96,107.79);

\path[draw=drawColor,line width= 0.6pt,line join=round] ( 68.96,116.91) --
	( 71.96,116.91);

\path[draw=drawColor,line width= 0.6pt,line join=round] ( 68.96,124.17) --
	( 71.96,124.17);
\end{scope}
\begin{scope}
\path[clip] (  0.00,  0.00) rectangle (505.89,614.29);
\definecolor[named]{drawColor}{rgb}{0.00,0.00,0.00}

\path[draw=drawColor,line width= 0.6pt,line join=round] (137.20, 33.84) --
	(137.20, 36.84);

\path[draw=drawColor,line width= 0.6pt,line join=round] (157.98, 33.84) --
	(157.98, 36.84);

\path[draw=drawColor,line width= 0.6pt,line join=round] (172.55, 33.84) --
	(172.55, 36.84);

\path[draw=drawColor,line width= 0.6pt,line join=round] (184.15, 33.84) --
	(184.15, 36.84);

\path[draw=drawColor,line width= 0.6pt,line join=round] (193.95, 33.84) --
	(193.95, 36.84);

\path[draw=drawColor,line width= 0.6pt,line join=round] (202.51, 33.84) --
	(202.51, 36.84);

\path[draw=drawColor,line width= 0.6pt,line join=round] (210.17, 33.84) --
	(210.17, 36.84);

\path[draw=drawColor,line width= 0.6pt,line join=round] (217.13, 33.84) --
	(217.13, 36.84);

\path[draw=drawColor,line width= 0.6pt,line join=round] (223.53, 33.84) --
	(223.53, 36.84);
\end{scope}
\begin{scope}
\path[clip] (  0.00,  0.00) rectangle (505.89,614.29);
\definecolor[named]{drawColor}{rgb}{0.00,0.00,0.00}

\node[text=drawColor,rotate= 50.00,anchor=base east,inner sep=0pt, outer sep=0pt, scale=  0.80] at (141.42, 27.90) {50};

\node[text=drawColor,rotate= 50.00,anchor=base east,inner sep=0pt, outer sep=0pt, scale=  0.80] at (162.20, 27.90) {100};

\node[text=drawColor,rotate= 50.00,anchor=base east,inner sep=0pt, outer sep=0pt, scale=  0.80] at (176.77, 27.90) {150};

\node[text=drawColor,rotate= 50.00,anchor=base east,inner sep=0pt, outer sep=0pt, scale=  0.80] at (188.37, 27.90) {200};

\node[text=drawColor,rotate= 50.00,anchor=base east,inner sep=0pt, outer sep=0pt, scale=  0.80] at (198.17, 27.90) {250};

\node[text=drawColor,rotate= 50.00,anchor=base east,inner sep=0pt, outer sep=0pt, scale=  0.80] at (206.73, 27.90) {300};

\node[text=drawColor,rotate= 50.00,anchor=base east,inner sep=0pt, outer sep=0pt, scale=  0.80] at (214.39, 27.90) {350};

\node[text=drawColor,rotate= 50.00,anchor=base east,inner sep=0pt, outer sep=0pt, scale=  0.80] at (221.35, 27.90) {400};

\node[text=drawColor,rotate= 50.00,anchor=base east,inner sep=0pt, outer sep=0pt, scale=  0.80] at (227.75, 27.90) {450};
\end{scope}
\begin{scope}
\path[clip] (  0.00,  0.00) rectangle (505.89,614.29);
\definecolor[named]{drawColor}{rgb}{0.00,0.00,0.00}

\node[text=drawColor,anchor=base,inner sep=0pt, outer sep=0pt, scale=  0.80] at (149.04,  8.40) {\# Instances};
\end{scope}
\begin{scope}
\path[clip] (  0.00,  0.00) rectangle (505.89,614.29);
\definecolor[named]{drawColor}{rgb}{0.00,0.00,0.00}

\node[text=drawColor,rotate= 90.00,anchor=base,inner sep=0pt, outer sep=0pt, scale=  0.80] at ( 42.35, 84.48) {1-(Best/Algorithm)};
\end{scope}
\begin{scope}
\path[clip] (  0.00,  0.00) rectangle (505.89,614.29);
\definecolor[named]{drawColor}{rgb}{0.00,0.00,0.00}

\node[text=drawColor,anchor=base,inner sep=0pt, outer sep=0pt, scale=  1.20] at (149.04,139.31) {\SPM};
\end{scope}
\begin{scope}
\path[clip] (  0.00,  0.00) rectangle (505.89,614.29);
\definecolor[named]{drawColor}{rgb}{0.00,0.00,0.00}

\node[text=drawColor,anchor=base west,inner sep=0pt, outer sep=0pt, scale=  1.00] at (297.49, 73.34) {\bfseries Algorithm};
\end{scope}
\begin{scope}
\path[clip] (  0.00,  0.00) rectangle (505.89,614.29);
\definecolor[named]{drawColor}{rgb}{0.89,0.10,0.11}
\definecolor[named]{fillColor}{rgb}{0.89,0.10,0.11}

\path[draw=drawColor,line width= 0.4pt,line join=round,line cap=round,fill=fillColor] (355.87, 97.26) circle (  2.50);
\end{scope}
\begin{scope}
\path[clip] (  0.00,  0.00) rectangle (505.89,614.29);
\definecolor[named]{drawColor}{rgb}{0.65,0.34,0.16}
\definecolor[named]{fillColor}{rgb}{0.65,0.34,0.16}

\path[draw=drawColor,line width= 0.4pt,line join=round,line cap=round,fill=fillColor] (355.87, 87.03) circle (  2.50);
\end{scope}
\begin{scope}
\path[clip] (  0.00,  0.00) rectangle (505.89,614.29);
\definecolor[named]{drawColor}{rgb}{0.22,0.49,0.72}
\definecolor[named]{fillColor}{rgb}{0.22,0.49,0.72}

\path[draw=drawColor,line width= 0.4pt,line join=round,line cap=round,fill=fillColor] (355.87, 76.79) circle (  2.50);
\end{scope}
\begin{scope}
\path[clip] (  0.00,  0.00) rectangle (505.89,614.29);
\definecolor[named]{drawColor}{rgb}{0.30,0.69,0.29}
\definecolor[named]{fillColor}{rgb}{0.30,0.69,0.29}

\path[draw=drawColor,line width= 0.4pt,line join=round,line cap=round,fill=fillColor] (355.87, 66.55) circle (  2.50);
\end{scope}
\begin{scope}
\path[clip] (  0.00,  0.00) rectangle (505.89,614.29);
\definecolor[named]{drawColor}{rgb}{0.60,0.31,0.64}
\definecolor[named]{fillColor}{rgb}{0.60,0.31,0.64}

\path[draw=drawColor,line width= 0.4pt,line join=round,line cap=round,fill=fillColor] (355.87, 56.31) circle (  2.50);
\end{scope}
\begin{scope}
\path[clip] (  0.00,  0.00) rectangle (505.89,614.29);
\definecolor[named]{drawColor}{rgb}{0.00,0.00,0.00}

\node[text=drawColor,anchor=base west,inner sep=0pt, outer sep=0pt, scale=  1.00] at (360.69, 93.82) {\KaHyPar{CA}};
\end{scope}
\begin{scope}
\path[clip] (  0.00,  0.00) rectangle (505.89,614.29);
\definecolor[named]{drawColor}{rgb}{0.00,0.00,0.00}

\node[text=drawColor,anchor=base west,inner sep=0pt, outer sep=0pt, scale=  1.00] at (360.69, 83.58) {\KaHyPar{MF}};
\end{scope}
\begin{scope}
\path[clip] (  0.00,  0.00) rectangle (505.89,614.29);
\definecolor[named]{drawColor}{rgb}{0.00,0.00,0.00}

\node[text=drawColor,anchor=base west,inner sep=0pt, outer sep=0pt, scale=  1.00] at (360.69, 73.34) {\KaHyParConfig{MF}{R1}};
\end{scope}
\begin{scope}
\path[clip] (  0.00,  0.00) rectangle (505.89,614.29);
\definecolor[named]{drawColor}{rgb}{0.00,0.00,0.00}

\node[text=drawColor,anchor=base west,inner sep=0pt, outer sep=0pt, scale=  1.00] at (360.69, 63.11) {\KaHyParConfig{MF}{R1,R2}};
\end{scope}
\begin{scope}
\path[clip] (  0.00,  0.00) rectangle (505.89,614.29);
\definecolor[named]{drawColor}{rgb}{0.00,0.00,0.00}

\node[text=drawColor,anchor=base west,inner sep=0pt, outer sep=0pt, scale=  1.00] at (360.69, 52.87) {\KaHyParConfig{MF}{R1,R2,R3}};
\end{scope}
\end{tikzpicture}
 %
\caption{Min-Cut performance plots comparing \KaHyPar{MF} with \KaHyPar{CA}. 
         The plots are explained in Section \ref{sec:methodology}.}
\label{fig:subset_flow}
\end{figure}  

\begin{table}[ht!]
\renewcommand{\arraystretch}{1.15}
\centering
\begin{tabular}{l|ccccccc}
\toprule
\multirow{2}{*}{Partitioner} & \multicolumn{7}{c}{Running Time $t[s]$} \\
\cmidrule{2-8}
 & \ALL & \DAC & \ISPD & \Primal & \Literal & \Dual & \SPM \\
\midrule%
\csname @@input\endcsname experiments/speed_up_heuristics/subset_flow_running_time.tex 
\bottomrule
\end{tabular} 
\caption{Comparing the average running time of \KaHyPar{MF} with \KaHyPar{CA}.}
\label{tbl:running_time} 
\end{table}

\newpage
\section{Detailed Comparison with other Hypergraph Partitioners}

\begin{table}[ht!]
\renewcommand{\arraystretch}{1.15}
\centering
\begin{tabular}{l|rrrrrrr}
\toprule
\multirow{2}{*}{Partitioner} & \multicolumn{7}{c}{Average $\lambda - 1$} \\
\cmidrule{2-8}
 & \ALL & \DAC & \ISPD & \Primal & \Literal & \Dual & \SPM \\
\midrule%
\csname @@input\endcsname experiments/final_flow/final_flow_km1_per_instance.tex 
\bottomrule
\end{tabular} 
\caption{Comparison of average $(\lambda - 1)$ metric of \KaHyPar{MF} with \KaHyPar{CA} and
         other partitioners on different benchmark types. The results are in percentage 
         relative to \KaHyPar{MF}.}
\label{tbl:full_quality} 
\end{table}

\begin{table}[ht!]
\renewcommand{\arraystretch}{1.15}
\centering
\begin{tabular}{l|rrrrrrr}
\toprule
\multirow{2}{*}{Partitioner} & \multicolumn{7}{c}{Average $\lambda - 1$} \\
\cmidrule{2-8}
 & $k = 2$ & $k = 4$ & $k = 8$ & $k = 16$ & $k = 32$ & $k = 64$ & $k = 128$ \\
\midrule%
\csname @@input\endcsname experiments/final_flow/final_flow_km1_per_k.tex 
\bottomrule
\end{tabular} 
\caption{Comparison of average $(\lambda - 1)$ metric of \KaHyPar{MF} with \KaHyPar{CA} and
         other partitioners for different values of $k$. The results are in percentage 
         relative to \KaHyPar{MF}.}
\label{tbl:full_quality_k} 
\end{table}

\begin{table}[ht!]
\renewcommand{\arraystretch}{1.15}
\centering
\begin{tabular}{l|rrrrrrr}
\toprule
\multirow{2}{*}{Partitioner} & \multicolumn{7}{c}{Running Time $t[s]$} \\
\cmidrule{2-8}
 & $k = 2$ & $k = 4$ & $k = 8$ & $k = 16$ & $k = 32$ & $k = 64$ & $k = 128$ \\
\midrule%
\csname @@input\endcsname experiments/final_flow/final_flow_running_time_per_k.tex
\bottomrule
\end{tabular} 
\caption{Comparing the average running time of \KaHyPar{MF} with \KaHyPar{CA} and
         other partitioners for different values of $k$.}
\label{tbl:full_running_time_k} 
\end{table}

\begin{figure}
\centering
% Created by tikzDevice version 0.6.2-92-0ad2792 on 2017-11-19 13:42:47
% !TEX encoding = UTF-8 Unicode
\begin{tikzpicture}[x=1pt,y=1pt]
\definecolor[named]{fillColor}{rgb}{1.00,1.00,1.00}
\path[use as bounding box,fill=fillColor,fill opacity=0.00] (0,0) rectangle (505.89,650.43);
\begin{scope}
\path[clip] ( 14.17,487.82) rectangle (238.78,650.43);
\definecolor[named]{drawColor}{rgb}{1.00,1.00,1.00}
\definecolor[named]{fillColor}{rgb}{1.00,1.00,1.00}

\path[draw=drawColor,line width= 0.6pt,line join=round,line cap=round,fill=fillColor] ( 14.17,487.82) rectangle (238.78,650.43);
\end{scope}
\begin{scope}
\path[clip] ( 67.36,526.73) rectangle (232.78,628.97);
\definecolor[named]{fillColor}{rgb}{1.00,1.00,1.00}

\path[fill=fillColor] ( 67.36,526.73) rectangle (232.78,628.97);
\definecolor[named]{drawColor}{rgb}{0.98,0.98,0.98}

\path[draw=drawColor,line width= 0.6pt,line join=round] ( 67.36,543.96) --
	(232.78,543.96);

\path[draw=drawColor,line width= 0.6pt,line join=round] ( 67.36,559.25) --
	(232.78,559.25);

\path[draw=drawColor,line width= 0.6pt,line join=round] ( 67.36,569.56) --
	(232.78,569.56);

\path[draw=drawColor,line width= 0.6pt,line join=round] ( 67.36,578.54) --
	(232.78,578.54);

\path[draw=drawColor,line width= 0.6pt,line join=round] ( 67.36,589.85) --
	(232.78,589.85);

\path[draw=drawColor,line width= 0.6pt,line join=round] ( 67.36,600.58) --
	(232.78,600.58);

\path[draw=drawColor,line width= 0.6pt,line join=round] ( 67.36,608.52) --
	(232.78,608.52);

\path[draw=drawColor,line width= 0.6pt,line join=round] ( 67.36,615.02) --
	(232.78,615.02);

\path[draw=drawColor,line width= 0.6pt,line join=round] ( 67.36,626.93) --
	(232.78,626.93);

\path[draw=drawColor,line width= 0.6pt,line join=round] (129.06,526.73) --
	(129.06,628.97);

\path[draw=drawColor,line width= 0.6pt,line join=round] (142.49,526.73) --
	(142.49,628.97);

\path[draw=drawColor,line width= 0.6pt,line join=round] (169.33,526.73) --
	(169.33,628.97);

\path[draw=drawColor,line width= 0.6pt,line join=round] (192.17,526.73) --
	(192.17,628.97);

\path[draw=drawColor,line width= 0.6pt,line join=round] (209.08,526.73) --
	(209.08,628.97);

\path[draw=drawColor,line width= 0.6pt,line join=round] (222.90,526.73) --
	(222.90,628.97);
\definecolor[named]{drawColor}{rgb}{0.75,0.75,0.75}

\path[draw=drawColor,line width= 0.6pt,dash pattern=on 1pt off 3pt ,line join=round] ( 67.36,535.03) --
	(232.78,535.03);

\path[draw=drawColor,line width= 0.6pt,dash pattern=on 1pt off 3pt ,line join=round] ( 67.36,552.90) --
	(232.78,552.90);

\path[draw=drawColor,line width= 0.6pt,dash pattern=on 1pt off 3pt ,line join=round] ( 67.36,565.59) --
	(232.78,565.59);

\path[draw=drawColor,line width= 0.6pt,dash pattern=on 1pt off 3pt ,line join=round] ( 67.36,573.54) --
	(232.78,573.54);

\path[draw=drawColor,line width= 0.6pt,dash pattern=on 1pt off 3pt ,line join=round] ( 67.36,583.54) --
	(232.78,583.54);

\path[draw=drawColor,line width= 0.6pt,dash pattern=on 1pt off 3pt ,line join=round] ( 67.36,596.16) --
	(232.78,596.16);

\path[draw=drawColor,line width= 0.6pt,dash pattern=on 1pt off 3pt ,line join=round] ( 67.36,605.00) --
	(232.78,605.00);

\path[draw=drawColor,line width= 0.6pt,dash pattern=on 1pt off 3pt ,line join=round] ( 67.36,612.04) --
	(232.78,612.04);

\path[draw=drawColor,line width= 0.6pt,dash pattern=on 1pt off 3pt ,line join=round] ( 67.36,617.99) --
	(232.78,617.99);

\path[draw=drawColor,line width= 0.6pt,dash pattern=on 1pt off 3pt ,line join=round] (155.91,526.73) --
	(155.91,628.97);

\path[draw=drawColor,line width= 0.6pt,dash pattern=on 1pt off 3pt ,line join=round] (182.75,526.73) --
	(182.75,628.97);

\path[draw=drawColor,line width= 0.6pt,dash pattern=on 1pt off 3pt ,line join=round] (201.58,526.73) --
	(201.58,628.97);

\path[draw=drawColor,line width= 0.6pt,dash pattern=on 1pt off 3pt ,line join=round] (216.58,526.73) --
	(216.58,628.97);

\path[draw=drawColor,line width= 0.6pt,dash pattern=on 1pt off 3pt ,line join=round] (229.23,526.73) --
	(229.23,628.97);
\definecolor[named]{drawColor}{rgb}{0.89,0.10,0.11}
\definecolor[named]{fillColor}{rgb}{0.89,0.10,0.11}

\path[draw=drawColor,line width= 0.4pt,line join=round,line cap=round,fill=fillColor] ( 74.88,597.16) circle (  1.16);

\path[draw=drawColor,line width= 0.4pt,line join=round,line cap=round,fill=fillColor] ( 80.66,596.50) circle (  1.16);

\path[draw=drawColor,line width= 0.4pt,line join=round,line cap=round,fill=fillColor] ( 84.72,594.74) circle (  1.16);

\path[draw=drawColor,line width= 0.4pt,line join=round,line cap=round,fill=fillColor] ( 87.95,594.15) circle (  1.16);

\path[draw=drawColor,line width= 0.4pt,line join=round,line cap=round,fill=fillColor] ( 90.68,592.47) circle (  1.16);

\path[draw=drawColor,line width= 0.4pt,line join=round,line cap=round,fill=fillColor] ( 93.06,591.28) circle (  1.16);

\path[draw=drawColor,line width= 0.4pt,line join=round,line cap=round,fill=fillColor] ( 95.19,591.08) circle (  1.16);

\path[draw=drawColor,line width= 0.4pt,line join=round,line cap=round,fill=fillColor] ( 97.13,589.72) circle (  1.16);

\path[draw=drawColor,line width= 0.4pt,line join=round,line cap=round,fill=fillColor] ( 98.91,588.83) circle (  1.16);

\path[draw=drawColor,line width= 0.4pt,line join=round,line cap=round,fill=fillColor] (100.57,588.47) circle (  1.16);

\path[draw=drawColor,line width= 0.4pt,line join=round,line cap=round,fill=fillColor] (102.11,588.21) circle (  1.16);

\path[draw=drawColor,line width= 0.4pt,line join=round,line cap=round,fill=fillColor] (103.57,588.17) circle (  1.16);

\path[draw=drawColor,line width= 0.4pt,line join=round,line cap=round,fill=fillColor] (104.95,587.70) circle (  1.16);

\path[draw=drawColor,line width= 0.4pt,line join=round,line cap=round,fill=fillColor] (106.26,587.62) circle (  1.16);

\path[draw=drawColor,line width= 0.4pt,line join=round,line cap=round,fill=fillColor] (107.50,587.61) circle (  1.16);

\path[draw=drawColor,line width= 0.4pt,line join=round,line cap=round,fill=fillColor] (108.70,586.94) circle (  1.16);

\path[draw=drawColor,line width= 0.4pt,line join=round,line cap=round,fill=fillColor] (109.84,586.93) circle (  1.16);

\path[draw=drawColor,line width= 0.4pt,line join=round,line cap=round,fill=fillColor] (110.94,585.12) circle (  1.16);

\path[draw=drawColor,line width= 0.4pt,line join=round,line cap=round,fill=fillColor] (112.00,585.02) circle (  1.16);

\path[draw=drawColor,line width= 0.4pt,line join=round,line cap=round,fill=fillColor] (113.03,584.79) circle (  1.16);

\path[draw=drawColor,line width= 0.4pt,line join=round,line cap=round,fill=fillColor] (114.02,584.54) circle (  1.16);

\path[draw=drawColor,line width= 0.4pt,line join=round,line cap=round,fill=fillColor] (114.98,583.40) circle (  1.16);

\path[draw=drawColor,line width= 0.4pt,line join=round,line cap=round,fill=fillColor] (115.91,580.68) circle (  1.16);

\path[draw=drawColor,line width= 0.4pt,line join=round,line cap=round,fill=fillColor] (116.81,580.62) circle (  1.16);

\path[draw=drawColor,line width= 0.4pt,line join=round,line cap=round,fill=fillColor] (117.69,579.20) circle (  1.16);

\path[draw=drawColor,line width= 0.4pt,line join=round,line cap=round,fill=fillColor] (118.55,578.81) circle (  1.16);

\path[draw=drawColor,line width= 0.4pt,line join=round,line cap=round,fill=fillColor] (119.38,577.82) circle (  1.16);

\path[draw=drawColor,line width= 0.4pt,line join=round,line cap=round,fill=fillColor] (120.20,577.71) circle (  1.16);

\path[draw=drawColor,line width= 0.4pt,line join=round,line cap=round,fill=fillColor] (120.99,577.44) circle (  1.16);

\path[draw=drawColor,line width= 0.4pt,line join=round,line cap=round,fill=fillColor] (121.77,576.69) circle (  1.16);

\path[draw=drawColor,line width= 0.4pt,line join=round,line cap=round,fill=fillColor] (122.53,576.18) circle (  1.16);

\path[draw=drawColor,line width= 0.4pt,line join=round,line cap=round,fill=fillColor] (123.27,575.75) circle (  1.16);

\path[draw=drawColor,line width= 0.4pt,line join=round,line cap=round,fill=fillColor] (124.00,575.59) circle (  1.16);

\path[draw=drawColor,line width= 0.4pt,line join=round,line cap=round,fill=fillColor] (124.71,575.52) circle (  1.16);

\path[draw=drawColor,line width= 0.4pt,line join=round,line cap=round,fill=fillColor] (125.41,575.17) circle (  1.16);

\path[draw=drawColor,line width= 0.4pt,line join=round,line cap=round,fill=fillColor] (126.10,574.91) circle (  1.16);

\path[draw=drawColor,line width= 0.4pt,line join=round,line cap=round,fill=fillColor] (126.77,574.72) circle (  1.16);

\path[draw=drawColor,line width= 0.4pt,line join=round,line cap=round,fill=fillColor] (127.44,574.59) circle (  1.16);

\path[draw=drawColor,line width= 0.4pt,line join=round,line cap=round,fill=fillColor] (128.09,573.88) circle (  1.16);

\path[draw=drawColor,line width= 0.4pt,line join=round,line cap=round,fill=fillColor] (128.73,573.72) circle (  1.16);

\path[draw=drawColor,line width= 0.4pt,line join=round,line cap=round,fill=fillColor] (129.35,573.39) circle (  1.16);

\path[draw=drawColor,line width= 0.4pt,line join=round,line cap=round,fill=fillColor] (129.97,573.35) circle (  1.16);

\path[draw=drawColor,line width= 0.4pt,line join=round,line cap=round,fill=fillColor] (130.58,573.18) circle (  1.16);

\path[draw=drawColor,line width= 0.4pt,line join=round,line cap=round,fill=fillColor] (131.18,573.00) circle (  1.16);

\path[draw=drawColor,line width= 0.4pt,line join=round,line cap=round,fill=fillColor] (131.77,572.62) circle (  1.16);

\path[draw=drawColor,line width= 0.4pt,line join=round,line cap=round,fill=fillColor] (132.35,572.57) circle (  1.16);

\path[draw=drawColor,line width= 0.4pt,line join=round,line cap=round,fill=fillColor] (132.93,571.90) circle (  1.16);

\path[draw=drawColor,line width= 0.4pt,line join=round,line cap=round,fill=fillColor] (133.49,571.83) circle (  1.16);

\path[draw=drawColor,line width= 0.4pt,line join=round,line cap=round,fill=fillColor] (134.05,571.75) circle (  1.16);

\path[draw=drawColor,line width= 0.4pt,line join=round,line cap=round,fill=fillColor] (134.60,571.49) circle (  1.16);

\path[draw=drawColor,line width= 0.4pt,line join=round,line cap=round,fill=fillColor] (135.14,571.34) circle (  1.16);

\path[draw=drawColor,line width= 0.4pt,line join=round,line cap=round,fill=fillColor] (135.68,571.33) circle (  1.16);

\path[draw=drawColor,line width= 0.4pt,line join=round,line cap=round,fill=fillColor] (136.21,571.28) circle (  1.16);

\path[draw=drawColor,line width= 0.4pt,line join=round,line cap=round,fill=fillColor] (136.73,571.25) circle (  1.16);

\path[draw=drawColor,line width= 0.4pt,line join=round,line cap=round,fill=fillColor] (137.25,571.06) circle (  1.16);

\path[draw=drawColor,line width= 0.4pt,line join=round,line cap=round,fill=fillColor] (137.76,570.70) circle (  1.16);

\path[draw=drawColor,line width= 0.4pt,line join=round,line cap=round,fill=fillColor] (138.26,569.92) circle (  1.16);

\path[draw=drawColor,line width= 0.4pt,line join=round,line cap=round,fill=fillColor] (138.76,569.91) circle (  1.16);

\path[draw=drawColor,line width= 0.4pt,line join=round,line cap=round,fill=fillColor] (139.25,569.91) circle (  1.16);

\path[draw=drawColor,line width= 0.4pt,line join=round,line cap=round,fill=fillColor] (139.74,569.45) circle (  1.16);

\path[draw=drawColor,line width= 0.4pt,line join=round,line cap=round,fill=fillColor] (140.22,568.65) circle (  1.16);

\path[draw=drawColor,line width= 0.4pt,line join=round,line cap=round,fill=fillColor] (140.70,568.35) circle (  1.16);

\path[draw=drawColor,line width= 0.4pt,line join=round,line cap=round,fill=fillColor] (141.17,568.34) circle (  1.16);

\path[draw=drawColor,line width= 0.4pt,line join=round,line cap=round,fill=fillColor] (141.63,568.26) circle (  1.16);

\path[draw=drawColor,line width= 0.4pt,line join=round,line cap=round,fill=fillColor] (142.09,568.22) circle (  1.16);

\path[draw=drawColor,line width= 0.4pt,line join=round,line cap=round,fill=fillColor] (142.55,568.19) circle (  1.16);

\path[draw=drawColor,line width= 0.4pt,line join=round,line cap=round,fill=fillColor] (143.00,568.19) circle (  1.16);

\path[draw=drawColor,line width= 0.4pt,line join=round,line cap=round,fill=fillColor] (143.45,568.01) circle (  1.16);

\path[draw=drawColor,line width= 0.4pt,line join=round,line cap=round,fill=fillColor] (143.89,568.00) circle (  1.16);

\path[draw=drawColor,line width= 0.4pt,line join=round,line cap=round,fill=fillColor] (144.33,567.91) circle (  1.16);

\path[draw=drawColor,line width= 0.4pt,line join=round,line cap=round,fill=fillColor] (144.77,567.87) circle (  1.16);

\path[draw=drawColor,line width= 0.4pt,line join=round,line cap=round,fill=fillColor] (145.20,567.75) circle (  1.16);

\path[draw=drawColor,line width= 0.4pt,line join=round,line cap=round,fill=fillColor] (145.62,567.40) circle (  1.16);

\path[draw=drawColor,line width= 0.4pt,line join=round,line cap=round,fill=fillColor] (146.05,567.14) circle (  1.16);

\path[draw=drawColor,line width= 0.4pt,line join=round,line cap=round,fill=fillColor] (146.46,567.12) circle (  1.16);

\path[draw=drawColor,line width= 0.4pt,line join=round,line cap=round,fill=fillColor] (146.88,566.83) circle (  1.16);

\path[draw=drawColor,line width= 0.4pt,line join=round,line cap=round,fill=fillColor] (147.29,566.48) circle (  1.16);

\path[draw=drawColor,line width= 0.4pt,line join=round,line cap=round,fill=fillColor] (147.70,566.12) circle (  1.16);

\path[draw=drawColor,line width= 0.4pt,line join=round,line cap=round,fill=fillColor] (148.10,565.90) circle (  1.16);

\path[draw=drawColor,line width= 0.4pt,line join=round,line cap=round,fill=fillColor] (148.50,565.48) circle (  1.16);

\path[draw=drawColor,line width= 0.4pt,line join=round,line cap=round,fill=fillColor] (148.90,565.37) circle (  1.16);

\path[draw=drawColor,line width= 0.4pt,line join=round,line cap=round,fill=fillColor] (149.30,565.08) circle (  1.16);

\path[draw=drawColor,line width= 0.4pt,line join=round,line cap=round,fill=fillColor] (149.69,564.99) circle (  1.16);

\path[draw=drawColor,line width= 0.4pt,line join=round,line cap=round,fill=fillColor] (150.08,564.93) circle (  1.16);

\path[draw=drawColor,line width= 0.4pt,line join=round,line cap=round,fill=fillColor] (150.46,564.86) circle (  1.16);

\path[draw=drawColor,line width= 0.4pt,line join=round,line cap=round,fill=fillColor] (150.84,564.49) circle (  1.16);

\path[draw=drawColor,line width= 0.4pt,line join=round,line cap=round,fill=fillColor] (151.22,564.42) circle (  1.16);

\path[draw=drawColor,line width= 0.4pt,line join=round,line cap=round,fill=fillColor] (151.60,564.25) circle (  1.16);

\path[draw=drawColor,line width= 0.4pt,line join=round,line cap=round,fill=fillColor] (151.97,564.22) circle (  1.16);

\path[draw=drawColor,line width= 0.4pt,line join=round,line cap=round,fill=fillColor] (152.34,564.20) circle (  1.16);

\path[draw=drawColor,line width= 0.4pt,line join=round,line cap=round,fill=fillColor] (152.71,564.06) circle (  1.16);

\path[draw=drawColor,line width= 0.4pt,line join=round,line cap=round,fill=fillColor] (153.08,563.82) circle (  1.16);

\path[draw=drawColor,line width= 0.4pt,line join=round,line cap=round,fill=fillColor] (153.44,563.63) circle (  1.16);

\path[draw=drawColor,line width= 0.4pt,line join=round,line cap=round,fill=fillColor] (153.80,563.50) circle (  1.16);

\path[draw=drawColor,line width= 0.4pt,line join=round,line cap=round,fill=fillColor] (154.16,563.40) circle (  1.16);

\path[draw=drawColor,line width= 0.4pt,line join=round,line cap=round,fill=fillColor] (154.51,563.19) circle (  1.16);

\path[draw=drawColor,line width= 0.4pt,line join=round,line cap=round,fill=fillColor] (154.86,563.15) circle (  1.16);

\path[draw=drawColor,line width= 0.4pt,line join=round,line cap=round,fill=fillColor] (155.22,562.90) circle (  1.16);

\path[draw=drawColor,line width= 0.4pt,line join=round,line cap=round,fill=fillColor] (155.56,562.85) circle (  1.16);

\path[draw=drawColor,line width= 0.4pt,line join=round,line cap=round,fill=fillColor] (155.91,562.68) circle (  1.16);

\path[draw=drawColor,line width= 0.4pt,line join=round,line cap=round,fill=fillColor] (156.25,562.68) circle (  1.16);

\path[draw=drawColor,line width= 0.4pt,line join=round,line cap=round,fill=fillColor] (156.59,562.31) circle (  1.16);

\path[draw=drawColor,line width= 0.4pt,line join=round,line cap=round,fill=fillColor] (156.93,562.29) circle (  1.16);

\path[draw=drawColor,line width= 0.4pt,line join=round,line cap=round,fill=fillColor] (157.27,562.18) circle (  1.16);

\path[draw=drawColor,line width= 0.4pt,line join=round,line cap=round,fill=fillColor] (157.60,562.03) circle (  1.16);

\path[draw=drawColor,line width= 0.4pt,line join=round,line cap=round,fill=fillColor] (157.93,561.99) circle (  1.16);

\path[draw=drawColor,line width= 0.4pt,line join=round,line cap=round,fill=fillColor] (158.26,561.89) circle (  1.16);

\path[draw=drawColor,line width= 0.4pt,line join=round,line cap=round,fill=fillColor] (158.59,561.71) circle (  1.16);

\path[draw=drawColor,line width= 0.4pt,line join=round,line cap=round,fill=fillColor] (158.92,561.63) circle (  1.16);

\path[draw=drawColor,line width= 0.4pt,line join=round,line cap=round,fill=fillColor] (159.24,561.52) circle (  1.16);

\path[draw=drawColor,line width= 0.4pt,line join=round,line cap=round,fill=fillColor] (159.56,561.48) circle (  1.16);

\path[draw=drawColor,line width= 0.4pt,line join=round,line cap=round,fill=fillColor] (159.88,561.42) circle (  1.16);

\path[draw=drawColor,line width= 0.4pt,line join=round,line cap=round,fill=fillColor] (160.20,561.40) circle (  1.16);

\path[draw=drawColor,line width= 0.4pt,line join=round,line cap=round,fill=fillColor] (160.52,561.30) circle (  1.16);

\path[draw=drawColor,line width= 0.4pt,line join=round,line cap=round,fill=fillColor] (160.83,561.22) circle (  1.16);

\path[draw=drawColor,line width= 0.4pt,line join=round,line cap=round,fill=fillColor] (161.15,560.91) circle (  1.16);

\path[draw=drawColor,line width= 0.4pt,line join=round,line cap=round,fill=fillColor] (161.46,560.89) circle (  1.16);

\path[draw=drawColor,line width= 0.4pt,line join=round,line cap=round,fill=fillColor] (161.77,560.84) circle (  1.16);

\path[draw=drawColor,line width= 0.4pt,line join=round,line cap=round,fill=fillColor] (162.07,560.68) circle (  1.16);

\path[draw=drawColor,line width= 0.4pt,line join=round,line cap=round,fill=fillColor] (162.38,560.17) circle (  1.16);

\path[draw=drawColor,line width= 0.4pt,line join=round,line cap=round,fill=fillColor] (162.68,560.11) circle (  1.16);

\path[draw=drawColor,line width= 0.4pt,line join=round,line cap=round,fill=fillColor] (162.99,560.08) circle (  1.16);

\path[draw=drawColor,line width= 0.4pt,line join=round,line cap=round,fill=fillColor] (163.29,560.00) circle (  1.16);

\path[draw=drawColor,line width= 0.4pt,line join=round,line cap=round,fill=fillColor] (163.59,559.95) circle (  1.16);

\path[draw=drawColor,line width= 0.4pt,line join=round,line cap=round,fill=fillColor] (163.88,559.80) circle (  1.16);

\path[draw=drawColor,line width= 0.4pt,line join=round,line cap=round,fill=fillColor] (164.18,559.64) circle (  1.16);

\path[draw=drawColor,line width= 0.4pt,line join=round,line cap=round,fill=fillColor] (164.47,559.57) circle (  1.16);

\path[draw=drawColor,line width= 0.4pt,line join=round,line cap=round,fill=fillColor] (164.77,559.45) circle (  1.16);

\path[draw=drawColor,line width= 0.4pt,line join=round,line cap=round,fill=fillColor] (165.06,559.45) circle (  1.16);

\path[draw=drawColor,line width= 0.4pt,line join=round,line cap=round,fill=fillColor] (165.35,559.41) circle (  1.16);

\path[draw=drawColor,line width= 0.4pt,line join=round,line cap=round,fill=fillColor] (165.64,559.18) circle (  1.16);

\path[draw=drawColor,line width= 0.4pt,line join=round,line cap=round,fill=fillColor] (165.92,559.09) circle (  1.16);

\path[draw=drawColor,line width= 0.4pt,line join=round,line cap=round,fill=fillColor] (166.21,558.68) circle (  1.16);

\path[draw=drawColor,line width= 0.4pt,line join=round,line cap=round,fill=fillColor] (166.49,558.67) circle (  1.16);

\path[draw=drawColor,line width= 0.4pt,line join=round,line cap=round,fill=fillColor] (166.77,558.62) circle (  1.16);

\path[draw=drawColor,line width= 0.4pt,line join=round,line cap=round,fill=fillColor] (167.06,558.61) circle (  1.16);

\path[draw=drawColor,line width= 0.4pt,line join=round,line cap=round,fill=fillColor] (167.34,558.53) circle (  1.16);

\path[draw=drawColor,line width= 0.4pt,line join=round,line cap=round,fill=fillColor] (167.61,558.41) circle (  1.16);

\path[draw=drawColor,line width= 0.4pt,line join=round,line cap=round,fill=fillColor] (167.89,558.14) circle (  1.16);

\path[draw=drawColor,line width= 0.4pt,line join=round,line cap=round,fill=fillColor] (168.17,558.03) circle (  1.16);

\path[draw=drawColor,line width= 0.4pt,line join=round,line cap=round,fill=fillColor] (168.44,557.86) circle (  1.16);

\path[draw=drawColor,line width= 0.4pt,line join=round,line cap=round,fill=fillColor] (168.71,557.74) circle (  1.16);

\path[draw=drawColor,line width= 0.4pt,line join=round,line cap=round,fill=fillColor] (168.99,557.45) circle (  1.16);

\path[draw=drawColor,line width= 0.4pt,line join=round,line cap=round,fill=fillColor] (169.26,557.22) circle (  1.16);

\path[draw=drawColor,line width= 0.4pt,line join=round,line cap=round,fill=fillColor] (169.53,557.12) circle (  1.16);

\path[draw=drawColor,line width= 0.4pt,line join=round,line cap=round,fill=fillColor] (169.79,557.10) circle (  1.16);

\path[draw=drawColor,line width= 0.4pt,line join=round,line cap=round,fill=fillColor] (170.06,557.09) circle (  1.16);

\path[draw=drawColor,line width= 0.4pt,line join=round,line cap=round,fill=fillColor] (170.33,557.01) circle (  1.16);

\path[draw=drawColor,line width= 0.4pt,line join=round,line cap=round,fill=fillColor] (170.59,557.00) circle (  1.16);

\path[draw=drawColor,line width= 0.4pt,line join=round,line cap=round,fill=fillColor] (170.85,556.98) circle (  1.16);

\path[draw=drawColor,line width= 0.4pt,line join=round,line cap=round,fill=fillColor] (171.12,556.94) circle (  1.16);

\path[draw=drawColor,line width= 0.4pt,line join=round,line cap=round,fill=fillColor] (171.38,556.84) circle (  1.16);

\path[draw=drawColor,line width= 0.4pt,line join=round,line cap=round,fill=fillColor] (171.64,556.47) circle (  1.16);

\path[draw=drawColor,line width= 0.4pt,line join=round,line cap=round,fill=fillColor] (171.90,556.24) circle (  1.16);

\path[draw=drawColor,line width= 0.4pt,line join=round,line cap=round,fill=fillColor] (172.15,556.12) circle (  1.16);

\path[draw=drawColor,line width= 0.4pt,line join=round,line cap=round,fill=fillColor] (172.41,556.10) circle (  1.16);

\path[draw=drawColor,line width= 0.4pt,line join=round,line cap=round,fill=fillColor] (172.67,556.09) circle (  1.16);

\path[draw=drawColor,line width= 0.4pt,line join=round,line cap=round,fill=fillColor] (172.92,556.09) circle (  1.16);

\path[draw=drawColor,line width= 0.4pt,line join=round,line cap=round,fill=fillColor] (173.17,556.06) circle (  1.16);

\path[draw=drawColor,line width= 0.4pt,line join=round,line cap=round,fill=fillColor] (173.43,555.92) circle (  1.16);

\path[draw=drawColor,line width= 0.4pt,line join=round,line cap=round,fill=fillColor] (173.68,555.85) circle (  1.16);

\path[draw=drawColor,line width= 0.4pt,line join=round,line cap=round,fill=fillColor] (173.93,555.70) circle (  1.16);

\path[draw=drawColor,line width= 0.4pt,line join=round,line cap=round,fill=fillColor] (174.18,555.47) circle (  1.16);

\path[draw=drawColor,line width= 0.4pt,line join=round,line cap=round,fill=fillColor] (174.42,555.30) circle (  1.16);

\path[draw=drawColor,line width= 0.4pt,line join=round,line cap=round,fill=fillColor] (174.67,555.23) circle (  1.16);

\path[draw=drawColor,line width= 0.4pt,line join=round,line cap=round,fill=fillColor] (174.92,555.17) circle (  1.16);

\path[draw=drawColor,line width= 0.4pt,line join=round,line cap=round,fill=fillColor] (175.16,555.13) circle (  1.16);

\path[draw=drawColor,line width= 0.4pt,line join=round,line cap=round,fill=fillColor] (175.41,555.09) circle (  1.16);

\path[draw=drawColor,line width= 0.4pt,line join=round,line cap=round,fill=fillColor] (175.65,555.06) circle (  1.16);

\path[draw=drawColor,line width= 0.4pt,line join=round,line cap=round,fill=fillColor] (175.89,555.06) circle (  1.16);

\path[draw=drawColor,line width= 0.4pt,line join=round,line cap=round,fill=fillColor] (176.13,554.97) circle (  1.16);

\path[draw=drawColor,line width= 0.4pt,line join=round,line cap=round,fill=fillColor] (176.37,554.62) circle (  1.16);

\path[draw=drawColor,line width= 0.4pt,line join=round,line cap=round,fill=fillColor] (176.61,554.61) circle (  1.16);

\path[draw=drawColor,line width= 0.4pt,line join=round,line cap=round,fill=fillColor] (176.85,554.52) circle (  1.16);

\path[draw=drawColor,line width= 0.4pt,line join=round,line cap=round,fill=fillColor] (177.09,554.44) circle (  1.16);

\path[draw=drawColor,line width= 0.4pt,line join=round,line cap=round,fill=fillColor] (177.32,554.35) circle (  1.16);

\path[draw=drawColor,line width= 0.4pt,line join=round,line cap=round,fill=fillColor] (177.56,554.28) circle (  1.16);

\path[draw=drawColor,line width= 0.4pt,line join=round,line cap=round,fill=fillColor] (177.80,554.12) circle (  1.16);

\path[draw=drawColor,line width= 0.4pt,line join=round,line cap=round,fill=fillColor] (178.03,554.08) circle (  1.16);

\path[draw=drawColor,line width= 0.4pt,line join=round,line cap=round,fill=fillColor] (178.26,554.04) circle (  1.16);

\path[draw=drawColor,line width= 0.4pt,line join=round,line cap=round,fill=fillColor] (178.49,554.03) circle (  1.16);

\path[draw=drawColor,line width= 0.4pt,line join=round,line cap=round,fill=fillColor] (178.73,554.02) circle (  1.16);

\path[draw=drawColor,line width= 0.4pt,line join=round,line cap=round,fill=fillColor] (178.96,553.61) circle (  1.16);

\path[draw=drawColor,line width= 0.4pt,line join=round,line cap=round,fill=fillColor] (179.19,553.59) circle (  1.16);

\path[draw=drawColor,line width= 0.4pt,line join=round,line cap=round,fill=fillColor] (179.41,553.53) circle (  1.16);

\path[draw=drawColor,line width= 0.4pt,line join=round,line cap=round,fill=fillColor] (179.64,553.34) circle (  1.16);

\path[draw=drawColor,line width= 0.4pt,line join=round,line cap=round,fill=fillColor] (179.87,553.30) circle (  1.16);

\path[draw=drawColor,line width= 0.4pt,line join=round,line cap=round,fill=fillColor] (180.10,553.27) circle (  1.16);

\path[draw=drawColor,line width= 0.4pt,line join=round,line cap=round,fill=fillColor] (180.32,553.13) circle (  1.16);

\path[draw=drawColor,line width= 0.4pt,line join=round,line cap=round,fill=fillColor] (180.55,553.04) circle (  1.16);

\path[draw=drawColor,line width= 0.4pt,line join=round,line cap=round,fill=fillColor] (180.77,552.94) circle (  1.16);

\path[draw=drawColor,line width= 0.4pt,line join=round,line cap=round,fill=fillColor] (180.99,552.86) circle (  1.16);

\path[draw=drawColor,line width= 0.4pt,line join=round,line cap=round,fill=fillColor] (181.22,552.78) circle (  1.16);

\path[draw=drawColor,line width= 0.4pt,line join=round,line cap=round,fill=fillColor] (181.44,552.71) circle (  1.16);

\path[draw=drawColor,line width= 0.4pt,line join=round,line cap=round,fill=fillColor] (181.66,552.65) circle (  1.16);

\path[draw=drawColor,line width= 0.4pt,line join=round,line cap=round,fill=fillColor] (181.88,552.48) circle (  1.16);

\path[draw=drawColor,line width= 0.4pt,line join=round,line cap=round,fill=fillColor] (182.10,552.45) circle (  1.16);

\path[draw=drawColor,line width= 0.4pt,line join=round,line cap=round,fill=fillColor] (182.32,552.43) circle (  1.16);

\path[draw=drawColor,line width= 0.4pt,line join=round,line cap=round,fill=fillColor] (182.54,552.32) circle (  1.16);

\path[draw=drawColor,line width= 0.4pt,line join=round,line cap=round,fill=fillColor] (182.75,552.31) circle (  1.16);

\path[draw=drawColor,line width= 0.4pt,line join=round,line cap=round,fill=fillColor] (182.97,552.30) circle (  1.16);

\path[draw=drawColor,line width= 0.4pt,line join=round,line cap=round,fill=fillColor] (183.19,552.29) circle (  1.16);

\path[draw=drawColor,line width= 0.4pt,line join=round,line cap=round,fill=fillColor] (183.40,552.19) circle (  1.16);

\path[draw=drawColor,line width= 0.4pt,line join=round,line cap=round,fill=fillColor] (183.61,552.10) circle (  1.16);

\path[draw=drawColor,line width= 0.4pt,line join=round,line cap=round,fill=fillColor] (183.83,552.07) circle (  1.16);

\path[draw=drawColor,line width= 0.4pt,line join=round,line cap=round,fill=fillColor] (184.04,552.01) circle (  1.16);

\path[draw=drawColor,line width= 0.4pt,line join=round,line cap=round,fill=fillColor] (184.25,552.00) circle (  1.16);

\path[draw=drawColor,line width= 0.4pt,line join=round,line cap=round,fill=fillColor] (184.47,551.99) circle (  1.16);

\path[draw=drawColor,line width= 0.4pt,line join=round,line cap=round,fill=fillColor] (184.68,551.99) circle (  1.16);

\path[draw=drawColor,line width= 0.4pt,line join=round,line cap=round,fill=fillColor] (184.89,551.99) circle (  1.16);

\path[draw=drawColor,line width= 0.4pt,line join=round,line cap=round,fill=fillColor] (185.10,551.82) circle (  1.16);

\path[draw=drawColor,line width= 0.4pt,line join=round,line cap=round,fill=fillColor] (185.31,551.69) circle (  1.16);

\path[draw=drawColor,line width= 0.4pt,line join=round,line cap=round,fill=fillColor] (185.51,551.69) circle (  1.16);

\path[draw=drawColor,line width= 0.4pt,line join=round,line cap=round,fill=fillColor] (185.72,551.63) circle (  1.16);

\path[draw=drawColor,line width= 0.4pt,line join=round,line cap=round,fill=fillColor] (185.93,551.41) circle (  1.16);

\path[draw=drawColor,line width= 0.4pt,line join=round,line cap=round,fill=fillColor] (186.13,551.37) circle (  1.16);

\path[draw=drawColor,line width= 0.4pt,line join=round,line cap=round,fill=fillColor] (186.34,551.29) circle (  1.16);

\path[draw=drawColor,line width= 0.4pt,line join=round,line cap=round,fill=fillColor] (186.55,551.25) circle (  1.16);

\path[draw=drawColor,line width= 0.4pt,line join=round,line cap=round,fill=fillColor] (186.75,550.98) circle (  1.16);

\path[draw=drawColor,line width= 0.4pt,line join=round,line cap=round,fill=fillColor] (186.95,550.88) circle (  1.16);

\path[draw=drawColor,line width= 0.4pt,line join=round,line cap=round,fill=fillColor] (187.16,550.69) circle (  1.16);

\path[draw=drawColor,line width= 0.4pt,line join=round,line cap=round,fill=fillColor] (187.36,550.49) circle (  1.16);

\path[draw=drawColor,line width= 0.4pt,line join=round,line cap=round,fill=fillColor] (187.56,550.37) circle (  1.16);

\path[draw=drawColor,line width= 0.4pt,line join=round,line cap=round,fill=fillColor] (187.76,550.35) circle (  1.16);

\path[draw=drawColor,line width= 0.4pt,line join=round,line cap=round,fill=fillColor] (187.96,550.24) circle (  1.16);

\path[draw=drawColor,line width= 0.4pt,line join=round,line cap=round,fill=fillColor] (188.16,550.17) circle (  1.16);

\path[draw=drawColor,line width= 0.4pt,line join=round,line cap=round,fill=fillColor] (188.36,550.12) circle (  1.16);

\path[draw=drawColor,line width= 0.4pt,line join=round,line cap=round,fill=fillColor] (188.56,550.01) circle (  1.16);

\path[draw=drawColor,line width= 0.4pt,line join=round,line cap=round,fill=fillColor] (188.76,549.90) circle (  1.16);

\path[draw=drawColor,line width= 0.4pt,line join=round,line cap=round,fill=fillColor] (188.96,549.84) circle (  1.16);

\path[draw=drawColor,line width= 0.4pt,line join=round,line cap=round,fill=fillColor] (189.16,549.82) circle (  1.16);

\path[draw=drawColor,line width= 0.4pt,line join=round,line cap=round,fill=fillColor] (189.35,549.67) circle (  1.16);

\path[draw=drawColor,line width= 0.4pt,line join=round,line cap=round,fill=fillColor] (189.55,549.67) circle (  1.16);

\path[draw=drawColor,line width= 0.4pt,line join=round,line cap=round,fill=fillColor] (189.74,549.58) circle (  1.16);

\path[draw=drawColor,line width= 0.4pt,line join=round,line cap=round,fill=fillColor] (189.94,549.56) circle (  1.16);

\path[draw=drawColor,line width= 0.4pt,line join=round,line cap=round,fill=fillColor] (190.13,549.53) circle (  1.16);

\path[draw=drawColor,line width= 0.4pt,line join=round,line cap=round,fill=fillColor] (190.33,549.47) circle (  1.16);

\path[draw=drawColor,line width= 0.4pt,line join=round,line cap=round,fill=fillColor] (190.52,549.36) circle (  1.16);

\path[draw=drawColor,line width= 0.4pt,line join=round,line cap=round,fill=fillColor] (190.71,549.35) circle (  1.16);

\path[draw=drawColor,line width= 0.4pt,line join=round,line cap=round,fill=fillColor] (190.91,549.18) circle (  1.16);

\path[draw=drawColor,line width= 0.4pt,line join=round,line cap=round,fill=fillColor] (191.10,549.04) circle (  1.16);

\path[draw=drawColor,line width= 0.4pt,line join=round,line cap=round,fill=fillColor] (191.29,549.02) circle (  1.16);

\path[draw=drawColor,line width= 0.4pt,line join=round,line cap=round,fill=fillColor] (191.48,548.84) circle (  1.16);

\path[draw=drawColor,line width= 0.4pt,line join=round,line cap=round,fill=fillColor] (191.67,548.55) circle (  1.16);

\path[draw=drawColor,line width= 0.4pt,line join=round,line cap=round,fill=fillColor] (191.86,548.49) circle (  1.16);

\path[draw=drawColor,line width= 0.4pt,line join=round,line cap=round,fill=fillColor] (192.05,548.46) circle (  1.16);

\path[draw=drawColor,line width= 0.4pt,line join=round,line cap=round,fill=fillColor] (192.24,548.39) circle (  1.16);

\path[draw=drawColor,line width= 0.4pt,line join=round,line cap=round,fill=fillColor] (192.43,548.27) circle (  1.16);

\path[draw=drawColor,line width= 0.4pt,line join=round,line cap=round,fill=fillColor] (192.61,548.25) circle (  1.16);

\path[draw=drawColor,line width= 0.4pt,line join=round,line cap=round,fill=fillColor] (192.80,547.84) circle (  1.16);

\path[draw=drawColor,line width= 0.4pt,line join=round,line cap=round,fill=fillColor] (192.99,547.76) circle (  1.16);

\path[draw=drawColor,line width= 0.4pt,line join=round,line cap=round,fill=fillColor] (193.17,547.41) circle (  1.16);

\path[draw=drawColor,line width= 0.4pt,line join=round,line cap=round,fill=fillColor] (193.36,547.30) circle (  1.16);

\path[draw=drawColor,line width= 0.4pt,line join=round,line cap=round,fill=fillColor] (193.54,547.30) circle (  1.16);

\path[draw=drawColor,line width= 0.4pt,line join=round,line cap=round,fill=fillColor] (193.73,547.27) circle (  1.16);

\path[draw=drawColor,line width= 0.4pt,line join=round,line cap=round,fill=fillColor] (193.91,547.17) circle (  1.16);

\path[draw=drawColor,line width= 0.4pt,line join=round,line cap=round,fill=fillColor] (194.10,547.09) circle (  1.16);

\path[draw=drawColor,line width= 0.4pt,line join=round,line cap=round,fill=fillColor] (194.28,546.99) circle (  1.16);

\path[draw=drawColor,line width= 0.4pt,line join=round,line cap=round,fill=fillColor] (194.46,546.96) circle (  1.16);

\path[draw=drawColor,line width= 0.4pt,line join=round,line cap=round,fill=fillColor] (194.65,546.92) circle (  1.16);

\path[draw=drawColor,line width= 0.4pt,line join=round,line cap=round,fill=fillColor] (194.83,546.90) circle (  1.16);

\path[draw=drawColor,line width= 0.4pt,line join=round,line cap=round,fill=fillColor] (195.01,546.87) circle (  1.16);

\path[draw=drawColor,line width= 0.4pt,line join=round,line cap=round,fill=fillColor] (195.19,546.87) circle (  1.16);

\path[draw=drawColor,line width= 0.4pt,line join=round,line cap=round,fill=fillColor] (195.37,546.45) circle (  1.16);

\path[draw=drawColor,line width= 0.4pt,line join=round,line cap=round,fill=fillColor] (195.55,546.33) circle (  1.16);

\path[draw=drawColor,line width= 0.4pt,line join=round,line cap=round,fill=fillColor] (195.73,546.10) circle (  1.16);

\path[draw=drawColor,line width= 0.4pt,line join=round,line cap=round,fill=fillColor] (195.91,546.01) circle (  1.16);

\path[draw=drawColor,line width= 0.4pt,line join=round,line cap=round,fill=fillColor] (196.09,545.96) circle (  1.16);

\path[draw=drawColor,line width= 0.4pt,line join=round,line cap=round,fill=fillColor] (196.27,545.93) circle (  1.16);

\path[draw=drawColor,line width= 0.4pt,line join=round,line cap=round,fill=fillColor] (196.44,545.92) circle (  1.16);

\path[draw=drawColor,line width= 0.4pt,line join=round,line cap=round,fill=fillColor] (196.62,545.71) circle (  1.16);

\path[draw=drawColor,line width= 0.4pt,line join=round,line cap=round,fill=fillColor] (196.80,545.40) circle (  1.16);

\path[draw=drawColor,line width= 0.4pt,line join=round,line cap=round,fill=fillColor] (196.97,545.28) circle (  1.16);

\path[draw=drawColor,line width= 0.4pt,line join=round,line cap=round,fill=fillColor] (197.15,544.98) circle (  1.16);

\path[draw=drawColor,line width= 0.4pt,line join=round,line cap=round,fill=fillColor] (197.33,544.87) circle (  1.16);

\path[draw=drawColor,line width= 0.4pt,line join=round,line cap=round,fill=fillColor] (197.50,544.79) circle (  1.16);

\path[draw=drawColor,line width= 0.4pt,line join=round,line cap=round,fill=fillColor] (197.68,544.68) circle (  1.16);

\path[draw=drawColor,line width= 0.4pt,line join=round,line cap=round,fill=fillColor] (197.85,544.67) circle (  1.16);

\path[draw=drawColor,line width= 0.4pt,line join=round,line cap=round,fill=fillColor] (198.02,544.66) circle (  1.16);

\path[draw=drawColor,line width= 0.4pt,line join=round,line cap=round,fill=fillColor] (198.20,543.98) circle (  1.16);

\path[draw=drawColor,line width= 0.4pt,line join=round,line cap=round,fill=fillColor] (198.37,543.88) circle (  1.16);

\path[draw=drawColor,line width= 0.4pt,line join=round,line cap=round,fill=fillColor] (198.54,543.73) circle (  1.16);

\path[draw=drawColor,line width= 0.4pt,line join=round,line cap=round,fill=fillColor] (198.72,543.66) circle (  1.16);

\path[draw=drawColor,line width= 0.4pt,line join=round,line cap=round,fill=fillColor] (198.89,543.41) circle (  1.16);

\path[draw=drawColor,line width= 0.4pt,line join=round,line cap=round,fill=fillColor] (199.06,543.39) circle (  1.16);

\path[draw=drawColor,line width= 0.4pt,line join=round,line cap=round,fill=fillColor] (199.23,543.38) circle (  1.16);

\path[draw=drawColor,line width= 0.4pt,line join=round,line cap=round,fill=fillColor] (199.40,543.15) circle (  1.16);

\path[draw=drawColor,line width= 0.4pt,line join=round,line cap=round,fill=fillColor] (199.57,542.97) circle (  1.16);

\path[draw=drawColor,line width= 0.4pt,line join=round,line cap=round,fill=fillColor] (199.74,542.52) circle (  1.16);

\path[draw=drawColor,line width= 0.4pt,line join=round,line cap=round,fill=fillColor] (199.91,542.50) circle (  1.16);

\path[draw=drawColor,line width= 0.4pt,line join=round,line cap=round,fill=fillColor] (200.08,542.23) circle (  1.16);

\path[draw=drawColor,line width= 0.4pt,line join=round,line cap=round,fill=fillColor] (200.25,542.19) circle (  1.16);

\path[draw=drawColor,line width= 0.4pt,line join=round,line cap=round,fill=fillColor] (200.42,542.11) circle (  1.16);

\path[draw=drawColor,line width= 0.4pt,line join=round,line cap=round,fill=fillColor] (200.58,542.11) circle (  1.16);

\path[draw=drawColor,line width= 0.4pt,line join=round,line cap=round,fill=fillColor] (200.75,541.90) circle (  1.16);

\path[draw=drawColor,line width= 0.4pt,line join=round,line cap=round,fill=fillColor] (200.92,541.86) circle (  1.16);

\path[draw=drawColor,line width= 0.4pt,line join=round,line cap=round,fill=fillColor] (201.09,541.66) circle (  1.16);

\path[draw=drawColor,line width= 0.4pt,line join=round,line cap=round,fill=fillColor] (201.25,540.31) circle (  1.16);

\path[draw=drawColor,line width= 0.4pt,line join=round,line cap=round,fill=fillColor] (201.42,539.22) circle (  1.16);

\path[draw=drawColor,line width= 0.4pt,line join=round,line cap=round,fill=fillColor] (201.58,538.83) circle (  1.16);

\path[draw=drawColor,line width= 0.4pt,line join=round,line cap=round,fill=fillColor] (201.75,538.25) circle (  1.16);

\path[draw=drawColor,line width= 0.4pt,line join=round,line cap=round,fill=fillColor] (201.91,537.32) circle (  1.16);

\path[draw=drawColor,line width= 0.4pt,line join=round,line cap=round,fill=fillColor] (202.08,537.18) circle (  1.16);

\path[draw=drawColor,line width= 0.4pt,line join=round,line cap=round,fill=fillColor] (202.24,537.09) circle (  1.16);

\path[draw=drawColor,line width= 0.4pt,line join=round,line cap=round,fill=fillColor] (202.41,536.74) circle (  1.16);

\path[draw=drawColor,line width= 0.4pt,line join=round,line cap=round,fill=fillColor] (202.57,535.03) circle (  1.16);

\path[draw=drawColor,line width= 0.4pt,line join=round,line cap=round,fill=fillColor] (202.73,535.03) circle (  1.16);

\path[draw=drawColor,line width= 0.4pt,line join=round,line cap=round,fill=fillColor] (202.90,535.03) circle (  1.16);

\path[draw=drawColor,line width= 0.4pt,line join=round,line cap=round,fill=fillColor] (203.06,535.03) circle (  1.16);

\path[draw=drawColor,line width= 0.4pt,line join=round,line cap=round,fill=fillColor] (203.22,535.03) circle (  1.16);

\path[draw=drawColor,line width= 0.4pt,line join=round,line cap=round,fill=fillColor] (203.38,535.03) circle (  1.16);

\path[draw=drawColor,line width= 0.4pt,line join=round,line cap=round,fill=fillColor] (203.54,535.03) circle (  1.16);

\path[draw=drawColor,line width= 0.4pt,line join=round,line cap=round,fill=fillColor] (203.71,535.03) circle (  1.16);

\path[draw=drawColor,line width= 0.4pt,line join=round,line cap=round,fill=fillColor] (203.87,535.03) circle (  1.16);

\path[draw=drawColor,line width= 0.4pt,line join=round,line cap=round,fill=fillColor] (204.03,535.03) circle (  1.16);

\path[draw=drawColor,line width= 0.4pt,line join=round,line cap=round,fill=fillColor] (204.19,535.03) circle (  1.16);

\path[draw=drawColor,line width= 0.4pt,line join=round,line cap=round,fill=fillColor] (204.35,535.03) circle (  1.16);

\path[draw=drawColor,line width= 0.4pt,line join=round,line cap=round,fill=fillColor] (204.51,535.03) circle (  1.16);

\path[draw=drawColor,line width= 0.4pt,line join=round,line cap=round,fill=fillColor] (204.66,535.03) circle (  1.16);

\path[draw=drawColor,line width= 0.4pt,line join=round,line cap=round,fill=fillColor] (204.82,535.03) circle (  1.16);

\path[draw=drawColor,line width= 0.4pt,line join=round,line cap=round,fill=fillColor] (204.98,535.03) circle (  1.16);

\path[draw=drawColor,line width= 0.4pt,line join=round,line cap=round,fill=fillColor] (205.14,535.03) circle (  1.16);

\path[draw=drawColor,line width= 0.4pt,line join=round,line cap=round,fill=fillColor] (205.30,535.03) circle (  1.16);

\path[draw=drawColor,line width= 0.4pt,line join=round,line cap=round,fill=fillColor] (205.45,535.03) circle (  1.16);

\path[draw=drawColor,line width= 0.4pt,line join=round,line cap=round,fill=fillColor] (205.61,535.03) circle (  1.16);

\path[draw=drawColor,line width= 0.4pt,line join=round,line cap=round,fill=fillColor] (205.77,535.03) circle (  1.16);

\path[draw=drawColor,line width= 0.4pt,line join=round,line cap=round,fill=fillColor] (205.92,535.03) circle (  1.16);

\path[draw=drawColor,line width= 0.4pt,line join=round,line cap=round,fill=fillColor] (206.08,535.03) circle (  1.16);

\path[draw=drawColor,line width= 0.4pt,line join=round,line cap=round,fill=fillColor] (206.24,535.03) circle (  1.16);

\path[draw=drawColor,line width= 0.4pt,line join=round,line cap=round,fill=fillColor] (206.39,535.03) circle (  1.16);

\path[draw=drawColor,line width= 0.4pt,line join=round,line cap=round,fill=fillColor] (206.55,535.03) circle (  1.16);

\path[draw=drawColor,line width= 0.4pt,line join=round,line cap=round,fill=fillColor] (206.70,535.03) circle (  1.16);

\path[draw=drawColor,line width= 0.4pt,line join=round,line cap=round,fill=fillColor] (206.86,535.03) circle (  1.16);

\path[draw=drawColor,line width= 0.4pt,line join=round,line cap=round,fill=fillColor] (207.01,535.03) circle (  1.16);

\path[draw=drawColor,line width= 0.4pt,line join=round,line cap=round,fill=fillColor] (207.16,535.03) circle (  1.16);

\path[draw=drawColor,line width= 0.4pt,line join=round,line cap=round,fill=fillColor] (207.32,535.03) circle (  1.16);

\path[draw=drawColor,line width= 0.4pt,line join=round,line cap=round,fill=fillColor] (207.47,535.03) circle (  1.16);

\path[draw=drawColor,line width= 0.4pt,line join=round,line cap=round,fill=fillColor] (207.62,535.03) circle (  1.16);

\path[draw=drawColor,line width= 0.4pt,line join=round,line cap=round,fill=fillColor] (207.78,535.03) circle (  1.16);

\path[draw=drawColor,line width= 0.4pt,line join=round,line cap=round,fill=fillColor] (207.93,535.03) circle (  1.16);

\path[draw=drawColor,line width= 0.4pt,line join=round,line cap=round,fill=fillColor] (208.08,535.03) circle (  1.16);

\path[draw=drawColor,line width= 0.4pt,line join=round,line cap=round,fill=fillColor] (208.23,535.03) circle (  1.16);

\path[draw=drawColor,line width= 0.4pt,line join=round,line cap=round,fill=fillColor] (208.39,535.03) circle (  1.16);

\path[draw=drawColor,line width= 0.4pt,line join=round,line cap=round,fill=fillColor] (208.54,535.03) circle (  1.16);

\path[draw=drawColor,line width= 0.4pt,line join=round,line cap=round,fill=fillColor] (208.69,535.03) circle (  1.16);

\path[draw=drawColor,line width= 0.4pt,line join=round,line cap=round,fill=fillColor] (208.84,535.03) circle (  1.16);

\path[draw=drawColor,line width= 0.4pt,line join=round,line cap=round,fill=fillColor] (208.99,535.03) circle (  1.16);

\path[draw=drawColor,line width= 0.4pt,line join=round,line cap=round,fill=fillColor] (209.14,535.03) circle (  1.16);

\path[draw=drawColor,line width= 0.4pt,line join=round,line cap=round,fill=fillColor] (209.29,535.03) circle (  1.16);

\path[draw=drawColor,line width= 0.4pt,line join=round,line cap=round,fill=fillColor] (209.44,535.03) circle (  1.16);

\path[draw=drawColor,line width= 0.4pt,line join=round,line cap=round,fill=fillColor] (209.59,535.03) circle (  1.16);

\path[draw=drawColor,line width= 0.4pt,line join=round,line cap=round,fill=fillColor] (209.74,535.03) circle (  1.16);

\path[draw=drawColor,line width= 0.4pt,line join=round,line cap=round,fill=fillColor] (209.88,535.03) circle (  1.16);

\path[draw=drawColor,line width= 0.4pt,line join=round,line cap=round,fill=fillColor] (210.03,535.03) circle (  1.16);

\path[draw=drawColor,line width= 0.4pt,line join=round,line cap=round,fill=fillColor] (210.18,535.03) circle (  1.16);

\path[draw=drawColor,line width= 0.4pt,line join=round,line cap=round,fill=fillColor] (210.33,535.03) circle (  1.16);

\path[draw=drawColor,line width= 0.4pt,line join=round,line cap=round,fill=fillColor] (210.48,535.03) circle (  1.16);

\path[draw=drawColor,line width= 0.4pt,line join=round,line cap=round,fill=fillColor] (210.62,535.03) circle (  1.16);

\path[draw=drawColor,line width= 0.4pt,line join=round,line cap=round,fill=fillColor] (210.77,535.03) circle (  1.16);

\path[draw=drawColor,line width= 0.4pt,line join=round,line cap=round,fill=fillColor] (210.92,535.03) circle (  1.16);

\path[draw=drawColor,line width= 0.4pt,line join=round,line cap=round,fill=fillColor] (211.06,535.03) circle (  1.16);

\path[draw=drawColor,line width= 0.4pt,line join=round,line cap=round,fill=fillColor] (211.21,535.03) circle (  1.16);

\path[draw=drawColor,line width= 0.4pt,line join=round,line cap=round,fill=fillColor] (211.36,535.03) circle (  1.16);

\path[draw=drawColor,line width= 0.4pt,line join=round,line cap=round,fill=fillColor] (211.50,535.03) circle (  1.16);

\path[draw=drawColor,line width= 0.4pt,line join=round,line cap=round,fill=fillColor] (211.65,535.03) circle (  1.16);

\path[draw=drawColor,line width= 0.4pt,line join=round,line cap=round,fill=fillColor] (211.79,535.03) circle (  1.16);

\path[draw=drawColor,line width= 0.4pt,line join=round,line cap=round,fill=fillColor] (211.94,535.03) circle (  1.16);

\path[draw=drawColor,line width= 0.4pt,line join=round,line cap=round,fill=fillColor] (212.08,535.03) circle (  1.16);

\path[draw=drawColor,line width= 0.4pt,line join=round,line cap=round,fill=fillColor] (212.23,535.03) circle (  1.16);

\path[draw=drawColor,line width= 0.4pt,line join=round,line cap=round,fill=fillColor] (212.37,535.03) circle (  1.16);

\path[draw=drawColor,line width= 0.4pt,line join=round,line cap=round,fill=fillColor] (212.51,535.03) circle (  1.16);

\path[draw=drawColor,line width= 0.4pt,line join=round,line cap=round,fill=fillColor] (212.66,535.03) circle (  1.16);

\path[draw=drawColor,line width= 0.4pt,line join=round,line cap=round,fill=fillColor] (212.80,535.03) circle (  1.16);

\path[draw=drawColor,line width= 0.4pt,line join=round,line cap=round,fill=fillColor] (212.94,535.03) circle (  1.16);

\path[draw=drawColor,line width= 0.4pt,line join=round,line cap=round,fill=fillColor] (213.09,535.03) circle (  1.16);

\path[draw=drawColor,line width= 0.4pt,line join=round,line cap=round,fill=fillColor] (213.23,535.03) circle (  1.16);

\path[draw=drawColor,line width= 0.4pt,line join=round,line cap=round,fill=fillColor] (213.37,535.03) circle (  1.16);

\path[draw=drawColor,line width= 0.4pt,line join=round,line cap=round,fill=fillColor] (213.51,535.03) circle (  1.16);

\path[draw=drawColor,line width= 0.4pt,line join=round,line cap=round,fill=fillColor] (213.65,535.03) circle (  1.16);

\path[draw=drawColor,line width= 0.4pt,line join=round,line cap=round,fill=fillColor] (213.80,535.03) circle (  1.16);

\path[draw=drawColor,line width= 0.4pt,line join=round,line cap=round,fill=fillColor] (213.94,535.03) circle (  1.16);

\path[draw=drawColor,line width= 0.4pt,line join=round,line cap=round,fill=fillColor] (214.08,535.03) circle (  1.16);

\path[draw=drawColor,line width= 0.4pt,line join=round,line cap=round,fill=fillColor] (214.22,535.03) circle (  1.16);

\path[draw=drawColor,line width= 0.4pt,line join=round,line cap=round,fill=fillColor] (214.36,535.03) circle (  1.16);

\path[draw=drawColor,line width= 0.4pt,line join=round,line cap=round,fill=fillColor] (214.50,535.03) circle (  1.16);

\path[draw=drawColor,line width= 0.4pt,line join=round,line cap=round,fill=fillColor] (214.64,535.03) circle (  1.16);

\path[draw=drawColor,line width= 0.4pt,line join=round,line cap=round,fill=fillColor] (214.78,535.03) circle (  1.16);

\path[draw=drawColor,line width= 0.4pt,line join=round,line cap=round,fill=fillColor] (214.92,535.03) circle (  1.16);

\path[draw=drawColor,line width= 0.4pt,line join=round,line cap=round,fill=fillColor] (215.06,535.03) circle (  1.16);

\path[draw=drawColor,line width= 0.4pt,line join=round,line cap=round,fill=fillColor] (215.20,535.03) circle (  1.16);

\path[draw=drawColor,line width= 0.4pt,line join=round,line cap=round,fill=fillColor] (215.34,535.03) circle (  1.16);

\path[draw=drawColor,line width= 0.4pt,line join=round,line cap=round,fill=fillColor] (215.47,535.03) circle (  1.16);

\path[draw=drawColor,line width= 0.4pt,line join=round,line cap=round,fill=fillColor] (215.61,535.03) circle (  1.16);

\path[draw=drawColor,line width= 0.4pt,line join=round,line cap=round,fill=fillColor] (215.75,535.03) circle (  1.16);

\path[draw=drawColor,line width= 0.4pt,line join=round,line cap=round,fill=fillColor] (215.89,535.03) circle (  1.16);

\path[draw=drawColor,line width= 0.4pt,line join=round,line cap=round,fill=fillColor] (216.03,535.03) circle (  1.16);

\path[draw=drawColor,line width= 0.4pt,line join=round,line cap=round,fill=fillColor] (216.16,535.03) circle (  1.16);

\path[draw=drawColor,line width= 0.4pt,line join=round,line cap=round,fill=fillColor] (216.30,535.03) circle (  1.16);

\path[draw=drawColor,line width= 0.4pt,line join=round,line cap=round,fill=fillColor] (216.44,535.03) circle (  1.16);

\path[draw=drawColor,line width= 0.4pt,line join=round,line cap=round,fill=fillColor] (216.58,535.03) circle (  1.16);

\path[draw=drawColor,line width= 0.4pt,line join=round,line cap=round,fill=fillColor] (216.71,535.03) circle (  1.16);

\path[draw=drawColor,line width= 0.4pt,line join=round,line cap=round,fill=fillColor] (216.85,535.03) circle (  1.16);

\path[draw=drawColor,line width= 0.4pt,line join=round,line cap=round,fill=fillColor] (216.98,535.03) circle (  1.16);

\path[draw=drawColor,line width= 0.4pt,line join=round,line cap=round,fill=fillColor] (217.12,535.03) circle (  1.16);

\path[draw=drawColor,line width= 0.4pt,line join=round,line cap=round,fill=fillColor] (217.26,535.03) circle (  1.16);

\path[draw=drawColor,line width= 0.4pt,line join=round,line cap=round,fill=fillColor] (217.39,535.03) circle (  1.16);

\path[draw=drawColor,line width= 0.4pt,line join=round,line cap=round,fill=fillColor] (217.53,535.03) circle (  1.16);

\path[draw=drawColor,line width= 0.4pt,line join=round,line cap=round,fill=fillColor] (217.66,535.03) circle (  1.16);

\path[draw=drawColor,line width= 0.4pt,line join=round,line cap=round,fill=fillColor] (217.80,535.03) circle (  1.16);

\path[draw=drawColor,line width= 0.4pt,line join=round,line cap=round,fill=fillColor] (217.93,535.03) circle (  1.16);

\path[draw=drawColor,line width= 0.4pt,line join=round,line cap=round,fill=fillColor] (218.06,535.03) circle (  1.16);

\path[draw=drawColor,line width= 0.4pt,line join=round,line cap=round,fill=fillColor] (218.20,535.03) circle (  1.16);

\path[draw=drawColor,line width= 0.4pt,line join=round,line cap=round,fill=fillColor] (218.33,535.03) circle (  1.16);

\path[draw=drawColor,line width= 0.4pt,line join=round,line cap=round,fill=fillColor] (218.47,535.03) circle (  1.16);

\path[draw=drawColor,line width= 0.4pt,line join=round,line cap=round,fill=fillColor] (218.60,535.03) circle (  1.16);

\path[draw=drawColor,line width= 0.4pt,line join=round,line cap=round,fill=fillColor] (218.73,535.03) circle (  1.16);

\path[draw=drawColor,line width= 0.4pt,line join=round,line cap=round,fill=fillColor] (218.87,535.03) circle (  1.16);

\path[draw=drawColor,line width= 0.4pt,line join=round,line cap=round,fill=fillColor] (219.00,535.03) circle (  1.16);

\path[draw=drawColor,line width= 0.4pt,line join=round,line cap=round,fill=fillColor] (219.13,535.03) circle (  1.16);

\path[draw=drawColor,line width= 0.4pt,line join=round,line cap=round,fill=fillColor] (219.26,535.03) circle (  1.16);

\path[draw=drawColor,line width= 0.4pt,line join=round,line cap=round,fill=fillColor] (219.40,535.03) circle (  1.16);

\path[draw=drawColor,line width= 0.4pt,line join=round,line cap=round,fill=fillColor] (219.53,535.03) circle (  1.16);

\path[draw=drawColor,line width= 0.4pt,line join=round,line cap=round,fill=fillColor] (219.66,535.03) circle (  1.16);

\path[draw=drawColor,line width= 0.4pt,line join=round,line cap=round,fill=fillColor] (219.79,535.03) circle (  1.16);

\path[draw=drawColor,line width= 0.4pt,line join=round,line cap=round,fill=fillColor] (219.92,535.03) circle (  1.16);

\path[draw=drawColor,line width= 0.4pt,line join=round,line cap=round,fill=fillColor] (220.05,535.03) circle (  1.16);

\path[draw=drawColor,line width= 0.4pt,line join=round,line cap=round,fill=fillColor] (220.18,535.03) circle (  1.16);

\path[draw=drawColor,line width= 0.4pt,line join=round,line cap=round,fill=fillColor] (220.31,535.03) circle (  1.16);

\path[draw=drawColor,line width= 0.4pt,line join=round,line cap=round,fill=fillColor] (220.45,535.03) circle (  1.16);

\path[draw=drawColor,line width= 0.4pt,line join=round,line cap=round,fill=fillColor] (220.58,535.03) circle (  1.16);

\path[draw=drawColor,line width= 0.4pt,line join=round,line cap=round,fill=fillColor] (220.71,535.03) circle (  1.16);

\path[draw=drawColor,line width= 0.4pt,line join=round,line cap=round,fill=fillColor] (220.84,535.03) circle (  1.16);

\path[draw=drawColor,line width= 0.4pt,line join=round,line cap=round,fill=fillColor] (220.97,535.03) circle (  1.16);

\path[draw=drawColor,line width= 0.4pt,line join=round,line cap=round,fill=fillColor] (221.09,535.03) circle (  1.16);

\path[draw=drawColor,line width= 0.4pt,line join=round,line cap=round,fill=fillColor] (221.22,535.03) circle (  1.16);

\path[draw=drawColor,line width= 0.4pt,line join=round,line cap=round,fill=fillColor] (221.35,535.03) circle (  1.16);

\path[draw=drawColor,line width= 0.4pt,line join=round,line cap=round,fill=fillColor] (221.48,535.03) circle (  1.16);

\path[draw=drawColor,line width= 0.4pt,line join=round,line cap=round,fill=fillColor] (221.61,535.03) circle (  1.16);

\path[draw=drawColor,line width= 0.4pt,line join=round,line cap=round,fill=fillColor] (221.74,535.03) circle (  1.16);

\path[draw=drawColor,line width= 0.4pt,line join=round,line cap=round,fill=fillColor] (221.87,535.03) circle (  1.16);

\path[draw=drawColor,line width= 0.4pt,line join=round,line cap=round,fill=fillColor] (222.00,535.03) circle (  1.16);

\path[draw=drawColor,line width= 0.4pt,line join=round,line cap=round,fill=fillColor] (222.12,535.03) circle (  1.16);

\path[draw=drawColor,line width= 0.4pt,line join=round,line cap=round,fill=fillColor] (222.25,535.03) circle (  1.16);

\path[draw=drawColor,line width= 0.4pt,line join=round,line cap=round,fill=fillColor] (222.38,535.03) circle (  1.16);

\path[draw=drawColor,line width= 0.4pt,line join=round,line cap=round,fill=fillColor] (222.51,535.03) circle (  1.16);

\path[draw=drawColor,line width= 0.4pt,line join=round,line cap=round,fill=fillColor] (222.63,535.03) circle (  1.16);

\path[draw=drawColor,line width= 0.4pt,line join=round,line cap=round,fill=fillColor] (222.76,535.03) circle (  1.16);

\path[draw=drawColor,line width= 0.4pt,line join=round,line cap=round,fill=fillColor] (222.89,535.03) circle (  1.16);

\path[draw=drawColor,line width= 0.4pt,line join=round,line cap=round,fill=fillColor] (223.01,535.03) circle (  1.16);

\path[draw=drawColor,line width= 0.4pt,line join=round,line cap=round,fill=fillColor] (223.14,535.03) circle (  1.16);

\path[draw=drawColor,line width= 0.4pt,line join=round,line cap=round,fill=fillColor] (223.27,535.03) circle (  1.16);

\path[draw=drawColor,line width= 0.4pt,line join=round,line cap=round,fill=fillColor] (223.39,535.03) circle (  1.16);

\path[draw=drawColor,line width= 0.4pt,line join=round,line cap=round,fill=fillColor] (223.52,535.03) circle (  1.16);

\path[draw=drawColor,line width= 0.4pt,line join=round,line cap=round,fill=fillColor] (223.64,535.03) circle (  1.16);

\path[draw=drawColor,line width= 0.4pt,line join=round,line cap=round,fill=fillColor] (223.77,535.03) circle (  1.16);

\path[draw=drawColor,line width= 0.4pt,line join=round,line cap=round,fill=fillColor] (223.89,535.03) circle (  1.16);

\path[draw=drawColor,line width= 0.4pt,line join=round,line cap=round,fill=fillColor] (224.02,535.03) circle (  1.16);

\path[draw=drawColor,line width= 0.4pt,line join=round,line cap=round,fill=fillColor] (224.14,535.03) circle (  1.16);

\path[draw=drawColor,line width= 0.4pt,line join=round,line cap=round,fill=fillColor] (224.27,535.03) circle (  1.16);

\path[draw=drawColor,line width= 0.4pt,line join=round,line cap=round,fill=fillColor] (224.39,535.03) circle (  1.16);

\path[draw=drawColor,line width= 0.4pt,line join=round,line cap=round,fill=fillColor] (224.52,535.03) circle (  1.16);

\path[draw=drawColor,line width= 0.4pt,line join=round,line cap=round,fill=fillColor] (224.64,535.03) circle (  1.16);

\path[draw=drawColor,line width= 0.4pt,line join=round,line cap=round,fill=fillColor] (224.77,535.03) circle (  1.16);

\path[draw=drawColor,line width= 0.4pt,line join=round,line cap=round,fill=fillColor] (224.89,535.03) circle (  1.16);

\path[draw=drawColor,line width= 0.4pt,line join=round,line cap=round,fill=fillColor] (225.01,535.03) circle (  1.16);

\path[draw=drawColor,line width= 0.4pt,line join=round,line cap=round,fill=fillColor] (225.14,535.03) circle (  1.16);

\path[draw=drawColor,line width= 0.4pt,line join=round,line cap=round,fill=fillColor] (225.26,535.03) circle (  1.16);
\definecolor[named]{drawColor}{rgb}{0.65,0.34,0.16}
\definecolor[named]{fillColor}{rgb}{0.65,0.34,0.16}

\path[draw=drawColor,line width= 0.4pt,line join=round,line cap=round,fill=fillColor] ( 74.88,594.74) circle (  1.16);

\path[draw=drawColor,line width= 0.4pt,line join=round,line cap=round,fill=fillColor] ( 80.66,594.15) circle (  1.16);

\path[draw=drawColor,line width= 0.4pt,line join=round,line cap=round,fill=fillColor] ( 84.72,592.03) circle (  1.16);

\path[draw=drawColor,line width= 0.4pt,line join=round,line cap=round,fill=fillColor] ( 87.95,591.28) circle (  1.16);

\path[draw=drawColor,line width= 0.4pt,line join=round,line cap=round,fill=fillColor] ( 90.68,591.08) circle (  1.16);

\path[draw=drawColor,line width= 0.4pt,line join=round,line cap=round,fill=fillColor] ( 93.06,588.83) circle (  1.16);

\path[draw=drawColor,line width= 0.4pt,line join=round,line cap=round,fill=fillColor] ( 95.19,588.47) circle (  1.16);

\path[draw=drawColor,line width= 0.4pt,line join=round,line cap=round,fill=fillColor] ( 97.13,586.93) circle (  1.16);

\path[draw=drawColor,line width= 0.4pt,line join=round,line cap=round,fill=fillColor] ( 98.91,586.65) circle (  1.16);

\path[draw=drawColor,line width= 0.4pt,line join=round,line cap=round,fill=fillColor] (100.57,585.38) circle (  1.16);

\path[draw=drawColor,line width= 0.4pt,line join=round,line cap=round,fill=fillColor] (102.11,585.02) circle (  1.16);

\path[draw=drawColor,line width= 0.4pt,line join=round,line cap=round,fill=fillColor] (103.57,584.03) circle (  1.16);

\path[draw=drawColor,line width= 0.4pt,line join=round,line cap=round,fill=fillColor] (104.95,583.24) circle (  1.16);

\path[draw=drawColor,line width= 0.4pt,line join=round,line cap=round,fill=fillColor] (106.26,582.62) circle (  1.16);

\path[draw=drawColor,line width= 0.4pt,line join=round,line cap=round,fill=fillColor] (107.50,582.03) circle (  1.16);

\path[draw=drawColor,line width= 0.4pt,line join=round,line cap=round,fill=fillColor] (108.70,580.68) circle (  1.16);

\path[draw=drawColor,line width= 0.4pt,line join=round,line cap=round,fill=fillColor] (109.84,579.27) circle (  1.16);

\path[draw=drawColor,line width= 0.4pt,line join=round,line cap=round,fill=fillColor] (110.94,578.90) circle (  1.16);

\path[draw=drawColor,line width= 0.4pt,line join=round,line cap=round,fill=fillColor] (112.00,578.75) circle (  1.16);

\path[draw=drawColor,line width= 0.4pt,line join=round,line cap=round,fill=fillColor] (113.03,578.65) circle (  1.16);

\path[draw=drawColor,line width= 0.4pt,line join=round,line cap=round,fill=fillColor] (114.02,577.99) circle (  1.16);

\path[draw=drawColor,line width= 0.4pt,line join=round,line cap=round,fill=fillColor] (114.98,577.79) circle (  1.16);

\path[draw=drawColor,line width= 0.4pt,line join=round,line cap=round,fill=fillColor] (115.91,577.69) circle (  1.16);

\path[draw=drawColor,line width= 0.4pt,line join=round,line cap=round,fill=fillColor] (116.81,577.67) circle (  1.16);

\path[draw=drawColor,line width= 0.4pt,line join=round,line cap=round,fill=fillColor] (117.69,575.81) circle (  1.16);

\path[draw=drawColor,line width= 0.4pt,line join=round,line cap=round,fill=fillColor] (118.55,575.19) circle (  1.16);

\path[draw=drawColor,line width= 0.4pt,line join=round,line cap=round,fill=fillColor] (119.38,575.16) circle (  1.16);

\path[draw=drawColor,line width= 0.4pt,line join=round,line cap=round,fill=fillColor] (120.20,574.91) circle (  1.16);

\path[draw=drawColor,line width= 0.4pt,line join=round,line cap=round,fill=fillColor] (120.99,574.59) circle (  1.16);

\path[draw=drawColor,line width= 0.4pt,line join=round,line cap=round,fill=fillColor] (121.77,573.88) circle (  1.16);

\path[draw=drawColor,line width= 0.4pt,line join=round,line cap=round,fill=fillColor] (122.53,573.72) circle (  1.16);

\path[draw=drawColor,line width= 0.4pt,line join=round,line cap=round,fill=fillColor] (123.27,572.57) circle (  1.16);

\path[draw=drawColor,line width= 0.4pt,line join=round,line cap=round,fill=fillColor] (124.00,571.78) circle (  1.16);

\path[draw=drawColor,line width= 0.4pt,line join=round,line cap=round,fill=fillColor] (124.71,571.75) circle (  1.16);

\path[draw=drawColor,line width= 0.4pt,line join=round,line cap=round,fill=fillColor] (125.41,571.54) circle (  1.16);

\path[draw=drawColor,line width= 0.4pt,line join=round,line cap=round,fill=fillColor] (126.10,571.28) circle (  1.16);

\path[draw=drawColor,line width= 0.4pt,line join=round,line cap=round,fill=fillColor] (126.77,570.64) circle (  1.16);

\path[draw=drawColor,line width= 0.4pt,line join=round,line cap=round,fill=fillColor] (127.44,570.13) circle (  1.16);

\path[draw=drawColor,line width= 0.4pt,line join=round,line cap=round,fill=fillColor] (128.09,569.92) circle (  1.16);

\path[draw=drawColor,line width= 0.4pt,line join=round,line cap=round,fill=fillColor] (128.73,569.91) circle (  1.16);

\path[draw=drawColor,line width= 0.4pt,line join=round,line cap=round,fill=fillColor] (129.35,569.91) circle (  1.16);

\path[draw=drawColor,line width= 0.4pt,line join=round,line cap=round,fill=fillColor] (129.97,569.45) circle (  1.16);

\path[draw=drawColor,line width= 0.4pt,line join=round,line cap=round,fill=fillColor] (130.58,568.72) circle (  1.16);

\path[draw=drawColor,line width= 0.4pt,line join=round,line cap=round,fill=fillColor] (131.18,568.40) circle (  1.16);

\path[draw=drawColor,line width= 0.4pt,line join=round,line cap=round,fill=fillColor] (131.77,568.19) circle (  1.16);

\path[draw=drawColor,line width= 0.4pt,line join=round,line cap=round,fill=fillColor] (132.35,567.99) circle (  1.16);

\path[draw=drawColor,line width= 0.4pt,line join=round,line cap=round,fill=fillColor] (132.93,567.87) circle (  1.16);

\path[draw=drawColor,line width= 0.4pt,line join=round,line cap=round,fill=fillColor] (133.49,567.81) circle (  1.16);

\path[draw=drawColor,line width= 0.4pt,line join=round,line cap=round,fill=fillColor] (134.05,567.53) circle (  1.16);

\path[draw=drawColor,line width= 0.4pt,line join=round,line cap=round,fill=fillColor] (134.60,567.52) circle (  1.16);

\path[draw=drawColor,line width= 0.4pt,line join=round,line cap=round,fill=fillColor] (135.14,567.12) circle (  1.16);

\path[draw=drawColor,line width= 0.4pt,line join=round,line cap=round,fill=fillColor] (135.68,566.98) circle (  1.16);

\path[draw=drawColor,line width= 0.4pt,line join=round,line cap=round,fill=fillColor] (136.21,566.83) circle (  1.16);

\path[draw=drawColor,line width= 0.4pt,line join=round,line cap=round,fill=fillColor] (136.73,565.55) circle (  1.16);

\path[draw=drawColor,line width= 0.4pt,line join=round,line cap=round,fill=fillColor] (137.25,565.06) circle (  1.16);

\path[draw=drawColor,line width= 0.4pt,line join=round,line cap=round,fill=fillColor] (137.76,564.86) circle (  1.16);

\path[draw=drawColor,line width= 0.4pt,line join=round,line cap=round,fill=fillColor] (138.26,564.69) circle (  1.16);

\path[draw=drawColor,line width= 0.4pt,line join=round,line cap=round,fill=fillColor] (138.76,564.25) circle (  1.16);

\path[draw=drawColor,line width= 0.4pt,line join=round,line cap=round,fill=fillColor] (139.25,564.22) circle (  1.16);

\path[draw=drawColor,line width= 0.4pt,line join=round,line cap=round,fill=fillColor] (139.74,563.72) circle (  1.16);

\path[draw=drawColor,line width= 0.4pt,line join=round,line cap=round,fill=fillColor] (140.22,563.40) circle (  1.16);

\path[draw=drawColor,line width= 0.4pt,line join=round,line cap=round,fill=fillColor] (140.70,563.19) circle (  1.16);

\path[draw=drawColor,line width= 0.4pt,line join=round,line cap=round,fill=fillColor] (141.17,563.15) circle (  1.16);

\path[draw=drawColor,line width= 0.4pt,line join=round,line cap=round,fill=fillColor] (141.63,562.68) circle (  1.16);

\path[draw=drawColor,line width= 0.4pt,line join=round,line cap=round,fill=fillColor] (142.09,562.31) circle (  1.16);

\path[draw=drawColor,line width= 0.4pt,line join=round,line cap=round,fill=fillColor] (142.55,562.03) circle (  1.16);

\path[draw=drawColor,line width= 0.4pt,line join=round,line cap=round,fill=fillColor] (143.00,561.91) circle (  1.16);

\path[draw=drawColor,line width= 0.4pt,line join=round,line cap=round,fill=fillColor] (143.45,561.90) circle (  1.16);

\path[draw=drawColor,line width= 0.4pt,line join=round,line cap=round,fill=fillColor] (143.89,561.71) circle (  1.16);

\path[draw=drawColor,line width= 0.4pt,line join=round,line cap=round,fill=fillColor] (144.33,561.52) circle (  1.16);

\path[draw=drawColor,line width= 0.4pt,line join=round,line cap=round,fill=fillColor] (144.77,561.48) circle (  1.16);

\path[draw=drawColor,line width= 0.4pt,line join=round,line cap=round,fill=fillColor] (145.20,561.40) circle (  1.16);

\path[draw=drawColor,line width= 0.4pt,line join=round,line cap=round,fill=fillColor] (145.62,561.27) circle (  1.16);

\path[draw=drawColor,line width= 0.4pt,line join=round,line cap=round,fill=fillColor] (146.05,561.26) circle (  1.16);

\path[draw=drawColor,line width= 0.4pt,line join=round,line cap=round,fill=fillColor] (146.46,561.13) circle (  1.16);

\path[draw=drawColor,line width= 0.4pt,line join=round,line cap=round,fill=fillColor] (146.88,560.93) circle (  1.16);

\path[draw=drawColor,line width= 0.4pt,line join=round,line cap=round,fill=fillColor] (147.29,560.91) circle (  1.16);

\path[draw=drawColor,line width= 0.4pt,line join=round,line cap=round,fill=fillColor] (147.70,560.79) circle (  1.16);

\path[draw=drawColor,line width= 0.4pt,line join=round,line cap=round,fill=fillColor] (148.10,560.73) circle (  1.16);

\path[draw=drawColor,line width= 0.4pt,line join=round,line cap=round,fill=fillColor] (148.50,559.84) circle (  1.16);

\path[draw=drawColor,line width= 0.4pt,line join=round,line cap=round,fill=fillColor] (148.90,559.84) circle (  1.16);

\path[draw=drawColor,line width= 0.4pt,line join=round,line cap=round,fill=fillColor] (149.30,559.79) circle (  1.16);

\path[draw=drawColor,line width= 0.4pt,line join=round,line cap=round,fill=fillColor] (149.69,559.63) circle (  1.16);

\path[draw=drawColor,line width= 0.4pt,line join=round,line cap=round,fill=fillColor] (150.08,559.52) circle (  1.16);

\path[draw=drawColor,line width= 0.4pt,line join=round,line cap=round,fill=fillColor] (150.46,559.33) circle (  1.16);

\path[draw=drawColor,line width= 0.4pt,line join=round,line cap=round,fill=fillColor] (150.84,559.18) circle (  1.16);

\path[draw=drawColor,line width= 0.4pt,line join=round,line cap=round,fill=fillColor] (151.22,558.44) circle (  1.16);

\path[draw=drawColor,line width= 0.4pt,line join=round,line cap=round,fill=fillColor] (151.60,558.41) circle (  1.16);

\path[draw=drawColor,line width= 0.4pt,line join=round,line cap=round,fill=fillColor] (151.97,557.63) circle (  1.16);

\path[draw=drawColor,line width= 0.4pt,line join=round,line cap=round,fill=fillColor] (152.34,557.53) circle (  1.16);

\path[draw=drawColor,line width= 0.4pt,line join=round,line cap=round,fill=fillColor] (152.71,557.43) circle (  1.16);

\path[draw=drawColor,line width= 0.4pt,line join=round,line cap=round,fill=fillColor] (153.08,557.38) circle (  1.16);

\path[draw=drawColor,line width= 0.4pt,line join=round,line cap=round,fill=fillColor] (153.44,557.10) circle (  1.16);

\path[draw=drawColor,line width= 0.4pt,line join=round,line cap=round,fill=fillColor] (153.80,556.64) circle (  1.16);

\path[draw=drawColor,line width= 0.4pt,line join=round,line cap=round,fill=fillColor] (154.16,556.61) circle (  1.16);

\path[draw=drawColor,line width= 0.4pt,line join=round,line cap=round,fill=fillColor] (154.51,556.24) circle (  1.16);

\path[draw=drawColor,line width= 0.4pt,line join=round,line cap=round,fill=fillColor] (154.86,556.12) circle (  1.16);

\path[draw=drawColor,line width= 0.4pt,line join=round,line cap=round,fill=fillColor] (155.22,556.10) circle (  1.16);

\path[draw=drawColor,line width= 0.4pt,line join=round,line cap=round,fill=fillColor] (155.56,555.92) circle (  1.16);

\path[draw=drawColor,line width= 0.4pt,line join=round,line cap=round,fill=fillColor] (155.91,555.92) circle (  1.16);

\path[draw=drawColor,line width= 0.4pt,line join=round,line cap=round,fill=fillColor] (156.25,555.61) circle (  1.16);

\path[draw=drawColor,line width= 0.4pt,line join=round,line cap=round,fill=fillColor] (156.59,555.35) circle (  1.16);

\path[draw=drawColor,line width= 0.4pt,line join=round,line cap=round,fill=fillColor] (156.93,555.17) circle (  1.16);

\path[draw=drawColor,line width= 0.4pt,line join=round,line cap=round,fill=fillColor] (157.27,555.12) circle (  1.16);

\path[draw=drawColor,line width= 0.4pt,line join=round,line cap=round,fill=fillColor] (157.60,555.01) circle (  1.16);

\path[draw=drawColor,line width= 0.4pt,line join=round,line cap=round,fill=fillColor] (157.93,554.97) circle (  1.16);

\path[draw=drawColor,line width= 0.4pt,line join=round,line cap=round,fill=fillColor] (158.26,554.69) circle (  1.16);

\path[draw=drawColor,line width= 0.4pt,line join=round,line cap=round,fill=fillColor] (158.59,554.07) circle (  1.16);

\path[draw=drawColor,line width= 0.4pt,line join=round,line cap=round,fill=fillColor] (158.92,553.96) circle (  1.16);

\path[draw=drawColor,line width= 0.4pt,line join=round,line cap=round,fill=fillColor] (159.24,553.93) circle (  1.16);

\path[draw=drawColor,line width= 0.4pt,line join=round,line cap=round,fill=fillColor] (159.56,553.86) circle (  1.16);

\path[draw=drawColor,line width= 0.4pt,line join=round,line cap=round,fill=fillColor] (159.88,553.61) circle (  1.16);

\path[draw=drawColor,line width= 0.4pt,line join=round,line cap=round,fill=fillColor] (160.20,553.27) circle (  1.16);

\path[draw=drawColor,line width= 0.4pt,line join=round,line cap=round,fill=fillColor] (160.52,553.13) circle (  1.16);

\path[draw=drawColor,line width= 0.4pt,line join=round,line cap=round,fill=fillColor] (160.83,553.04) circle (  1.16);

\path[draw=drawColor,line width= 0.4pt,line join=round,line cap=round,fill=fillColor] (161.15,552.81) circle (  1.16);

\path[draw=drawColor,line width= 0.4pt,line join=round,line cap=round,fill=fillColor] (161.46,552.71) circle (  1.16);

\path[draw=drawColor,line width= 0.4pt,line join=round,line cap=round,fill=fillColor] (161.77,552.64) circle (  1.16);

\path[draw=drawColor,line width= 0.4pt,line join=round,line cap=round,fill=fillColor] (162.07,552.56) circle (  1.16);

\path[draw=drawColor,line width= 0.4pt,line join=round,line cap=round,fill=fillColor] (162.38,552.51) circle (  1.16);

\path[draw=drawColor,line width= 0.4pt,line join=round,line cap=round,fill=fillColor] (162.68,552.32) circle (  1.16);

\path[draw=drawColor,line width= 0.4pt,line join=round,line cap=round,fill=fillColor] (162.99,552.01) circle (  1.16);

\path[draw=drawColor,line width= 0.4pt,line join=round,line cap=round,fill=fillColor] (163.29,551.86) circle (  1.16);

\path[draw=drawColor,line width= 0.4pt,line join=round,line cap=round,fill=fillColor] (163.59,551.82) circle (  1.16);

\path[draw=drawColor,line width= 0.4pt,line join=round,line cap=round,fill=fillColor] (163.88,551.69) circle (  1.16);

\path[draw=drawColor,line width= 0.4pt,line join=round,line cap=round,fill=fillColor] (164.18,551.69) circle (  1.16);

\path[draw=drawColor,line width= 0.4pt,line join=round,line cap=round,fill=fillColor] (164.47,551.50) circle (  1.16);

\path[draw=drawColor,line width= 0.4pt,line join=round,line cap=round,fill=fillColor] (164.77,550.32) circle (  1.16);

\path[draw=drawColor,line width= 0.4pt,line join=round,line cap=round,fill=fillColor] (165.06,549.72) circle (  1.16);

\path[draw=drawColor,line width= 0.4pt,line join=round,line cap=round,fill=fillColor] (165.35,549.21) circle (  1.16);

\path[draw=drawColor,line width= 0.4pt,line join=round,line cap=round,fill=fillColor] (165.64,549.02) circle (  1.16);

\path[draw=drawColor,line width= 0.4pt,line join=round,line cap=round,fill=fillColor] (165.92,548.60) circle (  1.16);

\path[draw=drawColor,line width= 0.4pt,line join=round,line cap=round,fill=fillColor] (166.21,548.23) circle (  1.16);

\path[draw=drawColor,line width= 0.4pt,line join=round,line cap=round,fill=fillColor] (166.49,548.12) circle (  1.16);

\path[draw=drawColor,line width= 0.4pt,line join=round,line cap=round,fill=fillColor] (166.77,547.95) circle (  1.16);

\path[draw=drawColor,line width= 0.4pt,line join=round,line cap=round,fill=fillColor] (167.06,547.93) circle (  1.16);

\path[draw=drawColor,line width= 0.4pt,line join=round,line cap=round,fill=fillColor] (167.34,547.41) circle (  1.16);

\path[draw=drawColor,line width= 0.4pt,line join=round,line cap=round,fill=fillColor] (167.61,547.30) circle (  1.16);

\path[draw=drawColor,line width= 0.4pt,line join=round,line cap=round,fill=fillColor] (167.89,547.09) circle (  1.16);

\path[draw=drawColor,line width= 0.4pt,line join=round,line cap=round,fill=fillColor] (168.17,546.93) circle (  1.16);

\path[draw=drawColor,line width= 0.4pt,line join=round,line cap=round,fill=fillColor] (168.44,546.90) circle (  1.16);

\path[draw=drawColor,line width= 0.4pt,line join=round,line cap=round,fill=fillColor] (168.71,546.85) circle (  1.16);

\path[draw=drawColor,line width= 0.4pt,line join=round,line cap=round,fill=fillColor] (168.99,546.81) circle (  1.16);

\path[draw=drawColor,line width= 0.4pt,line join=round,line cap=round,fill=fillColor] (169.26,546.77) circle (  1.16);

\path[draw=drawColor,line width= 0.4pt,line join=round,line cap=round,fill=fillColor] (169.53,546.44) circle (  1.16);

\path[draw=drawColor,line width= 0.4pt,line join=round,line cap=round,fill=fillColor] (169.79,546.37) circle (  1.16);

\path[draw=drawColor,line width= 0.4pt,line join=round,line cap=round,fill=fillColor] (170.06,546.27) circle (  1.16);

\path[draw=drawColor,line width= 0.4pt,line join=round,line cap=round,fill=fillColor] (170.33,545.88) circle (  1.16);

\path[draw=drawColor,line width= 0.4pt,line join=round,line cap=round,fill=fillColor] (170.59,545.86) circle (  1.16);

\path[draw=drawColor,line width= 0.4pt,line join=round,line cap=round,fill=fillColor] (170.85,545.46) circle (  1.16);

\path[draw=drawColor,line width= 0.4pt,line join=round,line cap=round,fill=fillColor] (171.12,545.46) circle (  1.16);

\path[draw=drawColor,line width= 0.4pt,line join=round,line cap=round,fill=fillColor] (171.38,544.68) circle (  1.16);

\path[draw=drawColor,line width= 0.4pt,line join=round,line cap=round,fill=fillColor] (171.64,544.36) circle (  1.16);

\path[draw=drawColor,line width= 0.4pt,line join=round,line cap=round,fill=fillColor] (171.90,544.15) circle (  1.16);

\path[draw=drawColor,line width= 0.4pt,line join=round,line cap=round,fill=fillColor] (172.15,543.98) circle (  1.16);

\path[draw=drawColor,line width= 0.4pt,line join=round,line cap=round,fill=fillColor] (172.41,543.84) circle (  1.16);

\path[draw=drawColor,line width= 0.4pt,line join=round,line cap=round,fill=fillColor] (172.67,543.73) circle (  1.16);

\path[draw=drawColor,line width= 0.4pt,line join=round,line cap=round,fill=fillColor] (172.92,543.45) circle (  1.16);

\path[draw=drawColor,line width= 0.4pt,line join=round,line cap=round,fill=fillColor] (173.17,542.81) circle (  1.16);

\path[draw=drawColor,line width= 0.4pt,line join=round,line cap=round,fill=fillColor] (173.43,542.75) circle (  1.16);

\path[draw=drawColor,line width= 0.4pt,line join=round,line cap=round,fill=fillColor] (173.68,542.59) circle (  1.16);

\path[draw=drawColor,line width= 0.4pt,line join=round,line cap=round,fill=fillColor] (173.93,541.90) circle (  1.16);

\path[draw=drawColor,line width= 0.4pt,line join=round,line cap=round,fill=fillColor] (174.18,540.71) circle (  1.16);

\path[draw=drawColor,line width= 0.4pt,line join=round,line cap=round,fill=fillColor] (174.42,540.68) circle (  1.16);

\path[draw=drawColor,line width= 0.4pt,line join=round,line cap=round,fill=fillColor] (174.67,540.47) circle (  1.16);

\path[draw=drawColor,line width= 0.4pt,line join=round,line cap=round,fill=fillColor] (174.92,540.33) circle (  1.16);

\path[draw=drawColor,line width= 0.4pt,line join=round,line cap=round,fill=fillColor] (175.16,540.29) circle (  1.16);

\path[draw=drawColor,line width= 0.4pt,line join=round,line cap=round,fill=fillColor] (175.41,540.16) circle (  1.16);

\path[draw=drawColor,line width= 0.4pt,line join=round,line cap=round,fill=fillColor] (175.65,540.15) circle (  1.16);

\path[draw=drawColor,line width= 0.4pt,line join=round,line cap=round,fill=fillColor] (175.89,537.01) circle (  1.16);

\path[draw=drawColor,line width= 0.4pt,line join=round,line cap=round,fill=fillColor] (176.13,535.03) circle (  1.16);

\path[draw=drawColor,line width= 0.4pt,line join=round,line cap=round,fill=fillColor] (176.37,535.03) circle (  1.16);

\path[draw=drawColor,line width= 0.4pt,line join=round,line cap=round,fill=fillColor] (176.61,535.03) circle (  1.16);

\path[draw=drawColor,line width= 0.4pt,line join=round,line cap=round,fill=fillColor] (176.85,535.03) circle (  1.16);

\path[draw=drawColor,line width= 0.4pt,line join=round,line cap=round,fill=fillColor] (177.09,535.03) circle (  1.16);

\path[draw=drawColor,line width= 0.4pt,line join=round,line cap=round,fill=fillColor] (177.32,535.03) circle (  1.16);

\path[draw=drawColor,line width= 0.4pt,line join=round,line cap=round,fill=fillColor] (177.56,535.03) circle (  1.16);

\path[draw=drawColor,line width= 0.4pt,line join=round,line cap=round,fill=fillColor] (177.80,535.03) circle (  1.16);

\path[draw=drawColor,line width= 0.4pt,line join=round,line cap=round,fill=fillColor] (178.03,535.03) circle (  1.16);

\path[draw=drawColor,line width= 0.4pt,line join=round,line cap=round,fill=fillColor] (178.26,535.03) circle (  1.16);

\path[draw=drawColor,line width= 0.4pt,line join=round,line cap=round,fill=fillColor] (178.49,535.03) circle (  1.16);

\path[draw=drawColor,line width= 0.4pt,line join=round,line cap=round,fill=fillColor] (178.73,535.03) circle (  1.16);

\path[draw=drawColor,line width= 0.4pt,line join=round,line cap=round,fill=fillColor] (178.96,535.03) circle (  1.16);

\path[draw=drawColor,line width= 0.4pt,line join=round,line cap=round,fill=fillColor] (179.19,535.03) circle (  1.16);

\path[draw=drawColor,line width= 0.4pt,line join=round,line cap=round,fill=fillColor] (179.41,535.03) circle (  1.16);

\path[draw=drawColor,line width= 0.4pt,line join=round,line cap=round,fill=fillColor] (179.64,535.03) circle (  1.16);

\path[draw=drawColor,line width= 0.4pt,line join=round,line cap=round,fill=fillColor] (179.87,535.03) circle (  1.16);

\path[draw=drawColor,line width= 0.4pt,line join=round,line cap=round,fill=fillColor] (180.10,535.03) circle (  1.16);

\path[draw=drawColor,line width= 0.4pt,line join=round,line cap=round,fill=fillColor] (180.32,535.03) circle (  1.16);

\path[draw=drawColor,line width= 0.4pt,line join=round,line cap=round,fill=fillColor] (180.55,535.03) circle (  1.16);

\path[draw=drawColor,line width= 0.4pt,line join=round,line cap=round,fill=fillColor] (180.77,535.03) circle (  1.16);

\path[draw=drawColor,line width= 0.4pt,line join=round,line cap=round,fill=fillColor] (180.99,535.03) circle (  1.16);

\path[draw=drawColor,line width= 0.4pt,line join=round,line cap=round,fill=fillColor] (181.22,535.03) circle (  1.16);

\path[draw=drawColor,line width= 0.4pt,line join=round,line cap=round,fill=fillColor] (181.44,535.03) circle (  1.16);

\path[draw=drawColor,line width= 0.4pt,line join=round,line cap=round,fill=fillColor] (181.66,535.03) circle (  1.16);

\path[draw=drawColor,line width= 0.4pt,line join=round,line cap=round,fill=fillColor] (181.88,535.03) circle (  1.16);

\path[draw=drawColor,line width= 0.4pt,line join=round,line cap=round,fill=fillColor] (182.10,535.03) circle (  1.16);

\path[draw=drawColor,line width= 0.4pt,line join=round,line cap=round,fill=fillColor] (182.32,535.03) circle (  1.16);

\path[draw=drawColor,line width= 0.4pt,line join=round,line cap=round,fill=fillColor] (182.54,535.03) circle (  1.16);

\path[draw=drawColor,line width= 0.4pt,line join=round,line cap=round,fill=fillColor] (182.75,535.03) circle (  1.16);

\path[draw=drawColor,line width= 0.4pt,line join=round,line cap=round,fill=fillColor] (182.97,535.03) circle (  1.16);

\path[draw=drawColor,line width= 0.4pt,line join=round,line cap=round,fill=fillColor] (183.19,535.03) circle (  1.16);

\path[draw=drawColor,line width= 0.4pt,line join=round,line cap=round,fill=fillColor] (183.40,535.03) circle (  1.16);

\path[draw=drawColor,line width= 0.4pt,line join=round,line cap=round,fill=fillColor] (183.61,535.03) circle (  1.16);

\path[draw=drawColor,line width= 0.4pt,line join=round,line cap=round,fill=fillColor] (183.83,535.03) circle (  1.16);

\path[draw=drawColor,line width= 0.4pt,line join=round,line cap=round,fill=fillColor] (184.04,535.03) circle (  1.16);

\path[draw=drawColor,line width= 0.4pt,line join=round,line cap=round,fill=fillColor] (184.25,535.03) circle (  1.16);

\path[draw=drawColor,line width= 0.4pt,line join=round,line cap=round,fill=fillColor] (184.47,535.03) circle (  1.16);

\path[draw=drawColor,line width= 0.4pt,line join=round,line cap=round,fill=fillColor] (184.68,535.03) circle (  1.16);

\path[draw=drawColor,line width= 0.4pt,line join=round,line cap=round,fill=fillColor] (184.89,535.03) circle (  1.16);

\path[draw=drawColor,line width= 0.4pt,line join=round,line cap=round,fill=fillColor] (185.10,535.03) circle (  1.16);

\path[draw=drawColor,line width= 0.4pt,line join=round,line cap=round,fill=fillColor] (185.31,535.03) circle (  1.16);

\path[draw=drawColor,line width= 0.4pt,line join=round,line cap=round,fill=fillColor] (185.51,535.03) circle (  1.16);

\path[draw=drawColor,line width= 0.4pt,line join=round,line cap=round,fill=fillColor] (185.72,535.03) circle (  1.16);

\path[draw=drawColor,line width= 0.4pt,line join=round,line cap=round,fill=fillColor] (185.93,535.03) circle (  1.16);

\path[draw=drawColor,line width= 0.4pt,line join=round,line cap=round,fill=fillColor] (186.13,535.03) circle (  1.16);

\path[draw=drawColor,line width= 0.4pt,line join=round,line cap=round,fill=fillColor] (186.34,535.03) circle (  1.16);

\path[draw=drawColor,line width= 0.4pt,line join=round,line cap=round,fill=fillColor] (186.55,535.03) circle (  1.16);

\path[draw=drawColor,line width= 0.4pt,line join=round,line cap=round,fill=fillColor] (186.75,535.03) circle (  1.16);

\path[draw=drawColor,line width= 0.4pt,line join=round,line cap=round,fill=fillColor] (186.95,535.03) circle (  1.16);

\path[draw=drawColor,line width= 0.4pt,line join=round,line cap=round,fill=fillColor] (187.16,535.03) circle (  1.16);

\path[draw=drawColor,line width= 0.4pt,line join=round,line cap=round,fill=fillColor] (187.36,535.03) circle (  1.16);

\path[draw=drawColor,line width= 0.4pt,line join=round,line cap=round,fill=fillColor] (187.56,535.03) circle (  1.16);

\path[draw=drawColor,line width= 0.4pt,line join=round,line cap=round,fill=fillColor] (187.76,535.03) circle (  1.16);

\path[draw=drawColor,line width= 0.4pt,line join=round,line cap=round,fill=fillColor] (187.96,535.03) circle (  1.16);

\path[draw=drawColor,line width= 0.4pt,line join=round,line cap=round,fill=fillColor] (188.16,535.03) circle (  1.16);

\path[draw=drawColor,line width= 0.4pt,line join=round,line cap=round,fill=fillColor] (188.36,535.03) circle (  1.16);

\path[draw=drawColor,line width= 0.4pt,line join=round,line cap=round,fill=fillColor] (188.56,535.03) circle (  1.16);

\path[draw=drawColor,line width= 0.4pt,line join=round,line cap=round,fill=fillColor] (188.76,535.03) circle (  1.16);

\path[draw=drawColor,line width= 0.4pt,line join=round,line cap=round,fill=fillColor] (188.96,535.03) circle (  1.16);

\path[draw=drawColor,line width= 0.4pt,line join=round,line cap=round,fill=fillColor] (189.16,535.03) circle (  1.16);

\path[draw=drawColor,line width= 0.4pt,line join=round,line cap=round,fill=fillColor] (189.35,535.03) circle (  1.16);

\path[draw=drawColor,line width= 0.4pt,line join=round,line cap=round,fill=fillColor] (189.55,535.03) circle (  1.16);

\path[draw=drawColor,line width= 0.4pt,line join=round,line cap=round,fill=fillColor] (189.74,535.03) circle (  1.16);

\path[draw=drawColor,line width= 0.4pt,line join=round,line cap=round,fill=fillColor] (189.94,535.03) circle (  1.16);

\path[draw=drawColor,line width= 0.4pt,line join=round,line cap=round,fill=fillColor] (190.13,535.03) circle (  1.16);

\path[draw=drawColor,line width= 0.4pt,line join=round,line cap=round,fill=fillColor] (190.33,535.03) circle (  1.16);

\path[draw=drawColor,line width= 0.4pt,line join=round,line cap=round,fill=fillColor] (190.52,535.03) circle (  1.16);

\path[draw=drawColor,line width= 0.4pt,line join=round,line cap=round,fill=fillColor] (190.71,535.03) circle (  1.16);

\path[draw=drawColor,line width= 0.4pt,line join=round,line cap=round,fill=fillColor] (190.91,535.03) circle (  1.16);

\path[draw=drawColor,line width= 0.4pt,line join=round,line cap=round,fill=fillColor] (191.10,535.03) circle (  1.16);

\path[draw=drawColor,line width= 0.4pt,line join=round,line cap=round,fill=fillColor] (191.29,535.03) circle (  1.16);

\path[draw=drawColor,line width= 0.4pt,line join=round,line cap=round,fill=fillColor] (191.48,535.03) circle (  1.16);

\path[draw=drawColor,line width= 0.4pt,line join=round,line cap=round,fill=fillColor] (191.67,535.03) circle (  1.16);

\path[draw=drawColor,line width= 0.4pt,line join=round,line cap=round,fill=fillColor] (191.86,535.03) circle (  1.16);

\path[draw=drawColor,line width= 0.4pt,line join=round,line cap=round,fill=fillColor] (192.05,535.03) circle (  1.16);

\path[draw=drawColor,line width= 0.4pt,line join=round,line cap=round,fill=fillColor] (192.24,535.03) circle (  1.16);

\path[draw=drawColor,line width= 0.4pt,line join=round,line cap=round,fill=fillColor] (192.43,535.03) circle (  1.16);

\path[draw=drawColor,line width= 0.4pt,line join=round,line cap=round,fill=fillColor] (192.61,535.03) circle (  1.16);

\path[draw=drawColor,line width= 0.4pt,line join=round,line cap=round,fill=fillColor] (192.80,535.03) circle (  1.16);

\path[draw=drawColor,line width= 0.4pt,line join=round,line cap=round,fill=fillColor] (192.99,535.03) circle (  1.16);

\path[draw=drawColor,line width= 0.4pt,line join=round,line cap=round,fill=fillColor] (193.17,535.03) circle (  1.16);

\path[draw=drawColor,line width= 0.4pt,line join=round,line cap=round,fill=fillColor] (193.36,535.03) circle (  1.16);

\path[draw=drawColor,line width= 0.4pt,line join=round,line cap=round,fill=fillColor] (193.54,535.03) circle (  1.16);

\path[draw=drawColor,line width= 0.4pt,line join=round,line cap=round,fill=fillColor] (193.73,535.03) circle (  1.16);

\path[draw=drawColor,line width= 0.4pt,line join=round,line cap=round,fill=fillColor] (193.91,535.03) circle (  1.16);

\path[draw=drawColor,line width= 0.4pt,line join=round,line cap=round,fill=fillColor] (194.10,535.03) circle (  1.16);

\path[draw=drawColor,line width= 0.4pt,line join=round,line cap=round,fill=fillColor] (194.28,535.03) circle (  1.16);

\path[draw=drawColor,line width= 0.4pt,line join=round,line cap=round,fill=fillColor] (194.46,535.03) circle (  1.16);

\path[draw=drawColor,line width= 0.4pt,line join=round,line cap=round,fill=fillColor] (194.65,535.03) circle (  1.16);

\path[draw=drawColor,line width= 0.4pt,line join=round,line cap=round,fill=fillColor] (194.83,535.03) circle (  1.16);

\path[draw=drawColor,line width= 0.4pt,line join=round,line cap=round,fill=fillColor] (195.01,535.03) circle (  1.16);

\path[draw=drawColor,line width= 0.4pt,line join=round,line cap=round,fill=fillColor] (195.19,535.03) circle (  1.16);

\path[draw=drawColor,line width= 0.4pt,line join=round,line cap=round,fill=fillColor] (195.37,535.03) circle (  1.16);

\path[draw=drawColor,line width= 0.4pt,line join=round,line cap=round,fill=fillColor] (195.55,535.03) circle (  1.16);

\path[draw=drawColor,line width= 0.4pt,line join=round,line cap=round,fill=fillColor] (195.73,535.03) circle (  1.16);

\path[draw=drawColor,line width= 0.4pt,line join=round,line cap=round,fill=fillColor] (195.91,535.03) circle (  1.16);

\path[draw=drawColor,line width= 0.4pt,line join=round,line cap=round,fill=fillColor] (196.09,535.03) circle (  1.16);

\path[draw=drawColor,line width= 0.4pt,line join=round,line cap=round,fill=fillColor] (196.27,535.03) circle (  1.16);

\path[draw=drawColor,line width= 0.4pt,line join=round,line cap=round,fill=fillColor] (196.44,535.03) circle (  1.16);

\path[draw=drawColor,line width= 0.4pt,line join=round,line cap=round,fill=fillColor] (196.62,535.03) circle (  1.16);

\path[draw=drawColor,line width= 0.4pt,line join=round,line cap=round,fill=fillColor] (196.80,535.03) circle (  1.16);

\path[draw=drawColor,line width= 0.4pt,line join=round,line cap=round,fill=fillColor] (196.97,535.03) circle (  1.16);

\path[draw=drawColor,line width= 0.4pt,line join=round,line cap=round,fill=fillColor] (197.15,535.03) circle (  1.16);

\path[draw=drawColor,line width= 0.4pt,line join=round,line cap=round,fill=fillColor] (197.33,535.03) circle (  1.16);

\path[draw=drawColor,line width= 0.4pt,line join=round,line cap=round,fill=fillColor] (197.50,535.03) circle (  1.16);

\path[draw=drawColor,line width= 0.4pt,line join=round,line cap=round,fill=fillColor] (197.68,535.03) circle (  1.16);

\path[draw=drawColor,line width= 0.4pt,line join=round,line cap=round,fill=fillColor] (197.85,535.03) circle (  1.16);

\path[draw=drawColor,line width= 0.4pt,line join=round,line cap=round,fill=fillColor] (198.02,535.03) circle (  1.16);

\path[draw=drawColor,line width= 0.4pt,line join=round,line cap=round,fill=fillColor] (198.20,535.03) circle (  1.16);

\path[draw=drawColor,line width= 0.4pt,line join=round,line cap=round,fill=fillColor] (198.37,535.03) circle (  1.16);

\path[draw=drawColor,line width= 0.4pt,line join=round,line cap=round,fill=fillColor] (198.54,535.03) circle (  1.16);

\path[draw=drawColor,line width= 0.4pt,line join=round,line cap=round,fill=fillColor] (198.72,535.03) circle (  1.16);

\path[draw=drawColor,line width= 0.4pt,line join=round,line cap=round,fill=fillColor] (198.89,535.03) circle (  1.16);

\path[draw=drawColor,line width= 0.4pt,line join=round,line cap=round,fill=fillColor] (199.06,535.03) circle (  1.16);

\path[draw=drawColor,line width= 0.4pt,line join=round,line cap=round,fill=fillColor] (199.23,535.03) circle (  1.16);

\path[draw=drawColor,line width= 0.4pt,line join=round,line cap=round,fill=fillColor] (199.40,535.03) circle (  1.16);

\path[draw=drawColor,line width= 0.4pt,line join=round,line cap=round,fill=fillColor] (199.57,535.03) circle (  1.16);

\path[draw=drawColor,line width= 0.4pt,line join=round,line cap=round,fill=fillColor] (199.74,535.03) circle (  1.16);

\path[draw=drawColor,line width= 0.4pt,line join=round,line cap=round,fill=fillColor] (199.91,535.03) circle (  1.16);

\path[draw=drawColor,line width= 0.4pt,line join=round,line cap=round,fill=fillColor] (200.08,535.03) circle (  1.16);

\path[draw=drawColor,line width= 0.4pt,line join=round,line cap=round,fill=fillColor] (200.25,535.03) circle (  1.16);

\path[draw=drawColor,line width= 0.4pt,line join=round,line cap=round,fill=fillColor] (200.42,535.03) circle (  1.16);

\path[draw=drawColor,line width= 0.4pt,line join=round,line cap=round,fill=fillColor] (200.58,535.03) circle (  1.16);

\path[draw=drawColor,line width= 0.4pt,line join=round,line cap=round,fill=fillColor] (200.75,535.03) circle (  1.16);

\path[draw=drawColor,line width= 0.4pt,line join=round,line cap=round,fill=fillColor] (200.92,535.03) circle (  1.16);

\path[draw=drawColor,line width= 0.4pt,line join=round,line cap=round,fill=fillColor] (201.09,535.03) circle (  1.16);

\path[draw=drawColor,line width= 0.4pt,line join=round,line cap=round,fill=fillColor] (201.25,535.03) circle (  1.16);

\path[draw=drawColor,line width= 0.4pt,line join=round,line cap=round,fill=fillColor] (201.42,535.03) circle (  1.16);

\path[draw=drawColor,line width= 0.4pt,line join=round,line cap=round,fill=fillColor] (201.58,535.03) circle (  1.16);

\path[draw=drawColor,line width= 0.4pt,line join=round,line cap=round,fill=fillColor] (201.75,535.03) circle (  1.16);

\path[draw=drawColor,line width= 0.4pt,line join=round,line cap=round,fill=fillColor] (201.91,535.03) circle (  1.16);

\path[draw=drawColor,line width= 0.4pt,line join=round,line cap=round,fill=fillColor] (202.08,535.03) circle (  1.16);

\path[draw=drawColor,line width= 0.4pt,line join=round,line cap=round,fill=fillColor] (202.24,535.03) circle (  1.16);

\path[draw=drawColor,line width= 0.4pt,line join=round,line cap=round,fill=fillColor] (202.41,535.03) circle (  1.16);

\path[draw=drawColor,line width= 0.4pt,line join=round,line cap=round,fill=fillColor] (202.57,535.03) circle (  1.16);

\path[draw=drawColor,line width= 0.4pt,line join=round,line cap=round,fill=fillColor] (202.73,535.03) circle (  1.16);

\path[draw=drawColor,line width= 0.4pt,line join=round,line cap=round,fill=fillColor] (202.90,535.03) circle (  1.16);

\path[draw=drawColor,line width= 0.4pt,line join=round,line cap=round,fill=fillColor] (203.06,535.03) circle (  1.16);

\path[draw=drawColor,line width= 0.4pt,line join=round,line cap=round,fill=fillColor] (203.22,535.03) circle (  1.16);

\path[draw=drawColor,line width= 0.4pt,line join=round,line cap=round,fill=fillColor] (203.38,535.03) circle (  1.16);

\path[draw=drawColor,line width= 0.4pt,line join=round,line cap=round,fill=fillColor] (203.54,535.03) circle (  1.16);

\path[draw=drawColor,line width= 0.4pt,line join=round,line cap=round,fill=fillColor] (203.71,535.03) circle (  1.16);

\path[draw=drawColor,line width= 0.4pt,line join=round,line cap=round,fill=fillColor] (203.87,535.03) circle (  1.16);

\path[draw=drawColor,line width= 0.4pt,line join=round,line cap=round,fill=fillColor] (204.03,535.03) circle (  1.16);

\path[draw=drawColor,line width= 0.4pt,line join=round,line cap=round,fill=fillColor] (204.19,535.03) circle (  1.16);

\path[draw=drawColor,line width= 0.4pt,line join=round,line cap=round,fill=fillColor] (204.35,535.03) circle (  1.16);

\path[draw=drawColor,line width= 0.4pt,line join=round,line cap=round,fill=fillColor] (204.51,535.03) circle (  1.16);

\path[draw=drawColor,line width= 0.4pt,line join=round,line cap=round,fill=fillColor] (204.66,535.03) circle (  1.16);

\path[draw=drawColor,line width= 0.4pt,line join=round,line cap=round,fill=fillColor] (204.82,535.03) circle (  1.16);

\path[draw=drawColor,line width= 0.4pt,line join=round,line cap=round,fill=fillColor] (204.98,535.03) circle (  1.16);

\path[draw=drawColor,line width= 0.4pt,line join=round,line cap=round,fill=fillColor] (205.14,535.03) circle (  1.16);

\path[draw=drawColor,line width= 0.4pt,line join=round,line cap=round,fill=fillColor] (205.30,535.03) circle (  1.16);

\path[draw=drawColor,line width= 0.4pt,line join=round,line cap=round,fill=fillColor] (205.45,535.03) circle (  1.16);

\path[draw=drawColor,line width= 0.4pt,line join=round,line cap=round,fill=fillColor] (205.61,535.03) circle (  1.16);

\path[draw=drawColor,line width= 0.4pt,line join=round,line cap=round,fill=fillColor] (205.77,535.03) circle (  1.16);

\path[draw=drawColor,line width= 0.4pt,line join=round,line cap=round,fill=fillColor] (205.92,535.03) circle (  1.16);

\path[draw=drawColor,line width= 0.4pt,line join=round,line cap=round,fill=fillColor] (206.08,535.03) circle (  1.16);

\path[draw=drawColor,line width= 0.4pt,line join=round,line cap=round,fill=fillColor] (206.24,535.03) circle (  1.16);

\path[draw=drawColor,line width= 0.4pt,line join=round,line cap=round,fill=fillColor] (206.39,535.03) circle (  1.16);

\path[draw=drawColor,line width= 0.4pt,line join=round,line cap=round,fill=fillColor] (206.55,535.03) circle (  1.16);

\path[draw=drawColor,line width= 0.4pt,line join=round,line cap=round,fill=fillColor] (206.70,535.03) circle (  1.16);

\path[draw=drawColor,line width= 0.4pt,line join=round,line cap=round,fill=fillColor] (206.86,535.03) circle (  1.16);

\path[draw=drawColor,line width= 0.4pt,line join=round,line cap=round,fill=fillColor] (207.01,535.03) circle (  1.16);

\path[draw=drawColor,line width= 0.4pt,line join=round,line cap=round,fill=fillColor] (207.16,535.03) circle (  1.16);

\path[draw=drawColor,line width= 0.4pt,line join=round,line cap=round,fill=fillColor] (207.32,535.03) circle (  1.16);

\path[draw=drawColor,line width= 0.4pt,line join=round,line cap=round,fill=fillColor] (207.47,535.03) circle (  1.16);

\path[draw=drawColor,line width= 0.4pt,line join=round,line cap=round,fill=fillColor] (207.62,535.03) circle (  1.16);

\path[draw=drawColor,line width= 0.4pt,line join=round,line cap=round,fill=fillColor] (207.78,535.03) circle (  1.16);

\path[draw=drawColor,line width= 0.4pt,line join=round,line cap=round,fill=fillColor] (207.93,535.03) circle (  1.16);

\path[draw=drawColor,line width= 0.4pt,line join=round,line cap=round,fill=fillColor] (208.08,535.03) circle (  1.16);

\path[draw=drawColor,line width= 0.4pt,line join=round,line cap=round,fill=fillColor] (208.23,535.03) circle (  1.16);

\path[draw=drawColor,line width= 0.4pt,line join=round,line cap=round,fill=fillColor] (208.39,535.03) circle (  1.16);

\path[draw=drawColor,line width= 0.4pt,line join=round,line cap=round,fill=fillColor] (208.54,535.03) circle (  1.16);

\path[draw=drawColor,line width= 0.4pt,line join=round,line cap=round,fill=fillColor] (208.69,535.03) circle (  1.16);

\path[draw=drawColor,line width= 0.4pt,line join=round,line cap=round,fill=fillColor] (208.84,535.03) circle (  1.16);

\path[draw=drawColor,line width= 0.4pt,line join=round,line cap=round,fill=fillColor] (208.99,535.03) circle (  1.16);

\path[draw=drawColor,line width= 0.4pt,line join=round,line cap=round,fill=fillColor] (209.14,535.03) circle (  1.16);

\path[draw=drawColor,line width= 0.4pt,line join=round,line cap=round,fill=fillColor] (209.29,535.03) circle (  1.16);

\path[draw=drawColor,line width= 0.4pt,line join=round,line cap=round,fill=fillColor] (209.44,535.03) circle (  1.16);

\path[draw=drawColor,line width= 0.4pt,line join=round,line cap=round,fill=fillColor] (209.59,535.03) circle (  1.16);

\path[draw=drawColor,line width= 0.4pt,line join=round,line cap=round,fill=fillColor] (209.74,535.03) circle (  1.16);

\path[draw=drawColor,line width= 0.4pt,line join=round,line cap=round,fill=fillColor] (209.88,535.03) circle (  1.16);

\path[draw=drawColor,line width= 0.4pt,line join=round,line cap=round,fill=fillColor] (210.03,535.03) circle (  1.16);

\path[draw=drawColor,line width= 0.4pt,line join=round,line cap=round,fill=fillColor] (210.18,535.03) circle (  1.16);

\path[draw=drawColor,line width= 0.4pt,line join=round,line cap=round,fill=fillColor] (210.33,535.03) circle (  1.16);

\path[draw=drawColor,line width= 0.4pt,line join=round,line cap=round,fill=fillColor] (210.48,535.03) circle (  1.16);

\path[draw=drawColor,line width= 0.4pt,line join=round,line cap=round,fill=fillColor] (210.62,535.03) circle (  1.16);

\path[draw=drawColor,line width= 0.4pt,line join=round,line cap=round,fill=fillColor] (210.77,535.03) circle (  1.16);

\path[draw=drawColor,line width= 0.4pt,line join=round,line cap=round,fill=fillColor] (210.92,535.03) circle (  1.16);

\path[draw=drawColor,line width= 0.4pt,line join=round,line cap=round,fill=fillColor] (211.06,535.03) circle (  1.16);

\path[draw=drawColor,line width= 0.4pt,line join=round,line cap=round,fill=fillColor] (211.21,535.03) circle (  1.16);

\path[draw=drawColor,line width= 0.4pt,line join=round,line cap=round,fill=fillColor] (211.36,535.03) circle (  1.16);

\path[draw=drawColor,line width= 0.4pt,line join=round,line cap=round,fill=fillColor] (211.50,535.03) circle (  1.16);

\path[draw=drawColor,line width= 0.4pt,line join=round,line cap=round,fill=fillColor] (211.65,535.03) circle (  1.16);

\path[draw=drawColor,line width= 0.4pt,line join=round,line cap=round,fill=fillColor] (211.79,535.03) circle (  1.16);

\path[draw=drawColor,line width= 0.4pt,line join=round,line cap=round,fill=fillColor] (211.94,535.03) circle (  1.16);

\path[draw=drawColor,line width= 0.4pt,line join=round,line cap=round,fill=fillColor] (212.08,535.03) circle (  1.16);

\path[draw=drawColor,line width= 0.4pt,line join=round,line cap=round,fill=fillColor] (212.23,535.03) circle (  1.16);

\path[draw=drawColor,line width= 0.4pt,line join=round,line cap=round,fill=fillColor] (212.37,535.03) circle (  1.16);

\path[draw=drawColor,line width= 0.4pt,line join=round,line cap=round,fill=fillColor] (212.51,535.03) circle (  1.16);

\path[draw=drawColor,line width= 0.4pt,line join=round,line cap=round,fill=fillColor] (212.66,535.03) circle (  1.16);

\path[draw=drawColor,line width= 0.4pt,line join=round,line cap=round,fill=fillColor] (212.80,535.03) circle (  1.16);

\path[draw=drawColor,line width= 0.4pt,line join=round,line cap=round,fill=fillColor] (212.94,535.03) circle (  1.16);

\path[draw=drawColor,line width= 0.4pt,line join=round,line cap=round,fill=fillColor] (213.09,535.03) circle (  1.16);

\path[draw=drawColor,line width= 0.4pt,line join=round,line cap=round,fill=fillColor] (213.23,535.03) circle (  1.16);

\path[draw=drawColor,line width= 0.4pt,line join=round,line cap=round,fill=fillColor] (213.37,535.03) circle (  1.16);

\path[draw=drawColor,line width= 0.4pt,line join=round,line cap=round,fill=fillColor] (213.51,535.03) circle (  1.16);

\path[draw=drawColor,line width= 0.4pt,line join=round,line cap=round,fill=fillColor] (213.65,535.03) circle (  1.16);

\path[draw=drawColor,line width= 0.4pt,line join=round,line cap=round,fill=fillColor] (213.80,535.03) circle (  1.16);

\path[draw=drawColor,line width= 0.4pt,line join=round,line cap=round,fill=fillColor] (213.94,535.03) circle (  1.16);

\path[draw=drawColor,line width= 0.4pt,line join=round,line cap=round,fill=fillColor] (214.08,535.03) circle (  1.16);

\path[draw=drawColor,line width= 0.4pt,line join=round,line cap=round,fill=fillColor] (214.22,535.03) circle (  1.16);

\path[draw=drawColor,line width= 0.4pt,line join=round,line cap=round,fill=fillColor] (214.36,535.03) circle (  1.16);

\path[draw=drawColor,line width= 0.4pt,line join=round,line cap=round,fill=fillColor] (214.50,535.03) circle (  1.16);

\path[draw=drawColor,line width= 0.4pt,line join=round,line cap=round,fill=fillColor] (214.64,535.03) circle (  1.16);

\path[draw=drawColor,line width= 0.4pt,line join=round,line cap=round,fill=fillColor] (214.78,535.03) circle (  1.16);

\path[draw=drawColor,line width= 0.4pt,line join=round,line cap=round,fill=fillColor] (214.92,535.03) circle (  1.16);

\path[draw=drawColor,line width= 0.4pt,line join=round,line cap=round,fill=fillColor] (215.06,535.03) circle (  1.16);

\path[draw=drawColor,line width= 0.4pt,line join=round,line cap=round,fill=fillColor] (215.20,535.03) circle (  1.16);

\path[draw=drawColor,line width= 0.4pt,line join=round,line cap=round,fill=fillColor] (215.34,535.03) circle (  1.16);

\path[draw=drawColor,line width= 0.4pt,line join=round,line cap=round,fill=fillColor] (215.47,535.03) circle (  1.16);

\path[draw=drawColor,line width= 0.4pt,line join=round,line cap=round,fill=fillColor] (215.61,535.03) circle (  1.16);

\path[draw=drawColor,line width= 0.4pt,line join=round,line cap=round,fill=fillColor] (215.75,535.03) circle (  1.16);

\path[draw=drawColor,line width= 0.4pt,line join=round,line cap=round,fill=fillColor] (215.89,535.03) circle (  1.16);

\path[draw=drawColor,line width= 0.4pt,line join=round,line cap=round,fill=fillColor] (216.03,535.03) circle (  1.16);

\path[draw=drawColor,line width= 0.4pt,line join=round,line cap=round,fill=fillColor] (216.16,535.03) circle (  1.16);

\path[draw=drawColor,line width= 0.4pt,line join=round,line cap=round,fill=fillColor] (216.30,535.03) circle (  1.16);

\path[draw=drawColor,line width= 0.4pt,line join=round,line cap=round,fill=fillColor] (216.44,535.03) circle (  1.16);

\path[draw=drawColor,line width= 0.4pt,line join=round,line cap=round,fill=fillColor] (216.58,535.03) circle (  1.16);

\path[draw=drawColor,line width= 0.4pt,line join=round,line cap=round,fill=fillColor] (216.71,535.03) circle (  1.16);

\path[draw=drawColor,line width= 0.4pt,line join=round,line cap=round,fill=fillColor] (216.85,535.03) circle (  1.16);

\path[draw=drawColor,line width= 0.4pt,line join=round,line cap=round,fill=fillColor] (216.98,535.03) circle (  1.16);

\path[draw=drawColor,line width= 0.4pt,line join=round,line cap=round,fill=fillColor] (217.12,535.03) circle (  1.16);

\path[draw=drawColor,line width= 0.4pt,line join=round,line cap=round,fill=fillColor] (217.26,535.03) circle (  1.16);

\path[draw=drawColor,line width= 0.4pt,line join=round,line cap=round,fill=fillColor] (217.39,535.03) circle (  1.16);

\path[draw=drawColor,line width= 0.4pt,line join=round,line cap=round,fill=fillColor] (217.53,535.03) circle (  1.16);

\path[draw=drawColor,line width= 0.4pt,line join=round,line cap=round,fill=fillColor] (217.66,535.03) circle (  1.16);

\path[draw=drawColor,line width= 0.4pt,line join=round,line cap=round,fill=fillColor] (217.80,535.03) circle (  1.16);

\path[draw=drawColor,line width= 0.4pt,line join=round,line cap=round,fill=fillColor] (217.93,535.03) circle (  1.16);

\path[draw=drawColor,line width= 0.4pt,line join=round,line cap=round,fill=fillColor] (218.06,535.03) circle (  1.16);

\path[draw=drawColor,line width= 0.4pt,line join=round,line cap=round,fill=fillColor] (218.20,535.03) circle (  1.16);

\path[draw=drawColor,line width= 0.4pt,line join=round,line cap=round,fill=fillColor] (218.33,535.03) circle (  1.16);

\path[draw=drawColor,line width= 0.4pt,line join=round,line cap=round,fill=fillColor] (218.47,535.03) circle (  1.16);

\path[draw=drawColor,line width= 0.4pt,line join=round,line cap=round,fill=fillColor] (218.60,535.03) circle (  1.16);

\path[draw=drawColor,line width= 0.4pt,line join=round,line cap=round,fill=fillColor] (218.73,535.03) circle (  1.16);

\path[draw=drawColor,line width= 0.4pt,line join=round,line cap=round,fill=fillColor] (218.87,535.03) circle (  1.16);

\path[draw=drawColor,line width= 0.4pt,line join=round,line cap=round,fill=fillColor] (219.00,535.03) circle (  1.16);

\path[draw=drawColor,line width= 0.4pt,line join=round,line cap=round,fill=fillColor] (219.13,535.03) circle (  1.16);

\path[draw=drawColor,line width= 0.4pt,line join=round,line cap=round,fill=fillColor] (219.26,535.03) circle (  1.16);

\path[draw=drawColor,line width= 0.4pt,line join=round,line cap=round,fill=fillColor] (219.40,535.03) circle (  1.16);

\path[draw=drawColor,line width= 0.4pt,line join=round,line cap=round,fill=fillColor] (219.53,535.03) circle (  1.16);

\path[draw=drawColor,line width= 0.4pt,line join=round,line cap=round,fill=fillColor] (219.66,535.03) circle (  1.16);

\path[draw=drawColor,line width= 0.4pt,line join=round,line cap=round,fill=fillColor] (219.79,535.03) circle (  1.16);

\path[draw=drawColor,line width= 0.4pt,line join=round,line cap=round,fill=fillColor] (219.92,535.03) circle (  1.16);

\path[draw=drawColor,line width= 0.4pt,line join=round,line cap=round,fill=fillColor] (220.05,535.03) circle (  1.16);

\path[draw=drawColor,line width= 0.4pt,line join=round,line cap=round,fill=fillColor] (220.18,535.03) circle (  1.16);

\path[draw=drawColor,line width= 0.4pt,line join=round,line cap=round,fill=fillColor] (220.31,535.03) circle (  1.16);

\path[draw=drawColor,line width= 0.4pt,line join=round,line cap=round,fill=fillColor] (220.45,535.03) circle (  1.16);

\path[draw=drawColor,line width= 0.4pt,line join=round,line cap=round,fill=fillColor] (220.58,535.03) circle (  1.16);

\path[draw=drawColor,line width= 0.4pt,line join=round,line cap=round,fill=fillColor] (220.71,535.03) circle (  1.16);

\path[draw=drawColor,line width= 0.4pt,line join=round,line cap=round,fill=fillColor] (220.84,535.03) circle (  1.16);

\path[draw=drawColor,line width= 0.4pt,line join=round,line cap=round,fill=fillColor] (220.97,535.03) circle (  1.16);

\path[draw=drawColor,line width= 0.4pt,line join=round,line cap=round,fill=fillColor] (221.09,535.03) circle (  1.16);

\path[draw=drawColor,line width= 0.4pt,line join=round,line cap=round,fill=fillColor] (221.22,535.03) circle (  1.16);

\path[draw=drawColor,line width= 0.4pt,line join=round,line cap=round,fill=fillColor] (221.35,535.03) circle (  1.16);

\path[draw=drawColor,line width= 0.4pt,line join=round,line cap=round,fill=fillColor] (221.48,535.03) circle (  1.16);

\path[draw=drawColor,line width= 0.4pt,line join=round,line cap=round,fill=fillColor] (221.61,535.03) circle (  1.16);

\path[draw=drawColor,line width= 0.4pt,line join=round,line cap=round,fill=fillColor] (221.74,535.03) circle (  1.16);

\path[draw=drawColor,line width= 0.4pt,line join=round,line cap=round,fill=fillColor] (221.87,535.03) circle (  1.16);

\path[draw=drawColor,line width= 0.4pt,line join=round,line cap=round,fill=fillColor] (222.00,535.03) circle (  1.16);

\path[draw=drawColor,line width= 0.4pt,line join=round,line cap=round,fill=fillColor] (222.12,535.03) circle (  1.16);

\path[draw=drawColor,line width= 0.4pt,line join=round,line cap=round,fill=fillColor] (222.25,535.03) circle (  1.16);

\path[draw=drawColor,line width= 0.4pt,line join=round,line cap=round,fill=fillColor] (222.38,535.03) circle (  1.16);

\path[draw=drawColor,line width= 0.4pt,line join=round,line cap=round,fill=fillColor] (222.51,535.03) circle (  1.16);

\path[draw=drawColor,line width= 0.4pt,line join=round,line cap=round,fill=fillColor] (222.63,535.03) circle (  1.16);

\path[draw=drawColor,line width= 0.4pt,line join=round,line cap=round,fill=fillColor] (222.76,535.03) circle (  1.16);

\path[draw=drawColor,line width= 0.4pt,line join=round,line cap=round,fill=fillColor] (222.89,535.03) circle (  1.16);

\path[draw=drawColor,line width= 0.4pt,line join=round,line cap=round,fill=fillColor] (223.01,535.03) circle (  1.16);

\path[draw=drawColor,line width= 0.4pt,line join=round,line cap=round,fill=fillColor] (223.14,535.03) circle (  1.16);

\path[draw=drawColor,line width= 0.4pt,line join=round,line cap=round,fill=fillColor] (223.27,535.03) circle (  1.16);

\path[draw=drawColor,line width= 0.4pt,line join=round,line cap=round,fill=fillColor] (223.39,535.03) circle (  1.16);

\path[draw=drawColor,line width= 0.4pt,line join=round,line cap=round,fill=fillColor] (223.52,535.03) circle (  1.16);

\path[draw=drawColor,line width= 0.4pt,line join=round,line cap=round,fill=fillColor] (223.64,535.03) circle (  1.16);

\path[draw=drawColor,line width= 0.4pt,line join=round,line cap=round,fill=fillColor] (223.77,535.03) circle (  1.16);

\path[draw=drawColor,line width= 0.4pt,line join=round,line cap=round,fill=fillColor] (223.89,535.03) circle (  1.16);

\path[draw=drawColor,line width= 0.4pt,line join=round,line cap=round,fill=fillColor] (224.02,535.03) circle (  1.16);

\path[draw=drawColor,line width= 0.4pt,line join=round,line cap=round,fill=fillColor] (224.14,535.03) circle (  1.16);

\path[draw=drawColor,line width= 0.4pt,line join=round,line cap=round,fill=fillColor] (224.27,535.03) circle (  1.16);

\path[draw=drawColor,line width= 0.4pt,line join=round,line cap=round,fill=fillColor] (224.39,535.03) circle (  1.16);

\path[draw=drawColor,line width= 0.4pt,line join=round,line cap=round,fill=fillColor] (224.52,535.03) circle (  1.16);

\path[draw=drawColor,line width= 0.4pt,line join=round,line cap=round,fill=fillColor] (224.64,535.03) circle (  1.16);

\path[draw=drawColor,line width= 0.4pt,line join=round,line cap=round,fill=fillColor] (224.77,535.03) circle (  1.16);

\path[draw=drawColor,line width= 0.4pt,line join=round,line cap=round,fill=fillColor] (224.89,535.03) circle (  1.16);

\path[draw=drawColor,line width= 0.4pt,line join=round,line cap=round,fill=fillColor] (225.01,535.03) circle (  1.16);

\path[draw=drawColor,line width= 0.4pt,line join=round,line cap=round,fill=fillColor] (225.14,535.03) circle (  1.16);

\path[draw=drawColor,line width= 0.4pt,line join=round,line cap=round,fill=fillColor] (225.26,535.03) circle (  1.16);
\definecolor[named]{drawColor}{rgb}{0.22,0.49,0.72}
\definecolor[named]{fillColor}{rgb}{0.22,0.49,0.72}

\path[draw=drawColor,line width= 0.4pt,line join=round,line cap=round,fill=fillColor] ( 74.88,617.98) circle (  1.16);

\path[draw=drawColor,line width= 0.4pt,line join=round,line cap=round,fill=fillColor] ( 80.66,617.79) circle (  1.16);

\path[draw=drawColor,line width= 0.4pt,line join=round,line cap=round,fill=fillColor] ( 84.72,617.77) circle (  1.16);

\path[draw=drawColor,line width= 0.4pt,line join=round,line cap=round,fill=fillColor] ( 87.95,617.75) circle (  1.16);

\path[draw=drawColor,line width= 0.4pt,line join=round,line cap=round,fill=fillColor] ( 90.68,617.73) circle (  1.16);

\path[draw=drawColor,line width= 0.4pt,line join=round,line cap=round,fill=fillColor] ( 93.06,617.57) circle (  1.16);

\path[draw=drawColor,line width= 0.4pt,line join=round,line cap=round,fill=fillColor] ( 95.19,617.53) circle (  1.16);

\path[draw=drawColor,line width= 0.4pt,line join=round,line cap=round,fill=fillColor] ( 97.13,617.24) circle (  1.16);

\path[draw=drawColor,line width= 0.4pt,line join=round,line cap=round,fill=fillColor] ( 98.91,617.18) circle (  1.16);

\path[draw=drawColor,line width= 0.4pt,line join=round,line cap=round,fill=fillColor] (100.57,617.16) circle (  1.16);

\path[draw=drawColor,line width= 0.4pt,line join=round,line cap=round,fill=fillColor] (102.11,616.70) circle (  1.16);

\path[draw=drawColor,line width= 0.4pt,line join=round,line cap=round,fill=fillColor] (103.57,616.57) circle (  1.16);

\path[draw=drawColor,line width= 0.4pt,line join=round,line cap=round,fill=fillColor] (104.95,616.54) circle (  1.16);

\path[draw=drawColor,line width= 0.4pt,line join=round,line cap=round,fill=fillColor] (106.26,616.31) circle (  1.16);

\path[draw=drawColor,line width= 0.4pt,line join=round,line cap=round,fill=fillColor] (107.50,616.23) circle (  1.16);

\path[draw=drawColor,line width= 0.4pt,line join=round,line cap=round,fill=fillColor] (108.70,616.03) circle (  1.16);

\path[draw=drawColor,line width= 0.4pt,line join=round,line cap=round,fill=fillColor] (109.84,615.83) circle (  1.16);

\path[draw=drawColor,line width= 0.4pt,line join=round,line cap=round,fill=fillColor] (110.94,615.82) circle (  1.16);

\path[draw=drawColor,line width= 0.4pt,line join=round,line cap=round,fill=fillColor] (112.00,615.80) circle (  1.16);

\path[draw=drawColor,line width= 0.4pt,line join=round,line cap=round,fill=fillColor] (113.03,615.64) circle (  1.16);

\path[draw=drawColor,line width= 0.4pt,line join=round,line cap=round,fill=fillColor] (114.02,615.62) circle (  1.16);

\path[draw=drawColor,line width= 0.4pt,line join=round,line cap=round,fill=fillColor] (114.98,615.54) circle (  1.16);

\path[draw=drawColor,line width= 0.4pt,line join=round,line cap=round,fill=fillColor] (115.91,615.42) circle (  1.16);

\path[draw=drawColor,line width= 0.4pt,line join=round,line cap=round,fill=fillColor] (116.81,615.25) circle (  1.16);

\path[draw=drawColor,line width= 0.4pt,line join=round,line cap=round,fill=fillColor] (117.69,615.20) circle (  1.16);

\path[draw=drawColor,line width= 0.4pt,line join=round,line cap=round,fill=fillColor] (118.55,615.14) circle (  1.16);

\path[draw=drawColor,line width= 0.4pt,line join=round,line cap=round,fill=fillColor] (119.38,615.13) circle (  1.16);

\path[draw=drawColor,line width= 0.4pt,line join=round,line cap=round,fill=fillColor] (120.20,614.50) circle (  1.16);

\path[draw=drawColor,line width= 0.4pt,line join=round,line cap=round,fill=fillColor] (120.99,613.42) circle (  1.16);

\path[draw=drawColor,line width= 0.4pt,line join=round,line cap=round,fill=fillColor] (121.77,613.07) circle (  1.16);

\path[draw=drawColor,line width= 0.4pt,line join=round,line cap=round,fill=fillColor] (122.53,612.72) circle (  1.16);

\path[draw=drawColor,line width= 0.4pt,line join=round,line cap=round,fill=fillColor] (123.27,612.47) circle (  1.16);

\path[draw=drawColor,line width= 0.4pt,line join=round,line cap=round,fill=fillColor] (124.00,612.19) circle (  1.16);

\path[draw=drawColor,line width= 0.4pt,line join=round,line cap=round,fill=fillColor] (124.71,611.67) circle (  1.16);

\path[draw=drawColor,line width= 0.4pt,line join=round,line cap=round,fill=fillColor] (125.41,611.30) circle (  1.16);

\path[draw=drawColor,line width= 0.4pt,line join=round,line cap=round,fill=fillColor] (126.10,609.28) circle (  1.16);

\path[draw=drawColor,line width= 0.4pt,line join=round,line cap=round,fill=fillColor] (126.77,609.14) circle (  1.16);

\path[draw=drawColor,line width= 0.4pt,line join=round,line cap=round,fill=fillColor] (127.44,607.62) circle (  1.16);

\path[draw=drawColor,line width= 0.4pt,line join=round,line cap=round,fill=fillColor] (128.09,607.35) circle (  1.16);

\path[draw=drawColor,line width= 0.4pt,line join=round,line cap=round,fill=fillColor] (128.73,605.02) circle (  1.16);

\path[draw=drawColor,line width= 0.4pt,line join=round,line cap=round,fill=fillColor] (129.35,604.85) circle (  1.16);

\path[draw=drawColor,line width= 0.4pt,line join=round,line cap=round,fill=fillColor] (129.97,604.13) circle (  1.16);

\path[draw=drawColor,line width= 0.4pt,line join=round,line cap=round,fill=fillColor] (130.58,603.81) circle (  1.16);

\path[draw=drawColor,line width= 0.4pt,line join=round,line cap=round,fill=fillColor] (131.18,601.82) circle (  1.16);

\path[draw=drawColor,line width= 0.4pt,line join=round,line cap=round,fill=fillColor] (131.77,601.78) circle (  1.16);

\path[draw=drawColor,line width= 0.4pt,line join=round,line cap=round,fill=fillColor] (132.35,601.57) circle (  1.16);

\path[draw=drawColor,line width= 0.4pt,line join=round,line cap=round,fill=fillColor] (132.93,600.63) circle (  1.16);

\path[draw=drawColor,line width= 0.4pt,line join=round,line cap=round,fill=fillColor] (133.49,599.97) circle (  1.16);

\path[draw=drawColor,line width= 0.4pt,line join=round,line cap=round,fill=fillColor] (134.05,599.96) circle (  1.16);

\path[draw=drawColor,line width= 0.4pt,line join=round,line cap=round,fill=fillColor] (134.60,599.05) circle (  1.16);

\path[draw=drawColor,line width= 0.4pt,line join=round,line cap=round,fill=fillColor] (135.14,598.07) circle (  1.16);

\path[draw=drawColor,line width= 0.4pt,line join=round,line cap=round,fill=fillColor] (135.68,597.21) circle (  1.16);

\path[draw=drawColor,line width= 0.4pt,line join=round,line cap=round,fill=fillColor] (136.21,597.04) circle (  1.16);

\path[draw=drawColor,line width= 0.4pt,line join=round,line cap=round,fill=fillColor] (136.73,595.81) circle (  1.16);

\path[draw=drawColor,line width= 0.4pt,line join=round,line cap=round,fill=fillColor] (137.25,594.48) circle (  1.16);

\path[draw=drawColor,line width= 0.4pt,line join=round,line cap=round,fill=fillColor] (137.76,588.92) circle (  1.16);

\path[draw=drawColor,line width= 0.4pt,line join=round,line cap=round,fill=fillColor] (138.26,588.37) circle (  1.16);

\path[draw=drawColor,line width= 0.4pt,line join=round,line cap=round,fill=fillColor] (138.76,587.55) circle (  1.16);

\path[draw=drawColor,line width= 0.4pt,line join=round,line cap=round,fill=fillColor] (139.25,585.57) circle (  1.16);

\path[draw=drawColor,line width= 0.4pt,line join=round,line cap=round,fill=fillColor] (139.74,584.46) circle (  1.16);

\path[draw=drawColor,line width= 0.4pt,line join=round,line cap=round,fill=fillColor] (140.22,583.19) circle (  1.16);

\path[draw=drawColor,line width= 0.4pt,line join=round,line cap=round,fill=fillColor] (140.70,582.07) circle (  1.16);

\path[draw=drawColor,line width= 0.4pt,line join=round,line cap=round,fill=fillColor] (141.17,581.63) circle (  1.16);

\path[draw=drawColor,line width= 0.4pt,line join=round,line cap=round,fill=fillColor] (141.63,579.26) circle (  1.16);

\path[draw=drawColor,line width= 0.4pt,line join=round,line cap=round,fill=fillColor] (142.09,578.38) circle (  1.16);

\path[draw=drawColor,line width= 0.4pt,line join=round,line cap=round,fill=fillColor] (142.55,578.11) circle (  1.16);

\path[draw=drawColor,line width= 0.4pt,line join=round,line cap=round,fill=fillColor] (143.00,577.65) circle (  1.16);

\path[draw=drawColor,line width= 0.4pt,line join=round,line cap=round,fill=fillColor] (143.45,576.67) circle (  1.16);

\path[draw=drawColor,line width= 0.4pt,line join=round,line cap=round,fill=fillColor] (143.89,576.62) circle (  1.16);

\path[draw=drawColor,line width= 0.4pt,line join=round,line cap=round,fill=fillColor] (144.33,576.57) circle (  1.16);

\path[draw=drawColor,line width= 0.4pt,line join=round,line cap=round,fill=fillColor] (144.77,574.91) circle (  1.16);

\path[draw=drawColor,line width= 0.4pt,line join=round,line cap=round,fill=fillColor] (145.20,574.53) circle (  1.16);

\path[draw=drawColor,line width= 0.4pt,line join=round,line cap=round,fill=fillColor] (145.62,573.66) circle (  1.16);

\path[draw=drawColor,line width= 0.4pt,line join=round,line cap=round,fill=fillColor] (146.05,573.62) circle (  1.16);

\path[draw=drawColor,line width= 0.4pt,line join=round,line cap=round,fill=fillColor] (146.46,573.57) circle (  1.16);

\path[draw=drawColor,line width= 0.4pt,line join=round,line cap=round,fill=fillColor] (146.88,573.48) circle (  1.16);

\path[draw=drawColor,line width= 0.4pt,line join=round,line cap=round,fill=fillColor] (147.29,572.83) circle (  1.16);

\path[draw=drawColor,line width= 0.4pt,line join=round,line cap=round,fill=fillColor] (147.70,571.57) circle (  1.16);

\path[draw=drawColor,line width= 0.4pt,line join=round,line cap=round,fill=fillColor] (148.10,570.94) circle (  1.16);

\path[draw=drawColor,line width= 0.4pt,line join=round,line cap=round,fill=fillColor] (148.50,570.77) circle (  1.16);

\path[draw=drawColor,line width= 0.4pt,line join=round,line cap=round,fill=fillColor] (148.90,569.73) circle (  1.16);

\path[draw=drawColor,line width= 0.4pt,line join=round,line cap=round,fill=fillColor] (149.30,568.98) circle (  1.16);

\path[draw=drawColor,line width= 0.4pt,line join=round,line cap=round,fill=fillColor] (149.69,568.93) circle (  1.16);

\path[draw=drawColor,line width= 0.4pt,line join=round,line cap=round,fill=fillColor] (150.08,568.88) circle (  1.16);

\path[draw=drawColor,line width= 0.4pt,line join=round,line cap=round,fill=fillColor] (150.46,568.72) circle (  1.16);

\path[draw=drawColor,line width= 0.4pt,line join=round,line cap=round,fill=fillColor] (150.84,568.42) circle (  1.16);

\path[draw=drawColor,line width= 0.4pt,line join=round,line cap=round,fill=fillColor] (151.22,568.03) circle (  1.16);

\path[draw=drawColor,line width= 0.4pt,line join=round,line cap=round,fill=fillColor] (151.60,567.12) circle (  1.16);

\path[draw=drawColor,line width= 0.4pt,line join=round,line cap=round,fill=fillColor] (151.97,566.68) circle (  1.16);

\path[draw=drawColor,line width= 0.4pt,line join=round,line cap=round,fill=fillColor] (152.34,566.56) circle (  1.16);

\path[draw=drawColor,line width= 0.4pt,line join=round,line cap=round,fill=fillColor] (152.71,566.03) circle (  1.16);

\path[draw=drawColor,line width= 0.4pt,line join=round,line cap=round,fill=fillColor] (153.08,565.99) circle (  1.16);

\path[draw=drawColor,line width= 0.4pt,line join=round,line cap=round,fill=fillColor] (153.44,565.65) circle (  1.16);

\path[draw=drawColor,line width= 0.4pt,line join=round,line cap=round,fill=fillColor] (153.80,564.03) circle (  1.16);

\path[draw=drawColor,line width= 0.4pt,line join=round,line cap=round,fill=fillColor] (154.16,564.03) circle (  1.16);

\path[draw=drawColor,line width= 0.4pt,line join=round,line cap=round,fill=fillColor] (154.51,563.75) circle (  1.16);

\path[draw=drawColor,line width= 0.4pt,line join=round,line cap=round,fill=fillColor] (154.86,563.73) circle (  1.16);

\path[draw=drawColor,line width= 0.4pt,line join=round,line cap=round,fill=fillColor] (155.22,563.33) circle (  1.16);

\path[draw=drawColor,line width= 0.4pt,line join=round,line cap=round,fill=fillColor] (155.56,563.18) circle (  1.16);

\path[draw=drawColor,line width= 0.4pt,line join=round,line cap=round,fill=fillColor] (155.91,562.37) circle (  1.16);

\path[draw=drawColor,line width= 0.4pt,line join=round,line cap=round,fill=fillColor] (156.25,562.32) circle (  1.16);

\path[draw=drawColor,line width= 0.4pt,line join=round,line cap=round,fill=fillColor] (156.59,561.91) circle (  1.16);

\path[draw=drawColor,line width= 0.4pt,line join=round,line cap=round,fill=fillColor] (156.93,561.84) circle (  1.16);

\path[draw=drawColor,line width= 0.4pt,line join=round,line cap=round,fill=fillColor] (157.27,561.59) circle (  1.16);

\path[draw=drawColor,line width= 0.4pt,line join=round,line cap=round,fill=fillColor] (157.60,561.58) circle (  1.16);

\path[draw=drawColor,line width= 0.4pt,line join=round,line cap=round,fill=fillColor] (157.93,561.41) circle (  1.16);

\path[draw=drawColor,line width= 0.4pt,line join=round,line cap=round,fill=fillColor] (158.26,561.36) circle (  1.16);

\path[draw=drawColor,line width= 0.4pt,line join=round,line cap=round,fill=fillColor] (158.59,561.08) circle (  1.16);

\path[draw=drawColor,line width= 0.4pt,line join=round,line cap=round,fill=fillColor] (158.92,560.71) circle (  1.16);

\path[draw=drawColor,line width= 0.4pt,line join=round,line cap=round,fill=fillColor] (159.24,560.39) circle (  1.16);

\path[draw=drawColor,line width= 0.4pt,line join=round,line cap=round,fill=fillColor] (159.56,560.08) circle (  1.16);

\path[draw=drawColor,line width= 0.4pt,line join=round,line cap=round,fill=fillColor] (159.88,560.06) circle (  1.16);

\path[draw=drawColor,line width= 0.4pt,line join=round,line cap=round,fill=fillColor] (160.20,558.68) circle (  1.16);

\path[draw=drawColor,line width= 0.4pt,line join=round,line cap=round,fill=fillColor] (160.52,557.74) circle (  1.16);

\path[draw=drawColor,line width= 0.4pt,line join=round,line cap=round,fill=fillColor] (160.83,557.06) circle (  1.16);

\path[draw=drawColor,line width= 0.4pt,line join=round,line cap=round,fill=fillColor] (161.15,556.90) circle (  1.16);

\path[draw=drawColor,line width= 0.4pt,line join=round,line cap=round,fill=fillColor] (161.46,556.80) circle (  1.16);

\path[draw=drawColor,line width= 0.4pt,line join=round,line cap=round,fill=fillColor] (161.77,556.78) circle (  1.16);

\path[draw=drawColor,line width= 0.4pt,line join=round,line cap=round,fill=fillColor] (162.07,556.42) circle (  1.16);

\path[draw=drawColor,line width= 0.4pt,line join=round,line cap=round,fill=fillColor] (162.38,555.89) circle (  1.16);

\path[draw=drawColor,line width= 0.4pt,line join=round,line cap=round,fill=fillColor] (162.68,555.87) circle (  1.16);

\path[draw=drawColor,line width= 0.4pt,line join=round,line cap=round,fill=fillColor] (162.99,555.85) circle (  1.16);

\path[draw=drawColor,line width= 0.4pt,line join=round,line cap=round,fill=fillColor] (163.29,555.71) circle (  1.16);

\path[draw=drawColor,line width= 0.4pt,line join=round,line cap=round,fill=fillColor] (163.59,555.09) circle (  1.16);

\path[draw=drawColor,line width= 0.4pt,line join=round,line cap=round,fill=fillColor] (163.88,555.00) circle (  1.16);

\path[draw=drawColor,line width= 0.4pt,line join=round,line cap=round,fill=fillColor] (164.18,554.60) circle (  1.16);

\path[draw=drawColor,line width= 0.4pt,line join=round,line cap=round,fill=fillColor] (164.47,554.42) circle (  1.16);

\path[draw=drawColor,line width= 0.4pt,line join=round,line cap=round,fill=fillColor] (164.77,554.31) circle (  1.16);

\path[draw=drawColor,line width= 0.4pt,line join=round,line cap=round,fill=fillColor] (165.06,554.28) circle (  1.16);

\path[draw=drawColor,line width= 0.4pt,line join=round,line cap=round,fill=fillColor] (165.35,554.23) circle (  1.16);

\path[draw=drawColor,line width= 0.4pt,line join=round,line cap=round,fill=fillColor] (165.64,554.09) circle (  1.16);

\path[draw=drawColor,line width= 0.4pt,line join=round,line cap=round,fill=fillColor] (165.92,554.05) circle (  1.16);

\path[draw=drawColor,line width= 0.4pt,line join=round,line cap=round,fill=fillColor] (166.21,554.05) circle (  1.16);

\path[draw=drawColor,line width= 0.4pt,line join=round,line cap=round,fill=fillColor] (166.49,553.82) circle (  1.16);

\path[draw=drawColor,line width= 0.4pt,line join=round,line cap=round,fill=fillColor] (166.77,553.75) circle (  1.16);

\path[draw=drawColor,line width= 0.4pt,line join=round,line cap=round,fill=fillColor] (167.06,553.61) circle (  1.16);

\path[draw=drawColor,line width= 0.4pt,line join=round,line cap=round,fill=fillColor] (167.34,553.59) circle (  1.16);

\path[draw=drawColor,line width= 0.4pt,line join=round,line cap=round,fill=fillColor] (167.61,553.57) circle (  1.16);

\path[draw=drawColor,line width= 0.4pt,line join=round,line cap=round,fill=fillColor] (167.89,553.51) circle (  1.16);

\path[draw=drawColor,line width= 0.4pt,line join=round,line cap=round,fill=fillColor] (168.17,553.31) circle (  1.16);

\path[draw=drawColor,line width= 0.4pt,line join=round,line cap=round,fill=fillColor] (168.44,553.09) circle (  1.16);

\path[draw=drawColor,line width= 0.4pt,line join=round,line cap=round,fill=fillColor] (168.71,552.77) circle (  1.16);

\path[draw=drawColor,line width= 0.4pt,line join=round,line cap=round,fill=fillColor] (168.99,552.54) circle (  1.16);

\path[draw=drawColor,line width= 0.4pt,line join=round,line cap=round,fill=fillColor] (169.26,552.53) circle (  1.16);

\path[draw=drawColor,line width= 0.4pt,line join=round,line cap=round,fill=fillColor] (169.53,552.39) circle (  1.16);

\path[draw=drawColor,line width= 0.4pt,line join=round,line cap=round,fill=fillColor] (169.79,552.20) circle (  1.16);

\path[draw=drawColor,line width= 0.4pt,line join=round,line cap=round,fill=fillColor] (170.06,552.06) circle (  1.16);

\path[draw=drawColor,line width= 0.4pt,line join=round,line cap=round,fill=fillColor] (170.33,551.96) circle (  1.16);

\path[draw=drawColor,line width= 0.4pt,line join=round,line cap=round,fill=fillColor] (170.59,551.81) circle (  1.16);

\path[draw=drawColor,line width= 0.4pt,line join=round,line cap=round,fill=fillColor] (170.85,551.69) circle (  1.16);

\path[draw=drawColor,line width= 0.4pt,line join=round,line cap=round,fill=fillColor] (171.12,551.55) circle (  1.16);

\path[draw=drawColor,line width= 0.4pt,line join=round,line cap=round,fill=fillColor] (171.38,551.40) circle (  1.16);

\path[draw=drawColor,line width= 0.4pt,line join=round,line cap=round,fill=fillColor] (171.64,551.26) circle (  1.16);

\path[draw=drawColor,line width= 0.4pt,line join=round,line cap=round,fill=fillColor] (171.90,551.20) circle (  1.16);

\path[draw=drawColor,line width= 0.4pt,line join=round,line cap=round,fill=fillColor] (172.15,551.16) circle (  1.16);

\path[draw=drawColor,line width= 0.4pt,line join=round,line cap=round,fill=fillColor] (172.41,551.12) circle (  1.16);

\path[draw=drawColor,line width= 0.4pt,line join=round,line cap=round,fill=fillColor] (172.67,551.06) circle (  1.16);

\path[draw=drawColor,line width= 0.4pt,line join=round,line cap=round,fill=fillColor] (172.92,550.83) circle (  1.16);

\path[draw=drawColor,line width= 0.4pt,line join=round,line cap=round,fill=fillColor] (173.17,550.79) circle (  1.16);

\path[draw=drawColor,line width= 0.4pt,line join=round,line cap=round,fill=fillColor] (173.43,550.53) circle (  1.16);

\path[draw=drawColor,line width= 0.4pt,line join=round,line cap=round,fill=fillColor] (173.68,550.39) circle (  1.16);

\path[draw=drawColor,line width= 0.4pt,line join=round,line cap=round,fill=fillColor] (173.93,550.01) circle (  1.16);

\path[draw=drawColor,line width= 0.4pt,line join=round,line cap=round,fill=fillColor] (174.18,549.59) circle (  1.16);

\path[draw=drawColor,line width= 0.4pt,line join=round,line cap=round,fill=fillColor] (174.42,549.18) circle (  1.16);

\path[draw=drawColor,line width= 0.4pt,line join=round,line cap=round,fill=fillColor] (174.67,549.18) circle (  1.16);

\path[draw=drawColor,line width= 0.4pt,line join=round,line cap=round,fill=fillColor] (174.92,549.14) circle (  1.16);

\path[draw=drawColor,line width= 0.4pt,line join=round,line cap=round,fill=fillColor] (175.16,549.05) circle (  1.16);

\path[draw=drawColor,line width= 0.4pt,line join=round,line cap=round,fill=fillColor] (175.41,549.00) circle (  1.16);

\path[draw=drawColor,line width= 0.4pt,line join=round,line cap=round,fill=fillColor] (175.65,548.93) circle (  1.16);

\path[draw=drawColor,line width= 0.4pt,line join=round,line cap=round,fill=fillColor] (175.89,548.67) circle (  1.16);

\path[draw=drawColor,line width= 0.4pt,line join=round,line cap=round,fill=fillColor] (176.13,548.55) circle (  1.16);

\path[draw=drawColor,line width= 0.4pt,line join=round,line cap=round,fill=fillColor] (176.37,548.53) circle (  1.16);

\path[draw=drawColor,line width= 0.4pt,line join=round,line cap=round,fill=fillColor] (176.61,548.32) circle (  1.16);

\path[draw=drawColor,line width= 0.4pt,line join=round,line cap=round,fill=fillColor] (176.85,548.28) circle (  1.16);

\path[draw=drawColor,line width= 0.4pt,line join=round,line cap=round,fill=fillColor] (177.09,548.23) circle (  1.16);

\path[draw=drawColor,line width= 0.4pt,line join=round,line cap=round,fill=fillColor] (177.32,548.09) circle (  1.16);

\path[draw=drawColor,line width= 0.4pt,line join=round,line cap=round,fill=fillColor] (177.56,548.05) circle (  1.16);

\path[draw=drawColor,line width= 0.4pt,line join=round,line cap=round,fill=fillColor] (177.80,547.66) circle (  1.16);

\path[draw=drawColor,line width= 0.4pt,line join=round,line cap=round,fill=fillColor] (178.03,547.02) circle (  1.16);

\path[draw=drawColor,line width= 0.4pt,line join=round,line cap=round,fill=fillColor] (178.26,546.74) circle (  1.16);

\path[draw=drawColor,line width= 0.4pt,line join=round,line cap=round,fill=fillColor] (178.49,546.69) circle (  1.16);

\path[draw=drawColor,line width= 0.4pt,line join=round,line cap=round,fill=fillColor] (178.73,546.60) circle (  1.16);

\path[draw=drawColor,line width= 0.4pt,line join=round,line cap=round,fill=fillColor] (178.96,546.07) circle (  1.16);

\path[draw=drawColor,line width= 0.4pt,line join=round,line cap=round,fill=fillColor] (179.19,545.87) circle (  1.16);

\path[draw=drawColor,line width= 0.4pt,line join=round,line cap=round,fill=fillColor] (179.41,545.74) circle (  1.16);

\path[draw=drawColor,line width= 0.4pt,line join=round,line cap=round,fill=fillColor] (179.64,545.59) circle (  1.16);

\path[draw=drawColor,line width= 0.4pt,line join=round,line cap=round,fill=fillColor] (179.87,545.31) circle (  1.16);

\path[draw=drawColor,line width= 0.4pt,line join=round,line cap=round,fill=fillColor] (180.10,545.04) circle (  1.16);

\path[draw=drawColor,line width= 0.4pt,line join=round,line cap=round,fill=fillColor] (180.32,544.69) circle (  1.16);

\path[draw=drawColor,line width= 0.4pt,line join=round,line cap=round,fill=fillColor] (180.55,544.56) circle (  1.16);

\path[draw=drawColor,line width= 0.4pt,line join=round,line cap=round,fill=fillColor] (180.77,544.38) circle (  1.16);

\path[draw=drawColor,line width= 0.4pt,line join=round,line cap=round,fill=fillColor] (180.99,544.36) circle (  1.16);

\path[draw=drawColor,line width= 0.4pt,line join=round,line cap=round,fill=fillColor] (181.22,544.29) circle (  1.16);

\path[draw=drawColor,line width= 0.4pt,line join=round,line cap=round,fill=fillColor] (181.44,544.26) circle (  1.16);

\path[draw=drawColor,line width= 0.4pt,line join=round,line cap=round,fill=fillColor] (181.66,543.57) circle (  1.16);

\path[draw=drawColor,line width= 0.4pt,line join=round,line cap=round,fill=fillColor] (181.88,543.45) circle (  1.16);

\path[draw=drawColor,line width= 0.4pt,line join=round,line cap=round,fill=fillColor] (182.10,543.28) circle (  1.16);

\path[draw=drawColor,line width= 0.4pt,line join=round,line cap=round,fill=fillColor] (182.32,542.97) circle (  1.16);

\path[draw=drawColor,line width= 0.4pt,line join=round,line cap=round,fill=fillColor] (182.54,542.91) circle (  1.16);

\path[draw=drawColor,line width= 0.4pt,line join=round,line cap=round,fill=fillColor] (182.75,542.87) circle (  1.16);

\path[draw=drawColor,line width= 0.4pt,line join=round,line cap=round,fill=fillColor] (182.97,542.79) circle (  1.16);

\path[draw=drawColor,line width= 0.4pt,line join=round,line cap=round,fill=fillColor] (183.19,541.42) circle (  1.16);

\path[draw=drawColor,line width= 0.4pt,line join=round,line cap=round,fill=fillColor] (183.40,540.24) circle (  1.16);

\path[draw=drawColor,line width= 0.4pt,line join=round,line cap=round,fill=fillColor] (183.61,535.03) circle (  1.16);

\path[draw=drawColor,line width= 0.4pt,line join=round,line cap=round,fill=fillColor] (183.83,535.03) circle (  1.16);

\path[draw=drawColor,line width= 0.4pt,line join=round,line cap=round,fill=fillColor] (184.04,535.03) circle (  1.16);

\path[draw=drawColor,line width= 0.4pt,line join=round,line cap=round,fill=fillColor] (184.25,535.03) circle (  1.16);

\path[draw=drawColor,line width= 0.4pt,line join=round,line cap=round,fill=fillColor] (184.47,535.03) circle (  1.16);

\path[draw=drawColor,line width= 0.4pt,line join=round,line cap=round,fill=fillColor] (184.68,535.03) circle (  1.16);

\path[draw=drawColor,line width= 0.4pt,line join=round,line cap=round,fill=fillColor] (184.89,535.03) circle (  1.16);

\path[draw=drawColor,line width= 0.4pt,line join=round,line cap=round,fill=fillColor] (185.10,535.03) circle (  1.16);

\path[draw=drawColor,line width= 0.4pt,line join=round,line cap=round,fill=fillColor] (185.31,535.03) circle (  1.16);

\path[draw=drawColor,line width= 0.4pt,line join=round,line cap=round,fill=fillColor] (185.51,535.03) circle (  1.16);

\path[draw=drawColor,line width= 0.4pt,line join=round,line cap=round,fill=fillColor] (185.72,535.03) circle (  1.16);

\path[draw=drawColor,line width= 0.4pt,line join=round,line cap=round,fill=fillColor] (185.93,535.03) circle (  1.16);

\path[draw=drawColor,line width= 0.4pt,line join=round,line cap=round,fill=fillColor] (186.13,535.03) circle (  1.16);

\path[draw=drawColor,line width= 0.4pt,line join=round,line cap=round,fill=fillColor] (186.34,535.03) circle (  1.16);

\path[draw=drawColor,line width= 0.4pt,line join=round,line cap=round,fill=fillColor] (186.55,535.03) circle (  1.16);

\path[draw=drawColor,line width= 0.4pt,line join=round,line cap=round,fill=fillColor] (186.75,535.03) circle (  1.16);

\path[draw=drawColor,line width= 0.4pt,line join=round,line cap=round,fill=fillColor] (186.95,535.03) circle (  1.16);

\path[draw=drawColor,line width= 0.4pt,line join=round,line cap=round,fill=fillColor] (187.16,535.03) circle (  1.16);

\path[draw=drawColor,line width= 0.4pt,line join=round,line cap=round,fill=fillColor] (187.36,535.03) circle (  1.16);

\path[draw=drawColor,line width= 0.4pt,line join=round,line cap=round,fill=fillColor] (187.56,535.03) circle (  1.16);

\path[draw=drawColor,line width= 0.4pt,line join=round,line cap=round,fill=fillColor] (187.76,535.03) circle (  1.16);

\path[draw=drawColor,line width= 0.4pt,line join=round,line cap=round,fill=fillColor] (187.96,535.03) circle (  1.16);

\path[draw=drawColor,line width= 0.4pt,line join=round,line cap=round,fill=fillColor] (188.16,535.03) circle (  1.16);

\path[draw=drawColor,line width= 0.4pt,line join=round,line cap=round,fill=fillColor] (188.36,535.03) circle (  1.16);

\path[draw=drawColor,line width= 0.4pt,line join=round,line cap=round,fill=fillColor] (188.56,535.03) circle (  1.16);

\path[draw=drawColor,line width= 0.4pt,line join=round,line cap=round,fill=fillColor] (188.76,535.03) circle (  1.16);

\path[draw=drawColor,line width= 0.4pt,line join=round,line cap=round,fill=fillColor] (188.96,535.03) circle (  1.16);

\path[draw=drawColor,line width= 0.4pt,line join=round,line cap=round,fill=fillColor] (189.16,535.03) circle (  1.16);

\path[draw=drawColor,line width= 0.4pt,line join=round,line cap=round,fill=fillColor] (189.35,535.03) circle (  1.16);

\path[draw=drawColor,line width= 0.4pt,line join=round,line cap=round,fill=fillColor] (189.55,535.03) circle (  1.16);

\path[draw=drawColor,line width= 0.4pt,line join=round,line cap=round,fill=fillColor] (189.74,535.03) circle (  1.16);

\path[draw=drawColor,line width= 0.4pt,line join=round,line cap=round,fill=fillColor] (189.94,535.03) circle (  1.16);

\path[draw=drawColor,line width= 0.4pt,line join=round,line cap=round,fill=fillColor] (190.13,535.03) circle (  1.16);

\path[draw=drawColor,line width= 0.4pt,line join=round,line cap=round,fill=fillColor] (190.33,535.03) circle (  1.16);

\path[draw=drawColor,line width= 0.4pt,line join=round,line cap=round,fill=fillColor] (190.52,535.03) circle (  1.16);

\path[draw=drawColor,line width= 0.4pt,line join=round,line cap=round,fill=fillColor] (190.71,535.03) circle (  1.16);

\path[draw=drawColor,line width= 0.4pt,line join=round,line cap=round,fill=fillColor] (190.91,535.03) circle (  1.16);

\path[draw=drawColor,line width= 0.4pt,line join=round,line cap=round,fill=fillColor] (191.10,535.03) circle (  1.16);

\path[draw=drawColor,line width= 0.4pt,line join=round,line cap=round,fill=fillColor] (191.29,535.03) circle (  1.16);

\path[draw=drawColor,line width= 0.4pt,line join=round,line cap=round,fill=fillColor] (191.48,535.03) circle (  1.16);

\path[draw=drawColor,line width= 0.4pt,line join=round,line cap=round,fill=fillColor] (191.67,535.03) circle (  1.16);

\path[draw=drawColor,line width= 0.4pt,line join=round,line cap=round,fill=fillColor] (191.86,535.03) circle (  1.16);

\path[draw=drawColor,line width= 0.4pt,line join=round,line cap=round,fill=fillColor] (192.05,535.03) circle (  1.16);

\path[draw=drawColor,line width= 0.4pt,line join=round,line cap=round,fill=fillColor] (192.24,535.03) circle (  1.16);

\path[draw=drawColor,line width= 0.4pt,line join=round,line cap=round,fill=fillColor] (192.43,535.03) circle (  1.16);

\path[draw=drawColor,line width= 0.4pt,line join=round,line cap=round,fill=fillColor] (192.61,535.03) circle (  1.16);

\path[draw=drawColor,line width= 0.4pt,line join=round,line cap=round,fill=fillColor] (192.80,535.03) circle (  1.16);

\path[draw=drawColor,line width= 0.4pt,line join=round,line cap=round,fill=fillColor] (192.99,535.03) circle (  1.16);

\path[draw=drawColor,line width= 0.4pt,line join=round,line cap=round,fill=fillColor] (193.17,535.03) circle (  1.16);

\path[draw=drawColor,line width= 0.4pt,line join=round,line cap=round,fill=fillColor] (193.36,535.03) circle (  1.16);

\path[draw=drawColor,line width= 0.4pt,line join=round,line cap=round,fill=fillColor] (193.54,535.03) circle (  1.16);

\path[draw=drawColor,line width= 0.4pt,line join=round,line cap=round,fill=fillColor] (193.73,535.03) circle (  1.16);

\path[draw=drawColor,line width= 0.4pt,line join=round,line cap=round,fill=fillColor] (193.91,535.03) circle (  1.16);

\path[draw=drawColor,line width= 0.4pt,line join=round,line cap=round,fill=fillColor] (194.10,535.03) circle (  1.16);

\path[draw=drawColor,line width= 0.4pt,line join=round,line cap=round,fill=fillColor] (194.28,535.03) circle (  1.16);

\path[draw=drawColor,line width= 0.4pt,line join=round,line cap=round,fill=fillColor] (194.46,535.03) circle (  1.16);

\path[draw=drawColor,line width= 0.4pt,line join=round,line cap=round,fill=fillColor] (194.65,535.03) circle (  1.16);

\path[draw=drawColor,line width= 0.4pt,line join=round,line cap=round,fill=fillColor] (194.83,535.03) circle (  1.16);

\path[draw=drawColor,line width= 0.4pt,line join=round,line cap=round,fill=fillColor] (195.01,535.03) circle (  1.16);

\path[draw=drawColor,line width= 0.4pt,line join=round,line cap=round,fill=fillColor] (195.19,535.03) circle (  1.16);

\path[draw=drawColor,line width= 0.4pt,line join=round,line cap=round,fill=fillColor] (195.37,535.03) circle (  1.16);

\path[draw=drawColor,line width= 0.4pt,line join=round,line cap=round,fill=fillColor] (195.55,535.03) circle (  1.16);

\path[draw=drawColor,line width= 0.4pt,line join=round,line cap=round,fill=fillColor] (195.73,535.03) circle (  1.16);

\path[draw=drawColor,line width= 0.4pt,line join=round,line cap=round,fill=fillColor] (195.91,535.03) circle (  1.16);

\path[draw=drawColor,line width= 0.4pt,line join=round,line cap=round,fill=fillColor] (196.09,535.03) circle (  1.16);

\path[draw=drawColor,line width= 0.4pt,line join=round,line cap=round,fill=fillColor] (196.27,535.03) circle (  1.16);

\path[draw=drawColor,line width= 0.4pt,line join=round,line cap=round,fill=fillColor] (196.44,535.03) circle (  1.16);

\path[draw=drawColor,line width= 0.4pt,line join=round,line cap=round,fill=fillColor] (196.62,535.03) circle (  1.16);

\path[draw=drawColor,line width= 0.4pt,line join=round,line cap=round,fill=fillColor] (196.80,535.03) circle (  1.16);

\path[draw=drawColor,line width= 0.4pt,line join=round,line cap=round,fill=fillColor] (196.97,535.03) circle (  1.16);

\path[draw=drawColor,line width= 0.4pt,line join=round,line cap=round,fill=fillColor] (197.15,535.03) circle (  1.16);

\path[draw=drawColor,line width= 0.4pt,line join=round,line cap=round,fill=fillColor] (197.33,535.03) circle (  1.16);

\path[draw=drawColor,line width= 0.4pt,line join=round,line cap=round,fill=fillColor] (197.50,535.03) circle (  1.16);

\path[draw=drawColor,line width= 0.4pt,line join=round,line cap=round,fill=fillColor] (197.68,535.03) circle (  1.16);

\path[draw=drawColor,line width= 0.4pt,line join=round,line cap=round,fill=fillColor] (197.85,535.03) circle (  1.16);

\path[draw=drawColor,line width= 0.4pt,line join=round,line cap=round,fill=fillColor] (198.02,535.03) circle (  1.16);

\path[draw=drawColor,line width= 0.4pt,line join=round,line cap=round,fill=fillColor] (198.20,535.03) circle (  1.16);

\path[draw=drawColor,line width= 0.4pt,line join=round,line cap=round,fill=fillColor] (198.37,535.03) circle (  1.16);

\path[draw=drawColor,line width= 0.4pt,line join=round,line cap=round,fill=fillColor] (198.54,535.03) circle (  1.16);

\path[draw=drawColor,line width= 0.4pt,line join=round,line cap=round,fill=fillColor] (198.72,535.03) circle (  1.16);

\path[draw=drawColor,line width= 0.4pt,line join=round,line cap=round,fill=fillColor] (198.89,535.03) circle (  1.16);

\path[draw=drawColor,line width= 0.4pt,line join=round,line cap=round,fill=fillColor] (199.06,535.03) circle (  1.16);

\path[draw=drawColor,line width= 0.4pt,line join=round,line cap=round,fill=fillColor] (199.23,535.03) circle (  1.16);

\path[draw=drawColor,line width= 0.4pt,line join=round,line cap=round,fill=fillColor] (199.40,535.03) circle (  1.16);

\path[draw=drawColor,line width= 0.4pt,line join=round,line cap=round,fill=fillColor] (199.57,535.03) circle (  1.16);

\path[draw=drawColor,line width= 0.4pt,line join=round,line cap=round,fill=fillColor] (199.74,535.03) circle (  1.16);

\path[draw=drawColor,line width= 0.4pt,line join=round,line cap=round,fill=fillColor] (199.91,535.03) circle (  1.16);

\path[draw=drawColor,line width= 0.4pt,line join=round,line cap=round,fill=fillColor] (200.08,535.03) circle (  1.16);

\path[draw=drawColor,line width= 0.4pt,line join=round,line cap=round,fill=fillColor] (200.25,535.03) circle (  1.16);

\path[draw=drawColor,line width= 0.4pt,line join=round,line cap=round,fill=fillColor] (200.42,535.03) circle (  1.16);

\path[draw=drawColor,line width= 0.4pt,line join=round,line cap=round,fill=fillColor] (200.58,535.03) circle (  1.16);

\path[draw=drawColor,line width= 0.4pt,line join=round,line cap=round,fill=fillColor] (200.75,535.03) circle (  1.16);

\path[draw=drawColor,line width= 0.4pt,line join=round,line cap=round,fill=fillColor] (200.92,535.03) circle (  1.16);

\path[draw=drawColor,line width= 0.4pt,line join=round,line cap=round,fill=fillColor] (201.09,535.03) circle (  1.16);

\path[draw=drawColor,line width= 0.4pt,line join=round,line cap=round,fill=fillColor] (201.25,535.03) circle (  1.16);

\path[draw=drawColor,line width= 0.4pt,line join=round,line cap=round,fill=fillColor] (201.42,535.03) circle (  1.16);

\path[draw=drawColor,line width= 0.4pt,line join=round,line cap=round,fill=fillColor] (201.58,535.03) circle (  1.16);

\path[draw=drawColor,line width= 0.4pt,line join=round,line cap=round,fill=fillColor] (201.75,535.03) circle (  1.16);

\path[draw=drawColor,line width= 0.4pt,line join=round,line cap=round,fill=fillColor] (201.91,535.03) circle (  1.16);

\path[draw=drawColor,line width= 0.4pt,line join=round,line cap=round,fill=fillColor] (202.08,535.03) circle (  1.16);

\path[draw=drawColor,line width= 0.4pt,line join=round,line cap=round,fill=fillColor] (202.24,535.03) circle (  1.16);

\path[draw=drawColor,line width= 0.4pt,line join=round,line cap=round,fill=fillColor] (202.41,535.03) circle (  1.16);

\path[draw=drawColor,line width= 0.4pt,line join=round,line cap=round,fill=fillColor] (202.57,535.03) circle (  1.16);

\path[draw=drawColor,line width= 0.4pt,line join=round,line cap=round,fill=fillColor] (202.73,535.03) circle (  1.16);

\path[draw=drawColor,line width= 0.4pt,line join=round,line cap=round,fill=fillColor] (202.90,535.03) circle (  1.16);

\path[draw=drawColor,line width= 0.4pt,line join=round,line cap=round,fill=fillColor] (203.06,535.03) circle (  1.16);

\path[draw=drawColor,line width= 0.4pt,line join=round,line cap=round,fill=fillColor] (203.22,535.03) circle (  1.16);

\path[draw=drawColor,line width= 0.4pt,line join=round,line cap=round,fill=fillColor] (203.38,535.03) circle (  1.16);

\path[draw=drawColor,line width= 0.4pt,line join=round,line cap=round,fill=fillColor] (203.54,535.03) circle (  1.16);

\path[draw=drawColor,line width= 0.4pt,line join=round,line cap=round,fill=fillColor] (203.71,535.03) circle (  1.16);

\path[draw=drawColor,line width= 0.4pt,line join=round,line cap=round,fill=fillColor] (203.87,535.03) circle (  1.16);

\path[draw=drawColor,line width= 0.4pt,line join=round,line cap=round,fill=fillColor] (204.03,535.03) circle (  1.16);

\path[draw=drawColor,line width= 0.4pt,line join=round,line cap=round,fill=fillColor] (204.19,535.03) circle (  1.16);

\path[draw=drawColor,line width= 0.4pt,line join=round,line cap=round,fill=fillColor] (204.35,535.03) circle (  1.16);

\path[draw=drawColor,line width= 0.4pt,line join=round,line cap=round,fill=fillColor] (204.51,535.03) circle (  1.16);

\path[draw=drawColor,line width= 0.4pt,line join=round,line cap=round,fill=fillColor] (204.66,535.03) circle (  1.16);

\path[draw=drawColor,line width= 0.4pt,line join=round,line cap=round,fill=fillColor] (204.82,535.03) circle (  1.16);

\path[draw=drawColor,line width= 0.4pt,line join=round,line cap=round,fill=fillColor] (204.98,535.03) circle (  1.16);

\path[draw=drawColor,line width= 0.4pt,line join=round,line cap=round,fill=fillColor] (205.14,535.03) circle (  1.16);

\path[draw=drawColor,line width= 0.4pt,line join=round,line cap=round,fill=fillColor] (205.30,535.03) circle (  1.16);

\path[draw=drawColor,line width= 0.4pt,line join=round,line cap=round,fill=fillColor] (205.45,535.03) circle (  1.16);

\path[draw=drawColor,line width= 0.4pt,line join=round,line cap=round,fill=fillColor] (205.61,535.03) circle (  1.16);

\path[draw=drawColor,line width= 0.4pt,line join=round,line cap=round,fill=fillColor] (205.77,535.03) circle (  1.16);

\path[draw=drawColor,line width= 0.4pt,line join=round,line cap=round,fill=fillColor] (205.92,535.03) circle (  1.16);

\path[draw=drawColor,line width= 0.4pt,line join=round,line cap=round,fill=fillColor] (206.08,535.03) circle (  1.16);

\path[draw=drawColor,line width= 0.4pt,line join=round,line cap=round,fill=fillColor] (206.24,535.03) circle (  1.16);

\path[draw=drawColor,line width= 0.4pt,line join=round,line cap=round,fill=fillColor] (206.39,535.03) circle (  1.16);

\path[draw=drawColor,line width= 0.4pt,line join=round,line cap=round,fill=fillColor] (206.55,535.03) circle (  1.16);

\path[draw=drawColor,line width= 0.4pt,line join=round,line cap=round,fill=fillColor] (206.70,535.03) circle (  1.16);

\path[draw=drawColor,line width= 0.4pt,line join=round,line cap=round,fill=fillColor] (206.86,535.03) circle (  1.16);

\path[draw=drawColor,line width= 0.4pt,line join=round,line cap=round,fill=fillColor] (207.01,535.03) circle (  1.16);

\path[draw=drawColor,line width= 0.4pt,line join=round,line cap=round,fill=fillColor] (207.16,535.03) circle (  1.16);

\path[draw=drawColor,line width= 0.4pt,line join=round,line cap=round,fill=fillColor] (207.32,535.03) circle (  1.16);

\path[draw=drawColor,line width= 0.4pt,line join=round,line cap=round,fill=fillColor] (207.47,535.03) circle (  1.16);

\path[draw=drawColor,line width= 0.4pt,line join=round,line cap=round,fill=fillColor] (207.62,535.03) circle (  1.16);

\path[draw=drawColor,line width= 0.4pt,line join=round,line cap=round,fill=fillColor] (207.78,535.03) circle (  1.16);

\path[draw=drawColor,line width= 0.4pt,line join=round,line cap=round,fill=fillColor] (207.93,535.03) circle (  1.16);

\path[draw=drawColor,line width= 0.4pt,line join=round,line cap=round,fill=fillColor] (208.08,535.03) circle (  1.16);

\path[draw=drawColor,line width= 0.4pt,line join=round,line cap=round,fill=fillColor] (208.23,535.03) circle (  1.16);

\path[draw=drawColor,line width= 0.4pt,line join=round,line cap=round,fill=fillColor] (208.39,535.03) circle (  1.16);

\path[draw=drawColor,line width= 0.4pt,line join=round,line cap=round,fill=fillColor] (208.54,535.03) circle (  1.16);

\path[draw=drawColor,line width= 0.4pt,line join=round,line cap=round,fill=fillColor] (208.69,535.03) circle (  1.16);

\path[draw=drawColor,line width= 0.4pt,line join=round,line cap=round,fill=fillColor] (208.84,535.03) circle (  1.16);

\path[draw=drawColor,line width= 0.4pt,line join=round,line cap=round,fill=fillColor] (208.99,535.03) circle (  1.16);

\path[draw=drawColor,line width= 0.4pt,line join=round,line cap=round,fill=fillColor] (209.14,535.03) circle (  1.16);

\path[draw=drawColor,line width= 0.4pt,line join=round,line cap=round,fill=fillColor] (209.29,535.03) circle (  1.16);

\path[draw=drawColor,line width= 0.4pt,line join=round,line cap=round,fill=fillColor] (209.44,535.03) circle (  1.16);

\path[draw=drawColor,line width= 0.4pt,line join=round,line cap=round,fill=fillColor] (209.59,535.03) circle (  1.16);

\path[draw=drawColor,line width= 0.4pt,line join=round,line cap=round,fill=fillColor] (209.74,535.03) circle (  1.16);

\path[draw=drawColor,line width= 0.4pt,line join=round,line cap=round,fill=fillColor] (209.88,535.03) circle (  1.16);

\path[draw=drawColor,line width= 0.4pt,line join=round,line cap=round,fill=fillColor] (210.03,535.03) circle (  1.16);

\path[draw=drawColor,line width= 0.4pt,line join=round,line cap=round,fill=fillColor] (210.18,535.03) circle (  1.16);

\path[draw=drawColor,line width= 0.4pt,line join=round,line cap=round,fill=fillColor] (210.33,535.03) circle (  1.16);

\path[draw=drawColor,line width= 0.4pt,line join=round,line cap=round,fill=fillColor] (210.48,535.03) circle (  1.16);

\path[draw=drawColor,line width= 0.4pt,line join=round,line cap=round,fill=fillColor] (210.62,535.03) circle (  1.16);

\path[draw=drawColor,line width= 0.4pt,line join=round,line cap=round,fill=fillColor] (210.77,535.03) circle (  1.16);

\path[draw=drawColor,line width= 0.4pt,line join=round,line cap=round,fill=fillColor] (210.92,535.03) circle (  1.16);

\path[draw=drawColor,line width= 0.4pt,line join=round,line cap=round,fill=fillColor] (211.06,535.03) circle (  1.16);

\path[draw=drawColor,line width= 0.4pt,line join=round,line cap=round,fill=fillColor] (211.21,535.03) circle (  1.16);

\path[draw=drawColor,line width= 0.4pt,line join=round,line cap=round,fill=fillColor] (211.36,535.03) circle (  1.16);

\path[draw=drawColor,line width= 0.4pt,line join=round,line cap=round,fill=fillColor] (211.50,535.03) circle (  1.16);

\path[draw=drawColor,line width= 0.4pt,line join=round,line cap=round,fill=fillColor] (211.65,535.03) circle (  1.16);

\path[draw=drawColor,line width= 0.4pt,line join=round,line cap=round,fill=fillColor] (211.79,535.03) circle (  1.16);

\path[draw=drawColor,line width= 0.4pt,line join=round,line cap=round,fill=fillColor] (211.94,535.03) circle (  1.16);

\path[draw=drawColor,line width= 0.4pt,line join=round,line cap=round,fill=fillColor] (212.08,535.03) circle (  1.16);

\path[draw=drawColor,line width= 0.4pt,line join=round,line cap=round,fill=fillColor] (212.23,535.03) circle (  1.16);

\path[draw=drawColor,line width= 0.4pt,line join=round,line cap=round,fill=fillColor] (212.37,535.03) circle (  1.16);

\path[draw=drawColor,line width= 0.4pt,line join=round,line cap=round,fill=fillColor] (212.51,535.03) circle (  1.16);

\path[draw=drawColor,line width= 0.4pt,line join=round,line cap=round,fill=fillColor] (212.66,535.03) circle (  1.16);

\path[draw=drawColor,line width= 0.4pt,line join=round,line cap=round,fill=fillColor] (212.80,535.03) circle (  1.16);

\path[draw=drawColor,line width= 0.4pt,line join=round,line cap=round,fill=fillColor] (212.94,535.03) circle (  1.16);

\path[draw=drawColor,line width= 0.4pt,line join=round,line cap=round,fill=fillColor] (213.09,535.03) circle (  1.16);

\path[draw=drawColor,line width= 0.4pt,line join=round,line cap=round,fill=fillColor] (213.23,535.03) circle (  1.16);

\path[draw=drawColor,line width= 0.4pt,line join=round,line cap=round,fill=fillColor] (213.37,535.03) circle (  1.16);

\path[draw=drawColor,line width= 0.4pt,line join=round,line cap=round,fill=fillColor] (213.51,535.03) circle (  1.16);

\path[draw=drawColor,line width= 0.4pt,line join=round,line cap=round,fill=fillColor] (213.65,535.03) circle (  1.16);

\path[draw=drawColor,line width= 0.4pt,line join=round,line cap=round,fill=fillColor] (213.80,535.03) circle (  1.16);

\path[draw=drawColor,line width= 0.4pt,line join=round,line cap=round,fill=fillColor] (213.94,535.03) circle (  1.16);

\path[draw=drawColor,line width= 0.4pt,line join=round,line cap=round,fill=fillColor] (214.08,535.03) circle (  1.16);

\path[draw=drawColor,line width= 0.4pt,line join=round,line cap=round,fill=fillColor] (214.22,535.03) circle (  1.16);

\path[draw=drawColor,line width= 0.4pt,line join=round,line cap=round,fill=fillColor] (214.36,535.03) circle (  1.16);

\path[draw=drawColor,line width= 0.4pt,line join=round,line cap=round,fill=fillColor] (214.50,535.03) circle (  1.16);

\path[draw=drawColor,line width= 0.4pt,line join=round,line cap=round,fill=fillColor] (214.64,535.03) circle (  1.16);

\path[draw=drawColor,line width= 0.4pt,line join=round,line cap=round,fill=fillColor] (214.78,535.03) circle (  1.16);

\path[draw=drawColor,line width= 0.4pt,line join=round,line cap=round,fill=fillColor] (214.92,535.03) circle (  1.16);

\path[draw=drawColor,line width= 0.4pt,line join=round,line cap=round,fill=fillColor] (215.06,535.03) circle (  1.16);

\path[draw=drawColor,line width= 0.4pt,line join=round,line cap=round,fill=fillColor] (215.20,535.03) circle (  1.16);

\path[draw=drawColor,line width= 0.4pt,line join=round,line cap=round,fill=fillColor] (215.34,535.03) circle (  1.16);

\path[draw=drawColor,line width= 0.4pt,line join=round,line cap=round,fill=fillColor] (215.47,535.03) circle (  1.16);

\path[draw=drawColor,line width= 0.4pt,line join=round,line cap=round,fill=fillColor] (215.61,535.03) circle (  1.16);

\path[draw=drawColor,line width= 0.4pt,line join=round,line cap=round,fill=fillColor] (215.75,535.03) circle (  1.16);

\path[draw=drawColor,line width= 0.4pt,line join=round,line cap=round,fill=fillColor] (215.89,535.03) circle (  1.16);

\path[draw=drawColor,line width= 0.4pt,line join=round,line cap=round,fill=fillColor] (216.03,535.03) circle (  1.16);

\path[draw=drawColor,line width= 0.4pt,line join=round,line cap=round,fill=fillColor] (216.16,535.03) circle (  1.16);

\path[draw=drawColor,line width= 0.4pt,line join=round,line cap=round,fill=fillColor] (216.30,535.03) circle (  1.16);

\path[draw=drawColor,line width= 0.4pt,line join=round,line cap=round,fill=fillColor] (216.44,535.03) circle (  1.16);

\path[draw=drawColor,line width= 0.4pt,line join=round,line cap=round,fill=fillColor] (216.58,535.03) circle (  1.16);

\path[draw=drawColor,line width= 0.4pt,line join=round,line cap=round,fill=fillColor] (216.71,535.03) circle (  1.16);

\path[draw=drawColor,line width= 0.4pt,line join=round,line cap=round,fill=fillColor] (216.85,535.03) circle (  1.16);

\path[draw=drawColor,line width= 0.4pt,line join=round,line cap=round,fill=fillColor] (216.98,535.03) circle (  1.16);

\path[draw=drawColor,line width= 0.4pt,line join=round,line cap=round,fill=fillColor] (217.12,535.03) circle (  1.16);

\path[draw=drawColor,line width= 0.4pt,line join=round,line cap=round,fill=fillColor] (217.26,535.03) circle (  1.16);

\path[draw=drawColor,line width= 0.4pt,line join=round,line cap=round,fill=fillColor] (217.39,535.03) circle (  1.16);

\path[draw=drawColor,line width= 0.4pt,line join=round,line cap=round,fill=fillColor] (217.53,535.03) circle (  1.16);

\path[draw=drawColor,line width= 0.4pt,line join=round,line cap=round,fill=fillColor] (217.66,535.03) circle (  1.16);

\path[draw=drawColor,line width= 0.4pt,line join=round,line cap=round,fill=fillColor] (217.80,535.03) circle (  1.16);

\path[draw=drawColor,line width= 0.4pt,line join=round,line cap=round,fill=fillColor] (217.93,535.03) circle (  1.16);

\path[draw=drawColor,line width= 0.4pt,line join=round,line cap=round,fill=fillColor] (218.06,535.03) circle (  1.16);

\path[draw=drawColor,line width= 0.4pt,line join=round,line cap=round,fill=fillColor] (218.20,535.03) circle (  1.16);

\path[draw=drawColor,line width= 0.4pt,line join=round,line cap=round,fill=fillColor] (218.33,535.03) circle (  1.16);

\path[draw=drawColor,line width= 0.4pt,line join=round,line cap=round,fill=fillColor] (218.47,535.03) circle (  1.16);

\path[draw=drawColor,line width= 0.4pt,line join=round,line cap=round,fill=fillColor] (218.60,535.03) circle (  1.16);

\path[draw=drawColor,line width= 0.4pt,line join=round,line cap=round,fill=fillColor] (218.73,535.03) circle (  1.16);

\path[draw=drawColor,line width= 0.4pt,line join=round,line cap=round,fill=fillColor] (218.87,535.03) circle (  1.16);

\path[draw=drawColor,line width= 0.4pt,line join=round,line cap=round,fill=fillColor] (219.00,535.03) circle (  1.16);

\path[draw=drawColor,line width= 0.4pt,line join=round,line cap=round,fill=fillColor] (219.13,535.03) circle (  1.16);

\path[draw=drawColor,line width= 0.4pt,line join=round,line cap=round,fill=fillColor] (219.26,535.03) circle (  1.16);

\path[draw=drawColor,line width= 0.4pt,line join=round,line cap=round,fill=fillColor] (219.40,535.03) circle (  1.16);

\path[draw=drawColor,line width= 0.4pt,line join=round,line cap=round,fill=fillColor] (219.53,535.03) circle (  1.16);

\path[draw=drawColor,line width= 0.4pt,line join=round,line cap=round,fill=fillColor] (219.66,535.03) circle (  1.16);

\path[draw=drawColor,line width= 0.4pt,line join=round,line cap=round,fill=fillColor] (219.79,535.03) circle (  1.16);

\path[draw=drawColor,line width= 0.4pt,line join=round,line cap=round,fill=fillColor] (219.92,535.03) circle (  1.16);

\path[draw=drawColor,line width= 0.4pt,line join=round,line cap=round,fill=fillColor] (220.05,535.03) circle (  1.16);

\path[draw=drawColor,line width= 0.4pt,line join=round,line cap=round,fill=fillColor] (220.18,535.03) circle (  1.16);

\path[draw=drawColor,line width= 0.4pt,line join=round,line cap=round,fill=fillColor] (220.31,535.03) circle (  1.16);

\path[draw=drawColor,line width= 0.4pt,line join=round,line cap=round,fill=fillColor] (220.45,535.03) circle (  1.16);

\path[draw=drawColor,line width= 0.4pt,line join=round,line cap=round,fill=fillColor] (220.58,535.03) circle (  1.16);

\path[draw=drawColor,line width= 0.4pt,line join=round,line cap=round,fill=fillColor] (220.71,535.03) circle (  1.16);

\path[draw=drawColor,line width= 0.4pt,line join=round,line cap=round,fill=fillColor] (220.84,535.03) circle (  1.16);

\path[draw=drawColor,line width= 0.4pt,line join=round,line cap=round,fill=fillColor] (220.97,535.03) circle (  1.16);

\path[draw=drawColor,line width= 0.4pt,line join=round,line cap=round,fill=fillColor] (221.09,535.03) circle (  1.16);

\path[draw=drawColor,line width= 0.4pt,line join=round,line cap=round,fill=fillColor] (221.22,535.03) circle (  1.16);

\path[draw=drawColor,line width= 0.4pt,line join=round,line cap=round,fill=fillColor] (221.35,535.03) circle (  1.16);

\path[draw=drawColor,line width= 0.4pt,line join=round,line cap=round,fill=fillColor] (221.48,535.03) circle (  1.16);

\path[draw=drawColor,line width= 0.4pt,line join=round,line cap=round,fill=fillColor] (221.61,535.03) circle (  1.16);

\path[draw=drawColor,line width= 0.4pt,line join=round,line cap=round,fill=fillColor] (221.74,535.03) circle (  1.16);

\path[draw=drawColor,line width= 0.4pt,line join=round,line cap=round,fill=fillColor] (221.87,535.03) circle (  1.16);

\path[draw=drawColor,line width= 0.4pt,line join=round,line cap=round,fill=fillColor] (222.00,535.03) circle (  1.16);

\path[draw=drawColor,line width= 0.4pt,line join=round,line cap=round,fill=fillColor] (222.12,535.03) circle (  1.16);

\path[draw=drawColor,line width= 0.4pt,line join=round,line cap=round,fill=fillColor] (222.25,535.03) circle (  1.16);

\path[draw=drawColor,line width= 0.4pt,line join=round,line cap=round,fill=fillColor] (222.38,535.03) circle (  1.16);

\path[draw=drawColor,line width= 0.4pt,line join=round,line cap=round,fill=fillColor] (222.51,535.03) circle (  1.16);

\path[draw=drawColor,line width= 0.4pt,line join=round,line cap=round,fill=fillColor] (222.63,535.03) circle (  1.16);

\path[draw=drawColor,line width= 0.4pt,line join=round,line cap=round,fill=fillColor] (222.76,535.03) circle (  1.16);

\path[draw=drawColor,line width= 0.4pt,line join=round,line cap=round,fill=fillColor] (222.89,535.03) circle (  1.16);

\path[draw=drawColor,line width= 0.4pt,line join=round,line cap=round,fill=fillColor] (223.01,535.03) circle (  1.16);

\path[draw=drawColor,line width= 0.4pt,line join=round,line cap=round,fill=fillColor] (223.14,535.03) circle (  1.16);

\path[draw=drawColor,line width= 0.4pt,line join=round,line cap=round,fill=fillColor] (223.27,535.03) circle (  1.16);

\path[draw=drawColor,line width= 0.4pt,line join=round,line cap=round,fill=fillColor] (223.39,535.03) circle (  1.16);

\path[draw=drawColor,line width= 0.4pt,line join=round,line cap=round,fill=fillColor] (223.52,535.03) circle (  1.16);

\path[draw=drawColor,line width= 0.4pt,line join=round,line cap=round,fill=fillColor] (223.64,535.03) circle (  1.16);

\path[draw=drawColor,line width= 0.4pt,line join=round,line cap=round,fill=fillColor] (223.77,535.03) circle (  1.16);

\path[draw=drawColor,line width= 0.4pt,line join=round,line cap=round,fill=fillColor] (223.89,535.03) circle (  1.16);

\path[draw=drawColor,line width= 0.4pt,line join=round,line cap=round,fill=fillColor] (224.02,535.03) circle (  1.16);

\path[draw=drawColor,line width= 0.4pt,line join=round,line cap=round,fill=fillColor] (224.14,535.03) circle (  1.16);

\path[draw=drawColor,line width= 0.4pt,line join=round,line cap=round,fill=fillColor] (224.27,535.03) circle (  1.16);

\path[draw=drawColor,line width= 0.4pt,line join=round,line cap=round,fill=fillColor] (224.39,535.03) circle (  1.16);

\path[draw=drawColor,line width= 0.4pt,line join=round,line cap=round,fill=fillColor] (224.52,535.03) circle (  1.16);

\path[draw=drawColor,line width= 0.4pt,line join=round,line cap=round,fill=fillColor] (224.64,535.03) circle (  1.16);

\path[draw=drawColor,line width= 0.4pt,line join=round,line cap=round,fill=fillColor] (224.77,535.03) circle (  1.16);

\path[draw=drawColor,line width= 0.4pt,line join=round,line cap=round,fill=fillColor] (224.89,535.03) circle (  1.16);

\path[draw=drawColor,line width= 0.4pt,line join=round,line cap=round,fill=fillColor] (225.01,535.03) circle (  1.16);

\path[draw=drawColor,line width= 0.4pt,line join=round,line cap=round,fill=fillColor] (225.14,535.03) circle (  1.16);

\path[draw=drawColor,line width= 0.4pt,line join=round,line cap=round,fill=fillColor] (225.26,535.03) circle (  1.16);
\definecolor[named]{drawColor}{rgb}{0.30,0.69,0.29}
\definecolor[named]{fillColor}{rgb}{0.30,0.69,0.29}

\path[draw=drawColor,line width= 0.4pt,line join=round,line cap=round,fill=fillColor] ( 74.88,617.99) circle (  1.16);

\path[draw=drawColor,line width= 0.4pt,line join=round,line cap=round,fill=fillColor] ( 80.66,617.98) circle (  1.16);

\path[draw=drawColor,line width= 0.4pt,line join=round,line cap=round,fill=fillColor] ( 84.72,617.77) circle (  1.16);

\path[draw=drawColor,line width= 0.4pt,line join=round,line cap=round,fill=fillColor] ( 87.95,617.76) circle (  1.16);

\path[draw=drawColor,line width= 0.4pt,line join=round,line cap=round,fill=fillColor] ( 90.68,617.75) circle (  1.16);

\path[draw=drawColor,line width= 0.4pt,line join=round,line cap=round,fill=fillColor] ( 93.06,617.45) circle (  1.16);

\path[draw=drawColor,line width= 0.4pt,line join=round,line cap=round,fill=fillColor] ( 95.19,617.43) circle (  1.16);

\path[draw=drawColor,line width= 0.4pt,line join=round,line cap=round,fill=fillColor] ( 97.13,616.98) circle (  1.16);

\path[draw=drawColor,line width= 0.4pt,line join=round,line cap=round,fill=fillColor] ( 98.91,616.76) circle (  1.16);

\path[draw=drawColor,line width= 0.4pt,line join=round,line cap=round,fill=fillColor] (100.57,616.37) circle (  1.16);

\path[draw=drawColor,line width= 0.4pt,line join=round,line cap=round,fill=fillColor] (102.11,616.25) circle (  1.16);

\path[draw=drawColor,line width= 0.4pt,line join=round,line cap=round,fill=fillColor] (103.57,616.07) circle (  1.16);

\path[draw=drawColor,line width= 0.4pt,line join=round,line cap=round,fill=fillColor] (104.95,615.26) circle (  1.16);

\path[draw=drawColor,line width= 0.4pt,line join=round,line cap=round,fill=fillColor] (106.26,614.74) circle (  1.16);

\path[draw=drawColor,line width= 0.4pt,line join=round,line cap=round,fill=fillColor] (107.50,614.67) circle (  1.16);

\path[draw=drawColor,line width= 0.4pt,line join=round,line cap=round,fill=fillColor] (108.70,614.55) circle (  1.16);

\path[draw=drawColor,line width= 0.4pt,line join=round,line cap=round,fill=fillColor] (109.84,614.22) circle (  1.16);

\path[draw=drawColor,line width= 0.4pt,line join=round,line cap=round,fill=fillColor] (110.94,614.20) circle (  1.16);

\path[draw=drawColor,line width= 0.4pt,line join=round,line cap=round,fill=fillColor] (112.00,613.92) circle (  1.16);

\path[draw=drawColor,line width= 0.4pt,line join=round,line cap=round,fill=fillColor] (113.03,613.81) circle (  1.16);

\path[draw=drawColor,line width= 0.4pt,line join=round,line cap=round,fill=fillColor] (114.02,613.80) circle (  1.16);

\path[draw=drawColor,line width= 0.4pt,line join=round,line cap=round,fill=fillColor] (114.98,613.66) circle (  1.16);

\path[draw=drawColor,line width= 0.4pt,line join=round,line cap=round,fill=fillColor] (115.91,613.12) circle (  1.16);

\path[draw=drawColor,line width= 0.4pt,line join=round,line cap=round,fill=fillColor] (116.81,613.12) circle (  1.16);

\path[draw=drawColor,line width= 0.4pt,line join=round,line cap=round,fill=fillColor] (117.69,612.19) circle (  1.16);

\path[draw=drawColor,line width= 0.4pt,line join=round,line cap=round,fill=fillColor] (118.55,612.19) circle (  1.16);

\path[draw=drawColor,line width= 0.4pt,line join=round,line cap=round,fill=fillColor] (119.38,611.46) circle (  1.16);

\path[draw=drawColor,line width= 0.4pt,line join=round,line cap=round,fill=fillColor] (120.20,611.38) circle (  1.16);

\path[draw=drawColor,line width= 0.4pt,line join=round,line cap=round,fill=fillColor] (120.99,611.36) circle (  1.16);

\path[draw=drawColor,line width= 0.4pt,line join=round,line cap=round,fill=fillColor] (121.77,610.70) circle (  1.16);

\path[draw=drawColor,line width= 0.4pt,line join=round,line cap=round,fill=fillColor] (122.53,609.91) circle (  1.16);

\path[draw=drawColor,line width= 0.4pt,line join=round,line cap=round,fill=fillColor] (123.27,609.52) circle (  1.16);

\path[draw=drawColor,line width= 0.4pt,line join=round,line cap=round,fill=fillColor] (124.00,609.39) circle (  1.16);

\path[draw=drawColor,line width= 0.4pt,line join=round,line cap=round,fill=fillColor] (124.71,608.97) circle (  1.16);

\path[draw=drawColor,line width= 0.4pt,line join=round,line cap=round,fill=fillColor] (125.41,607.35) circle (  1.16);

\path[draw=drawColor,line width= 0.4pt,line join=round,line cap=round,fill=fillColor] (126.10,606.81) circle (  1.16);

\path[draw=drawColor,line width= 0.4pt,line join=round,line cap=round,fill=fillColor] (126.77,606.62) circle (  1.16);

\path[draw=drawColor,line width= 0.4pt,line join=round,line cap=round,fill=fillColor] (127.44,605.68) circle (  1.16);

\path[draw=drawColor,line width= 0.4pt,line join=round,line cap=round,fill=fillColor] (128.09,605.35) circle (  1.16);

\path[draw=drawColor,line width= 0.4pt,line join=round,line cap=round,fill=fillColor] (128.73,604.95) circle (  1.16);

\path[draw=drawColor,line width= 0.4pt,line join=round,line cap=round,fill=fillColor] (129.35,604.67) circle (  1.16);

\path[draw=drawColor,line width= 0.4pt,line join=round,line cap=round,fill=fillColor] (129.97,604.31) circle (  1.16);

\path[draw=drawColor,line width= 0.4pt,line join=round,line cap=round,fill=fillColor] (130.58,603.66) circle (  1.16);

\path[draw=drawColor,line width= 0.4pt,line join=round,line cap=round,fill=fillColor] (131.18,602.45) circle (  1.16);

\path[draw=drawColor,line width= 0.4pt,line join=round,line cap=round,fill=fillColor] (131.77,602.21) circle (  1.16);

\path[draw=drawColor,line width= 0.4pt,line join=round,line cap=round,fill=fillColor] (132.35,602.19) circle (  1.16);

\path[draw=drawColor,line width= 0.4pt,line join=round,line cap=round,fill=fillColor] (132.93,602.17) circle (  1.16);

\path[draw=drawColor,line width= 0.4pt,line join=round,line cap=round,fill=fillColor] (133.49,601.96) circle (  1.16);

\path[draw=drawColor,line width= 0.4pt,line join=round,line cap=round,fill=fillColor] (134.05,599.98) circle (  1.16);

\path[draw=drawColor,line width= 0.4pt,line join=round,line cap=round,fill=fillColor] (134.60,599.14) circle (  1.16);

\path[draw=drawColor,line width= 0.4pt,line join=round,line cap=round,fill=fillColor] (135.14,599.05) circle (  1.16);

\path[draw=drawColor,line width= 0.4pt,line join=round,line cap=round,fill=fillColor] (135.68,598.75) circle (  1.16);

\path[draw=drawColor,line width= 0.4pt,line join=round,line cap=round,fill=fillColor] (136.21,597.21) circle (  1.16);

\path[draw=drawColor,line width= 0.4pt,line join=round,line cap=round,fill=fillColor] (136.73,597.17) circle (  1.16);

\path[draw=drawColor,line width= 0.4pt,line join=round,line cap=round,fill=fillColor] (137.25,596.75) circle (  1.16);

\path[draw=drawColor,line width= 0.4pt,line join=round,line cap=round,fill=fillColor] (137.76,595.32) circle (  1.16);

\path[draw=drawColor,line width= 0.4pt,line join=round,line cap=round,fill=fillColor] (138.26,593.34) circle (  1.16);

\path[draw=drawColor,line width= 0.4pt,line join=round,line cap=round,fill=fillColor] (138.76,591.99) circle (  1.16);

\path[draw=drawColor,line width= 0.4pt,line join=round,line cap=round,fill=fillColor] (139.25,591.08) circle (  1.16);

\path[draw=drawColor,line width= 0.4pt,line join=round,line cap=round,fill=fillColor] (139.74,589.97) circle (  1.16);

\path[draw=drawColor,line width= 0.4pt,line join=round,line cap=round,fill=fillColor] (140.22,589.85) circle (  1.16);

\path[draw=drawColor,line width= 0.4pt,line join=round,line cap=round,fill=fillColor] (140.70,588.70) circle (  1.16);

\path[draw=drawColor,line width= 0.4pt,line join=round,line cap=round,fill=fillColor] (141.17,586.76) circle (  1.16);

\path[draw=drawColor,line width= 0.4pt,line join=round,line cap=round,fill=fillColor] (141.63,586.04) circle (  1.16);

\path[draw=drawColor,line width= 0.4pt,line join=round,line cap=round,fill=fillColor] (142.09,585.64) circle (  1.16);

\path[draw=drawColor,line width= 0.4pt,line join=round,line cap=round,fill=fillColor] (142.55,585.19) circle (  1.16);

\path[draw=drawColor,line width= 0.4pt,line join=round,line cap=round,fill=fillColor] (143.00,582.86) circle (  1.16);

\path[draw=drawColor,line width= 0.4pt,line join=round,line cap=round,fill=fillColor] (143.45,582.80) circle (  1.16);

\path[draw=drawColor,line width= 0.4pt,line join=round,line cap=round,fill=fillColor] (143.89,582.79) circle (  1.16);

\path[draw=drawColor,line width= 0.4pt,line join=round,line cap=round,fill=fillColor] (144.33,582.39) circle (  1.16);

\path[draw=drawColor,line width= 0.4pt,line join=round,line cap=round,fill=fillColor] (144.77,582.21) circle (  1.16);

\path[draw=drawColor,line width= 0.4pt,line join=round,line cap=round,fill=fillColor] (145.20,581.82) circle (  1.16);

\path[draw=drawColor,line width= 0.4pt,line join=round,line cap=round,fill=fillColor] (145.62,581.66) circle (  1.16);

\path[draw=drawColor,line width= 0.4pt,line join=round,line cap=round,fill=fillColor] (146.05,581.04) circle (  1.16);

\path[draw=drawColor,line width= 0.4pt,line join=round,line cap=round,fill=fillColor] (146.46,579.79) circle (  1.16);

\path[draw=drawColor,line width= 0.4pt,line join=round,line cap=round,fill=fillColor] (146.88,579.64) circle (  1.16);

\path[draw=drawColor,line width= 0.4pt,line join=round,line cap=round,fill=fillColor] (147.29,578.78) circle (  1.16);

\path[draw=drawColor,line width= 0.4pt,line join=round,line cap=round,fill=fillColor] (147.70,578.66) circle (  1.16);

\path[draw=drawColor,line width= 0.4pt,line join=round,line cap=round,fill=fillColor] (148.10,578.22) circle (  1.16);

\path[draw=drawColor,line width= 0.4pt,line join=round,line cap=round,fill=fillColor] (148.50,578.11) circle (  1.16);

\path[draw=drawColor,line width= 0.4pt,line join=round,line cap=round,fill=fillColor] (148.90,577.65) circle (  1.16);

\path[draw=drawColor,line width= 0.4pt,line join=round,line cap=round,fill=fillColor] (149.30,577.22) circle (  1.16);

\path[draw=drawColor,line width= 0.4pt,line join=round,line cap=round,fill=fillColor] (149.69,577.21) circle (  1.16);

\path[draw=drawColor,line width= 0.4pt,line join=round,line cap=round,fill=fillColor] (150.08,577.08) circle (  1.16);

\path[draw=drawColor,line width= 0.4pt,line join=round,line cap=round,fill=fillColor] (150.46,576.57) circle (  1.16);

\path[draw=drawColor,line width= 0.4pt,line join=round,line cap=round,fill=fillColor] (150.84,575.89) circle (  1.16);

\path[draw=drawColor,line width= 0.4pt,line join=round,line cap=round,fill=fillColor] (151.22,575.88) circle (  1.16);

\path[draw=drawColor,line width= 0.4pt,line join=round,line cap=round,fill=fillColor] (151.60,575.84) circle (  1.16);

\path[draw=drawColor,line width= 0.4pt,line join=round,line cap=round,fill=fillColor] (151.97,575.79) circle (  1.16);

\path[draw=drawColor,line width= 0.4pt,line join=round,line cap=round,fill=fillColor] (152.34,575.22) circle (  1.16);

\path[draw=drawColor,line width= 0.4pt,line join=round,line cap=round,fill=fillColor] (152.71,574.91) circle (  1.16);

\path[draw=drawColor,line width= 0.4pt,line join=round,line cap=round,fill=fillColor] (153.08,574.63) circle (  1.16);

\path[draw=drawColor,line width= 0.4pt,line join=round,line cap=round,fill=fillColor] (153.44,574.53) circle (  1.16);

\path[draw=drawColor,line width= 0.4pt,line join=round,line cap=round,fill=fillColor] (153.80,574.02) circle (  1.16);

\path[draw=drawColor,line width= 0.4pt,line join=round,line cap=round,fill=fillColor] (154.16,573.71) circle (  1.16);

\path[draw=drawColor,line width= 0.4pt,line join=round,line cap=round,fill=fillColor] (154.51,573.66) circle (  1.16);

\path[draw=drawColor,line width= 0.4pt,line join=round,line cap=round,fill=fillColor] (154.86,573.24) circle (  1.16);

\path[draw=drawColor,line width= 0.4pt,line join=round,line cap=round,fill=fillColor] (155.22,573.02) circle (  1.16);

\path[draw=drawColor,line width= 0.4pt,line join=round,line cap=round,fill=fillColor] (155.56,572.80) circle (  1.16);

\path[draw=drawColor,line width= 0.4pt,line join=round,line cap=round,fill=fillColor] (155.91,571.67) circle (  1.16);

\path[draw=drawColor,line width= 0.4pt,line join=round,line cap=round,fill=fillColor] (156.25,571.15) circle (  1.16);

\path[draw=drawColor,line width= 0.4pt,line join=round,line cap=round,fill=fillColor] (156.59,570.92) circle (  1.16);

\path[draw=drawColor,line width= 0.4pt,line join=round,line cap=round,fill=fillColor] (156.93,570.81) circle (  1.16);

\path[draw=drawColor,line width= 0.4pt,line join=round,line cap=round,fill=fillColor] (157.27,570.72) circle (  1.16);

\path[draw=drawColor,line width= 0.4pt,line join=round,line cap=round,fill=fillColor] (157.60,569.83) circle (  1.16);

\path[draw=drawColor,line width= 0.4pt,line join=round,line cap=round,fill=fillColor] (157.93,569.35) circle (  1.16);

\path[draw=drawColor,line width= 0.4pt,line join=round,line cap=round,fill=fillColor] (158.26,569.34) circle (  1.16);

\path[draw=drawColor,line width= 0.4pt,line join=round,line cap=round,fill=fillColor] (158.59,569.29) circle (  1.16);

\path[draw=drawColor,line width= 0.4pt,line join=round,line cap=round,fill=fillColor] (158.92,569.26) circle (  1.16);

\path[draw=drawColor,line width= 0.4pt,line join=round,line cap=round,fill=fillColor] (159.24,569.20) circle (  1.16);

\path[draw=drawColor,line width= 0.4pt,line join=round,line cap=round,fill=fillColor] (159.56,569.20) circle (  1.16);

\path[draw=drawColor,line width= 0.4pt,line join=round,line cap=round,fill=fillColor] (159.88,569.06) circle (  1.16);

\path[draw=drawColor,line width= 0.4pt,line join=round,line cap=round,fill=fillColor] (160.20,568.94) circle (  1.16);

\path[draw=drawColor,line width= 0.4pt,line join=round,line cap=round,fill=fillColor] (160.52,568.67) circle (  1.16);

\path[draw=drawColor,line width= 0.4pt,line join=round,line cap=round,fill=fillColor] (160.83,568.56) circle (  1.16);

\path[draw=drawColor,line width= 0.4pt,line join=round,line cap=round,fill=fillColor] (161.15,568.42) circle (  1.16);

\path[draw=drawColor,line width= 0.4pt,line join=round,line cap=round,fill=fillColor] (161.46,568.23) circle (  1.16);

\path[draw=drawColor,line width= 0.4pt,line join=round,line cap=round,fill=fillColor] (161.77,567.33) circle (  1.16);

\path[draw=drawColor,line width= 0.4pt,line join=round,line cap=round,fill=fillColor] (162.07,566.95) circle (  1.16);

\path[draw=drawColor,line width= 0.4pt,line join=round,line cap=round,fill=fillColor] (162.38,566.46) circle (  1.16);

\path[draw=drawColor,line width= 0.4pt,line join=round,line cap=round,fill=fillColor] (162.68,566.44) circle (  1.16);

\path[draw=drawColor,line width= 0.4pt,line join=round,line cap=round,fill=fillColor] (162.99,566.29) circle (  1.16);

\path[draw=drawColor,line width= 0.4pt,line join=round,line cap=round,fill=fillColor] (163.29,566.19) circle (  1.16);

\path[draw=drawColor,line width= 0.4pt,line join=round,line cap=round,fill=fillColor] (163.59,566.07) circle (  1.16);

\path[draw=drawColor,line width= 0.4pt,line join=round,line cap=round,fill=fillColor] (163.88,565.90) circle (  1.16);

\path[draw=drawColor,line width= 0.4pt,line join=round,line cap=round,fill=fillColor] (164.18,565.75) circle (  1.16);

\path[draw=drawColor,line width= 0.4pt,line join=round,line cap=round,fill=fillColor] (164.47,565.67) circle (  1.16);

\path[draw=drawColor,line width= 0.4pt,line join=round,line cap=round,fill=fillColor] (164.77,565.60) circle (  1.16);

\path[draw=drawColor,line width= 0.4pt,line join=round,line cap=round,fill=fillColor] (165.06,565.43) circle (  1.16);

\path[draw=drawColor,line width= 0.4pt,line join=round,line cap=round,fill=fillColor] (165.35,565.28) circle (  1.16);

\path[draw=drawColor,line width= 0.4pt,line join=round,line cap=round,fill=fillColor] (165.64,565.18) circle (  1.16);

\path[draw=drawColor,line width= 0.4pt,line join=round,line cap=round,fill=fillColor] (165.92,565.04) circle (  1.16);

\path[draw=drawColor,line width= 0.4pt,line join=round,line cap=round,fill=fillColor] (166.21,564.96) circle (  1.16);

\path[draw=drawColor,line width= 0.4pt,line join=round,line cap=round,fill=fillColor] (166.49,564.83) circle (  1.16);

\path[draw=drawColor,line width= 0.4pt,line join=round,line cap=round,fill=fillColor] (166.77,564.46) circle (  1.16);

\path[draw=drawColor,line width= 0.4pt,line join=round,line cap=round,fill=fillColor] (167.06,564.28) circle (  1.16);

\path[draw=drawColor,line width= 0.4pt,line join=round,line cap=round,fill=fillColor] (167.34,564.15) circle (  1.16);

\path[draw=drawColor,line width= 0.4pt,line join=round,line cap=round,fill=fillColor] (167.61,564.15) circle (  1.16);

\path[draw=drawColor,line width= 0.4pt,line join=round,line cap=round,fill=fillColor] (167.89,564.03) circle (  1.16);

\path[draw=drawColor,line width= 0.4pt,line join=round,line cap=round,fill=fillColor] (168.17,563.90) circle (  1.16);

\path[draw=drawColor,line width= 0.4pt,line join=round,line cap=round,fill=fillColor] (168.44,563.85) circle (  1.16);

\path[draw=drawColor,line width= 0.4pt,line join=round,line cap=round,fill=fillColor] (168.71,563.79) circle (  1.16);

\path[draw=drawColor,line width= 0.4pt,line join=round,line cap=round,fill=fillColor] (168.99,563.76) circle (  1.16);

\path[draw=drawColor,line width= 0.4pt,line join=round,line cap=round,fill=fillColor] (169.26,563.75) circle (  1.16);

\path[draw=drawColor,line width= 0.4pt,line join=round,line cap=round,fill=fillColor] (169.53,563.42) circle (  1.16);

\path[draw=drawColor,line width= 0.4pt,line join=round,line cap=round,fill=fillColor] (169.79,563.39) circle (  1.16);

\path[draw=drawColor,line width= 0.4pt,line join=round,line cap=round,fill=fillColor] (170.06,562.83) circle (  1.16);

\path[draw=drawColor,line width= 0.4pt,line join=round,line cap=round,fill=fillColor] (170.33,562.81) circle (  1.16);

\path[draw=drawColor,line width= 0.4pt,line join=round,line cap=round,fill=fillColor] (170.59,562.55) circle (  1.16);

\path[draw=drawColor,line width= 0.4pt,line join=round,line cap=round,fill=fillColor] (170.85,562.51) circle (  1.16);

\path[draw=drawColor,line width= 0.4pt,line join=round,line cap=round,fill=fillColor] (171.12,562.19) circle (  1.16);

\path[draw=drawColor,line width= 0.4pt,line join=round,line cap=round,fill=fillColor] (171.38,562.17) circle (  1.16);

\path[draw=drawColor,line width= 0.4pt,line join=round,line cap=round,fill=fillColor] (171.64,562.07) circle (  1.16);

\path[draw=drawColor,line width= 0.4pt,line join=round,line cap=round,fill=fillColor] (171.90,562.03) circle (  1.16);

\path[draw=drawColor,line width= 0.4pt,line join=round,line cap=round,fill=fillColor] (172.15,561.80) circle (  1.16);

\path[draw=drawColor,line width= 0.4pt,line join=round,line cap=round,fill=fillColor] (172.41,561.65) circle (  1.16);

\path[draw=drawColor,line width= 0.4pt,line join=round,line cap=round,fill=fillColor] (172.67,561.44) circle (  1.16);

\path[draw=drawColor,line width= 0.4pt,line join=round,line cap=round,fill=fillColor] (172.92,561.24) circle (  1.16);

\path[draw=drawColor,line width= 0.4pt,line join=round,line cap=round,fill=fillColor] (173.17,561.18) circle (  1.16);

\path[draw=drawColor,line width= 0.4pt,line join=round,line cap=round,fill=fillColor] (173.43,561.17) circle (  1.16);

\path[draw=drawColor,line width= 0.4pt,line join=round,line cap=round,fill=fillColor] (173.68,561.07) circle (  1.16);

\path[draw=drawColor,line width= 0.4pt,line join=round,line cap=round,fill=fillColor] (173.93,560.96) circle (  1.16);

\path[draw=drawColor,line width= 0.4pt,line join=round,line cap=round,fill=fillColor] (174.18,560.89) circle (  1.16);

\path[draw=drawColor,line width= 0.4pt,line join=round,line cap=round,fill=fillColor] (174.42,560.86) circle (  1.16);

\path[draw=drawColor,line width= 0.4pt,line join=round,line cap=round,fill=fillColor] (174.67,560.85) circle (  1.16);

\path[draw=drawColor,line width= 0.4pt,line join=round,line cap=round,fill=fillColor] (174.92,560.79) circle (  1.16);

\path[draw=drawColor,line width= 0.4pt,line join=round,line cap=round,fill=fillColor] (175.16,560.77) circle (  1.16);

\path[draw=drawColor,line width= 0.4pt,line join=round,line cap=round,fill=fillColor] (175.41,560.68) circle (  1.16);

\path[draw=drawColor,line width= 0.4pt,line join=round,line cap=round,fill=fillColor] (175.65,560.58) circle (  1.16);

\path[draw=drawColor,line width= 0.4pt,line join=round,line cap=round,fill=fillColor] (175.89,560.55) circle (  1.16);

\path[draw=drawColor,line width= 0.4pt,line join=round,line cap=round,fill=fillColor] (176.13,560.54) circle (  1.16);

\path[draw=drawColor,line width= 0.4pt,line join=round,line cap=round,fill=fillColor] (176.37,560.24) circle (  1.16);

\path[draw=drawColor,line width= 0.4pt,line join=round,line cap=round,fill=fillColor] (176.61,560.19) circle (  1.16);

\path[draw=drawColor,line width= 0.4pt,line join=round,line cap=round,fill=fillColor] (176.85,560.17) circle (  1.16);

\path[draw=drawColor,line width= 0.4pt,line join=round,line cap=round,fill=fillColor] (177.09,560.09) circle (  1.16);

\path[draw=drawColor,line width= 0.4pt,line join=round,line cap=round,fill=fillColor] (177.32,560.01) circle (  1.16);

\path[draw=drawColor,line width= 0.4pt,line join=round,line cap=round,fill=fillColor] (177.56,559.83) circle (  1.16);

\path[draw=drawColor,line width= 0.4pt,line join=round,line cap=round,fill=fillColor] (177.80,559.52) circle (  1.16);

\path[draw=drawColor,line width= 0.4pt,line join=round,line cap=round,fill=fillColor] (178.03,559.29) circle (  1.16);

\path[draw=drawColor,line width= 0.4pt,line join=round,line cap=round,fill=fillColor] (178.26,559.15) circle (  1.16);

\path[draw=drawColor,line width= 0.4pt,line join=round,line cap=round,fill=fillColor] (178.49,558.90) circle (  1.16);

\path[draw=drawColor,line width= 0.4pt,line join=round,line cap=round,fill=fillColor] (178.73,558.89) circle (  1.16);

\path[draw=drawColor,line width= 0.4pt,line join=round,line cap=round,fill=fillColor] (178.96,558.70) circle (  1.16);

\path[draw=drawColor,line width= 0.4pt,line join=round,line cap=round,fill=fillColor] (179.19,558.64) circle (  1.16);

\path[draw=drawColor,line width= 0.4pt,line join=round,line cap=round,fill=fillColor] (179.41,558.58) circle (  1.16);

\path[draw=drawColor,line width= 0.4pt,line join=round,line cap=round,fill=fillColor] (179.64,558.35) circle (  1.16);

\path[draw=drawColor,line width= 0.4pt,line join=round,line cap=round,fill=fillColor] (179.87,558.30) circle (  1.16);

\path[draw=drawColor,line width= 0.4pt,line join=round,line cap=round,fill=fillColor] (180.10,557.98) circle (  1.16);

\path[draw=drawColor,line width= 0.4pt,line join=round,line cap=round,fill=fillColor] (180.32,557.84) circle (  1.16);

\path[draw=drawColor,line width= 0.4pt,line join=round,line cap=round,fill=fillColor] (180.55,557.83) circle (  1.16);

\path[draw=drawColor,line width= 0.4pt,line join=round,line cap=round,fill=fillColor] (180.77,557.68) circle (  1.16);

\path[draw=drawColor,line width= 0.4pt,line join=round,line cap=round,fill=fillColor] (180.99,557.68) circle (  1.16);

\path[draw=drawColor,line width= 0.4pt,line join=round,line cap=round,fill=fillColor] (181.22,557.64) circle (  1.16);

\path[draw=drawColor,line width= 0.4pt,line join=round,line cap=round,fill=fillColor] (181.44,557.44) circle (  1.16);

\path[draw=drawColor,line width= 0.4pt,line join=round,line cap=round,fill=fillColor] (181.66,557.35) circle (  1.16);

\path[draw=drawColor,line width= 0.4pt,line join=round,line cap=round,fill=fillColor] (181.88,557.31) circle (  1.16);

\path[draw=drawColor,line width= 0.4pt,line join=round,line cap=round,fill=fillColor] (182.10,557.25) circle (  1.16);

\path[draw=drawColor,line width= 0.4pt,line join=round,line cap=round,fill=fillColor] (182.32,557.23) circle (  1.16);

\path[draw=drawColor,line width= 0.4pt,line join=round,line cap=round,fill=fillColor] (182.54,556.99) circle (  1.16);

\path[draw=drawColor,line width= 0.4pt,line join=round,line cap=round,fill=fillColor] (182.75,556.96) circle (  1.16);

\path[draw=drawColor,line width= 0.4pt,line join=round,line cap=round,fill=fillColor] (182.97,556.94) circle (  1.16);

\path[draw=drawColor,line width= 0.4pt,line join=round,line cap=round,fill=fillColor] (183.19,556.93) circle (  1.16);

\path[draw=drawColor,line width= 0.4pt,line join=round,line cap=round,fill=fillColor] (183.40,556.90) circle (  1.16);

\path[draw=drawColor,line width= 0.4pt,line join=round,line cap=round,fill=fillColor] (183.61,556.84) circle (  1.16);

\path[draw=drawColor,line width= 0.4pt,line join=round,line cap=round,fill=fillColor] (183.83,556.69) circle (  1.16);

\path[draw=drawColor,line width= 0.4pt,line join=round,line cap=round,fill=fillColor] (184.04,556.58) circle (  1.16);

\path[draw=drawColor,line width= 0.4pt,line join=round,line cap=round,fill=fillColor] (184.25,556.50) circle (  1.16);

\path[draw=drawColor,line width= 0.4pt,line join=round,line cap=round,fill=fillColor] (184.47,556.02) circle (  1.16);

\path[draw=drawColor,line width= 0.4pt,line join=round,line cap=round,fill=fillColor] (184.68,555.96) circle (  1.16);

\path[draw=drawColor,line width= 0.4pt,line join=round,line cap=round,fill=fillColor] (184.89,555.88) circle (  1.16);

\path[draw=drawColor,line width= 0.4pt,line join=round,line cap=round,fill=fillColor] (185.10,555.77) circle (  1.16);

\path[draw=drawColor,line width= 0.4pt,line join=round,line cap=round,fill=fillColor] (185.31,555.62) circle (  1.16);

\path[draw=drawColor,line width= 0.4pt,line join=round,line cap=round,fill=fillColor] (185.51,555.48) circle (  1.16);

\path[draw=drawColor,line width= 0.4pt,line join=round,line cap=round,fill=fillColor] (185.72,555.43) circle (  1.16);

\path[draw=drawColor,line width= 0.4pt,line join=round,line cap=round,fill=fillColor] (185.93,555.35) circle (  1.16);

\path[draw=drawColor,line width= 0.4pt,line join=round,line cap=round,fill=fillColor] (186.13,555.30) circle (  1.16);

\path[draw=drawColor,line width= 0.4pt,line join=round,line cap=round,fill=fillColor] (186.34,555.20) circle (  1.16);

\path[draw=drawColor,line width= 0.4pt,line join=round,line cap=round,fill=fillColor] (186.55,555.06) circle (  1.16);

\path[draw=drawColor,line width= 0.4pt,line join=round,line cap=round,fill=fillColor] (186.75,554.70) circle (  1.16);

\path[draw=drawColor,line width= 0.4pt,line join=round,line cap=round,fill=fillColor] (186.95,554.63) circle (  1.16);

\path[draw=drawColor,line width= 0.4pt,line join=round,line cap=round,fill=fillColor] (187.16,554.42) circle (  1.16);

\path[draw=drawColor,line width= 0.4pt,line join=round,line cap=round,fill=fillColor] (187.36,554.23) circle (  1.16);

\path[draw=drawColor,line width= 0.4pt,line join=round,line cap=round,fill=fillColor] (187.56,554.15) circle (  1.16);

\path[draw=drawColor,line width= 0.4pt,line join=round,line cap=round,fill=fillColor] (187.76,554.08) circle (  1.16);

\path[draw=drawColor,line width= 0.4pt,line join=round,line cap=round,fill=fillColor] (187.96,554.07) circle (  1.16);

\path[draw=drawColor,line width= 0.4pt,line join=round,line cap=round,fill=fillColor] (188.16,554.06) circle (  1.16);

\path[draw=drawColor,line width= 0.4pt,line join=round,line cap=round,fill=fillColor] (188.36,553.88) circle (  1.16);

\path[draw=drawColor,line width= 0.4pt,line join=round,line cap=round,fill=fillColor] (188.56,553.82) circle (  1.16);

\path[draw=drawColor,line width= 0.4pt,line join=round,line cap=round,fill=fillColor] (188.76,553.73) circle (  1.16);

\path[draw=drawColor,line width= 0.4pt,line join=round,line cap=round,fill=fillColor] (188.96,553.67) circle (  1.16);

\path[draw=drawColor,line width= 0.4pt,line join=round,line cap=round,fill=fillColor] (189.16,553.57) circle (  1.16);

\path[draw=drawColor,line width= 0.4pt,line join=round,line cap=round,fill=fillColor] (189.35,553.40) circle (  1.16);

\path[draw=drawColor,line width= 0.4pt,line join=round,line cap=round,fill=fillColor] (189.55,553.33) circle (  1.16);

\path[draw=drawColor,line width= 0.4pt,line join=round,line cap=round,fill=fillColor] (189.74,553.27) circle (  1.16);

\path[draw=drawColor,line width= 0.4pt,line join=round,line cap=round,fill=fillColor] (189.94,553.09) circle (  1.16);

\path[draw=drawColor,line width= 0.4pt,line join=round,line cap=round,fill=fillColor] (190.13,553.08) circle (  1.16);

\path[draw=drawColor,line width= 0.4pt,line join=round,line cap=round,fill=fillColor] (190.33,553.08) circle (  1.16);

\path[draw=drawColor,line width= 0.4pt,line join=round,line cap=round,fill=fillColor] (190.52,552.90) circle (  1.16);

\path[draw=drawColor,line width= 0.4pt,line join=round,line cap=round,fill=fillColor] (190.71,552.79) circle (  1.16);

\path[draw=drawColor,line width= 0.4pt,line join=round,line cap=round,fill=fillColor] (190.91,552.64) circle (  1.16);

\path[draw=drawColor,line width= 0.4pt,line join=round,line cap=round,fill=fillColor] (191.10,552.64) circle (  1.16);

\path[draw=drawColor,line width= 0.4pt,line join=round,line cap=round,fill=fillColor] (191.29,552.58) circle (  1.16);

\path[draw=drawColor,line width= 0.4pt,line join=round,line cap=round,fill=fillColor] (191.48,552.47) circle (  1.16);

\path[draw=drawColor,line width= 0.4pt,line join=round,line cap=round,fill=fillColor] (191.67,552.43) circle (  1.16);

\path[draw=drawColor,line width= 0.4pt,line join=round,line cap=round,fill=fillColor] (191.86,552.41) circle (  1.16);

\path[draw=drawColor,line width= 0.4pt,line join=round,line cap=round,fill=fillColor] (192.05,552.27) circle (  1.16);

\path[draw=drawColor,line width= 0.4pt,line join=round,line cap=round,fill=fillColor] (192.24,551.95) circle (  1.16);

\path[draw=drawColor,line width= 0.4pt,line join=round,line cap=round,fill=fillColor] (192.43,551.74) circle (  1.16);

\path[draw=drawColor,line width= 0.4pt,line join=round,line cap=round,fill=fillColor] (192.61,551.57) circle (  1.16);

\path[draw=drawColor,line width= 0.4pt,line join=round,line cap=round,fill=fillColor] (192.80,551.11) circle (  1.16);

\path[draw=drawColor,line width= 0.4pt,line join=round,line cap=round,fill=fillColor] (192.99,551.10) circle (  1.16);

\path[draw=drawColor,line width= 0.4pt,line join=round,line cap=round,fill=fillColor] (193.17,551.06) circle (  1.16);

\path[draw=drawColor,line width= 0.4pt,line join=round,line cap=round,fill=fillColor] (193.36,551.04) circle (  1.16);

\path[draw=drawColor,line width= 0.4pt,line join=round,line cap=round,fill=fillColor] (193.54,551.01) circle (  1.16);

\path[draw=drawColor,line width= 0.4pt,line join=round,line cap=round,fill=fillColor] (193.73,550.98) circle (  1.16);

\path[draw=drawColor,line width= 0.4pt,line join=round,line cap=round,fill=fillColor] (193.91,550.94) circle (  1.16);

\path[draw=drawColor,line width= 0.4pt,line join=round,line cap=round,fill=fillColor] (194.10,550.79) circle (  1.16);

\path[draw=drawColor,line width= 0.4pt,line join=round,line cap=round,fill=fillColor] (194.28,550.78) circle (  1.16);

\path[draw=drawColor,line width= 0.4pt,line join=round,line cap=round,fill=fillColor] (194.46,550.72) circle (  1.16);

\path[draw=drawColor,line width= 0.4pt,line join=round,line cap=round,fill=fillColor] (194.65,550.71) circle (  1.16);

\path[draw=drawColor,line width= 0.4pt,line join=round,line cap=round,fill=fillColor] (194.83,550.56) circle (  1.16);

\path[draw=drawColor,line width= 0.4pt,line join=round,line cap=round,fill=fillColor] (195.01,550.02) circle (  1.16);

\path[draw=drawColor,line width= 0.4pt,line join=round,line cap=round,fill=fillColor] (195.19,549.90) circle (  1.16);

\path[draw=drawColor,line width= 0.4pt,line join=round,line cap=round,fill=fillColor] (195.37,549.83) circle (  1.16);

\path[draw=drawColor,line width= 0.4pt,line join=round,line cap=round,fill=fillColor] (195.55,549.75) circle (  1.16);

\path[draw=drawColor,line width= 0.4pt,line join=round,line cap=round,fill=fillColor] (195.73,549.71) circle (  1.16);

\path[draw=drawColor,line width= 0.4pt,line join=round,line cap=round,fill=fillColor] (195.91,549.66) circle (  1.16);

\path[draw=drawColor,line width= 0.4pt,line join=round,line cap=round,fill=fillColor] (196.09,549.36) circle (  1.16);

\path[draw=drawColor,line width= 0.4pt,line join=round,line cap=round,fill=fillColor] (196.27,549.32) circle (  1.16);

\path[draw=drawColor,line width= 0.4pt,line join=round,line cap=round,fill=fillColor] (196.44,549.25) circle (  1.16);

\path[draw=drawColor,line width= 0.4pt,line join=round,line cap=round,fill=fillColor] (196.62,549.19) circle (  1.16);

\path[draw=drawColor,line width= 0.4pt,line join=round,line cap=round,fill=fillColor] (196.80,549.14) circle (  1.16);

\path[draw=drawColor,line width= 0.4pt,line join=round,line cap=round,fill=fillColor] (196.97,548.97) circle (  1.16);

\path[draw=drawColor,line width= 0.4pt,line join=round,line cap=round,fill=fillColor] (197.15,548.88) circle (  1.16);

\path[draw=drawColor,line width= 0.4pt,line join=round,line cap=round,fill=fillColor] (197.33,548.78) circle (  1.16);

\path[draw=drawColor,line width= 0.4pt,line join=round,line cap=round,fill=fillColor] (197.50,548.73) circle (  1.16);

\path[draw=drawColor,line width= 0.4pt,line join=round,line cap=round,fill=fillColor] (197.68,548.70) circle (  1.16);

\path[draw=drawColor,line width= 0.4pt,line join=round,line cap=round,fill=fillColor] (197.85,548.70) circle (  1.16);

\path[draw=drawColor,line width= 0.4pt,line join=round,line cap=round,fill=fillColor] (198.02,548.66) circle (  1.16);

\path[draw=drawColor,line width= 0.4pt,line join=round,line cap=round,fill=fillColor] (198.20,547.91) circle (  1.16);

\path[draw=drawColor,line width= 0.4pt,line join=round,line cap=round,fill=fillColor] (198.37,547.78) circle (  1.16);

\path[draw=drawColor,line width= 0.4pt,line join=round,line cap=round,fill=fillColor] (198.54,547.77) circle (  1.16);

\path[draw=drawColor,line width= 0.4pt,line join=round,line cap=round,fill=fillColor] (198.72,547.74) circle (  1.16);

\path[draw=drawColor,line width= 0.4pt,line join=round,line cap=round,fill=fillColor] (198.89,547.62) circle (  1.16);

\path[draw=drawColor,line width= 0.4pt,line join=round,line cap=round,fill=fillColor] (199.06,547.56) circle (  1.16);

\path[draw=drawColor,line width= 0.4pt,line join=round,line cap=round,fill=fillColor] (199.23,547.56) circle (  1.16);

\path[draw=drawColor,line width= 0.4pt,line join=round,line cap=round,fill=fillColor] (199.40,547.54) circle (  1.16);

\path[draw=drawColor,line width= 0.4pt,line join=round,line cap=round,fill=fillColor] (199.57,547.53) circle (  1.16);

\path[draw=drawColor,line width= 0.4pt,line join=round,line cap=round,fill=fillColor] (199.74,547.53) circle (  1.16);

\path[draw=drawColor,line width= 0.4pt,line join=round,line cap=round,fill=fillColor] (199.91,547.45) circle (  1.16);

\path[draw=drawColor,line width= 0.4pt,line join=round,line cap=round,fill=fillColor] (200.08,547.23) circle (  1.16);

\path[draw=drawColor,line width= 0.4pt,line join=round,line cap=round,fill=fillColor] (200.25,547.09) circle (  1.16);

\path[draw=drawColor,line width= 0.4pt,line join=round,line cap=round,fill=fillColor] (200.42,546.97) circle (  1.16);

\path[draw=drawColor,line width= 0.4pt,line join=round,line cap=round,fill=fillColor] (200.58,546.85) circle (  1.16);

\path[draw=drawColor,line width= 0.4pt,line join=round,line cap=round,fill=fillColor] (200.75,546.78) circle (  1.16);

\path[draw=drawColor,line width= 0.4pt,line join=round,line cap=round,fill=fillColor] (200.92,546.70) circle (  1.16);

\path[draw=drawColor,line width= 0.4pt,line join=round,line cap=round,fill=fillColor] (201.09,546.45) circle (  1.16);

\path[draw=drawColor,line width= 0.4pt,line join=round,line cap=round,fill=fillColor] (201.25,546.15) circle (  1.16);

\path[draw=drawColor,line width= 0.4pt,line join=round,line cap=round,fill=fillColor] (201.42,546.01) circle (  1.16);

\path[draw=drawColor,line width= 0.4pt,line join=round,line cap=round,fill=fillColor] (201.58,545.87) circle (  1.16);

\path[draw=drawColor,line width= 0.4pt,line join=round,line cap=round,fill=fillColor] (201.75,545.63) circle (  1.16);

\path[draw=drawColor,line width= 0.4pt,line join=round,line cap=round,fill=fillColor] (201.91,545.22) circle (  1.16);

\path[draw=drawColor,line width= 0.4pt,line join=round,line cap=round,fill=fillColor] (202.08,545.21) circle (  1.16);

\path[draw=drawColor,line width= 0.4pt,line join=round,line cap=round,fill=fillColor] (202.24,545.21) circle (  1.16);

\path[draw=drawColor,line width= 0.4pt,line join=round,line cap=round,fill=fillColor] (202.41,544.97) circle (  1.16);

\path[draw=drawColor,line width= 0.4pt,line join=round,line cap=round,fill=fillColor] (202.57,544.80) circle (  1.16);

\path[draw=drawColor,line width= 0.4pt,line join=round,line cap=round,fill=fillColor] (202.73,544.50) circle (  1.16);

\path[draw=drawColor,line width= 0.4pt,line join=round,line cap=round,fill=fillColor] (202.90,544.41) circle (  1.16);

\path[draw=drawColor,line width= 0.4pt,line join=round,line cap=round,fill=fillColor] (203.06,543.82) circle (  1.16);

\path[draw=drawColor,line width= 0.4pt,line join=round,line cap=round,fill=fillColor] (203.22,543.16) circle (  1.16);

\path[draw=drawColor,line width= 0.4pt,line join=round,line cap=round,fill=fillColor] (203.38,542.85) circle (  1.16);

\path[draw=drawColor,line width= 0.4pt,line join=round,line cap=round,fill=fillColor] (203.54,540.76) circle (  1.16);

\path[draw=drawColor,line width= 0.4pt,line join=round,line cap=round,fill=fillColor] (203.71,535.03) circle (  1.16);

\path[draw=drawColor,line width= 0.4pt,line join=round,line cap=round,fill=fillColor] (203.87,535.03) circle (  1.16);

\path[draw=drawColor,line width= 0.4pt,line join=round,line cap=round,fill=fillColor] (204.03,535.03) circle (  1.16);

\path[draw=drawColor,line width= 0.4pt,line join=round,line cap=round,fill=fillColor] (204.19,535.03) circle (  1.16);

\path[draw=drawColor,line width= 0.4pt,line join=round,line cap=round,fill=fillColor] (204.35,535.03) circle (  1.16);

\path[draw=drawColor,line width= 0.4pt,line join=round,line cap=round,fill=fillColor] (204.51,535.03) circle (  1.16);

\path[draw=drawColor,line width= 0.4pt,line join=round,line cap=round,fill=fillColor] (204.66,535.03) circle (  1.16);

\path[draw=drawColor,line width= 0.4pt,line join=round,line cap=round,fill=fillColor] (204.82,535.03) circle (  1.16);

\path[draw=drawColor,line width= 0.4pt,line join=round,line cap=round,fill=fillColor] (204.98,535.03) circle (  1.16);

\path[draw=drawColor,line width= 0.4pt,line join=round,line cap=round,fill=fillColor] (205.14,535.03) circle (  1.16);

\path[draw=drawColor,line width= 0.4pt,line join=round,line cap=round,fill=fillColor] (205.30,535.03) circle (  1.16);

\path[draw=drawColor,line width= 0.4pt,line join=round,line cap=round,fill=fillColor] (205.45,535.03) circle (  1.16);

\path[draw=drawColor,line width= 0.4pt,line join=round,line cap=round,fill=fillColor] (205.61,535.03) circle (  1.16);

\path[draw=drawColor,line width= 0.4pt,line join=round,line cap=round,fill=fillColor] (205.77,535.03) circle (  1.16);

\path[draw=drawColor,line width= 0.4pt,line join=round,line cap=round,fill=fillColor] (205.92,535.03) circle (  1.16);

\path[draw=drawColor,line width= 0.4pt,line join=round,line cap=round,fill=fillColor] (206.08,535.03) circle (  1.16);

\path[draw=drawColor,line width= 0.4pt,line join=round,line cap=round,fill=fillColor] (206.24,535.03) circle (  1.16);

\path[draw=drawColor,line width= 0.4pt,line join=round,line cap=round,fill=fillColor] (206.39,535.03) circle (  1.16);

\path[draw=drawColor,line width= 0.4pt,line join=round,line cap=round,fill=fillColor] (206.55,535.03) circle (  1.16);

\path[draw=drawColor,line width= 0.4pt,line join=round,line cap=round,fill=fillColor] (206.70,535.03) circle (  1.16);

\path[draw=drawColor,line width= 0.4pt,line join=round,line cap=round,fill=fillColor] (206.86,535.03) circle (  1.16);

\path[draw=drawColor,line width= 0.4pt,line join=round,line cap=round,fill=fillColor] (207.01,535.03) circle (  1.16);

\path[draw=drawColor,line width= 0.4pt,line join=round,line cap=round,fill=fillColor] (207.16,535.03) circle (  1.16);

\path[draw=drawColor,line width= 0.4pt,line join=round,line cap=round,fill=fillColor] (207.32,535.03) circle (  1.16);

\path[draw=drawColor,line width= 0.4pt,line join=round,line cap=round,fill=fillColor] (207.47,535.03) circle (  1.16);

\path[draw=drawColor,line width= 0.4pt,line join=round,line cap=round,fill=fillColor] (207.62,535.03) circle (  1.16);

\path[draw=drawColor,line width= 0.4pt,line join=round,line cap=round,fill=fillColor] (207.78,535.03) circle (  1.16);

\path[draw=drawColor,line width= 0.4pt,line join=round,line cap=round,fill=fillColor] (207.93,535.03) circle (  1.16);

\path[draw=drawColor,line width= 0.4pt,line join=round,line cap=round,fill=fillColor] (208.08,535.03) circle (  1.16);

\path[draw=drawColor,line width= 0.4pt,line join=round,line cap=round,fill=fillColor] (208.23,535.03) circle (  1.16);

\path[draw=drawColor,line width= 0.4pt,line join=round,line cap=round,fill=fillColor] (208.39,535.03) circle (  1.16);

\path[draw=drawColor,line width= 0.4pt,line join=round,line cap=round,fill=fillColor] (208.54,535.03) circle (  1.16);

\path[draw=drawColor,line width= 0.4pt,line join=round,line cap=round,fill=fillColor] (208.69,535.03) circle (  1.16);

\path[draw=drawColor,line width= 0.4pt,line join=round,line cap=round,fill=fillColor] (208.84,535.03) circle (  1.16);

\path[draw=drawColor,line width= 0.4pt,line join=round,line cap=round,fill=fillColor] (208.99,535.03) circle (  1.16);

\path[draw=drawColor,line width= 0.4pt,line join=round,line cap=round,fill=fillColor] (209.14,535.03) circle (  1.16);

\path[draw=drawColor,line width= 0.4pt,line join=round,line cap=round,fill=fillColor] (209.29,535.03) circle (  1.16);

\path[draw=drawColor,line width= 0.4pt,line join=round,line cap=round,fill=fillColor] (209.44,535.03) circle (  1.16);

\path[draw=drawColor,line width= 0.4pt,line join=round,line cap=round,fill=fillColor] (209.59,535.03) circle (  1.16);

\path[draw=drawColor,line width= 0.4pt,line join=round,line cap=round,fill=fillColor] (209.74,535.03) circle (  1.16);

\path[draw=drawColor,line width= 0.4pt,line join=round,line cap=round,fill=fillColor] (209.88,535.03) circle (  1.16);

\path[draw=drawColor,line width= 0.4pt,line join=round,line cap=round,fill=fillColor] (210.03,535.03) circle (  1.16);

\path[draw=drawColor,line width= 0.4pt,line join=round,line cap=round,fill=fillColor] (210.18,535.03) circle (  1.16);

\path[draw=drawColor,line width= 0.4pt,line join=round,line cap=round,fill=fillColor] (210.33,535.03) circle (  1.16);

\path[draw=drawColor,line width= 0.4pt,line join=round,line cap=round,fill=fillColor] (210.48,535.03) circle (  1.16);

\path[draw=drawColor,line width= 0.4pt,line join=round,line cap=round,fill=fillColor] (210.62,535.03) circle (  1.16);

\path[draw=drawColor,line width= 0.4pt,line join=round,line cap=round,fill=fillColor] (210.77,535.03) circle (  1.16);

\path[draw=drawColor,line width= 0.4pt,line join=round,line cap=round,fill=fillColor] (210.92,535.03) circle (  1.16);

\path[draw=drawColor,line width= 0.4pt,line join=round,line cap=round,fill=fillColor] (211.06,535.03) circle (  1.16);

\path[draw=drawColor,line width= 0.4pt,line join=round,line cap=round,fill=fillColor] (211.21,535.03) circle (  1.16);

\path[draw=drawColor,line width= 0.4pt,line join=round,line cap=round,fill=fillColor] (211.36,535.03) circle (  1.16);

\path[draw=drawColor,line width= 0.4pt,line join=round,line cap=round,fill=fillColor] (211.50,535.03) circle (  1.16);

\path[draw=drawColor,line width= 0.4pt,line join=round,line cap=round,fill=fillColor] (211.65,535.03) circle (  1.16);

\path[draw=drawColor,line width= 0.4pt,line join=round,line cap=round,fill=fillColor] (211.79,535.03) circle (  1.16);

\path[draw=drawColor,line width= 0.4pt,line join=round,line cap=round,fill=fillColor] (211.94,535.03) circle (  1.16);

\path[draw=drawColor,line width= 0.4pt,line join=round,line cap=round,fill=fillColor] (212.08,535.03) circle (  1.16);

\path[draw=drawColor,line width= 0.4pt,line join=round,line cap=round,fill=fillColor] (212.23,535.03) circle (  1.16);

\path[draw=drawColor,line width= 0.4pt,line join=round,line cap=round,fill=fillColor] (212.37,535.03) circle (  1.16);

\path[draw=drawColor,line width= 0.4pt,line join=round,line cap=round,fill=fillColor] (212.51,535.03) circle (  1.16);

\path[draw=drawColor,line width= 0.4pt,line join=round,line cap=round,fill=fillColor] (212.66,535.03) circle (  1.16);

\path[draw=drawColor,line width= 0.4pt,line join=round,line cap=round,fill=fillColor] (212.80,535.03) circle (  1.16);

\path[draw=drawColor,line width= 0.4pt,line join=round,line cap=round,fill=fillColor] (212.94,535.03) circle (  1.16);

\path[draw=drawColor,line width= 0.4pt,line join=round,line cap=round,fill=fillColor] (213.09,535.03) circle (  1.16);

\path[draw=drawColor,line width= 0.4pt,line join=round,line cap=round,fill=fillColor] (213.23,535.03) circle (  1.16);

\path[draw=drawColor,line width= 0.4pt,line join=round,line cap=round,fill=fillColor] (213.37,535.03) circle (  1.16);

\path[draw=drawColor,line width= 0.4pt,line join=round,line cap=round,fill=fillColor] (213.51,535.03) circle (  1.16);

\path[draw=drawColor,line width= 0.4pt,line join=round,line cap=round,fill=fillColor] (213.65,535.03) circle (  1.16);

\path[draw=drawColor,line width= 0.4pt,line join=round,line cap=round,fill=fillColor] (213.80,535.03) circle (  1.16);

\path[draw=drawColor,line width= 0.4pt,line join=round,line cap=round,fill=fillColor] (213.94,535.03) circle (  1.16);

\path[draw=drawColor,line width= 0.4pt,line join=round,line cap=round,fill=fillColor] (214.08,535.03) circle (  1.16);

\path[draw=drawColor,line width= 0.4pt,line join=round,line cap=round,fill=fillColor] (214.22,535.03) circle (  1.16);

\path[draw=drawColor,line width= 0.4pt,line join=round,line cap=round,fill=fillColor] (214.36,535.03) circle (  1.16);

\path[draw=drawColor,line width= 0.4pt,line join=round,line cap=round,fill=fillColor] (214.50,535.03) circle (  1.16);

\path[draw=drawColor,line width= 0.4pt,line join=round,line cap=round,fill=fillColor] (214.64,535.03) circle (  1.16);

\path[draw=drawColor,line width= 0.4pt,line join=round,line cap=round,fill=fillColor] (214.78,535.03) circle (  1.16);

\path[draw=drawColor,line width= 0.4pt,line join=round,line cap=round,fill=fillColor] (214.92,535.03) circle (  1.16);

\path[draw=drawColor,line width= 0.4pt,line join=round,line cap=round,fill=fillColor] (215.06,535.03) circle (  1.16);

\path[draw=drawColor,line width= 0.4pt,line join=round,line cap=round,fill=fillColor] (215.20,535.03) circle (  1.16);

\path[draw=drawColor,line width= 0.4pt,line join=round,line cap=round,fill=fillColor] (215.34,535.03) circle (  1.16);

\path[draw=drawColor,line width= 0.4pt,line join=round,line cap=round,fill=fillColor] (215.47,535.03) circle (  1.16);

\path[draw=drawColor,line width= 0.4pt,line join=round,line cap=round,fill=fillColor] (215.61,535.03) circle (  1.16);

\path[draw=drawColor,line width= 0.4pt,line join=round,line cap=round,fill=fillColor] (215.75,535.03) circle (  1.16);

\path[draw=drawColor,line width= 0.4pt,line join=round,line cap=round,fill=fillColor] (215.89,535.03) circle (  1.16);

\path[draw=drawColor,line width= 0.4pt,line join=round,line cap=round,fill=fillColor] (216.03,535.03) circle (  1.16);

\path[draw=drawColor,line width= 0.4pt,line join=round,line cap=round,fill=fillColor] (216.16,535.03) circle (  1.16);

\path[draw=drawColor,line width= 0.4pt,line join=round,line cap=round,fill=fillColor] (216.30,535.03) circle (  1.16);

\path[draw=drawColor,line width= 0.4pt,line join=round,line cap=round,fill=fillColor] (216.44,535.03) circle (  1.16);

\path[draw=drawColor,line width= 0.4pt,line join=round,line cap=round,fill=fillColor] (216.58,535.03) circle (  1.16);

\path[draw=drawColor,line width= 0.4pt,line join=round,line cap=round,fill=fillColor] (216.71,535.03) circle (  1.16);

\path[draw=drawColor,line width= 0.4pt,line join=round,line cap=round,fill=fillColor] (216.85,535.03) circle (  1.16);

\path[draw=drawColor,line width= 0.4pt,line join=round,line cap=round,fill=fillColor] (216.98,535.03) circle (  1.16);

\path[draw=drawColor,line width= 0.4pt,line join=round,line cap=round,fill=fillColor] (217.12,535.03) circle (  1.16);

\path[draw=drawColor,line width= 0.4pt,line join=round,line cap=round,fill=fillColor] (217.26,535.03) circle (  1.16);

\path[draw=drawColor,line width= 0.4pt,line join=round,line cap=round,fill=fillColor] (217.39,535.03) circle (  1.16);

\path[draw=drawColor,line width= 0.4pt,line join=round,line cap=round,fill=fillColor] (217.53,535.03) circle (  1.16);

\path[draw=drawColor,line width= 0.4pt,line join=round,line cap=round,fill=fillColor] (217.66,535.03) circle (  1.16);

\path[draw=drawColor,line width= 0.4pt,line join=round,line cap=round,fill=fillColor] (217.80,535.03) circle (  1.16);

\path[draw=drawColor,line width= 0.4pt,line join=round,line cap=round,fill=fillColor] (217.93,535.03) circle (  1.16);

\path[draw=drawColor,line width= 0.4pt,line join=round,line cap=round,fill=fillColor] (218.06,535.03) circle (  1.16);

\path[draw=drawColor,line width= 0.4pt,line join=round,line cap=round,fill=fillColor] (218.20,535.03) circle (  1.16);

\path[draw=drawColor,line width= 0.4pt,line join=round,line cap=round,fill=fillColor] (218.33,535.03) circle (  1.16);

\path[draw=drawColor,line width= 0.4pt,line join=round,line cap=round,fill=fillColor] (218.47,535.03) circle (  1.16);

\path[draw=drawColor,line width= 0.4pt,line join=round,line cap=round,fill=fillColor] (218.60,535.03) circle (  1.16);

\path[draw=drawColor,line width= 0.4pt,line join=round,line cap=round,fill=fillColor] (218.73,535.03) circle (  1.16);

\path[draw=drawColor,line width= 0.4pt,line join=round,line cap=round,fill=fillColor] (218.87,535.03) circle (  1.16);

\path[draw=drawColor,line width= 0.4pt,line join=round,line cap=round,fill=fillColor] (219.00,535.03) circle (  1.16);

\path[draw=drawColor,line width= 0.4pt,line join=round,line cap=round,fill=fillColor] (219.13,535.03) circle (  1.16);

\path[draw=drawColor,line width= 0.4pt,line join=round,line cap=round,fill=fillColor] (219.26,535.03) circle (  1.16);

\path[draw=drawColor,line width= 0.4pt,line join=round,line cap=round,fill=fillColor] (219.40,535.03) circle (  1.16);

\path[draw=drawColor,line width= 0.4pt,line join=round,line cap=round,fill=fillColor] (219.53,535.03) circle (  1.16);

\path[draw=drawColor,line width= 0.4pt,line join=round,line cap=round,fill=fillColor] (219.66,535.03) circle (  1.16);

\path[draw=drawColor,line width= 0.4pt,line join=round,line cap=round,fill=fillColor] (219.79,535.03) circle (  1.16);

\path[draw=drawColor,line width= 0.4pt,line join=round,line cap=round,fill=fillColor] (219.92,535.03) circle (  1.16);

\path[draw=drawColor,line width= 0.4pt,line join=round,line cap=round,fill=fillColor] (220.05,535.03) circle (  1.16);

\path[draw=drawColor,line width= 0.4pt,line join=round,line cap=round,fill=fillColor] (220.18,535.03) circle (  1.16);

\path[draw=drawColor,line width= 0.4pt,line join=round,line cap=round,fill=fillColor] (220.31,535.03) circle (  1.16);

\path[draw=drawColor,line width= 0.4pt,line join=round,line cap=round,fill=fillColor] (220.45,535.03) circle (  1.16);

\path[draw=drawColor,line width= 0.4pt,line join=round,line cap=round,fill=fillColor] (220.58,535.03) circle (  1.16);

\path[draw=drawColor,line width= 0.4pt,line join=round,line cap=round,fill=fillColor] (220.71,535.03) circle (  1.16);

\path[draw=drawColor,line width= 0.4pt,line join=round,line cap=round,fill=fillColor] (220.84,535.03) circle (  1.16);

\path[draw=drawColor,line width= 0.4pt,line join=round,line cap=round,fill=fillColor] (220.97,535.03) circle (  1.16);

\path[draw=drawColor,line width= 0.4pt,line join=round,line cap=round,fill=fillColor] (221.09,535.03) circle (  1.16);

\path[draw=drawColor,line width= 0.4pt,line join=round,line cap=round,fill=fillColor] (221.22,535.03) circle (  1.16);

\path[draw=drawColor,line width= 0.4pt,line join=round,line cap=round,fill=fillColor] (221.35,535.03) circle (  1.16);

\path[draw=drawColor,line width= 0.4pt,line join=round,line cap=round,fill=fillColor] (221.48,535.03) circle (  1.16);

\path[draw=drawColor,line width= 0.4pt,line join=round,line cap=round,fill=fillColor] (221.61,535.03) circle (  1.16);

\path[draw=drawColor,line width= 0.4pt,line join=round,line cap=round,fill=fillColor] (221.74,535.03) circle (  1.16);

\path[draw=drawColor,line width= 0.4pt,line join=round,line cap=round,fill=fillColor] (221.87,535.03) circle (  1.16);

\path[draw=drawColor,line width= 0.4pt,line join=round,line cap=round,fill=fillColor] (222.00,535.03) circle (  1.16);

\path[draw=drawColor,line width= 0.4pt,line join=round,line cap=round,fill=fillColor] (222.12,535.03) circle (  1.16);

\path[draw=drawColor,line width= 0.4pt,line join=round,line cap=round,fill=fillColor] (222.25,535.03) circle (  1.16);

\path[draw=drawColor,line width= 0.4pt,line join=round,line cap=round,fill=fillColor] (222.38,535.03) circle (  1.16);

\path[draw=drawColor,line width= 0.4pt,line join=round,line cap=round,fill=fillColor] (222.51,535.03) circle (  1.16);

\path[draw=drawColor,line width= 0.4pt,line join=round,line cap=round,fill=fillColor] (222.63,535.03) circle (  1.16);

\path[draw=drawColor,line width= 0.4pt,line join=round,line cap=round,fill=fillColor] (222.76,535.03) circle (  1.16);

\path[draw=drawColor,line width= 0.4pt,line join=round,line cap=round,fill=fillColor] (222.89,535.03) circle (  1.16);

\path[draw=drawColor,line width= 0.4pt,line join=round,line cap=round,fill=fillColor] (223.01,535.03) circle (  1.16);

\path[draw=drawColor,line width= 0.4pt,line join=round,line cap=round,fill=fillColor] (223.14,535.03) circle (  1.16);

\path[draw=drawColor,line width= 0.4pt,line join=round,line cap=round,fill=fillColor] (223.27,535.03) circle (  1.16);

\path[draw=drawColor,line width= 0.4pt,line join=round,line cap=round,fill=fillColor] (223.39,535.03) circle (  1.16);

\path[draw=drawColor,line width= 0.4pt,line join=round,line cap=round,fill=fillColor] (223.52,535.03) circle (  1.16);

\path[draw=drawColor,line width= 0.4pt,line join=round,line cap=round,fill=fillColor] (223.64,535.03) circle (  1.16);

\path[draw=drawColor,line width= 0.4pt,line join=round,line cap=round,fill=fillColor] (223.77,535.03) circle (  1.16);

\path[draw=drawColor,line width= 0.4pt,line join=round,line cap=round,fill=fillColor] (223.89,535.03) circle (  1.16);

\path[draw=drawColor,line width= 0.4pt,line join=round,line cap=round,fill=fillColor] (224.02,535.03) circle (  1.16);

\path[draw=drawColor,line width= 0.4pt,line join=round,line cap=round,fill=fillColor] (224.14,535.03) circle (  1.16);

\path[draw=drawColor,line width= 0.4pt,line join=round,line cap=round,fill=fillColor] (224.27,535.03) circle (  1.16);

\path[draw=drawColor,line width= 0.4pt,line join=round,line cap=round,fill=fillColor] (224.39,535.03) circle (  1.16);

\path[draw=drawColor,line width= 0.4pt,line join=round,line cap=round,fill=fillColor] (224.52,535.03) circle (  1.16);

\path[draw=drawColor,line width= 0.4pt,line join=round,line cap=round,fill=fillColor] (224.64,535.03) circle (  1.16);

\path[draw=drawColor,line width= 0.4pt,line join=round,line cap=round,fill=fillColor] (224.77,535.03) circle (  1.16);

\path[draw=drawColor,line width= 0.4pt,line join=round,line cap=round,fill=fillColor] (224.89,535.03) circle (  1.16);

\path[draw=drawColor,line width= 0.4pt,line join=round,line cap=round,fill=fillColor] (225.01,535.03) circle (  1.16);

\path[draw=drawColor,line width= 0.4pt,line join=round,line cap=round,fill=fillColor] (225.14,535.03) circle (  1.16);

\path[draw=drawColor,line width= 0.4pt,line join=round,line cap=round,fill=fillColor] (225.26,535.03) circle (  1.16);
\definecolor[named]{drawColor}{rgb}{0.60,0.31,0.64}
\definecolor[named]{fillColor}{rgb}{0.60,0.31,0.64}

\path[draw=drawColor,line width= 0.4pt,line join=round,line cap=round,fill=fillColor] ( 74.88,617.93) circle (  1.16);

\path[draw=drawColor,line width= 0.4pt,line join=round,line cap=round,fill=fillColor] ( 80.66,617.91) circle (  1.16);

\path[draw=drawColor,line width= 0.4pt,line join=round,line cap=round,fill=fillColor] ( 84.72,616.30) circle (  1.16);

\path[draw=drawColor,line width= 0.4pt,line join=round,line cap=round,fill=fillColor] ( 87.95,613.53) circle (  1.16);

\path[draw=drawColor,line width= 0.4pt,line join=round,line cap=round,fill=fillColor] ( 90.68,611.26) circle (  1.16);

\path[draw=drawColor,line width= 0.4pt,line join=round,line cap=round,fill=fillColor] ( 93.06,610.75) circle (  1.16);

\path[draw=drawColor,line width= 0.4pt,line join=round,line cap=round,fill=fillColor] ( 95.19,610.40) circle (  1.16);

\path[draw=drawColor,line width= 0.4pt,line join=round,line cap=round,fill=fillColor] ( 97.13,610.05) circle (  1.16);

\path[draw=drawColor,line width= 0.4pt,line join=round,line cap=round,fill=fillColor] ( 98.91,608.91) circle (  1.16);

\path[draw=drawColor,line width= 0.4pt,line join=round,line cap=round,fill=fillColor] (100.57,608.75) circle (  1.16);

\path[draw=drawColor,line width= 0.4pt,line join=round,line cap=round,fill=fillColor] (102.11,608.55) circle (  1.16);

\path[draw=drawColor,line width= 0.4pt,line join=round,line cap=round,fill=fillColor] (103.57,608.30) circle (  1.16);

\path[draw=drawColor,line width= 0.4pt,line join=round,line cap=round,fill=fillColor] (104.95,607.86) circle (  1.16);

\path[draw=drawColor,line width= 0.4pt,line join=round,line cap=round,fill=fillColor] (106.26,607.56) circle (  1.16);

\path[draw=drawColor,line width= 0.4pt,line join=round,line cap=round,fill=fillColor] (107.50,607.13) circle (  1.16);

\path[draw=drawColor,line width= 0.4pt,line join=round,line cap=round,fill=fillColor] (108.70,606.33) circle (  1.16);

\path[draw=drawColor,line width= 0.4pt,line join=round,line cap=round,fill=fillColor] (109.84,605.53) circle (  1.16);

\path[draw=drawColor,line width= 0.4pt,line join=round,line cap=round,fill=fillColor] (110.94,605.21) circle (  1.16);

\path[draw=drawColor,line width= 0.4pt,line join=round,line cap=round,fill=fillColor] (112.00,604.09) circle (  1.16);

\path[draw=drawColor,line width= 0.4pt,line join=round,line cap=round,fill=fillColor] (113.03,603.73) circle (  1.16);

\path[draw=drawColor,line width= 0.4pt,line join=round,line cap=round,fill=fillColor] (114.02,603.33) circle (  1.16);

\path[draw=drawColor,line width= 0.4pt,line join=round,line cap=round,fill=fillColor] (114.98,602.96) circle (  1.16);

\path[draw=drawColor,line width= 0.4pt,line join=round,line cap=round,fill=fillColor] (115.91,602.76) circle (  1.16);

\path[draw=drawColor,line width= 0.4pt,line join=round,line cap=round,fill=fillColor] (116.81,602.45) circle (  1.16);

\path[draw=drawColor,line width= 0.4pt,line join=round,line cap=round,fill=fillColor] (117.69,602.22) circle (  1.16);

\path[draw=drawColor,line width= 0.4pt,line join=round,line cap=round,fill=fillColor] (118.55,602.18) circle (  1.16);

\path[draw=drawColor,line width= 0.4pt,line join=round,line cap=round,fill=fillColor] (119.38,600.91) circle (  1.16);

\path[draw=drawColor,line width= 0.4pt,line join=round,line cap=round,fill=fillColor] (120.20,600.11) circle (  1.16);

\path[draw=drawColor,line width= 0.4pt,line join=round,line cap=round,fill=fillColor] (120.99,598.99) circle (  1.16);

\path[draw=drawColor,line width= 0.4pt,line join=round,line cap=round,fill=fillColor] (121.77,598.30) circle (  1.16);

\path[draw=drawColor,line width= 0.4pt,line join=round,line cap=round,fill=fillColor] (122.53,597.68) circle (  1.16);

\path[draw=drawColor,line width= 0.4pt,line join=round,line cap=round,fill=fillColor] (123.27,596.76) circle (  1.16);

\path[draw=drawColor,line width= 0.4pt,line join=round,line cap=round,fill=fillColor] (124.00,596.41) circle (  1.16);

\path[draw=drawColor,line width= 0.4pt,line join=round,line cap=round,fill=fillColor] (124.71,595.19) circle (  1.16);

\path[draw=drawColor,line width= 0.4pt,line join=round,line cap=round,fill=fillColor] (125.41,595.14) circle (  1.16);

\path[draw=drawColor,line width= 0.4pt,line join=round,line cap=round,fill=fillColor] (126.10,594.86) circle (  1.16);

\path[draw=drawColor,line width= 0.4pt,line join=round,line cap=round,fill=fillColor] (126.77,594.85) circle (  1.16);

\path[draw=drawColor,line width= 0.4pt,line join=round,line cap=round,fill=fillColor] (127.44,594.45) circle (  1.16);

\path[draw=drawColor,line width= 0.4pt,line join=round,line cap=round,fill=fillColor] (128.09,594.28) circle (  1.16);

\path[draw=drawColor,line width= 0.4pt,line join=round,line cap=round,fill=fillColor] (128.73,593.85) circle (  1.16);

\path[draw=drawColor,line width= 0.4pt,line join=round,line cap=round,fill=fillColor] (129.35,593.57) circle (  1.16);

\path[draw=drawColor,line width= 0.4pt,line join=round,line cap=round,fill=fillColor] (129.97,593.32) circle (  1.16);

\path[draw=drawColor,line width= 0.4pt,line join=round,line cap=round,fill=fillColor] (130.58,593.13) circle (  1.16);

\path[draw=drawColor,line width= 0.4pt,line join=round,line cap=round,fill=fillColor] (131.18,592.70) circle (  1.16);

\path[draw=drawColor,line width= 0.4pt,line join=round,line cap=round,fill=fillColor] (131.77,592.55) circle (  1.16);

\path[draw=drawColor,line width= 0.4pt,line join=round,line cap=round,fill=fillColor] (132.35,592.35) circle (  1.16);

\path[draw=drawColor,line width= 0.4pt,line join=round,line cap=round,fill=fillColor] (132.93,592.20) circle (  1.16);

\path[draw=drawColor,line width= 0.4pt,line join=round,line cap=round,fill=fillColor] (133.49,592.01) circle (  1.16);

\path[draw=drawColor,line width= 0.4pt,line join=round,line cap=round,fill=fillColor] (134.05,591.86) circle (  1.16);

\path[draw=drawColor,line width= 0.4pt,line join=round,line cap=round,fill=fillColor] (134.60,591.73) circle (  1.16);

\path[draw=drawColor,line width= 0.4pt,line join=round,line cap=round,fill=fillColor] (135.14,591.55) circle (  1.16);

\path[draw=drawColor,line width= 0.4pt,line join=round,line cap=round,fill=fillColor] (135.68,591.28) circle (  1.16);

\path[draw=drawColor,line width= 0.4pt,line join=round,line cap=round,fill=fillColor] (136.21,591.22) circle (  1.16);

\path[draw=drawColor,line width= 0.4pt,line join=round,line cap=round,fill=fillColor] (136.73,590.69) circle (  1.16);

\path[draw=drawColor,line width= 0.4pt,line join=round,line cap=round,fill=fillColor] (137.25,590.08) circle (  1.16);

\path[draw=drawColor,line width= 0.4pt,line join=round,line cap=round,fill=fillColor] (137.76,589.99) circle (  1.16);

\path[draw=drawColor,line width= 0.4pt,line join=round,line cap=round,fill=fillColor] (138.26,589.74) circle (  1.16);

\path[draw=drawColor,line width= 0.4pt,line join=round,line cap=round,fill=fillColor] (138.76,589.58) circle (  1.16);

\path[draw=drawColor,line width= 0.4pt,line join=round,line cap=round,fill=fillColor] (139.25,589.52) circle (  1.16);

\path[draw=drawColor,line width= 0.4pt,line join=round,line cap=round,fill=fillColor] (139.74,589.30) circle (  1.16);

\path[draw=drawColor,line width= 0.4pt,line join=round,line cap=round,fill=fillColor] (140.22,588.54) circle (  1.16);

\path[draw=drawColor,line width= 0.4pt,line join=round,line cap=round,fill=fillColor] (140.70,588.38) circle (  1.16);

\path[draw=drawColor,line width= 0.4pt,line join=round,line cap=round,fill=fillColor] (141.17,588.35) circle (  1.16);

\path[draw=drawColor,line width= 0.4pt,line join=round,line cap=round,fill=fillColor] (141.63,588.32) circle (  1.16);

\path[draw=drawColor,line width= 0.4pt,line join=round,line cap=round,fill=fillColor] (142.09,588.16) circle (  1.16);

\path[draw=drawColor,line width= 0.4pt,line join=round,line cap=round,fill=fillColor] (142.55,587.85) circle (  1.16);

\path[draw=drawColor,line width= 0.4pt,line join=round,line cap=round,fill=fillColor] (143.00,587.69) circle (  1.16);

\path[draw=drawColor,line width= 0.4pt,line join=round,line cap=round,fill=fillColor] (143.45,587.22) circle (  1.16);

\path[draw=drawColor,line width= 0.4pt,line join=round,line cap=round,fill=fillColor] (143.89,587.13) circle (  1.16);

\path[draw=drawColor,line width= 0.4pt,line join=round,line cap=round,fill=fillColor] (144.33,587.06) circle (  1.16);

\path[draw=drawColor,line width= 0.4pt,line join=round,line cap=round,fill=fillColor] (144.77,586.87) circle (  1.16);

\path[draw=drawColor,line width= 0.4pt,line join=round,line cap=round,fill=fillColor] (145.20,586.63) circle (  1.16);

\path[draw=drawColor,line width= 0.4pt,line join=round,line cap=round,fill=fillColor] (145.62,586.25) circle (  1.16);

\path[draw=drawColor,line width= 0.4pt,line join=round,line cap=round,fill=fillColor] (146.05,584.90) circle (  1.16);

\path[draw=drawColor,line width= 0.4pt,line join=round,line cap=round,fill=fillColor] (146.46,584.83) circle (  1.16);

\path[draw=drawColor,line width= 0.4pt,line join=round,line cap=round,fill=fillColor] (146.88,584.76) circle (  1.16);

\path[draw=drawColor,line width= 0.4pt,line join=round,line cap=round,fill=fillColor] (147.29,584.70) circle (  1.16);

\path[draw=drawColor,line width= 0.4pt,line join=round,line cap=round,fill=fillColor] (147.70,584.43) circle (  1.16);

\path[draw=drawColor,line width= 0.4pt,line join=round,line cap=round,fill=fillColor] (148.10,584.24) circle (  1.16);

\path[draw=drawColor,line width= 0.4pt,line join=round,line cap=round,fill=fillColor] (148.50,584.19) circle (  1.16);

\path[draw=drawColor,line width= 0.4pt,line join=round,line cap=round,fill=fillColor] (148.90,584.00) circle (  1.16);

\path[draw=drawColor,line width= 0.4pt,line join=round,line cap=round,fill=fillColor] (149.30,583.65) circle (  1.16);

\path[draw=drawColor,line width= 0.4pt,line join=round,line cap=round,fill=fillColor] (149.69,583.63) circle (  1.16);

\path[draw=drawColor,line width= 0.4pt,line join=round,line cap=round,fill=fillColor] (150.08,583.46) circle (  1.16);

\path[draw=drawColor,line width= 0.4pt,line join=round,line cap=round,fill=fillColor] (150.46,583.45) circle (  1.16);

\path[draw=drawColor,line width= 0.4pt,line join=round,line cap=round,fill=fillColor] (150.84,583.45) circle (  1.16);

\path[draw=drawColor,line width= 0.4pt,line join=round,line cap=round,fill=fillColor] (151.22,583.42) circle (  1.16);

\path[draw=drawColor,line width= 0.4pt,line join=round,line cap=round,fill=fillColor] (151.60,583.37) circle (  1.16);

\path[draw=drawColor,line width= 0.4pt,line join=round,line cap=round,fill=fillColor] (151.97,583.23) circle (  1.16);

\path[draw=drawColor,line width= 0.4pt,line join=round,line cap=round,fill=fillColor] (152.34,583.11) circle (  1.16);

\path[draw=drawColor,line width= 0.4pt,line join=round,line cap=round,fill=fillColor] (152.71,583.02) circle (  1.16);

\path[draw=drawColor,line width= 0.4pt,line join=round,line cap=round,fill=fillColor] (153.08,582.48) circle (  1.16);

\path[draw=drawColor,line width= 0.4pt,line join=round,line cap=round,fill=fillColor] (153.44,582.26) circle (  1.16);

\path[draw=drawColor,line width= 0.4pt,line join=round,line cap=round,fill=fillColor] (153.80,582.22) circle (  1.16);

\path[draw=drawColor,line width= 0.4pt,line join=round,line cap=round,fill=fillColor] (154.16,582.12) circle (  1.16);

\path[draw=drawColor,line width= 0.4pt,line join=round,line cap=round,fill=fillColor] (154.51,581.75) circle (  1.16);

\path[draw=drawColor,line width= 0.4pt,line join=round,line cap=round,fill=fillColor] (154.86,581.74) circle (  1.16);

\path[draw=drawColor,line width= 0.4pt,line join=round,line cap=round,fill=fillColor] (155.22,581.65) circle (  1.16);

\path[draw=drawColor,line width= 0.4pt,line join=round,line cap=round,fill=fillColor] (155.56,581.62) circle (  1.16);

\path[draw=drawColor,line width= 0.4pt,line join=round,line cap=round,fill=fillColor] (155.91,581.58) circle (  1.16);

\path[draw=drawColor,line width= 0.4pt,line join=round,line cap=round,fill=fillColor] (156.25,581.09) circle (  1.16);

\path[draw=drawColor,line width= 0.4pt,line join=round,line cap=round,fill=fillColor] (156.59,580.97) circle (  1.16);

\path[draw=drawColor,line width= 0.4pt,line join=round,line cap=round,fill=fillColor] (156.93,580.90) circle (  1.16);

\path[draw=drawColor,line width= 0.4pt,line join=round,line cap=round,fill=fillColor] (157.27,580.76) circle (  1.16);

\path[draw=drawColor,line width= 0.4pt,line join=round,line cap=round,fill=fillColor] (157.60,580.71) circle (  1.16);

\path[draw=drawColor,line width= 0.4pt,line join=round,line cap=round,fill=fillColor] (157.93,580.60) circle (  1.16);

\path[draw=drawColor,line width= 0.4pt,line join=round,line cap=round,fill=fillColor] (158.26,580.58) circle (  1.16);

\path[draw=drawColor,line width= 0.4pt,line join=round,line cap=round,fill=fillColor] (158.59,580.50) circle (  1.16);

\path[draw=drawColor,line width= 0.4pt,line join=round,line cap=round,fill=fillColor] (158.92,580.18) circle (  1.16);

\path[draw=drawColor,line width= 0.4pt,line join=round,line cap=round,fill=fillColor] (159.24,580.10) circle (  1.16);

\path[draw=drawColor,line width= 0.4pt,line join=round,line cap=round,fill=fillColor] (159.56,579.97) circle (  1.16);

\path[draw=drawColor,line width= 0.4pt,line join=round,line cap=round,fill=fillColor] (159.88,579.75) circle (  1.16);

\path[draw=drawColor,line width= 0.4pt,line join=round,line cap=round,fill=fillColor] (160.20,579.72) circle (  1.16);

\path[draw=drawColor,line width= 0.4pt,line join=round,line cap=round,fill=fillColor] (160.52,579.55) circle (  1.16);

\path[draw=drawColor,line width= 0.4pt,line join=round,line cap=round,fill=fillColor] (160.83,579.33) circle (  1.16);

\path[draw=drawColor,line width= 0.4pt,line join=round,line cap=round,fill=fillColor] (161.15,579.25) circle (  1.16);

\path[draw=drawColor,line width= 0.4pt,line join=round,line cap=round,fill=fillColor] (161.46,579.19) circle (  1.16);

\path[draw=drawColor,line width= 0.4pt,line join=round,line cap=round,fill=fillColor] (161.77,578.88) circle (  1.16);

\path[draw=drawColor,line width= 0.4pt,line join=round,line cap=round,fill=fillColor] (162.07,578.82) circle (  1.16);

\path[draw=drawColor,line width= 0.4pt,line join=round,line cap=round,fill=fillColor] (162.38,578.41) circle (  1.16);

\path[draw=drawColor,line width= 0.4pt,line join=round,line cap=round,fill=fillColor] (162.68,578.35) circle (  1.16);

\path[draw=drawColor,line width= 0.4pt,line join=round,line cap=round,fill=fillColor] (162.99,578.04) circle (  1.16);

\path[draw=drawColor,line width= 0.4pt,line join=round,line cap=round,fill=fillColor] (163.29,577.98) circle (  1.16);

\path[draw=drawColor,line width= 0.4pt,line join=round,line cap=round,fill=fillColor] (163.59,577.92) circle (  1.16);

\path[draw=drawColor,line width= 0.4pt,line join=round,line cap=round,fill=fillColor] (163.88,577.89) circle (  1.16);

\path[draw=drawColor,line width= 0.4pt,line join=round,line cap=round,fill=fillColor] (164.18,577.74) circle (  1.16);

\path[draw=drawColor,line width= 0.4pt,line join=round,line cap=round,fill=fillColor] (164.47,577.72) circle (  1.16);

\path[draw=drawColor,line width= 0.4pt,line join=round,line cap=round,fill=fillColor] (164.77,577.65) circle (  1.16);

\path[draw=drawColor,line width= 0.4pt,line join=round,line cap=round,fill=fillColor] (165.06,577.64) circle (  1.16);

\path[draw=drawColor,line width= 0.4pt,line join=round,line cap=round,fill=fillColor] (165.35,577.40) circle (  1.16);

\path[draw=drawColor,line width= 0.4pt,line join=round,line cap=round,fill=fillColor] (165.64,577.37) circle (  1.16);

\path[draw=drawColor,line width= 0.4pt,line join=round,line cap=round,fill=fillColor] (165.92,577.36) circle (  1.16);

\path[draw=drawColor,line width= 0.4pt,line join=round,line cap=round,fill=fillColor] (166.21,577.27) circle (  1.16);

\path[draw=drawColor,line width= 0.4pt,line join=round,line cap=round,fill=fillColor] (166.49,576.93) circle (  1.16);

\path[draw=drawColor,line width= 0.4pt,line join=round,line cap=round,fill=fillColor] (166.77,576.79) circle (  1.16);

\path[draw=drawColor,line width= 0.4pt,line join=round,line cap=round,fill=fillColor] (167.06,576.22) circle (  1.16);

\path[draw=drawColor,line width= 0.4pt,line join=round,line cap=round,fill=fillColor] (167.34,576.18) circle (  1.16);

\path[draw=drawColor,line width= 0.4pt,line join=round,line cap=round,fill=fillColor] (167.61,576.08) circle (  1.16);

\path[draw=drawColor,line width= 0.4pt,line join=round,line cap=round,fill=fillColor] (167.89,576.06) circle (  1.16);

\path[draw=drawColor,line width= 0.4pt,line join=round,line cap=round,fill=fillColor] (168.17,575.91) circle (  1.16);

\path[draw=drawColor,line width= 0.4pt,line join=round,line cap=round,fill=fillColor] (168.44,575.86) circle (  1.16);

\path[draw=drawColor,line width= 0.4pt,line join=round,line cap=round,fill=fillColor] (168.71,575.85) circle (  1.16);

\path[draw=drawColor,line width= 0.4pt,line join=round,line cap=round,fill=fillColor] (168.99,575.79) circle (  1.16);

\path[draw=drawColor,line width= 0.4pt,line join=round,line cap=round,fill=fillColor] (169.26,575.71) circle (  1.16);

\path[draw=drawColor,line width= 0.4pt,line join=round,line cap=round,fill=fillColor] (169.53,575.64) circle (  1.16);

\path[draw=drawColor,line width= 0.4pt,line join=round,line cap=round,fill=fillColor] (169.79,575.57) circle (  1.16);

\path[draw=drawColor,line width= 0.4pt,line join=round,line cap=round,fill=fillColor] (170.06,575.34) circle (  1.16);

\path[draw=drawColor,line width= 0.4pt,line join=round,line cap=round,fill=fillColor] (170.33,575.26) circle (  1.16);

\path[draw=drawColor,line width= 0.4pt,line join=round,line cap=round,fill=fillColor] (170.59,575.21) circle (  1.16);

\path[draw=drawColor,line width= 0.4pt,line join=round,line cap=round,fill=fillColor] (170.85,575.16) circle (  1.16);

\path[draw=drawColor,line width= 0.4pt,line join=round,line cap=round,fill=fillColor] (171.12,574.96) circle (  1.16);

\path[draw=drawColor,line width= 0.4pt,line join=round,line cap=round,fill=fillColor] (171.38,574.71) circle (  1.16);

\path[draw=drawColor,line width= 0.4pt,line join=round,line cap=round,fill=fillColor] (171.64,574.70) circle (  1.16);

\path[draw=drawColor,line width= 0.4pt,line join=round,line cap=round,fill=fillColor] (171.90,574.61) circle (  1.16);

\path[draw=drawColor,line width= 0.4pt,line join=round,line cap=round,fill=fillColor] (172.15,574.50) circle (  1.16);

\path[draw=drawColor,line width= 0.4pt,line join=round,line cap=round,fill=fillColor] (172.41,574.48) circle (  1.16);

\path[draw=drawColor,line width= 0.4pt,line join=round,line cap=round,fill=fillColor] (172.67,574.37) circle (  1.16);

\path[draw=drawColor,line width= 0.4pt,line join=round,line cap=round,fill=fillColor] (172.92,574.26) circle (  1.16);

\path[draw=drawColor,line width= 0.4pt,line join=round,line cap=round,fill=fillColor] (173.17,574.08) circle (  1.16);

\path[draw=drawColor,line width= 0.4pt,line join=round,line cap=round,fill=fillColor] (173.43,573.88) circle (  1.16);

\path[draw=drawColor,line width= 0.4pt,line join=round,line cap=round,fill=fillColor] (173.68,573.68) circle (  1.16);

\path[draw=drawColor,line width= 0.4pt,line join=round,line cap=round,fill=fillColor] (173.93,573.50) circle (  1.16);

\path[draw=drawColor,line width= 0.4pt,line join=round,line cap=round,fill=fillColor] (174.18,573.48) circle (  1.16);

\path[draw=drawColor,line width= 0.4pt,line join=round,line cap=round,fill=fillColor] (174.42,573.48) circle (  1.16);

\path[draw=drawColor,line width= 0.4pt,line join=round,line cap=round,fill=fillColor] (174.67,573.46) circle (  1.16);

\path[draw=drawColor,line width= 0.4pt,line join=round,line cap=round,fill=fillColor] (174.92,573.39) circle (  1.16);

\path[draw=drawColor,line width= 0.4pt,line join=round,line cap=round,fill=fillColor] (175.16,573.38) circle (  1.16);

\path[draw=drawColor,line width= 0.4pt,line join=round,line cap=round,fill=fillColor] (175.41,573.33) circle (  1.16);

\path[draw=drawColor,line width= 0.4pt,line join=round,line cap=round,fill=fillColor] (175.65,573.29) circle (  1.16);

\path[draw=drawColor,line width= 0.4pt,line join=round,line cap=round,fill=fillColor] (175.89,573.03) circle (  1.16);

\path[draw=drawColor,line width= 0.4pt,line join=round,line cap=round,fill=fillColor] (176.13,572.98) circle (  1.16);

\path[draw=drawColor,line width= 0.4pt,line join=round,line cap=round,fill=fillColor] (176.37,572.81) circle (  1.16);

\path[draw=drawColor,line width= 0.4pt,line join=round,line cap=round,fill=fillColor] (176.61,572.72) circle (  1.16);

\path[draw=drawColor,line width= 0.4pt,line join=round,line cap=round,fill=fillColor] (176.85,572.72) circle (  1.16);

\path[draw=drawColor,line width= 0.4pt,line join=round,line cap=round,fill=fillColor] (177.09,572.66) circle (  1.16);

\path[draw=drawColor,line width= 0.4pt,line join=round,line cap=round,fill=fillColor] (177.32,572.56) circle (  1.16);

\path[draw=drawColor,line width= 0.4pt,line join=round,line cap=round,fill=fillColor] (177.56,572.52) circle (  1.16);

\path[draw=drawColor,line width= 0.4pt,line join=round,line cap=round,fill=fillColor] (177.80,572.51) circle (  1.16);

\path[draw=drawColor,line width= 0.4pt,line join=round,line cap=round,fill=fillColor] (178.03,572.51) circle (  1.16);

\path[draw=drawColor,line width= 0.4pt,line join=round,line cap=round,fill=fillColor] (178.26,572.22) circle (  1.16);

\path[draw=drawColor,line width= 0.4pt,line join=round,line cap=round,fill=fillColor] (178.49,572.19) circle (  1.16);

\path[draw=drawColor,line width= 0.4pt,line join=round,line cap=round,fill=fillColor] (178.73,572.04) circle (  1.16);

\path[draw=drawColor,line width= 0.4pt,line join=round,line cap=round,fill=fillColor] (178.96,572.03) circle (  1.16);

\path[draw=drawColor,line width= 0.4pt,line join=round,line cap=round,fill=fillColor] (179.19,571.91) circle (  1.16);

\path[draw=drawColor,line width= 0.4pt,line join=round,line cap=round,fill=fillColor] (179.41,571.75) circle (  1.16);

\path[draw=drawColor,line width= 0.4pt,line join=round,line cap=round,fill=fillColor] (179.64,571.73) circle (  1.16);

\path[draw=drawColor,line width= 0.4pt,line join=round,line cap=round,fill=fillColor] (179.87,571.55) circle (  1.16);

\path[draw=drawColor,line width= 0.4pt,line join=round,line cap=round,fill=fillColor] (180.10,571.29) circle (  1.16);

\path[draw=drawColor,line width= 0.4pt,line join=round,line cap=round,fill=fillColor] (180.32,571.16) circle (  1.16);

\path[draw=drawColor,line width= 0.4pt,line join=round,line cap=round,fill=fillColor] (180.55,571.05) circle (  1.16);

\path[draw=drawColor,line width= 0.4pt,line join=round,line cap=round,fill=fillColor] (180.77,571.05) circle (  1.16);

\path[draw=drawColor,line width= 0.4pt,line join=round,line cap=round,fill=fillColor] (180.99,570.98) circle (  1.16);

\path[draw=drawColor,line width= 0.4pt,line join=round,line cap=round,fill=fillColor] (181.22,570.89) circle (  1.16);

\path[draw=drawColor,line width= 0.4pt,line join=round,line cap=round,fill=fillColor] (181.44,570.85) circle (  1.16);

\path[draw=drawColor,line width= 0.4pt,line join=round,line cap=round,fill=fillColor] (181.66,570.82) circle (  1.16);

\path[draw=drawColor,line width= 0.4pt,line join=round,line cap=round,fill=fillColor] (181.88,570.79) circle (  1.16);

\path[draw=drawColor,line width= 0.4pt,line join=round,line cap=round,fill=fillColor] (182.10,570.62) circle (  1.16);

\path[draw=drawColor,line width= 0.4pt,line join=round,line cap=round,fill=fillColor] (182.32,570.58) circle (  1.16);

\path[draw=drawColor,line width= 0.4pt,line join=round,line cap=round,fill=fillColor] (182.54,570.40) circle (  1.16);

\path[draw=drawColor,line width= 0.4pt,line join=round,line cap=round,fill=fillColor] (182.75,570.24) circle (  1.16);

\path[draw=drawColor,line width= 0.4pt,line join=round,line cap=round,fill=fillColor] (182.97,569.96) circle (  1.16);

\path[draw=drawColor,line width= 0.4pt,line join=round,line cap=round,fill=fillColor] (183.19,569.93) circle (  1.16);

\path[draw=drawColor,line width= 0.4pt,line join=round,line cap=round,fill=fillColor] (183.40,569.92) circle (  1.16);

\path[draw=drawColor,line width= 0.4pt,line join=round,line cap=round,fill=fillColor] (183.61,569.83) circle (  1.16);

\path[draw=drawColor,line width= 0.4pt,line join=round,line cap=round,fill=fillColor] (183.83,569.81) circle (  1.16);

\path[draw=drawColor,line width= 0.4pt,line join=round,line cap=round,fill=fillColor] (184.04,569.76) circle (  1.16);

\path[draw=drawColor,line width= 0.4pt,line join=round,line cap=round,fill=fillColor] (184.25,569.71) circle (  1.16);

\path[draw=drawColor,line width= 0.4pt,line join=round,line cap=round,fill=fillColor] (184.47,569.59) circle (  1.16);

\path[draw=drawColor,line width= 0.4pt,line join=round,line cap=round,fill=fillColor] (184.68,569.45) circle (  1.16);

\path[draw=drawColor,line width= 0.4pt,line join=round,line cap=round,fill=fillColor] (184.89,569.29) circle (  1.16);

\path[draw=drawColor,line width= 0.4pt,line join=round,line cap=round,fill=fillColor] (185.10,569.28) circle (  1.16);

\path[draw=drawColor,line width= 0.4pt,line join=round,line cap=round,fill=fillColor] (185.31,569.18) circle (  1.16);

\path[draw=drawColor,line width= 0.4pt,line join=round,line cap=round,fill=fillColor] (185.51,569.04) circle (  1.16);

\path[draw=drawColor,line width= 0.4pt,line join=round,line cap=round,fill=fillColor] (185.72,569.02) circle (  1.16);

\path[draw=drawColor,line width= 0.4pt,line join=round,line cap=round,fill=fillColor] (185.93,569.00) circle (  1.16);

\path[draw=drawColor,line width= 0.4pt,line join=round,line cap=round,fill=fillColor] (186.13,568.94) circle (  1.16);

\path[draw=drawColor,line width= 0.4pt,line join=round,line cap=round,fill=fillColor] (186.34,568.88) circle (  1.16);

\path[draw=drawColor,line width= 0.4pt,line join=round,line cap=round,fill=fillColor] (186.55,568.61) circle (  1.16);

\path[draw=drawColor,line width= 0.4pt,line join=round,line cap=round,fill=fillColor] (186.75,568.58) circle (  1.16);

\path[draw=drawColor,line width= 0.4pt,line join=round,line cap=round,fill=fillColor] (186.95,568.46) circle (  1.16);

\path[draw=drawColor,line width= 0.4pt,line join=round,line cap=round,fill=fillColor] (187.16,568.46) circle (  1.16);

\path[draw=drawColor,line width= 0.4pt,line join=round,line cap=round,fill=fillColor] (187.36,568.43) circle (  1.16);

\path[draw=drawColor,line width= 0.4pt,line join=round,line cap=round,fill=fillColor] (187.56,568.40) circle (  1.16);

\path[draw=drawColor,line width= 0.4pt,line join=round,line cap=round,fill=fillColor] (187.76,568.05) circle (  1.16);

\path[draw=drawColor,line width= 0.4pt,line join=round,line cap=round,fill=fillColor] (187.96,567.95) circle (  1.16);

\path[draw=drawColor,line width= 0.4pt,line join=round,line cap=round,fill=fillColor] (188.16,567.92) circle (  1.16);

\path[draw=drawColor,line width= 0.4pt,line join=round,line cap=round,fill=fillColor] (188.36,567.89) circle (  1.16);

\path[draw=drawColor,line width= 0.4pt,line join=round,line cap=round,fill=fillColor] (188.56,567.82) circle (  1.16);

\path[draw=drawColor,line width= 0.4pt,line join=round,line cap=round,fill=fillColor] (188.76,567.73) circle (  1.16);

\path[draw=drawColor,line width= 0.4pt,line join=round,line cap=round,fill=fillColor] (188.96,567.69) circle (  1.16);

\path[draw=drawColor,line width= 0.4pt,line join=round,line cap=round,fill=fillColor] (189.16,567.56) circle (  1.16);

\path[draw=drawColor,line width= 0.4pt,line join=round,line cap=round,fill=fillColor] (189.35,567.33) circle (  1.16);

\path[draw=drawColor,line width= 0.4pt,line join=round,line cap=round,fill=fillColor] (189.55,567.33) circle (  1.16);

\path[draw=drawColor,line width= 0.4pt,line join=round,line cap=round,fill=fillColor] (189.74,567.32) circle (  1.16);

\path[draw=drawColor,line width= 0.4pt,line join=round,line cap=round,fill=fillColor] (189.94,567.06) circle (  1.16);

\path[draw=drawColor,line width= 0.4pt,line join=round,line cap=round,fill=fillColor] (190.13,567.03) circle (  1.16);

\path[draw=drawColor,line width= 0.4pt,line join=round,line cap=round,fill=fillColor] (190.33,566.74) circle (  1.16);

\path[draw=drawColor,line width= 0.4pt,line join=round,line cap=round,fill=fillColor] (190.52,566.68) circle (  1.16);

\path[draw=drawColor,line width= 0.4pt,line join=round,line cap=round,fill=fillColor] (190.71,566.57) circle (  1.16);

\path[draw=drawColor,line width= 0.4pt,line join=round,line cap=round,fill=fillColor] (190.91,566.56) circle (  1.16);

\path[draw=drawColor,line width= 0.4pt,line join=round,line cap=round,fill=fillColor] (191.10,566.35) circle (  1.16);

\path[draw=drawColor,line width= 0.4pt,line join=round,line cap=round,fill=fillColor] (191.29,566.32) circle (  1.16);

\path[draw=drawColor,line width= 0.4pt,line join=round,line cap=round,fill=fillColor] (191.48,566.29) circle (  1.16);

\path[draw=drawColor,line width= 0.4pt,line join=round,line cap=round,fill=fillColor] (191.67,566.20) circle (  1.16);

\path[draw=drawColor,line width= 0.4pt,line join=round,line cap=round,fill=fillColor] (191.86,565.84) circle (  1.16);

\path[draw=drawColor,line width= 0.4pt,line join=round,line cap=round,fill=fillColor] (192.05,565.76) circle (  1.16);

\path[draw=drawColor,line width= 0.4pt,line join=round,line cap=round,fill=fillColor] (192.24,565.63) circle (  1.16);

\path[draw=drawColor,line width= 0.4pt,line join=round,line cap=round,fill=fillColor] (192.43,565.59) circle (  1.16);

\path[draw=drawColor,line width= 0.4pt,line join=round,line cap=round,fill=fillColor] (192.61,565.58) circle (  1.16);

\path[draw=drawColor,line width= 0.4pt,line join=round,line cap=round,fill=fillColor] (192.80,565.54) circle (  1.16);

\path[draw=drawColor,line width= 0.4pt,line join=round,line cap=round,fill=fillColor] (192.99,565.52) circle (  1.16);

\path[draw=drawColor,line width= 0.4pt,line join=round,line cap=round,fill=fillColor] (193.17,565.46) circle (  1.16);

\path[draw=drawColor,line width= 0.4pt,line join=round,line cap=round,fill=fillColor] (193.36,565.41) circle (  1.16);

\path[draw=drawColor,line width= 0.4pt,line join=round,line cap=round,fill=fillColor] (193.54,565.31) circle (  1.16);

\path[draw=drawColor,line width= 0.4pt,line join=round,line cap=round,fill=fillColor] (193.73,565.26) circle (  1.16);

\path[draw=drawColor,line width= 0.4pt,line join=round,line cap=round,fill=fillColor] (193.91,565.17) circle (  1.16);

\path[draw=drawColor,line width= 0.4pt,line join=round,line cap=round,fill=fillColor] (194.10,565.17) circle (  1.16);

\path[draw=drawColor,line width= 0.4pt,line join=round,line cap=round,fill=fillColor] (194.28,565.14) circle (  1.16);

\path[draw=drawColor,line width= 0.4pt,line join=round,line cap=round,fill=fillColor] (194.46,565.11) circle (  1.16);

\path[draw=drawColor,line width= 0.4pt,line join=round,line cap=round,fill=fillColor] (194.65,564.99) circle (  1.16);

\path[draw=drawColor,line width= 0.4pt,line join=round,line cap=round,fill=fillColor] (194.83,564.98) circle (  1.16);

\path[draw=drawColor,line width= 0.4pt,line join=round,line cap=round,fill=fillColor] (195.01,564.98) circle (  1.16);

\path[draw=drawColor,line width= 0.4pt,line join=round,line cap=round,fill=fillColor] (195.19,564.97) circle (  1.16);

\path[draw=drawColor,line width= 0.4pt,line join=round,line cap=round,fill=fillColor] (195.37,564.95) circle (  1.16);

\path[draw=drawColor,line width= 0.4pt,line join=round,line cap=round,fill=fillColor] (195.55,564.94) circle (  1.16);

\path[draw=drawColor,line width= 0.4pt,line join=round,line cap=round,fill=fillColor] (195.73,564.73) circle (  1.16);

\path[draw=drawColor,line width= 0.4pt,line join=round,line cap=round,fill=fillColor] (195.91,564.71) circle (  1.16);

\path[draw=drawColor,line width= 0.4pt,line join=round,line cap=round,fill=fillColor] (196.09,564.56) circle (  1.16);

\path[draw=drawColor,line width= 0.4pt,line join=round,line cap=round,fill=fillColor] (196.27,564.53) circle (  1.16);

\path[draw=drawColor,line width= 0.4pt,line join=round,line cap=round,fill=fillColor] (196.44,564.49) circle (  1.16);

\path[draw=drawColor,line width= 0.4pt,line join=round,line cap=round,fill=fillColor] (196.62,564.48) circle (  1.16);

\path[draw=drawColor,line width= 0.4pt,line join=round,line cap=round,fill=fillColor] (196.80,564.28) circle (  1.16);

\path[draw=drawColor,line width= 0.4pt,line join=round,line cap=round,fill=fillColor] (196.97,564.18) circle (  1.16);

\path[draw=drawColor,line width= 0.4pt,line join=round,line cap=round,fill=fillColor] (197.15,564.05) circle (  1.16);

\path[draw=drawColor,line width= 0.4pt,line join=round,line cap=round,fill=fillColor] (197.33,564.02) circle (  1.16);

\path[draw=drawColor,line width= 0.4pt,line join=round,line cap=round,fill=fillColor] (197.50,563.87) circle (  1.16);

\path[draw=drawColor,line width= 0.4pt,line join=round,line cap=round,fill=fillColor] (197.68,563.82) circle (  1.16);

\path[draw=drawColor,line width= 0.4pt,line join=round,line cap=round,fill=fillColor] (197.85,563.76) circle (  1.16);

\path[draw=drawColor,line width= 0.4pt,line join=round,line cap=round,fill=fillColor] (198.02,563.63) circle (  1.16);

\path[draw=drawColor,line width= 0.4pt,line join=round,line cap=round,fill=fillColor] (198.20,563.33) circle (  1.16);

\path[draw=drawColor,line width= 0.4pt,line join=round,line cap=round,fill=fillColor] (198.37,563.20) circle (  1.16);

\path[draw=drawColor,line width= 0.4pt,line join=round,line cap=round,fill=fillColor] (198.54,563.17) circle (  1.16);

\path[draw=drawColor,line width= 0.4pt,line join=round,line cap=round,fill=fillColor] (198.72,563.14) circle (  1.16);

\path[draw=drawColor,line width= 0.4pt,line join=round,line cap=round,fill=fillColor] (198.89,563.07) circle (  1.16);

\path[draw=drawColor,line width= 0.4pt,line join=round,line cap=round,fill=fillColor] (199.06,562.95) circle (  1.16);

\path[draw=drawColor,line width= 0.4pt,line join=round,line cap=round,fill=fillColor] (199.23,562.68) circle (  1.16);

\path[draw=drawColor,line width= 0.4pt,line join=round,line cap=round,fill=fillColor] (199.40,562.66) circle (  1.16);

\path[draw=drawColor,line width= 0.4pt,line join=round,line cap=round,fill=fillColor] (199.57,562.61) circle (  1.16);

\path[draw=drawColor,line width= 0.4pt,line join=round,line cap=round,fill=fillColor] (199.74,562.57) circle (  1.16);

\path[draw=drawColor,line width= 0.4pt,line join=round,line cap=round,fill=fillColor] (199.91,562.49) circle (  1.16);

\path[draw=drawColor,line width= 0.4pt,line join=round,line cap=round,fill=fillColor] (200.08,562.44) circle (  1.16);

\path[draw=drawColor,line width= 0.4pt,line join=round,line cap=round,fill=fillColor] (200.25,562.43) circle (  1.16);

\path[draw=drawColor,line width= 0.4pt,line join=round,line cap=round,fill=fillColor] (200.42,562.39) circle (  1.16);

\path[draw=drawColor,line width= 0.4pt,line join=round,line cap=round,fill=fillColor] (200.58,562.20) circle (  1.16);

\path[draw=drawColor,line width= 0.4pt,line join=round,line cap=round,fill=fillColor] (200.75,561.98) circle (  1.16);

\path[draw=drawColor,line width= 0.4pt,line join=round,line cap=round,fill=fillColor] (200.92,561.97) circle (  1.16);

\path[draw=drawColor,line width= 0.4pt,line join=round,line cap=round,fill=fillColor] (201.09,561.89) circle (  1.16);

\path[draw=drawColor,line width= 0.4pt,line join=round,line cap=round,fill=fillColor] (201.25,561.85) circle (  1.16);

\path[draw=drawColor,line width= 0.4pt,line join=round,line cap=round,fill=fillColor] (201.42,561.68) circle (  1.16);

\path[draw=drawColor,line width= 0.4pt,line join=round,line cap=round,fill=fillColor] (201.58,561.54) circle (  1.16);

\path[draw=drawColor,line width= 0.4pt,line join=round,line cap=round,fill=fillColor] (201.75,561.46) circle (  1.16);

\path[draw=drawColor,line width= 0.4pt,line join=round,line cap=round,fill=fillColor] (201.91,561.00) circle (  1.16);

\path[draw=drawColor,line width= 0.4pt,line join=round,line cap=round,fill=fillColor] (202.08,560.99) circle (  1.16);

\path[draw=drawColor,line width= 0.4pt,line join=round,line cap=round,fill=fillColor] (202.24,560.89) circle (  1.16);

\path[draw=drawColor,line width= 0.4pt,line join=round,line cap=round,fill=fillColor] (202.41,560.89) circle (  1.16);

\path[draw=drawColor,line width= 0.4pt,line join=round,line cap=round,fill=fillColor] (202.57,560.56) circle (  1.16);

\path[draw=drawColor,line width= 0.4pt,line join=round,line cap=round,fill=fillColor] (202.73,560.51) circle (  1.16);

\path[draw=drawColor,line width= 0.4pt,line join=round,line cap=round,fill=fillColor] (202.90,560.48) circle (  1.16);

\path[draw=drawColor,line width= 0.4pt,line join=round,line cap=round,fill=fillColor] (203.06,560.28) circle (  1.16);

\path[draw=drawColor,line width= 0.4pt,line join=round,line cap=round,fill=fillColor] (203.22,560.23) circle (  1.16);

\path[draw=drawColor,line width= 0.4pt,line join=round,line cap=round,fill=fillColor] (203.38,560.20) circle (  1.16);

\path[draw=drawColor,line width= 0.4pt,line join=round,line cap=round,fill=fillColor] (203.54,559.95) circle (  1.16);

\path[draw=drawColor,line width= 0.4pt,line join=round,line cap=round,fill=fillColor] (203.71,559.80) circle (  1.16);

\path[draw=drawColor,line width= 0.4pt,line join=round,line cap=round,fill=fillColor] (203.87,559.65) circle (  1.16);

\path[draw=drawColor,line width= 0.4pt,line join=round,line cap=round,fill=fillColor] (204.03,559.59) circle (  1.16);

\path[draw=drawColor,line width= 0.4pt,line join=round,line cap=round,fill=fillColor] (204.19,559.54) circle (  1.16);

\path[draw=drawColor,line width= 0.4pt,line join=round,line cap=round,fill=fillColor] (204.35,558.79) circle (  1.16);

\path[draw=drawColor,line width= 0.4pt,line join=round,line cap=round,fill=fillColor] (204.51,558.67) circle (  1.16);

\path[draw=drawColor,line width= 0.4pt,line join=round,line cap=round,fill=fillColor] (204.66,558.50) circle (  1.16);

\path[draw=drawColor,line width= 0.4pt,line join=round,line cap=round,fill=fillColor] (204.82,558.47) circle (  1.16);

\path[draw=drawColor,line width= 0.4pt,line join=round,line cap=round,fill=fillColor] (204.98,558.44) circle (  1.16);

\path[draw=drawColor,line width= 0.4pt,line join=round,line cap=round,fill=fillColor] (205.14,558.37) circle (  1.16);

\path[draw=drawColor,line width= 0.4pt,line join=round,line cap=round,fill=fillColor] (205.30,558.35) circle (  1.16);

\path[draw=drawColor,line width= 0.4pt,line join=round,line cap=round,fill=fillColor] (205.45,558.22) circle (  1.16);

\path[draw=drawColor,line width= 0.4pt,line join=round,line cap=round,fill=fillColor] (205.61,558.14) circle (  1.16);

\path[draw=drawColor,line width= 0.4pt,line join=round,line cap=round,fill=fillColor] (205.77,558.14) circle (  1.16);

\path[draw=drawColor,line width= 0.4pt,line join=round,line cap=round,fill=fillColor] (205.92,558.08) circle (  1.16);

\path[draw=drawColor,line width= 0.4pt,line join=round,line cap=round,fill=fillColor] (206.08,558.02) circle (  1.16);

\path[draw=drawColor,line width= 0.4pt,line join=round,line cap=round,fill=fillColor] (206.24,557.97) circle (  1.16);

\path[draw=drawColor,line width= 0.4pt,line join=round,line cap=round,fill=fillColor] (206.39,557.81) circle (  1.16);

\path[draw=drawColor,line width= 0.4pt,line join=round,line cap=round,fill=fillColor] (206.55,557.78) circle (  1.16);

\path[draw=drawColor,line width= 0.4pt,line join=round,line cap=round,fill=fillColor] (206.70,557.07) circle (  1.16);

\path[draw=drawColor,line width= 0.4pt,line join=round,line cap=round,fill=fillColor] (206.86,557.00) circle (  1.16);

\path[draw=drawColor,line width= 0.4pt,line join=round,line cap=round,fill=fillColor] (207.01,556.84) circle (  1.16);

\path[draw=drawColor,line width= 0.4pt,line join=round,line cap=round,fill=fillColor] (207.16,556.40) circle (  1.16);

\path[draw=drawColor,line width= 0.4pt,line join=round,line cap=round,fill=fillColor] (207.32,556.37) circle (  1.16);

\path[draw=drawColor,line width= 0.4pt,line join=round,line cap=round,fill=fillColor] (207.47,556.13) circle (  1.16);

\path[draw=drawColor,line width= 0.4pt,line join=round,line cap=round,fill=fillColor] (207.62,555.97) circle (  1.16);

\path[draw=drawColor,line width= 0.4pt,line join=round,line cap=round,fill=fillColor] (207.78,555.94) circle (  1.16);

\path[draw=drawColor,line width= 0.4pt,line join=round,line cap=round,fill=fillColor] (207.93,555.88) circle (  1.16);

\path[draw=drawColor,line width= 0.4pt,line join=round,line cap=round,fill=fillColor] (208.08,555.88) circle (  1.16);

\path[draw=drawColor,line width= 0.4pt,line join=round,line cap=round,fill=fillColor] (208.23,555.80) circle (  1.16);

\path[draw=drawColor,line width= 0.4pt,line join=round,line cap=round,fill=fillColor] (208.39,555.63) circle (  1.16);

\path[draw=drawColor,line width= 0.4pt,line join=round,line cap=round,fill=fillColor] (208.54,555.59) circle (  1.16);

\path[draw=drawColor,line width= 0.4pt,line join=round,line cap=round,fill=fillColor] (208.69,555.16) circle (  1.16);

\path[draw=drawColor,line width= 0.4pt,line join=round,line cap=round,fill=fillColor] (208.84,555.09) circle (  1.16);

\path[draw=drawColor,line width= 0.4pt,line join=round,line cap=round,fill=fillColor] (208.99,555.09) circle (  1.16);

\path[draw=drawColor,line width= 0.4pt,line join=round,line cap=round,fill=fillColor] (209.14,555.06) circle (  1.16);

\path[draw=drawColor,line width= 0.4pt,line join=round,line cap=round,fill=fillColor] (209.29,554.89) circle (  1.16);

\path[draw=drawColor,line width= 0.4pt,line join=round,line cap=round,fill=fillColor] (209.44,554.81) circle (  1.16);

\path[draw=drawColor,line width= 0.4pt,line join=round,line cap=round,fill=fillColor] (209.59,554.79) circle (  1.16);

\path[draw=drawColor,line width= 0.4pt,line join=round,line cap=round,fill=fillColor] (209.74,554.54) circle (  1.16);

\path[draw=drawColor,line width= 0.4pt,line join=round,line cap=round,fill=fillColor] (209.88,554.28) circle (  1.16);

\path[draw=drawColor,line width= 0.4pt,line join=round,line cap=round,fill=fillColor] (210.03,554.16) circle (  1.16);

\path[draw=drawColor,line width= 0.4pt,line join=round,line cap=round,fill=fillColor] (210.18,553.88) circle (  1.16);

\path[draw=drawColor,line width= 0.4pt,line join=round,line cap=round,fill=fillColor] (210.33,553.36) circle (  1.16);

\path[draw=drawColor,line width= 0.4pt,line join=round,line cap=round,fill=fillColor] (210.48,553.15) circle (  1.16);

\path[draw=drawColor,line width= 0.4pt,line join=round,line cap=round,fill=fillColor] (210.62,553.05) circle (  1.16);

\path[draw=drawColor,line width= 0.4pt,line join=round,line cap=round,fill=fillColor] (210.77,553.02) circle (  1.16);

\path[draw=drawColor,line width= 0.4pt,line join=round,line cap=round,fill=fillColor] (210.92,552.82) circle (  1.16);

\path[draw=drawColor,line width= 0.4pt,line join=round,line cap=round,fill=fillColor] (211.06,552.81) circle (  1.16);

\path[draw=drawColor,line width= 0.4pt,line join=round,line cap=round,fill=fillColor] (211.21,552.75) circle (  1.16);

\path[draw=drawColor,line width= 0.4pt,line join=round,line cap=round,fill=fillColor] (211.36,552.71) circle (  1.16);

\path[draw=drawColor,line width= 0.4pt,line join=round,line cap=round,fill=fillColor] (211.50,552.64) circle (  1.16);

\path[draw=drawColor,line width= 0.4pt,line join=round,line cap=round,fill=fillColor] (211.65,552.54) circle (  1.16);

\path[draw=drawColor,line width= 0.4pt,line join=round,line cap=round,fill=fillColor] (211.79,552.43) circle (  1.16);

\path[draw=drawColor,line width= 0.4pt,line join=round,line cap=round,fill=fillColor] (211.94,552.23) circle (  1.16);

\path[draw=drawColor,line width= 0.4pt,line join=round,line cap=round,fill=fillColor] (212.08,552.00) circle (  1.16);

\path[draw=drawColor,line width= 0.4pt,line join=round,line cap=round,fill=fillColor] (212.23,551.97) circle (  1.16);

\path[draw=drawColor,line width= 0.4pt,line join=round,line cap=round,fill=fillColor] (212.37,551.94) circle (  1.16);

\path[draw=drawColor,line width= 0.4pt,line join=round,line cap=round,fill=fillColor] (212.51,551.84) circle (  1.16);

\path[draw=drawColor,line width= 0.4pt,line join=round,line cap=round,fill=fillColor] (212.66,551.75) circle (  1.16);

\path[draw=drawColor,line width= 0.4pt,line join=round,line cap=round,fill=fillColor] (212.80,550.87) circle (  1.16);

\path[draw=drawColor,line width= 0.4pt,line join=round,line cap=round,fill=fillColor] (212.94,550.80) circle (  1.16);

\path[draw=drawColor,line width= 0.4pt,line join=round,line cap=round,fill=fillColor] (213.09,550.76) circle (  1.16);

\path[draw=drawColor,line width= 0.4pt,line join=round,line cap=round,fill=fillColor] (213.23,550.64) circle (  1.16);

\path[draw=drawColor,line width= 0.4pt,line join=round,line cap=round,fill=fillColor] (213.37,550.56) circle (  1.16);

\path[draw=drawColor,line width= 0.4pt,line join=round,line cap=round,fill=fillColor] (213.51,550.29) circle (  1.16);

\path[draw=drawColor,line width= 0.4pt,line join=round,line cap=round,fill=fillColor] (213.65,550.08) circle (  1.16);

\path[draw=drawColor,line width= 0.4pt,line join=round,line cap=round,fill=fillColor] (213.80,550.08) circle (  1.16);

\path[draw=drawColor,line width= 0.4pt,line join=round,line cap=round,fill=fillColor] (213.94,550.08) circle (  1.16);

\path[draw=drawColor,line width= 0.4pt,line join=round,line cap=round,fill=fillColor] (214.08,550.05) circle (  1.16);

\path[draw=drawColor,line width= 0.4pt,line join=round,line cap=round,fill=fillColor] (214.22,549.60) circle (  1.16);

\path[draw=drawColor,line width= 0.4pt,line join=round,line cap=round,fill=fillColor] (214.36,549.48) circle (  1.16);

\path[draw=drawColor,line width= 0.4pt,line join=round,line cap=round,fill=fillColor] (214.50,549.40) circle (  1.16);

\path[draw=drawColor,line width= 0.4pt,line join=round,line cap=round,fill=fillColor] (214.64,548.67) circle (  1.16);

\path[draw=drawColor,line width= 0.4pt,line join=round,line cap=round,fill=fillColor] (214.78,547.88) circle (  1.16);

\path[draw=drawColor,line width= 0.4pt,line join=round,line cap=round,fill=fillColor] (214.92,547.55) circle (  1.16);

\path[draw=drawColor,line width= 0.4pt,line join=round,line cap=round,fill=fillColor] (215.06,547.53) circle (  1.16);

\path[draw=drawColor,line width= 0.4pt,line join=round,line cap=round,fill=fillColor] (215.20,547.53) circle (  1.16);

\path[draw=drawColor,line width= 0.4pt,line join=round,line cap=round,fill=fillColor] (215.34,547.33) circle (  1.16);

\path[draw=drawColor,line width= 0.4pt,line join=round,line cap=round,fill=fillColor] (215.47,547.27) circle (  1.16);

\path[draw=drawColor,line width= 0.4pt,line join=round,line cap=round,fill=fillColor] (215.61,546.45) circle (  1.16);

\path[draw=drawColor,line width= 0.4pt,line join=round,line cap=round,fill=fillColor] (215.75,546.15) circle (  1.16);

\path[draw=drawColor,line width= 0.4pt,line join=round,line cap=round,fill=fillColor] (215.89,545.88) circle (  1.16);

\path[draw=drawColor,line width= 0.4pt,line join=round,line cap=round,fill=fillColor] (216.03,545.86) circle (  1.16);

\path[draw=drawColor,line width= 0.4pt,line join=round,line cap=round,fill=fillColor] (216.16,544.08) circle (  1.16);

\path[draw=drawColor,line width= 0.4pt,line join=round,line cap=round,fill=fillColor] (216.30,543.14) circle (  1.16);

\path[draw=drawColor,line width= 0.4pt,line join=round,line cap=round,fill=fillColor] (216.44,538.55) circle (  1.16);

\path[draw=drawColor,line width= 0.4pt,line join=round,line cap=round,fill=fillColor] (216.58,535.03) circle (  1.16);

\path[draw=drawColor,line width= 0.4pt,line join=round,line cap=round,fill=fillColor] (216.71,535.03) circle (  1.16);

\path[draw=drawColor,line width= 0.4pt,line join=round,line cap=round,fill=fillColor] (216.85,535.03) circle (  1.16);

\path[draw=drawColor,line width= 0.4pt,line join=round,line cap=round,fill=fillColor] (216.98,535.03) circle (  1.16);

\path[draw=drawColor,line width= 0.4pt,line join=round,line cap=round,fill=fillColor] (217.12,535.03) circle (  1.16);

\path[draw=drawColor,line width= 0.4pt,line join=round,line cap=round,fill=fillColor] (217.26,535.03) circle (  1.16);

\path[draw=drawColor,line width= 0.4pt,line join=round,line cap=round,fill=fillColor] (217.39,535.03) circle (  1.16);

\path[draw=drawColor,line width= 0.4pt,line join=round,line cap=round,fill=fillColor] (217.53,535.03) circle (  1.16);

\path[draw=drawColor,line width= 0.4pt,line join=round,line cap=round,fill=fillColor] (217.66,535.03) circle (  1.16);

\path[draw=drawColor,line width= 0.4pt,line join=round,line cap=round,fill=fillColor] (217.80,535.03) circle (  1.16);

\path[draw=drawColor,line width= 0.4pt,line join=round,line cap=round,fill=fillColor] (217.93,535.03) circle (  1.16);

\path[draw=drawColor,line width= 0.4pt,line join=round,line cap=round,fill=fillColor] (218.06,535.03) circle (  1.16);

\path[draw=drawColor,line width= 0.4pt,line join=round,line cap=round,fill=fillColor] (218.20,535.03) circle (  1.16);

\path[draw=drawColor,line width= 0.4pt,line join=round,line cap=round,fill=fillColor] (218.33,535.03) circle (  1.16);

\path[draw=drawColor,line width= 0.4pt,line join=round,line cap=round,fill=fillColor] (218.47,535.03) circle (  1.16);

\path[draw=drawColor,line width= 0.4pt,line join=round,line cap=round,fill=fillColor] (218.60,535.03) circle (  1.16);

\path[draw=drawColor,line width= 0.4pt,line join=round,line cap=round,fill=fillColor] (218.73,535.03) circle (  1.16);

\path[draw=drawColor,line width= 0.4pt,line join=round,line cap=round,fill=fillColor] (218.87,535.03) circle (  1.16);

\path[draw=drawColor,line width= 0.4pt,line join=round,line cap=round,fill=fillColor] (219.00,535.03) circle (  1.16);

\path[draw=drawColor,line width= 0.4pt,line join=round,line cap=round,fill=fillColor] (219.13,535.03) circle (  1.16);

\path[draw=drawColor,line width= 0.4pt,line join=round,line cap=round,fill=fillColor] (219.26,535.03) circle (  1.16);

\path[draw=drawColor,line width= 0.4pt,line join=round,line cap=round,fill=fillColor] (219.40,535.03) circle (  1.16);

\path[draw=drawColor,line width= 0.4pt,line join=round,line cap=round,fill=fillColor] (219.53,535.03) circle (  1.16);

\path[draw=drawColor,line width= 0.4pt,line join=round,line cap=round,fill=fillColor] (219.66,535.03) circle (  1.16);

\path[draw=drawColor,line width= 0.4pt,line join=round,line cap=round,fill=fillColor] (219.79,535.03) circle (  1.16);

\path[draw=drawColor,line width= 0.4pt,line join=round,line cap=round,fill=fillColor] (219.92,535.03) circle (  1.16);

\path[draw=drawColor,line width= 0.4pt,line join=round,line cap=round,fill=fillColor] (220.05,535.03) circle (  1.16);

\path[draw=drawColor,line width= 0.4pt,line join=round,line cap=round,fill=fillColor] (220.18,535.03) circle (  1.16);

\path[draw=drawColor,line width= 0.4pt,line join=round,line cap=round,fill=fillColor] (220.31,535.03) circle (  1.16);

\path[draw=drawColor,line width= 0.4pt,line join=round,line cap=round,fill=fillColor] (220.45,535.03) circle (  1.16);

\path[draw=drawColor,line width= 0.4pt,line join=round,line cap=round,fill=fillColor] (220.58,535.03) circle (  1.16);

\path[draw=drawColor,line width= 0.4pt,line join=round,line cap=round,fill=fillColor] (220.71,535.03) circle (  1.16);

\path[draw=drawColor,line width= 0.4pt,line join=round,line cap=round,fill=fillColor] (220.84,535.03) circle (  1.16);

\path[draw=drawColor,line width= 0.4pt,line join=round,line cap=round,fill=fillColor] (220.97,535.03) circle (  1.16);

\path[draw=drawColor,line width= 0.4pt,line join=round,line cap=round,fill=fillColor] (221.09,535.03) circle (  1.16);

\path[draw=drawColor,line width= 0.4pt,line join=round,line cap=round,fill=fillColor] (221.22,535.03) circle (  1.16);

\path[draw=drawColor,line width= 0.4pt,line join=round,line cap=round,fill=fillColor] (221.35,535.03) circle (  1.16);

\path[draw=drawColor,line width= 0.4pt,line join=round,line cap=round,fill=fillColor] (221.48,535.03) circle (  1.16);

\path[draw=drawColor,line width= 0.4pt,line join=round,line cap=round,fill=fillColor] (221.61,535.03) circle (  1.16);

\path[draw=drawColor,line width= 0.4pt,line join=round,line cap=round,fill=fillColor] (221.74,535.03) circle (  1.16);

\path[draw=drawColor,line width= 0.4pt,line join=round,line cap=round,fill=fillColor] (221.87,535.03) circle (  1.16);

\path[draw=drawColor,line width= 0.4pt,line join=round,line cap=round,fill=fillColor] (222.00,535.03) circle (  1.16);

\path[draw=drawColor,line width= 0.4pt,line join=round,line cap=round,fill=fillColor] (222.12,535.03) circle (  1.16);

\path[draw=drawColor,line width= 0.4pt,line join=round,line cap=round,fill=fillColor] (222.25,535.03) circle (  1.16);

\path[draw=drawColor,line width= 0.4pt,line join=round,line cap=round,fill=fillColor] (222.38,535.03) circle (  1.16);

\path[draw=drawColor,line width= 0.4pt,line join=round,line cap=round,fill=fillColor] (222.51,535.03) circle (  1.16);

\path[draw=drawColor,line width= 0.4pt,line join=round,line cap=round,fill=fillColor] (222.63,535.03) circle (  1.16);

\path[draw=drawColor,line width= 0.4pt,line join=round,line cap=round,fill=fillColor] (222.76,535.03) circle (  1.16);

\path[draw=drawColor,line width= 0.4pt,line join=round,line cap=round,fill=fillColor] (222.89,535.03) circle (  1.16);

\path[draw=drawColor,line width= 0.4pt,line join=round,line cap=round,fill=fillColor] (223.01,535.03) circle (  1.16);

\path[draw=drawColor,line width= 0.4pt,line join=round,line cap=round,fill=fillColor] (223.14,535.03) circle (  1.16);

\path[draw=drawColor,line width= 0.4pt,line join=round,line cap=round,fill=fillColor] (223.27,535.03) circle (  1.16);

\path[draw=drawColor,line width= 0.4pt,line join=round,line cap=round,fill=fillColor] (223.39,535.03) circle (  1.16);

\path[draw=drawColor,line width= 0.4pt,line join=round,line cap=round,fill=fillColor] (223.52,535.03) circle (  1.16);

\path[draw=drawColor,line width= 0.4pt,line join=round,line cap=round,fill=fillColor] (223.64,535.03) circle (  1.16);

\path[draw=drawColor,line width= 0.4pt,line join=round,line cap=round,fill=fillColor] (223.77,535.03) circle (  1.16);

\path[draw=drawColor,line width= 0.4pt,line join=round,line cap=round,fill=fillColor] (223.89,535.03) circle (  1.16);

\path[draw=drawColor,line width= 0.4pt,line join=round,line cap=round,fill=fillColor] (224.02,535.03) circle (  1.16);

\path[draw=drawColor,line width= 0.4pt,line join=round,line cap=round,fill=fillColor] (224.14,535.03) circle (  1.16);

\path[draw=drawColor,line width= 0.4pt,line join=round,line cap=round,fill=fillColor] (224.27,535.03) circle (  1.16);

\path[draw=drawColor,line width= 0.4pt,line join=round,line cap=round,fill=fillColor] (224.39,535.03) circle (  1.16);

\path[draw=drawColor,line width= 0.4pt,line join=round,line cap=round,fill=fillColor] (224.52,535.03) circle (  1.16);

\path[draw=drawColor,line width= 0.4pt,line join=round,line cap=round,fill=fillColor] (224.64,535.03) circle (  1.16);

\path[draw=drawColor,line width= 0.4pt,line join=round,line cap=round,fill=fillColor] (224.77,535.03) circle (  1.16);

\path[draw=drawColor,line width= 0.4pt,line join=round,line cap=round,fill=fillColor] (224.89,535.03) circle (  1.16);

\path[draw=drawColor,line width= 0.4pt,line join=round,line cap=round,fill=fillColor] (225.01,535.03) circle (  1.16);

\path[draw=drawColor,line width= 0.4pt,line join=round,line cap=round,fill=fillColor] (225.14,535.03) circle (  1.16);

\path[draw=drawColor,line width= 0.4pt,line join=round,line cap=round,fill=fillColor] (225.26,535.03) circle (  1.16);
\definecolor[named]{drawColor}{rgb}{1.00,0.50,0.00}
\definecolor[named]{fillColor}{rgb}{1.00,0.50,0.00}

\path[draw=drawColor,line width= 0.4pt,line join=round,line cap=round,fill=fillColor] ( 74.88,620.67) circle (  1.16);

\path[draw=drawColor,line width= 0.4pt,line join=round,line cap=round,fill=fillColor] ( 80.66,614.92) circle (  1.16);

\path[draw=drawColor,line width= 0.4pt,line join=round,line cap=round,fill=fillColor] ( 84.72,614.66) circle (  1.16);

\path[draw=drawColor,line width= 0.4pt,line join=round,line cap=round,fill=fillColor] ( 87.95,611.32) circle (  1.16);

\path[draw=drawColor,line width= 0.4pt,line join=round,line cap=round,fill=fillColor] ( 90.68,608.85) circle (  1.16);

\path[draw=drawColor,line width= 0.4pt,line join=round,line cap=round,fill=fillColor] ( 93.06,608.81) circle (  1.16);

\path[draw=drawColor,line width= 0.4pt,line join=round,line cap=round,fill=fillColor] ( 95.19,606.94) circle (  1.16);

\path[draw=drawColor,line width= 0.4pt,line join=round,line cap=round,fill=fillColor] ( 97.13,606.62) circle (  1.16);

\path[draw=drawColor,line width= 0.4pt,line join=round,line cap=round,fill=fillColor] ( 98.91,606.27) circle (  1.16);

\path[draw=drawColor,line width= 0.4pt,line join=round,line cap=round,fill=fillColor] (100.57,605.08) circle (  1.16);

\path[draw=drawColor,line width= 0.4pt,line join=round,line cap=round,fill=fillColor] (102.11,604.95) circle (  1.16);

\path[draw=drawColor,line width= 0.4pt,line join=round,line cap=round,fill=fillColor] (103.57,603.98) circle (  1.16);

\path[draw=drawColor,line width= 0.4pt,line join=round,line cap=round,fill=fillColor] (104.95,603.46) circle (  1.16);

\path[draw=drawColor,line width= 0.4pt,line join=round,line cap=round,fill=fillColor] (106.26,602.28) circle (  1.16);

\path[draw=drawColor,line width= 0.4pt,line join=round,line cap=round,fill=fillColor] (107.50,601.99) circle (  1.16);

\path[draw=drawColor,line width= 0.4pt,line join=round,line cap=round,fill=fillColor] (108.70,601.98) circle (  1.16);

\path[draw=drawColor,line width= 0.4pt,line join=round,line cap=round,fill=fillColor] (109.84,601.73) circle (  1.16);

\path[draw=drawColor,line width= 0.4pt,line join=round,line cap=round,fill=fillColor] (110.94,601.21) circle (  1.16);

\path[draw=drawColor,line width= 0.4pt,line join=round,line cap=round,fill=fillColor] (112.00,601.08) circle (  1.16);

\path[draw=drawColor,line width= 0.4pt,line join=round,line cap=round,fill=fillColor] (113.03,601.05) circle (  1.16);

\path[draw=drawColor,line width= 0.4pt,line join=round,line cap=round,fill=fillColor] (114.02,600.88) circle (  1.16);

\path[draw=drawColor,line width= 0.4pt,line join=round,line cap=round,fill=fillColor] (114.98,598.82) circle (  1.16);

\path[draw=drawColor,line width= 0.4pt,line join=round,line cap=round,fill=fillColor] (115.91,598.18) circle (  1.16);

\path[draw=drawColor,line width= 0.4pt,line join=round,line cap=round,fill=fillColor] (116.81,597.46) circle (  1.16);

\path[draw=drawColor,line width= 0.4pt,line join=round,line cap=round,fill=fillColor] (117.69,594.29) circle (  1.16);

\path[draw=drawColor,line width= 0.4pt,line join=round,line cap=round,fill=fillColor] (118.55,594.18) circle (  1.16);

\path[draw=drawColor,line width= 0.4pt,line join=round,line cap=round,fill=fillColor] (119.38,593.86) circle (  1.16);

\path[draw=drawColor,line width= 0.4pt,line join=round,line cap=round,fill=fillColor] (120.20,593.16) circle (  1.16);

\path[draw=drawColor,line width= 0.4pt,line join=round,line cap=round,fill=fillColor] (120.99,592.81) circle (  1.16);

\path[draw=drawColor,line width= 0.4pt,line join=round,line cap=round,fill=fillColor] (121.77,592.20) circle (  1.16);

\path[draw=drawColor,line width= 0.4pt,line join=round,line cap=round,fill=fillColor] (122.53,591.85) circle (  1.16);

\path[draw=drawColor,line width= 0.4pt,line join=round,line cap=round,fill=fillColor] (123.27,591.47) circle (  1.16);

\path[draw=drawColor,line width= 0.4pt,line join=round,line cap=round,fill=fillColor] (124.00,591.04) circle (  1.16);

\path[draw=drawColor,line width= 0.4pt,line join=round,line cap=round,fill=fillColor] (124.71,590.63) circle (  1.16);

\path[draw=drawColor,line width= 0.4pt,line join=round,line cap=round,fill=fillColor] (125.41,590.46) circle (  1.16);

\path[draw=drawColor,line width= 0.4pt,line join=round,line cap=round,fill=fillColor] (126.10,589.68) circle (  1.16);

\path[draw=drawColor,line width= 0.4pt,line join=round,line cap=round,fill=fillColor] (126.77,589.41) circle (  1.16);

\path[draw=drawColor,line width= 0.4pt,line join=round,line cap=round,fill=fillColor] (127.44,589.30) circle (  1.16);

\path[draw=drawColor,line width= 0.4pt,line join=round,line cap=round,fill=fillColor] (128.09,588.55) circle (  1.16);

\path[draw=drawColor,line width= 0.4pt,line join=round,line cap=round,fill=fillColor] (128.73,587.84) circle (  1.16);

\path[draw=drawColor,line width= 0.4pt,line join=round,line cap=round,fill=fillColor] (129.35,587.63) circle (  1.16);

\path[draw=drawColor,line width= 0.4pt,line join=round,line cap=round,fill=fillColor] (129.97,587.62) circle (  1.16);

\path[draw=drawColor,line width= 0.4pt,line join=round,line cap=round,fill=fillColor] (130.58,586.98) circle (  1.16);

\path[draw=drawColor,line width= 0.4pt,line join=round,line cap=round,fill=fillColor] (131.18,586.89) circle (  1.16);

\path[draw=drawColor,line width= 0.4pt,line join=round,line cap=round,fill=fillColor] (131.77,586.74) circle (  1.16);

\path[draw=drawColor,line width= 0.4pt,line join=round,line cap=round,fill=fillColor] (132.35,586.55) circle (  1.16);

\path[draw=drawColor,line width= 0.4pt,line join=round,line cap=round,fill=fillColor] (132.93,586.54) circle (  1.16);

\path[draw=drawColor,line width= 0.4pt,line join=round,line cap=round,fill=fillColor] (133.49,586.42) circle (  1.16);

\path[draw=drawColor,line width= 0.4pt,line join=round,line cap=round,fill=fillColor] (134.05,586.03) circle (  1.16);

\path[draw=drawColor,line width= 0.4pt,line join=round,line cap=round,fill=fillColor] (134.60,585.91) circle (  1.16);

\path[draw=drawColor,line width= 0.4pt,line join=round,line cap=round,fill=fillColor] (135.14,585.88) circle (  1.16);

\path[draw=drawColor,line width= 0.4pt,line join=round,line cap=round,fill=fillColor] (135.68,585.63) circle (  1.16);

\path[draw=drawColor,line width= 0.4pt,line join=round,line cap=round,fill=fillColor] (136.21,585.16) circle (  1.16);

\path[draw=drawColor,line width= 0.4pt,line join=round,line cap=round,fill=fillColor] (136.73,584.80) circle (  1.16);

\path[draw=drawColor,line width= 0.4pt,line join=round,line cap=round,fill=fillColor] (137.25,584.69) circle (  1.16);

\path[draw=drawColor,line width= 0.4pt,line join=round,line cap=round,fill=fillColor] (137.76,584.16) circle (  1.16);

\path[draw=drawColor,line width= 0.4pt,line join=round,line cap=round,fill=fillColor] (138.26,583.60) circle (  1.16);

\path[draw=drawColor,line width= 0.4pt,line join=round,line cap=round,fill=fillColor] (138.76,583.56) circle (  1.16);

\path[draw=drawColor,line width= 0.4pt,line join=round,line cap=round,fill=fillColor] (139.25,583.16) circle (  1.16);

\path[draw=drawColor,line width= 0.4pt,line join=round,line cap=round,fill=fillColor] (139.74,582.42) circle (  1.16);

\path[draw=drawColor,line width= 0.4pt,line join=round,line cap=round,fill=fillColor] (140.22,582.34) circle (  1.16);

\path[draw=drawColor,line width= 0.4pt,line join=round,line cap=round,fill=fillColor] (140.70,582.27) circle (  1.16);

\path[draw=drawColor,line width= 0.4pt,line join=round,line cap=round,fill=fillColor] (141.17,581.65) circle (  1.16);

\path[draw=drawColor,line width= 0.4pt,line join=round,line cap=round,fill=fillColor] (141.63,581.44) circle (  1.16);

\path[draw=drawColor,line width= 0.4pt,line join=round,line cap=round,fill=fillColor] (142.09,580.49) circle (  1.16);

\path[draw=drawColor,line width= 0.4pt,line join=round,line cap=round,fill=fillColor] (142.55,579.89) circle (  1.16);

\path[draw=drawColor,line width= 0.4pt,line join=round,line cap=round,fill=fillColor] (143.00,579.74) circle (  1.16);

\path[draw=drawColor,line width= 0.4pt,line join=round,line cap=round,fill=fillColor] (143.45,579.52) circle (  1.16);

\path[draw=drawColor,line width= 0.4pt,line join=round,line cap=round,fill=fillColor] (143.89,579.43) circle (  1.16);

\path[draw=drawColor,line width= 0.4pt,line join=round,line cap=round,fill=fillColor] (144.33,579.11) circle (  1.16);

\path[draw=drawColor,line width= 0.4pt,line join=round,line cap=round,fill=fillColor] (144.77,579.10) circle (  1.16);

\path[draw=drawColor,line width= 0.4pt,line join=round,line cap=round,fill=fillColor] (145.20,578.27) circle (  1.16);

\path[draw=drawColor,line width= 0.4pt,line join=round,line cap=round,fill=fillColor] (145.62,578.06) circle (  1.16);

\path[draw=drawColor,line width= 0.4pt,line join=round,line cap=round,fill=fillColor] (146.05,577.99) circle (  1.16);

\path[draw=drawColor,line width= 0.4pt,line join=round,line cap=round,fill=fillColor] (146.46,577.99) circle (  1.16);

\path[draw=drawColor,line width= 0.4pt,line join=round,line cap=round,fill=fillColor] (146.88,577.91) circle (  1.16);

\path[draw=drawColor,line width= 0.4pt,line join=round,line cap=round,fill=fillColor] (147.29,577.88) circle (  1.16);

\path[draw=drawColor,line width= 0.4pt,line join=round,line cap=round,fill=fillColor] (147.70,577.82) circle (  1.16);

\path[draw=drawColor,line width= 0.4pt,line join=round,line cap=round,fill=fillColor] (148.10,577.65) circle (  1.16);

\path[draw=drawColor,line width= 0.4pt,line join=round,line cap=round,fill=fillColor] (148.50,577.54) circle (  1.16);

\path[draw=drawColor,line width= 0.4pt,line join=round,line cap=round,fill=fillColor] (148.90,577.41) circle (  1.16);

\path[draw=drawColor,line width= 0.4pt,line join=round,line cap=round,fill=fillColor] (149.30,577.25) circle (  1.16);

\path[draw=drawColor,line width= 0.4pt,line join=round,line cap=round,fill=fillColor] (149.69,577.25) circle (  1.16);

\path[draw=drawColor,line width= 0.4pt,line join=round,line cap=round,fill=fillColor] (150.08,577.20) circle (  1.16);

\path[draw=drawColor,line width= 0.4pt,line join=round,line cap=round,fill=fillColor] (150.46,576.88) circle (  1.16);

\path[draw=drawColor,line width= 0.4pt,line join=round,line cap=round,fill=fillColor] (150.84,576.82) circle (  1.16);

\path[draw=drawColor,line width= 0.4pt,line join=round,line cap=round,fill=fillColor] (151.22,576.49) circle (  1.16);

\path[draw=drawColor,line width= 0.4pt,line join=round,line cap=round,fill=fillColor] (151.60,576.24) circle (  1.16);

\path[draw=drawColor,line width= 0.4pt,line join=round,line cap=round,fill=fillColor] (151.97,575.93) circle (  1.16);

\path[draw=drawColor,line width= 0.4pt,line join=round,line cap=round,fill=fillColor] (152.34,575.80) circle (  1.16);

\path[draw=drawColor,line width= 0.4pt,line join=round,line cap=round,fill=fillColor] (152.71,575.77) circle (  1.16);

\path[draw=drawColor,line width= 0.4pt,line join=round,line cap=round,fill=fillColor] (153.08,575.68) circle (  1.16);

\path[draw=drawColor,line width= 0.4pt,line join=round,line cap=round,fill=fillColor] (153.44,575.65) circle (  1.16);

\path[draw=drawColor,line width= 0.4pt,line join=round,line cap=round,fill=fillColor] (153.80,575.65) circle (  1.16);

\path[draw=drawColor,line width= 0.4pt,line join=round,line cap=round,fill=fillColor] (154.16,575.21) circle (  1.16);

\path[draw=drawColor,line width= 0.4pt,line join=round,line cap=round,fill=fillColor] (154.51,575.15) circle (  1.16);

\path[draw=drawColor,line width= 0.4pt,line join=round,line cap=round,fill=fillColor] (154.86,575.14) circle (  1.16);

\path[draw=drawColor,line width= 0.4pt,line join=round,line cap=round,fill=fillColor] (155.22,574.51) circle (  1.16);

\path[draw=drawColor,line width= 0.4pt,line join=round,line cap=round,fill=fillColor] (155.56,574.47) circle (  1.16);

\path[draw=drawColor,line width= 0.4pt,line join=round,line cap=round,fill=fillColor] (155.91,574.28) circle (  1.16);

\path[draw=drawColor,line width= 0.4pt,line join=round,line cap=round,fill=fillColor] (156.25,574.15) circle (  1.16);

\path[draw=drawColor,line width= 0.4pt,line join=round,line cap=round,fill=fillColor] (156.59,573.99) circle (  1.16);

\path[draw=drawColor,line width= 0.4pt,line join=round,line cap=round,fill=fillColor] (156.93,573.99) circle (  1.16);

\path[draw=drawColor,line width= 0.4pt,line join=round,line cap=round,fill=fillColor] (157.27,573.67) circle (  1.16);

\path[draw=drawColor,line width= 0.4pt,line join=round,line cap=round,fill=fillColor] (157.60,573.63) circle (  1.16);

\path[draw=drawColor,line width= 0.4pt,line join=round,line cap=round,fill=fillColor] (157.93,573.53) circle (  1.16);

\path[draw=drawColor,line width= 0.4pt,line join=round,line cap=round,fill=fillColor] (158.26,573.35) circle (  1.16);

\path[draw=drawColor,line width= 0.4pt,line join=round,line cap=round,fill=fillColor] (158.59,573.28) circle (  1.16);

\path[draw=drawColor,line width= 0.4pt,line join=round,line cap=round,fill=fillColor] (158.92,573.21) circle (  1.16);

\path[draw=drawColor,line width= 0.4pt,line join=round,line cap=round,fill=fillColor] (159.24,573.20) circle (  1.16);

\path[draw=drawColor,line width= 0.4pt,line join=round,line cap=round,fill=fillColor] (159.56,573.00) circle (  1.16);

\path[draw=drawColor,line width= 0.4pt,line join=round,line cap=round,fill=fillColor] (159.88,572.78) circle (  1.16);

\path[draw=drawColor,line width= 0.4pt,line join=round,line cap=round,fill=fillColor] (160.20,572.76) circle (  1.16);

\path[draw=drawColor,line width= 0.4pt,line join=round,line cap=round,fill=fillColor] (160.52,572.44) circle (  1.16);

\path[draw=drawColor,line width= 0.4pt,line join=round,line cap=round,fill=fillColor] (160.83,572.13) circle (  1.16);

\path[draw=drawColor,line width= 0.4pt,line join=round,line cap=round,fill=fillColor] (161.15,572.07) circle (  1.16);

\path[draw=drawColor,line width= 0.4pt,line join=round,line cap=round,fill=fillColor] (161.46,572.04) circle (  1.16);

\path[draw=drawColor,line width= 0.4pt,line join=round,line cap=round,fill=fillColor] (161.77,571.95) circle (  1.16);

\path[draw=drawColor,line width= 0.4pt,line join=round,line cap=round,fill=fillColor] (162.07,571.95) circle (  1.16);

\path[draw=drawColor,line width= 0.4pt,line join=round,line cap=round,fill=fillColor] (162.38,571.70) circle (  1.16);

\path[draw=drawColor,line width= 0.4pt,line join=round,line cap=round,fill=fillColor] (162.68,571.56) circle (  1.16);

\path[draw=drawColor,line width= 0.4pt,line join=round,line cap=round,fill=fillColor] (162.99,571.53) circle (  1.16);

\path[draw=drawColor,line width= 0.4pt,line join=round,line cap=round,fill=fillColor] (163.29,571.47) circle (  1.16);

\path[draw=drawColor,line width= 0.4pt,line join=round,line cap=round,fill=fillColor] (163.59,571.39) circle (  1.16);

\path[draw=drawColor,line width= 0.4pt,line join=round,line cap=round,fill=fillColor] (163.88,571.25) circle (  1.16);

\path[draw=drawColor,line width= 0.4pt,line join=round,line cap=round,fill=fillColor] (164.18,571.09) circle (  1.16);

\path[draw=drawColor,line width= 0.4pt,line join=round,line cap=round,fill=fillColor] (164.47,571.08) circle (  1.16);

\path[draw=drawColor,line width= 0.4pt,line join=round,line cap=round,fill=fillColor] (164.77,571.04) circle (  1.16);

\path[draw=drawColor,line width= 0.4pt,line join=round,line cap=round,fill=fillColor] (165.06,570.73) circle (  1.16);

\path[draw=drawColor,line width= 0.4pt,line join=round,line cap=round,fill=fillColor] (165.35,570.56) circle (  1.16);

\path[draw=drawColor,line width= 0.4pt,line join=round,line cap=round,fill=fillColor] (165.64,570.49) circle (  1.16);

\path[draw=drawColor,line width= 0.4pt,line join=round,line cap=round,fill=fillColor] (165.92,570.41) circle (  1.16);

\path[draw=drawColor,line width= 0.4pt,line join=round,line cap=round,fill=fillColor] (166.21,570.36) circle (  1.16);

\path[draw=drawColor,line width= 0.4pt,line join=round,line cap=round,fill=fillColor] (166.49,570.35) circle (  1.16);

\path[draw=drawColor,line width= 0.4pt,line join=round,line cap=round,fill=fillColor] (166.77,570.18) circle (  1.16);

\path[draw=drawColor,line width= 0.4pt,line join=round,line cap=round,fill=fillColor] (167.06,569.99) circle (  1.16);

\path[draw=drawColor,line width= 0.4pt,line join=round,line cap=round,fill=fillColor] (167.34,569.92) circle (  1.16);

\path[draw=drawColor,line width= 0.4pt,line join=round,line cap=round,fill=fillColor] (167.61,569.90) circle (  1.16);

\path[draw=drawColor,line width= 0.4pt,line join=round,line cap=round,fill=fillColor] (167.89,569.82) circle (  1.16);

\path[draw=drawColor,line width= 0.4pt,line join=round,line cap=round,fill=fillColor] (168.17,569.81) circle (  1.16);

\path[draw=drawColor,line width= 0.4pt,line join=round,line cap=round,fill=fillColor] (168.44,569.80) circle (  1.16);

\path[draw=drawColor,line width= 0.4pt,line join=round,line cap=round,fill=fillColor] (168.71,569.54) circle (  1.16);

\path[draw=drawColor,line width= 0.4pt,line join=round,line cap=round,fill=fillColor] (168.99,569.45) circle (  1.16);

\path[draw=drawColor,line width= 0.4pt,line join=round,line cap=round,fill=fillColor] (169.26,569.19) circle (  1.16);

\path[draw=drawColor,line width= 0.4pt,line join=round,line cap=round,fill=fillColor] (169.53,569.15) circle (  1.16);

\path[draw=drawColor,line width= 0.4pt,line join=round,line cap=round,fill=fillColor] (169.79,568.80) circle (  1.16);

\path[draw=drawColor,line width= 0.4pt,line join=round,line cap=round,fill=fillColor] (170.06,568.66) circle (  1.16);

\path[draw=drawColor,line width= 0.4pt,line join=round,line cap=round,fill=fillColor] (170.33,568.64) circle (  1.16);

\path[draw=drawColor,line width= 0.4pt,line join=round,line cap=round,fill=fillColor] (170.59,568.61) circle (  1.16);

\path[draw=drawColor,line width= 0.4pt,line join=round,line cap=round,fill=fillColor] (170.85,568.42) circle (  1.16);

\path[draw=drawColor,line width= 0.4pt,line join=round,line cap=round,fill=fillColor] (171.12,568.21) circle (  1.16);

\path[draw=drawColor,line width= 0.4pt,line join=round,line cap=round,fill=fillColor] (171.38,568.18) circle (  1.16);

\path[draw=drawColor,line width= 0.4pt,line join=round,line cap=round,fill=fillColor] (171.64,568.06) circle (  1.16);

\path[draw=drawColor,line width= 0.4pt,line join=round,line cap=round,fill=fillColor] (171.90,567.96) circle (  1.16);

\path[draw=drawColor,line width= 0.4pt,line join=round,line cap=round,fill=fillColor] (172.15,567.86) circle (  1.16);

\path[draw=drawColor,line width= 0.4pt,line join=round,line cap=round,fill=fillColor] (172.41,567.82) circle (  1.16);

\path[draw=drawColor,line width= 0.4pt,line join=round,line cap=round,fill=fillColor] (172.67,567.79) circle (  1.16);

\path[draw=drawColor,line width= 0.4pt,line join=round,line cap=round,fill=fillColor] (172.92,567.76) circle (  1.16);

\path[draw=drawColor,line width= 0.4pt,line join=round,line cap=round,fill=fillColor] (173.17,567.65) circle (  1.16);

\path[draw=drawColor,line width= 0.4pt,line join=round,line cap=round,fill=fillColor] (173.43,567.33) circle (  1.16);

\path[draw=drawColor,line width= 0.4pt,line join=round,line cap=round,fill=fillColor] (173.68,567.23) circle (  1.16);

\path[draw=drawColor,line width= 0.4pt,line join=round,line cap=round,fill=fillColor] (173.93,567.21) circle (  1.16);

\path[draw=drawColor,line width= 0.4pt,line join=round,line cap=round,fill=fillColor] (174.18,567.15) circle (  1.16);

\path[draw=drawColor,line width= 0.4pt,line join=round,line cap=round,fill=fillColor] (174.42,567.08) circle (  1.16);

\path[draw=drawColor,line width= 0.4pt,line join=round,line cap=round,fill=fillColor] (174.67,567.03) circle (  1.16);

\path[draw=drawColor,line width= 0.4pt,line join=round,line cap=round,fill=fillColor] (174.92,566.94) circle (  1.16);

\path[draw=drawColor,line width= 0.4pt,line join=round,line cap=round,fill=fillColor] (175.16,566.94) circle (  1.16);

\path[draw=drawColor,line width= 0.4pt,line join=round,line cap=round,fill=fillColor] (175.41,566.94) circle (  1.16);

\path[draw=drawColor,line width= 0.4pt,line join=round,line cap=round,fill=fillColor] (175.65,566.64) circle (  1.16);

\path[draw=drawColor,line width= 0.4pt,line join=round,line cap=round,fill=fillColor] (175.89,566.57) circle (  1.16);

\path[draw=drawColor,line width= 0.4pt,line join=round,line cap=round,fill=fillColor] (176.13,566.31) circle (  1.16);

\path[draw=drawColor,line width= 0.4pt,line join=round,line cap=round,fill=fillColor] (176.37,566.23) circle (  1.16);

\path[draw=drawColor,line width= 0.4pt,line join=round,line cap=round,fill=fillColor] (176.61,566.11) circle (  1.16);

\path[draw=drawColor,line width= 0.4pt,line join=round,line cap=round,fill=fillColor] (176.85,566.05) circle (  1.16);

\path[draw=drawColor,line width= 0.4pt,line join=round,line cap=round,fill=fillColor] (177.09,566.04) circle (  1.16);

\path[draw=drawColor,line width= 0.4pt,line join=round,line cap=round,fill=fillColor] (177.32,565.99) circle (  1.16);

\path[draw=drawColor,line width= 0.4pt,line join=round,line cap=round,fill=fillColor] (177.56,565.84) circle (  1.16);

\path[draw=drawColor,line width= 0.4pt,line join=round,line cap=round,fill=fillColor] (177.80,565.80) circle (  1.16);

\path[draw=drawColor,line width= 0.4pt,line join=round,line cap=round,fill=fillColor] (178.03,565.79) circle (  1.16);

\path[draw=drawColor,line width= 0.4pt,line join=round,line cap=round,fill=fillColor] (178.26,565.64) circle (  1.16);

\path[draw=drawColor,line width= 0.4pt,line join=round,line cap=round,fill=fillColor] (178.49,565.63) circle (  1.16);

\path[draw=drawColor,line width= 0.4pt,line join=round,line cap=round,fill=fillColor] (178.73,565.47) circle (  1.16);

\path[draw=drawColor,line width= 0.4pt,line join=round,line cap=round,fill=fillColor] (178.96,565.43) circle (  1.16);

\path[draw=drawColor,line width= 0.4pt,line join=round,line cap=round,fill=fillColor] (179.19,565.42) circle (  1.16);

\path[draw=drawColor,line width= 0.4pt,line join=round,line cap=round,fill=fillColor] (179.41,565.39) circle (  1.16);

\path[draw=drawColor,line width= 0.4pt,line join=round,line cap=round,fill=fillColor] (179.64,565.38) circle (  1.16);

\path[draw=drawColor,line width= 0.4pt,line join=round,line cap=round,fill=fillColor] (179.87,565.32) circle (  1.16);

\path[draw=drawColor,line width= 0.4pt,line join=round,line cap=round,fill=fillColor] (180.10,565.28) circle (  1.16);

\path[draw=drawColor,line width= 0.4pt,line join=round,line cap=round,fill=fillColor] (180.32,565.22) circle (  1.16);

\path[draw=drawColor,line width= 0.4pt,line join=round,line cap=round,fill=fillColor] (180.55,565.12) circle (  1.16);

\path[draw=drawColor,line width= 0.4pt,line join=round,line cap=round,fill=fillColor] (180.77,565.06) circle (  1.16);

\path[draw=drawColor,line width= 0.4pt,line join=round,line cap=round,fill=fillColor] (180.99,564.99) circle (  1.16);

\path[draw=drawColor,line width= 0.4pt,line join=round,line cap=round,fill=fillColor] (181.22,564.93) circle (  1.16);

\path[draw=drawColor,line width= 0.4pt,line join=round,line cap=round,fill=fillColor] (181.44,564.92) circle (  1.16);

\path[draw=drawColor,line width= 0.4pt,line join=round,line cap=round,fill=fillColor] (181.66,564.88) circle (  1.16);

\path[draw=drawColor,line width= 0.4pt,line join=round,line cap=round,fill=fillColor] (181.88,564.68) circle (  1.16);

\path[draw=drawColor,line width= 0.4pt,line join=round,line cap=round,fill=fillColor] (182.10,564.67) circle (  1.16);

\path[draw=drawColor,line width= 0.4pt,line join=round,line cap=round,fill=fillColor] (182.32,564.57) circle (  1.16);

\path[draw=drawColor,line width= 0.4pt,line join=round,line cap=round,fill=fillColor] (182.54,564.47) circle (  1.16);

\path[draw=drawColor,line width= 0.4pt,line join=round,line cap=round,fill=fillColor] (182.75,564.34) circle (  1.16);

\path[draw=drawColor,line width= 0.4pt,line join=round,line cap=round,fill=fillColor] (182.97,564.33) circle (  1.16);

\path[draw=drawColor,line width= 0.4pt,line join=round,line cap=round,fill=fillColor] (183.19,564.16) circle (  1.16);

\path[draw=drawColor,line width= 0.4pt,line join=round,line cap=round,fill=fillColor] (183.40,564.15) circle (  1.16);

\path[draw=drawColor,line width= 0.4pt,line join=round,line cap=round,fill=fillColor] (183.61,564.07) circle (  1.16);

\path[draw=drawColor,line width= 0.4pt,line join=round,line cap=round,fill=fillColor] (183.83,564.02) circle (  1.16);

\path[draw=drawColor,line width= 0.4pt,line join=round,line cap=round,fill=fillColor] (184.04,563.60) circle (  1.16);

\path[draw=drawColor,line width= 0.4pt,line join=round,line cap=round,fill=fillColor] (184.25,563.57) circle (  1.16);

\path[draw=drawColor,line width= 0.4pt,line join=round,line cap=round,fill=fillColor] (184.47,563.46) circle (  1.16);

\path[draw=drawColor,line width= 0.4pt,line join=round,line cap=round,fill=fillColor] (184.68,563.44) circle (  1.16);

\path[draw=drawColor,line width= 0.4pt,line join=round,line cap=round,fill=fillColor] (184.89,563.37) circle (  1.16);

\path[draw=drawColor,line width= 0.4pt,line join=round,line cap=round,fill=fillColor] (185.10,563.35) circle (  1.16);

\path[draw=drawColor,line width= 0.4pt,line join=round,line cap=round,fill=fillColor] (185.31,563.16) circle (  1.16);

\path[draw=drawColor,line width= 0.4pt,line join=round,line cap=round,fill=fillColor] (185.51,563.09) circle (  1.16);

\path[draw=drawColor,line width= 0.4pt,line join=round,line cap=round,fill=fillColor] (185.72,563.07) circle (  1.16);

\path[draw=drawColor,line width= 0.4pt,line join=round,line cap=round,fill=fillColor] (185.93,563.02) circle (  1.16);

\path[draw=drawColor,line width= 0.4pt,line join=round,line cap=round,fill=fillColor] (186.13,563.01) circle (  1.16);

\path[draw=drawColor,line width= 0.4pt,line join=round,line cap=round,fill=fillColor] (186.34,562.95) circle (  1.16);

\path[draw=drawColor,line width= 0.4pt,line join=round,line cap=round,fill=fillColor] (186.55,562.81) circle (  1.16);

\path[draw=drawColor,line width= 0.4pt,line join=round,line cap=round,fill=fillColor] (186.75,562.67) circle (  1.16);

\path[draw=drawColor,line width= 0.4pt,line join=round,line cap=round,fill=fillColor] (186.95,562.61) circle (  1.16);

\path[draw=drawColor,line width= 0.4pt,line join=round,line cap=round,fill=fillColor] (187.16,562.57) circle (  1.16);

\path[draw=drawColor,line width= 0.4pt,line join=round,line cap=round,fill=fillColor] (187.36,562.56) circle (  1.16);

\path[draw=drawColor,line width= 0.4pt,line join=round,line cap=round,fill=fillColor] (187.56,562.44) circle (  1.16);

\path[draw=drawColor,line width= 0.4pt,line join=round,line cap=round,fill=fillColor] (187.76,562.37) circle (  1.16);

\path[draw=drawColor,line width= 0.4pt,line join=round,line cap=round,fill=fillColor] (187.96,562.15) circle (  1.16);

\path[draw=drawColor,line width= 0.4pt,line join=round,line cap=round,fill=fillColor] (188.16,562.01) circle (  1.16);

\path[draw=drawColor,line width= 0.4pt,line join=round,line cap=round,fill=fillColor] (188.36,561.84) circle (  1.16);

\path[draw=drawColor,line width= 0.4pt,line join=round,line cap=round,fill=fillColor] (188.56,561.73) circle (  1.16);

\path[draw=drawColor,line width= 0.4pt,line join=round,line cap=round,fill=fillColor] (188.76,561.58) circle (  1.16);

\path[draw=drawColor,line width= 0.4pt,line join=round,line cap=round,fill=fillColor] (188.96,561.51) circle (  1.16);

\path[draw=drawColor,line width= 0.4pt,line join=round,line cap=round,fill=fillColor] (189.16,561.44) circle (  1.16);

\path[draw=drawColor,line width= 0.4pt,line join=round,line cap=round,fill=fillColor] (189.35,561.42) circle (  1.16);

\path[draw=drawColor,line width= 0.4pt,line join=round,line cap=round,fill=fillColor] (189.55,561.40) circle (  1.16);

\path[draw=drawColor,line width= 0.4pt,line join=round,line cap=round,fill=fillColor] (189.74,561.34) circle (  1.16);

\path[draw=drawColor,line width= 0.4pt,line join=round,line cap=round,fill=fillColor] (189.94,561.29) circle (  1.16);

\path[draw=drawColor,line width= 0.4pt,line join=round,line cap=round,fill=fillColor] (190.13,561.29) circle (  1.16);

\path[draw=drawColor,line width= 0.4pt,line join=round,line cap=round,fill=fillColor] (190.33,561.26) circle (  1.16);

\path[draw=drawColor,line width= 0.4pt,line join=round,line cap=round,fill=fillColor] (190.52,561.26) circle (  1.16);

\path[draw=drawColor,line width= 0.4pt,line join=round,line cap=round,fill=fillColor] (190.71,561.25) circle (  1.16);

\path[draw=drawColor,line width= 0.4pt,line join=round,line cap=round,fill=fillColor] (190.91,561.22) circle (  1.16);

\path[draw=drawColor,line width= 0.4pt,line join=round,line cap=round,fill=fillColor] (191.10,561.01) circle (  1.16);

\path[draw=drawColor,line width= 0.4pt,line join=round,line cap=round,fill=fillColor] (191.29,560.84) circle (  1.16);

\path[draw=drawColor,line width= 0.4pt,line join=round,line cap=round,fill=fillColor] (191.48,560.73) circle (  1.16);

\path[draw=drawColor,line width= 0.4pt,line join=round,line cap=round,fill=fillColor] (191.67,560.67) circle (  1.16);

\path[draw=drawColor,line width= 0.4pt,line join=round,line cap=round,fill=fillColor] (191.86,560.54) circle (  1.16);

\path[draw=drawColor,line width= 0.4pt,line join=round,line cap=round,fill=fillColor] (192.05,560.54) circle (  1.16);

\path[draw=drawColor,line width= 0.4pt,line join=round,line cap=round,fill=fillColor] (192.24,560.49) circle (  1.16);

\path[draw=drawColor,line width= 0.4pt,line join=round,line cap=round,fill=fillColor] (192.43,560.47) circle (  1.16);

\path[draw=drawColor,line width= 0.4pt,line join=round,line cap=round,fill=fillColor] (192.61,560.45) circle (  1.16);

\path[draw=drawColor,line width= 0.4pt,line join=round,line cap=round,fill=fillColor] (192.80,560.36) circle (  1.16);

\path[draw=drawColor,line width= 0.4pt,line join=round,line cap=round,fill=fillColor] (192.99,560.19) circle (  1.16);

\path[draw=drawColor,line width= 0.4pt,line join=round,line cap=round,fill=fillColor] (193.17,560.03) circle (  1.16);

\path[draw=drawColor,line width= 0.4pt,line join=round,line cap=round,fill=fillColor] (193.36,560.00) circle (  1.16);

\path[draw=drawColor,line width= 0.4pt,line join=round,line cap=round,fill=fillColor] (193.54,559.38) circle (  1.16);

\path[draw=drawColor,line width= 0.4pt,line join=round,line cap=round,fill=fillColor] (193.73,559.25) circle (  1.16);

\path[draw=drawColor,line width= 0.4pt,line join=round,line cap=round,fill=fillColor] (193.91,559.19) circle (  1.16);

\path[draw=drawColor,line width= 0.4pt,line join=round,line cap=round,fill=fillColor] (194.10,559.11) circle (  1.16);

\path[draw=drawColor,line width= 0.4pt,line join=round,line cap=round,fill=fillColor] (194.28,559.03) circle (  1.16);

\path[draw=drawColor,line width= 0.4pt,line join=round,line cap=round,fill=fillColor] (194.46,558.79) circle (  1.16);

\path[draw=drawColor,line width= 0.4pt,line join=round,line cap=round,fill=fillColor] (194.65,558.73) circle (  1.16);

\path[draw=drawColor,line width= 0.4pt,line join=round,line cap=round,fill=fillColor] (194.83,558.71) circle (  1.16);

\path[draw=drawColor,line width= 0.4pt,line join=round,line cap=round,fill=fillColor] (195.01,558.67) circle (  1.16);

\path[draw=drawColor,line width= 0.4pt,line join=round,line cap=round,fill=fillColor] (195.19,558.64) circle (  1.16);

\path[draw=drawColor,line width= 0.4pt,line join=round,line cap=round,fill=fillColor] (195.37,558.56) circle (  1.16);

\path[draw=drawColor,line width= 0.4pt,line join=round,line cap=round,fill=fillColor] (195.55,558.54) circle (  1.16);

\path[draw=drawColor,line width= 0.4pt,line join=round,line cap=round,fill=fillColor] (195.73,558.35) circle (  1.16);

\path[draw=drawColor,line width= 0.4pt,line join=round,line cap=round,fill=fillColor] (195.91,558.35) circle (  1.16);

\path[draw=drawColor,line width= 0.4pt,line join=round,line cap=round,fill=fillColor] (196.09,558.04) circle (  1.16);

\path[draw=drawColor,line width= 0.4pt,line join=round,line cap=round,fill=fillColor] (196.27,557.92) circle (  1.16);

\path[draw=drawColor,line width= 0.4pt,line join=round,line cap=round,fill=fillColor] (196.44,557.78) circle (  1.16);

\path[draw=drawColor,line width= 0.4pt,line join=round,line cap=round,fill=fillColor] (196.62,557.74) circle (  1.16);

\path[draw=drawColor,line width= 0.4pt,line join=round,line cap=round,fill=fillColor] (196.80,557.72) circle (  1.16);

\path[draw=drawColor,line width= 0.4pt,line join=round,line cap=round,fill=fillColor] (196.97,557.55) circle (  1.16);

\path[draw=drawColor,line width= 0.4pt,line join=round,line cap=round,fill=fillColor] (197.15,557.49) circle (  1.16);

\path[draw=drawColor,line width= 0.4pt,line join=round,line cap=round,fill=fillColor] (197.33,557.11) circle (  1.16);

\path[draw=drawColor,line width= 0.4pt,line join=round,line cap=round,fill=fillColor] (197.50,557.01) circle (  1.16);

\path[draw=drawColor,line width= 0.4pt,line join=round,line cap=round,fill=fillColor] (197.68,556.84) circle (  1.16);

\path[draw=drawColor,line width= 0.4pt,line join=round,line cap=round,fill=fillColor] (197.85,556.67) circle (  1.16);

\path[draw=drawColor,line width= 0.4pt,line join=round,line cap=round,fill=fillColor] (198.02,556.66) circle (  1.16);

\path[draw=drawColor,line width= 0.4pt,line join=round,line cap=round,fill=fillColor] (198.20,556.59) circle (  1.16);

\path[draw=drawColor,line width= 0.4pt,line join=round,line cap=round,fill=fillColor] (198.37,556.58) circle (  1.16);

\path[draw=drawColor,line width= 0.4pt,line join=round,line cap=round,fill=fillColor] (198.54,556.40) circle (  1.16);

\path[draw=drawColor,line width= 0.4pt,line join=round,line cap=round,fill=fillColor] (198.72,556.28) circle (  1.16);

\path[draw=drawColor,line width= 0.4pt,line join=round,line cap=round,fill=fillColor] (198.89,556.19) circle (  1.16);

\path[draw=drawColor,line width= 0.4pt,line join=round,line cap=round,fill=fillColor] (199.06,555.96) circle (  1.16);

\path[draw=drawColor,line width= 0.4pt,line join=round,line cap=round,fill=fillColor] (199.23,555.88) circle (  1.16);

\path[draw=drawColor,line width= 0.4pt,line join=round,line cap=round,fill=fillColor] (199.40,555.77) circle (  1.16);

\path[draw=drawColor,line width= 0.4pt,line join=round,line cap=round,fill=fillColor] (199.57,555.57) circle (  1.16);

\path[draw=drawColor,line width= 0.4pt,line join=round,line cap=round,fill=fillColor] (199.74,555.42) circle (  1.16);

\path[draw=drawColor,line width= 0.4pt,line join=round,line cap=round,fill=fillColor] (199.91,555.35) circle (  1.16);

\path[draw=drawColor,line width= 0.4pt,line join=round,line cap=round,fill=fillColor] (200.08,555.23) circle (  1.16);

\path[draw=drawColor,line width= 0.4pt,line join=round,line cap=round,fill=fillColor] (200.25,555.17) circle (  1.16);

\path[draw=drawColor,line width= 0.4pt,line join=round,line cap=round,fill=fillColor] (200.42,555.16) circle (  1.16);

\path[draw=drawColor,line width= 0.4pt,line join=round,line cap=round,fill=fillColor] (200.58,555.06) circle (  1.16);

\path[draw=drawColor,line width= 0.4pt,line join=round,line cap=round,fill=fillColor] (200.75,554.99) circle (  1.16);

\path[draw=drawColor,line width= 0.4pt,line join=round,line cap=round,fill=fillColor] (200.92,554.97) circle (  1.16);

\path[draw=drawColor,line width= 0.4pt,line join=round,line cap=round,fill=fillColor] (201.09,554.60) circle (  1.16);

\path[draw=drawColor,line width= 0.4pt,line join=round,line cap=round,fill=fillColor] (201.25,554.59) circle (  1.16);

\path[draw=drawColor,line width= 0.4pt,line join=round,line cap=round,fill=fillColor] (201.42,554.44) circle (  1.16);

\path[draw=drawColor,line width= 0.4pt,line join=round,line cap=round,fill=fillColor] (201.58,554.43) circle (  1.16);

\path[draw=drawColor,line width= 0.4pt,line join=round,line cap=round,fill=fillColor] (201.75,554.39) circle (  1.16);

\path[draw=drawColor,line width= 0.4pt,line join=round,line cap=round,fill=fillColor] (201.91,554.39) circle (  1.16);

\path[draw=drawColor,line width= 0.4pt,line join=round,line cap=round,fill=fillColor] (202.08,554.21) circle (  1.16);

\path[draw=drawColor,line width= 0.4pt,line join=round,line cap=round,fill=fillColor] (202.24,554.07) circle (  1.16);

\path[draw=drawColor,line width= 0.4pt,line join=round,line cap=round,fill=fillColor] (202.41,553.96) circle (  1.16);

\path[draw=drawColor,line width= 0.4pt,line join=round,line cap=round,fill=fillColor] (202.57,553.44) circle (  1.16);

\path[draw=drawColor,line width= 0.4pt,line join=round,line cap=round,fill=fillColor] (202.73,553.44) circle (  1.16);

\path[draw=drawColor,line width= 0.4pt,line join=round,line cap=round,fill=fillColor] (202.90,553.43) circle (  1.16);

\path[draw=drawColor,line width= 0.4pt,line join=round,line cap=round,fill=fillColor] (203.06,553.30) circle (  1.16);

\path[draw=drawColor,line width= 0.4pt,line join=round,line cap=round,fill=fillColor] (203.22,553.03) circle (  1.16);

\path[draw=drawColor,line width= 0.4pt,line join=round,line cap=round,fill=fillColor] (203.38,552.81) circle (  1.16);

\path[draw=drawColor,line width= 0.4pt,line join=round,line cap=round,fill=fillColor] (203.54,552.71) circle (  1.16);

\path[draw=drawColor,line width= 0.4pt,line join=round,line cap=round,fill=fillColor] (203.71,552.64) circle (  1.16);

\path[draw=drawColor,line width= 0.4pt,line join=round,line cap=round,fill=fillColor] (203.87,552.64) circle (  1.16);

\path[draw=drawColor,line width= 0.4pt,line join=round,line cap=round,fill=fillColor] (204.03,552.64) circle (  1.16);

\path[draw=drawColor,line width= 0.4pt,line join=round,line cap=round,fill=fillColor] (204.19,552.43) circle (  1.16);

\path[draw=drawColor,line width= 0.4pt,line join=round,line cap=round,fill=fillColor] (204.35,552.29) circle (  1.16);

\path[draw=drawColor,line width= 0.4pt,line join=round,line cap=round,fill=fillColor] (204.51,551.97) circle (  1.16);

\path[draw=drawColor,line width= 0.4pt,line join=round,line cap=round,fill=fillColor] (204.66,551.97) circle (  1.16);

\path[draw=drawColor,line width= 0.4pt,line join=round,line cap=round,fill=fillColor] (204.82,551.97) circle (  1.16);

\path[draw=drawColor,line width= 0.4pt,line join=round,line cap=round,fill=fillColor] (204.98,551.70) circle (  1.16);

\path[draw=drawColor,line width= 0.4pt,line join=round,line cap=round,fill=fillColor] (205.14,551.26) circle (  1.16);

\path[draw=drawColor,line width= 0.4pt,line join=round,line cap=round,fill=fillColor] (205.30,551.11) circle (  1.16);

\path[draw=drawColor,line width= 0.4pt,line join=round,line cap=round,fill=fillColor] (205.45,551.04) circle (  1.16);

\path[draw=drawColor,line width= 0.4pt,line join=round,line cap=round,fill=fillColor] (205.61,550.93) circle (  1.16);

\path[draw=drawColor,line width= 0.4pt,line join=round,line cap=round,fill=fillColor] (205.77,550.80) circle (  1.16);

\path[draw=drawColor,line width= 0.4pt,line join=round,line cap=round,fill=fillColor] (205.92,550.75) circle (  1.16);

\path[draw=drawColor,line width= 0.4pt,line join=round,line cap=round,fill=fillColor] (206.08,550.63) circle (  1.16);

\path[draw=drawColor,line width= 0.4pt,line join=round,line cap=round,fill=fillColor] (206.24,550.43) circle (  1.16);

\path[draw=drawColor,line width= 0.4pt,line join=round,line cap=round,fill=fillColor] (206.39,550.43) circle (  1.16);

\path[draw=drawColor,line width= 0.4pt,line join=round,line cap=round,fill=fillColor] (206.55,550.35) circle (  1.16);

\path[draw=drawColor,line width= 0.4pt,line join=round,line cap=round,fill=fillColor] (206.70,550.34) circle (  1.16);

\path[draw=drawColor,line width= 0.4pt,line join=round,line cap=round,fill=fillColor] (206.86,550.08) circle (  1.16);

\path[draw=drawColor,line width= 0.4pt,line join=round,line cap=round,fill=fillColor] (207.01,550.06) circle (  1.16);

\path[draw=drawColor,line width= 0.4pt,line join=round,line cap=round,fill=fillColor] (207.16,549.97) circle (  1.16);

\path[draw=drawColor,line width= 0.4pt,line join=round,line cap=round,fill=fillColor] (207.32,549.41) circle (  1.16);

\path[draw=drawColor,line width= 0.4pt,line join=round,line cap=round,fill=fillColor] (207.47,549.33) circle (  1.16);

\path[draw=drawColor,line width= 0.4pt,line join=round,line cap=round,fill=fillColor] (207.62,549.33) circle (  1.16);

\path[draw=drawColor,line width= 0.4pt,line join=round,line cap=round,fill=fillColor] (207.78,549.33) circle (  1.16);

\path[draw=drawColor,line width= 0.4pt,line join=round,line cap=round,fill=fillColor] (207.93,549.33) circle (  1.16);

\path[draw=drawColor,line width= 0.4pt,line join=round,line cap=round,fill=fillColor] (208.08,549.18) circle (  1.16);

\path[draw=drawColor,line width= 0.4pt,line join=round,line cap=round,fill=fillColor] (208.23,548.32) circle (  1.16);

\path[draw=drawColor,line width= 0.4pt,line join=round,line cap=round,fill=fillColor] (208.39,547.53) circle (  1.16);

\path[draw=drawColor,line width= 0.4pt,line join=round,line cap=round,fill=fillColor] (208.54,547.53) circle (  1.16);

\path[draw=drawColor,line width= 0.4pt,line join=round,line cap=round,fill=fillColor] (208.69,547.37) circle (  1.16);

\path[draw=drawColor,line width= 0.4pt,line join=round,line cap=round,fill=fillColor] (208.84,547.36) circle (  1.16);

\path[draw=drawColor,line width= 0.4pt,line join=round,line cap=round,fill=fillColor] (208.99,546.96) circle (  1.16);

\path[draw=drawColor,line width= 0.4pt,line join=round,line cap=round,fill=fillColor] (209.14,546.60) circle (  1.16);

\path[draw=drawColor,line width= 0.4pt,line join=round,line cap=round,fill=fillColor] (209.29,546.39) circle (  1.16);

\path[draw=drawColor,line width= 0.4pt,line join=round,line cap=round,fill=fillColor] (209.44,546.37) circle (  1.16);

\path[draw=drawColor,line width= 0.4pt,line join=round,line cap=round,fill=fillColor] (209.59,546.15) circle (  1.16);

\path[draw=drawColor,line width= 0.4pt,line join=round,line cap=round,fill=fillColor] (209.74,545.88) circle (  1.16);

\path[draw=drawColor,line width= 0.4pt,line join=round,line cap=round,fill=fillColor] (209.88,545.46) circle (  1.16);

\path[draw=drawColor,line width= 0.4pt,line join=round,line cap=round,fill=fillColor] (210.03,545.46) circle (  1.16);

\path[draw=drawColor,line width= 0.4pt,line join=round,line cap=round,fill=fillColor] (210.18,545.45) circle (  1.16);

\path[draw=drawColor,line width= 0.4pt,line join=round,line cap=round,fill=fillColor] (210.33,544.94) circle (  1.16);

\path[draw=drawColor,line width= 0.4pt,line join=round,line cap=round,fill=fillColor] (210.48,544.78) circle (  1.16);

\path[draw=drawColor,line width= 0.4pt,line join=round,line cap=round,fill=fillColor] (210.62,544.38) circle (  1.16);

\path[draw=drawColor,line width= 0.4pt,line join=round,line cap=round,fill=fillColor] (210.77,543.30) circle (  1.16);

\path[draw=drawColor,line width= 0.4pt,line join=round,line cap=round,fill=fillColor] (210.92,542.89) circle (  1.16);

\path[draw=drawColor,line width= 0.4pt,line join=round,line cap=round,fill=fillColor] (211.06,542.80) circle (  1.16);

\path[draw=drawColor,line width= 0.4pt,line join=round,line cap=round,fill=fillColor] (211.21,541.51) circle (  1.16);

\path[draw=drawColor,line width= 0.4pt,line join=round,line cap=round,fill=fillColor] (211.36,535.03) circle (  1.16);

\path[draw=drawColor,line width= 0.4pt,line join=round,line cap=round,fill=fillColor] (211.50,535.03) circle (  1.16);

\path[draw=drawColor,line width= 0.4pt,line join=round,line cap=round,fill=fillColor] (211.65,535.03) circle (  1.16);

\path[draw=drawColor,line width= 0.4pt,line join=round,line cap=round,fill=fillColor] (211.79,535.03) circle (  1.16);

\path[draw=drawColor,line width= 0.4pt,line join=round,line cap=round,fill=fillColor] (211.94,535.03) circle (  1.16);

\path[draw=drawColor,line width= 0.4pt,line join=round,line cap=round,fill=fillColor] (212.08,535.03) circle (  1.16);

\path[draw=drawColor,line width= 0.4pt,line join=round,line cap=round,fill=fillColor] (212.23,535.03) circle (  1.16);

\path[draw=drawColor,line width= 0.4pt,line join=round,line cap=round,fill=fillColor] (212.37,535.03) circle (  1.16);

\path[draw=drawColor,line width= 0.4pt,line join=round,line cap=round,fill=fillColor] (212.51,535.03) circle (  1.16);

\path[draw=drawColor,line width= 0.4pt,line join=round,line cap=round,fill=fillColor] (212.66,535.03) circle (  1.16);

\path[draw=drawColor,line width= 0.4pt,line join=round,line cap=round,fill=fillColor] (212.80,535.03) circle (  1.16);

\path[draw=drawColor,line width= 0.4pt,line join=round,line cap=round,fill=fillColor] (212.94,535.03) circle (  1.16);

\path[draw=drawColor,line width= 0.4pt,line join=round,line cap=round,fill=fillColor] (213.09,535.03) circle (  1.16);

\path[draw=drawColor,line width= 0.4pt,line join=round,line cap=round,fill=fillColor] (213.23,535.03) circle (  1.16);

\path[draw=drawColor,line width= 0.4pt,line join=round,line cap=round,fill=fillColor] (213.37,535.03) circle (  1.16);

\path[draw=drawColor,line width= 0.4pt,line join=round,line cap=round,fill=fillColor] (213.51,535.03) circle (  1.16);

\path[draw=drawColor,line width= 0.4pt,line join=round,line cap=round,fill=fillColor] (213.65,535.03) circle (  1.16);

\path[draw=drawColor,line width= 0.4pt,line join=round,line cap=round,fill=fillColor] (213.80,535.03) circle (  1.16);

\path[draw=drawColor,line width= 0.4pt,line join=round,line cap=round,fill=fillColor] (213.94,535.03) circle (  1.16);

\path[draw=drawColor,line width= 0.4pt,line join=round,line cap=round,fill=fillColor] (214.08,535.03) circle (  1.16);

\path[draw=drawColor,line width= 0.4pt,line join=round,line cap=round,fill=fillColor] (214.22,535.03) circle (  1.16);

\path[draw=drawColor,line width= 0.4pt,line join=round,line cap=round,fill=fillColor] (214.36,535.03) circle (  1.16);

\path[draw=drawColor,line width= 0.4pt,line join=round,line cap=round,fill=fillColor] (214.50,535.03) circle (  1.16);

\path[draw=drawColor,line width= 0.4pt,line join=round,line cap=round,fill=fillColor] (214.64,535.03) circle (  1.16);

\path[draw=drawColor,line width= 0.4pt,line join=round,line cap=round,fill=fillColor] (214.78,535.03) circle (  1.16);

\path[draw=drawColor,line width= 0.4pt,line join=round,line cap=round,fill=fillColor] (214.92,535.03) circle (  1.16);

\path[draw=drawColor,line width= 0.4pt,line join=round,line cap=round,fill=fillColor] (215.06,535.03) circle (  1.16);

\path[draw=drawColor,line width= 0.4pt,line join=round,line cap=round,fill=fillColor] (215.20,535.03) circle (  1.16);

\path[draw=drawColor,line width= 0.4pt,line join=round,line cap=round,fill=fillColor] (215.34,535.03) circle (  1.16);

\path[draw=drawColor,line width= 0.4pt,line join=round,line cap=round,fill=fillColor] (215.47,535.03) circle (  1.16);

\path[draw=drawColor,line width= 0.4pt,line join=round,line cap=round,fill=fillColor] (215.61,535.03) circle (  1.16);

\path[draw=drawColor,line width= 0.4pt,line join=round,line cap=round,fill=fillColor] (215.75,535.03) circle (  1.16);

\path[draw=drawColor,line width= 0.4pt,line join=round,line cap=round,fill=fillColor] (215.89,535.03) circle (  1.16);

\path[draw=drawColor,line width= 0.4pt,line join=round,line cap=round,fill=fillColor] (216.03,535.03) circle (  1.16);

\path[draw=drawColor,line width= 0.4pt,line join=round,line cap=round,fill=fillColor] (216.16,535.03) circle (  1.16);

\path[draw=drawColor,line width= 0.4pt,line join=round,line cap=round,fill=fillColor] (216.30,535.03) circle (  1.16);

\path[draw=drawColor,line width= 0.4pt,line join=round,line cap=round,fill=fillColor] (216.44,535.03) circle (  1.16);

\path[draw=drawColor,line width= 0.4pt,line join=round,line cap=round,fill=fillColor] (216.58,535.03) circle (  1.16);

\path[draw=drawColor,line width= 0.4pt,line join=round,line cap=round,fill=fillColor] (216.71,535.03) circle (  1.16);

\path[draw=drawColor,line width= 0.4pt,line join=round,line cap=round,fill=fillColor] (216.85,535.03) circle (  1.16);

\path[draw=drawColor,line width= 0.4pt,line join=round,line cap=round,fill=fillColor] (216.98,535.03) circle (  1.16);

\path[draw=drawColor,line width= 0.4pt,line join=round,line cap=round,fill=fillColor] (217.12,535.03) circle (  1.16);

\path[draw=drawColor,line width= 0.4pt,line join=round,line cap=round,fill=fillColor] (217.26,535.03) circle (  1.16);

\path[draw=drawColor,line width= 0.4pt,line join=round,line cap=round,fill=fillColor] (217.39,535.03) circle (  1.16);

\path[draw=drawColor,line width= 0.4pt,line join=round,line cap=round,fill=fillColor] (217.53,535.03) circle (  1.16);

\path[draw=drawColor,line width= 0.4pt,line join=round,line cap=round,fill=fillColor] (217.66,535.03) circle (  1.16);

\path[draw=drawColor,line width= 0.4pt,line join=round,line cap=round,fill=fillColor] (217.80,535.03) circle (  1.16);

\path[draw=drawColor,line width= 0.4pt,line join=round,line cap=round,fill=fillColor] (217.93,535.03) circle (  1.16);

\path[draw=drawColor,line width= 0.4pt,line join=round,line cap=round,fill=fillColor] (218.06,535.03) circle (  1.16);

\path[draw=drawColor,line width= 0.4pt,line join=round,line cap=round,fill=fillColor] (218.20,535.03) circle (  1.16);

\path[draw=drawColor,line width= 0.4pt,line join=round,line cap=round,fill=fillColor] (218.33,535.03) circle (  1.16);

\path[draw=drawColor,line width= 0.4pt,line join=round,line cap=round,fill=fillColor] (218.47,535.03) circle (  1.16);

\path[draw=drawColor,line width= 0.4pt,line join=round,line cap=round,fill=fillColor] (218.60,535.03) circle (  1.16);

\path[draw=drawColor,line width= 0.4pt,line join=round,line cap=round,fill=fillColor] (218.73,535.03) circle (  1.16);

\path[draw=drawColor,line width= 0.4pt,line join=round,line cap=round,fill=fillColor] (218.87,535.03) circle (  1.16);

\path[draw=drawColor,line width= 0.4pt,line join=round,line cap=round,fill=fillColor] (219.00,535.03) circle (  1.16);

\path[draw=drawColor,line width= 0.4pt,line join=round,line cap=round,fill=fillColor] (219.13,535.03) circle (  1.16);

\path[draw=drawColor,line width= 0.4pt,line join=round,line cap=round,fill=fillColor] (219.26,535.03) circle (  1.16);

\path[draw=drawColor,line width= 0.4pt,line join=round,line cap=round,fill=fillColor] (219.40,535.03) circle (  1.16);

\path[draw=drawColor,line width= 0.4pt,line join=round,line cap=round,fill=fillColor] (219.53,535.03) circle (  1.16);

\path[draw=drawColor,line width= 0.4pt,line join=round,line cap=round,fill=fillColor] (219.66,535.03) circle (  1.16);

\path[draw=drawColor,line width= 0.4pt,line join=round,line cap=round,fill=fillColor] (219.79,535.03) circle (  1.16);

\path[draw=drawColor,line width= 0.4pt,line join=round,line cap=round,fill=fillColor] (219.92,535.03) circle (  1.16);

\path[draw=drawColor,line width= 0.4pt,line join=round,line cap=round,fill=fillColor] (220.05,535.03) circle (  1.16);

\path[draw=drawColor,line width= 0.4pt,line join=round,line cap=round,fill=fillColor] (220.18,535.03) circle (  1.16);

\path[draw=drawColor,line width= 0.4pt,line join=round,line cap=round,fill=fillColor] (220.31,535.03) circle (  1.16);

\path[draw=drawColor,line width= 0.4pt,line join=round,line cap=round,fill=fillColor] (220.45,535.03) circle (  1.16);

\path[draw=drawColor,line width= 0.4pt,line join=round,line cap=round,fill=fillColor] (220.58,535.03) circle (  1.16);

\path[draw=drawColor,line width= 0.4pt,line join=round,line cap=round,fill=fillColor] (220.71,535.03) circle (  1.16);

\path[draw=drawColor,line width= 0.4pt,line join=round,line cap=round,fill=fillColor] (220.84,535.03) circle (  1.16);

\path[draw=drawColor,line width= 0.4pt,line join=round,line cap=round,fill=fillColor] (220.97,535.03) circle (  1.16);

\path[draw=drawColor,line width= 0.4pt,line join=round,line cap=round,fill=fillColor] (221.09,535.03) circle (  1.16);

\path[draw=drawColor,line width= 0.4pt,line join=round,line cap=round,fill=fillColor] (221.22,535.03) circle (  1.16);

\path[draw=drawColor,line width= 0.4pt,line join=round,line cap=round,fill=fillColor] (221.35,535.03) circle (  1.16);

\path[draw=drawColor,line width= 0.4pt,line join=round,line cap=round,fill=fillColor] (221.48,535.03) circle (  1.16);

\path[draw=drawColor,line width= 0.4pt,line join=round,line cap=round,fill=fillColor] (221.61,535.03) circle (  1.16);

\path[draw=drawColor,line width= 0.4pt,line join=round,line cap=round,fill=fillColor] (221.74,535.03) circle (  1.16);

\path[draw=drawColor,line width= 0.4pt,line join=round,line cap=round,fill=fillColor] (221.87,535.03) circle (  1.16);

\path[draw=drawColor,line width= 0.4pt,line join=round,line cap=round,fill=fillColor] (222.00,535.03) circle (  1.16);

\path[draw=drawColor,line width= 0.4pt,line join=round,line cap=round,fill=fillColor] (222.12,535.03) circle (  1.16);

\path[draw=drawColor,line width= 0.4pt,line join=round,line cap=round,fill=fillColor] (222.25,535.03) circle (  1.16);

\path[draw=drawColor,line width= 0.4pt,line join=round,line cap=round,fill=fillColor] (222.38,535.03) circle (  1.16);

\path[draw=drawColor,line width= 0.4pt,line join=round,line cap=round,fill=fillColor] (222.51,535.03) circle (  1.16);

\path[draw=drawColor,line width= 0.4pt,line join=round,line cap=round,fill=fillColor] (222.63,535.03) circle (  1.16);

\path[draw=drawColor,line width= 0.4pt,line join=round,line cap=round,fill=fillColor] (222.76,535.03) circle (  1.16);

\path[draw=drawColor,line width= 0.4pt,line join=round,line cap=round,fill=fillColor] (222.89,535.03) circle (  1.16);

\path[draw=drawColor,line width= 0.4pt,line join=round,line cap=round,fill=fillColor] (223.01,535.03) circle (  1.16);

\path[draw=drawColor,line width= 0.4pt,line join=round,line cap=round,fill=fillColor] (223.14,535.03) circle (  1.16);

\path[draw=drawColor,line width= 0.4pt,line join=round,line cap=round,fill=fillColor] (223.27,535.03) circle (  1.16);

\path[draw=drawColor,line width= 0.4pt,line join=round,line cap=round,fill=fillColor] (223.39,535.03) circle (  1.16);

\path[draw=drawColor,line width= 0.4pt,line join=round,line cap=round,fill=fillColor] (223.52,535.03) circle (  1.16);

\path[draw=drawColor,line width= 0.4pt,line join=round,line cap=round,fill=fillColor] (223.64,535.03) circle (  1.16);

\path[draw=drawColor,line width= 0.4pt,line join=round,line cap=round,fill=fillColor] (223.77,535.03) circle (  1.16);

\path[draw=drawColor,line width= 0.4pt,line join=round,line cap=round,fill=fillColor] (223.89,535.03) circle (  1.16);

\path[draw=drawColor,line width= 0.4pt,line join=round,line cap=round,fill=fillColor] (224.02,535.03) circle (  1.16);

\path[draw=drawColor,line width= 0.4pt,line join=round,line cap=round,fill=fillColor] (224.14,535.03) circle (  1.16);

\path[draw=drawColor,line width= 0.4pt,line join=round,line cap=round,fill=fillColor] (224.27,535.03) circle (  1.16);

\path[draw=drawColor,line width= 0.4pt,line join=round,line cap=round,fill=fillColor] (224.39,535.03) circle (  1.16);

\path[draw=drawColor,line width= 0.4pt,line join=round,line cap=round,fill=fillColor] (224.52,535.03) circle (  1.16);

\path[draw=drawColor,line width= 0.4pt,line join=round,line cap=round,fill=fillColor] (224.64,535.03) circle (  1.16);

\path[draw=drawColor,line width= 0.4pt,line join=round,line cap=round,fill=fillColor] (224.77,535.03) circle (  1.16);

\path[draw=drawColor,line width= 0.4pt,line join=round,line cap=round,fill=fillColor] (224.89,535.03) circle (  1.16);

\path[draw=drawColor,line width= 0.4pt,line join=round,line cap=round,fill=fillColor] (225.01,535.03) circle (  1.16);

\path[draw=drawColor,line width= 0.4pt,line join=round,line cap=round,fill=fillColor] (225.14,535.03) circle (  1.16);

\path[draw=drawColor,line width= 0.4pt,line join=round,line cap=round,fill=fillColor] (225.26,535.03) circle (  1.16);
\definecolor[named]{drawColor}{rgb}{0.00,0.00,0.00}
\definecolor[named]{fillColor}{rgb}{0.00,0.00,0.00}

\path[draw=drawColor,line width= 0.6pt,line join=round,fill=fillColor] ( 67.36,617.99) -- (232.78,617.99);

\node[text=drawColor,anchor=base east,inner sep=0pt, outer sep=0pt, scale=  0.85] at (229.28,620.15) {infeasible solutions};

\path[draw=drawColor,line width= 0.6pt,line join=round,line cap=round] ( 67.36,526.73) rectangle (232.78,628.97);
\end{scope}
\begin{scope}
\path[clip] (  0.00,  0.00) rectangle (505.89,650.43);
\definecolor[named]{drawColor}{rgb}{0.00,0.00,0.00}

\node[text=drawColor,anchor=base east,inner sep=0pt, outer sep=0pt, scale=  0.80] at ( 61.96,532.27) {0.00};

\node[text=drawColor,anchor=base east,inner sep=0pt, outer sep=0pt, scale=  0.80] at ( 61.96,550.15) {0.01};

\node[text=drawColor,anchor=base east,inner sep=0pt, outer sep=0pt, scale=  0.80] at ( 61.96,562.84) {0.05};

\node[text=drawColor,anchor=base east,inner sep=0pt, outer sep=0pt, scale=  0.80] at ( 61.96,570.78) {0.10};

\node[text=drawColor,anchor=base east,inner sep=0pt, outer sep=0pt, scale=  0.80] at ( 61.96,580.79) {0.20};

\node[text=drawColor,anchor=base east,inner sep=0pt, outer sep=0pt, scale=  0.80] at ( 61.96,593.40) {0.40};

\node[text=drawColor,anchor=base east,inner sep=0pt, outer sep=0pt, scale=  0.80] at ( 61.96,602.25) {0.60};

\node[text=drawColor,anchor=base east,inner sep=0pt, outer sep=0pt, scale=  0.80] at ( 61.96,609.29) {0.80};

\node[text=drawColor,anchor=base east,inner sep=0pt, outer sep=0pt, scale=  0.80] at ( 61.96,615.24) {1.00};
\end{scope}
\begin{scope}
\path[clip] (  0.00,  0.00) rectangle (505.89,650.43);
\definecolor[named]{drawColor}{rgb}{0.00,0.00,0.00}

\path[draw=drawColor,line width= 0.6pt,line join=round] ( 64.36,535.03) --
	( 67.36,535.03);

\path[draw=drawColor,line width= 0.6pt,line join=round] ( 64.36,552.90) --
	( 67.36,552.90);

\path[draw=drawColor,line width= 0.6pt,line join=round] ( 64.36,565.59) --
	( 67.36,565.59);

\path[draw=drawColor,line width= 0.6pt,line join=round] ( 64.36,573.54) --
	( 67.36,573.54);

\path[draw=drawColor,line width= 0.6pt,line join=round] ( 64.36,583.54) --
	( 67.36,583.54);

\path[draw=drawColor,line width= 0.6pt,line join=round] ( 64.36,596.16) --
	( 67.36,596.16);

\path[draw=drawColor,line width= 0.6pt,line join=round] ( 64.36,605.00) --
	( 67.36,605.00);

\path[draw=drawColor,line width= 0.6pt,line join=round] ( 64.36,612.04) --
	( 67.36,612.04);

\path[draw=drawColor,line width= 0.6pt,line join=round] ( 64.36,617.99) --
	( 67.36,617.99);
\end{scope}
\begin{scope}
\path[clip] (  0.00,  0.00) rectangle (505.89,650.43);
\definecolor[named]{drawColor}{rgb}{0.00,0.00,0.00}

\path[draw=drawColor,line width= 0.6pt,line join=round] (155.91,523.73) --
	(155.91,526.73);

\path[draw=drawColor,line width= 0.6pt,line join=round] (182.75,523.73) --
	(182.75,526.73);

\path[draw=drawColor,line width= 0.6pt,line join=round] (201.58,523.73) --
	(201.58,526.73);

\path[draw=drawColor,line width= 0.6pt,line join=round] (216.58,523.73) --
	(216.58,526.73);

\path[draw=drawColor,line width= 0.6pt,line join=round] (229.23,523.73) --
	(229.23,526.73);
\end{scope}
\begin{scope}
\path[clip] (  0.00,  0.00) rectangle (505.89,650.43);
\definecolor[named]{drawColor}{rgb}{0.00,0.00,0.00}

\node[text=drawColor,rotate= 50.00,anchor=base east,inner sep=0pt, outer sep=0pt, scale=  0.80] at (160.13,517.79) {100};

\node[text=drawColor,rotate= 50.00,anchor=base east,inner sep=0pt, outer sep=0pt, scale=  0.80] at (186.97,517.79) {200};

\node[text=drawColor,rotate= 50.00,anchor=base east,inner sep=0pt, outer sep=0pt, scale=  0.80] at (205.80,517.79) {300};

\node[text=drawColor,rotate= 50.00,anchor=base east,inner sep=0pt, outer sep=0pt, scale=  0.80] at (220.80,517.79) {400};

\node[text=drawColor,rotate= 50.00,anchor=base east,inner sep=0pt, outer sep=0pt, scale=  0.80] at (233.46,517.79) {500};
\end{scope}
\begin{scope}
\path[clip] (  0.00,  0.00) rectangle (505.89,650.43);
\definecolor[named]{drawColor}{rgb}{0.00,0.00,0.00}

\node[text=drawColor,anchor=base,inner sep=0pt, outer sep=0pt, scale=  1.10] at (150.07,496.22) {\# Instances};
\end{scope}
\begin{scope}
\path[clip] (  0.00,  0.00) rectangle (505.89,650.43);
\definecolor[named]{drawColor}{rgb}{0.00,0.00,0.00}

\node[text=drawColor,rotate= 90.00,anchor=base,inner sep=0pt, outer sep=0pt, scale=  1.10] at ( 37.74,577.85) {1-(Best/Algorithm)};
\end{scope}
\begin{scope}
\path[clip] (  0.00,  0.00) rectangle (505.89,650.43);
\definecolor[named]{drawColor}{rgb}{0.00,0.00,0.00}

\node[text=drawColor,anchor=base,inner sep=0pt, outer sep=0pt, scale=  1.20] at (150.07,636.17) {$k=2$};
\end{scope}
\begin{scope}
\path[clip] (267.11,487.82) rectangle (491.72,650.43);
\definecolor[named]{drawColor}{rgb}{1.00,1.00,1.00}
\definecolor[named]{fillColor}{rgb}{1.00,1.00,1.00}

\path[draw=drawColor,line width= 0.6pt,line join=round,line cap=round,fill=fillColor] (267.11,487.82) rectangle (491.72,650.43);
\end{scope}
\begin{scope}
\path[clip] (320.31,526.73) rectangle (485.72,628.97);
\definecolor[named]{fillColor}{rgb}{1.00,1.00,1.00}

\path[fill=fillColor] (320.31,526.73) rectangle (485.72,628.97);
\definecolor[named]{drawColor}{rgb}{0.98,0.98,0.98}

\path[draw=drawColor,line width= 0.6pt,line join=round] (320.31,543.96) --
	(485.72,543.96);

\path[draw=drawColor,line width= 0.6pt,line join=round] (320.31,559.25) --
	(485.72,559.25);

\path[draw=drawColor,line width= 0.6pt,line join=round] (320.31,569.56) --
	(485.72,569.56);

\path[draw=drawColor,line width= 0.6pt,line join=round] (320.31,578.54) --
	(485.72,578.54);

\path[draw=drawColor,line width= 0.6pt,line join=round] (320.31,589.85) --
	(485.72,589.85);

\path[draw=drawColor,line width= 0.6pt,line join=round] (320.31,600.58) --
	(485.72,600.58);

\path[draw=drawColor,line width= 0.6pt,line join=round] (320.31,608.52) --
	(485.72,608.52);

\path[draw=drawColor,line width= 0.6pt,line join=round] (320.31,615.02) --
	(485.72,615.02);

\path[draw=drawColor,line width= 0.6pt,line join=round] (320.31,626.93) --
	(485.72,626.93);

\path[draw=drawColor,line width= 0.6pt,line join=round] (382.10,526.73) --
	(382.10,628.97);

\path[draw=drawColor,line width= 0.6pt,line join=round] (395.54,526.73) --
	(395.54,628.97);

\path[draw=drawColor,line width= 0.6pt,line join=round] (422.43,526.73) --
	(422.43,628.97);

\path[draw=drawColor,line width= 0.6pt,line join=round] (445.31,526.73) --
	(445.31,628.97);

\path[draw=drawColor,line width= 0.6pt,line join=round] (462.25,526.73) --
	(462.25,628.97);

\path[draw=drawColor,line width= 0.6pt,line join=round] (476.09,526.73) --
	(476.09,628.97);
\definecolor[named]{drawColor}{rgb}{0.75,0.75,0.75}

\path[draw=drawColor,line width= 0.6pt,dash pattern=on 1pt off 3pt ,line join=round] (320.31,535.03) --
	(485.72,535.03);

\path[draw=drawColor,line width= 0.6pt,dash pattern=on 1pt off 3pt ,line join=round] (320.31,552.90) --
	(485.72,552.90);

\path[draw=drawColor,line width= 0.6pt,dash pattern=on 1pt off 3pt ,line join=round] (320.31,565.59) --
	(485.72,565.59);

\path[draw=drawColor,line width= 0.6pt,dash pattern=on 1pt off 3pt ,line join=round] (320.31,573.54) --
	(485.72,573.54);

\path[draw=drawColor,line width= 0.6pt,dash pattern=on 1pt off 3pt ,line join=round] (320.31,583.54) --
	(485.72,583.54);

\path[draw=drawColor,line width= 0.6pt,dash pattern=on 1pt off 3pt ,line join=round] (320.31,596.16) --
	(485.72,596.16);

\path[draw=drawColor,line width= 0.6pt,dash pattern=on 1pt off 3pt ,line join=round] (320.31,605.00) --
	(485.72,605.00);

\path[draw=drawColor,line width= 0.6pt,dash pattern=on 1pt off 3pt ,line join=round] (320.31,612.04) --
	(485.72,612.04);

\path[draw=drawColor,line width= 0.6pt,dash pattern=on 1pt off 3pt ,line join=round] (320.31,617.99) --
	(485.72,617.99);

\path[draw=drawColor,line width= 0.6pt,dash pattern=on 1pt off 3pt ,line join=round] (408.99,526.73) --
	(408.99,628.97);

\path[draw=drawColor,line width= 0.6pt,dash pattern=on 1pt off 3pt ,line join=round] (435.88,526.73) --
	(435.88,628.97);

\path[draw=drawColor,line width= 0.6pt,dash pattern=on 1pt off 3pt ,line join=round] (454.74,526.73) --
	(454.74,628.97);

\path[draw=drawColor,line width= 0.6pt,dash pattern=on 1pt off 3pt ,line join=round] (469.75,526.73) --
	(469.75,628.97);

\path[draw=drawColor,line width= 0.6pt,dash pattern=on 1pt off 3pt ,line join=round] (482.43,526.73) --
	(482.43,628.97);
\definecolor[named]{drawColor}{rgb}{0.89,0.10,0.11}
\definecolor[named]{fillColor}{rgb}{0.89,0.10,0.11}

\path[draw=drawColor,line width= 0.4pt,line join=round,line cap=round,fill=fillColor] (327.82,592.71) circle (  1.16);

\path[draw=drawColor,line width= 0.4pt,line join=round,line cap=round,fill=fillColor] (333.62,587.52) circle (  1.16);

\path[draw=drawColor,line width= 0.4pt,line join=round,line cap=round,fill=fillColor] (337.68,587.03) circle (  1.16);

\path[draw=drawColor,line width= 0.4pt,line join=round,line cap=round,fill=fillColor] (340.92,586.54) circle (  1.16);

\path[draw=drawColor,line width= 0.4pt,line join=round,line cap=round,fill=fillColor] (343.65,585.25) circle (  1.16);

\path[draw=drawColor,line width= 0.4pt,line join=round,line cap=round,fill=fillColor] (346.04,584.62) circle (  1.16);

\path[draw=drawColor,line width= 0.4pt,line join=round,line cap=round,fill=fillColor] (348.17,583.59) circle (  1.16);

\path[draw=drawColor,line width= 0.4pt,line join=round,line cap=round,fill=fillColor] (350.11,583.44) circle (  1.16);

\path[draw=drawColor,line width= 0.4pt,line join=round,line cap=round,fill=fillColor] (351.90,583.42) circle (  1.16);

\path[draw=drawColor,line width= 0.4pt,line join=round,line cap=round,fill=fillColor] (353.55,582.21) circle (  1.16);

\path[draw=drawColor,line width= 0.4pt,line join=round,line cap=round,fill=fillColor] (355.10,582.21) circle (  1.16);

\path[draw=drawColor,line width= 0.4pt,line join=round,line cap=round,fill=fillColor] (356.56,582.00) circle (  1.16);

\path[draw=drawColor,line width= 0.4pt,line join=round,line cap=round,fill=fillColor] (357.94,580.77) circle (  1.16);

\path[draw=drawColor,line width= 0.4pt,line join=round,line cap=round,fill=fillColor] (359.25,580.66) circle (  1.16);

\path[draw=drawColor,line width= 0.4pt,line join=round,line cap=round,fill=fillColor] (360.50,580.57) circle (  1.16);

\path[draw=drawColor,line width= 0.4pt,line join=round,line cap=round,fill=fillColor] (361.70,580.51) circle (  1.16);

\path[draw=drawColor,line width= 0.4pt,line join=round,line cap=round,fill=fillColor] (362.84,580.23) circle (  1.16);

\path[draw=drawColor,line width= 0.4pt,line join=round,line cap=round,fill=fillColor] (363.95,579.90) circle (  1.16);

\path[draw=drawColor,line width= 0.4pt,line join=round,line cap=round,fill=fillColor] (365.01,579.41) circle (  1.16);

\path[draw=drawColor,line width= 0.4pt,line join=round,line cap=round,fill=fillColor] (366.03,579.11) circle (  1.16);

\path[draw=drawColor,line width= 0.4pt,line join=round,line cap=round,fill=fillColor] (367.03,579.01) circle (  1.16);

\path[draw=drawColor,line width= 0.4pt,line join=round,line cap=round,fill=fillColor] (367.99,578.97) circle (  1.16);

\path[draw=drawColor,line width= 0.4pt,line join=round,line cap=round,fill=fillColor] (368.92,578.75) circle (  1.16);

\path[draw=drawColor,line width= 0.4pt,line join=round,line cap=round,fill=fillColor] (369.83,578.69) circle (  1.16);

\path[draw=drawColor,line width= 0.4pt,line join=round,line cap=round,fill=fillColor] (370.71,577.96) circle (  1.16);

\path[draw=drawColor,line width= 0.4pt,line join=round,line cap=round,fill=fillColor] (371.56,577.93) circle (  1.16);

\path[draw=drawColor,line width= 0.4pt,line join=round,line cap=round,fill=fillColor] (372.40,577.74) circle (  1.16);

\path[draw=drawColor,line width= 0.4pt,line join=round,line cap=round,fill=fillColor] (373.21,577.12) circle (  1.16);

\path[draw=drawColor,line width= 0.4pt,line join=round,line cap=round,fill=fillColor] (374.01,576.72) circle (  1.16);

\path[draw=drawColor,line width= 0.4pt,line join=round,line cap=round,fill=fillColor] (374.79,576.01) circle (  1.16);

\path[draw=drawColor,line width= 0.4pt,line join=round,line cap=round,fill=fillColor] (375.55,575.89) circle (  1.16);

\path[draw=drawColor,line width= 0.4pt,line join=round,line cap=round,fill=fillColor] (376.30,574.76) circle (  1.16);

\path[draw=drawColor,line width= 0.4pt,line join=round,line cap=round,fill=fillColor] (377.02,574.65) circle (  1.16);

\path[draw=drawColor,line width= 0.4pt,line join=round,line cap=round,fill=fillColor] (377.74,574.63) circle (  1.16);

\path[draw=drawColor,line width= 0.4pt,line join=round,line cap=round,fill=fillColor] (378.44,574.55) circle (  1.16);

\path[draw=drawColor,line width= 0.4pt,line join=round,line cap=round,fill=fillColor] (379.13,574.04) circle (  1.16);

\path[draw=drawColor,line width= 0.4pt,line join=round,line cap=round,fill=fillColor] (379.80,574.04) circle (  1.16);

\path[draw=drawColor,line width= 0.4pt,line join=round,line cap=round,fill=fillColor] (380.47,573.48) circle (  1.16);

\path[draw=drawColor,line width= 0.4pt,line join=round,line cap=round,fill=fillColor] (381.12,573.19) circle (  1.16);

\path[draw=drawColor,line width= 0.4pt,line join=round,line cap=round,fill=fillColor] (381.76,572.87) circle (  1.16);

\path[draw=drawColor,line width= 0.4pt,line join=round,line cap=round,fill=fillColor] (382.39,572.39) circle (  1.16);

\path[draw=drawColor,line width= 0.4pt,line join=round,line cap=round,fill=fillColor] (383.01,572.37) circle (  1.16);

\path[draw=drawColor,line width= 0.4pt,line join=round,line cap=round,fill=fillColor] (383.62,572.23) circle (  1.16);

\path[draw=drawColor,line width= 0.4pt,line join=round,line cap=round,fill=fillColor] (384.22,572.21) circle (  1.16);

\path[draw=drawColor,line width= 0.4pt,line join=round,line cap=round,fill=fillColor] (384.81,572.05) circle (  1.16);

\path[draw=drawColor,line width= 0.4pt,line join=round,line cap=round,fill=fillColor] (385.39,572.01) circle (  1.16);

\path[draw=drawColor,line width= 0.4pt,line join=round,line cap=round,fill=fillColor] (385.97,571.98) circle (  1.16);

\path[draw=drawColor,line width= 0.4pt,line join=round,line cap=round,fill=fillColor] (386.54,571.89) circle (  1.16);

\path[draw=drawColor,line width= 0.4pt,line join=round,line cap=round,fill=fillColor] (387.09,571.80) circle (  1.16);

\path[draw=drawColor,line width= 0.4pt,line join=round,line cap=round,fill=fillColor] (387.64,571.69) circle (  1.16);

\path[draw=drawColor,line width= 0.4pt,line join=round,line cap=round,fill=fillColor] (388.19,571.60) circle (  1.16);

\path[draw=drawColor,line width= 0.4pt,line join=round,line cap=round,fill=fillColor] (388.73,571.58) circle (  1.16);

\path[draw=drawColor,line width= 0.4pt,line join=round,line cap=round,fill=fillColor] (389.26,571.33) circle (  1.16);

\path[draw=drawColor,line width= 0.4pt,line join=round,line cap=round,fill=fillColor] (389.78,571.26) circle (  1.16);

\path[draw=drawColor,line width= 0.4pt,line join=round,line cap=round,fill=fillColor] (390.30,571.17) circle (  1.16);

\path[draw=drawColor,line width= 0.4pt,line join=round,line cap=round,fill=fillColor] (390.81,570.93) circle (  1.16);

\path[draw=drawColor,line width= 0.4pt,line join=round,line cap=round,fill=fillColor] (391.31,570.53) circle (  1.16);

\path[draw=drawColor,line width= 0.4pt,line join=round,line cap=round,fill=fillColor] (391.81,570.53) circle (  1.16);

\path[draw=drawColor,line width= 0.4pt,line join=round,line cap=round,fill=fillColor] (392.30,570.48) circle (  1.16);

\path[draw=drawColor,line width= 0.4pt,line join=round,line cap=round,fill=fillColor] (392.79,570.37) circle (  1.16);

\path[draw=drawColor,line width= 0.4pt,line join=round,line cap=round,fill=fillColor] (393.27,570.16) circle (  1.16);

\path[draw=drawColor,line width= 0.4pt,line join=round,line cap=round,fill=fillColor] (393.75,570.12) circle (  1.16);

\path[draw=drawColor,line width= 0.4pt,line join=round,line cap=round,fill=fillColor] (394.22,570.10) circle (  1.16);

\path[draw=drawColor,line width= 0.4pt,line join=round,line cap=round,fill=fillColor] (394.69,569.96) circle (  1.16);

\path[draw=drawColor,line width= 0.4pt,line join=round,line cap=round,fill=fillColor] (395.15,569.70) circle (  1.16);

\path[draw=drawColor,line width= 0.4pt,line join=round,line cap=round,fill=fillColor] (395.61,569.68) circle (  1.16);

\path[draw=drawColor,line width= 0.4pt,line join=round,line cap=round,fill=fillColor] (396.06,569.57) circle (  1.16);

\path[draw=drawColor,line width= 0.4pt,line join=round,line cap=round,fill=fillColor] (396.51,569.52) circle (  1.16);

\path[draw=drawColor,line width= 0.4pt,line join=round,line cap=round,fill=fillColor] (396.95,569.48) circle (  1.16);

\path[draw=drawColor,line width= 0.4pt,line join=round,line cap=round,fill=fillColor] (397.39,569.41) circle (  1.16);

\path[draw=drawColor,line width= 0.4pt,line join=round,line cap=round,fill=fillColor] (397.83,569.31) circle (  1.16);

\path[draw=drawColor,line width= 0.4pt,line join=round,line cap=round,fill=fillColor] (398.26,569.28) circle (  1.16);

\path[draw=drawColor,line width= 0.4pt,line join=round,line cap=round,fill=fillColor] (398.68,569.18) circle (  1.16);

\path[draw=drawColor,line width= 0.4pt,line join=round,line cap=round,fill=fillColor] (399.11,569.16) circle (  1.16);

\path[draw=drawColor,line width= 0.4pt,line join=round,line cap=round,fill=fillColor] (399.53,569.09) circle (  1.16);

\path[draw=drawColor,line width= 0.4pt,line join=round,line cap=round,fill=fillColor] (399.94,569.01) circle (  1.16);

\path[draw=drawColor,line width= 0.4pt,line join=round,line cap=round,fill=fillColor] (400.36,568.98) circle (  1.16);

\path[draw=drawColor,line width= 0.4pt,line join=round,line cap=round,fill=fillColor] (400.76,568.92) circle (  1.16);

\path[draw=drawColor,line width= 0.4pt,line join=round,line cap=round,fill=fillColor] (401.17,568.89) circle (  1.16);

\path[draw=drawColor,line width= 0.4pt,line join=round,line cap=round,fill=fillColor] (401.57,568.82) circle (  1.16);

\path[draw=drawColor,line width= 0.4pt,line join=round,line cap=round,fill=fillColor] (401.97,568.81) circle (  1.16);

\path[draw=drawColor,line width= 0.4pt,line join=round,line cap=round,fill=fillColor] (402.36,568.70) circle (  1.16);

\path[draw=drawColor,line width= 0.4pt,line join=round,line cap=round,fill=fillColor] (402.76,568.70) circle (  1.16);

\path[draw=drawColor,line width= 0.4pt,line join=round,line cap=round,fill=fillColor] (403.15,568.69) circle (  1.16);

\path[draw=drawColor,line width= 0.4pt,line join=round,line cap=round,fill=fillColor] (403.53,568.68) circle (  1.16);

\path[draw=drawColor,line width= 0.4pt,line join=round,line cap=round,fill=fillColor] (403.91,568.67) circle (  1.16);

\path[draw=drawColor,line width= 0.4pt,line join=round,line cap=round,fill=fillColor] (404.29,568.66) circle (  1.16);

\path[draw=drawColor,line width= 0.4pt,line join=round,line cap=round,fill=fillColor] (404.67,568.55) circle (  1.16);

\path[draw=drawColor,line width= 0.4pt,line join=round,line cap=round,fill=fillColor] (405.05,568.12) circle (  1.16);

\path[draw=drawColor,line width= 0.4pt,line join=round,line cap=round,fill=fillColor] (405.42,567.94) circle (  1.16);

\path[draw=drawColor,line width= 0.4pt,line join=round,line cap=round,fill=fillColor] (405.78,567.76) circle (  1.16);

\path[draw=drawColor,line width= 0.4pt,line join=round,line cap=round,fill=fillColor] (406.15,567.66) circle (  1.16);

\path[draw=drawColor,line width= 0.4pt,line join=round,line cap=round,fill=fillColor] (406.51,567.63) circle (  1.16);

\path[draw=drawColor,line width= 0.4pt,line join=round,line cap=round,fill=fillColor] (406.87,567.54) circle (  1.16);

\path[draw=drawColor,line width= 0.4pt,line join=round,line cap=round,fill=fillColor] (407.23,567.38) circle (  1.16);

\path[draw=drawColor,line width= 0.4pt,line join=round,line cap=round,fill=fillColor] (407.59,567.36) circle (  1.16);

\path[draw=drawColor,line width= 0.4pt,line join=round,line cap=round,fill=fillColor] (407.94,567.26) circle (  1.16);

\path[draw=drawColor,line width= 0.4pt,line join=round,line cap=round,fill=fillColor] (408.29,567.17) circle (  1.16);

\path[draw=drawColor,line width= 0.4pt,line join=round,line cap=round,fill=fillColor] (408.64,567.11) circle (  1.16);

\path[draw=drawColor,line width= 0.4pt,line join=round,line cap=round,fill=fillColor] (408.99,567.03) circle (  1.16);

\path[draw=drawColor,line width= 0.4pt,line join=round,line cap=round,fill=fillColor] (409.33,566.94) circle (  1.16);

\path[draw=drawColor,line width= 0.4pt,line join=round,line cap=round,fill=fillColor] (409.67,566.52) circle (  1.16);

\path[draw=drawColor,line width= 0.4pt,line join=round,line cap=round,fill=fillColor] (410.01,566.51) circle (  1.16);

\path[draw=drawColor,line width= 0.4pt,line join=round,line cap=round,fill=fillColor] (410.35,566.48) circle (  1.16);

\path[draw=drawColor,line width= 0.4pt,line join=round,line cap=round,fill=fillColor] (410.68,566.43) circle (  1.16);

\path[draw=drawColor,line width= 0.4pt,line join=round,line cap=round,fill=fillColor] (411.02,566.33) circle (  1.16);

\path[draw=drawColor,line width= 0.4pt,line join=round,line cap=round,fill=fillColor] (411.35,566.23) circle (  1.16);

\path[draw=drawColor,line width= 0.4pt,line join=round,line cap=round,fill=fillColor] (411.67,566.07) circle (  1.16);

\path[draw=drawColor,line width= 0.4pt,line join=round,line cap=round,fill=fillColor] (412.00,565.90) circle (  1.16);

\path[draw=drawColor,line width= 0.4pt,line join=round,line cap=round,fill=fillColor] (412.33,565.88) circle (  1.16);

\path[draw=drawColor,line width= 0.4pt,line join=round,line cap=round,fill=fillColor] (412.65,565.84) circle (  1.16);

\path[draw=drawColor,line width= 0.4pt,line join=round,line cap=round,fill=fillColor] (412.97,565.72) circle (  1.16);

\path[draw=drawColor,line width= 0.4pt,line join=round,line cap=round,fill=fillColor] (413.29,565.71) circle (  1.16);

\path[draw=drawColor,line width= 0.4pt,line join=round,line cap=round,fill=fillColor] (413.60,565.60) circle (  1.16);

\path[draw=drawColor,line width= 0.4pt,line join=round,line cap=round,fill=fillColor] (413.92,565.58) circle (  1.16);

\path[draw=drawColor,line width= 0.4pt,line join=round,line cap=round,fill=fillColor] (414.23,565.41) circle (  1.16);

\path[draw=drawColor,line width= 0.4pt,line join=round,line cap=round,fill=fillColor] (414.54,565.14) circle (  1.16);

\path[draw=drawColor,line width= 0.4pt,line join=round,line cap=round,fill=fillColor] (414.85,564.97) circle (  1.16);

\path[draw=drawColor,line width= 0.4pt,line join=round,line cap=round,fill=fillColor] (415.16,564.92) circle (  1.16);

\path[draw=drawColor,line width= 0.4pt,line join=round,line cap=round,fill=fillColor] (415.47,564.72) circle (  1.16);

\path[draw=drawColor,line width= 0.4pt,line join=round,line cap=round,fill=fillColor] (415.77,564.65) circle (  1.16);

\path[draw=drawColor,line width= 0.4pt,line join=round,line cap=round,fill=fillColor] (416.08,564.54) circle (  1.16);

\path[draw=drawColor,line width= 0.4pt,line join=round,line cap=round,fill=fillColor] (416.38,564.49) circle (  1.16);

\path[draw=drawColor,line width= 0.4pt,line join=round,line cap=round,fill=fillColor] (416.68,564.34) circle (  1.16);

\path[draw=drawColor,line width= 0.4pt,line join=round,line cap=round,fill=fillColor] (416.97,564.31) circle (  1.16);

\path[draw=drawColor,line width= 0.4pt,line join=round,line cap=round,fill=fillColor] (417.27,564.14) circle (  1.16);

\path[draw=drawColor,line width= 0.4pt,line join=round,line cap=round,fill=fillColor] (417.57,564.12) circle (  1.16);

\path[draw=drawColor,line width= 0.4pt,line join=round,line cap=round,fill=fillColor] (417.86,563.85) circle (  1.16);

\path[draw=drawColor,line width= 0.4pt,line join=round,line cap=round,fill=fillColor] (418.15,563.79) circle (  1.16);

\path[draw=drawColor,line width= 0.4pt,line join=round,line cap=round,fill=fillColor] (418.44,563.69) circle (  1.16);

\path[draw=drawColor,line width= 0.4pt,line join=round,line cap=round,fill=fillColor] (418.73,563.62) circle (  1.16);

\path[draw=drawColor,line width= 0.4pt,line join=round,line cap=round,fill=fillColor] (419.02,563.60) circle (  1.16);

\path[draw=drawColor,line width= 0.4pt,line join=round,line cap=round,fill=fillColor] (419.30,563.57) circle (  1.16);

\path[draw=drawColor,line width= 0.4pt,line join=round,line cap=round,fill=fillColor] (419.59,563.55) circle (  1.16);

\path[draw=drawColor,line width= 0.4pt,line join=round,line cap=round,fill=fillColor] (419.87,563.52) circle (  1.16);

\path[draw=drawColor,line width= 0.4pt,line join=round,line cap=round,fill=fillColor] (420.15,563.51) circle (  1.16);

\path[draw=drawColor,line width= 0.4pt,line join=round,line cap=round,fill=fillColor] (420.43,563.41) circle (  1.16);

\path[draw=drawColor,line width= 0.4pt,line join=round,line cap=round,fill=fillColor] (420.71,563.30) circle (  1.16);

\path[draw=drawColor,line width= 0.4pt,line join=round,line cap=round,fill=fillColor] (420.99,563.28) circle (  1.16);

\path[draw=drawColor,line width= 0.4pt,line join=round,line cap=round,fill=fillColor] (421.26,563.20) circle (  1.16);

\path[draw=drawColor,line width= 0.4pt,line join=round,line cap=round,fill=fillColor] (421.54,563.06) circle (  1.16);

\path[draw=drawColor,line width= 0.4pt,line join=round,line cap=round,fill=fillColor] (421.81,562.76) circle (  1.16);

\path[draw=drawColor,line width= 0.4pt,line join=round,line cap=round,fill=fillColor] (422.09,562.65) circle (  1.16);

\path[draw=drawColor,line width= 0.4pt,line join=round,line cap=round,fill=fillColor] (422.36,562.48) circle (  1.16);

\path[draw=drawColor,line width= 0.4pt,line join=round,line cap=round,fill=fillColor] (422.63,562.44) circle (  1.16);

\path[draw=drawColor,line width= 0.4pt,line join=round,line cap=round,fill=fillColor] (422.90,562.39) circle (  1.16);

\path[draw=drawColor,line width= 0.4pt,line join=round,line cap=round,fill=fillColor] (423.16,562.30) circle (  1.16);

\path[draw=drawColor,line width= 0.4pt,line join=round,line cap=round,fill=fillColor] (423.43,562.26) circle (  1.16);

\path[draw=drawColor,line width= 0.4pt,line join=round,line cap=round,fill=fillColor] (423.69,562.24) circle (  1.16);

\path[draw=drawColor,line width= 0.4pt,line join=round,line cap=round,fill=fillColor] (423.96,562.20) circle (  1.16);

\path[draw=drawColor,line width= 0.4pt,line join=round,line cap=round,fill=fillColor] (424.22,562.13) circle (  1.16);

\path[draw=drawColor,line width= 0.4pt,line join=round,line cap=round,fill=fillColor] (424.48,562.08) circle (  1.16);

\path[draw=drawColor,line width= 0.4pt,line join=round,line cap=round,fill=fillColor] (424.74,561.97) circle (  1.16);

\path[draw=drawColor,line width= 0.4pt,line join=round,line cap=round,fill=fillColor] (425.00,561.86) circle (  1.16);

\path[draw=drawColor,line width= 0.4pt,line join=round,line cap=round,fill=fillColor] (425.26,561.81) circle (  1.16);

\path[draw=drawColor,line width= 0.4pt,line join=round,line cap=round,fill=fillColor] (425.52,561.69) circle (  1.16);

\path[draw=drawColor,line width= 0.4pt,line join=round,line cap=round,fill=fillColor] (425.77,561.68) circle (  1.16);

\path[draw=drawColor,line width= 0.4pt,line join=round,line cap=round,fill=fillColor] (426.03,561.63) circle (  1.16);

\path[draw=drawColor,line width= 0.4pt,line join=round,line cap=round,fill=fillColor] (426.28,561.63) circle (  1.16);

\path[draw=drawColor,line width= 0.4pt,line join=round,line cap=round,fill=fillColor] (426.53,561.34) circle (  1.16);

\path[draw=drawColor,line width= 0.4pt,line join=round,line cap=round,fill=fillColor] (426.78,561.16) circle (  1.16);

\path[draw=drawColor,line width= 0.4pt,line join=round,line cap=round,fill=fillColor] (427.03,561.13) circle (  1.16);

\path[draw=drawColor,line width= 0.4pt,line join=round,line cap=round,fill=fillColor] (427.28,561.01) circle (  1.16);

\path[draw=drawColor,line width= 0.4pt,line join=round,line cap=round,fill=fillColor] (427.53,561.00) circle (  1.16);

\path[draw=drawColor,line width= 0.4pt,line join=round,line cap=round,fill=fillColor] (427.78,560.70) circle (  1.16);

\path[draw=drawColor,line width= 0.4pt,line join=round,line cap=round,fill=fillColor] (428.03,560.66) circle (  1.16);

\path[draw=drawColor,line width= 0.4pt,line join=round,line cap=round,fill=fillColor] (428.27,560.61) circle (  1.16);

\path[draw=drawColor,line width= 0.4pt,line join=round,line cap=round,fill=fillColor] (428.52,560.49) circle (  1.16);

\path[draw=drawColor,line width= 0.4pt,line join=round,line cap=round,fill=fillColor] (428.76,560.42) circle (  1.16);

\path[draw=drawColor,line width= 0.4pt,line join=round,line cap=round,fill=fillColor] (429.00,560.41) circle (  1.16);

\path[draw=drawColor,line width= 0.4pt,line join=round,line cap=round,fill=fillColor] (429.24,560.37) circle (  1.16);

\path[draw=drawColor,line width= 0.4pt,line join=round,line cap=round,fill=fillColor] (429.48,559.98) circle (  1.16);

\path[draw=drawColor,line width= 0.4pt,line join=round,line cap=round,fill=fillColor] (429.72,559.86) circle (  1.16);

\path[draw=drawColor,line width= 0.4pt,line join=round,line cap=round,fill=fillColor] (429.96,559.83) circle (  1.16);

\path[draw=drawColor,line width= 0.4pt,line join=round,line cap=round,fill=fillColor] (430.20,559.81) circle (  1.16);

\path[draw=drawColor,line width= 0.4pt,line join=round,line cap=round,fill=fillColor] (430.44,559.80) circle (  1.16);

\path[draw=drawColor,line width= 0.4pt,line join=round,line cap=round,fill=fillColor] (430.67,559.78) circle (  1.16);

\path[draw=drawColor,line width= 0.4pt,line join=round,line cap=round,fill=fillColor] (430.91,559.77) circle (  1.16);

\path[draw=drawColor,line width= 0.4pt,line join=round,line cap=round,fill=fillColor] (431.14,559.72) circle (  1.16);

\path[draw=drawColor,line width= 0.4pt,line join=round,line cap=round,fill=fillColor] (431.38,559.70) circle (  1.16);

\path[draw=drawColor,line width= 0.4pt,line join=round,line cap=round,fill=fillColor] (431.61,559.67) circle (  1.16);

\path[draw=drawColor,line width= 0.4pt,line join=round,line cap=round,fill=fillColor] (431.84,559.63) circle (  1.16);

\path[draw=drawColor,line width= 0.4pt,line join=round,line cap=round,fill=fillColor] (432.07,559.62) circle (  1.16);

\path[draw=drawColor,line width= 0.4pt,line join=round,line cap=round,fill=fillColor] (432.30,559.43) circle (  1.16);

\path[draw=drawColor,line width= 0.4pt,line join=round,line cap=round,fill=fillColor] (432.53,559.41) circle (  1.16);

\path[draw=drawColor,line width= 0.4pt,line join=round,line cap=round,fill=fillColor] (432.76,559.36) circle (  1.16);

\path[draw=drawColor,line width= 0.4pt,line join=round,line cap=round,fill=fillColor] (432.99,559.34) circle (  1.16);

\path[draw=drawColor,line width= 0.4pt,line join=round,line cap=round,fill=fillColor] (433.21,559.31) circle (  1.16);

\path[draw=drawColor,line width= 0.4pt,line join=round,line cap=round,fill=fillColor] (433.44,559.22) circle (  1.16);

\path[draw=drawColor,line width= 0.4pt,line join=round,line cap=round,fill=fillColor] (433.67,559.20) circle (  1.16);

\path[draw=drawColor,line width= 0.4pt,line join=round,line cap=round,fill=fillColor] (433.89,559.18) circle (  1.16);

\path[draw=drawColor,line width= 0.4pt,line join=round,line cap=round,fill=fillColor] (434.11,559.09) circle (  1.16);

\path[draw=drawColor,line width= 0.4pt,line join=round,line cap=round,fill=fillColor] (434.34,559.08) circle (  1.16);

\path[draw=drawColor,line width= 0.4pt,line join=round,line cap=round,fill=fillColor] (434.56,559.06) circle (  1.16);

\path[draw=drawColor,line width= 0.4pt,line join=round,line cap=round,fill=fillColor] (434.78,558.89) circle (  1.16);

\path[draw=drawColor,line width= 0.4pt,line join=round,line cap=round,fill=fillColor] (435.00,558.80) circle (  1.16);

\path[draw=drawColor,line width= 0.4pt,line join=round,line cap=round,fill=fillColor] (435.22,558.70) circle (  1.16);

\path[draw=drawColor,line width= 0.4pt,line join=round,line cap=round,fill=fillColor] (435.44,558.64) circle (  1.16);

\path[draw=drawColor,line width= 0.4pt,line join=round,line cap=round,fill=fillColor] (435.66,558.59) circle (  1.16);

\path[draw=drawColor,line width= 0.4pt,line join=round,line cap=round,fill=fillColor] (435.88,558.57) circle (  1.16);

\path[draw=drawColor,line width= 0.4pt,line join=round,line cap=round,fill=fillColor] (436.09,558.53) circle (  1.16);

\path[draw=drawColor,line width= 0.4pt,line join=round,line cap=round,fill=fillColor] (436.31,558.46) circle (  1.16);

\path[draw=drawColor,line width= 0.4pt,line join=round,line cap=round,fill=fillColor] (436.52,558.43) circle (  1.16);

\path[draw=drawColor,line width= 0.4pt,line join=round,line cap=round,fill=fillColor] (436.74,558.42) circle (  1.16);

\path[draw=drawColor,line width= 0.4pt,line join=round,line cap=round,fill=fillColor] (436.95,558.32) circle (  1.16);

\path[draw=drawColor,line width= 0.4pt,line join=round,line cap=round,fill=fillColor] (437.17,558.23) circle (  1.16);

\path[draw=drawColor,line width= 0.4pt,line join=round,line cap=round,fill=fillColor] (437.38,558.21) circle (  1.16);

\path[draw=drawColor,line width= 0.4pt,line join=round,line cap=round,fill=fillColor] (437.59,558.15) circle (  1.16);

\path[draw=drawColor,line width= 0.4pt,line join=round,line cap=round,fill=fillColor] (437.80,558.05) circle (  1.16);

\path[draw=drawColor,line width= 0.4pt,line join=round,line cap=round,fill=fillColor] (438.01,557.99) circle (  1.16);

\path[draw=drawColor,line width= 0.4pt,line join=round,line cap=round,fill=fillColor] (438.22,557.99) circle (  1.16);

\path[draw=drawColor,line width= 0.4pt,line join=round,line cap=round,fill=fillColor] (438.43,557.86) circle (  1.16);

\path[draw=drawColor,line width= 0.4pt,line join=round,line cap=round,fill=fillColor] (438.64,557.69) circle (  1.16);

\path[draw=drawColor,line width= 0.4pt,line join=round,line cap=round,fill=fillColor] (438.85,557.53) circle (  1.16);

\path[draw=drawColor,line width= 0.4pt,line join=round,line cap=round,fill=fillColor] (439.06,557.50) circle (  1.16);

\path[draw=drawColor,line width= 0.4pt,line join=round,line cap=round,fill=fillColor] (439.26,557.46) circle (  1.16);

\path[draw=drawColor,line width= 0.4pt,line join=round,line cap=round,fill=fillColor] (439.47,557.45) circle (  1.16);

\path[draw=drawColor,line width= 0.4pt,line join=round,line cap=round,fill=fillColor] (439.67,557.41) circle (  1.16);

\path[draw=drawColor,line width= 0.4pt,line join=round,line cap=round,fill=fillColor] (439.88,557.35) circle (  1.16);

\path[draw=drawColor,line width= 0.4pt,line join=round,line cap=round,fill=fillColor] (440.08,557.34) circle (  1.16);

\path[draw=drawColor,line width= 0.4pt,line join=round,line cap=round,fill=fillColor] (440.29,557.31) circle (  1.16);

\path[draw=drawColor,line width= 0.4pt,line join=round,line cap=round,fill=fillColor] (440.49,557.31) circle (  1.16);

\path[draw=drawColor,line width= 0.4pt,line join=round,line cap=round,fill=fillColor] (440.69,557.21) circle (  1.16);

\path[draw=drawColor,line width= 0.4pt,line join=round,line cap=round,fill=fillColor] (440.89,557.07) circle (  1.16);

\path[draw=drawColor,line width= 0.4pt,line join=round,line cap=round,fill=fillColor] (441.09,557.06) circle (  1.16);

\path[draw=drawColor,line width= 0.4pt,line join=round,line cap=round,fill=fillColor] (441.29,557.06) circle (  1.16);

\path[draw=drawColor,line width= 0.4pt,line join=round,line cap=round,fill=fillColor] (441.49,557.03) circle (  1.16);

\path[draw=drawColor,line width= 0.4pt,line join=round,line cap=round,fill=fillColor] (441.69,557.01) circle (  1.16);

\path[draw=drawColor,line width= 0.4pt,line join=round,line cap=round,fill=fillColor] (441.89,556.97) circle (  1.16);

\path[draw=drawColor,line width= 0.4pt,line join=round,line cap=round,fill=fillColor] (442.09,556.88) circle (  1.16);

\path[draw=drawColor,line width= 0.4pt,line join=round,line cap=round,fill=fillColor] (442.29,556.86) circle (  1.16);

\path[draw=drawColor,line width= 0.4pt,line join=round,line cap=round,fill=fillColor] (442.49,556.85) circle (  1.16);

\path[draw=drawColor,line width= 0.4pt,line join=round,line cap=round,fill=fillColor] (442.68,556.84) circle (  1.16);

\path[draw=drawColor,line width= 0.4pt,line join=round,line cap=round,fill=fillColor] (442.88,556.74) circle (  1.16);

\path[draw=drawColor,line width= 0.4pt,line join=round,line cap=round,fill=fillColor] (443.07,556.62) circle (  1.16);

\path[draw=drawColor,line width= 0.4pt,line join=round,line cap=round,fill=fillColor] (443.27,556.58) circle (  1.16);

\path[draw=drawColor,line width= 0.4pt,line join=round,line cap=round,fill=fillColor] (443.46,556.58) circle (  1.16);

\path[draw=drawColor,line width= 0.4pt,line join=round,line cap=round,fill=fillColor] (443.66,556.52) circle (  1.16);

\path[draw=drawColor,line width= 0.4pt,line join=round,line cap=round,fill=fillColor] (443.85,556.50) circle (  1.16);

\path[draw=drawColor,line width= 0.4pt,line join=round,line cap=round,fill=fillColor] (444.04,556.47) circle (  1.16);

\path[draw=drawColor,line width= 0.4pt,line join=round,line cap=round,fill=fillColor] (444.23,556.41) circle (  1.16);

\path[draw=drawColor,line width= 0.4pt,line join=round,line cap=round,fill=fillColor] (444.43,556.37) circle (  1.16);

\path[draw=drawColor,line width= 0.4pt,line join=round,line cap=round,fill=fillColor] (444.62,556.28) circle (  1.16);

\path[draw=drawColor,line width= 0.4pt,line join=round,line cap=round,fill=fillColor] (444.81,556.26) circle (  1.16);

\path[draw=drawColor,line width= 0.4pt,line join=round,line cap=round,fill=fillColor] (445.00,556.21) circle (  1.16);

\path[draw=drawColor,line width= 0.4pt,line join=round,line cap=round,fill=fillColor] (445.19,556.10) circle (  1.16);

\path[draw=drawColor,line width= 0.4pt,line join=round,line cap=round,fill=fillColor] (445.38,556.04) circle (  1.16);

\path[draw=drawColor,line width= 0.4pt,line join=round,line cap=round,fill=fillColor] (445.56,555.98) circle (  1.16);

\path[draw=drawColor,line width= 0.4pt,line join=round,line cap=round,fill=fillColor] (445.75,555.95) circle (  1.16);

\path[draw=drawColor,line width= 0.4pt,line join=round,line cap=round,fill=fillColor] (445.94,555.82) circle (  1.16);

\path[draw=drawColor,line width= 0.4pt,line join=round,line cap=round,fill=fillColor] (446.13,555.70) circle (  1.16);

\path[draw=drawColor,line width= 0.4pt,line join=round,line cap=round,fill=fillColor] (446.31,555.62) circle (  1.16);

\path[draw=drawColor,line width= 0.4pt,line join=round,line cap=round,fill=fillColor] (446.50,555.50) circle (  1.16);

\path[draw=drawColor,line width= 0.4pt,line join=round,line cap=round,fill=fillColor] (446.68,555.45) circle (  1.16);

\path[draw=drawColor,line width= 0.4pt,line join=round,line cap=round,fill=fillColor] (446.87,555.45) circle (  1.16);

\path[draw=drawColor,line width= 0.4pt,line join=round,line cap=round,fill=fillColor] (447.05,555.45) circle (  1.16);

\path[draw=drawColor,line width= 0.4pt,line join=round,line cap=round,fill=fillColor] (447.24,555.38) circle (  1.16);

\path[draw=drawColor,line width= 0.4pt,line join=round,line cap=round,fill=fillColor] (447.42,555.37) circle (  1.16);

\path[draw=drawColor,line width= 0.4pt,line join=round,line cap=round,fill=fillColor] (447.60,555.24) circle (  1.16);

\path[draw=drawColor,line width= 0.4pt,line join=round,line cap=round,fill=fillColor] (447.79,555.22) circle (  1.16);

\path[draw=drawColor,line width= 0.4pt,line join=round,line cap=round,fill=fillColor] (447.97,555.14) circle (  1.16);

\path[draw=drawColor,line width= 0.4pt,line join=round,line cap=round,fill=fillColor] (448.15,555.06) circle (  1.16);

\path[draw=drawColor,line width= 0.4pt,line join=round,line cap=round,fill=fillColor] (448.33,555.01) circle (  1.16);

\path[draw=drawColor,line width= 0.4pt,line join=round,line cap=round,fill=fillColor] (448.51,555.00) circle (  1.16);

\path[draw=drawColor,line width= 0.4pt,line join=round,line cap=round,fill=fillColor] (448.69,554.66) circle (  1.16);

\path[draw=drawColor,line width= 0.4pt,line join=round,line cap=round,fill=fillColor] (448.87,554.61) circle (  1.16);

\path[draw=drawColor,line width= 0.4pt,line join=round,line cap=round,fill=fillColor] (449.05,554.17) circle (  1.16);

\path[draw=drawColor,line width= 0.4pt,line join=round,line cap=round,fill=fillColor] (449.23,554.15) circle (  1.16);

\path[draw=drawColor,line width= 0.4pt,line join=round,line cap=round,fill=fillColor] (449.41,554.03) circle (  1.16);

\path[draw=drawColor,line width= 0.4pt,line join=round,line cap=round,fill=fillColor] (449.59,553.79) circle (  1.16);

\path[draw=drawColor,line width= 0.4pt,line join=round,line cap=round,fill=fillColor] (449.77,553.69) circle (  1.16);

\path[draw=drawColor,line width= 0.4pt,line join=round,line cap=round,fill=fillColor] (449.94,553.58) circle (  1.16);

\path[draw=drawColor,line width= 0.4pt,line join=round,line cap=round,fill=fillColor] (450.12,553.56) circle (  1.16);

\path[draw=drawColor,line width= 0.4pt,line join=round,line cap=round,fill=fillColor] (450.30,553.41) circle (  1.16);

\path[draw=drawColor,line width= 0.4pt,line join=round,line cap=round,fill=fillColor] (450.47,553.34) circle (  1.16);

\path[draw=drawColor,line width= 0.4pt,line join=round,line cap=round,fill=fillColor] (450.65,553.32) circle (  1.16);

\path[draw=drawColor,line width= 0.4pt,line join=round,line cap=round,fill=fillColor] (450.82,553.31) circle (  1.16);

\path[draw=drawColor,line width= 0.4pt,line join=round,line cap=round,fill=fillColor] (451.00,553.31) circle (  1.16);

\path[draw=drawColor,line width= 0.4pt,line join=round,line cap=round,fill=fillColor] (451.17,553.23) circle (  1.16);

\path[draw=drawColor,line width= 0.4pt,line join=round,line cap=round,fill=fillColor] (451.35,553.22) circle (  1.16);

\path[draw=drawColor,line width= 0.4pt,line join=round,line cap=round,fill=fillColor] (451.52,553.22) circle (  1.16);

\path[draw=drawColor,line width= 0.4pt,line join=round,line cap=round,fill=fillColor] (451.69,553.19) circle (  1.16);

\path[draw=drawColor,line width= 0.4pt,line join=round,line cap=round,fill=fillColor] (451.86,553.01) circle (  1.16);

\path[draw=drawColor,line width= 0.4pt,line join=round,line cap=round,fill=fillColor] (452.04,552.95) circle (  1.16);

\path[draw=drawColor,line width= 0.4pt,line join=round,line cap=round,fill=fillColor] (452.21,552.92) circle (  1.16);

\path[draw=drawColor,line width= 0.4pt,line join=round,line cap=round,fill=fillColor] (452.38,552.89) circle (  1.16);

\path[draw=drawColor,line width= 0.4pt,line join=round,line cap=round,fill=fillColor] (452.55,552.85) circle (  1.16);

\path[draw=drawColor,line width= 0.4pt,line join=round,line cap=round,fill=fillColor] (452.72,552.83) circle (  1.16);

\path[draw=drawColor,line width= 0.4pt,line join=round,line cap=round,fill=fillColor] (452.89,552.75) circle (  1.16);

\path[draw=drawColor,line width= 0.4pt,line join=round,line cap=round,fill=fillColor] (453.06,552.66) circle (  1.16);

\path[draw=drawColor,line width= 0.4pt,line join=round,line cap=round,fill=fillColor] (453.23,552.64) circle (  1.16);

\path[draw=drawColor,line width= 0.4pt,line join=round,line cap=round,fill=fillColor] (453.40,552.57) circle (  1.16);

\path[draw=drawColor,line width= 0.4pt,line join=round,line cap=round,fill=fillColor] (453.57,552.55) circle (  1.16);

\path[draw=drawColor,line width= 0.4pt,line join=round,line cap=round,fill=fillColor] (453.74,552.52) circle (  1.16);

\path[draw=drawColor,line width= 0.4pt,line join=round,line cap=round,fill=fillColor] (453.90,552.33) circle (  1.16);

\path[draw=drawColor,line width= 0.4pt,line join=round,line cap=round,fill=fillColor] (454.07,552.32) circle (  1.16);

\path[draw=drawColor,line width= 0.4pt,line join=round,line cap=round,fill=fillColor] (454.24,552.32) circle (  1.16);

\path[draw=drawColor,line width= 0.4pt,line join=round,line cap=round,fill=fillColor] (454.40,552.24) circle (  1.16);

\path[draw=drawColor,line width= 0.4pt,line join=round,line cap=round,fill=fillColor] (454.57,552.22) circle (  1.16);

\path[draw=drawColor,line width= 0.4pt,line join=round,line cap=round,fill=fillColor] (454.74,552.12) circle (  1.16);

\path[draw=drawColor,line width= 0.4pt,line join=round,line cap=round,fill=fillColor] (454.90,552.01) circle (  1.16);

\path[draw=drawColor,line width= 0.4pt,line join=round,line cap=round,fill=fillColor] (455.07,551.99) circle (  1.16);

\path[draw=drawColor,line width= 0.4pt,line join=round,line cap=round,fill=fillColor] (455.23,551.93) circle (  1.16);

\path[draw=drawColor,line width= 0.4pt,line join=round,line cap=round,fill=fillColor] (455.40,551.75) circle (  1.16);

\path[draw=drawColor,line width= 0.4pt,line join=round,line cap=round,fill=fillColor] (455.56,551.72) circle (  1.16);

\path[draw=drawColor,line width= 0.4pt,line join=round,line cap=round,fill=fillColor] (455.73,551.59) circle (  1.16);

\path[draw=drawColor,line width= 0.4pt,line join=round,line cap=round,fill=fillColor] (455.89,551.56) circle (  1.16);

\path[draw=drawColor,line width= 0.4pt,line join=round,line cap=round,fill=fillColor] (456.05,551.49) circle (  1.16);

\path[draw=drawColor,line width= 0.4pt,line join=round,line cap=round,fill=fillColor] (456.21,551.44) circle (  1.16);

\path[draw=drawColor,line width= 0.4pt,line join=round,line cap=round,fill=fillColor] (456.38,551.41) circle (  1.16);

\path[draw=drawColor,line width= 0.4pt,line join=round,line cap=round,fill=fillColor] (456.54,551.32) circle (  1.16);

\path[draw=drawColor,line width= 0.4pt,line join=round,line cap=round,fill=fillColor] (456.70,551.31) circle (  1.16);

\path[draw=drawColor,line width= 0.4pt,line join=round,line cap=round,fill=fillColor] (456.86,551.30) circle (  1.16);

\path[draw=drawColor,line width= 0.4pt,line join=round,line cap=round,fill=fillColor] (457.02,551.21) circle (  1.16);

\path[draw=drawColor,line width= 0.4pt,line join=round,line cap=round,fill=fillColor] (457.18,551.19) circle (  1.16);

\path[draw=drawColor,line width= 0.4pt,line join=round,line cap=round,fill=fillColor] (457.34,550.88) circle (  1.16);

\path[draw=drawColor,line width= 0.4pt,line join=round,line cap=round,fill=fillColor] (457.50,550.57) circle (  1.16);

\path[draw=drawColor,line width= 0.4pt,line join=round,line cap=round,fill=fillColor] (457.66,550.54) circle (  1.16);

\path[draw=drawColor,line width= 0.4pt,line join=round,line cap=round,fill=fillColor] (457.82,550.18) circle (  1.16);

\path[draw=drawColor,line width= 0.4pt,line join=round,line cap=round,fill=fillColor] (457.98,550.10) circle (  1.16);

\path[draw=drawColor,line width= 0.4pt,line join=round,line cap=round,fill=fillColor] (458.14,550.10) circle (  1.16);

\path[draw=drawColor,line width= 0.4pt,line join=round,line cap=round,fill=fillColor] (458.30,550.09) circle (  1.16);

\path[draw=drawColor,line width= 0.4pt,line join=round,line cap=round,fill=fillColor] (458.46,550.08) circle (  1.16);

\path[draw=drawColor,line width= 0.4pt,line join=round,line cap=round,fill=fillColor] (458.61,549.89) circle (  1.16);

\path[draw=drawColor,line width= 0.4pt,line join=round,line cap=round,fill=fillColor] (458.77,549.82) circle (  1.16);

\path[draw=drawColor,line width= 0.4pt,line join=round,line cap=round,fill=fillColor] (458.93,549.47) circle (  1.16);

\path[draw=drawColor,line width= 0.4pt,line join=round,line cap=round,fill=fillColor] (459.09,549.43) circle (  1.16);

\path[draw=drawColor,line width= 0.4pt,line join=round,line cap=round,fill=fillColor] (459.24,549.40) circle (  1.16);

\path[draw=drawColor,line width= 0.4pt,line join=round,line cap=round,fill=fillColor] (459.40,549.40) circle (  1.16);

\path[draw=drawColor,line width= 0.4pt,line join=round,line cap=round,fill=fillColor] (459.55,549.24) circle (  1.16);

\path[draw=drawColor,line width= 0.4pt,line join=round,line cap=round,fill=fillColor] (459.71,549.11) circle (  1.16);

\path[draw=drawColor,line width= 0.4pt,line join=round,line cap=round,fill=fillColor] (459.86,549.08) circle (  1.16);

\path[draw=drawColor,line width= 0.4pt,line join=round,line cap=round,fill=fillColor] (460.02,548.91) circle (  1.16);

\path[draw=drawColor,line width= 0.4pt,line join=round,line cap=round,fill=fillColor] (460.17,548.88) circle (  1.16);

\path[draw=drawColor,line width= 0.4pt,line join=round,line cap=round,fill=fillColor] (460.33,548.81) circle (  1.16);

\path[draw=drawColor,line width= 0.4pt,line join=round,line cap=round,fill=fillColor] (460.48,548.79) circle (  1.16);

\path[draw=drawColor,line width= 0.4pt,line join=round,line cap=round,fill=fillColor] (460.63,548.77) circle (  1.16);

\path[draw=drawColor,line width= 0.4pt,line join=round,line cap=round,fill=fillColor] (460.79,548.76) circle (  1.16);

\path[draw=drawColor,line width= 0.4pt,line join=round,line cap=round,fill=fillColor] (460.94,548.75) circle (  1.16);

\path[draw=drawColor,line width= 0.4pt,line join=round,line cap=round,fill=fillColor] (461.09,548.64) circle (  1.16);

\path[draw=drawColor,line width= 0.4pt,line join=round,line cap=round,fill=fillColor] (461.25,548.45) circle (  1.16);

\path[draw=drawColor,line width= 0.4pt,line join=round,line cap=round,fill=fillColor] (461.40,548.39) circle (  1.16);

\path[draw=drawColor,line width= 0.4pt,line join=round,line cap=round,fill=fillColor] (461.55,548.39) circle (  1.16);

\path[draw=drawColor,line width= 0.4pt,line join=round,line cap=round,fill=fillColor] (461.70,548.29) circle (  1.16);

\path[draw=drawColor,line width= 0.4pt,line join=round,line cap=round,fill=fillColor] (461.85,548.13) circle (  1.16);

\path[draw=drawColor,line width= 0.4pt,line join=round,line cap=round,fill=fillColor] (462.00,548.08) circle (  1.16);

\path[draw=drawColor,line width= 0.4pt,line join=round,line cap=round,fill=fillColor] (462.15,547.96) circle (  1.16);

\path[draw=drawColor,line width= 0.4pt,line join=round,line cap=round,fill=fillColor] (462.30,547.89) circle (  1.16);

\path[draw=drawColor,line width= 0.4pt,line join=round,line cap=round,fill=fillColor] (462.45,547.89) circle (  1.16);

\path[draw=drawColor,line width= 0.4pt,line join=round,line cap=round,fill=fillColor] (462.60,547.88) circle (  1.16);

\path[draw=drawColor,line width= 0.4pt,line join=round,line cap=round,fill=fillColor] (462.75,547.70) circle (  1.16);

\path[draw=drawColor,line width= 0.4pt,line join=round,line cap=round,fill=fillColor] (462.90,547.52) circle (  1.16);

\path[draw=drawColor,line width= 0.4pt,line join=round,line cap=round,fill=fillColor] (463.05,547.47) circle (  1.16);

\path[draw=drawColor,line width= 0.4pt,line join=round,line cap=round,fill=fillColor] (463.20,547.26) circle (  1.16);

\path[draw=drawColor,line width= 0.4pt,line join=round,line cap=round,fill=fillColor] (463.35,547.03) circle (  1.16);

\path[draw=drawColor,line width= 0.4pt,line join=round,line cap=round,fill=fillColor] (463.50,546.60) circle (  1.16);

\path[draw=drawColor,line width= 0.4pt,line join=round,line cap=round,fill=fillColor] (463.64,546.53) circle (  1.16);

\path[draw=drawColor,line width= 0.4pt,line join=round,line cap=round,fill=fillColor] (463.79,546.44) circle (  1.16);

\path[draw=drawColor,line width= 0.4pt,line join=round,line cap=round,fill=fillColor] (463.94,546.26) circle (  1.16);

\path[draw=drawColor,line width= 0.4pt,line join=round,line cap=round,fill=fillColor] (464.09,546.15) circle (  1.16);

\path[draw=drawColor,line width= 0.4pt,line join=round,line cap=round,fill=fillColor] (464.23,545.97) circle (  1.16);

\path[draw=drawColor,line width= 0.4pt,line join=round,line cap=round,fill=fillColor] (464.38,545.96) circle (  1.16);

\path[draw=drawColor,line width= 0.4pt,line join=round,line cap=round,fill=fillColor] (464.53,545.89) circle (  1.16);

\path[draw=drawColor,line width= 0.4pt,line join=round,line cap=round,fill=fillColor] (464.67,545.66) circle (  1.16);

\path[draw=drawColor,line width= 0.4pt,line join=round,line cap=round,fill=fillColor] (464.82,545.57) circle (  1.16);

\path[draw=drawColor,line width= 0.4pt,line join=round,line cap=round,fill=fillColor] (464.96,545.52) circle (  1.16);

\path[draw=drawColor,line width= 0.4pt,line join=round,line cap=round,fill=fillColor] (465.11,545.38) circle (  1.16);

\path[draw=drawColor,line width= 0.4pt,line join=round,line cap=round,fill=fillColor] (465.25,545.33) circle (  1.16);

\path[draw=drawColor,line width= 0.4pt,line join=round,line cap=round,fill=fillColor] (465.40,544.97) circle (  1.16);

\path[draw=drawColor,line width= 0.4pt,line join=round,line cap=round,fill=fillColor] (465.54,544.97) circle (  1.16);

\path[draw=drawColor,line width= 0.4pt,line join=round,line cap=round,fill=fillColor] (465.68,544.97) circle (  1.16);

\path[draw=drawColor,line width= 0.4pt,line join=round,line cap=round,fill=fillColor] (465.83,544.94) circle (  1.16);

\path[draw=drawColor,line width= 0.4pt,line join=round,line cap=round,fill=fillColor] (465.97,544.54) circle (  1.16);

\path[draw=drawColor,line width= 0.4pt,line join=round,line cap=round,fill=fillColor] (466.12,544.52) circle (  1.16);

\path[draw=drawColor,line width= 0.4pt,line join=round,line cap=round,fill=fillColor] (466.26,543.71) circle (  1.16);

\path[draw=drawColor,line width= 0.4pt,line join=round,line cap=round,fill=fillColor] (466.40,543.71) circle (  1.16);

\path[draw=drawColor,line width= 0.4pt,line join=round,line cap=round,fill=fillColor] (466.54,543.71) circle (  1.16);

\path[draw=drawColor,line width= 0.4pt,line join=round,line cap=round,fill=fillColor] (466.69,543.70) circle (  1.16);

\path[draw=drawColor,line width= 0.4pt,line join=round,line cap=round,fill=fillColor] (466.83,543.70) circle (  1.16);

\path[draw=drawColor,line width= 0.4pt,line join=round,line cap=round,fill=fillColor] (466.97,543.70) circle (  1.16);

\path[draw=drawColor,line width= 0.4pt,line join=round,line cap=round,fill=fillColor] (467.11,543.70) circle (  1.16);

\path[draw=drawColor,line width= 0.4pt,line join=round,line cap=round,fill=fillColor] (467.25,543.70) circle (  1.16);

\path[draw=drawColor,line width= 0.4pt,line join=round,line cap=round,fill=fillColor] (467.39,543.62) circle (  1.16);

\path[draw=drawColor,line width= 0.4pt,line join=round,line cap=round,fill=fillColor] (467.53,543.40) circle (  1.16);

\path[draw=drawColor,line width= 0.4pt,line join=round,line cap=round,fill=fillColor] (467.67,543.28) circle (  1.16);

\path[draw=drawColor,line width= 0.4pt,line join=round,line cap=round,fill=fillColor] (467.81,543.24) circle (  1.16);

\path[draw=drawColor,line width= 0.4pt,line join=round,line cap=round,fill=fillColor] (467.95,543.02) circle (  1.16);

\path[draw=drawColor,line width= 0.4pt,line join=round,line cap=round,fill=fillColor] (468.09,542.75) circle (  1.16);

\path[draw=drawColor,line width= 0.4pt,line join=round,line cap=round,fill=fillColor] (468.23,542.55) circle (  1.16);

\path[draw=drawColor,line width= 0.4pt,line join=round,line cap=round,fill=fillColor] (468.37,542.43) circle (  1.16);

\path[draw=drawColor,line width= 0.4pt,line join=round,line cap=round,fill=fillColor] (468.51,542.32) circle (  1.16);

\path[draw=drawColor,line width= 0.4pt,line join=round,line cap=round,fill=fillColor] (468.65,542.26) circle (  1.16);

\path[draw=drawColor,line width= 0.4pt,line join=round,line cap=round,fill=fillColor] (468.79,542.21) circle (  1.16);

\path[draw=drawColor,line width= 0.4pt,line join=round,line cap=round,fill=fillColor] (468.93,542.08) circle (  1.16);

\path[draw=drawColor,line width= 0.4pt,line join=round,line cap=round,fill=fillColor] (469.07,539.04) circle (  1.16);

\path[draw=drawColor,line width= 0.4pt,line join=round,line cap=round,fill=fillColor] (469.20,535.03) circle (  1.16);

\path[draw=drawColor,line width= 0.4pt,line join=round,line cap=round,fill=fillColor] (469.34,535.03) circle (  1.16);

\path[draw=drawColor,line width= 0.4pt,line join=round,line cap=round,fill=fillColor] (469.48,535.03) circle (  1.16);

\path[draw=drawColor,line width= 0.4pt,line join=round,line cap=round,fill=fillColor] (469.62,535.03) circle (  1.16);

\path[draw=drawColor,line width= 0.4pt,line join=round,line cap=round,fill=fillColor] (469.75,535.03) circle (  1.16);

\path[draw=drawColor,line width= 0.4pt,line join=round,line cap=round,fill=fillColor] (469.89,535.03) circle (  1.16);

\path[draw=drawColor,line width= 0.4pt,line join=round,line cap=round,fill=fillColor] (470.03,535.03) circle (  1.16);

\path[draw=drawColor,line width= 0.4pt,line join=round,line cap=round,fill=fillColor] (470.16,535.03) circle (  1.16);

\path[draw=drawColor,line width= 0.4pt,line join=round,line cap=round,fill=fillColor] (470.30,535.03) circle (  1.16);

\path[draw=drawColor,line width= 0.4pt,line join=round,line cap=round,fill=fillColor] (470.43,535.03) circle (  1.16);

\path[draw=drawColor,line width= 0.4pt,line join=round,line cap=round,fill=fillColor] (470.57,535.03) circle (  1.16);

\path[draw=drawColor,line width= 0.4pt,line join=round,line cap=round,fill=fillColor] (470.71,535.03) circle (  1.16);

\path[draw=drawColor,line width= 0.4pt,line join=round,line cap=round,fill=fillColor] (470.84,535.03) circle (  1.16);

\path[draw=drawColor,line width= 0.4pt,line join=round,line cap=round,fill=fillColor] (470.98,535.03) circle (  1.16);

\path[draw=drawColor,line width= 0.4pt,line join=round,line cap=round,fill=fillColor] (471.11,535.03) circle (  1.16);

\path[draw=drawColor,line width= 0.4pt,line join=round,line cap=round,fill=fillColor] (471.24,535.03) circle (  1.16);

\path[draw=drawColor,line width= 0.4pt,line join=round,line cap=round,fill=fillColor] (471.38,535.03) circle (  1.16);

\path[draw=drawColor,line width= 0.4pt,line join=round,line cap=round,fill=fillColor] (471.51,535.03) circle (  1.16);

\path[draw=drawColor,line width= 0.4pt,line join=round,line cap=round,fill=fillColor] (471.65,535.03) circle (  1.16);

\path[draw=drawColor,line width= 0.4pt,line join=round,line cap=round,fill=fillColor] (471.78,535.03) circle (  1.16);

\path[draw=drawColor,line width= 0.4pt,line join=round,line cap=round,fill=fillColor] (471.91,535.03) circle (  1.16);

\path[draw=drawColor,line width= 0.4pt,line join=round,line cap=round,fill=fillColor] (472.05,535.03) circle (  1.16);

\path[draw=drawColor,line width= 0.4pt,line join=round,line cap=round,fill=fillColor] (472.18,535.03) circle (  1.16);

\path[draw=drawColor,line width= 0.4pt,line join=round,line cap=round,fill=fillColor] (472.31,535.03) circle (  1.16);

\path[draw=drawColor,line width= 0.4pt,line join=round,line cap=round,fill=fillColor] (472.45,535.03) circle (  1.16);

\path[draw=drawColor,line width= 0.4pt,line join=round,line cap=round,fill=fillColor] (472.58,535.03) circle (  1.16);

\path[draw=drawColor,line width= 0.4pt,line join=round,line cap=round,fill=fillColor] (472.71,535.03) circle (  1.16);

\path[draw=drawColor,line width= 0.4pt,line join=round,line cap=round,fill=fillColor] (472.84,535.03) circle (  1.16);

\path[draw=drawColor,line width= 0.4pt,line join=round,line cap=round,fill=fillColor] (472.97,535.03) circle (  1.16);

\path[draw=drawColor,line width= 0.4pt,line join=round,line cap=round,fill=fillColor] (473.11,535.03) circle (  1.16);

\path[draw=drawColor,line width= 0.4pt,line join=round,line cap=round,fill=fillColor] (473.24,535.03) circle (  1.16);

\path[draw=drawColor,line width= 0.4pt,line join=round,line cap=round,fill=fillColor] (473.37,535.03) circle (  1.16);

\path[draw=drawColor,line width= 0.4pt,line join=round,line cap=round,fill=fillColor] (473.50,535.03) circle (  1.16);

\path[draw=drawColor,line width= 0.4pt,line join=round,line cap=round,fill=fillColor] (473.63,535.03) circle (  1.16);

\path[draw=drawColor,line width= 0.4pt,line join=round,line cap=round,fill=fillColor] (473.76,535.03) circle (  1.16);

\path[draw=drawColor,line width= 0.4pt,line join=round,line cap=round,fill=fillColor] (473.89,535.03) circle (  1.16);

\path[draw=drawColor,line width= 0.4pt,line join=round,line cap=round,fill=fillColor] (474.02,535.03) circle (  1.16);

\path[draw=drawColor,line width= 0.4pt,line join=round,line cap=round,fill=fillColor] (474.15,535.03) circle (  1.16);

\path[draw=drawColor,line width= 0.4pt,line join=round,line cap=round,fill=fillColor] (474.28,535.03) circle (  1.16);

\path[draw=drawColor,line width= 0.4pt,line join=round,line cap=round,fill=fillColor] (474.41,535.03) circle (  1.16);

\path[draw=drawColor,line width= 0.4pt,line join=round,line cap=round,fill=fillColor] (474.54,535.03) circle (  1.16);

\path[draw=drawColor,line width= 0.4pt,line join=round,line cap=round,fill=fillColor] (474.67,535.03) circle (  1.16);

\path[draw=drawColor,line width= 0.4pt,line join=round,line cap=round,fill=fillColor] (474.80,535.03) circle (  1.16);

\path[draw=drawColor,line width= 0.4pt,line join=round,line cap=round,fill=fillColor] (474.93,535.03) circle (  1.16);

\path[draw=drawColor,line width= 0.4pt,line join=round,line cap=round,fill=fillColor] (475.05,535.03) circle (  1.16);

\path[draw=drawColor,line width= 0.4pt,line join=round,line cap=round,fill=fillColor] (475.18,535.03) circle (  1.16);

\path[draw=drawColor,line width= 0.4pt,line join=round,line cap=round,fill=fillColor] (475.31,535.03) circle (  1.16);

\path[draw=drawColor,line width= 0.4pt,line join=round,line cap=round,fill=fillColor] (475.44,535.03) circle (  1.16);

\path[draw=drawColor,line width= 0.4pt,line join=round,line cap=round,fill=fillColor] (475.57,535.03) circle (  1.16);

\path[draw=drawColor,line width= 0.4pt,line join=round,line cap=round,fill=fillColor] (475.69,535.03) circle (  1.16);

\path[draw=drawColor,line width= 0.4pt,line join=round,line cap=round,fill=fillColor] (475.82,535.03) circle (  1.16);

\path[draw=drawColor,line width= 0.4pt,line join=round,line cap=round,fill=fillColor] (475.95,535.03) circle (  1.16);

\path[draw=drawColor,line width= 0.4pt,line join=round,line cap=round,fill=fillColor] (476.08,535.03) circle (  1.16);

\path[draw=drawColor,line width= 0.4pt,line join=round,line cap=round,fill=fillColor] (476.20,535.03) circle (  1.16);

\path[draw=drawColor,line width= 0.4pt,line join=round,line cap=round,fill=fillColor] (476.33,535.03) circle (  1.16);

\path[draw=drawColor,line width= 0.4pt,line join=round,line cap=round,fill=fillColor] (476.46,535.03) circle (  1.16);

\path[draw=drawColor,line width= 0.4pt,line join=round,line cap=round,fill=fillColor] (476.58,535.03) circle (  1.16);

\path[draw=drawColor,line width= 0.4pt,line join=round,line cap=round,fill=fillColor] (476.71,535.03) circle (  1.16);

\path[draw=drawColor,line width= 0.4pt,line join=round,line cap=round,fill=fillColor] (476.83,535.03) circle (  1.16);

\path[draw=drawColor,line width= 0.4pt,line join=round,line cap=round,fill=fillColor] (476.96,535.03) circle (  1.16);

\path[draw=drawColor,line width= 0.4pt,line join=round,line cap=round,fill=fillColor] (477.08,535.03) circle (  1.16);

\path[draw=drawColor,line width= 0.4pt,line join=round,line cap=round,fill=fillColor] (477.21,535.03) circle (  1.16);

\path[draw=drawColor,line width= 0.4pt,line join=round,line cap=round,fill=fillColor] (477.33,535.03) circle (  1.16);

\path[draw=drawColor,line width= 0.4pt,line join=round,line cap=round,fill=fillColor] (477.46,535.03) circle (  1.16);

\path[draw=drawColor,line width= 0.4pt,line join=round,line cap=round,fill=fillColor] (477.58,535.03) circle (  1.16);

\path[draw=drawColor,line width= 0.4pt,line join=round,line cap=round,fill=fillColor] (477.71,535.03) circle (  1.16);

\path[draw=drawColor,line width= 0.4pt,line join=round,line cap=round,fill=fillColor] (477.83,535.03) circle (  1.16);

\path[draw=drawColor,line width= 0.4pt,line join=round,line cap=round,fill=fillColor] (477.96,535.03) circle (  1.16);

\path[draw=drawColor,line width= 0.4pt,line join=round,line cap=round,fill=fillColor] (478.08,535.03) circle (  1.16);

\path[draw=drawColor,line width= 0.4pt,line join=round,line cap=round,fill=fillColor] (478.21,535.03) circle (  1.16);
\definecolor[named]{drawColor}{rgb}{0.65,0.34,0.16}
\definecolor[named]{fillColor}{rgb}{0.65,0.34,0.16}

\path[draw=drawColor,line width= 0.4pt,line join=round,line cap=round,fill=fillColor] (327.82,589.31) circle (  1.16);

\path[draw=drawColor,line width= 0.4pt,line join=round,line cap=round,fill=fillColor] (333.62,584.11) circle (  1.16);

\path[draw=drawColor,line width= 0.4pt,line join=round,line cap=round,fill=fillColor] (337.68,584.06) circle (  1.16);

\path[draw=drawColor,line width= 0.4pt,line join=round,line cap=round,fill=fillColor] (340.92,583.48) circle (  1.16);

\path[draw=drawColor,line width= 0.4pt,line join=round,line cap=round,fill=fillColor] (343.65,582.00) circle (  1.16);

\path[draw=drawColor,line width= 0.4pt,line join=round,line cap=round,fill=fillColor] (346.04,581.90) circle (  1.16);

\path[draw=drawColor,line width= 0.4pt,line join=round,line cap=round,fill=fillColor] (348.17,581.27) circle (  1.16);

\path[draw=drawColor,line width= 0.4pt,line join=round,line cap=round,fill=fillColor] (350.11,581.11) circle (  1.16);

\path[draw=drawColor,line width= 0.4pt,line join=round,line cap=round,fill=fillColor] (351.90,580.91) circle (  1.16);

\path[draw=drawColor,line width= 0.4pt,line join=round,line cap=round,fill=fillColor] (353.55,580.77) circle (  1.16);

\path[draw=drawColor,line width= 0.4pt,line join=round,line cap=round,fill=fillColor] (355.10,579.90) circle (  1.16);

\path[draw=drawColor,line width= 0.4pt,line join=round,line cap=round,fill=fillColor] (356.56,579.23) circle (  1.16);

\path[draw=drawColor,line width= 0.4pt,line join=round,line cap=round,fill=fillColor] (357.94,579.07) circle (  1.16);

\path[draw=drawColor,line width= 0.4pt,line join=round,line cap=round,fill=fillColor] (359.25,578.85) circle (  1.16);

\path[draw=drawColor,line width= 0.4pt,line join=round,line cap=round,fill=fillColor] (360.50,578.56) circle (  1.16);

\path[draw=drawColor,line width= 0.4pt,line join=round,line cap=round,fill=fillColor] (361.70,577.96) circle (  1.16);

\path[draw=drawColor,line width= 0.4pt,line join=round,line cap=round,fill=fillColor] (362.84,577.63) circle (  1.16);

\path[draw=drawColor,line width= 0.4pt,line join=round,line cap=round,fill=fillColor] (363.95,577.20) circle (  1.16);

\path[draw=drawColor,line width= 0.4pt,line join=round,line cap=round,fill=fillColor] (365.01,576.91) circle (  1.16);

\path[draw=drawColor,line width= 0.4pt,line join=round,line cap=round,fill=fillColor] (366.03,576.78) circle (  1.16);

\path[draw=drawColor,line width= 0.4pt,line join=round,line cap=round,fill=fillColor] (367.03,576.71) circle (  1.16);

\path[draw=drawColor,line width= 0.4pt,line join=round,line cap=round,fill=fillColor] (367.99,575.25) circle (  1.16);

\path[draw=drawColor,line width= 0.4pt,line join=round,line cap=round,fill=fillColor] (368.92,574.30) circle (  1.16);

\path[draw=drawColor,line width= 0.4pt,line join=round,line cap=round,fill=fillColor] (369.83,574.29) circle (  1.16);

\path[draw=drawColor,line width= 0.4pt,line join=round,line cap=round,fill=fillColor] (370.71,572.07) circle (  1.16);

\path[draw=drawColor,line width= 0.4pt,line join=round,line cap=round,fill=fillColor] (371.56,572.02) circle (  1.16);

\path[draw=drawColor,line width= 0.4pt,line join=round,line cap=round,fill=fillColor] (372.40,571.63) circle (  1.16);

\path[draw=drawColor,line width= 0.4pt,line join=round,line cap=round,fill=fillColor] (373.21,571.52) circle (  1.16);

\path[draw=drawColor,line width= 0.4pt,line join=round,line cap=round,fill=fillColor] (374.01,571.49) circle (  1.16);

\path[draw=drawColor,line width= 0.4pt,line join=round,line cap=round,fill=fillColor] (374.79,570.95) circle (  1.16);

\path[draw=drawColor,line width= 0.4pt,line join=round,line cap=round,fill=fillColor] (375.55,570.94) circle (  1.16);

\path[draw=drawColor,line width= 0.4pt,line join=round,line cap=round,fill=fillColor] (376.30,570.87) circle (  1.16);

\path[draw=drawColor,line width= 0.4pt,line join=round,line cap=round,fill=fillColor] (377.02,570.62) circle (  1.16);

\path[draw=drawColor,line width= 0.4pt,line join=round,line cap=round,fill=fillColor] (377.74,570.56) circle (  1.16);

\path[draw=drawColor,line width= 0.4pt,line join=round,line cap=round,fill=fillColor] (378.44,570.48) circle (  1.16);

\path[draw=drawColor,line width= 0.4pt,line join=round,line cap=round,fill=fillColor] (379.13,570.19) circle (  1.16);

\path[draw=drawColor,line width= 0.4pt,line join=round,line cap=round,fill=fillColor] (379.80,570.16) circle (  1.16);

\path[draw=drawColor,line width= 0.4pt,line join=round,line cap=round,fill=fillColor] (380.47,569.96) circle (  1.16);

\path[draw=drawColor,line width= 0.4pt,line join=round,line cap=round,fill=fillColor] (381.12,569.83) circle (  1.16);

\path[draw=drawColor,line width= 0.4pt,line join=round,line cap=round,fill=fillColor] (381.76,569.77) circle (  1.16);

\path[draw=drawColor,line width= 0.4pt,line join=round,line cap=round,fill=fillColor] (382.39,569.64) circle (  1.16);

\path[draw=drawColor,line width= 0.4pt,line join=round,line cap=round,fill=fillColor] (383.01,569.63) circle (  1.16);

\path[draw=drawColor,line width= 0.4pt,line join=round,line cap=round,fill=fillColor] (383.62,569.49) circle (  1.16);

\path[draw=drawColor,line width= 0.4pt,line join=round,line cap=round,fill=fillColor] (384.22,569.15) circle (  1.16);

\path[draw=drawColor,line width= 0.4pt,line join=round,line cap=round,fill=fillColor] (384.81,569.07) circle (  1.16);

\path[draw=drawColor,line width= 0.4pt,line join=round,line cap=round,fill=fillColor] (385.39,568.62) circle (  1.16);

\path[draw=drawColor,line width= 0.4pt,line join=round,line cap=round,fill=fillColor] (385.97,568.36) circle (  1.16);

\path[draw=drawColor,line width= 0.4pt,line join=round,line cap=round,fill=fillColor] (386.54,567.85) circle (  1.16);

\path[draw=drawColor,line width= 0.4pt,line join=round,line cap=round,fill=fillColor] (387.09,567.76) circle (  1.16);

\path[draw=drawColor,line width= 0.4pt,line join=round,line cap=round,fill=fillColor] (387.64,567.67) circle (  1.16);

\path[draw=drawColor,line width= 0.4pt,line join=round,line cap=round,fill=fillColor] (388.19,567.56) circle (  1.16);

\path[draw=drawColor,line width= 0.4pt,line join=round,line cap=round,fill=fillColor] (388.73,567.53) circle (  1.16);

\path[draw=drawColor,line width= 0.4pt,line join=round,line cap=round,fill=fillColor] (389.26,567.06) circle (  1.16);

\path[draw=drawColor,line width= 0.4pt,line join=round,line cap=round,fill=fillColor] (389.78,566.56) circle (  1.16);

\path[draw=drawColor,line width= 0.4pt,line join=round,line cap=round,fill=fillColor] (390.30,566.48) circle (  1.16);

\path[draw=drawColor,line width= 0.4pt,line join=round,line cap=round,fill=fillColor] (390.81,566.41) circle (  1.16);

\path[draw=drawColor,line width= 0.4pt,line join=round,line cap=round,fill=fillColor] (391.31,566.36) circle (  1.16);

\path[draw=drawColor,line width= 0.4pt,line join=round,line cap=round,fill=fillColor] (391.81,566.31) circle (  1.16);

\path[draw=drawColor,line width= 0.4pt,line join=round,line cap=round,fill=fillColor] (392.30,565.76) circle (  1.16);

\path[draw=drawColor,line width= 0.4pt,line join=round,line cap=round,fill=fillColor] (392.79,565.51) circle (  1.16);

\path[draw=drawColor,line width= 0.4pt,line join=round,line cap=round,fill=fillColor] (393.27,565.10) circle (  1.16);

\path[draw=drawColor,line width= 0.4pt,line join=round,line cap=round,fill=fillColor] (393.75,564.72) circle (  1.16);

\path[draw=drawColor,line width= 0.4pt,line join=round,line cap=round,fill=fillColor] (394.22,564.54) circle (  1.16);

\path[draw=drawColor,line width= 0.4pt,line join=round,line cap=round,fill=fillColor] (394.69,564.48) circle (  1.16);

\path[draw=drawColor,line width= 0.4pt,line join=round,line cap=round,fill=fillColor] (395.15,564.12) circle (  1.16);

\path[draw=drawColor,line width= 0.4pt,line join=round,line cap=round,fill=fillColor] (395.61,564.10) circle (  1.16);

\path[draw=drawColor,line width= 0.4pt,line join=round,line cap=round,fill=fillColor] (396.06,563.93) circle (  1.16);

\path[draw=drawColor,line width= 0.4pt,line join=round,line cap=round,fill=fillColor] (396.51,563.80) circle (  1.16);

\path[draw=drawColor,line width= 0.4pt,line join=round,line cap=round,fill=fillColor] (396.95,563.60) circle (  1.16);

\path[draw=drawColor,line width= 0.4pt,line join=round,line cap=round,fill=fillColor] (397.39,563.38) circle (  1.16);

\path[draw=drawColor,line width= 0.4pt,line join=round,line cap=round,fill=fillColor] (397.83,563.30) circle (  1.16);

\path[draw=drawColor,line width= 0.4pt,line join=round,line cap=round,fill=fillColor] (398.26,563.18) circle (  1.16);

\path[draw=drawColor,line width= 0.4pt,line join=round,line cap=round,fill=fillColor] (398.68,563.15) circle (  1.16);

\path[draw=drawColor,line width= 0.4pt,line join=round,line cap=round,fill=fillColor] (399.11,562.26) circle (  1.16);

\path[draw=drawColor,line width= 0.4pt,line join=round,line cap=round,fill=fillColor] (399.53,562.21) circle (  1.16);

\path[draw=drawColor,line width= 0.4pt,line join=round,line cap=round,fill=fillColor] (399.94,561.82) circle (  1.16);

\path[draw=drawColor,line width= 0.4pt,line join=round,line cap=round,fill=fillColor] (400.36,561.69) circle (  1.16);

\path[draw=drawColor,line width= 0.4pt,line join=round,line cap=round,fill=fillColor] (400.76,561.51) circle (  1.16);

\path[draw=drawColor,line width= 0.4pt,line join=round,line cap=round,fill=fillColor] (401.17,561.10) circle (  1.16);

\path[draw=drawColor,line width= 0.4pt,line join=round,line cap=round,fill=fillColor] (401.57,560.78) circle (  1.16);

\path[draw=drawColor,line width= 0.4pt,line join=round,line cap=round,fill=fillColor] (401.97,560.63) circle (  1.16);

\path[draw=drawColor,line width= 0.4pt,line join=round,line cap=round,fill=fillColor] (402.36,560.57) circle (  1.16);

\path[draw=drawColor,line width= 0.4pt,line join=round,line cap=round,fill=fillColor] (402.76,560.14) circle (  1.16);

\path[draw=drawColor,line width= 0.4pt,line join=round,line cap=round,fill=fillColor] (403.15,560.06) circle (  1.16);

\path[draw=drawColor,line width= 0.4pt,line join=round,line cap=round,fill=fillColor] (403.53,560.02) circle (  1.16);

\path[draw=drawColor,line width= 0.4pt,line join=round,line cap=round,fill=fillColor] (403.91,559.98) circle (  1.16);

\path[draw=drawColor,line width= 0.4pt,line join=round,line cap=round,fill=fillColor] (404.29,559.83) circle (  1.16);

\path[draw=drawColor,line width= 0.4pt,line join=round,line cap=round,fill=fillColor] (404.67,559.44) circle (  1.16);

\path[draw=drawColor,line width= 0.4pt,line join=round,line cap=round,fill=fillColor] (405.05,559.41) circle (  1.16);

\path[draw=drawColor,line width= 0.4pt,line join=round,line cap=round,fill=fillColor] (405.42,559.28) circle (  1.16);

\path[draw=drawColor,line width= 0.4pt,line join=round,line cap=round,fill=fillColor] (405.78,558.81) circle (  1.16);

\path[draw=drawColor,line width= 0.4pt,line join=round,line cap=round,fill=fillColor] (406.15,558.75) circle (  1.16);

\path[draw=drawColor,line width= 0.4pt,line join=round,line cap=round,fill=fillColor] (406.51,558.72) circle (  1.16);

\path[draw=drawColor,line width= 0.4pt,line join=round,line cap=round,fill=fillColor] (406.87,558.46) circle (  1.16);

\path[draw=drawColor,line width= 0.4pt,line join=round,line cap=round,fill=fillColor] (407.23,558.42) circle (  1.16);

\path[draw=drawColor,line width= 0.4pt,line join=round,line cap=round,fill=fillColor] (407.59,558.32) circle (  1.16);

\path[draw=drawColor,line width= 0.4pt,line join=round,line cap=round,fill=fillColor] (407.94,558.26) circle (  1.16);

\path[draw=drawColor,line width= 0.4pt,line join=round,line cap=round,fill=fillColor] (408.29,557.83) circle (  1.16);

\path[draw=drawColor,line width= 0.4pt,line join=round,line cap=round,fill=fillColor] (408.64,557.27) circle (  1.16);

\path[draw=drawColor,line width= 0.4pt,line join=round,line cap=round,fill=fillColor] (408.99,557.15) circle (  1.16);

\path[draw=drawColor,line width= 0.4pt,line join=round,line cap=round,fill=fillColor] (409.33,557.13) circle (  1.16);

\path[draw=drawColor,line width= 0.4pt,line join=round,line cap=round,fill=fillColor] (409.67,557.03) circle (  1.16);

\path[draw=drawColor,line width= 0.4pt,line join=round,line cap=round,fill=fillColor] (410.01,556.65) circle (  1.16);

\path[draw=drawColor,line width= 0.4pt,line join=round,line cap=round,fill=fillColor] (410.35,556.48) circle (  1.16);

\path[draw=drawColor,line width= 0.4pt,line join=round,line cap=round,fill=fillColor] (410.68,556.35) circle (  1.16);

\path[draw=drawColor,line width= 0.4pt,line join=round,line cap=round,fill=fillColor] (411.02,556.32) circle (  1.16);

\path[draw=drawColor,line width= 0.4pt,line join=round,line cap=round,fill=fillColor] (411.35,556.30) circle (  1.16);

\path[draw=drawColor,line width= 0.4pt,line join=round,line cap=round,fill=fillColor] (411.67,556.01) circle (  1.16);

\path[draw=drawColor,line width= 0.4pt,line join=round,line cap=round,fill=fillColor] (412.00,555.91) circle (  1.16);

\path[draw=drawColor,line width= 0.4pt,line join=round,line cap=round,fill=fillColor] (412.33,555.52) circle (  1.16);

\path[draw=drawColor,line width= 0.4pt,line join=round,line cap=round,fill=fillColor] (412.65,555.48) circle (  1.16);

\path[draw=drawColor,line width= 0.4pt,line join=round,line cap=round,fill=fillColor] (412.97,555.47) circle (  1.16);

\path[draw=drawColor,line width= 0.4pt,line join=round,line cap=round,fill=fillColor] (413.29,554.97) circle (  1.16);

\path[draw=drawColor,line width= 0.4pt,line join=round,line cap=round,fill=fillColor] (413.60,554.66) circle (  1.16);

\path[draw=drawColor,line width= 0.4pt,line join=round,line cap=round,fill=fillColor] (413.92,554.20) circle (  1.16);

\path[draw=drawColor,line width= 0.4pt,line join=round,line cap=round,fill=fillColor] (414.23,553.93) circle (  1.16);

\path[draw=drawColor,line width= 0.4pt,line join=round,line cap=round,fill=fillColor] (414.54,553.84) circle (  1.16);

\path[draw=drawColor,line width= 0.4pt,line join=round,line cap=round,fill=fillColor] (414.85,553.84) circle (  1.16);

\path[draw=drawColor,line width= 0.4pt,line join=round,line cap=round,fill=fillColor] (415.16,553.64) circle (  1.16);

\path[draw=drawColor,line width= 0.4pt,line join=round,line cap=round,fill=fillColor] (415.47,553.45) circle (  1.16);

\path[draw=drawColor,line width= 0.4pt,line join=round,line cap=round,fill=fillColor] (415.77,553.34) circle (  1.16);

\path[draw=drawColor,line width= 0.4pt,line join=round,line cap=round,fill=fillColor] (416.08,553.20) circle (  1.16);

\path[draw=drawColor,line width= 0.4pt,line join=round,line cap=round,fill=fillColor] (416.38,552.65) circle (  1.16);

\path[draw=drawColor,line width= 0.4pt,line join=round,line cap=round,fill=fillColor] (416.68,552.57) circle (  1.16);

\path[draw=drawColor,line width= 0.4pt,line join=round,line cap=round,fill=fillColor] (416.97,552.43) circle (  1.16);

\path[draw=drawColor,line width= 0.4pt,line join=round,line cap=round,fill=fillColor] (417.27,552.37) circle (  1.16);

\path[draw=drawColor,line width= 0.4pt,line join=round,line cap=round,fill=fillColor] (417.57,552.35) circle (  1.16);

\path[draw=drawColor,line width= 0.4pt,line join=round,line cap=round,fill=fillColor] (417.86,552.28) circle (  1.16);

\path[draw=drawColor,line width= 0.4pt,line join=round,line cap=round,fill=fillColor] (418.15,552.00) circle (  1.16);

\path[draw=drawColor,line width= 0.4pt,line join=round,line cap=round,fill=fillColor] (418.44,551.88) circle (  1.16);

\path[draw=drawColor,line width= 0.4pt,line join=round,line cap=round,fill=fillColor] (418.73,551.75) circle (  1.16);

\path[draw=drawColor,line width= 0.4pt,line join=round,line cap=round,fill=fillColor] (419.02,551.55) circle (  1.16);

\path[draw=drawColor,line width= 0.4pt,line join=round,line cap=round,fill=fillColor] (419.30,551.21) circle (  1.16);

\path[draw=drawColor,line width= 0.4pt,line join=round,line cap=round,fill=fillColor] (419.59,551.19) circle (  1.16);

\path[draw=drawColor,line width= 0.4pt,line join=round,line cap=round,fill=fillColor] (419.87,551.16) circle (  1.16);

\path[draw=drawColor,line width= 0.4pt,line join=round,line cap=round,fill=fillColor] (420.15,551.16) circle (  1.16);

\path[draw=drawColor,line width= 0.4pt,line join=round,line cap=round,fill=fillColor] (420.43,550.25) circle (  1.16);

\path[draw=drawColor,line width= 0.4pt,line join=round,line cap=round,fill=fillColor] (420.71,550.08) circle (  1.16);

\path[draw=drawColor,line width= 0.4pt,line join=round,line cap=round,fill=fillColor] (420.99,549.87) circle (  1.16);

\path[draw=drawColor,line width= 0.4pt,line join=round,line cap=round,fill=fillColor] (421.26,549.65) circle (  1.16);

\path[draw=drawColor,line width= 0.4pt,line join=round,line cap=round,fill=fillColor] (421.54,549.57) circle (  1.16);

\path[draw=drawColor,line width= 0.4pt,line join=round,line cap=round,fill=fillColor] (421.81,549.50) circle (  1.16);

\path[draw=drawColor,line width= 0.4pt,line join=round,line cap=round,fill=fillColor] (422.09,549.30) circle (  1.16);

\path[draw=drawColor,line width= 0.4pt,line join=round,line cap=round,fill=fillColor] (422.36,549.20) circle (  1.16);

\path[draw=drawColor,line width= 0.4pt,line join=round,line cap=round,fill=fillColor] (422.63,549.07) circle (  1.16);

\path[draw=drawColor,line width= 0.4pt,line join=round,line cap=round,fill=fillColor] (422.90,548.96) circle (  1.16);

\path[draw=drawColor,line width= 0.4pt,line join=round,line cap=round,fill=fillColor] (423.16,548.89) circle (  1.16);

\path[draw=drawColor,line width= 0.4pt,line join=round,line cap=round,fill=fillColor] (423.43,548.50) circle (  1.16);

\path[draw=drawColor,line width= 0.4pt,line join=round,line cap=round,fill=fillColor] (423.69,548.28) circle (  1.16);

\path[draw=drawColor,line width= 0.4pt,line join=round,line cap=round,fill=fillColor] (423.96,548.08) circle (  1.16);

\path[draw=drawColor,line width= 0.4pt,line join=round,line cap=round,fill=fillColor] (424.22,547.77) circle (  1.16);

\path[draw=drawColor,line width= 0.4pt,line join=round,line cap=round,fill=fillColor] (424.48,547.67) circle (  1.16);

\path[draw=drawColor,line width= 0.4pt,line join=round,line cap=round,fill=fillColor] (424.74,547.49) circle (  1.16);

\path[draw=drawColor,line width= 0.4pt,line join=round,line cap=round,fill=fillColor] (425.00,547.31) circle (  1.16);

\path[draw=drawColor,line width= 0.4pt,line join=round,line cap=round,fill=fillColor] (425.26,546.56) circle (  1.16);

\path[draw=drawColor,line width= 0.4pt,line join=round,line cap=round,fill=fillColor] (425.52,546.44) circle (  1.16);

\path[draw=drawColor,line width= 0.4pt,line join=round,line cap=round,fill=fillColor] (425.77,545.86) circle (  1.16);

\path[draw=drawColor,line width= 0.4pt,line join=round,line cap=round,fill=fillColor] (426.03,545.21) circle (  1.16);

\path[draw=drawColor,line width= 0.4pt,line join=round,line cap=round,fill=fillColor] (426.28,544.57) circle (  1.16);

\path[draw=drawColor,line width= 0.4pt,line join=round,line cap=round,fill=fillColor] (426.53,544.35) circle (  1.16);

\path[draw=drawColor,line width= 0.4pt,line join=round,line cap=round,fill=fillColor] (426.78,544.23) circle (  1.16);

\path[draw=drawColor,line width= 0.4pt,line join=round,line cap=round,fill=fillColor] (427.03,543.33) circle (  1.16);

\path[draw=drawColor,line width= 0.4pt,line join=round,line cap=round,fill=fillColor] (427.28,543.25) circle (  1.16);

\path[draw=drawColor,line width= 0.4pt,line join=round,line cap=round,fill=fillColor] (427.53,542.77) circle (  1.16);

\path[draw=drawColor,line width= 0.4pt,line join=round,line cap=round,fill=fillColor] (427.78,542.32) circle (  1.16);

\path[draw=drawColor,line width= 0.4pt,line join=round,line cap=round,fill=fillColor] (428.03,541.96) circle (  1.16);

\path[draw=drawColor,line width= 0.4pt,line join=round,line cap=round,fill=fillColor] (428.27,541.48) circle (  1.16);

\path[draw=drawColor,line width= 0.4pt,line join=round,line cap=round,fill=fillColor] (428.52,535.03) circle (  1.16);

\path[draw=drawColor,line width= 0.4pt,line join=round,line cap=round,fill=fillColor] (428.76,535.03) circle (  1.16);

\path[draw=drawColor,line width= 0.4pt,line join=round,line cap=round,fill=fillColor] (429.00,535.03) circle (  1.16);

\path[draw=drawColor,line width= 0.4pt,line join=round,line cap=round,fill=fillColor] (429.24,535.03) circle (  1.16);

\path[draw=drawColor,line width= 0.4pt,line join=round,line cap=round,fill=fillColor] (429.48,535.03) circle (  1.16);

\path[draw=drawColor,line width= 0.4pt,line join=round,line cap=round,fill=fillColor] (429.72,535.03) circle (  1.16);

\path[draw=drawColor,line width= 0.4pt,line join=round,line cap=round,fill=fillColor] (429.96,535.03) circle (  1.16);

\path[draw=drawColor,line width= 0.4pt,line join=round,line cap=round,fill=fillColor] (430.20,535.03) circle (  1.16);

\path[draw=drawColor,line width= 0.4pt,line join=round,line cap=round,fill=fillColor] (430.44,535.03) circle (  1.16);

\path[draw=drawColor,line width= 0.4pt,line join=round,line cap=round,fill=fillColor] (430.67,535.03) circle (  1.16);

\path[draw=drawColor,line width= 0.4pt,line join=round,line cap=round,fill=fillColor] (430.91,535.03) circle (  1.16);

\path[draw=drawColor,line width= 0.4pt,line join=round,line cap=round,fill=fillColor] (431.14,535.03) circle (  1.16);

\path[draw=drawColor,line width= 0.4pt,line join=round,line cap=round,fill=fillColor] (431.38,535.03) circle (  1.16);

\path[draw=drawColor,line width= 0.4pt,line join=round,line cap=round,fill=fillColor] (431.61,535.03) circle (  1.16);

\path[draw=drawColor,line width= 0.4pt,line join=round,line cap=round,fill=fillColor] (431.84,535.03) circle (  1.16);

\path[draw=drawColor,line width= 0.4pt,line join=round,line cap=round,fill=fillColor] (432.07,535.03) circle (  1.16);

\path[draw=drawColor,line width= 0.4pt,line join=round,line cap=round,fill=fillColor] (432.30,535.03) circle (  1.16);

\path[draw=drawColor,line width= 0.4pt,line join=round,line cap=round,fill=fillColor] (432.53,535.03) circle (  1.16);

\path[draw=drawColor,line width= 0.4pt,line join=round,line cap=round,fill=fillColor] (432.76,535.03) circle (  1.16);

\path[draw=drawColor,line width= 0.4pt,line join=round,line cap=round,fill=fillColor] (432.99,535.03) circle (  1.16);

\path[draw=drawColor,line width= 0.4pt,line join=round,line cap=round,fill=fillColor] (433.21,535.03) circle (  1.16);

\path[draw=drawColor,line width= 0.4pt,line join=round,line cap=round,fill=fillColor] (433.44,535.03) circle (  1.16);

\path[draw=drawColor,line width= 0.4pt,line join=round,line cap=round,fill=fillColor] (433.67,535.03) circle (  1.16);

\path[draw=drawColor,line width= 0.4pt,line join=round,line cap=round,fill=fillColor] (433.89,535.03) circle (  1.16);

\path[draw=drawColor,line width= 0.4pt,line join=round,line cap=round,fill=fillColor] (434.11,535.03) circle (  1.16);

\path[draw=drawColor,line width= 0.4pt,line join=round,line cap=round,fill=fillColor] (434.34,535.03) circle (  1.16);

\path[draw=drawColor,line width= 0.4pt,line join=round,line cap=round,fill=fillColor] (434.56,535.03) circle (  1.16);

\path[draw=drawColor,line width= 0.4pt,line join=round,line cap=round,fill=fillColor] (434.78,535.03) circle (  1.16);

\path[draw=drawColor,line width= 0.4pt,line join=round,line cap=round,fill=fillColor] (435.00,535.03) circle (  1.16);

\path[draw=drawColor,line width= 0.4pt,line join=round,line cap=round,fill=fillColor] (435.22,535.03) circle (  1.16);

\path[draw=drawColor,line width= 0.4pt,line join=round,line cap=round,fill=fillColor] (435.44,535.03) circle (  1.16);

\path[draw=drawColor,line width= 0.4pt,line join=round,line cap=round,fill=fillColor] (435.66,535.03) circle (  1.16);

\path[draw=drawColor,line width= 0.4pt,line join=round,line cap=round,fill=fillColor] (435.88,535.03) circle (  1.16);

\path[draw=drawColor,line width= 0.4pt,line join=round,line cap=round,fill=fillColor] (436.09,535.03) circle (  1.16);

\path[draw=drawColor,line width= 0.4pt,line join=round,line cap=round,fill=fillColor] (436.31,535.03) circle (  1.16);

\path[draw=drawColor,line width= 0.4pt,line join=round,line cap=round,fill=fillColor] (436.52,535.03) circle (  1.16);

\path[draw=drawColor,line width= 0.4pt,line join=round,line cap=round,fill=fillColor] (436.74,535.03) circle (  1.16);

\path[draw=drawColor,line width= 0.4pt,line join=round,line cap=round,fill=fillColor] (436.95,535.03) circle (  1.16);

\path[draw=drawColor,line width= 0.4pt,line join=round,line cap=round,fill=fillColor] (437.17,535.03) circle (  1.16);

\path[draw=drawColor,line width= 0.4pt,line join=round,line cap=round,fill=fillColor] (437.38,535.03) circle (  1.16);

\path[draw=drawColor,line width= 0.4pt,line join=round,line cap=round,fill=fillColor] (437.59,535.03) circle (  1.16);

\path[draw=drawColor,line width= 0.4pt,line join=round,line cap=round,fill=fillColor] (437.80,535.03) circle (  1.16);

\path[draw=drawColor,line width= 0.4pt,line join=round,line cap=round,fill=fillColor] (438.01,535.03) circle (  1.16);

\path[draw=drawColor,line width= 0.4pt,line join=round,line cap=round,fill=fillColor] (438.22,535.03) circle (  1.16);

\path[draw=drawColor,line width= 0.4pt,line join=round,line cap=round,fill=fillColor] (438.43,535.03) circle (  1.16);

\path[draw=drawColor,line width= 0.4pt,line join=round,line cap=round,fill=fillColor] (438.64,535.03) circle (  1.16);

\path[draw=drawColor,line width= 0.4pt,line join=round,line cap=round,fill=fillColor] (438.85,535.03) circle (  1.16);

\path[draw=drawColor,line width= 0.4pt,line join=round,line cap=round,fill=fillColor] (439.06,535.03) circle (  1.16);

\path[draw=drawColor,line width= 0.4pt,line join=round,line cap=round,fill=fillColor] (439.26,535.03) circle (  1.16);

\path[draw=drawColor,line width= 0.4pt,line join=round,line cap=round,fill=fillColor] (439.47,535.03) circle (  1.16);

\path[draw=drawColor,line width= 0.4pt,line join=round,line cap=round,fill=fillColor] (439.67,535.03) circle (  1.16);

\path[draw=drawColor,line width= 0.4pt,line join=round,line cap=round,fill=fillColor] (439.88,535.03) circle (  1.16);

\path[draw=drawColor,line width= 0.4pt,line join=round,line cap=round,fill=fillColor] (440.08,535.03) circle (  1.16);

\path[draw=drawColor,line width= 0.4pt,line join=round,line cap=round,fill=fillColor] (440.29,535.03) circle (  1.16);

\path[draw=drawColor,line width= 0.4pt,line join=round,line cap=round,fill=fillColor] (440.49,535.03) circle (  1.16);

\path[draw=drawColor,line width= 0.4pt,line join=round,line cap=round,fill=fillColor] (440.69,535.03) circle (  1.16);

\path[draw=drawColor,line width= 0.4pt,line join=round,line cap=round,fill=fillColor] (440.89,535.03) circle (  1.16);

\path[draw=drawColor,line width= 0.4pt,line join=round,line cap=round,fill=fillColor] (441.09,535.03) circle (  1.16);

\path[draw=drawColor,line width= 0.4pt,line join=round,line cap=round,fill=fillColor] (441.29,535.03) circle (  1.16);

\path[draw=drawColor,line width= 0.4pt,line join=round,line cap=round,fill=fillColor] (441.49,535.03) circle (  1.16);

\path[draw=drawColor,line width= 0.4pt,line join=round,line cap=round,fill=fillColor] (441.69,535.03) circle (  1.16);

\path[draw=drawColor,line width= 0.4pt,line join=round,line cap=round,fill=fillColor] (441.89,535.03) circle (  1.16);

\path[draw=drawColor,line width= 0.4pt,line join=round,line cap=round,fill=fillColor] (442.09,535.03) circle (  1.16);

\path[draw=drawColor,line width= 0.4pt,line join=round,line cap=round,fill=fillColor] (442.29,535.03) circle (  1.16);

\path[draw=drawColor,line width= 0.4pt,line join=round,line cap=round,fill=fillColor] (442.49,535.03) circle (  1.16);

\path[draw=drawColor,line width= 0.4pt,line join=round,line cap=round,fill=fillColor] (442.68,535.03) circle (  1.16);

\path[draw=drawColor,line width= 0.4pt,line join=round,line cap=round,fill=fillColor] (442.88,535.03) circle (  1.16);

\path[draw=drawColor,line width= 0.4pt,line join=round,line cap=round,fill=fillColor] (443.07,535.03) circle (  1.16);

\path[draw=drawColor,line width= 0.4pt,line join=round,line cap=round,fill=fillColor] (443.27,535.03) circle (  1.16);

\path[draw=drawColor,line width= 0.4pt,line join=round,line cap=round,fill=fillColor] (443.46,535.03) circle (  1.16);

\path[draw=drawColor,line width= 0.4pt,line join=round,line cap=round,fill=fillColor] (443.66,535.03) circle (  1.16);

\path[draw=drawColor,line width= 0.4pt,line join=round,line cap=round,fill=fillColor] (443.85,535.03) circle (  1.16);

\path[draw=drawColor,line width= 0.4pt,line join=round,line cap=round,fill=fillColor] (444.04,535.03) circle (  1.16);

\path[draw=drawColor,line width= 0.4pt,line join=round,line cap=round,fill=fillColor] (444.23,535.03) circle (  1.16);

\path[draw=drawColor,line width= 0.4pt,line join=round,line cap=round,fill=fillColor] (444.43,535.03) circle (  1.16);

\path[draw=drawColor,line width= 0.4pt,line join=round,line cap=round,fill=fillColor] (444.62,535.03) circle (  1.16);

\path[draw=drawColor,line width= 0.4pt,line join=round,line cap=round,fill=fillColor] (444.81,535.03) circle (  1.16);

\path[draw=drawColor,line width= 0.4pt,line join=round,line cap=round,fill=fillColor] (445.00,535.03) circle (  1.16);

\path[draw=drawColor,line width= 0.4pt,line join=round,line cap=round,fill=fillColor] (445.19,535.03) circle (  1.16);

\path[draw=drawColor,line width= 0.4pt,line join=round,line cap=round,fill=fillColor] (445.38,535.03) circle (  1.16);

\path[draw=drawColor,line width= 0.4pt,line join=round,line cap=round,fill=fillColor] (445.56,535.03) circle (  1.16);

\path[draw=drawColor,line width= 0.4pt,line join=round,line cap=round,fill=fillColor] (445.75,535.03) circle (  1.16);

\path[draw=drawColor,line width= 0.4pt,line join=round,line cap=round,fill=fillColor] (445.94,535.03) circle (  1.16);

\path[draw=drawColor,line width= 0.4pt,line join=round,line cap=round,fill=fillColor] (446.13,535.03) circle (  1.16);

\path[draw=drawColor,line width= 0.4pt,line join=round,line cap=round,fill=fillColor] (446.31,535.03) circle (  1.16);

\path[draw=drawColor,line width= 0.4pt,line join=round,line cap=round,fill=fillColor] (446.50,535.03) circle (  1.16);

\path[draw=drawColor,line width= 0.4pt,line join=round,line cap=round,fill=fillColor] (446.68,535.03) circle (  1.16);

\path[draw=drawColor,line width= 0.4pt,line join=round,line cap=round,fill=fillColor] (446.87,535.03) circle (  1.16);

\path[draw=drawColor,line width= 0.4pt,line join=round,line cap=round,fill=fillColor] (447.05,535.03) circle (  1.16);

\path[draw=drawColor,line width= 0.4pt,line join=round,line cap=round,fill=fillColor] (447.24,535.03) circle (  1.16);

\path[draw=drawColor,line width= 0.4pt,line join=round,line cap=round,fill=fillColor] (447.42,535.03) circle (  1.16);

\path[draw=drawColor,line width= 0.4pt,line join=round,line cap=round,fill=fillColor] (447.60,535.03) circle (  1.16);

\path[draw=drawColor,line width= 0.4pt,line join=round,line cap=round,fill=fillColor] (447.79,535.03) circle (  1.16);

\path[draw=drawColor,line width= 0.4pt,line join=round,line cap=round,fill=fillColor] (447.97,535.03) circle (  1.16);

\path[draw=drawColor,line width= 0.4pt,line join=round,line cap=round,fill=fillColor] (448.15,535.03) circle (  1.16);

\path[draw=drawColor,line width= 0.4pt,line join=round,line cap=round,fill=fillColor] (448.33,535.03) circle (  1.16);

\path[draw=drawColor,line width= 0.4pt,line join=round,line cap=round,fill=fillColor] (448.51,535.03) circle (  1.16);

\path[draw=drawColor,line width= 0.4pt,line join=round,line cap=round,fill=fillColor] (448.69,535.03) circle (  1.16);

\path[draw=drawColor,line width= 0.4pt,line join=round,line cap=round,fill=fillColor] (448.87,535.03) circle (  1.16);

\path[draw=drawColor,line width= 0.4pt,line join=round,line cap=round,fill=fillColor] (449.05,535.03) circle (  1.16);

\path[draw=drawColor,line width= 0.4pt,line join=round,line cap=round,fill=fillColor] (449.23,535.03) circle (  1.16);

\path[draw=drawColor,line width= 0.4pt,line join=round,line cap=round,fill=fillColor] (449.41,535.03) circle (  1.16);

\path[draw=drawColor,line width= 0.4pt,line join=round,line cap=round,fill=fillColor] (449.59,535.03) circle (  1.16);

\path[draw=drawColor,line width= 0.4pt,line join=round,line cap=round,fill=fillColor] (449.77,535.03) circle (  1.16);

\path[draw=drawColor,line width= 0.4pt,line join=round,line cap=round,fill=fillColor] (449.94,535.03) circle (  1.16);

\path[draw=drawColor,line width= 0.4pt,line join=round,line cap=round,fill=fillColor] (450.12,535.03) circle (  1.16);

\path[draw=drawColor,line width= 0.4pt,line join=round,line cap=round,fill=fillColor] (450.30,535.03) circle (  1.16);

\path[draw=drawColor,line width= 0.4pt,line join=round,line cap=round,fill=fillColor] (450.47,535.03) circle (  1.16);

\path[draw=drawColor,line width= 0.4pt,line join=round,line cap=round,fill=fillColor] (450.65,535.03) circle (  1.16);

\path[draw=drawColor,line width= 0.4pt,line join=round,line cap=round,fill=fillColor] (450.82,535.03) circle (  1.16);

\path[draw=drawColor,line width= 0.4pt,line join=round,line cap=round,fill=fillColor] (451.00,535.03) circle (  1.16);

\path[draw=drawColor,line width= 0.4pt,line join=round,line cap=round,fill=fillColor] (451.17,535.03) circle (  1.16);

\path[draw=drawColor,line width= 0.4pt,line join=round,line cap=round,fill=fillColor] (451.35,535.03) circle (  1.16);

\path[draw=drawColor,line width= 0.4pt,line join=round,line cap=round,fill=fillColor] (451.52,535.03) circle (  1.16);

\path[draw=drawColor,line width= 0.4pt,line join=round,line cap=round,fill=fillColor] (451.69,535.03) circle (  1.16);

\path[draw=drawColor,line width= 0.4pt,line join=round,line cap=round,fill=fillColor] (451.86,535.03) circle (  1.16);

\path[draw=drawColor,line width= 0.4pt,line join=round,line cap=round,fill=fillColor] (452.04,535.03) circle (  1.16);

\path[draw=drawColor,line width= 0.4pt,line join=round,line cap=round,fill=fillColor] (452.21,535.03) circle (  1.16);

\path[draw=drawColor,line width= 0.4pt,line join=round,line cap=round,fill=fillColor] (452.38,535.03) circle (  1.16);

\path[draw=drawColor,line width= 0.4pt,line join=round,line cap=round,fill=fillColor] (452.55,535.03) circle (  1.16);

\path[draw=drawColor,line width= 0.4pt,line join=round,line cap=round,fill=fillColor] (452.72,535.03) circle (  1.16);

\path[draw=drawColor,line width= 0.4pt,line join=round,line cap=round,fill=fillColor] (452.89,535.03) circle (  1.16);

\path[draw=drawColor,line width= 0.4pt,line join=round,line cap=round,fill=fillColor] (453.06,535.03) circle (  1.16);

\path[draw=drawColor,line width= 0.4pt,line join=round,line cap=round,fill=fillColor] (453.23,535.03) circle (  1.16);

\path[draw=drawColor,line width= 0.4pt,line join=round,line cap=round,fill=fillColor] (453.40,535.03) circle (  1.16);

\path[draw=drawColor,line width= 0.4pt,line join=round,line cap=round,fill=fillColor] (453.57,535.03) circle (  1.16);

\path[draw=drawColor,line width= 0.4pt,line join=round,line cap=round,fill=fillColor] (453.74,535.03) circle (  1.16);

\path[draw=drawColor,line width= 0.4pt,line join=round,line cap=round,fill=fillColor] (453.90,535.03) circle (  1.16);

\path[draw=drawColor,line width= 0.4pt,line join=round,line cap=round,fill=fillColor] (454.07,535.03) circle (  1.16);

\path[draw=drawColor,line width= 0.4pt,line join=round,line cap=round,fill=fillColor] (454.24,535.03) circle (  1.16);

\path[draw=drawColor,line width= 0.4pt,line join=round,line cap=round,fill=fillColor] (454.40,535.03) circle (  1.16);

\path[draw=drawColor,line width= 0.4pt,line join=round,line cap=round,fill=fillColor] (454.57,535.03) circle (  1.16);

\path[draw=drawColor,line width= 0.4pt,line join=round,line cap=round,fill=fillColor] (454.74,535.03) circle (  1.16);

\path[draw=drawColor,line width= 0.4pt,line join=round,line cap=round,fill=fillColor] (454.90,535.03) circle (  1.16);

\path[draw=drawColor,line width= 0.4pt,line join=round,line cap=round,fill=fillColor] (455.07,535.03) circle (  1.16);

\path[draw=drawColor,line width= 0.4pt,line join=round,line cap=round,fill=fillColor] (455.23,535.03) circle (  1.16);

\path[draw=drawColor,line width= 0.4pt,line join=round,line cap=round,fill=fillColor] (455.40,535.03) circle (  1.16);

\path[draw=drawColor,line width= 0.4pt,line join=round,line cap=round,fill=fillColor] (455.56,535.03) circle (  1.16);

\path[draw=drawColor,line width= 0.4pt,line join=round,line cap=round,fill=fillColor] (455.73,535.03) circle (  1.16);

\path[draw=drawColor,line width= 0.4pt,line join=round,line cap=round,fill=fillColor] (455.89,535.03) circle (  1.16);

\path[draw=drawColor,line width= 0.4pt,line join=round,line cap=round,fill=fillColor] (456.05,535.03) circle (  1.16);

\path[draw=drawColor,line width= 0.4pt,line join=round,line cap=round,fill=fillColor] (456.21,535.03) circle (  1.16);

\path[draw=drawColor,line width= 0.4pt,line join=round,line cap=round,fill=fillColor] (456.38,535.03) circle (  1.16);

\path[draw=drawColor,line width= 0.4pt,line join=round,line cap=round,fill=fillColor] (456.54,535.03) circle (  1.16);

\path[draw=drawColor,line width= 0.4pt,line join=round,line cap=round,fill=fillColor] (456.70,535.03) circle (  1.16);

\path[draw=drawColor,line width= 0.4pt,line join=round,line cap=round,fill=fillColor] (456.86,535.03) circle (  1.16);

\path[draw=drawColor,line width= 0.4pt,line join=round,line cap=round,fill=fillColor] (457.02,535.03) circle (  1.16);

\path[draw=drawColor,line width= 0.4pt,line join=round,line cap=round,fill=fillColor] (457.18,535.03) circle (  1.16);

\path[draw=drawColor,line width= 0.4pt,line join=round,line cap=round,fill=fillColor] (457.34,535.03) circle (  1.16);

\path[draw=drawColor,line width= 0.4pt,line join=round,line cap=round,fill=fillColor] (457.50,535.03) circle (  1.16);

\path[draw=drawColor,line width= 0.4pt,line join=round,line cap=round,fill=fillColor] (457.66,535.03) circle (  1.16);

\path[draw=drawColor,line width= 0.4pt,line join=round,line cap=round,fill=fillColor] (457.82,535.03) circle (  1.16);

\path[draw=drawColor,line width= 0.4pt,line join=round,line cap=round,fill=fillColor] (457.98,535.03) circle (  1.16);

\path[draw=drawColor,line width= 0.4pt,line join=round,line cap=round,fill=fillColor] (458.14,535.03) circle (  1.16);

\path[draw=drawColor,line width= 0.4pt,line join=round,line cap=round,fill=fillColor] (458.30,535.03) circle (  1.16);

\path[draw=drawColor,line width= 0.4pt,line join=round,line cap=round,fill=fillColor] (458.46,535.03) circle (  1.16);

\path[draw=drawColor,line width= 0.4pt,line join=round,line cap=round,fill=fillColor] (458.61,535.03) circle (  1.16);

\path[draw=drawColor,line width= 0.4pt,line join=round,line cap=round,fill=fillColor] (458.77,535.03) circle (  1.16);

\path[draw=drawColor,line width= 0.4pt,line join=round,line cap=round,fill=fillColor] (458.93,535.03) circle (  1.16);

\path[draw=drawColor,line width= 0.4pt,line join=round,line cap=round,fill=fillColor] (459.09,535.03) circle (  1.16);

\path[draw=drawColor,line width= 0.4pt,line join=round,line cap=round,fill=fillColor] (459.24,535.03) circle (  1.16);

\path[draw=drawColor,line width= 0.4pt,line join=round,line cap=round,fill=fillColor] (459.40,535.03) circle (  1.16);

\path[draw=drawColor,line width= 0.4pt,line join=round,line cap=round,fill=fillColor] (459.55,535.03) circle (  1.16);

\path[draw=drawColor,line width= 0.4pt,line join=round,line cap=round,fill=fillColor] (459.71,535.03) circle (  1.16);

\path[draw=drawColor,line width= 0.4pt,line join=round,line cap=round,fill=fillColor] (459.86,535.03) circle (  1.16);

\path[draw=drawColor,line width= 0.4pt,line join=round,line cap=round,fill=fillColor] (460.02,535.03) circle (  1.16);

\path[draw=drawColor,line width= 0.4pt,line join=round,line cap=round,fill=fillColor] (460.17,535.03) circle (  1.16);

\path[draw=drawColor,line width= 0.4pt,line join=round,line cap=round,fill=fillColor] (460.33,535.03) circle (  1.16);

\path[draw=drawColor,line width= 0.4pt,line join=round,line cap=round,fill=fillColor] (460.48,535.03) circle (  1.16);

\path[draw=drawColor,line width= 0.4pt,line join=round,line cap=round,fill=fillColor] (460.63,535.03) circle (  1.16);

\path[draw=drawColor,line width= 0.4pt,line join=round,line cap=round,fill=fillColor] (460.79,535.03) circle (  1.16);

\path[draw=drawColor,line width= 0.4pt,line join=round,line cap=round,fill=fillColor] (460.94,535.03) circle (  1.16);

\path[draw=drawColor,line width= 0.4pt,line join=round,line cap=round,fill=fillColor] (461.09,535.03) circle (  1.16);

\path[draw=drawColor,line width= 0.4pt,line join=round,line cap=round,fill=fillColor] (461.25,535.03) circle (  1.16);

\path[draw=drawColor,line width= 0.4pt,line join=round,line cap=round,fill=fillColor] (461.40,535.03) circle (  1.16);

\path[draw=drawColor,line width= 0.4pt,line join=round,line cap=round,fill=fillColor] (461.55,535.03) circle (  1.16);

\path[draw=drawColor,line width= 0.4pt,line join=round,line cap=round,fill=fillColor] (461.70,535.03) circle (  1.16);

\path[draw=drawColor,line width= 0.4pt,line join=round,line cap=round,fill=fillColor] (461.85,535.03) circle (  1.16);

\path[draw=drawColor,line width= 0.4pt,line join=round,line cap=round,fill=fillColor] (462.00,535.03) circle (  1.16);

\path[draw=drawColor,line width= 0.4pt,line join=round,line cap=round,fill=fillColor] (462.15,535.03) circle (  1.16);

\path[draw=drawColor,line width= 0.4pt,line join=round,line cap=round,fill=fillColor] (462.30,535.03) circle (  1.16);

\path[draw=drawColor,line width= 0.4pt,line join=round,line cap=round,fill=fillColor] (462.45,535.03) circle (  1.16);

\path[draw=drawColor,line width= 0.4pt,line join=round,line cap=round,fill=fillColor] (462.60,535.03) circle (  1.16);

\path[draw=drawColor,line width= 0.4pt,line join=round,line cap=round,fill=fillColor] (462.75,535.03) circle (  1.16);

\path[draw=drawColor,line width= 0.4pt,line join=round,line cap=round,fill=fillColor] (462.90,535.03) circle (  1.16);

\path[draw=drawColor,line width= 0.4pt,line join=round,line cap=round,fill=fillColor] (463.05,535.03) circle (  1.16);

\path[draw=drawColor,line width= 0.4pt,line join=round,line cap=round,fill=fillColor] (463.20,535.03) circle (  1.16);

\path[draw=drawColor,line width= 0.4pt,line join=round,line cap=round,fill=fillColor] (463.35,535.03) circle (  1.16);

\path[draw=drawColor,line width= 0.4pt,line join=round,line cap=round,fill=fillColor] (463.50,535.03) circle (  1.16);

\path[draw=drawColor,line width= 0.4pt,line join=round,line cap=round,fill=fillColor] (463.64,535.03) circle (  1.16);

\path[draw=drawColor,line width= 0.4pt,line join=round,line cap=round,fill=fillColor] (463.79,535.03) circle (  1.16);

\path[draw=drawColor,line width= 0.4pt,line join=round,line cap=round,fill=fillColor] (463.94,535.03) circle (  1.16);

\path[draw=drawColor,line width= 0.4pt,line join=round,line cap=round,fill=fillColor] (464.09,535.03) circle (  1.16);

\path[draw=drawColor,line width= 0.4pt,line join=round,line cap=round,fill=fillColor] (464.23,535.03) circle (  1.16);

\path[draw=drawColor,line width= 0.4pt,line join=round,line cap=round,fill=fillColor] (464.38,535.03) circle (  1.16);

\path[draw=drawColor,line width= 0.4pt,line join=round,line cap=round,fill=fillColor] (464.53,535.03) circle (  1.16);

\path[draw=drawColor,line width= 0.4pt,line join=round,line cap=round,fill=fillColor] (464.67,535.03) circle (  1.16);

\path[draw=drawColor,line width= 0.4pt,line join=round,line cap=round,fill=fillColor] (464.82,535.03) circle (  1.16);

\path[draw=drawColor,line width= 0.4pt,line join=round,line cap=round,fill=fillColor] (464.96,535.03) circle (  1.16);

\path[draw=drawColor,line width= 0.4pt,line join=round,line cap=round,fill=fillColor] (465.11,535.03) circle (  1.16);

\path[draw=drawColor,line width= 0.4pt,line join=round,line cap=round,fill=fillColor] (465.25,535.03) circle (  1.16);

\path[draw=drawColor,line width= 0.4pt,line join=round,line cap=round,fill=fillColor] (465.40,535.03) circle (  1.16);

\path[draw=drawColor,line width= 0.4pt,line join=round,line cap=round,fill=fillColor] (465.54,535.03) circle (  1.16);

\path[draw=drawColor,line width= 0.4pt,line join=round,line cap=round,fill=fillColor] (465.68,535.03) circle (  1.16);

\path[draw=drawColor,line width= 0.4pt,line join=round,line cap=round,fill=fillColor] (465.83,535.03) circle (  1.16);

\path[draw=drawColor,line width= 0.4pt,line join=round,line cap=round,fill=fillColor] (465.97,535.03) circle (  1.16);

\path[draw=drawColor,line width= 0.4pt,line join=round,line cap=round,fill=fillColor] (466.12,535.03) circle (  1.16);

\path[draw=drawColor,line width= 0.4pt,line join=round,line cap=round,fill=fillColor] (466.26,535.03) circle (  1.16);

\path[draw=drawColor,line width= 0.4pt,line join=round,line cap=round,fill=fillColor] (466.40,535.03) circle (  1.16);

\path[draw=drawColor,line width= 0.4pt,line join=round,line cap=round,fill=fillColor] (466.54,535.03) circle (  1.16);

\path[draw=drawColor,line width= 0.4pt,line join=round,line cap=round,fill=fillColor] (466.69,535.03) circle (  1.16);

\path[draw=drawColor,line width= 0.4pt,line join=round,line cap=round,fill=fillColor] (466.83,535.03) circle (  1.16);

\path[draw=drawColor,line width= 0.4pt,line join=round,line cap=round,fill=fillColor] (466.97,535.03) circle (  1.16);

\path[draw=drawColor,line width= 0.4pt,line join=round,line cap=round,fill=fillColor] (467.11,535.03) circle (  1.16);

\path[draw=drawColor,line width= 0.4pt,line join=round,line cap=round,fill=fillColor] (467.25,535.03) circle (  1.16);

\path[draw=drawColor,line width= 0.4pt,line join=round,line cap=round,fill=fillColor] (467.39,535.03) circle (  1.16);

\path[draw=drawColor,line width= 0.4pt,line join=round,line cap=round,fill=fillColor] (467.53,535.03) circle (  1.16);

\path[draw=drawColor,line width= 0.4pt,line join=round,line cap=round,fill=fillColor] (467.67,535.03) circle (  1.16);

\path[draw=drawColor,line width= 0.4pt,line join=round,line cap=round,fill=fillColor] (467.81,535.03) circle (  1.16);

\path[draw=drawColor,line width= 0.4pt,line join=round,line cap=round,fill=fillColor] (467.95,535.03) circle (  1.16);

\path[draw=drawColor,line width= 0.4pt,line join=round,line cap=round,fill=fillColor] (468.09,535.03) circle (  1.16);

\path[draw=drawColor,line width= 0.4pt,line join=round,line cap=round,fill=fillColor] (468.23,535.03) circle (  1.16);

\path[draw=drawColor,line width= 0.4pt,line join=round,line cap=round,fill=fillColor] (468.37,535.03) circle (  1.16);

\path[draw=drawColor,line width= 0.4pt,line join=round,line cap=round,fill=fillColor] (468.51,535.03) circle (  1.16);

\path[draw=drawColor,line width= 0.4pt,line join=round,line cap=round,fill=fillColor] (468.65,535.03) circle (  1.16);

\path[draw=drawColor,line width= 0.4pt,line join=round,line cap=round,fill=fillColor] (468.79,535.03) circle (  1.16);

\path[draw=drawColor,line width= 0.4pt,line join=round,line cap=round,fill=fillColor] (468.93,535.03) circle (  1.16);

\path[draw=drawColor,line width= 0.4pt,line join=round,line cap=round,fill=fillColor] (469.07,535.03) circle (  1.16);

\path[draw=drawColor,line width= 0.4pt,line join=round,line cap=round,fill=fillColor] (469.20,535.03) circle (  1.16);

\path[draw=drawColor,line width= 0.4pt,line join=round,line cap=round,fill=fillColor] (469.34,535.03) circle (  1.16);

\path[draw=drawColor,line width= 0.4pt,line join=round,line cap=round,fill=fillColor] (469.48,535.03) circle (  1.16);

\path[draw=drawColor,line width= 0.4pt,line join=round,line cap=round,fill=fillColor] (469.62,535.03) circle (  1.16);

\path[draw=drawColor,line width= 0.4pt,line join=round,line cap=round,fill=fillColor] (469.75,535.03) circle (  1.16);

\path[draw=drawColor,line width= 0.4pt,line join=round,line cap=round,fill=fillColor] (469.89,535.03) circle (  1.16);

\path[draw=drawColor,line width= 0.4pt,line join=round,line cap=round,fill=fillColor] (470.03,535.03) circle (  1.16);

\path[draw=drawColor,line width= 0.4pt,line join=round,line cap=round,fill=fillColor] (470.16,535.03) circle (  1.16);

\path[draw=drawColor,line width= 0.4pt,line join=round,line cap=round,fill=fillColor] (470.30,535.03) circle (  1.16);

\path[draw=drawColor,line width= 0.4pt,line join=round,line cap=round,fill=fillColor] (470.43,535.03) circle (  1.16);

\path[draw=drawColor,line width= 0.4pt,line join=round,line cap=round,fill=fillColor] (470.57,535.03) circle (  1.16);

\path[draw=drawColor,line width= 0.4pt,line join=round,line cap=round,fill=fillColor] (470.71,535.03) circle (  1.16);

\path[draw=drawColor,line width= 0.4pt,line join=round,line cap=round,fill=fillColor] (470.84,535.03) circle (  1.16);

\path[draw=drawColor,line width= 0.4pt,line join=round,line cap=round,fill=fillColor] (470.98,535.03) circle (  1.16);

\path[draw=drawColor,line width= 0.4pt,line join=round,line cap=round,fill=fillColor] (471.11,535.03) circle (  1.16);

\path[draw=drawColor,line width= 0.4pt,line join=round,line cap=round,fill=fillColor] (471.24,535.03) circle (  1.16);

\path[draw=drawColor,line width= 0.4pt,line join=round,line cap=round,fill=fillColor] (471.38,535.03) circle (  1.16);

\path[draw=drawColor,line width= 0.4pt,line join=round,line cap=round,fill=fillColor] (471.51,535.03) circle (  1.16);

\path[draw=drawColor,line width= 0.4pt,line join=round,line cap=round,fill=fillColor] (471.65,535.03) circle (  1.16);

\path[draw=drawColor,line width= 0.4pt,line join=round,line cap=round,fill=fillColor] (471.78,535.03) circle (  1.16);

\path[draw=drawColor,line width= 0.4pt,line join=round,line cap=round,fill=fillColor] (471.91,535.03) circle (  1.16);

\path[draw=drawColor,line width= 0.4pt,line join=round,line cap=round,fill=fillColor] (472.05,535.03) circle (  1.16);

\path[draw=drawColor,line width= 0.4pt,line join=round,line cap=round,fill=fillColor] (472.18,535.03) circle (  1.16);

\path[draw=drawColor,line width= 0.4pt,line join=round,line cap=round,fill=fillColor] (472.31,535.03) circle (  1.16);

\path[draw=drawColor,line width= 0.4pt,line join=round,line cap=round,fill=fillColor] (472.45,535.03) circle (  1.16);

\path[draw=drawColor,line width= 0.4pt,line join=round,line cap=round,fill=fillColor] (472.58,535.03) circle (  1.16);

\path[draw=drawColor,line width= 0.4pt,line join=round,line cap=round,fill=fillColor] (472.71,535.03) circle (  1.16);

\path[draw=drawColor,line width= 0.4pt,line join=round,line cap=round,fill=fillColor] (472.84,535.03) circle (  1.16);

\path[draw=drawColor,line width= 0.4pt,line join=round,line cap=round,fill=fillColor] (472.97,535.03) circle (  1.16);

\path[draw=drawColor,line width= 0.4pt,line join=round,line cap=round,fill=fillColor] (473.11,535.03) circle (  1.16);

\path[draw=drawColor,line width= 0.4pt,line join=round,line cap=round,fill=fillColor] (473.24,535.03) circle (  1.16);

\path[draw=drawColor,line width= 0.4pt,line join=round,line cap=round,fill=fillColor] (473.37,535.03) circle (  1.16);

\path[draw=drawColor,line width= 0.4pt,line join=round,line cap=round,fill=fillColor] (473.50,535.03) circle (  1.16);

\path[draw=drawColor,line width= 0.4pt,line join=round,line cap=round,fill=fillColor] (473.63,535.03) circle (  1.16);

\path[draw=drawColor,line width= 0.4pt,line join=round,line cap=round,fill=fillColor] (473.76,535.03) circle (  1.16);

\path[draw=drawColor,line width= 0.4pt,line join=round,line cap=round,fill=fillColor] (473.89,535.03) circle (  1.16);

\path[draw=drawColor,line width= 0.4pt,line join=round,line cap=round,fill=fillColor] (474.02,535.03) circle (  1.16);

\path[draw=drawColor,line width= 0.4pt,line join=round,line cap=round,fill=fillColor] (474.15,535.03) circle (  1.16);

\path[draw=drawColor,line width= 0.4pt,line join=round,line cap=round,fill=fillColor] (474.28,535.03) circle (  1.16);

\path[draw=drawColor,line width= 0.4pt,line join=round,line cap=round,fill=fillColor] (474.41,535.03) circle (  1.16);

\path[draw=drawColor,line width= 0.4pt,line join=round,line cap=round,fill=fillColor] (474.54,535.03) circle (  1.16);

\path[draw=drawColor,line width= 0.4pt,line join=round,line cap=round,fill=fillColor] (474.67,535.03) circle (  1.16);

\path[draw=drawColor,line width= 0.4pt,line join=round,line cap=round,fill=fillColor] (474.80,535.03) circle (  1.16);

\path[draw=drawColor,line width= 0.4pt,line join=round,line cap=round,fill=fillColor] (474.93,535.03) circle (  1.16);

\path[draw=drawColor,line width= 0.4pt,line join=round,line cap=round,fill=fillColor] (475.05,535.03) circle (  1.16);

\path[draw=drawColor,line width= 0.4pt,line join=round,line cap=round,fill=fillColor] (475.18,535.03) circle (  1.16);

\path[draw=drawColor,line width= 0.4pt,line join=round,line cap=round,fill=fillColor] (475.31,535.03) circle (  1.16);

\path[draw=drawColor,line width= 0.4pt,line join=round,line cap=round,fill=fillColor] (475.44,535.03) circle (  1.16);

\path[draw=drawColor,line width= 0.4pt,line join=round,line cap=round,fill=fillColor] (475.57,535.03) circle (  1.16);

\path[draw=drawColor,line width= 0.4pt,line join=round,line cap=round,fill=fillColor] (475.69,535.03) circle (  1.16);

\path[draw=drawColor,line width= 0.4pt,line join=round,line cap=round,fill=fillColor] (475.82,535.03) circle (  1.16);

\path[draw=drawColor,line width= 0.4pt,line join=round,line cap=round,fill=fillColor] (475.95,535.03) circle (  1.16);

\path[draw=drawColor,line width= 0.4pt,line join=round,line cap=round,fill=fillColor] (476.08,535.03) circle (  1.16);

\path[draw=drawColor,line width= 0.4pt,line join=round,line cap=round,fill=fillColor] (476.20,535.03) circle (  1.16);

\path[draw=drawColor,line width= 0.4pt,line join=round,line cap=round,fill=fillColor] (476.33,535.03) circle (  1.16);

\path[draw=drawColor,line width= 0.4pt,line join=round,line cap=round,fill=fillColor] (476.46,535.03) circle (  1.16);

\path[draw=drawColor,line width= 0.4pt,line join=round,line cap=round,fill=fillColor] (476.58,535.03) circle (  1.16);

\path[draw=drawColor,line width= 0.4pt,line join=round,line cap=round,fill=fillColor] (476.71,535.03) circle (  1.16);

\path[draw=drawColor,line width= 0.4pt,line join=round,line cap=round,fill=fillColor] (476.83,535.03) circle (  1.16);

\path[draw=drawColor,line width= 0.4pt,line join=round,line cap=round,fill=fillColor] (476.96,535.03) circle (  1.16);

\path[draw=drawColor,line width= 0.4pt,line join=round,line cap=round,fill=fillColor] (477.08,535.03) circle (  1.16);

\path[draw=drawColor,line width= 0.4pt,line join=round,line cap=round,fill=fillColor] (477.21,535.03) circle (  1.16);

\path[draw=drawColor,line width= 0.4pt,line join=round,line cap=round,fill=fillColor] (477.33,535.03) circle (  1.16);

\path[draw=drawColor,line width= 0.4pt,line join=round,line cap=round,fill=fillColor] (477.46,535.03) circle (  1.16);

\path[draw=drawColor,line width= 0.4pt,line join=round,line cap=round,fill=fillColor] (477.58,535.03) circle (  1.16);

\path[draw=drawColor,line width= 0.4pt,line join=round,line cap=round,fill=fillColor] (477.71,535.03) circle (  1.16);

\path[draw=drawColor,line width= 0.4pt,line join=round,line cap=round,fill=fillColor] (477.83,535.03) circle (  1.16);

\path[draw=drawColor,line width= 0.4pt,line join=round,line cap=round,fill=fillColor] (477.96,535.03) circle (  1.16);

\path[draw=drawColor,line width= 0.4pt,line join=round,line cap=round,fill=fillColor] (478.08,535.03) circle (  1.16);

\path[draw=drawColor,line width= 0.4pt,line join=round,line cap=round,fill=fillColor] (478.21,535.03) circle (  1.16);
\definecolor[named]{drawColor}{rgb}{0.22,0.49,0.72}
\definecolor[named]{fillColor}{rgb}{0.22,0.49,0.72}

\path[draw=drawColor,line width= 0.4pt,line join=round,line cap=round,fill=fillColor] (327.82,617.76) circle (  1.16);

\path[draw=drawColor,line width= 0.4pt,line join=round,line cap=round,fill=fillColor] (333.62,617.37) circle (  1.16);

\path[draw=drawColor,line width= 0.4pt,line join=round,line cap=round,fill=fillColor] (337.68,616.72) circle (  1.16);

\path[draw=drawColor,line width= 0.4pt,line join=round,line cap=round,fill=fillColor] (340.92,616.40) circle (  1.16);

\path[draw=drawColor,line width= 0.4pt,line join=round,line cap=round,fill=fillColor] (343.65,616.32) circle (  1.16);

\path[draw=drawColor,line width= 0.4pt,line join=round,line cap=round,fill=fillColor] (346.04,616.21) circle (  1.16);

\path[draw=drawColor,line width= 0.4pt,line join=round,line cap=round,fill=fillColor] (348.17,616.15) circle (  1.16);

\path[draw=drawColor,line width= 0.4pt,line join=round,line cap=round,fill=fillColor] (350.11,616.03) circle (  1.16);

\path[draw=drawColor,line width= 0.4pt,line join=round,line cap=round,fill=fillColor] (351.90,615.98) circle (  1.16);

\path[draw=drawColor,line width= 0.4pt,line join=round,line cap=round,fill=fillColor] (353.55,615.84) circle (  1.16);

\path[draw=drawColor,line width= 0.4pt,line join=round,line cap=round,fill=fillColor] (355.10,615.72) circle (  1.16);

\path[draw=drawColor,line width= 0.4pt,line join=round,line cap=round,fill=fillColor] (356.56,615.30) circle (  1.16);

\path[draw=drawColor,line width= 0.4pt,line join=round,line cap=round,fill=fillColor] (357.94,615.01) circle (  1.16);

\path[draw=drawColor,line width= 0.4pt,line join=round,line cap=round,fill=fillColor] (359.25,614.63) circle (  1.16);

\path[draw=drawColor,line width= 0.4pt,line join=round,line cap=round,fill=fillColor] (360.50,614.42) circle (  1.16);

\path[draw=drawColor,line width= 0.4pt,line join=round,line cap=round,fill=fillColor] (361.70,614.30) circle (  1.16);

\path[draw=drawColor,line width= 0.4pt,line join=round,line cap=round,fill=fillColor] (362.84,614.28) circle (  1.16);

\path[draw=drawColor,line width= 0.4pt,line join=round,line cap=round,fill=fillColor] (363.95,613.93) circle (  1.16);

\path[draw=drawColor,line width= 0.4pt,line join=round,line cap=round,fill=fillColor] (365.01,613.68) circle (  1.16);

\path[draw=drawColor,line width= 0.4pt,line join=round,line cap=round,fill=fillColor] (366.03,613.04) circle (  1.16);

\path[draw=drawColor,line width= 0.4pt,line join=round,line cap=round,fill=fillColor] (367.03,612.91) circle (  1.16);

\path[draw=drawColor,line width= 0.4pt,line join=round,line cap=round,fill=fillColor] (367.99,612.89) circle (  1.16);

\path[draw=drawColor,line width= 0.4pt,line join=round,line cap=round,fill=fillColor] (368.92,612.71) circle (  1.16);

\path[draw=drawColor,line width= 0.4pt,line join=round,line cap=round,fill=fillColor] (369.83,612.41) circle (  1.16);

\path[draw=drawColor,line width= 0.4pt,line join=round,line cap=round,fill=fillColor] (370.71,612.40) circle (  1.16);

\path[draw=drawColor,line width= 0.4pt,line join=round,line cap=round,fill=fillColor] (371.56,612.23) circle (  1.16);

\path[draw=drawColor,line width= 0.4pt,line join=round,line cap=round,fill=fillColor] (372.40,610.35) circle (  1.16);

\path[draw=drawColor,line width= 0.4pt,line join=round,line cap=round,fill=fillColor] (373.21,609.77) circle (  1.16);

\path[draw=drawColor,line width= 0.4pt,line join=round,line cap=round,fill=fillColor] (374.01,608.63) circle (  1.16);

\path[draw=drawColor,line width= 0.4pt,line join=round,line cap=round,fill=fillColor] (374.79,607.05) circle (  1.16);

\path[draw=drawColor,line width= 0.4pt,line join=round,line cap=round,fill=fillColor] (375.55,606.21) circle (  1.16);

\path[draw=drawColor,line width= 0.4pt,line join=round,line cap=round,fill=fillColor] (376.30,605.43) circle (  1.16);

\path[draw=drawColor,line width= 0.4pt,line join=round,line cap=round,fill=fillColor] (377.02,603.71) circle (  1.16);

\path[draw=drawColor,line width= 0.4pt,line join=round,line cap=round,fill=fillColor] (377.74,603.31) circle (  1.16);

\path[draw=drawColor,line width= 0.4pt,line join=round,line cap=round,fill=fillColor] (378.44,603.30) circle (  1.16);

\path[draw=drawColor,line width= 0.4pt,line join=round,line cap=round,fill=fillColor] (379.13,601.28) circle (  1.16);

\path[draw=drawColor,line width= 0.4pt,line join=round,line cap=round,fill=fillColor] (379.80,599.78) circle (  1.16);

\path[draw=drawColor,line width= 0.4pt,line join=round,line cap=round,fill=fillColor] (380.47,598.55) circle (  1.16);

\path[draw=drawColor,line width= 0.4pt,line join=round,line cap=round,fill=fillColor] (381.12,597.84) circle (  1.16);

\path[draw=drawColor,line width= 0.4pt,line join=round,line cap=round,fill=fillColor] (381.76,597.54) circle (  1.16);

\path[draw=drawColor,line width= 0.4pt,line join=round,line cap=round,fill=fillColor] (382.39,595.04) circle (  1.16);

\path[draw=drawColor,line width= 0.4pt,line join=round,line cap=round,fill=fillColor] (383.01,594.40) circle (  1.16);

\path[draw=drawColor,line width= 0.4pt,line join=round,line cap=round,fill=fillColor] (383.62,591.62) circle (  1.16);

\path[draw=drawColor,line width= 0.4pt,line join=round,line cap=round,fill=fillColor] (384.22,591.24) circle (  1.16);

\path[draw=drawColor,line width= 0.4pt,line join=round,line cap=round,fill=fillColor] (384.81,590.80) circle (  1.16);

\path[draw=drawColor,line width= 0.4pt,line join=round,line cap=round,fill=fillColor] (385.39,590.52) circle (  1.16);

\path[draw=drawColor,line width= 0.4pt,line join=round,line cap=round,fill=fillColor] (385.97,590.40) circle (  1.16);

\path[draw=drawColor,line width= 0.4pt,line join=round,line cap=round,fill=fillColor] (386.54,589.79) circle (  1.16);

\path[draw=drawColor,line width= 0.4pt,line join=round,line cap=round,fill=fillColor] (387.09,589.21) circle (  1.16);

\path[draw=drawColor,line width= 0.4pt,line join=round,line cap=round,fill=fillColor] (387.64,588.57) circle (  1.16);

\path[draw=drawColor,line width= 0.4pt,line join=round,line cap=round,fill=fillColor] (388.19,588.09) circle (  1.16);

\path[draw=drawColor,line width= 0.4pt,line join=round,line cap=round,fill=fillColor] (388.73,587.64) circle (  1.16);

\path[draw=drawColor,line width= 0.4pt,line join=round,line cap=round,fill=fillColor] (389.26,586.82) circle (  1.16);

\path[draw=drawColor,line width= 0.4pt,line join=round,line cap=round,fill=fillColor] (389.78,586.41) circle (  1.16);

\path[draw=drawColor,line width= 0.4pt,line join=round,line cap=round,fill=fillColor] (390.30,586.31) circle (  1.16);

\path[draw=drawColor,line width= 0.4pt,line join=round,line cap=round,fill=fillColor] (390.81,585.99) circle (  1.16);

\path[draw=drawColor,line width= 0.4pt,line join=round,line cap=round,fill=fillColor] (391.31,585.94) circle (  1.16);

\path[draw=drawColor,line width= 0.4pt,line join=round,line cap=round,fill=fillColor] (391.81,585.00) circle (  1.16);

\path[draw=drawColor,line width= 0.4pt,line join=round,line cap=round,fill=fillColor] (392.30,583.30) circle (  1.16);

\path[draw=drawColor,line width= 0.4pt,line join=round,line cap=round,fill=fillColor] (392.79,583.23) circle (  1.16);

\path[draw=drawColor,line width= 0.4pt,line join=round,line cap=round,fill=fillColor] (393.27,582.50) circle (  1.16);

\path[draw=drawColor,line width= 0.4pt,line join=round,line cap=round,fill=fillColor] (393.75,582.15) circle (  1.16);

\path[draw=drawColor,line width= 0.4pt,line join=round,line cap=round,fill=fillColor] (394.22,582.08) circle (  1.16);

\path[draw=drawColor,line width= 0.4pt,line join=round,line cap=round,fill=fillColor] (394.69,581.59) circle (  1.16);

\path[draw=drawColor,line width= 0.4pt,line join=round,line cap=round,fill=fillColor] (395.15,581.27) circle (  1.16);

\path[draw=drawColor,line width= 0.4pt,line join=round,line cap=round,fill=fillColor] (395.61,580.56) circle (  1.16);

\path[draw=drawColor,line width= 0.4pt,line join=round,line cap=round,fill=fillColor] (396.06,580.43) circle (  1.16);

\path[draw=drawColor,line width= 0.4pt,line join=round,line cap=round,fill=fillColor] (396.51,579.77) circle (  1.16);

\path[draw=drawColor,line width= 0.4pt,line join=round,line cap=round,fill=fillColor] (396.95,579.65) circle (  1.16);

\path[draw=drawColor,line width= 0.4pt,line join=round,line cap=round,fill=fillColor] (397.39,579.62) circle (  1.16);

\path[draw=drawColor,line width= 0.4pt,line join=round,line cap=round,fill=fillColor] (397.83,579.45) circle (  1.16);

\path[draw=drawColor,line width= 0.4pt,line join=round,line cap=round,fill=fillColor] (398.26,579.42) circle (  1.16);

\path[draw=drawColor,line width= 0.4pt,line join=round,line cap=round,fill=fillColor] (398.68,579.42) circle (  1.16);

\path[draw=drawColor,line width= 0.4pt,line join=round,line cap=round,fill=fillColor] (399.11,579.42) circle (  1.16);

\path[draw=drawColor,line width= 0.4pt,line join=round,line cap=round,fill=fillColor] (399.53,578.89) circle (  1.16);

\path[draw=drawColor,line width= 0.4pt,line join=round,line cap=round,fill=fillColor] (399.94,578.86) circle (  1.16);

\path[draw=drawColor,line width= 0.4pt,line join=round,line cap=round,fill=fillColor] (400.36,578.66) circle (  1.16);

\path[draw=drawColor,line width= 0.4pt,line join=round,line cap=round,fill=fillColor] (400.76,578.48) circle (  1.16);

\path[draw=drawColor,line width= 0.4pt,line join=round,line cap=round,fill=fillColor] (401.17,578.29) circle (  1.16);

\path[draw=drawColor,line width= 0.4pt,line join=round,line cap=round,fill=fillColor] (401.57,578.25) circle (  1.16);

\path[draw=drawColor,line width= 0.4pt,line join=round,line cap=round,fill=fillColor] (401.97,577.80) circle (  1.16);

\path[draw=drawColor,line width= 0.4pt,line join=round,line cap=round,fill=fillColor] (402.36,577.53) circle (  1.16);

\path[draw=drawColor,line width= 0.4pt,line join=round,line cap=round,fill=fillColor] (402.76,577.51) circle (  1.16);

\path[draw=drawColor,line width= 0.4pt,line join=round,line cap=round,fill=fillColor] (403.15,577.37) circle (  1.16);

\path[draw=drawColor,line width= 0.4pt,line join=round,line cap=round,fill=fillColor] (403.53,577.35) circle (  1.16);

\path[draw=drawColor,line width= 0.4pt,line join=round,line cap=round,fill=fillColor] (403.91,577.03) circle (  1.16);

\path[draw=drawColor,line width= 0.4pt,line join=round,line cap=round,fill=fillColor] (404.29,576.54) circle (  1.16);

\path[draw=drawColor,line width= 0.4pt,line join=round,line cap=round,fill=fillColor] (404.67,576.23) circle (  1.16);

\path[draw=drawColor,line width= 0.4pt,line join=round,line cap=round,fill=fillColor] (405.05,576.11) circle (  1.16);

\path[draw=drawColor,line width= 0.4pt,line join=round,line cap=round,fill=fillColor] (405.42,575.83) circle (  1.16);

\path[draw=drawColor,line width= 0.4pt,line join=round,line cap=round,fill=fillColor] (405.78,575.64) circle (  1.16);

\path[draw=drawColor,line width= 0.4pt,line join=round,line cap=round,fill=fillColor] (406.15,575.56) circle (  1.16);

\path[draw=drawColor,line width= 0.4pt,line join=round,line cap=round,fill=fillColor] (406.51,575.47) circle (  1.16);

\path[draw=drawColor,line width= 0.4pt,line join=round,line cap=round,fill=fillColor] (406.87,574.99) circle (  1.16);

\path[draw=drawColor,line width= 0.4pt,line join=round,line cap=round,fill=fillColor] (407.23,574.93) circle (  1.16);

\path[draw=drawColor,line width= 0.4pt,line join=round,line cap=round,fill=fillColor] (407.59,574.92) circle (  1.16);

\path[draw=drawColor,line width= 0.4pt,line join=round,line cap=round,fill=fillColor] (407.94,574.88) circle (  1.16);

\path[draw=drawColor,line width= 0.4pt,line join=round,line cap=round,fill=fillColor] (408.29,574.88) circle (  1.16);

\path[draw=drawColor,line width= 0.4pt,line join=round,line cap=round,fill=fillColor] (408.64,574.63) circle (  1.16);

\path[draw=drawColor,line width= 0.4pt,line join=round,line cap=round,fill=fillColor] (408.99,574.43) circle (  1.16);

\path[draw=drawColor,line width= 0.4pt,line join=round,line cap=round,fill=fillColor] (409.33,573.97) circle (  1.16);

\path[draw=drawColor,line width= 0.4pt,line join=round,line cap=round,fill=fillColor] (409.67,573.93) circle (  1.16);

\path[draw=drawColor,line width= 0.4pt,line join=round,line cap=round,fill=fillColor] (410.01,573.77) circle (  1.16);

\path[draw=drawColor,line width= 0.4pt,line join=round,line cap=round,fill=fillColor] (410.35,573.74) circle (  1.16);

\path[draw=drawColor,line width= 0.4pt,line join=round,line cap=round,fill=fillColor] (410.68,573.47) circle (  1.16);

\path[draw=drawColor,line width= 0.4pt,line join=round,line cap=round,fill=fillColor] (411.02,573.43) circle (  1.16);

\path[draw=drawColor,line width= 0.4pt,line join=round,line cap=round,fill=fillColor] (411.35,573.36) circle (  1.16);

\path[draw=drawColor,line width= 0.4pt,line join=round,line cap=round,fill=fillColor] (411.67,573.22) circle (  1.16);

\path[draw=drawColor,line width= 0.4pt,line join=round,line cap=round,fill=fillColor] (412.00,572.73) circle (  1.16);

\path[draw=drawColor,line width= 0.4pt,line join=round,line cap=round,fill=fillColor] (412.33,572.61) circle (  1.16);

\path[draw=drawColor,line width= 0.4pt,line join=round,line cap=round,fill=fillColor] (412.65,572.56) circle (  1.16);

\path[draw=drawColor,line width= 0.4pt,line join=round,line cap=round,fill=fillColor] (412.97,572.18) circle (  1.16);

\path[draw=drawColor,line width= 0.4pt,line join=round,line cap=round,fill=fillColor] (413.29,572.04) circle (  1.16);

\path[draw=drawColor,line width= 0.4pt,line join=round,line cap=round,fill=fillColor] (413.60,571.94) circle (  1.16);

\path[draw=drawColor,line width= 0.4pt,line join=round,line cap=round,fill=fillColor] (413.92,571.94) circle (  1.16);

\path[draw=drawColor,line width= 0.4pt,line join=round,line cap=round,fill=fillColor] (414.23,571.75) circle (  1.16);

\path[draw=drawColor,line width= 0.4pt,line join=round,line cap=round,fill=fillColor] (414.54,571.32) circle (  1.16);

\path[draw=drawColor,line width= 0.4pt,line join=round,line cap=round,fill=fillColor] (414.85,571.32) circle (  1.16);

\path[draw=drawColor,line width= 0.4pt,line join=round,line cap=round,fill=fillColor] (415.16,571.25) circle (  1.16);

\path[draw=drawColor,line width= 0.4pt,line join=round,line cap=round,fill=fillColor] (415.47,571.21) circle (  1.16);

\path[draw=drawColor,line width= 0.4pt,line join=round,line cap=round,fill=fillColor] (415.77,571.01) circle (  1.16);

\path[draw=drawColor,line width= 0.4pt,line join=round,line cap=round,fill=fillColor] (416.08,570.99) circle (  1.16);

\path[draw=drawColor,line width= 0.4pt,line join=round,line cap=round,fill=fillColor] (416.38,570.92) circle (  1.16);

\path[draw=drawColor,line width= 0.4pt,line join=round,line cap=round,fill=fillColor] (416.68,570.85) circle (  1.16);

\path[draw=drawColor,line width= 0.4pt,line join=round,line cap=round,fill=fillColor] (416.97,570.83) circle (  1.16);

\path[draw=drawColor,line width= 0.4pt,line join=round,line cap=round,fill=fillColor] (417.27,570.74) circle (  1.16);

\path[draw=drawColor,line width= 0.4pt,line join=round,line cap=round,fill=fillColor] (417.57,570.67) circle (  1.16);

\path[draw=drawColor,line width= 0.4pt,line join=round,line cap=round,fill=fillColor] (417.86,570.46) circle (  1.16);

\path[draw=drawColor,line width= 0.4pt,line join=round,line cap=round,fill=fillColor] (418.15,570.35) circle (  1.16);

\path[draw=drawColor,line width= 0.4pt,line join=round,line cap=round,fill=fillColor] (418.44,570.34) circle (  1.16);

\path[draw=drawColor,line width= 0.4pt,line join=round,line cap=round,fill=fillColor] (418.73,570.30) circle (  1.16);

\path[draw=drawColor,line width= 0.4pt,line join=round,line cap=round,fill=fillColor] (419.02,570.17) circle (  1.16);

\path[draw=drawColor,line width= 0.4pt,line join=round,line cap=round,fill=fillColor] (419.30,570.11) circle (  1.16);

\path[draw=drawColor,line width= 0.4pt,line join=round,line cap=round,fill=fillColor] (419.59,569.93) circle (  1.16);

\path[draw=drawColor,line width= 0.4pt,line join=round,line cap=round,fill=fillColor] (419.87,569.88) circle (  1.16);

\path[draw=drawColor,line width= 0.4pt,line join=round,line cap=round,fill=fillColor] (420.15,569.45) circle (  1.16);

\path[draw=drawColor,line width= 0.4pt,line join=round,line cap=round,fill=fillColor] (420.43,569.32) circle (  1.16);

\path[draw=drawColor,line width= 0.4pt,line join=round,line cap=round,fill=fillColor] (420.71,569.23) circle (  1.16);

\path[draw=drawColor,line width= 0.4pt,line join=round,line cap=round,fill=fillColor] (420.99,569.07) circle (  1.16);

\path[draw=drawColor,line width= 0.4pt,line join=round,line cap=round,fill=fillColor] (421.26,569.05) circle (  1.16);

\path[draw=drawColor,line width= 0.4pt,line join=round,line cap=round,fill=fillColor] (421.54,568.93) circle (  1.16);

\path[draw=drawColor,line width= 0.4pt,line join=round,line cap=round,fill=fillColor] (421.81,568.91) circle (  1.16);

\path[draw=drawColor,line width= 0.4pt,line join=round,line cap=round,fill=fillColor] (422.09,568.76) circle (  1.16);

\path[draw=drawColor,line width= 0.4pt,line join=round,line cap=round,fill=fillColor] (422.36,568.73) circle (  1.16);

\path[draw=drawColor,line width= 0.4pt,line join=round,line cap=round,fill=fillColor] (422.63,568.73) circle (  1.16);

\path[draw=drawColor,line width= 0.4pt,line join=round,line cap=round,fill=fillColor] (422.90,568.57) circle (  1.16);

\path[draw=drawColor,line width= 0.4pt,line join=round,line cap=round,fill=fillColor] (423.16,568.46) circle (  1.16);

\path[draw=drawColor,line width= 0.4pt,line join=round,line cap=round,fill=fillColor] (423.43,568.28) circle (  1.16);

\path[draw=drawColor,line width= 0.4pt,line join=round,line cap=round,fill=fillColor] (423.69,568.24) circle (  1.16);

\path[draw=drawColor,line width= 0.4pt,line join=round,line cap=round,fill=fillColor] (423.96,568.06) circle (  1.16);

\path[draw=drawColor,line width= 0.4pt,line join=round,line cap=round,fill=fillColor] (424.22,567.94) circle (  1.16);

\path[draw=drawColor,line width= 0.4pt,line join=round,line cap=round,fill=fillColor] (424.48,567.85) circle (  1.16);

\path[draw=drawColor,line width= 0.4pt,line join=round,line cap=round,fill=fillColor] (424.74,567.71) circle (  1.16);

\path[draw=drawColor,line width= 0.4pt,line join=round,line cap=round,fill=fillColor] (425.00,567.64) circle (  1.16);

\path[draw=drawColor,line width= 0.4pt,line join=round,line cap=round,fill=fillColor] (425.26,567.48) circle (  1.16);

\path[draw=drawColor,line width= 0.4pt,line join=round,line cap=round,fill=fillColor] (425.52,567.40) circle (  1.16);

\path[draw=drawColor,line width= 0.4pt,line join=round,line cap=round,fill=fillColor] (425.77,567.33) circle (  1.16);

\path[draw=drawColor,line width= 0.4pt,line join=round,line cap=round,fill=fillColor] (426.03,567.33) circle (  1.16);

\path[draw=drawColor,line width= 0.4pt,line join=round,line cap=round,fill=fillColor] (426.28,566.81) circle (  1.16);

\path[draw=drawColor,line width= 0.4pt,line join=round,line cap=round,fill=fillColor] (426.53,566.71) circle (  1.16);

\path[draw=drawColor,line width= 0.4pt,line join=round,line cap=round,fill=fillColor] (426.78,566.66) circle (  1.16);

\path[draw=drawColor,line width= 0.4pt,line join=round,line cap=round,fill=fillColor] (427.03,566.60) circle (  1.16);

\path[draw=drawColor,line width= 0.4pt,line join=round,line cap=round,fill=fillColor] (427.28,566.52) circle (  1.16);

\path[draw=drawColor,line width= 0.4pt,line join=round,line cap=round,fill=fillColor] (427.53,566.45) circle (  1.16);

\path[draw=drawColor,line width= 0.4pt,line join=round,line cap=round,fill=fillColor] (427.78,566.35) circle (  1.16);

\path[draw=drawColor,line width= 0.4pt,line join=round,line cap=round,fill=fillColor] (428.03,566.01) circle (  1.16);

\path[draw=drawColor,line width= 0.4pt,line join=round,line cap=round,fill=fillColor] (428.27,565.96) circle (  1.16);

\path[draw=drawColor,line width= 0.4pt,line join=round,line cap=round,fill=fillColor] (428.52,565.93) circle (  1.16);

\path[draw=drawColor,line width= 0.4pt,line join=round,line cap=round,fill=fillColor] (428.76,565.90) circle (  1.16);

\path[draw=drawColor,line width= 0.4pt,line join=round,line cap=round,fill=fillColor] (429.00,565.90) circle (  1.16);

\path[draw=drawColor,line width= 0.4pt,line join=round,line cap=round,fill=fillColor] (429.24,565.77) circle (  1.16);

\path[draw=drawColor,line width= 0.4pt,line join=round,line cap=round,fill=fillColor] (429.48,565.76) circle (  1.16);

\path[draw=drawColor,line width= 0.4pt,line join=round,line cap=round,fill=fillColor] (429.72,565.73) circle (  1.16);

\path[draw=drawColor,line width= 0.4pt,line join=round,line cap=round,fill=fillColor] (429.96,565.65) circle (  1.16);

\path[draw=drawColor,line width= 0.4pt,line join=round,line cap=round,fill=fillColor] (430.20,565.61) circle (  1.16);

\path[draw=drawColor,line width= 0.4pt,line join=round,line cap=round,fill=fillColor] (430.44,565.59) circle (  1.16);

\path[draw=drawColor,line width= 0.4pt,line join=round,line cap=round,fill=fillColor] (430.67,565.56) circle (  1.16);

\path[draw=drawColor,line width= 0.4pt,line join=round,line cap=round,fill=fillColor] (430.91,565.53) circle (  1.16);

\path[draw=drawColor,line width= 0.4pt,line join=round,line cap=round,fill=fillColor] (431.14,565.50) circle (  1.16);

\path[draw=drawColor,line width= 0.4pt,line join=round,line cap=round,fill=fillColor] (431.38,565.33) circle (  1.16);

\path[draw=drawColor,line width= 0.4pt,line join=round,line cap=round,fill=fillColor] (431.61,565.31) circle (  1.16);

\path[draw=drawColor,line width= 0.4pt,line join=round,line cap=round,fill=fillColor] (431.84,565.16) circle (  1.16);

\path[draw=drawColor,line width= 0.4pt,line join=round,line cap=round,fill=fillColor] (432.07,565.08) circle (  1.16);

\path[draw=drawColor,line width= 0.4pt,line join=round,line cap=round,fill=fillColor] (432.30,565.06) circle (  1.16);

\path[draw=drawColor,line width= 0.4pt,line join=round,line cap=round,fill=fillColor] (432.53,564.69) circle (  1.16);

\path[draw=drawColor,line width= 0.4pt,line join=round,line cap=round,fill=fillColor] (432.76,564.58) circle (  1.16);

\path[draw=drawColor,line width= 0.4pt,line join=round,line cap=round,fill=fillColor] (432.99,564.51) circle (  1.16);

\path[draw=drawColor,line width= 0.4pt,line join=round,line cap=round,fill=fillColor] (433.21,564.45) circle (  1.16);

\path[draw=drawColor,line width= 0.4pt,line join=round,line cap=round,fill=fillColor] (433.44,564.11) circle (  1.16);

\path[draw=drawColor,line width= 0.4pt,line join=round,line cap=round,fill=fillColor] (433.67,564.06) circle (  1.16);

\path[draw=drawColor,line width= 0.4pt,line join=round,line cap=round,fill=fillColor] (433.89,564.05) circle (  1.16);

\path[draw=drawColor,line width= 0.4pt,line join=round,line cap=round,fill=fillColor] (434.11,563.97) circle (  1.16);

\path[draw=drawColor,line width= 0.4pt,line join=round,line cap=round,fill=fillColor] (434.34,563.79) circle (  1.16);

\path[draw=drawColor,line width= 0.4pt,line join=round,line cap=round,fill=fillColor] (434.56,563.79) circle (  1.16);

\path[draw=drawColor,line width= 0.4pt,line join=round,line cap=round,fill=fillColor] (434.78,563.78) circle (  1.16);

\path[draw=drawColor,line width= 0.4pt,line join=round,line cap=round,fill=fillColor] (435.00,563.45) circle (  1.16);

\path[draw=drawColor,line width= 0.4pt,line join=round,line cap=round,fill=fillColor] (435.22,563.37) circle (  1.16);

\path[draw=drawColor,line width= 0.4pt,line join=round,line cap=round,fill=fillColor] (435.44,563.32) circle (  1.16);

\path[draw=drawColor,line width= 0.4pt,line join=round,line cap=round,fill=fillColor] (435.66,563.01) circle (  1.16);

\path[draw=drawColor,line width= 0.4pt,line join=round,line cap=round,fill=fillColor] (435.88,562.95) circle (  1.16);

\path[draw=drawColor,line width= 0.4pt,line join=round,line cap=round,fill=fillColor] (436.09,562.85) circle (  1.16);

\path[draw=drawColor,line width= 0.4pt,line join=round,line cap=round,fill=fillColor] (436.31,562.75) circle (  1.16);

\path[draw=drawColor,line width= 0.4pt,line join=round,line cap=round,fill=fillColor] (436.52,562.67) circle (  1.16);

\path[draw=drawColor,line width= 0.4pt,line join=round,line cap=round,fill=fillColor] (436.74,562.55) circle (  1.16);

\path[draw=drawColor,line width= 0.4pt,line join=round,line cap=round,fill=fillColor] (436.95,562.49) circle (  1.16);

\path[draw=drawColor,line width= 0.4pt,line join=round,line cap=round,fill=fillColor] (437.17,562.36) circle (  1.16);

\path[draw=drawColor,line width= 0.4pt,line join=round,line cap=round,fill=fillColor] (437.38,562.28) circle (  1.16);

\path[draw=drawColor,line width= 0.4pt,line join=round,line cap=round,fill=fillColor] (437.59,562.27) circle (  1.16);

\path[draw=drawColor,line width= 0.4pt,line join=round,line cap=round,fill=fillColor] (437.80,562.13) circle (  1.16);

\path[draw=drawColor,line width= 0.4pt,line join=round,line cap=round,fill=fillColor] (438.01,562.08) circle (  1.16);

\path[draw=drawColor,line width= 0.4pt,line join=round,line cap=round,fill=fillColor] (438.22,562.07) circle (  1.16);

\path[draw=drawColor,line width= 0.4pt,line join=round,line cap=round,fill=fillColor] (438.43,562.03) circle (  1.16);

\path[draw=drawColor,line width= 0.4pt,line join=round,line cap=round,fill=fillColor] (438.64,561.99) circle (  1.16);

\path[draw=drawColor,line width= 0.4pt,line join=round,line cap=round,fill=fillColor] (438.85,561.98) circle (  1.16);

\path[draw=drawColor,line width= 0.4pt,line join=round,line cap=round,fill=fillColor] (439.06,561.77) circle (  1.16);

\path[draw=drawColor,line width= 0.4pt,line join=round,line cap=round,fill=fillColor] (439.26,561.65) circle (  1.16);

\path[draw=drawColor,line width= 0.4pt,line join=round,line cap=round,fill=fillColor] (439.47,561.62) circle (  1.16);

\path[draw=drawColor,line width= 0.4pt,line join=round,line cap=round,fill=fillColor] (439.67,561.55) circle (  1.16);

\path[draw=drawColor,line width= 0.4pt,line join=round,line cap=round,fill=fillColor] (439.88,561.41) circle (  1.16);

\path[draw=drawColor,line width= 0.4pt,line join=round,line cap=round,fill=fillColor] (440.08,561.39) circle (  1.16);

\path[draw=drawColor,line width= 0.4pt,line join=round,line cap=round,fill=fillColor] (440.29,561.08) circle (  1.16);

\path[draw=drawColor,line width= 0.4pt,line join=round,line cap=round,fill=fillColor] (440.49,561.07) circle (  1.16);

\path[draw=drawColor,line width= 0.4pt,line join=round,line cap=round,fill=fillColor] (440.69,560.79) circle (  1.16);

\path[draw=drawColor,line width= 0.4pt,line join=round,line cap=round,fill=fillColor] (440.89,560.63) circle (  1.16);

\path[draw=drawColor,line width= 0.4pt,line join=round,line cap=round,fill=fillColor] (441.09,560.51) circle (  1.16);

\path[draw=drawColor,line width= 0.4pt,line join=round,line cap=round,fill=fillColor] (441.29,560.41) circle (  1.16);

\path[draw=drawColor,line width= 0.4pt,line join=round,line cap=round,fill=fillColor] (441.49,560.19) circle (  1.16);

\path[draw=drawColor,line width= 0.4pt,line join=round,line cap=round,fill=fillColor] (441.69,560.18) circle (  1.16);

\path[draw=drawColor,line width= 0.4pt,line join=round,line cap=round,fill=fillColor] (441.89,560.11) circle (  1.16);

\path[draw=drawColor,line width= 0.4pt,line join=round,line cap=round,fill=fillColor] (442.09,560.09) circle (  1.16);

\path[draw=drawColor,line width= 0.4pt,line join=round,line cap=round,fill=fillColor] (442.29,560.04) circle (  1.16);

\path[draw=drawColor,line width= 0.4pt,line join=round,line cap=round,fill=fillColor] (442.49,559.98) circle (  1.16);

\path[draw=drawColor,line width= 0.4pt,line join=round,line cap=round,fill=fillColor] (442.68,559.96) circle (  1.16);

\path[draw=drawColor,line width= 0.4pt,line join=round,line cap=round,fill=fillColor] (442.88,559.80) circle (  1.16);

\path[draw=drawColor,line width= 0.4pt,line join=round,line cap=round,fill=fillColor] (443.07,559.53) circle (  1.16);

\path[draw=drawColor,line width= 0.4pt,line join=round,line cap=round,fill=fillColor] (443.27,559.35) circle (  1.16);

\path[draw=drawColor,line width= 0.4pt,line join=round,line cap=round,fill=fillColor] (443.46,559.01) circle (  1.16);

\path[draw=drawColor,line width= 0.4pt,line join=round,line cap=round,fill=fillColor] (443.66,559.00) circle (  1.16);

\path[draw=drawColor,line width= 0.4pt,line join=round,line cap=round,fill=fillColor] (443.85,558.95) circle (  1.16);

\path[draw=drawColor,line width= 0.4pt,line join=round,line cap=round,fill=fillColor] (444.04,558.95) circle (  1.16);

\path[draw=drawColor,line width= 0.4pt,line join=round,line cap=round,fill=fillColor] (444.23,558.93) circle (  1.16);

\path[draw=drawColor,line width= 0.4pt,line join=round,line cap=round,fill=fillColor] (444.43,558.79) circle (  1.16);

\path[draw=drawColor,line width= 0.4pt,line join=round,line cap=round,fill=fillColor] (444.62,558.73) circle (  1.16);

\path[draw=drawColor,line width= 0.4pt,line join=round,line cap=round,fill=fillColor] (444.81,558.70) circle (  1.16);

\path[draw=drawColor,line width= 0.4pt,line join=round,line cap=round,fill=fillColor] (445.00,558.65) circle (  1.16);

\path[draw=drawColor,line width= 0.4pt,line join=round,line cap=round,fill=fillColor] (445.19,558.64) circle (  1.16);

\path[draw=drawColor,line width= 0.4pt,line join=round,line cap=round,fill=fillColor] (445.38,558.53) circle (  1.16);

\path[draw=drawColor,line width= 0.4pt,line join=round,line cap=round,fill=fillColor] (445.56,558.53) circle (  1.16);

\path[draw=drawColor,line width= 0.4pt,line join=round,line cap=round,fill=fillColor] (445.75,558.49) circle (  1.16);

\path[draw=drawColor,line width= 0.4pt,line join=round,line cap=round,fill=fillColor] (445.94,558.44) circle (  1.16);

\path[draw=drawColor,line width= 0.4pt,line join=round,line cap=round,fill=fillColor] (446.13,558.41) circle (  1.16);

\path[draw=drawColor,line width= 0.4pt,line join=round,line cap=round,fill=fillColor] (446.31,557.84) circle (  1.16);

\path[draw=drawColor,line width= 0.4pt,line join=round,line cap=round,fill=fillColor] (446.50,557.84) circle (  1.16);

\path[draw=drawColor,line width= 0.4pt,line join=round,line cap=round,fill=fillColor] (446.68,557.67) circle (  1.16);

\path[draw=drawColor,line width= 0.4pt,line join=round,line cap=round,fill=fillColor] (446.87,557.65) circle (  1.16);

\path[draw=drawColor,line width= 0.4pt,line join=round,line cap=round,fill=fillColor] (447.05,557.65) circle (  1.16);

\path[draw=drawColor,line width= 0.4pt,line join=round,line cap=round,fill=fillColor] (447.24,557.31) circle (  1.16);

\path[draw=drawColor,line width= 0.4pt,line join=round,line cap=round,fill=fillColor] (447.42,557.15) circle (  1.16);

\path[draw=drawColor,line width= 0.4pt,line join=round,line cap=round,fill=fillColor] (447.60,557.10) circle (  1.16);

\path[draw=drawColor,line width= 0.4pt,line join=round,line cap=round,fill=fillColor] (447.79,557.07) circle (  1.16);

\path[draw=drawColor,line width= 0.4pt,line join=round,line cap=round,fill=fillColor] (447.97,557.06) circle (  1.16);

\path[draw=drawColor,line width= 0.4pt,line join=round,line cap=round,fill=fillColor] (448.15,557.06) circle (  1.16);

\path[draw=drawColor,line width= 0.4pt,line join=round,line cap=round,fill=fillColor] (448.33,557.01) circle (  1.16);

\path[draw=drawColor,line width= 0.4pt,line join=round,line cap=round,fill=fillColor] (448.51,556.74) circle (  1.16);

\path[draw=drawColor,line width= 0.4pt,line join=round,line cap=round,fill=fillColor] (448.69,556.71) circle (  1.16);

\path[draw=drawColor,line width= 0.4pt,line join=round,line cap=round,fill=fillColor] (448.87,556.70) circle (  1.16);

\path[draw=drawColor,line width= 0.4pt,line join=round,line cap=round,fill=fillColor] (449.05,556.57) circle (  1.16);

\path[draw=drawColor,line width= 0.4pt,line join=round,line cap=round,fill=fillColor] (449.23,556.23) circle (  1.16);

\path[draw=drawColor,line width= 0.4pt,line join=round,line cap=round,fill=fillColor] (449.41,556.07) circle (  1.16);

\path[draw=drawColor,line width= 0.4pt,line join=round,line cap=round,fill=fillColor] (449.59,555.96) circle (  1.16);

\path[draw=drawColor,line width= 0.4pt,line join=round,line cap=round,fill=fillColor] (449.77,555.77) circle (  1.16);

\path[draw=drawColor,line width= 0.4pt,line join=round,line cap=round,fill=fillColor] (449.94,555.75) circle (  1.16);

\path[draw=drawColor,line width= 0.4pt,line join=round,line cap=round,fill=fillColor] (450.12,555.73) circle (  1.16);

\path[draw=drawColor,line width= 0.4pt,line join=round,line cap=round,fill=fillColor] (450.30,555.71) circle (  1.16);

\path[draw=drawColor,line width= 0.4pt,line join=round,line cap=round,fill=fillColor] (450.47,555.70) circle (  1.16);

\path[draw=drawColor,line width= 0.4pt,line join=round,line cap=round,fill=fillColor] (450.65,555.57) circle (  1.16);

\path[draw=drawColor,line width= 0.4pt,line join=round,line cap=round,fill=fillColor] (450.82,555.49) circle (  1.16);

\path[draw=drawColor,line width= 0.4pt,line join=round,line cap=round,fill=fillColor] (451.00,555.45) circle (  1.16);

\path[draw=drawColor,line width= 0.4pt,line join=round,line cap=round,fill=fillColor] (451.17,555.43) circle (  1.16);

\path[draw=drawColor,line width= 0.4pt,line join=round,line cap=round,fill=fillColor] (451.35,555.37) circle (  1.16);

\path[draw=drawColor,line width= 0.4pt,line join=round,line cap=round,fill=fillColor] (451.52,555.13) circle (  1.16);

\path[draw=drawColor,line width= 0.4pt,line join=round,line cap=round,fill=fillColor] (451.69,555.10) circle (  1.16);

\path[draw=drawColor,line width= 0.4pt,line join=round,line cap=round,fill=fillColor] (451.86,555.04) circle (  1.16);

\path[draw=drawColor,line width= 0.4pt,line join=round,line cap=round,fill=fillColor] (452.04,555.00) circle (  1.16);

\path[draw=drawColor,line width= 0.4pt,line join=round,line cap=round,fill=fillColor] (452.21,555.00) circle (  1.16);

\path[draw=drawColor,line width= 0.4pt,line join=round,line cap=round,fill=fillColor] (452.38,554.99) circle (  1.16);

\path[draw=drawColor,line width= 0.4pt,line join=round,line cap=round,fill=fillColor] (452.55,554.86) circle (  1.16);

\path[draw=drawColor,line width= 0.4pt,line join=round,line cap=round,fill=fillColor] (452.72,554.85) circle (  1.16);

\path[draw=drawColor,line width= 0.4pt,line join=round,line cap=round,fill=fillColor] (452.89,554.72) circle (  1.16);

\path[draw=drawColor,line width= 0.4pt,line join=round,line cap=round,fill=fillColor] (453.06,554.64) circle (  1.16);

\path[draw=drawColor,line width= 0.4pt,line join=round,line cap=round,fill=fillColor] (453.23,554.60) circle (  1.16);

\path[draw=drawColor,line width= 0.4pt,line join=round,line cap=round,fill=fillColor] (453.40,554.45) circle (  1.16);

\path[draw=drawColor,line width= 0.4pt,line join=round,line cap=round,fill=fillColor] (453.57,554.32) circle (  1.16);

\path[draw=drawColor,line width= 0.4pt,line join=round,line cap=round,fill=fillColor] (453.74,554.30) circle (  1.16);

\path[draw=drawColor,line width= 0.4pt,line join=round,line cap=round,fill=fillColor] (453.90,554.28) circle (  1.16);

\path[draw=drawColor,line width= 0.4pt,line join=round,line cap=round,fill=fillColor] (454.07,554.27) circle (  1.16);

\path[draw=drawColor,line width= 0.4pt,line join=round,line cap=round,fill=fillColor] (454.24,554.13) circle (  1.16);

\path[draw=drawColor,line width= 0.4pt,line join=round,line cap=round,fill=fillColor] (454.40,554.10) circle (  1.16);

\path[draw=drawColor,line width= 0.4pt,line join=round,line cap=round,fill=fillColor] (454.57,553.98) circle (  1.16);

\path[draw=drawColor,line width= 0.4pt,line join=round,line cap=round,fill=fillColor] (454.74,553.80) circle (  1.16);

\path[draw=drawColor,line width= 0.4pt,line join=round,line cap=round,fill=fillColor] (454.90,553.60) circle (  1.16);

\path[draw=drawColor,line width= 0.4pt,line join=round,line cap=round,fill=fillColor] (455.07,553.54) circle (  1.16);

\path[draw=drawColor,line width= 0.4pt,line join=round,line cap=round,fill=fillColor] (455.23,553.44) circle (  1.16);

\path[draw=drawColor,line width= 0.4pt,line join=round,line cap=round,fill=fillColor] (455.40,552.99) circle (  1.16);

\path[draw=drawColor,line width= 0.4pt,line join=round,line cap=round,fill=fillColor] (455.56,552.84) circle (  1.16);

\path[draw=drawColor,line width= 0.4pt,line join=round,line cap=round,fill=fillColor] (455.73,552.58) circle (  1.16);

\path[draw=drawColor,line width= 0.4pt,line join=round,line cap=round,fill=fillColor] (455.89,552.47) circle (  1.16);

\path[draw=drawColor,line width= 0.4pt,line join=round,line cap=round,fill=fillColor] (456.05,552.40) circle (  1.16);

\path[draw=drawColor,line width= 0.4pt,line join=round,line cap=round,fill=fillColor] (456.21,552.35) circle (  1.16);

\path[draw=drawColor,line width= 0.4pt,line join=round,line cap=round,fill=fillColor] (456.38,552.22) circle (  1.16);

\path[draw=drawColor,line width= 0.4pt,line join=round,line cap=round,fill=fillColor] (456.54,552.16) circle (  1.16);

\path[draw=drawColor,line width= 0.4pt,line join=round,line cap=round,fill=fillColor] (456.70,552.11) circle (  1.16);

\path[draw=drawColor,line width= 0.4pt,line join=round,line cap=round,fill=fillColor] (456.86,551.96) circle (  1.16);

\path[draw=drawColor,line width= 0.4pt,line join=round,line cap=round,fill=fillColor] (457.02,551.87) circle (  1.16);

\path[draw=drawColor,line width= 0.4pt,line join=round,line cap=round,fill=fillColor] (457.18,551.75) circle (  1.16);

\path[draw=drawColor,line width= 0.4pt,line join=round,line cap=round,fill=fillColor] (457.34,551.70) circle (  1.16);

\path[draw=drawColor,line width= 0.4pt,line join=round,line cap=round,fill=fillColor] (457.50,551.69) circle (  1.16);

\path[draw=drawColor,line width= 0.4pt,line join=round,line cap=round,fill=fillColor] (457.66,551.37) circle (  1.16);

\path[draw=drawColor,line width= 0.4pt,line join=round,line cap=round,fill=fillColor] (457.82,551.06) circle (  1.16);

\path[draw=drawColor,line width= 0.4pt,line join=round,line cap=round,fill=fillColor] (457.98,550.91) circle (  1.16);

\path[draw=drawColor,line width= 0.4pt,line join=round,line cap=round,fill=fillColor] (458.14,550.91) circle (  1.16);

\path[draw=drawColor,line width= 0.4pt,line join=round,line cap=round,fill=fillColor] (458.30,550.59) circle (  1.16);

\path[draw=drawColor,line width= 0.4pt,line join=round,line cap=round,fill=fillColor] (458.46,550.59) circle (  1.16);

\path[draw=drawColor,line width= 0.4pt,line join=round,line cap=round,fill=fillColor] (458.61,550.57) circle (  1.16);

\path[draw=drawColor,line width= 0.4pt,line join=round,line cap=round,fill=fillColor] (458.77,550.50) circle (  1.16);

\path[draw=drawColor,line width= 0.4pt,line join=round,line cap=round,fill=fillColor] (458.93,550.30) circle (  1.16);

\path[draw=drawColor,line width= 0.4pt,line join=round,line cap=round,fill=fillColor] (459.09,550.26) circle (  1.16);

\path[draw=drawColor,line width= 0.4pt,line join=round,line cap=round,fill=fillColor] (459.24,550.04) circle (  1.16);

\path[draw=drawColor,line width= 0.4pt,line join=round,line cap=round,fill=fillColor] (459.40,549.83) circle (  1.16);

\path[draw=drawColor,line width= 0.4pt,line join=round,line cap=round,fill=fillColor] (459.55,549.78) circle (  1.16);

\path[draw=drawColor,line width= 0.4pt,line join=round,line cap=round,fill=fillColor] (459.71,549.39) circle (  1.16);

\path[draw=drawColor,line width= 0.4pt,line join=round,line cap=round,fill=fillColor] (459.86,549.26) circle (  1.16);

\path[draw=drawColor,line width= 0.4pt,line join=round,line cap=round,fill=fillColor] (460.02,549.17) circle (  1.16);

\path[draw=drawColor,line width= 0.4pt,line join=round,line cap=round,fill=fillColor] (460.17,548.85) circle (  1.16);

\path[draw=drawColor,line width= 0.4pt,line join=round,line cap=round,fill=fillColor] (460.33,548.76) circle (  1.16);

\path[draw=drawColor,line width= 0.4pt,line join=round,line cap=round,fill=fillColor] (460.48,548.76) circle (  1.16);

\path[draw=drawColor,line width= 0.4pt,line join=round,line cap=round,fill=fillColor] (460.63,548.70) circle (  1.16);

\path[draw=drawColor,line width= 0.4pt,line join=round,line cap=round,fill=fillColor] (460.79,548.41) circle (  1.16);

\path[draw=drawColor,line width= 0.4pt,line join=round,line cap=round,fill=fillColor] (460.94,547.96) circle (  1.16);

\path[draw=drawColor,line width= 0.4pt,line join=round,line cap=round,fill=fillColor] (461.09,547.95) circle (  1.16);

\path[draw=drawColor,line width= 0.4pt,line join=round,line cap=round,fill=fillColor] (461.25,547.94) circle (  1.16);

\path[draw=drawColor,line width= 0.4pt,line join=round,line cap=round,fill=fillColor] (461.40,547.84) circle (  1.16);

\path[draw=drawColor,line width= 0.4pt,line join=round,line cap=round,fill=fillColor] (461.55,547.75) circle (  1.16);

\path[draw=drawColor,line width= 0.4pt,line join=round,line cap=round,fill=fillColor] (461.70,547.57) circle (  1.16);

\path[draw=drawColor,line width= 0.4pt,line join=round,line cap=round,fill=fillColor] (461.85,547.04) circle (  1.16);

\path[draw=drawColor,line width= 0.4pt,line join=round,line cap=round,fill=fillColor] (462.00,547.00) circle (  1.16);

\path[draw=drawColor,line width= 0.4pt,line join=round,line cap=round,fill=fillColor] (462.15,546.88) circle (  1.16);

\path[draw=drawColor,line width= 0.4pt,line join=round,line cap=round,fill=fillColor] (462.30,546.74) circle (  1.16);

\path[draw=drawColor,line width= 0.4pt,line join=round,line cap=round,fill=fillColor] (462.45,546.49) circle (  1.16);

\path[draw=drawColor,line width= 0.4pt,line join=round,line cap=round,fill=fillColor] (462.60,546.18) circle (  1.16);

\path[draw=drawColor,line width= 0.4pt,line join=round,line cap=round,fill=fillColor] (462.75,546.11) circle (  1.16);

\path[draw=drawColor,line width= 0.4pt,line join=round,line cap=round,fill=fillColor] (462.90,545.96) circle (  1.16);

\path[draw=drawColor,line width= 0.4pt,line join=round,line cap=round,fill=fillColor] (463.05,545.91) circle (  1.16);

\path[draw=drawColor,line width= 0.4pt,line join=round,line cap=round,fill=fillColor] (463.20,545.87) circle (  1.16);

\path[draw=drawColor,line width= 0.4pt,line join=round,line cap=round,fill=fillColor] (463.35,545.76) circle (  1.16);

\path[draw=drawColor,line width= 0.4pt,line join=round,line cap=round,fill=fillColor] (463.50,545.38) circle (  1.16);

\path[draw=drawColor,line width= 0.4pt,line join=round,line cap=round,fill=fillColor] (463.64,545.33) circle (  1.16);

\path[draw=drawColor,line width= 0.4pt,line join=round,line cap=round,fill=fillColor] (463.79,545.00) circle (  1.16);

\path[draw=drawColor,line width= 0.4pt,line join=round,line cap=round,fill=fillColor] (463.94,544.82) circle (  1.16);

\path[draw=drawColor,line width= 0.4pt,line join=round,line cap=round,fill=fillColor] (464.09,544.03) circle (  1.16);

\path[draw=drawColor,line width= 0.4pt,line join=round,line cap=round,fill=fillColor] (464.23,543.95) circle (  1.16);

\path[draw=drawColor,line width= 0.4pt,line join=round,line cap=round,fill=fillColor] (464.38,543.51) circle (  1.16);

\path[draw=drawColor,line width= 0.4pt,line join=round,line cap=round,fill=fillColor] (464.53,542.91) circle (  1.16);

\path[draw=drawColor,line width= 0.4pt,line join=round,line cap=round,fill=fillColor] (464.67,542.76) circle (  1.16);

\path[draw=drawColor,line width= 0.4pt,line join=round,line cap=round,fill=fillColor] (464.82,542.66) circle (  1.16);

\path[draw=drawColor,line width= 0.4pt,line join=round,line cap=round,fill=fillColor] (464.96,542.45) circle (  1.16);

\path[draw=drawColor,line width= 0.4pt,line join=round,line cap=round,fill=fillColor] (465.11,542.29) circle (  1.16);

\path[draw=drawColor,line width= 0.4pt,line join=round,line cap=round,fill=fillColor] (465.25,541.92) circle (  1.16);

\path[draw=drawColor,line width= 0.4pt,line join=round,line cap=round,fill=fillColor] (465.40,541.88) circle (  1.16);

\path[draw=drawColor,line width= 0.4pt,line join=round,line cap=round,fill=fillColor] (465.54,541.77) circle (  1.16);

\path[draw=drawColor,line width= 0.4pt,line join=round,line cap=round,fill=fillColor] (465.68,540.74) circle (  1.16);

\path[draw=drawColor,line width= 0.4pt,line join=round,line cap=round,fill=fillColor] (465.83,540.49) circle (  1.16);

\path[draw=drawColor,line width= 0.4pt,line join=round,line cap=round,fill=fillColor] (465.97,539.87) circle (  1.16);

\path[draw=drawColor,line width= 0.4pt,line join=round,line cap=round,fill=fillColor] (466.12,535.03) circle (  1.16);

\path[draw=drawColor,line width= 0.4pt,line join=round,line cap=round,fill=fillColor] (466.26,535.03) circle (  1.16);

\path[draw=drawColor,line width= 0.4pt,line join=round,line cap=round,fill=fillColor] (466.40,535.03) circle (  1.16);

\path[draw=drawColor,line width= 0.4pt,line join=round,line cap=round,fill=fillColor] (466.54,535.03) circle (  1.16);

\path[draw=drawColor,line width= 0.4pt,line join=round,line cap=round,fill=fillColor] (466.69,535.03) circle (  1.16);

\path[draw=drawColor,line width= 0.4pt,line join=round,line cap=round,fill=fillColor] (466.83,535.03) circle (  1.16);

\path[draw=drawColor,line width= 0.4pt,line join=round,line cap=round,fill=fillColor] (466.97,535.03) circle (  1.16);

\path[draw=drawColor,line width= 0.4pt,line join=round,line cap=round,fill=fillColor] (467.11,535.03) circle (  1.16);

\path[draw=drawColor,line width= 0.4pt,line join=round,line cap=round,fill=fillColor] (467.25,535.03) circle (  1.16);

\path[draw=drawColor,line width= 0.4pt,line join=round,line cap=round,fill=fillColor] (467.39,535.03) circle (  1.16);

\path[draw=drawColor,line width= 0.4pt,line join=round,line cap=round,fill=fillColor] (467.53,535.03) circle (  1.16);

\path[draw=drawColor,line width= 0.4pt,line join=round,line cap=round,fill=fillColor] (467.67,535.03) circle (  1.16);

\path[draw=drawColor,line width= 0.4pt,line join=round,line cap=round,fill=fillColor] (467.81,535.03) circle (  1.16);

\path[draw=drawColor,line width= 0.4pt,line join=round,line cap=round,fill=fillColor] (467.95,535.03) circle (  1.16);

\path[draw=drawColor,line width= 0.4pt,line join=round,line cap=round,fill=fillColor] (468.09,535.03) circle (  1.16);

\path[draw=drawColor,line width= 0.4pt,line join=round,line cap=round,fill=fillColor] (468.23,535.03) circle (  1.16);

\path[draw=drawColor,line width= 0.4pt,line join=round,line cap=round,fill=fillColor] (468.37,535.03) circle (  1.16);

\path[draw=drawColor,line width= 0.4pt,line join=round,line cap=round,fill=fillColor] (468.51,535.03) circle (  1.16);

\path[draw=drawColor,line width= 0.4pt,line join=round,line cap=round,fill=fillColor] (468.65,535.03) circle (  1.16);

\path[draw=drawColor,line width= 0.4pt,line join=round,line cap=round,fill=fillColor] (468.79,535.03) circle (  1.16);

\path[draw=drawColor,line width= 0.4pt,line join=round,line cap=round,fill=fillColor] (468.93,535.03) circle (  1.16);

\path[draw=drawColor,line width= 0.4pt,line join=round,line cap=round,fill=fillColor] (469.07,535.03) circle (  1.16);

\path[draw=drawColor,line width= 0.4pt,line join=round,line cap=round,fill=fillColor] (469.20,535.03) circle (  1.16);

\path[draw=drawColor,line width= 0.4pt,line join=round,line cap=round,fill=fillColor] (469.34,535.03) circle (  1.16);

\path[draw=drawColor,line width= 0.4pt,line join=round,line cap=round,fill=fillColor] (469.48,535.03) circle (  1.16);

\path[draw=drawColor,line width= 0.4pt,line join=round,line cap=round,fill=fillColor] (469.62,535.03) circle (  1.16);

\path[draw=drawColor,line width= 0.4pt,line join=round,line cap=round,fill=fillColor] (469.75,535.03) circle (  1.16);

\path[draw=drawColor,line width= 0.4pt,line join=round,line cap=round,fill=fillColor] (469.89,535.03) circle (  1.16);

\path[draw=drawColor,line width= 0.4pt,line join=round,line cap=round,fill=fillColor] (470.03,535.03) circle (  1.16);

\path[draw=drawColor,line width= 0.4pt,line join=round,line cap=round,fill=fillColor] (470.16,535.03) circle (  1.16);

\path[draw=drawColor,line width= 0.4pt,line join=round,line cap=round,fill=fillColor] (470.30,535.03) circle (  1.16);

\path[draw=drawColor,line width= 0.4pt,line join=round,line cap=round,fill=fillColor] (470.43,535.03) circle (  1.16);

\path[draw=drawColor,line width= 0.4pt,line join=round,line cap=round,fill=fillColor] (470.57,535.03) circle (  1.16);

\path[draw=drawColor,line width= 0.4pt,line join=round,line cap=round,fill=fillColor] (470.71,535.03) circle (  1.16);

\path[draw=drawColor,line width= 0.4pt,line join=round,line cap=round,fill=fillColor] (470.84,535.03) circle (  1.16);

\path[draw=drawColor,line width= 0.4pt,line join=round,line cap=round,fill=fillColor] (470.98,535.03) circle (  1.16);

\path[draw=drawColor,line width= 0.4pt,line join=round,line cap=round,fill=fillColor] (471.11,535.03) circle (  1.16);

\path[draw=drawColor,line width= 0.4pt,line join=round,line cap=round,fill=fillColor] (471.24,535.03) circle (  1.16);

\path[draw=drawColor,line width= 0.4pt,line join=round,line cap=round,fill=fillColor] (471.38,535.03) circle (  1.16);

\path[draw=drawColor,line width= 0.4pt,line join=round,line cap=round,fill=fillColor] (471.51,535.03) circle (  1.16);

\path[draw=drawColor,line width= 0.4pt,line join=round,line cap=round,fill=fillColor] (471.65,535.03) circle (  1.16);

\path[draw=drawColor,line width= 0.4pt,line join=round,line cap=round,fill=fillColor] (471.78,535.03) circle (  1.16);

\path[draw=drawColor,line width= 0.4pt,line join=round,line cap=round,fill=fillColor] (471.91,535.03) circle (  1.16);

\path[draw=drawColor,line width= 0.4pt,line join=round,line cap=round,fill=fillColor] (472.05,535.03) circle (  1.16);

\path[draw=drawColor,line width= 0.4pt,line join=round,line cap=round,fill=fillColor] (472.18,535.03) circle (  1.16);

\path[draw=drawColor,line width= 0.4pt,line join=round,line cap=round,fill=fillColor] (472.31,535.03) circle (  1.16);

\path[draw=drawColor,line width= 0.4pt,line join=round,line cap=round,fill=fillColor] (472.45,535.03) circle (  1.16);

\path[draw=drawColor,line width= 0.4pt,line join=round,line cap=round,fill=fillColor] (472.58,535.03) circle (  1.16);

\path[draw=drawColor,line width= 0.4pt,line join=round,line cap=round,fill=fillColor] (472.71,535.03) circle (  1.16);

\path[draw=drawColor,line width= 0.4pt,line join=round,line cap=round,fill=fillColor] (472.84,535.03) circle (  1.16);

\path[draw=drawColor,line width= 0.4pt,line join=round,line cap=round,fill=fillColor] (472.97,535.03) circle (  1.16);

\path[draw=drawColor,line width= 0.4pt,line join=round,line cap=round,fill=fillColor] (473.11,535.03) circle (  1.16);

\path[draw=drawColor,line width= 0.4pt,line join=round,line cap=round,fill=fillColor] (473.24,535.03) circle (  1.16);

\path[draw=drawColor,line width= 0.4pt,line join=round,line cap=round,fill=fillColor] (473.37,535.03) circle (  1.16);

\path[draw=drawColor,line width= 0.4pt,line join=round,line cap=round,fill=fillColor] (473.50,535.03) circle (  1.16);

\path[draw=drawColor,line width= 0.4pt,line join=round,line cap=round,fill=fillColor] (473.63,535.03) circle (  1.16);

\path[draw=drawColor,line width= 0.4pt,line join=round,line cap=round,fill=fillColor] (473.76,535.03) circle (  1.16);

\path[draw=drawColor,line width= 0.4pt,line join=round,line cap=round,fill=fillColor] (473.89,535.03) circle (  1.16);

\path[draw=drawColor,line width= 0.4pt,line join=round,line cap=round,fill=fillColor] (474.02,535.03) circle (  1.16);

\path[draw=drawColor,line width= 0.4pt,line join=round,line cap=round,fill=fillColor] (474.15,535.03) circle (  1.16);

\path[draw=drawColor,line width= 0.4pt,line join=round,line cap=round,fill=fillColor] (474.28,535.03) circle (  1.16);

\path[draw=drawColor,line width= 0.4pt,line join=round,line cap=round,fill=fillColor] (474.41,535.03) circle (  1.16);

\path[draw=drawColor,line width= 0.4pt,line join=round,line cap=round,fill=fillColor] (474.54,535.03) circle (  1.16);

\path[draw=drawColor,line width= 0.4pt,line join=round,line cap=round,fill=fillColor] (474.67,535.03) circle (  1.16);

\path[draw=drawColor,line width= 0.4pt,line join=round,line cap=round,fill=fillColor] (474.80,535.03) circle (  1.16);

\path[draw=drawColor,line width= 0.4pt,line join=round,line cap=round,fill=fillColor] (474.93,535.03) circle (  1.16);

\path[draw=drawColor,line width= 0.4pt,line join=round,line cap=round,fill=fillColor] (475.05,535.03) circle (  1.16);

\path[draw=drawColor,line width= 0.4pt,line join=round,line cap=round,fill=fillColor] (475.18,535.03) circle (  1.16);

\path[draw=drawColor,line width= 0.4pt,line join=round,line cap=round,fill=fillColor] (475.31,535.03) circle (  1.16);

\path[draw=drawColor,line width= 0.4pt,line join=round,line cap=round,fill=fillColor] (475.44,535.03) circle (  1.16);

\path[draw=drawColor,line width= 0.4pt,line join=round,line cap=round,fill=fillColor] (475.57,535.03) circle (  1.16);

\path[draw=drawColor,line width= 0.4pt,line join=round,line cap=round,fill=fillColor] (475.69,535.03) circle (  1.16);

\path[draw=drawColor,line width= 0.4pt,line join=round,line cap=round,fill=fillColor] (475.82,535.03) circle (  1.16);

\path[draw=drawColor,line width= 0.4pt,line join=round,line cap=round,fill=fillColor] (475.95,535.03) circle (  1.16);

\path[draw=drawColor,line width= 0.4pt,line join=round,line cap=round,fill=fillColor] (476.08,535.03) circle (  1.16);

\path[draw=drawColor,line width= 0.4pt,line join=round,line cap=round,fill=fillColor] (476.20,535.03) circle (  1.16);

\path[draw=drawColor,line width= 0.4pt,line join=round,line cap=round,fill=fillColor] (476.33,535.03) circle (  1.16);

\path[draw=drawColor,line width= 0.4pt,line join=round,line cap=round,fill=fillColor] (476.46,535.03) circle (  1.16);

\path[draw=drawColor,line width= 0.4pt,line join=round,line cap=round,fill=fillColor] (476.58,535.03) circle (  1.16);

\path[draw=drawColor,line width= 0.4pt,line join=round,line cap=round,fill=fillColor] (476.71,535.03) circle (  1.16);

\path[draw=drawColor,line width= 0.4pt,line join=round,line cap=round,fill=fillColor] (476.83,535.03) circle (  1.16);

\path[draw=drawColor,line width= 0.4pt,line join=round,line cap=round,fill=fillColor] (476.96,535.03) circle (  1.16);

\path[draw=drawColor,line width= 0.4pt,line join=round,line cap=round,fill=fillColor] (477.08,535.03) circle (  1.16);

\path[draw=drawColor,line width= 0.4pt,line join=round,line cap=round,fill=fillColor] (477.21,535.03) circle (  1.16);

\path[draw=drawColor,line width= 0.4pt,line join=round,line cap=round,fill=fillColor] (477.33,535.03) circle (  1.16);

\path[draw=drawColor,line width= 0.4pt,line join=round,line cap=round,fill=fillColor] (477.46,535.03) circle (  1.16);

\path[draw=drawColor,line width= 0.4pt,line join=round,line cap=round,fill=fillColor] (477.58,535.03) circle (  1.16);

\path[draw=drawColor,line width= 0.4pt,line join=round,line cap=round,fill=fillColor] (477.71,535.03) circle (  1.16);

\path[draw=drawColor,line width= 0.4pt,line join=round,line cap=round,fill=fillColor] (477.83,535.03) circle (  1.16);

\path[draw=drawColor,line width= 0.4pt,line join=round,line cap=round,fill=fillColor] (477.96,535.03) circle (  1.16);

\path[draw=drawColor,line width= 0.4pt,line join=round,line cap=round,fill=fillColor] (478.08,535.03) circle (  1.16);

\path[draw=drawColor,line width= 0.4pt,line join=round,line cap=round,fill=fillColor] (478.21,535.03) circle (  1.16);
\definecolor[named]{drawColor}{rgb}{0.30,0.69,0.29}
\definecolor[named]{fillColor}{rgb}{0.30,0.69,0.29}

\path[draw=drawColor,line width= 0.4pt,line join=round,line cap=round,fill=fillColor] (327.82,620.67) circle (  1.16);

\path[draw=drawColor,line width= 0.4pt,line join=round,line cap=round,fill=fillColor] (333.62,620.67) circle (  1.16);

\path[draw=drawColor,line width= 0.4pt,line join=round,line cap=round,fill=fillColor] (337.68,620.67) circle (  1.16);

\path[draw=drawColor,line width= 0.4pt,line join=round,line cap=round,fill=fillColor] (340.92,620.67) circle (  1.16);

\path[draw=drawColor,line width= 0.4pt,line join=round,line cap=round,fill=fillColor] (343.65,620.67) circle (  1.16);

\path[draw=drawColor,line width= 0.4pt,line join=round,line cap=round,fill=fillColor] (346.04,620.67) circle (  1.16);

\path[draw=drawColor,line width= 0.4pt,line join=round,line cap=round,fill=fillColor] (348.17,620.67) circle (  1.16);

\path[draw=drawColor,line width= 0.4pt,line join=round,line cap=round,fill=fillColor] (350.11,620.67) circle (  1.16);

\path[draw=drawColor,line width= 0.4pt,line join=round,line cap=round,fill=fillColor] (351.90,620.67) circle (  1.16);

\path[draw=drawColor,line width= 0.4pt,line join=round,line cap=round,fill=fillColor] (353.55,620.67) circle (  1.16);

\path[draw=drawColor,line width= 0.4pt,line join=round,line cap=round,fill=fillColor] (355.10,620.67) circle (  1.16);

\path[draw=drawColor,line width= 0.4pt,line join=round,line cap=round,fill=fillColor] (356.56,620.67) circle (  1.16);

\path[draw=drawColor,line width= 0.4pt,line join=round,line cap=round,fill=fillColor] (357.94,620.67) circle (  1.16);

\path[draw=drawColor,line width= 0.4pt,line join=round,line cap=round,fill=fillColor] (359.25,620.67) circle (  1.16);

\path[draw=drawColor,line width= 0.4pt,line join=round,line cap=round,fill=fillColor] (360.50,620.67) circle (  1.16);

\path[draw=drawColor,line width= 0.4pt,line join=round,line cap=round,fill=fillColor] (361.70,620.67) circle (  1.16);

\path[draw=drawColor,line width= 0.4pt,line join=round,line cap=round,fill=fillColor] (362.84,620.67) circle (  1.16);

\path[draw=drawColor,line width= 0.4pt,line join=round,line cap=round,fill=fillColor] (363.95,617.14) circle (  1.16);

\path[draw=drawColor,line width= 0.4pt,line join=round,line cap=round,fill=fillColor] (365.01,615.97) circle (  1.16);

\path[draw=drawColor,line width= 0.4pt,line join=round,line cap=round,fill=fillColor] (366.03,615.50) circle (  1.16);

\path[draw=drawColor,line width= 0.4pt,line join=round,line cap=round,fill=fillColor] (367.03,614.52) circle (  1.16);

\path[draw=drawColor,line width= 0.4pt,line join=round,line cap=round,fill=fillColor] (367.99,613.73) circle (  1.16);

\path[draw=drawColor,line width= 0.4pt,line join=round,line cap=round,fill=fillColor] (368.92,613.38) circle (  1.16);

\path[draw=drawColor,line width= 0.4pt,line join=round,line cap=round,fill=fillColor] (369.83,613.03) circle (  1.16);

\path[draw=drawColor,line width= 0.4pt,line join=round,line cap=round,fill=fillColor] (370.71,612.79) circle (  1.16);

\path[draw=drawColor,line width= 0.4pt,line join=round,line cap=round,fill=fillColor] (371.56,612.39) circle (  1.16);

\path[draw=drawColor,line width= 0.4pt,line join=round,line cap=round,fill=fillColor] (372.40,612.13) circle (  1.16);

\path[draw=drawColor,line width= 0.4pt,line join=round,line cap=round,fill=fillColor] (373.21,612.01) circle (  1.16);

\path[draw=drawColor,line width= 0.4pt,line join=round,line cap=round,fill=fillColor] (374.01,611.88) circle (  1.16);

\path[draw=drawColor,line width= 0.4pt,line join=round,line cap=round,fill=fillColor] (374.79,610.77) circle (  1.16);

\path[draw=drawColor,line width= 0.4pt,line join=round,line cap=round,fill=fillColor] (375.55,610.73) circle (  1.16);

\path[draw=drawColor,line width= 0.4pt,line join=round,line cap=round,fill=fillColor] (376.30,610.50) circle (  1.16);

\path[draw=drawColor,line width= 0.4pt,line join=round,line cap=round,fill=fillColor] (377.02,610.04) circle (  1.16);

\path[draw=drawColor,line width= 0.4pt,line join=round,line cap=round,fill=fillColor] (377.74,609.86) circle (  1.16);

\path[draw=drawColor,line width= 0.4pt,line join=round,line cap=round,fill=fillColor] (378.44,609.75) circle (  1.16);

\path[draw=drawColor,line width= 0.4pt,line join=round,line cap=round,fill=fillColor] (379.13,608.69) circle (  1.16);

\path[draw=drawColor,line width= 0.4pt,line join=round,line cap=round,fill=fillColor] (379.80,608.39) circle (  1.16);

\path[draw=drawColor,line width= 0.4pt,line join=round,line cap=round,fill=fillColor] (380.47,607.46) circle (  1.16);

\path[draw=drawColor,line width= 0.4pt,line join=round,line cap=round,fill=fillColor] (381.12,607.34) circle (  1.16);

\path[draw=drawColor,line width= 0.4pt,line join=round,line cap=round,fill=fillColor] (381.76,607.28) circle (  1.16);

\path[draw=drawColor,line width= 0.4pt,line join=round,line cap=round,fill=fillColor] (382.39,606.06) circle (  1.16);

\path[draw=drawColor,line width= 0.4pt,line join=round,line cap=round,fill=fillColor] (383.01,606.04) circle (  1.16);

\path[draw=drawColor,line width= 0.4pt,line join=round,line cap=round,fill=fillColor] (383.62,605.31) circle (  1.16);

\path[draw=drawColor,line width= 0.4pt,line join=round,line cap=round,fill=fillColor] (384.22,604.57) circle (  1.16);

\path[draw=drawColor,line width= 0.4pt,line join=round,line cap=round,fill=fillColor] (384.81,603.69) circle (  1.16);

\path[draw=drawColor,line width= 0.4pt,line join=round,line cap=round,fill=fillColor] (385.39,603.48) circle (  1.16);

\path[draw=drawColor,line width= 0.4pt,line join=round,line cap=round,fill=fillColor] (385.97,601.70) circle (  1.16);

\path[draw=drawColor,line width= 0.4pt,line join=round,line cap=round,fill=fillColor] (386.54,601.52) circle (  1.16);

\path[draw=drawColor,line width= 0.4pt,line join=round,line cap=round,fill=fillColor] (387.09,601.36) circle (  1.16);

\path[draw=drawColor,line width= 0.4pt,line join=round,line cap=round,fill=fillColor] (387.64,601.17) circle (  1.16);

\path[draw=drawColor,line width= 0.4pt,line join=round,line cap=round,fill=fillColor] (388.19,600.21) circle (  1.16);

\path[draw=drawColor,line width= 0.4pt,line join=round,line cap=round,fill=fillColor] (388.73,599.99) circle (  1.16);

\path[draw=drawColor,line width= 0.4pt,line join=round,line cap=round,fill=fillColor] (389.26,599.90) circle (  1.16);

\path[draw=drawColor,line width= 0.4pt,line join=round,line cap=round,fill=fillColor] (389.78,599.53) circle (  1.16);

\path[draw=drawColor,line width= 0.4pt,line join=round,line cap=round,fill=fillColor] (390.30,599.49) circle (  1.16);

\path[draw=drawColor,line width= 0.4pt,line join=round,line cap=round,fill=fillColor] (390.81,598.61) circle (  1.16);

\path[draw=drawColor,line width= 0.4pt,line join=round,line cap=round,fill=fillColor] (391.31,598.27) circle (  1.16);

\path[draw=drawColor,line width= 0.4pt,line join=round,line cap=round,fill=fillColor] (391.81,598.11) circle (  1.16);

\path[draw=drawColor,line width= 0.4pt,line join=round,line cap=round,fill=fillColor] (392.30,597.91) circle (  1.16);

\path[draw=drawColor,line width= 0.4pt,line join=round,line cap=round,fill=fillColor] (392.79,597.02) circle (  1.16);

\path[draw=drawColor,line width= 0.4pt,line join=round,line cap=round,fill=fillColor] (393.27,594.85) circle (  1.16);

\path[draw=drawColor,line width= 0.4pt,line join=round,line cap=round,fill=fillColor] (393.75,594.03) circle (  1.16);

\path[draw=drawColor,line width= 0.4pt,line join=round,line cap=round,fill=fillColor] (394.22,592.87) circle (  1.16);

\path[draw=drawColor,line width= 0.4pt,line join=round,line cap=round,fill=fillColor] (394.69,592.41) circle (  1.16);

\path[draw=drawColor,line width= 0.4pt,line join=round,line cap=round,fill=fillColor] (395.15,592.20) circle (  1.16);

\path[draw=drawColor,line width= 0.4pt,line join=round,line cap=round,fill=fillColor] (395.61,591.49) circle (  1.16);

\path[draw=drawColor,line width= 0.4pt,line join=round,line cap=round,fill=fillColor] (396.06,591.27) circle (  1.16);

\path[draw=drawColor,line width= 0.4pt,line join=round,line cap=round,fill=fillColor] (396.51,591.24) circle (  1.16);

\path[draw=drawColor,line width= 0.4pt,line join=round,line cap=round,fill=fillColor] (396.95,591.10) circle (  1.16);

\path[draw=drawColor,line width= 0.4pt,line join=round,line cap=round,fill=fillColor] (397.39,590.32) circle (  1.16);

\path[draw=drawColor,line width= 0.4pt,line join=round,line cap=round,fill=fillColor] (397.83,590.03) circle (  1.16);

\path[draw=drawColor,line width= 0.4pt,line join=round,line cap=round,fill=fillColor] (398.26,589.40) circle (  1.16);

\path[draw=drawColor,line width= 0.4pt,line join=round,line cap=round,fill=fillColor] (398.68,587.16) circle (  1.16);

\path[draw=drawColor,line width= 0.4pt,line join=round,line cap=round,fill=fillColor] (399.11,586.81) circle (  1.16);

\path[draw=drawColor,line width= 0.4pt,line join=round,line cap=round,fill=fillColor] (399.53,586.74) circle (  1.16);

\path[draw=drawColor,line width= 0.4pt,line join=round,line cap=round,fill=fillColor] (399.94,585.52) circle (  1.16);

\path[draw=drawColor,line width= 0.4pt,line join=round,line cap=round,fill=fillColor] (400.36,585.48) circle (  1.16);

\path[draw=drawColor,line width= 0.4pt,line join=round,line cap=round,fill=fillColor] (400.76,585.04) circle (  1.16);

\path[draw=drawColor,line width= 0.4pt,line join=round,line cap=round,fill=fillColor] (401.17,583.93) circle (  1.16);

\path[draw=drawColor,line width= 0.4pt,line join=round,line cap=round,fill=fillColor] (401.57,583.16) circle (  1.16);

\path[draw=drawColor,line width= 0.4pt,line join=round,line cap=round,fill=fillColor] (401.97,583.06) circle (  1.16);

\path[draw=drawColor,line width= 0.4pt,line join=round,line cap=round,fill=fillColor] (402.36,582.64) circle (  1.16);

\path[draw=drawColor,line width= 0.4pt,line join=round,line cap=round,fill=fillColor] (402.76,582.30) circle (  1.16);

\path[draw=drawColor,line width= 0.4pt,line join=round,line cap=round,fill=fillColor] (403.15,582.21) circle (  1.16);

\path[draw=drawColor,line width= 0.4pt,line join=round,line cap=round,fill=fillColor] (403.53,581.87) circle (  1.16);

\path[draw=drawColor,line width= 0.4pt,line join=round,line cap=round,fill=fillColor] (403.91,581.40) circle (  1.16);

\path[draw=drawColor,line width= 0.4pt,line join=round,line cap=round,fill=fillColor] (404.29,580.98) circle (  1.16);

\path[draw=drawColor,line width= 0.4pt,line join=round,line cap=round,fill=fillColor] (404.67,580.87) circle (  1.16);

\path[draw=drawColor,line width= 0.4pt,line join=round,line cap=round,fill=fillColor] (405.05,580.03) circle (  1.16);

\path[draw=drawColor,line width= 0.4pt,line join=round,line cap=round,fill=fillColor] (405.42,579.79) circle (  1.16);

\path[draw=drawColor,line width= 0.4pt,line join=round,line cap=round,fill=fillColor] (405.78,579.77) circle (  1.16);

\path[draw=drawColor,line width= 0.4pt,line join=round,line cap=round,fill=fillColor] (406.15,579.52) circle (  1.16);

\path[draw=drawColor,line width= 0.4pt,line join=round,line cap=round,fill=fillColor] (406.51,579.23) circle (  1.16);

\path[draw=drawColor,line width= 0.4pt,line join=round,line cap=round,fill=fillColor] (406.87,579.20) circle (  1.16);

\path[draw=drawColor,line width= 0.4pt,line join=round,line cap=round,fill=fillColor] (407.23,578.73) circle (  1.16);

\path[draw=drawColor,line width= 0.4pt,line join=round,line cap=round,fill=fillColor] (407.59,578.51) circle (  1.16);

\path[draw=drawColor,line width= 0.4pt,line join=round,line cap=round,fill=fillColor] (407.94,578.43) circle (  1.16);

\path[draw=drawColor,line width= 0.4pt,line join=round,line cap=round,fill=fillColor] (408.29,578.27) circle (  1.16);

\path[draw=drawColor,line width= 0.4pt,line join=round,line cap=round,fill=fillColor] (408.64,578.04) circle (  1.16);

\path[draw=drawColor,line width= 0.4pt,line join=round,line cap=round,fill=fillColor] (408.99,577.95) circle (  1.16);

\path[draw=drawColor,line width= 0.4pt,line join=round,line cap=round,fill=fillColor] (409.33,577.61) circle (  1.16);

\path[draw=drawColor,line width= 0.4pt,line join=round,line cap=round,fill=fillColor] (409.67,577.52) circle (  1.16);

\path[draw=drawColor,line width= 0.4pt,line join=round,line cap=round,fill=fillColor] (410.01,577.14) circle (  1.16);

\path[draw=drawColor,line width= 0.4pt,line join=round,line cap=round,fill=fillColor] (410.35,577.10) circle (  1.16);

\path[draw=drawColor,line width= 0.4pt,line join=round,line cap=round,fill=fillColor] (410.68,577.08) circle (  1.16);

\path[draw=drawColor,line width= 0.4pt,line join=round,line cap=round,fill=fillColor] (411.02,576.98) circle (  1.16);

\path[draw=drawColor,line width= 0.4pt,line join=round,line cap=round,fill=fillColor] (411.35,576.95) circle (  1.16);

\path[draw=drawColor,line width= 0.4pt,line join=round,line cap=round,fill=fillColor] (411.67,575.89) circle (  1.16);

\path[draw=drawColor,line width= 0.4pt,line join=round,line cap=round,fill=fillColor] (412.00,575.47) circle (  1.16);

\path[draw=drawColor,line width= 0.4pt,line join=round,line cap=round,fill=fillColor] (412.33,574.88) circle (  1.16);

\path[draw=drawColor,line width= 0.4pt,line join=round,line cap=round,fill=fillColor] (412.65,574.63) circle (  1.16);

\path[draw=drawColor,line width= 0.4pt,line join=round,line cap=round,fill=fillColor] (412.97,574.61) circle (  1.16);

\path[draw=drawColor,line width= 0.4pt,line join=round,line cap=round,fill=fillColor] (413.29,574.02) circle (  1.16);

\path[draw=drawColor,line width= 0.4pt,line join=round,line cap=round,fill=fillColor] (413.60,573.97) circle (  1.16);

\path[draw=drawColor,line width= 0.4pt,line join=round,line cap=round,fill=fillColor] (413.92,573.82) circle (  1.16);

\path[draw=drawColor,line width= 0.4pt,line join=round,line cap=round,fill=fillColor] (414.23,573.49) circle (  1.16);

\path[draw=drawColor,line width= 0.4pt,line join=round,line cap=round,fill=fillColor] (414.54,573.23) circle (  1.16);

\path[draw=drawColor,line width= 0.4pt,line join=round,line cap=round,fill=fillColor] (414.85,573.21) circle (  1.16);

\path[draw=drawColor,line width= 0.4pt,line join=round,line cap=round,fill=fillColor] (415.16,572.51) circle (  1.16);

\path[draw=drawColor,line width= 0.4pt,line join=round,line cap=round,fill=fillColor] (415.47,572.38) circle (  1.16);

\path[draw=drawColor,line width= 0.4pt,line join=round,line cap=round,fill=fillColor] (415.77,572.27) circle (  1.16);

\path[draw=drawColor,line width= 0.4pt,line join=round,line cap=round,fill=fillColor] (416.08,572.04) circle (  1.16);

\path[draw=drawColor,line width= 0.4pt,line join=round,line cap=round,fill=fillColor] (416.38,571.93) circle (  1.16);

\path[draw=drawColor,line width= 0.4pt,line join=round,line cap=round,fill=fillColor] (416.68,571.87) circle (  1.16);

\path[draw=drawColor,line width= 0.4pt,line join=round,line cap=round,fill=fillColor] (416.97,571.38) circle (  1.16);

\path[draw=drawColor,line width= 0.4pt,line join=round,line cap=round,fill=fillColor] (417.27,571.23) circle (  1.16);

\path[draw=drawColor,line width= 0.4pt,line join=round,line cap=round,fill=fillColor] (417.57,570.87) circle (  1.16);

\path[draw=drawColor,line width= 0.4pt,line join=round,line cap=round,fill=fillColor] (417.86,570.78) circle (  1.16);

\path[draw=drawColor,line width= 0.4pt,line join=round,line cap=round,fill=fillColor] (418.15,570.75) circle (  1.16);

\path[draw=drawColor,line width= 0.4pt,line join=round,line cap=round,fill=fillColor] (418.44,570.57) circle (  1.16);

\path[draw=drawColor,line width= 0.4pt,line join=round,line cap=round,fill=fillColor] (418.73,570.50) circle (  1.16);

\path[draw=drawColor,line width= 0.4pt,line join=round,line cap=round,fill=fillColor] (419.02,570.31) circle (  1.16);

\path[draw=drawColor,line width= 0.4pt,line join=round,line cap=round,fill=fillColor] (419.30,570.23) circle (  1.16);

\path[draw=drawColor,line width= 0.4pt,line join=round,line cap=round,fill=fillColor] (419.59,569.98) circle (  1.16);

\path[draw=drawColor,line width= 0.4pt,line join=round,line cap=round,fill=fillColor] (419.87,569.96) circle (  1.16);

\path[draw=drawColor,line width= 0.4pt,line join=round,line cap=round,fill=fillColor] (420.15,569.94) circle (  1.16);

\path[draw=drawColor,line width= 0.4pt,line join=round,line cap=round,fill=fillColor] (420.43,569.45) circle (  1.16);

\path[draw=drawColor,line width= 0.4pt,line join=round,line cap=round,fill=fillColor] (420.71,569.04) circle (  1.16);

\path[draw=drawColor,line width= 0.4pt,line join=round,line cap=round,fill=fillColor] (420.99,569.03) circle (  1.16);

\path[draw=drawColor,line width= 0.4pt,line join=round,line cap=round,fill=fillColor] (421.26,568.95) circle (  1.16);

\path[draw=drawColor,line width= 0.4pt,line join=round,line cap=round,fill=fillColor] (421.54,568.87) circle (  1.16);

\path[draw=drawColor,line width= 0.4pt,line join=round,line cap=round,fill=fillColor] (421.81,568.75) circle (  1.16);

\path[draw=drawColor,line width= 0.4pt,line join=round,line cap=round,fill=fillColor] (422.09,568.74) circle (  1.16);

\path[draw=drawColor,line width= 0.4pt,line join=round,line cap=round,fill=fillColor] (422.36,568.73) circle (  1.16);

\path[draw=drawColor,line width= 0.4pt,line join=round,line cap=round,fill=fillColor] (422.63,568.72) circle (  1.16);

\path[draw=drawColor,line width= 0.4pt,line join=round,line cap=round,fill=fillColor] (422.90,568.70) circle (  1.16);

\path[draw=drawColor,line width= 0.4pt,line join=round,line cap=round,fill=fillColor] (423.16,568.59) circle (  1.16);

\path[draw=drawColor,line width= 0.4pt,line join=round,line cap=round,fill=fillColor] (423.43,568.28) circle (  1.16);

\path[draw=drawColor,line width= 0.4pt,line join=round,line cap=round,fill=fillColor] (423.69,567.46) circle (  1.16);

\path[draw=drawColor,line width= 0.4pt,line join=round,line cap=round,fill=fillColor] (423.96,567.40) circle (  1.16);

\path[draw=drawColor,line width= 0.4pt,line join=round,line cap=round,fill=fillColor] (424.22,567.38) circle (  1.16);

\path[draw=drawColor,line width= 0.4pt,line join=round,line cap=round,fill=fillColor] (424.48,567.25) circle (  1.16);

\path[draw=drawColor,line width= 0.4pt,line join=round,line cap=round,fill=fillColor] (424.74,567.22) circle (  1.16);

\path[draw=drawColor,line width= 0.4pt,line join=round,line cap=round,fill=fillColor] (425.00,567.16) circle (  1.16);

\path[draw=drawColor,line width= 0.4pt,line join=round,line cap=round,fill=fillColor] (425.26,567.03) circle (  1.16);

\path[draw=drawColor,line width= 0.4pt,line join=round,line cap=round,fill=fillColor] (425.52,566.85) circle (  1.16);

\path[draw=drawColor,line width= 0.4pt,line join=round,line cap=round,fill=fillColor] (425.77,566.80) circle (  1.16);

\path[draw=drawColor,line width= 0.4pt,line join=round,line cap=round,fill=fillColor] (426.03,566.74) circle (  1.16);

\path[draw=drawColor,line width= 0.4pt,line join=round,line cap=round,fill=fillColor] (426.28,566.65) circle (  1.16);

\path[draw=drawColor,line width= 0.4pt,line join=round,line cap=round,fill=fillColor] (426.53,566.45) circle (  1.16);

\path[draw=drawColor,line width= 0.4pt,line join=round,line cap=round,fill=fillColor] (426.78,566.22) circle (  1.16);

\path[draw=drawColor,line width= 0.4pt,line join=round,line cap=round,fill=fillColor] (427.03,565.96) circle (  1.16);

\path[draw=drawColor,line width= 0.4pt,line join=round,line cap=round,fill=fillColor] (427.28,565.88) circle (  1.16);

\path[draw=drawColor,line width= 0.4pt,line join=round,line cap=round,fill=fillColor] (427.53,565.88) circle (  1.16);

\path[draw=drawColor,line width= 0.4pt,line join=round,line cap=round,fill=fillColor] (427.78,565.80) circle (  1.16);

\path[draw=drawColor,line width= 0.4pt,line join=round,line cap=round,fill=fillColor] (428.03,565.76) circle (  1.16);

\path[draw=drawColor,line width= 0.4pt,line join=round,line cap=round,fill=fillColor] (428.27,565.67) circle (  1.16);

\path[draw=drawColor,line width= 0.4pt,line join=round,line cap=round,fill=fillColor] (428.52,565.53) circle (  1.16);

\path[draw=drawColor,line width= 0.4pt,line join=round,line cap=round,fill=fillColor] (428.76,565.46) circle (  1.16);

\path[draw=drawColor,line width= 0.4pt,line join=round,line cap=round,fill=fillColor] (429.00,565.43) circle (  1.16);

\path[draw=drawColor,line width= 0.4pt,line join=round,line cap=round,fill=fillColor] (429.24,565.38) circle (  1.16);

\path[draw=drawColor,line width= 0.4pt,line join=round,line cap=round,fill=fillColor] (429.48,564.86) circle (  1.16);

\path[draw=drawColor,line width= 0.4pt,line join=round,line cap=round,fill=fillColor] (429.72,564.80) circle (  1.16);

\path[draw=drawColor,line width= 0.4pt,line join=round,line cap=round,fill=fillColor] (429.96,564.80) circle (  1.16);

\path[draw=drawColor,line width= 0.4pt,line join=round,line cap=round,fill=fillColor] (430.20,564.57) circle (  1.16);

\path[draw=drawColor,line width= 0.4pt,line join=round,line cap=round,fill=fillColor] (430.44,564.56) circle (  1.16);

\path[draw=drawColor,line width= 0.4pt,line join=round,line cap=round,fill=fillColor] (430.67,564.21) circle (  1.16);

\path[draw=drawColor,line width= 0.4pt,line join=round,line cap=round,fill=fillColor] (430.91,564.00) circle (  1.16);

\path[draw=drawColor,line width= 0.4pt,line join=round,line cap=round,fill=fillColor] (431.14,563.97) circle (  1.16);

\path[draw=drawColor,line width= 0.4pt,line join=round,line cap=round,fill=fillColor] (431.38,563.95) circle (  1.16);

\path[draw=drawColor,line width= 0.4pt,line join=round,line cap=round,fill=fillColor] (431.61,563.91) circle (  1.16);

\path[draw=drawColor,line width= 0.4pt,line join=round,line cap=round,fill=fillColor] (431.84,563.79) circle (  1.16);

\path[draw=drawColor,line width= 0.4pt,line join=round,line cap=round,fill=fillColor] (432.07,563.78) circle (  1.16);

\path[draw=drawColor,line width= 0.4pt,line join=round,line cap=round,fill=fillColor] (432.30,563.73) circle (  1.16);

\path[draw=drawColor,line width= 0.4pt,line join=round,line cap=round,fill=fillColor] (432.53,563.63) circle (  1.16);

\path[draw=drawColor,line width= 0.4pt,line join=round,line cap=round,fill=fillColor] (432.76,563.55) circle (  1.16);

\path[draw=drawColor,line width= 0.4pt,line join=round,line cap=round,fill=fillColor] (432.99,563.43) circle (  1.16);

\path[draw=drawColor,line width= 0.4pt,line join=round,line cap=round,fill=fillColor] (433.21,563.43) circle (  1.16);

\path[draw=drawColor,line width= 0.4pt,line join=round,line cap=round,fill=fillColor] (433.44,563.19) circle (  1.16);

\path[draw=drawColor,line width= 0.4pt,line join=round,line cap=round,fill=fillColor] (433.67,563.03) circle (  1.16);

\path[draw=drawColor,line width= 0.4pt,line join=round,line cap=round,fill=fillColor] (433.89,562.91) circle (  1.16);

\path[draw=drawColor,line width= 0.4pt,line join=round,line cap=round,fill=fillColor] (434.11,562.86) circle (  1.16);

\path[draw=drawColor,line width= 0.4pt,line join=round,line cap=round,fill=fillColor] (434.34,562.85) circle (  1.16);

\path[draw=drawColor,line width= 0.4pt,line join=round,line cap=round,fill=fillColor] (434.56,562.75) circle (  1.16);

\path[draw=drawColor,line width= 0.4pt,line join=round,line cap=round,fill=fillColor] (434.78,562.65) circle (  1.16);

\path[draw=drawColor,line width= 0.4pt,line join=round,line cap=round,fill=fillColor] (435.00,562.61) circle (  1.16);

\path[draw=drawColor,line width= 0.4pt,line join=round,line cap=round,fill=fillColor] (435.22,562.57) circle (  1.16);

\path[draw=drawColor,line width= 0.4pt,line join=round,line cap=round,fill=fillColor] (435.44,562.55) circle (  1.16);

\path[draw=drawColor,line width= 0.4pt,line join=round,line cap=round,fill=fillColor] (435.66,562.51) circle (  1.16);

\path[draw=drawColor,line width= 0.4pt,line join=round,line cap=round,fill=fillColor] (435.88,562.51) circle (  1.16);

\path[draw=drawColor,line width= 0.4pt,line join=round,line cap=round,fill=fillColor] (436.09,562.43) circle (  1.16);

\path[draw=drawColor,line width= 0.4pt,line join=round,line cap=round,fill=fillColor] (436.31,562.26) circle (  1.16);

\path[draw=drawColor,line width= 0.4pt,line join=round,line cap=round,fill=fillColor] (436.52,561.93) circle (  1.16);

\path[draw=drawColor,line width= 0.4pt,line join=round,line cap=round,fill=fillColor] (436.74,561.90) circle (  1.16);

\path[draw=drawColor,line width= 0.4pt,line join=round,line cap=round,fill=fillColor] (436.95,561.89) circle (  1.16);

\path[draw=drawColor,line width= 0.4pt,line join=round,line cap=round,fill=fillColor] (437.17,561.67) circle (  1.16);

\path[draw=drawColor,line width= 0.4pt,line join=round,line cap=round,fill=fillColor] (437.38,561.64) circle (  1.16);

\path[draw=drawColor,line width= 0.4pt,line join=round,line cap=round,fill=fillColor] (437.59,561.54) circle (  1.16);

\path[draw=drawColor,line width= 0.4pt,line join=round,line cap=round,fill=fillColor] (437.80,561.41) circle (  1.16);

\path[draw=drawColor,line width= 0.4pt,line join=round,line cap=round,fill=fillColor] (438.01,561.36) circle (  1.16);

\path[draw=drawColor,line width= 0.4pt,line join=round,line cap=round,fill=fillColor] (438.22,561.24) circle (  1.16);

\path[draw=drawColor,line width= 0.4pt,line join=round,line cap=round,fill=fillColor] (438.43,561.20) circle (  1.16);

\path[draw=drawColor,line width= 0.4pt,line join=round,line cap=round,fill=fillColor] (438.64,561.00) circle (  1.16);

\path[draw=drawColor,line width= 0.4pt,line join=round,line cap=round,fill=fillColor] (438.85,560.74) circle (  1.16);

\path[draw=drawColor,line width= 0.4pt,line join=round,line cap=round,fill=fillColor] (439.06,560.73) circle (  1.16);

\path[draw=drawColor,line width= 0.4pt,line join=round,line cap=round,fill=fillColor] (439.26,560.54) circle (  1.16);

\path[draw=drawColor,line width= 0.4pt,line join=round,line cap=round,fill=fillColor] (439.47,560.50) circle (  1.16);

\path[draw=drawColor,line width= 0.4pt,line join=round,line cap=round,fill=fillColor] (439.67,560.41) circle (  1.16);

\path[draw=drawColor,line width= 0.4pt,line join=round,line cap=round,fill=fillColor] (439.88,560.41) circle (  1.16);

\path[draw=drawColor,line width= 0.4pt,line join=round,line cap=round,fill=fillColor] (440.08,560.37) circle (  1.16);

\path[draw=drawColor,line width= 0.4pt,line join=round,line cap=round,fill=fillColor] (440.29,560.15) circle (  1.16);

\path[draw=drawColor,line width= 0.4pt,line join=round,line cap=round,fill=fillColor] (440.49,560.12) circle (  1.16);

\path[draw=drawColor,line width= 0.4pt,line join=round,line cap=round,fill=fillColor] (440.69,560.04) circle (  1.16);

\path[draw=drawColor,line width= 0.4pt,line join=round,line cap=round,fill=fillColor] (440.89,559.81) circle (  1.16);

\path[draw=drawColor,line width= 0.4pt,line join=round,line cap=round,fill=fillColor] (441.09,559.81) circle (  1.16);

\path[draw=drawColor,line width= 0.4pt,line join=round,line cap=round,fill=fillColor] (441.29,559.80) circle (  1.16);

\path[draw=drawColor,line width= 0.4pt,line join=round,line cap=round,fill=fillColor] (441.49,559.78) circle (  1.16);

\path[draw=drawColor,line width= 0.4pt,line join=round,line cap=round,fill=fillColor] (441.69,559.69) circle (  1.16);

\path[draw=drawColor,line width= 0.4pt,line join=round,line cap=round,fill=fillColor] (441.89,559.54) circle (  1.16);

\path[draw=drawColor,line width= 0.4pt,line join=round,line cap=round,fill=fillColor] (442.09,559.46) circle (  1.16);

\path[draw=drawColor,line width= 0.4pt,line join=round,line cap=round,fill=fillColor] (442.29,559.30) circle (  1.16);

\path[draw=drawColor,line width= 0.4pt,line join=round,line cap=round,fill=fillColor] (442.49,559.20) circle (  1.16);

\path[draw=drawColor,line width= 0.4pt,line join=round,line cap=round,fill=fillColor] (442.68,559.19) circle (  1.16);

\path[draw=drawColor,line width= 0.4pt,line join=round,line cap=round,fill=fillColor] (442.88,558.98) circle (  1.16);

\path[draw=drawColor,line width= 0.4pt,line join=round,line cap=round,fill=fillColor] (443.07,558.96) circle (  1.16);

\path[draw=drawColor,line width= 0.4pt,line join=round,line cap=round,fill=fillColor] (443.27,558.66) circle (  1.16);

\path[draw=drawColor,line width= 0.4pt,line join=round,line cap=round,fill=fillColor] (443.46,558.65) circle (  1.16);

\path[draw=drawColor,line width= 0.4pt,line join=round,line cap=round,fill=fillColor] (443.66,558.61) circle (  1.16);

\path[draw=drawColor,line width= 0.4pt,line join=round,line cap=round,fill=fillColor] (443.85,558.56) circle (  1.16);

\path[draw=drawColor,line width= 0.4pt,line join=round,line cap=round,fill=fillColor] (444.04,558.53) circle (  1.16);

\path[draw=drawColor,line width= 0.4pt,line join=round,line cap=round,fill=fillColor] (444.23,558.40) circle (  1.16);

\path[draw=drawColor,line width= 0.4pt,line join=round,line cap=round,fill=fillColor] (444.43,558.21) circle (  1.16);

\path[draw=drawColor,line width= 0.4pt,line join=round,line cap=round,fill=fillColor] (444.62,557.93) circle (  1.16);

\path[draw=drawColor,line width= 0.4pt,line join=round,line cap=round,fill=fillColor] (444.81,557.85) circle (  1.16);

\path[draw=drawColor,line width= 0.4pt,line join=round,line cap=round,fill=fillColor] (445.00,557.67) circle (  1.16);

\path[draw=drawColor,line width= 0.4pt,line join=round,line cap=round,fill=fillColor] (445.19,557.58) circle (  1.16);

\path[draw=drawColor,line width= 0.4pt,line join=round,line cap=round,fill=fillColor] (445.38,557.37) circle (  1.16);

\path[draw=drawColor,line width= 0.4pt,line join=round,line cap=round,fill=fillColor] (445.56,557.30) circle (  1.16);

\path[draw=drawColor,line width= 0.4pt,line join=round,line cap=round,fill=fillColor] (445.75,557.09) circle (  1.16);

\path[draw=drawColor,line width= 0.4pt,line join=round,line cap=round,fill=fillColor] (445.94,557.08) circle (  1.16);

\path[draw=drawColor,line width= 0.4pt,line join=round,line cap=round,fill=fillColor] (446.13,557.01) circle (  1.16);

\path[draw=drawColor,line width= 0.4pt,line join=round,line cap=round,fill=fillColor] (446.31,556.92) circle (  1.16);

\path[draw=drawColor,line width= 0.4pt,line join=round,line cap=round,fill=fillColor] (446.50,556.89) circle (  1.16);

\path[draw=drawColor,line width= 0.4pt,line join=round,line cap=round,fill=fillColor] (446.68,556.78) circle (  1.16);

\path[draw=drawColor,line width= 0.4pt,line join=round,line cap=round,fill=fillColor] (446.87,556.68) circle (  1.16);

\path[draw=drawColor,line width= 0.4pt,line join=round,line cap=round,fill=fillColor] (447.05,556.61) circle (  1.16);

\path[draw=drawColor,line width= 0.4pt,line join=round,line cap=round,fill=fillColor] (447.24,556.58) circle (  1.16);

\path[draw=drawColor,line width= 0.4pt,line join=round,line cap=round,fill=fillColor] (447.42,556.39) circle (  1.16);

\path[draw=drawColor,line width= 0.4pt,line join=round,line cap=round,fill=fillColor] (447.60,556.21) circle (  1.16);

\path[draw=drawColor,line width= 0.4pt,line join=round,line cap=round,fill=fillColor] (447.79,556.19) circle (  1.16);

\path[draw=drawColor,line width= 0.4pt,line join=round,line cap=round,fill=fillColor] (447.97,556.02) circle (  1.16);

\path[draw=drawColor,line width= 0.4pt,line join=round,line cap=round,fill=fillColor] (448.15,555.96) circle (  1.16);

\path[draw=drawColor,line width= 0.4pt,line join=round,line cap=round,fill=fillColor] (448.33,555.86) circle (  1.16);

\path[draw=drawColor,line width= 0.4pt,line join=round,line cap=round,fill=fillColor] (448.51,555.76) circle (  1.16);

\path[draw=drawColor,line width= 0.4pt,line join=round,line cap=round,fill=fillColor] (448.69,555.56) circle (  1.16);

\path[draw=drawColor,line width= 0.4pt,line join=round,line cap=round,fill=fillColor] (448.87,555.45) circle (  1.16);

\path[draw=drawColor,line width= 0.4pt,line join=round,line cap=round,fill=fillColor] (449.05,555.39) circle (  1.16);

\path[draw=drawColor,line width= 0.4pt,line join=round,line cap=round,fill=fillColor] (449.23,555.17) circle (  1.16);

\path[draw=drawColor,line width= 0.4pt,line join=round,line cap=round,fill=fillColor] (449.41,555.08) circle (  1.16);

\path[draw=drawColor,line width= 0.4pt,line join=round,line cap=round,fill=fillColor] (449.59,555.07) circle (  1.16);

\path[draw=drawColor,line width= 0.4pt,line join=round,line cap=round,fill=fillColor] (449.77,554.89) circle (  1.16);

\path[draw=drawColor,line width= 0.4pt,line join=round,line cap=round,fill=fillColor] (449.94,554.86) circle (  1.16);

\path[draw=drawColor,line width= 0.4pt,line join=round,line cap=round,fill=fillColor] (450.12,554.83) circle (  1.16);

\path[draw=drawColor,line width= 0.4pt,line join=round,line cap=round,fill=fillColor] (450.30,554.77) circle (  1.16);

\path[draw=drawColor,line width= 0.4pt,line join=round,line cap=round,fill=fillColor] (450.47,554.67) circle (  1.16);

\path[draw=drawColor,line width= 0.4pt,line join=round,line cap=round,fill=fillColor] (450.65,554.64) circle (  1.16);

\path[draw=drawColor,line width= 0.4pt,line join=round,line cap=round,fill=fillColor] (450.82,554.60) circle (  1.16);

\path[draw=drawColor,line width= 0.4pt,line join=round,line cap=round,fill=fillColor] (451.00,554.60) circle (  1.16);

\path[draw=drawColor,line width= 0.4pt,line join=round,line cap=round,fill=fillColor] (451.17,554.56) circle (  1.16);

\path[draw=drawColor,line width= 0.4pt,line join=round,line cap=round,fill=fillColor] (451.35,554.48) circle (  1.16);

\path[draw=drawColor,line width= 0.4pt,line join=round,line cap=round,fill=fillColor] (451.52,554.40) circle (  1.16);

\path[draw=drawColor,line width= 0.4pt,line join=round,line cap=round,fill=fillColor] (451.69,554.33) circle (  1.16);

\path[draw=drawColor,line width= 0.4pt,line join=round,line cap=round,fill=fillColor] (451.86,554.32) circle (  1.16);

\path[draw=drawColor,line width= 0.4pt,line join=round,line cap=round,fill=fillColor] (452.04,554.31) circle (  1.16);

\path[draw=drawColor,line width= 0.4pt,line join=round,line cap=round,fill=fillColor] (452.21,554.09) circle (  1.16);

\path[draw=drawColor,line width= 0.4pt,line join=round,line cap=round,fill=fillColor] (452.38,554.05) circle (  1.16);

\path[draw=drawColor,line width= 0.4pt,line join=round,line cap=round,fill=fillColor] (452.55,553.98) circle (  1.16);

\path[draw=drawColor,line width= 0.4pt,line join=round,line cap=round,fill=fillColor] (452.72,553.98) circle (  1.16);

\path[draw=drawColor,line width= 0.4pt,line join=round,line cap=round,fill=fillColor] (452.89,553.88) circle (  1.16);

\path[draw=drawColor,line width= 0.4pt,line join=round,line cap=round,fill=fillColor] (453.06,553.81) circle (  1.16);

\path[draw=drawColor,line width= 0.4pt,line join=round,line cap=round,fill=fillColor] (453.23,553.78) circle (  1.16);

\path[draw=drawColor,line width= 0.4pt,line join=round,line cap=round,fill=fillColor] (453.40,553.73) circle (  1.16);

\path[draw=drawColor,line width= 0.4pt,line join=round,line cap=round,fill=fillColor] (453.57,553.70) circle (  1.16);

\path[draw=drawColor,line width= 0.4pt,line join=round,line cap=round,fill=fillColor] (453.74,553.59) circle (  1.16);

\path[draw=drawColor,line width= 0.4pt,line join=round,line cap=round,fill=fillColor] (453.90,553.53) circle (  1.16);

\path[draw=drawColor,line width= 0.4pt,line join=round,line cap=round,fill=fillColor] (454.07,553.53) circle (  1.16);

\path[draw=drawColor,line width= 0.4pt,line join=round,line cap=round,fill=fillColor] (454.24,553.47) circle (  1.16);

\path[draw=drawColor,line width= 0.4pt,line join=round,line cap=round,fill=fillColor] (454.40,553.32) circle (  1.16);

\path[draw=drawColor,line width= 0.4pt,line join=round,line cap=round,fill=fillColor] (454.57,553.31) circle (  1.16);

\path[draw=drawColor,line width= 0.4pt,line join=round,line cap=round,fill=fillColor] (454.74,553.31) circle (  1.16);

\path[draw=drawColor,line width= 0.4pt,line join=round,line cap=round,fill=fillColor] (454.90,553.21) circle (  1.16);

\path[draw=drawColor,line width= 0.4pt,line join=round,line cap=round,fill=fillColor] (455.07,553.20) circle (  1.16);

\path[draw=drawColor,line width= 0.4pt,line join=round,line cap=round,fill=fillColor] (455.23,553.11) circle (  1.16);

\path[draw=drawColor,line width= 0.4pt,line join=round,line cap=round,fill=fillColor] (455.40,552.66) circle (  1.16);

\path[draw=drawColor,line width= 0.4pt,line join=round,line cap=round,fill=fillColor] (455.56,552.36) circle (  1.16);

\path[draw=drawColor,line width= 0.4pt,line join=round,line cap=round,fill=fillColor] (455.73,552.11) circle (  1.16);

\path[draw=drawColor,line width= 0.4pt,line join=round,line cap=round,fill=fillColor] (455.89,551.97) circle (  1.16);

\path[draw=drawColor,line width= 0.4pt,line join=round,line cap=round,fill=fillColor] (456.05,551.96) circle (  1.16);

\path[draw=drawColor,line width= 0.4pt,line join=round,line cap=round,fill=fillColor] (456.21,551.91) circle (  1.16);

\path[draw=drawColor,line width= 0.4pt,line join=round,line cap=round,fill=fillColor] (456.38,551.91) circle (  1.16);

\path[draw=drawColor,line width= 0.4pt,line join=round,line cap=round,fill=fillColor] (456.54,551.83) circle (  1.16);

\path[draw=drawColor,line width= 0.4pt,line join=round,line cap=round,fill=fillColor] (456.70,551.75) circle (  1.16);

\path[draw=drawColor,line width= 0.4pt,line join=round,line cap=round,fill=fillColor] (456.86,551.52) circle (  1.16);

\path[draw=drawColor,line width= 0.4pt,line join=round,line cap=round,fill=fillColor] (457.02,551.39) circle (  1.16);

\path[draw=drawColor,line width= 0.4pt,line join=round,line cap=round,fill=fillColor] (457.18,551.32) circle (  1.16);

\path[draw=drawColor,line width= 0.4pt,line join=round,line cap=round,fill=fillColor] (457.34,551.15) circle (  1.16);

\path[draw=drawColor,line width= 0.4pt,line join=round,line cap=round,fill=fillColor] (457.50,550.40) circle (  1.16);

\path[draw=drawColor,line width= 0.4pt,line join=round,line cap=round,fill=fillColor] (457.66,550.30) circle (  1.16);

\path[draw=drawColor,line width= 0.4pt,line join=round,line cap=round,fill=fillColor] (457.82,550.00) circle (  1.16);

\path[draw=drawColor,line width= 0.4pt,line join=round,line cap=round,fill=fillColor] (457.98,549.95) circle (  1.16);

\path[draw=drawColor,line width= 0.4pt,line join=round,line cap=round,fill=fillColor] (458.14,549.88) circle (  1.16);

\path[draw=drawColor,line width= 0.4pt,line join=round,line cap=round,fill=fillColor] (458.30,549.84) circle (  1.16);

\path[draw=drawColor,line width= 0.4pt,line join=round,line cap=round,fill=fillColor] (458.46,549.81) circle (  1.16);

\path[draw=drawColor,line width= 0.4pt,line join=round,line cap=round,fill=fillColor] (458.61,549.68) circle (  1.16);

\path[draw=drawColor,line width= 0.4pt,line join=round,line cap=round,fill=fillColor] (458.77,549.32) circle (  1.16);

\path[draw=drawColor,line width= 0.4pt,line join=round,line cap=round,fill=fillColor] (458.93,549.26) circle (  1.16);

\path[draw=drawColor,line width= 0.4pt,line join=round,line cap=round,fill=fillColor] (459.09,549.21) circle (  1.16);

\path[draw=drawColor,line width= 0.4pt,line join=round,line cap=round,fill=fillColor] (459.24,549.08) circle (  1.16);

\path[draw=drawColor,line width= 0.4pt,line join=round,line cap=round,fill=fillColor] (459.40,548.79) circle (  1.16);

\path[draw=drawColor,line width= 0.4pt,line join=round,line cap=round,fill=fillColor] (459.55,548.79) circle (  1.16);

\path[draw=drawColor,line width= 0.4pt,line join=round,line cap=round,fill=fillColor] (459.71,548.58) circle (  1.16);

\path[draw=drawColor,line width= 0.4pt,line join=round,line cap=round,fill=fillColor] (459.86,548.44) circle (  1.16);

\path[draw=drawColor,line width= 0.4pt,line join=round,line cap=round,fill=fillColor] (460.02,548.39) circle (  1.16);

\path[draw=drawColor,line width= 0.4pt,line join=round,line cap=round,fill=fillColor] (460.17,547.93) circle (  1.16);

\path[draw=drawColor,line width= 0.4pt,line join=round,line cap=round,fill=fillColor] (460.33,547.87) circle (  1.16);

\path[draw=drawColor,line width= 0.4pt,line join=round,line cap=round,fill=fillColor] (460.48,547.74) circle (  1.16);

\path[draw=drawColor,line width= 0.4pt,line join=round,line cap=round,fill=fillColor] (460.63,547.62) circle (  1.16);

\path[draw=drawColor,line width= 0.4pt,line join=round,line cap=round,fill=fillColor] (460.79,547.53) circle (  1.16);

\path[draw=drawColor,line width= 0.4pt,line join=round,line cap=round,fill=fillColor] (460.94,547.52) circle (  1.16);

\path[draw=drawColor,line width= 0.4pt,line join=round,line cap=round,fill=fillColor] (461.09,547.34) circle (  1.16);

\path[draw=drawColor,line width= 0.4pt,line join=round,line cap=round,fill=fillColor] (461.25,547.17) circle (  1.16);

\path[draw=drawColor,line width= 0.4pt,line join=round,line cap=round,fill=fillColor] (461.40,546.80) circle (  1.16);

\path[draw=drawColor,line width= 0.4pt,line join=round,line cap=round,fill=fillColor] (461.55,546.57) circle (  1.16);

\path[draw=drawColor,line width= 0.4pt,line join=round,line cap=round,fill=fillColor] (461.70,546.39) circle (  1.16);

\path[draw=drawColor,line width= 0.4pt,line join=round,line cap=round,fill=fillColor] (461.85,546.15) circle (  1.16);

\path[draw=drawColor,line width= 0.4pt,line join=round,line cap=round,fill=fillColor] (462.00,546.08) circle (  1.16);

\path[draw=drawColor,line width= 0.4pt,line join=round,line cap=round,fill=fillColor] (462.15,546.06) circle (  1.16);

\path[draw=drawColor,line width= 0.4pt,line join=round,line cap=round,fill=fillColor] (462.30,545.97) circle (  1.16);

\path[draw=drawColor,line width= 0.4pt,line join=round,line cap=round,fill=fillColor] (462.45,545.96) circle (  1.16);

\path[draw=drawColor,line width= 0.4pt,line join=round,line cap=round,fill=fillColor] (462.60,545.96) circle (  1.16);

\path[draw=drawColor,line width= 0.4pt,line join=round,line cap=round,fill=fillColor] (462.75,545.96) circle (  1.16);

\path[draw=drawColor,line width= 0.4pt,line join=round,line cap=round,fill=fillColor] (462.90,545.91) circle (  1.16);

\path[draw=drawColor,line width= 0.4pt,line join=round,line cap=round,fill=fillColor] (463.05,545.38) circle (  1.16);

\path[draw=drawColor,line width= 0.4pt,line join=round,line cap=round,fill=fillColor] (463.20,545.37) circle (  1.16);

\path[draw=drawColor,line width= 0.4pt,line join=round,line cap=round,fill=fillColor] (463.35,545.17) circle (  1.16);

\path[draw=drawColor,line width= 0.4pt,line join=round,line cap=round,fill=fillColor] (463.50,545.14) circle (  1.16);

\path[draw=drawColor,line width= 0.4pt,line join=round,line cap=round,fill=fillColor] (463.64,544.96) circle (  1.16);

\path[draw=drawColor,line width= 0.4pt,line join=round,line cap=round,fill=fillColor] (463.79,544.45) circle (  1.16);

\path[draw=drawColor,line width= 0.4pt,line join=round,line cap=round,fill=fillColor] (463.94,544.22) circle (  1.16);

\path[draw=drawColor,line width= 0.4pt,line join=round,line cap=round,fill=fillColor] (464.09,543.71) circle (  1.16);

\path[draw=drawColor,line width= 0.4pt,line join=round,line cap=round,fill=fillColor] (464.23,543.70) circle (  1.16);

\path[draw=drawColor,line width= 0.4pt,line join=round,line cap=round,fill=fillColor] (464.38,543.20) circle (  1.16);

\path[draw=drawColor,line width= 0.4pt,line join=round,line cap=round,fill=fillColor] (464.53,542.91) circle (  1.16);

\path[draw=drawColor,line width= 0.4pt,line join=round,line cap=round,fill=fillColor] (464.67,542.55) circle (  1.16);

\path[draw=drawColor,line width= 0.4pt,line join=round,line cap=round,fill=fillColor] (464.82,535.03) circle (  1.16);

\path[draw=drawColor,line width= 0.4pt,line join=round,line cap=round,fill=fillColor] (464.96,535.03) circle (  1.16);

\path[draw=drawColor,line width= 0.4pt,line join=round,line cap=round,fill=fillColor] (465.11,535.03) circle (  1.16);

\path[draw=drawColor,line width= 0.4pt,line join=round,line cap=round,fill=fillColor] (465.25,535.03) circle (  1.16);

\path[draw=drawColor,line width= 0.4pt,line join=round,line cap=round,fill=fillColor] (465.40,535.03) circle (  1.16);

\path[draw=drawColor,line width= 0.4pt,line join=round,line cap=round,fill=fillColor] (465.54,535.03) circle (  1.16);

\path[draw=drawColor,line width= 0.4pt,line join=round,line cap=round,fill=fillColor] (465.68,535.03) circle (  1.16);

\path[draw=drawColor,line width= 0.4pt,line join=round,line cap=round,fill=fillColor] (465.83,535.03) circle (  1.16);

\path[draw=drawColor,line width= 0.4pt,line join=round,line cap=round,fill=fillColor] (465.97,535.03) circle (  1.16);

\path[draw=drawColor,line width= 0.4pt,line join=round,line cap=round,fill=fillColor] (466.12,535.03) circle (  1.16);

\path[draw=drawColor,line width= 0.4pt,line join=round,line cap=round,fill=fillColor] (466.26,535.03) circle (  1.16);

\path[draw=drawColor,line width= 0.4pt,line join=round,line cap=round,fill=fillColor] (466.40,535.03) circle (  1.16);

\path[draw=drawColor,line width= 0.4pt,line join=round,line cap=round,fill=fillColor] (466.54,535.03) circle (  1.16);

\path[draw=drawColor,line width= 0.4pt,line join=round,line cap=round,fill=fillColor] (466.69,535.03) circle (  1.16);

\path[draw=drawColor,line width= 0.4pt,line join=round,line cap=round,fill=fillColor] (466.83,535.03) circle (  1.16);

\path[draw=drawColor,line width= 0.4pt,line join=round,line cap=round,fill=fillColor] (466.97,535.03) circle (  1.16);

\path[draw=drawColor,line width= 0.4pt,line join=round,line cap=round,fill=fillColor] (467.11,535.03) circle (  1.16);

\path[draw=drawColor,line width= 0.4pt,line join=round,line cap=round,fill=fillColor] (467.25,535.03) circle (  1.16);

\path[draw=drawColor,line width= 0.4pt,line join=round,line cap=round,fill=fillColor] (467.39,535.03) circle (  1.16);

\path[draw=drawColor,line width= 0.4pt,line join=round,line cap=round,fill=fillColor] (467.53,535.03) circle (  1.16);

\path[draw=drawColor,line width= 0.4pt,line join=round,line cap=round,fill=fillColor] (467.67,535.03) circle (  1.16);

\path[draw=drawColor,line width= 0.4pt,line join=round,line cap=round,fill=fillColor] (467.81,535.03) circle (  1.16);

\path[draw=drawColor,line width= 0.4pt,line join=round,line cap=round,fill=fillColor] (467.95,535.03) circle (  1.16);

\path[draw=drawColor,line width= 0.4pt,line join=round,line cap=round,fill=fillColor] (468.09,535.03) circle (  1.16);

\path[draw=drawColor,line width= 0.4pt,line join=round,line cap=round,fill=fillColor] (468.23,535.03) circle (  1.16);

\path[draw=drawColor,line width= 0.4pt,line join=round,line cap=round,fill=fillColor] (468.37,535.03) circle (  1.16);

\path[draw=drawColor,line width= 0.4pt,line join=round,line cap=round,fill=fillColor] (468.51,535.03) circle (  1.16);

\path[draw=drawColor,line width= 0.4pt,line join=round,line cap=round,fill=fillColor] (468.65,535.03) circle (  1.16);

\path[draw=drawColor,line width= 0.4pt,line join=round,line cap=round,fill=fillColor] (468.79,535.03) circle (  1.16);

\path[draw=drawColor,line width= 0.4pt,line join=round,line cap=round,fill=fillColor] (468.93,535.03) circle (  1.16);

\path[draw=drawColor,line width= 0.4pt,line join=round,line cap=round,fill=fillColor] (469.07,535.03) circle (  1.16);

\path[draw=drawColor,line width= 0.4pt,line join=round,line cap=round,fill=fillColor] (469.20,535.03) circle (  1.16);

\path[draw=drawColor,line width= 0.4pt,line join=round,line cap=round,fill=fillColor] (469.34,535.03) circle (  1.16);

\path[draw=drawColor,line width= 0.4pt,line join=round,line cap=round,fill=fillColor] (469.48,535.03) circle (  1.16);

\path[draw=drawColor,line width= 0.4pt,line join=round,line cap=round,fill=fillColor] (469.62,535.03) circle (  1.16);

\path[draw=drawColor,line width= 0.4pt,line join=round,line cap=round,fill=fillColor] (469.75,535.03) circle (  1.16);

\path[draw=drawColor,line width= 0.4pt,line join=round,line cap=round,fill=fillColor] (469.89,535.03) circle (  1.16);

\path[draw=drawColor,line width= 0.4pt,line join=round,line cap=round,fill=fillColor] (470.03,535.03) circle (  1.16);

\path[draw=drawColor,line width= 0.4pt,line join=round,line cap=round,fill=fillColor] (470.16,535.03) circle (  1.16);

\path[draw=drawColor,line width= 0.4pt,line join=round,line cap=round,fill=fillColor] (470.30,535.03) circle (  1.16);

\path[draw=drawColor,line width= 0.4pt,line join=round,line cap=round,fill=fillColor] (470.43,535.03) circle (  1.16);

\path[draw=drawColor,line width= 0.4pt,line join=round,line cap=round,fill=fillColor] (470.57,535.03) circle (  1.16);

\path[draw=drawColor,line width= 0.4pt,line join=round,line cap=round,fill=fillColor] (470.71,535.03) circle (  1.16);

\path[draw=drawColor,line width= 0.4pt,line join=round,line cap=round,fill=fillColor] (470.84,535.03) circle (  1.16);

\path[draw=drawColor,line width= 0.4pt,line join=round,line cap=round,fill=fillColor] (470.98,535.03) circle (  1.16);

\path[draw=drawColor,line width= 0.4pt,line join=round,line cap=round,fill=fillColor] (471.11,535.03) circle (  1.16);

\path[draw=drawColor,line width= 0.4pt,line join=round,line cap=round,fill=fillColor] (471.24,535.03) circle (  1.16);

\path[draw=drawColor,line width= 0.4pt,line join=round,line cap=round,fill=fillColor] (471.38,535.03) circle (  1.16);

\path[draw=drawColor,line width= 0.4pt,line join=round,line cap=round,fill=fillColor] (471.51,535.03) circle (  1.16);

\path[draw=drawColor,line width= 0.4pt,line join=round,line cap=round,fill=fillColor] (471.65,535.03) circle (  1.16);

\path[draw=drawColor,line width= 0.4pt,line join=round,line cap=round,fill=fillColor] (471.78,535.03) circle (  1.16);

\path[draw=drawColor,line width= 0.4pt,line join=round,line cap=round,fill=fillColor] (471.91,535.03) circle (  1.16);

\path[draw=drawColor,line width= 0.4pt,line join=round,line cap=round,fill=fillColor] (472.05,535.03) circle (  1.16);

\path[draw=drawColor,line width= 0.4pt,line join=round,line cap=round,fill=fillColor] (472.18,535.03) circle (  1.16);

\path[draw=drawColor,line width= 0.4pt,line join=round,line cap=round,fill=fillColor] (472.31,535.03) circle (  1.16);

\path[draw=drawColor,line width= 0.4pt,line join=round,line cap=round,fill=fillColor] (472.45,535.03) circle (  1.16);

\path[draw=drawColor,line width= 0.4pt,line join=round,line cap=round,fill=fillColor] (472.58,535.03) circle (  1.16);

\path[draw=drawColor,line width= 0.4pt,line join=round,line cap=round,fill=fillColor] (472.71,535.03) circle (  1.16);

\path[draw=drawColor,line width= 0.4pt,line join=round,line cap=round,fill=fillColor] (472.84,535.03) circle (  1.16);

\path[draw=drawColor,line width= 0.4pt,line join=round,line cap=round,fill=fillColor] (472.97,535.03) circle (  1.16);

\path[draw=drawColor,line width= 0.4pt,line join=round,line cap=round,fill=fillColor] (473.11,535.03) circle (  1.16);

\path[draw=drawColor,line width= 0.4pt,line join=round,line cap=round,fill=fillColor] (473.24,535.03) circle (  1.16);

\path[draw=drawColor,line width= 0.4pt,line join=round,line cap=round,fill=fillColor] (473.37,535.03) circle (  1.16);

\path[draw=drawColor,line width= 0.4pt,line join=round,line cap=round,fill=fillColor] (473.50,535.03) circle (  1.16);

\path[draw=drawColor,line width= 0.4pt,line join=round,line cap=round,fill=fillColor] (473.63,535.03) circle (  1.16);

\path[draw=drawColor,line width= 0.4pt,line join=round,line cap=round,fill=fillColor] (473.76,535.03) circle (  1.16);

\path[draw=drawColor,line width= 0.4pt,line join=round,line cap=round,fill=fillColor] (473.89,535.03) circle (  1.16);

\path[draw=drawColor,line width= 0.4pt,line join=round,line cap=round,fill=fillColor] (474.02,535.03) circle (  1.16);

\path[draw=drawColor,line width= 0.4pt,line join=round,line cap=round,fill=fillColor] (474.15,535.03) circle (  1.16);

\path[draw=drawColor,line width= 0.4pt,line join=round,line cap=round,fill=fillColor] (474.28,535.03) circle (  1.16);

\path[draw=drawColor,line width= 0.4pt,line join=round,line cap=round,fill=fillColor] (474.41,535.03) circle (  1.16);

\path[draw=drawColor,line width= 0.4pt,line join=round,line cap=round,fill=fillColor] (474.54,535.03) circle (  1.16);

\path[draw=drawColor,line width= 0.4pt,line join=round,line cap=round,fill=fillColor] (474.67,535.03) circle (  1.16);

\path[draw=drawColor,line width= 0.4pt,line join=round,line cap=round,fill=fillColor] (474.80,535.03) circle (  1.16);

\path[draw=drawColor,line width= 0.4pt,line join=round,line cap=round,fill=fillColor] (474.93,535.03) circle (  1.16);

\path[draw=drawColor,line width= 0.4pt,line join=round,line cap=round,fill=fillColor] (475.05,535.03) circle (  1.16);

\path[draw=drawColor,line width= 0.4pt,line join=round,line cap=round,fill=fillColor] (475.18,535.03) circle (  1.16);

\path[draw=drawColor,line width= 0.4pt,line join=round,line cap=round,fill=fillColor] (475.31,535.03) circle (  1.16);

\path[draw=drawColor,line width= 0.4pt,line join=round,line cap=round,fill=fillColor] (475.44,535.03) circle (  1.16);

\path[draw=drawColor,line width= 0.4pt,line join=round,line cap=round,fill=fillColor] (475.57,535.03) circle (  1.16);

\path[draw=drawColor,line width= 0.4pt,line join=round,line cap=round,fill=fillColor] (475.69,535.03) circle (  1.16);

\path[draw=drawColor,line width= 0.4pt,line join=round,line cap=round,fill=fillColor] (475.82,535.03) circle (  1.16);

\path[draw=drawColor,line width= 0.4pt,line join=round,line cap=round,fill=fillColor] (475.95,535.03) circle (  1.16);

\path[draw=drawColor,line width= 0.4pt,line join=round,line cap=round,fill=fillColor] (476.08,535.03) circle (  1.16);

\path[draw=drawColor,line width= 0.4pt,line join=round,line cap=round,fill=fillColor] (476.20,535.03) circle (  1.16);

\path[draw=drawColor,line width= 0.4pt,line join=round,line cap=round,fill=fillColor] (476.33,535.03) circle (  1.16);

\path[draw=drawColor,line width= 0.4pt,line join=round,line cap=round,fill=fillColor] (476.46,535.03) circle (  1.16);

\path[draw=drawColor,line width= 0.4pt,line join=round,line cap=round,fill=fillColor] (476.58,535.03) circle (  1.16);

\path[draw=drawColor,line width= 0.4pt,line join=round,line cap=round,fill=fillColor] (476.71,535.03) circle (  1.16);

\path[draw=drawColor,line width= 0.4pt,line join=round,line cap=round,fill=fillColor] (476.83,535.03) circle (  1.16);

\path[draw=drawColor,line width= 0.4pt,line join=round,line cap=round,fill=fillColor] (476.96,535.03) circle (  1.16);

\path[draw=drawColor,line width= 0.4pt,line join=round,line cap=round,fill=fillColor] (477.08,535.03) circle (  1.16);

\path[draw=drawColor,line width= 0.4pt,line join=round,line cap=round,fill=fillColor] (477.21,535.03) circle (  1.16);

\path[draw=drawColor,line width= 0.4pt,line join=round,line cap=round,fill=fillColor] (477.33,535.03) circle (  1.16);

\path[draw=drawColor,line width= 0.4pt,line join=round,line cap=round,fill=fillColor] (477.46,535.03) circle (  1.16);

\path[draw=drawColor,line width= 0.4pt,line join=round,line cap=round,fill=fillColor] (477.58,535.03) circle (  1.16);

\path[draw=drawColor,line width= 0.4pt,line join=round,line cap=round,fill=fillColor] (477.71,535.03) circle (  1.16);

\path[draw=drawColor,line width= 0.4pt,line join=round,line cap=round,fill=fillColor] (477.83,535.03) circle (  1.16);

\path[draw=drawColor,line width= 0.4pt,line join=round,line cap=round,fill=fillColor] (477.96,535.03) circle (  1.16);

\path[draw=drawColor,line width= 0.4pt,line join=round,line cap=round,fill=fillColor] (478.08,535.03) circle (  1.16);

\path[draw=drawColor,line width= 0.4pt,line join=round,line cap=round,fill=fillColor] (478.21,535.03) circle (  1.16);
\definecolor[named]{drawColor}{rgb}{0.60,0.31,0.64}
\definecolor[named]{fillColor}{rgb}{0.60,0.31,0.64}

\path[draw=drawColor,line width= 0.4pt,line join=round,line cap=round,fill=fillColor] (327.82,608.21) circle (  1.16);

\path[draw=drawColor,line width= 0.4pt,line join=round,line cap=round,fill=fillColor] (333.62,606.79) circle (  1.16);

\path[draw=drawColor,line width= 0.4pt,line join=round,line cap=round,fill=fillColor] (337.68,605.60) circle (  1.16);

\path[draw=drawColor,line width= 0.4pt,line join=round,line cap=round,fill=fillColor] (340.92,603.03) circle (  1.16);

\path[draw=drawColor,line width= 0.4pt,line join=round,line cap=round,fill=fillColor] (343.65,602.45) circle (  1.16);

\path[draw=drawColor,line width= 0.4pt,line join=round,line cap=round,fill=fillColor] (346.04,601.93) circle (  1.16);

\path[draw=drawColor,line width= 0.4pt,line join=round,line cap=round,fill=fillColor] (348.17,600.88) circle (  1.16);

\path[draw=drawColor,line width= 0.4pt,line join=round,line cap=round,fill=fillColor] (350.11,600.61) circle (  1.16);

\path[draw=drawColor,line width= 0.4pt,line join=round,line cap=round,fill=fillColor] (351.90,600.41) circle (  1.16);

\path[draw=drawColor,line width= 0.4pt,line join=round,line cap=round,fill=fillColor] (353.55,600.07) circle (  1.16);

\path[draw=drawColor,line width= 0.4pt,line join=round,line cap=round,fill=fillColor] (355.10,600.07) circle (  1.16);

\path[draw=drawColor,line width= 0.4pt,line join=round,line cap=round,fill=fillColor] (356.56,599.93) circle (  1.16);

\path[draw=drawColor,line width= 0.4pt,line join=round,line cap=round,fill=fillColor] (357.94,599.88) circle (  1.16);

\path[draw=drawColor,line width= 0.4pt,line join=round,line cap=round,fill=fillColor] (359.25,599.52) circle (  1.16);

\path[draw=drawColor,line width= 0.4pt,line join=round,line cap=round,fill=fillColor] (360.50,599.17) circle (  1.16);

\path[draw=drawColor,line width= 0.4pt,line join=round,line cap=round,fill=fillColor] (361.70,598.69) circle (  1.16);

\path[draw=drawColor,line width= 0.4pt,line join=round,line cap=round,fill=fillColor] (362.84,598.11) circle (  1.16);

\path[draw=drawColor,line width= 0.4pt,line join=round,line cap=round,fill=fillColor] (363.95,597.69) circle (  1.16);

\path[draw=drawColor,line width= 0.4pt,line join=round,line cap=round,fill=fillColor] (365.01,597.30) circle (  1.16);

\path[draw=drawColor,line width= 0.4pt,line join=round,line cap=round,fill=fillColor] (366.03,596.20) circle (  1.16);

\path[draw=drawColor,line width= 0.4pt,line join=round,line cap=round,fill=fillColor] (367.03,595.88) circle (  1.16);

\path[draw=drawColor,line width= 0.4pt,line join=round,line cap=round,fill=fillColor] (367.99,595.55) circle (  1.16);

\path[draw=drawColor,line width= 0.4pt,line join=round,line cap=round,fill=fillColor] (368.92,595.38) circle (  1.16);

\path[draw=drawColor,line width= 0.4pt,line join=round,line cap=round,fill=fillColor] (369.83,594.89) circle (  1.16);

\path[draw=drawColor,line width= 0.4pt,line join=round,line cap=round,fill=fillColor] (370.71,594.75) circle (  1.16);

\path[draw=drawColor,line width= 0.4pt,line join=round,line cap=round,fill=fillColor] (371.56,594.22) circle (  1.16);

\path[draw=drawColor,line width= 0.4pt,line join=round,line cap=round,fill=fillColor] (372.40,594.21) circle (  1.16);

\path[draw=drawColor,line width= 0.4pt,line join=round,line cap=round,fill=fillColor] (373.21,593.87) circle (  1.16);

\path[draw=drawColor,line width= 0.4pt,line join=round,line cap=round,fill=fillColor] (374.01,593.33) circle (  1.16);

\path[draw=drawColor,line width= 0.4pt,line join=round,line cap=round,fill=fillColor] (374.79,593.20) circle (  1.16);

\path[draw=drawColor,line width= 0.4pt,line join=round,line cap=round,fill=fillColor] (375.55,592.69) circle (  1.16);

\path[draw=drawColor,line width= 0.4pt,line join=round,line cap=round,fill=fillColor] (376.30,592.68) circle (  1.16);

\path[draw=drawColor,line width= 0.4pt,line join=round,line cap=round,fill=fillColor] (377.02,591.64) circle (  1.16);

\path[draw=drawColor,line width= 0.4pt,line join=round,line cap=round,fill=fillColor] (377.74,591.36) circle (  1.16);

\path[draw=drawColor,line width= 0.4pt,line join=round,line cap=round,fill=fillColor] (378.44,591.33) circle (  1.16);

\path[draw=drawColor,line width= 0.4pt,line join=round,line cap=round,fill=fillColor] (379.13,591.16) circle (  1.16);

\path[draw=drawColor,line width= 0.4pt,line join=round,line cap=round,fill=fillColor] (379.80,590.46) circle (  1.16);

\path[draw=drawColor,line width= 0.4pt,line join=round,line cap=round,fill=fillColor] (380.47,589.95) circle (  1.16);

\path[draw=drawColor,line width= 0.4pt,line join=round,line cap=round,fill=fillColor] (381.12,589.91) circle (  1.16);

\path[draw=drawColor,line width= 0.4pt,line join=round,line cap=round,fill=fillColor] (381.76,589.48) circle (  1.16);

\path[draw=drawColor,line width= 0.4pt,line join=round,line cap=round,fill=fillColor] (382.39,589.38) circle (  1.16);

\path[draw=drawColor,line width= 0.4pt,line join=round,line cap=round,fill=fillColor] (383.01,589.37) circle (  1.16);

\path[draw=drawColor,line width= 0.4pt,line join=round,line cap=round,fill=fillColor] (383.62,589.06) circle (  1.16);

\path[draw=drawColor,line width= 0.4pt,line join=round,line cap=round,fill=fillColor] (384.22,589.04) circle (  1.16);

\path[draw=drawColor,line width= 0.4pt,line join=round,line cap=round,fill=fillColor] (384.81,589.00) circle (  1.16);

\path[draw=drawColor,line width= 0.4pt,line join=round,line cap=round,fill=fillColor] (385.39,588.67) circle (  1.16);

\path[draw=drawColor,line width= 0.4pt,line join=round,line cap=round,fill=fillColor] (385.97,588.40) circle (  1.16);

\path[draw=drawColor,line width= 0.4pt,line join=round,line cap=round,fill=fillColor] (386.54,588.24) circle (  1.16);

\path[draw=drawColor,line width= 0.4pt,line join=round,line cap=round,fill=fillColor] (387.09,587.57) circle (  1.16);

\path[draw=drawColor,line width= 0.4pt,line join=round,line cap=round,fill=fillColor] (387.64,587.29) circle (  1.16);

\path[draw=drawColor,line width= 0.4pt,line join=round,line cap=round,fill=fillColor] (388.19,587.23) circle (  1.16);

\path[draw=drawColor,line width= 0.4pt,line join=round,line cap=round,fill=fillColor] (388.73,587.18) circle (  1.16);

\path[draw=drawColor,line width= 0.4pt,line join=round,line cap=round,fill=fillColor] (389.26,587.04) circle (  1.16);

\path[draw=drawColor,line width= 0.4pt,line join=round,line cap=round,fill=fillColor] (389.78,587.00) circle (  1.16);

\path[draw=drawColor,line width= 0.4pt,line join=round,line cap=round,fill=fillColor] (390.30,586.82) circle (  1.16);

\path[draw=drawColor,line width= 0.4pt,line join=round,line cap=round,fill=fillColor] (390.81,586.52) circle (  1.16);

\path[draw=drawColor,line width= 0.4pt,line join=round,line cap=round,fill=fillColor] (391.31,586.41) circle (  1.16);

\path[draw=drawColor,line width= 0.4pt,line join=round,line cap=round,fill=fillColor] (391.81,586.39) circle (  1.16);

\path[draw=drawColor,line width= 0.4pt,line join=round,line cap=round,fill=fillColor] (392.30,586.18) circle (  1.16);

\path[draw=drawColor,line width= 0.4pt,line join=round,line cap=round,fill=fillColor] (392.79,586.02) circle (  1.16);

\path[draw=drawColor,line width= 0.4pt,line join=round,line cap=round,fill=fillColor] (393.27,585.94) circle (  1.16);

\path[draw=drawColor,line width= 0.4pt,line join=round,line cap=round,fill=fillColor] (393.75,585.92) circle (  1.16);

\path[draw=drawColor,line width= 0.4pt,line join=round,line cap=round,fill=fillColor] (394.22,585.82) circle (  1.16);

\path[draw=drawColor,line width= 0.4pt,line join=round,line cap=round,fill=fillColor] (394.69,585.76) circle (  1.16);

\path[draw=drawColor,line width= 0.4pt,line join=round,line cap=round,fill=fillColor] (395.15,585.62) circle (  1.16);

\path[draw=drawColor,line width= 0.4pt,line join=round,line cap=round,fill=fillColor] (395.61,585.57) circle (  1.16);

\path[draw=drawColor,line width= 0.4pt,line join=round,line cap=round,fill=fillColor] (396.06,585.53) circle (  1.16);

\path[draw=drawColor,line width= 0.4pt,line join=round,line cap=round,fill=fillColor] (396.51,585.49) circle (  1.16);

\path[draw=drawColor,line width= 0.4pt,line join=round,line cap=round,fill=fillColor] (396.95,585.49) circle (  1.16);

\path[draw=drawColor,line width= 0.4pt,line join=round,line cap=round,fill=fillColor] (397.39,585.28) circle (  1.16);

\path[draw=drawColor,line width= 0.4pt,line join=round,line cap=round,fill=fillColor] (397.83,585.25) circle (  1.16);

\path[draw=drawColor,line width= 0.4pt,line join=round,line cap=round,fill=fillColor] (398.26,585.19) circle (  1.16);

\path[draw=drawColor,line width= 0.4pt,line join=round,line cap=round,fill=fillColor] (398.68,585.18) circle (  1.16);

\path[draw=drawColor,line width= 0.4pt,line join=round,line cap=round,fill=fillColor] (399.11,585.12) circle (  1.16);

\path[draw=drawColor,line width= 0.4pt,line join=round,line cap=round,fill=fillColor] (399.53,585.07) circle (  1.16);

\path[draw=drawColor,line width= 0.4pt,line join=round,line cap=round,fill=fillColor] (399.94,585.04) circle (  1.16);

\path[draw=drawColor,line width= 0.4pt,line join=round,line cap=round,fill=fillColor] (400.36,584.99) circle (  1.16);

\path[draw=drawColor,line width= 0.4pt,line join=round,line cap=round,fill=fillColor] (400.76,584.96) circle (  1.16);

\path[draw=drawColor,line width= 0.4pt,line join=round,line cap=round,fill=fillColor] (401.17,584.94) circle (  1.16);

\path[draw=drawColor,line width= 0.4pt,line join=round,line cap=round,fill=fillColor] (401.57,584.88) circle (  1.16);

\path[draw=drawColor,line width= 0.4pt,line join=round,line cap=round,fill=fillColor] (401.97,584.87) circle (  1.16);

\path[draw=drawColor,line width= 0.4pt,line join=round,line cap=round,fill=fillColor] (402.36,584.82) circle (  1.16);

\path[draw=drawColor,line width= 0.4pt,line join=round,line cap=round,fill=fillColor] (402.76,584.81) circle (  1.16);

\path[draw=drawColor,line width= 0.4pt,line join=round,line cap=round,fill=fillColor] (403.15,584.65) circle (  1.16);

\path[draw=drawColor,line width= 0.4pt,line join=round,line cap=round,fill=fillColor] (403.53,584.63) circle (  1.16);

\path[draw=drawColor,line width= 0.4pt,line join=round,line cap=round,fill=fillColor] (403.91,584.61) circle (  1.16);

\path[draw=drawColor,line width= 0.4pt,line join=round,line cap=round,fill=fillColor] (404.29,584.60) circle (  1.16);

\path[draw=drawColor,line width= 0.4pt,line join=round,line cap=round,fill=fillColor] (404.67,584.54) circle (  1.16);

\path[draw=drawColor,line width= 0.4pt,line join=round,line cap=round,fill=fillColor] (405.05,584.51) circle (  1.16);

\path[draw=drawColor,line width= 0.4pt,line join=round,line cap=round,fill=fillColor] (405.42,584.41) circle (  1.16);

\path[draw=drawColor,line width= 0.4pt,line join=round,line cap=round,fill=fillColor] (405.78,584.26) circle (  1.16);

\path[draw=drawColor,line width= 0.4pt,line join=round,line cap=round,fill=fillColor] (406.15,584.26) circle (  1.16);

\path[draw=drawColor,line width= 0.4pt,line join=round,line cap=round,fill=fillColor] (406.51,584.22) circle (  1.16);

\path[draw=drawColor,line width= 0.4pt,line join=round,line cap=round,fill=fillColor] (406.87,584.03) circle (  1.16);

\path[draw=drawColor,line width= 0.4pt,line join=round,line cap=round,fill=fillColor] (407.23,583.98) circle (  1.16);

\path[draw=drawColor,line width= 0.4pt,line join=round,line cap=round,fill=fillColor] (407.59,583.94) circle (  1.16);

\path[draw=drawColor,line width= 0.4pt,line join=round,line cap=round,fill=fillColor] (407.94,583.94) circle (  1.16);

\path[draw=drawColor,line width= 0.4pt,line join=round,line cap=round,fill=fillColor] (408.29,583.62) circle (  1.16);

\path[draw=drawColor,line width= 0.4pt,line join=round,line cap=round,fill=fillColor] (408.64,583.51) circle (  1.16);

\path[draw=drawColor,line width= 0.4pt,line join=round,line cap=round,fill=fillColor] (408.99,583.51) circle (  1.16);

\path[draw=drawColor,line width= 0.4pt,line join=round,line cap=round,fill=fillColor] (409.33,583.44) circle (  1.16);

\path[draw=drawColor,line width= 0.4pt,line join=round,line cap=round,fill=fillColor] (409.67,583.41) circle (  1.16);

\path[draw=drawColor,line width= 0.4pt,line join=round,line cap=round,fill=fillColor] (410.01,583.25) circle (  1.16);

\path[draw=drawColor,line width= 0.4pt,line join=round,line cap=round,fill=fillColor] (410.35,583.15) circle (  1.16);

\path[draw=drawColor,line width= 0.4pt,line join=round,line cap=round,fill=fillColor] (410.68,583.11) circle (  1.16);

\path[draw=drawColor,line width= 0.4pt,line join=round,line cap=round,fill=fillColor] (411.02,583.06) circle (  1.16);

\path[draw=drawColor,line width= 0.4pt,line join=round,line cap=round,fill=fillColor] (411.35,583.00) circle (  1.16);

\path[draw=drawColor,line width= 0.4pt,line join=round,line cap=round,fill=fillColor] (411.67,582.99) circle (  1.16);

\path[draw=drawColor,line width= 0.4pt,line join=round,line cap=round,fill=fillColor] (412.00,582.80) circle (  1.16);

\path[draw=drawColor,line width= 0.4pt,line join=round,line cap=round,fill=fillColor] (412.33,582.71) circle (  1.16);

\path[draw=drawColor,line width= 0.4pt,line join=round,line cap=round,fill=fillColor] (412.65,582.67) circle (  1.16);

\path[draw=drawColor,line width= 0.4pt,line join=round,line cap=round,fill=fillColor] (412.97,582.25) circle (  1.16);

\path[draw=drawColor,line width= 0.4pt,line join=round,line cap=round,fill=fillColor] (413.29,582.20) circle (  1.16);

\path[draw=drawColor,line width= 0.4pt,line join=round,line cap=round,fill=fillColor] (413.60,582.16) circle (  1.16);

\path[draw=drawColor,line width= 0.4pt,line join=round,line cap=round,fill=fillColor] (413.92,581.90) circle (  1.16);

\path[draw=drawColor,line width= 0.4pt,line join=round,line cap=round,fill=fillColor] (414.23,581.88) circle (  1.16);

\path[draw=drawColor,line width= 0.4pt,line join=round,line cap=round,fill=fillColor] (414.54,581.76) circle (  1.16);

\path[draw=drawColor,line width= 0.4pt,line join=round,line cap=round,fill=fillColor] (414.85,581.74) circle (  1.16);

\path[draw=drawColor,line width= 0.4pt,line join=round,line cap=round,fill=fillColor] (415.16,581.74) circle (  1.16);

\path[draw=drawColor,line width= 0.4pt,line join=round,line cap=round,fill=fillColor] (415.47,581.48) circle (  1.16);

\path[draw=drawColor,line width= 0.4pt,line join=round,line cap=round,fill=fillColor] (415.77,581.36) circle (  1.16);

\path[draw=drawColor,line width= 0.4pt,line join=round,line cap=round,fill=fillColor] (416.08,581.34) circle (  1.16);

\path[draw=drawColor,line width= 0.4pt,line join=round,line cap=round,fill=fillColor] (416.38,581.27) circle (  1.16);

\path[draw=drawColor,line width= 0.4pt,line join=round,line cap=round,fill=fillColor] (416.68,581.25) circle (  1.16);

\path[draw=drawColor,line width= 0.4pt,line join=round,line cap=round,fill=fillColor] (416.97,581.12) circle (  1.16);

\path[draw=drawColor,line width= 0.4pt,line join=round,line cap=round,fill=fillColor] (417.27,581.04) circle (  1.16);

\path[draw=drawColor,line width= 0.4pt,line join=round,line cap=round,fill=fillColor] (417.57,580.98) circle (  1.16);

\path[draw=drawColor,line width= 0.4pt,line join=round,line cap=round,fill=fillColor] (417.86,580.94) circle (  1.16);

\path[draw=drawColor,line width= 0.4pt,line join=round,line cap=round,fill=fillColor] (418.15,580.86) circle (  1.16);

\path[draw=drawColor,line width= 0.4pt,line join=round,line cap=round,fill=fillColor] (418.44,580.81) circle (  1.16);

\path[draw=drawColor,line width= 0.4pt,line join=round,line cap=round,fill=fillColor] (418.73,580.78) circle (  1.16);

\path[draw=drawColor,line width= 0.4pt,line join=round,line cap=round,fill=fillColor] (419.02,580.56) circle (  1.16);

\path[draw=drawColor,line width= 0.4pt,line join=round,line cap=round,fill=fillColor] (419.30,580.39) circle (  1.16);

\path[draw=drawColor,line width= 0.4pt,line join=round,line cap=round,fill=fillColor] (419.59,580.35) circle (  1.16);

\path[draw=drawColor,line width= 0.4pt,line join=round,line cap=round,fill=fillColor] (419.87,580.33) circle (  1.16);

\path[draw=drawColor,line width= 0.4pt,line join=round,line cap=round,fill=fillColor] (420.15,580.26) circle (  1.16);

\path[draw=drawColor,line width= 0.4pt,line join=round,line cap=round,fill=fillColor] (420.43,580.15) circle (  1.16);

\path[draw=drawColor,line width= 0.4pt,line join=round,line cap=round,fill=fillColor] (420.71,580.05) circle (  1.16);

\path[draw=drawColor,line width= 0.4pt,line join=round,line cap=round,fill=fillColor] (420.99,580.03) circle (  1.16);

\path[draw=drawColor,line width= 0.4pt,line join=round,line cap=round,fill=fillColor] (421.26,579.96) circle (  1.16);

\path[draw=drawColor,line width= 0.4pt,line join=round,line cap=round,fill=fillColor] (421.54,579.91) circle (  1.16);

\path[draw=drawColor,line width= 0.4pt,line join=round,line cap=round,fill=fillColor] (421.81,579.75) circle (  1.16);

\path[draw=drawColor,line width= 0.4pt,line join=round,line cap=round,fill=fillColor] (422.09,579.63) circle (  1.16);

\path[draw=drawColor,line width= 0.4pt,line join=round,line cap=round,fill=fillColor] (422.36,579.58) circle (  1.16);

\path[draw=drawColor,line width= 0.4pt,line join=round,line cap=round,fill=fillColor] (422.63,579.52) circle (  1.16);

\path[draw=drawColor,line width= 0.4pt,line join=round,line cap=round,fill=fillColor] (422.90,579.48) circle (  1.16);

\path[draw=drawColor,line width= 0.4pt,line join=round,line cap=round,fill=fillColor] (423.16,579.46) circle (  1.16);

\path[draw=drawColor,line width= 0.4pt,line join=round,line cap=round,fill=fillColor] (423.43,579.29) circle (  1.16);

\path[draw=drawColor,line width= 0.4pt,line join=round,line cap=round,fill=fillColor] (423.69,579.26) circle (  1.16);

\path[draw=drawColor,line width= 0.4pt,line join=round,line cap=round,fill=fillColor] (423.96,579.21) circle (  1.16);

\path[draw=drawColor,line width= 0.4pt,line join=round,line cap=round,fill=fillColor] (424.22,579.17) circle (  1.16);

\path[draw=drawColor,line width= 0.4pt,line join=round,line cap=round,fill=fillColor] (424.48,579.16) circle (  1.16);

\path[draw=drawColor,line width= 0.4pt,line join=round,line cap=round,fill=fillColor] (424.74,579.02) circle (  1.16);

\path[draw=drawColor,line width= 0.4pt,line join=round,line cap=round,fill=fillColor] (425.00,579.00) circle (  1.16);

\path[draw=drawColor,line width= 0.4pt,line join=round,line cap=round,fill=fillColor] (425.26,578.96) circle (  1.16);

\path[draw=drawColor,line width= 0.4pt,line join=round,line cap=round,fill=fillColor] (425.52,578.88) circle (  1.16);

\path[draw=drawColor,line width= 0.4pt,line join=round,line cap=round,fill=fillColor] (425.77,578.62) circle (  1.16);

\path[draw=drawColor,line width= 0.4pt,line join=round,line cap=round,fill=fillColor] (426.03,578.62) circle (  1.16);

\path[draw=drawColor,line width= 0.4pt,line join=round,line cap=round,fill=fillColor] (426.28,578.49) circle (  1.16);

\path[draw=drawColor,line width= 0.4pt,line join=round,line cap=round,fill=fillColor] (426.53,578.41) circle (  1.16);

\path[draw=drawColor,line width= 0.4pt,line join=round,line cap=round,fill=fillColor] (426.78,578.25) circle (  1.16);

\path[draw=drawColor,line width= 0.4pt,line join=round,line cap=round,fill=fillColor] (427.03,578.25) circle (  1.16);

\path[draw=drawColor,line width= 0.4pt,line join=round,line cap=round,fill=fillColor] (427.28,578.21) circle (  1.16);

\path[draw=drawColor,line width= 0.4pt,line join=round,line cap=round,fill=fillColor] (427.53,578.13) circle (  1.16);

\path[draw=drawColor,line width= 0.4pt,line join=round,line cap=round,fill=fillColor] (427.78,578.01) circle (  1.16);

\path[draw=drawColor,line width= 0.4pt,line join=round,line cap=round,fill=fillColor] (428.03,577.97) circle (  1.16);

\path[draw=drawColor,line width= 0.4pt,line join=round,line cap=round,fill=fillColor] (428.27,577.91) circle (  1.16);

\path[draw=drawColor,line width= 0.4pt,line join=round,line cap=round,fill=fillColor] (428.52,577.89) circle (  1.16);

\path[draw=drawColor,line width= 0.4pt,line join=round,line cap=round,fill=fillColor] (428.76,577.87) circle (  1.16);

\path[draw=drawColor,line width= 0.4pt,line join=round,line cap=round,fill=fillColor] (429.00,577.81) circle (  1.16);

\path[draw=drawColor,line width= 0.4pt,line join=round,line cap=round,fill=fillColor] (429.24,577.75) circle (  1.16);

\path[draw=drawColor,line width= 0.4pt,line join=round,line cap=round,fill=fillColor] (429.48,577.72) circle (  1.16);

\path[draw=drawColor,line width= 0.4pt,line join=round,line cap=round,fill=fillColor] (429.72,577.60) circle (  1.16);

\path[draw=drawColor,line width= 0.4pt,line join=round,line cap=round,fill=fillColor] (429.96,577.58) circle (  1.16);

\path[draw=drawColor,line width= 0.4pt,line join=round,line cap=round,fill=fillColor] (430.20,577.51) circle (  1.16);

\path[draw=drawColor,line width= 0.4pt,line join=round,line cap=round,fill=fillColor] (430.44,577.46) circle (  1.16);

\path[draw=drawColor,line width= 0.4pt,line join=round,line cap=round,fill=fillColor] (430.67,577.43) circle (  1.16);

\path[draw=drawColor,line width= 0.4pt,line join=round,line cap=round,fill=fillColor] (430.91,577.35) circle (  1.16);

\path[draw=drawColor,line width= 0.4pt,line join=round,line cap=round,fill=fillColor] (431.14,577.34) circle (  1.16);

\path[draw=drawColor,line width= 0.4pt,line join=round,line cap=round,fill=fillColor] (431.38,577.31) circle (  1.16);

\path[draw=drawColor,line width= 0.4pt,line join=round,line cap=round,fill=fillColor] (431.61,577.28) circle (  1.16);

\path[draw=drawColor,line width= 0.4pt,line join=round,line cap=round,fill=fillColor] (431.84,577.23) circle (  1.16);

\path[draw=drawColor,line width= 0.4pt,line join=round,line cap=round,fill=fillColor] (432.07,577.15) circle (  1.16);

\path[draw=drawColor,line width= 0.4pt,line join=round,line cap=round,fill=fillColor] (432.30,577.03) circle (  1.16);

\path[draw=drawColor,line width= 0.4pt,line join=round,line cap=round,fill=fillColor] (432.53,577.02) circle (  1.16);

\path[draw=drawColor,line width= 0.4pt,line join=round,line cap=round,fill=fillColor] (432.76,576.99) circle (  1.16);

\path[draw=drawColor,line width= 0.4pt,line join=round,line cap=round,fill=fillColor] (432.99,576.92) circle (  1.16);

\path[draw=drawColor,line width= 0.4pt,line join=round,line cap=round,fill=fillColor] (433.21,576.87) circle (  1.16);

\path[draw=drawColor,line width= 0.4pt,line join=round,line cap=round,fill=fillColor] (433.44,576.86) circle (  1.16);

\path[draw=drawColor,line width= 0.4pt,line join=round,line cap=round,fill=fillColor] (433.67,576.84) circle (  1.16);

\path[draw=drawColor,line width= 0.4pt,line join=round,line cap=round,fill=fillColor] (433.89,576.72) circle (  1.16);

\path[draw=drawColor,line width= 0.4pt,line join=round,line cap=round,fill=fillColor] (434.11,576.64) circle (  1.16);

\path[draw=drawColor,line width= 0.4pt,line join=round,line cap=round,fill=fillColor] (434.34,576.62) circle (  1.16);

\path[draw=drawColor,line width= 0.4pt,line join=round,line cap=round,fill=fillColor] (434.56,576.60) circle (  1.16);

\path[draw=drawColor,line width= 0.4pt,line join=round,line cap=round,fill=fillColor] (434.78,576.58) circle (  1.16);

\path[draw=drawColor,line width= 0.4pt,line join=round,line cap=round,fill=fillColor] (435.00,576.48) circle (  1.16);

\path[draw=drawColor,line width= 0.4pt,line join=round,line cap=round,fill=fillColor] (435.22,576.44) circle (  1.16);

\path[draw=drawColor,line width= 0.4pt,line join=round,line cap=round,fill=fillColor] (435.44,576.38) circle (  1.16);

\path[draw=drawColor,line width= 0.4pt,line join=round,line cap=round,fill=fillColor] (435.66,576.28) circle (  1.16);

\path[draw=drawColor,line width= 0.4pt,line join=round,line cap=round,fill=fillColor] (435.88,576.28) circle (  1.16);

\path[draw=drawColor,line width= 0.4pt,line join=round,line cap=round,fill=fillColor] (436.09,576.17) circle (  1.16);

\path[draw=drawColor,line width= 0.4pt,line join=round,line cap=round,fill=fillColor] (436.31,576.14) circle (  1.16);

\path[draw=drawColor,line width= 0.4pt,line join=round,line cap=round,fill=fillColor] (436.52,576.12) circle (  1.16);

\path[draw=drawColor,line width= 0.4pt,line join=round,line cap=round,fill=fillColor] (436.74,575.81) circle (  1.16);

\path[draw=drawColor,line width= 0.4pt,line join=round,line cap=round,fill=fillColor] (436.95,575.81) circle (  1.16);

\path[draw=drawColor,line width= 0.4pt,line join=round,line cap=round,fill=fillColor] (437.17,575.70) circle (  1.16);

\path[draw=drawColor,line width= 0.4pt,line join=round,line cap=round,fill=fillColor] (437.38,575.69) circle (  1.16);

\path[draw=drawColor,line width= 0.4pt,line join=round,line cap=round,fill=fillColor] (437.59,575.53) circle (  1.16);

\path[draw=drawColor,line width= 0.4pt,line join=round,line cap=round,fill=fillColor] (437.80,575.15) circle (  1.16);

\path[draw=drawColor,line width= 0.4pt,line join=round,line cap=round,fill=fillColor] (438.01,575.13) circle (  1.16);

\path[draw=drawColor,line width= 0.4pt,line join=round,line cap=round,fill=fillColor] (438.22,575.12) circle (  1.16);

\path[draw=drawColor,line width= 0.4pt,line join=round,line cap=round,fill=fillColor] (438.43,575.08) circle (  1.16);

\path[draw=drawColor,line width= 0.4pt,line join=round,line cap=round,fill=fillColor] (438.64,575.02) circle (  1.16);

\path[draw=drawColor,line width= 0.4pt,line join=round,line cap=round,fill=fillColor] (438.85,575.00) circle (  1.16);

\path[draw=drawColor,line width= 0.4pt,line join=round,line cap=round,fill=fillColor] (439.06,574.91) circle (  1.16);

\path[draw=drawColor,line width= 0.4pt,line join=round,line cap=round,fill=fillColor] (439.26,574.61) circle (  1.16);

\path[draw=drawColor,line width= 0.4pt,line join=round,line cap=round,fill=fillColor] (439.47,574.54) circle (  1.16);

\path[draw=drawColor,line width= 0.4pt,line join=round,line cap=round,fill=fillColor] (439.67,574.47) circle (  1.16);

\path[draw=drawColor,line width= 0.4pt,line join=round,line cap=round,fill=fillColor] (439.88,574.46) circle (  1.16);

\path[draw=drawColor,line width= 0.4pt,line join=round,line cap=round,fill=fillColor] (440.08,574.21) circle (  1.16);

\path[draw=drawColor,line width= 0.4pt,line join=round,line cap=round,fill=fillColor] (440.29,574.10) circle (  1.16);

\path[draw=drawColor,line width= 0.4pt,line join=round,line cap=round,fill=fillColor] (440.49,574.05) circle (  1.16);

\path[draw=drawColor,line width= 0.4pt,line join=round,line cap=round,fill=fillColor] (440.69,573.97) circle (  1.16);

\path[draw=drawColor,line width= 0.4pt,line join=round,line cap=round,fill=fillColor] (440.89,573.97) circle (  1.16);

\path[draw=drawColor,line width= 0.4pt,line join=round,line cap=round,fill=fillColor] (441.09,573.83) circle (  1.16);

\path[draw=drawColor,line width= 0.4pt,line join=round,line cap=round,fill=fillColor] (441.29,573.79) circle (  1.16);

\path[draw=drawColor,line width= 0.4pt,line join=round,line cap=round,fill=fillColor] (441.49,573.72) circle (  1.16);

\path[draw=drawColor,line width= 0.4pt,line join=round,line cap=round,fill=fillColor] (441.69,573.69) circle (  1.16);

\path[draw=drawColor,line width= 0.4pt,line join=round,line cap=round,fill=fillColor] (441.89,573.63) circle (  1.16);

\path[draw=drawColor,line width= 0.4pt,line join=round,line cap=round,fill=fillColor] (442.09,573.58) circle (  1.16);

\path[draw=drawColor,line width= 0.4pt,line join=round,line cap=round,fill=fillColor] (442.29,573.52) circle (  1.16);

\path[draw=drawColor,line width= 0.4pt,line join=round,line cap=round,fill=fillColor] (442.49,573.43) circle (  1.16);

\path[draw=drawColor,line width= 0.4pt,line join=round,line cap=round,fill=fillColor] (442.68,573.41) circle (  1.16);

\path[draw=drawColor,line width= 0.4pt,line join=round,line cap=round,fill=fillColor] (442.88,573.40) circle (  1.16);

\path[draw=drawColor,line width= 0.4pt,line join=round,line cap=round,fill=fillColor] (443.07,573.36) circle (  1.16);

\path[draw=drawColor,line width= 0.4pt,line join=round,line cap=round,fill=fillColor] (443.27,573.28) circle (  1.16);

\path[draw=drawColor,line width= 0.4pt,line join=round,line cap=round,fill=fillColor] (443.46,573.16) circle (  1.16);

\path[draw=drawColor,line width= 0.4pt,line join=round,line cap=round,fill=fillColor] (443.66,573.11) circle (  1.16);

\path[draw=drawColor,line width= 0.4pt,line join=round,line cap=round,fill=fillColor] (443.85,573.05) circle (  1.16);

\path[draw=drawColor,line width= 0.4pt,line join=round,line cap=round,fill=fillColor] (444.04,572.97) circle (  1.16);

\path[draw=drawColor,line width= 0.4pt,line join=round,line cap=round,fill=fillColor] (444.23,572.79) circle (  1.16);

\path[draw=drawColor,line width= 0.4pt,line join=round,line cap=round,fill=fillColor] (444.43,572.70) circle (  1.16);

\path[draw=drawColor,line width= 0.4pt,line join=round,line cap=round,fill=fillColor] (444.62,572.65) circle (  1.16);

\path[draw=drawColor,line width= 0.4pt,line join=round,line cap=round,fill=fillColor] (444.81,572.64) circle (  1.16);

\path[draw=drawColor,line width= 0.4pt,line join=round,line cap=round,fill=fillColor] (445.00,572.63) circle (  1.16);

\path[draw=drawColor,line width= 0.4pt,line join=round,line cap=round,fill=fillColor] (445.19,572.60) circle (  1.16);

\path[draw=drawColor,line width= 0.4pt,line join=round,line cap=round,fill=fillColor] (445.38,572.52) circle (  1.16);

\path[draw=drawColor,line width= 0.4pt,line join=round,line cap=round,fill=fillColor] (445.56,572.51) circle (  1.16);

\path[draw=drawColor,line width= 0.4pt,line join=round,line cap=round,fill=fillColor] (445.75,572.49) circle (  1.16);

\path[draw=drawColor,line width= 0.4pt,line join=round,line cap=round,fill=fillColor] (445.94,572.30) circle (  1.16);

\path[draw=drawColor,line width= 0.4pt,line join=round,line cap=round,fill=fillColor] (446.13,572.22) circle (  1.16);

\path[draw=drawColor,line width= 0.4pt,line join=round,line cap=round,fill=fillColor] (446.31,572.18) circle (  1.16);

\path[draw=drawColor,line width= 0.4pt,line join=round,line cap=round,fill=fillColor] (446.50,572.16) circle (  1.16);

\path[draw=drawColor,line width= 0.4pt,line join=round,line cap=round,fill=fillColor] (446.68,572.10) circle (  1.16);

\path[draw=drawColor,line width= 0.4pt,line join=round,line cap=round,fill=fillColor] (446.87,572.10) circle (  1.16);

\path[draw=drawColor,line width= 0.4pt,line join=round,line cap=round,fill=fillColor] (447.05,571.93) circle (  1.16);

\path[draw=drawColor,line width= 0.4pt,line join=round,line cap=round,fill=fillColor] (447.24,571.93) circle (  1.16);

\path[draw=drawColor,line width= 0.4pt,line join=round,line cap=round,fill=fillColor] (447.42,571.90) circle (  1.16);

\path[draw=drawColor,line width= 0.4pt,line join=round,line cap=round,fill=fillColor] (447.60,571.83) circle (  1.16);

\path[draw=drawColor,line width= 0.4pt,line join=round,line cap=round,fill=fillColor] (447.79,571.82) circle (  1.16);

\path[draw=drawColor,line width= 0.4pt,line join=round,line cap=round,fill=fillColor] (447.97,571.79) circle (  1.16);

\path[draw=drawColor,line width= 0.4pt,line join=round,line cap=round,fill=fillColor] (448.15,571.78) circle (  1.16);

\path[draw=drawColor,line width= 0.4pt,line join=round,line cap=round,fill=fillColor] (448.33,571.74) circle (  1.16);

\path[draw=drawColor,line width= 0.4pt,line join=round,line cap=round,fill=fillColor] (448.51,571.67) circle (  1.16);

\path[draw=drawColor,line width= 0.4pt,line join=round,line cap=round,fill=fillColor] (448.69,571.54) circle (  1.16);

\path[draw=drawColor,line width= 0.4pt,line join=round,line cap=round,fill=fillColor] (448.87,571.50) circle (  1.16);

\path[draw=drawColor,line width= 0.4pt,line join=round,line cap=round,fill=fillColor] (449.05,571.40) circle (  1.16);

\path[draw=drawColor,line width= 0.4pt,line join=round,line cap=round,fill=fillColor] (449.23,571.40) circle (  1.16);

\path[draw=drawColor,line width= 0.4pt,line join=round,line cap=round,fill=fillColor] (449.41,571.38) circle (  1.16);

\path[draw=drawColor,line width= 0.4pt,line join=round,line cap=round,fill=fillColor] (449.59,571.37) circle (  1.16);

\path[draw=drawColor,line width= 0.4pt,line join=round,line cap=round,fill=fillColor] (449.77,571.30) circle (  1.16);

\path[draw=drawColor,line width= 0.4pt,line join=round,line cap=round,fill=fillColor] (449.94,571.22) circle (  1.16);

\path[draw=drawColor,line width= 0.4pt,line join=round,line cap=round,fill=fillColor] (450.12,571.19) circle (  1.16);

\path[draw=drawColor,line width= 0.4pt,line join=round,line cap=round,fill=fillColor] (450.30,571.18) circle (  1.16);

\path[draw=drawColor,line width= 0.4pt,line join=round,line cap=round,fill=fillColor] (450.47,571.11) circle (  1.16);

\path[draw=drawColor,line width= 0.4pt,line join=round,line cap=round,fill=fillColor] (450.65,571.07) circle (  1.16);

\path[draw=drawColor,line width= 0.4pt,line join=round,line cap=round,fill=fillColor] (450.82,571.00) circle (  1.16);

\path[draw=drawColor,line width= 0.4pt,line join=round,line cap=round,fill=fillColor] (451.00,570.57) circle (  1.16);

\path[draw=drawColor,line width= 0.4pt,line join=round,line cap=round,fill=fillColor] (451.17,570.53) circle (  1.16);

\path[draw=drawColor,line width= 0.4pt,line join=round,line cap=round,fill=fillColor] (451.35,570.46) circle (  1.16);

\path[draw=drawColor,line width= 0.4pt,line join=round,line cap=round,fill=fillColor] (451.52,570.43) circle (  1.16);

\path[draw=drawColor,line width= 0.4pt,line join=round,line cap=round,fill=fillColor] (451.69,570.43) circle (  1.16);

\path[draw=drawColor,line width= 0.4pt,line join=round,line cap=round,fill=fillColor] (451.86,570.42) circle (  1.16);

\path[draw=drawColor,line width= 0.4pt,line join=round,line cap=round,fill=fillColor] (452.04,570.33) circle (  1.16);

\path[draw=drawColor,line width= 0.4pt,line join=round,line cap=round,fill=fillColor] (452.21,570.31) circle (  1.16);

\path[draw=drawColor,line width= 0.4pt,line join=round,line cap=round,fill=fillColor] (452.38,570.31) circle (  1.16);

\path[draw=drawColor,line width= 0.4pt,line join=round,line cap=round,fill=fillColor] (452.55,570.28) circle (  1.16);

\path[draw=drawColor,line width= 0.4pt,line join=round,line cap=round,fill=fillColor] (452.72,570.28) circle (  1.16);

\path[draw=drawColor,line width= 0.4pt,line join=round,line cap=round,fill=fillColor] (452.89,570.28) circle (  1.16);

\path[draw=drawColor,line width= 0.4pt,line join=round,line cap=round,fill=fillColor] (453.06,570.17) circle (  1.16);

\path[draw=drawColor,line width= 0.4pt,line join=round,line cap=round,fill=fillColor] (453.23,570.06) circle (  1.16);

\path[draw=drawColor,line width= 0.4pt,line join=round,line cap=round,fill=fillColor] (453.40,569.87) circle (  1.16);

\path[draw=drawColor,line width= 0.4pt,line join=round,line cap=round,fill=fillColor] (453.57,569.85) circle (  1.16);

\path[draw=drawColor,line width= 0.4pt,line join=round,line cap=round,fill=fillColor] (453.74,569.83) circle (  1.16);

\path[draw=drawColor,line width= 0.4pt,line join=round,line cap=round,fill=fillColor] (453.90,569.82) circle (  1.16);

\path[draw=drawColor,line width= 0.4pt,line join=round,line cap=round,fill=fillColor] (454.07,569.78) circle (  1.16);

\path[draw=drawColor,line width= 0.4pt,line join=round,line cap=round,fill=fillColor] (454.24,569.75) circle (  1.16);

\path[draw=drawColor,line width= 0.4pt,line join=round,line cap=round,fill=fillColor] (454.40,569.71) circle (  1.16);

\path[draw=drawColor,line width= 0.4pt,line join=round,line cap=round,fill=fillColor] (454.57,569.49) circle (  1.16);

\path[draw=drawColor,line width= 0.4pt,line join=round,line cap=round,fill=fillColor] (454.74,569.48) circle (  1.16);

\path[draw=drawColor,line width= 0.4pt,line join=round,line cap=round,fill=fillColor] (454.90,569.40) circle (  1.16);

\path[draw=drawColor,line width= 0.4pt,line join=round,line cap=round,fill=fillColor] (455.07,569.25) circle (  1.16);

\path[draw=drawColor,line width= 0.4pt,line join=round,line cap=round,fill=fillColor] (455.23,569.21) circle (  1.16);

\path[draw=drawColor,line width= 0.4pt,line join=round,line cap=round,fill=fillColor] (455.40,569.21) circle (  1.16);

\path[draw=drawColor,line width= 0.4pt,line join=round,line cap=round,fill=fillColor] (455.56,569.10) circle (  1.16);

\path[draw=drawColor,line width= 0.4pt,line join=round,line cap=round,fill=fillColor] (455.73,568.99) circle (  1.16);

\path[draw=drawColor,line width= 0.4pt,line join=round,line cap=round,fill=fillColor] (455.89,568.89) circle (  1.16);

\path[draw=drawColor,line width= 0.4pt,line join=round,line cap=round,fill=fillColor] (456.05,568.80) circle (  1.16);

\path[draw=drawColor,line width= 0.4pt,line join=round,line cap=round,fill=fillColor] (456.21,568.67) circle (  1.16);

\path[draw=drawColor,line width= 0.4pt,line join=round,line cap=round,fill=fillColor] (456.38,568.56) circle (  1.16);

\path[draw=drawColor,line width= 0.4pt,line join=round,line cap=round,fill=fillColor] (456.54,568.43) circle (  1.16);

\path[draw=drawColor,line width= 0.4pt,line join=round,line cap=round,fill=fillColor] (456.70,568.43) circle (  1.16);

\path[draw=drawColor,line width= 0.4pt,line join=round,line cap=round,fill=fillColor] (456.86,568.38) circle (  1.16);

\path[draw=drawColor,line width= 0.4pt,line join=round,line cap=round,fill=fillColor] (457.02,568.34) circle (  1.16);

\path[draw=drawColor,line width= 0.4pt,line join=round,line cap=round,fill=fillColor] (457.18,568.30) circle (  1.16);

\path[draw=drawColor,line width= 0.4pt,line join=round,line cap=round,fill=fillColor] (457.34,568.11) circle (  1.16);

\path[draw=drawColor,line width= 0.4pt,line join=round,line cap=round,fill=fillColor] (457.50,568.01) circle (  1.16);

\path[draw=drawColor,line width= 0.4pt,line join=round,line cap=round,fill=fillColor] (457.66,567.90) circle (  1.16);

\path[draw=drawColor,line width= 0.4pt,line join=round,line cap=round,fill=fillColor] (457.82,567.81) circle (  1.16);

\path[draw=drawColor,line width= 0.4pt,line join=round,line cap=round,fill=fillColor] (457.98,567.76) circle (  1.16);

\path[draw=drawColor,line width= 0.4pt,line join=round,line cap=round,fill=fillColor] (458.14,567.73) circle (  1.16);

\path[draw=drawColor,line width= 0.4pt,line join=round,line cap=round,fill=fillColor] (458.30,567.19) circle (  1.16);

\path[draw=drawColor,line width= 0.4pt,line join=round,line cap=round,fill=fillColor] (458.46,567.16) circle (  1.16);

\path[draw=drawColor,line width= 0.4pt,line join=round,line cap=round,fill=fillColor] (458.61,567.07) circle (  1.16);

\path[draw=drawColor,line width= 0.4pt,line join=round,line cap=round,fill=fillColor] (458.77,567.07) circle (  1.16);

\path[draw=drawColor,line width= 0.4pt,line join=round,line cap=round,fill=fillColor] (458.93,566.87) circle (  1.16);

\path[draw=drawColor,line width= 0.4pt,line join=round,line cap=round,fill=fillColor] (459.09,566.83) circle (  1.16);

\path[draw=drawColor,line width= 0.4pt,line join=round,line cap=round,fill=fillColor] (459.24,566.83) circle (  1.16);

\path[draw=drawColor,line width= 0.4pt,line join=round,line cap=round,fill=fillColor] (459.40,566.64) circle (  1.16);

\path[draw=drawColor,line width= 0.4pt,line join=round,line cap=round,fill=fillColor] (459.55,566.63) circle (  1.16);

\path[draw=drawColor,line width= 0.4pt,line join=round,line cap=round,fill=fillColor] (459.71,566.55) circle (  1.16);

\path[draw=drawColor,line width= 0.4pt,line join=round,line cap=round,fill=fillColor] (459.86,566.49) circle (  1.16);

\path[draw=drawColor,line width= 0.4pt,line join=round,line cap=round,fill=fillColor] (460.02,566.41) circle (  1.16);

\path[draw=drawColor,line width= 0.4pt,line join=round,line cap=round,fill=fillColor] (460.17,566.37) circle (  1.16);

\path[draw=drawColor,line width= 0.4pt,line join=round,line cap=round,fill=fillColor] (460.33,566.23) circle (  1.16);

\path[draw=drawColor,line width= 0.4pt,line join=round,line cap=round,fill=fillColor] (460.48,566.12) circle (  1.16);

\path[draw=drawColor,line width= 0.4pt,line join=round,line cap=round,fill=fillColor] (460.63,565.91) circle (  1.16);

\path[draw=drawColor,line width= 0.4pt,line join=round,line cap=round,fill=fillColor] (460.79,565.90) circle (  1.16);

\path[draw=drawColor,line width= 0.4pt,line join=round,line cap=round,fill=fillColor] (460.94,565.87) circle (  1.16);

\path[draw=drawColor,line width= 0.4pt,line join=round,line cap=round,fill=fillColor] (461.09,565.80) circle (  1.16);

\path[draw=drawColor,line width= 0.4pt,line join=round,line cap=round,fill=fillColor] (461.25,565.71) circle (  1.16);

\path[draw=drawColor,line width= 0.4pt,line join=round,line cap=round,fill=fillColor] (461.40,565.59) circle (  1.16);

\path[draw=drawColor,line width= 0.4pt,line join=round,line cap=round,fill=fillColor] (461.55,565.48) circle (  1.16);

\path[draw=drawColor,line width= 0.4pt,line join=round,line cap=round,fill=fillColor] (461.70,565.41) circle (  1.16);

\path[draw=drawColor,line width= 0.4pt,line join=round,line cap=round,fill=fillColor] (461.85,565.35) circle (  1.16);

\path[draw=drawColor,line width= 0.4pt,line join=round,line cap=round,fill=fillColor] (462.00,565.32) circle (  1.16);

\path[draw=drawColor,line width= 0.4pt,line join=round,line cap=round,fill=fillColor] (462.15,565.31) circle (  1.16);

\path[draw=drawColor,line width= 0.4pt,line join=round,line cap=round,fill=fillColor] (462.30,565.26) circle (  1.16);

\path[draw=drawColor,line width= 0.4pt,line join=round,line cap=round,fill=fillColor] (462.45,565.14) circle (  1.16);

\path[draw=drawColor,line width= 0.4pt,line join=round,line cap=round,fill=fillColor] (462.60,564.99) circle (  1.16);

\path[draw=drawColor,line width= 0.4pt,line join=round,line cap=round,fill=fillColor] (462.75,564.93) circle (  1.16);

\path[draw=drawColor,line width= 0.4pt,line join=round,line cap=round,fill=fillColor] (462.90,564.85) circle (  1.16);

\path[draw=drawColor,line width= 0.4pt,line join=round,line cap=round,fill=fillColor] (463.05,564.66) circle (  1.16);

\path[draw=drawColor,line width= 0.4pt,line join=round,line cap=round,fill=fillColor] (463.20,564.62) circle (  1.16);

\path[draw=drawColor,line width= 0.4pt,line join=round,line cap=round,fill=fillColor] (463.35,564.33) circle (  1.16);

\path[draw=drawColor,line width= 0.4pt,line join=round,line cap=round,fill=fillColor] (463.50,564.26) circle (  1.16);

\path[draw=drawColor,line width= 0.4pt,line join=round,line cap=round,fill=fillColor] (463.64,564.23) circle (  1.16);

\path[draw=drawColor,line width= 0.4pt,line join=round,line cap=round,fill=fillColor] (463.79,564.09) circle (  1.16);

\path[draw=drawColor,line width= 0.4pt,line join=round,line cap=round,fill=fillColor] (463.94,563.93) circle (  1.16);

\path[draw=drawColor,line width= 0.4pt,line join=round,line cap=round,fill=fillColor] (464.09,563.83) circle (  1.16);

\path[draw=drawColor,line width= 0.4pt,line join=round,line cap=round,fill=fillColor] (464.23,563.77) circle (  1.16);

\path[draw=drawColor,line width= 0.4pt,line join=round,line cap=round,fill=fillColor] (464.38,563.68) circle (  1.16);

\path[draw=drawColor,line width= 0.4pt,line join=round,line cap=round,fill=fillColor] (464.53,563.59) circle (  1.16);

\path[draw=drawColor,line width= 0.4pt,line join=round,line cap=round,fill=fillColor] (464.67,563.55) circle (  1.16);

\path[draw=drawColor,line width= 0.4pt,line join=round,line cap=round,fill=fillColor] (464.82,563.50) circle (  1.16);

\path[draw=drawColor,line width= 0.4pt,line join=round,line cap=round,fill=fillColor] (464.96,563.40) circle (  1.16);

\path[draw=drawColor,line width= 0.4pt,line join=round,line cap=round,fill=fillColor] (465.11,563.26) circle (  1.16);

\path[draw=drawColor,line width= 0.4pt,line join=round,line cap=round,fill=fillColor] (465.25,563.12) circle (  1.16);

\path[draw=drawColor,line width= 0.4pt,line join=round,line cap=round,fill=fillColor] (465.40,562.98) circle (  1.16);

\path[draw=drawColor,line width= 0.4pt,line join=round,line cap=round,fill=fillColor] (465.54,562.61) circle (  1.16);

\path[draw=drawColor,line width= 0.4pt,line join=round,line cap=round,fill=fillColor] (465.68,562.39) circle (  1.16);

\path[draw=drawColor,line width= 0.4pt,line join=round,line cap=round,fill=fillColor] (465.83,562.09) circle (  1.16);

\path[draw=drawColor,line width= 0.4pt,line join=round,line cap=round,fill=fillColor] (465.97,561.82) circle (  1.16);

\path[draw=drawColor,line width= 0.4pt,line join=round,line cap=round,fill=fillColor] (466.12,561.75) circle (  1.16);

\path[draw=drawColor,line width= 0.4pt,line join=round,line cap=round,fill=fillColor] (466.26,561.71) circle (  1.16);

\path[draw=drawColor,line width= 0.4pt,line join=round,line cap=round,fill=fillColor] (466.40,561.56) circle (  1.16);

\path[draw=drawColor,line width= 0.4pt,line join=round,line cap=round,fill=fillColor] (466.54,561.52) circle (  1.16);

\path[draw=drawColor,line width= 0.4pt,line join=round,line cap=round,fill=fillColor] (466.69,561.44) circle (  1.16);

\path[draw=drawColor,line width= 0.4pt,line join=round,line cap=round,fill=fillColor] (466.83,561.41) circle (  1.16);

\path[draw=drawColor,line width= 0.4pt,line join=round,line cap=round,fill=fillColor] (466.97,561.06) circle (  1.16);

\path[draw=drawColor,line width= 0.4pt,line join=round,line cap=round,fill=fillColor] (467.11,560.95) circle (  1.16);

\path[draw=drawColor,line width= 0.4pt,line join=round,line cap=round,fill=fillColor] (467.25,560.93) circle (  1.16);

\path[draw=drawColor,line width= 0.4pt,line join=round,line cap=round,fill=fillColor] (467.39,560.76) circle (  1.16);

\path[draw=drawColor,line width= 0.4pt,line join=round,line cap=round,fill=fillColor] (467.53,560.74) circle (  1.16);

\path[draw=drawColor,line width= 0.4pt,line join=round,line cap=round,fill=fillColor] (467.67,560.70) circle (  1.16);

\path[draw=drawColor,line width= 0.4pt,line join=round,line cap=round,fill=fillColor] (467.81,560.61) circle (  1.16);

\path[draw=drawColor,line width= 0.4pt,line join=round,line cap=round,fill=fillColor] (467.95,560.57) circle (  1.16);

\path[draw=drawColor,line width= 0.4pt,line join=round,line cap=round,fill=fillColor] (468.09,560.41) circle (  1.16);

\path[draw=drawColor,line width= 0.4pt,line join=round,line cap=round,fill=fillColor] (468.23,560.41) circle (  1.16);

\path[draw=drawColor,line width= 0.4pt,line join=round,line cap=round,fill=fillColor] (468.37,560.34) circle (  1.16);

\path[draw=drawColor,line width= 0.4pt,line join=round,line cap=round,fill=fillColor] (468.51,560.31) circle (  1.16);

\path[draw=drawColor,line width= 0.4pt,line join=round,line cap=round,fill=fillColor] (468.65,559.98) circle (  1.16);

\path[draw=drawColor,line width= 0.4pt,line join=round,line cap=round,fill=fillColor] (468.79,559.42) circle (  1.16);

\path[draw=drawColor,line width= 0.4pt,line join=round,line cap=round,fill=fillColor] (468.93,559.37) circle (  1.16);

\path[draw=drawColor,line width= 0.4pt,line join=round,line cap=round,fill=fillColor] (469.07,559.25) circle (  1.16);

\path[draw=drawColor,line width= 0.4pt,line join=round,line cap=round,fill=fillColor] (469.20,559.06) circle (  1.16);

\path[draw=drawColor,line width= 0.4pt,line join=round,line cap=round,fill=fillColor] (469.34,558.86) circle (  1.16);

\path[draw=drawColor,line width= 0.4pt,line join=round,line cap=round,fill=fillColor] (469.48,558.75) circle (  1.16);

\path[draw=drawColor,line width= 0.4pt,line join=round,line cap=round,fill=fillColor] (469.62,558.42) circle (  1.16);

\path[draw=drawColor,line width= 0.4pt,line join=round,line cap=round,fill=fillColor] (469.75,558.32) circle (  1.16);

\path[draw=drawColor,line width= 0.4pt,line join=round,line cap=round,fill=fillColor] (469.89,558.05) circle (  1.16);

\path[draw=drawColor,line width= 0.4pt,line join=round,line cap=round,fill=fillColor] (470.03,557.82) circle (  1.16);

\path[draw=drawColor,line width= 0.4pt,line join=round,line cap=round,fill=fillColor] (470.16,557.73) circle (  1.16);

\path[draw=drawColor,line width= 0.4pt,line join=round,line cap=round,fill=fillColor] (470.30,557.50) circle (  1.16);

\path[draw=drawColor,line width= 0.4pt,line join=round,line cap=round,fill=fillColor] (470.43,557.35) circle (  1.16);

\path[draw=drawColor,line width= 0.4pt,line join=round,line cap=round,fill=fillColor] (470.57,557.08) circle (  1.16);

\path[draw=drawColor,line width= 0.4pt,line join=round,line cap=round,fill=fillColor] (470.71,556.83) circle (  1.16);

\path[draw=drawColor,line width= 0.4pt,line join=round,line cap=round,fill=fillColor] (470.84,556.69) circle (  1.16);

\path[draw=drawColor,line width= 0.4pt,line join=round,line cap=round,fill=fillColor] (470.98,556.65) circle (  1.16);

\path[draw=drawColor,line width= 0.4pt,line join=round,line cap=round,fill=fillColor] (471.11,556.08) circle (  1.16);

\path[draw=drawColor,line width= 0.4pt,line join=round,line cap=round,fill=fillColor] (471.24,555.37) circle (  1.16);

\path[draw=drawColor,line width= 0.4pt,line join=round,line cap=round,fill=fillColor] (471.38,555.30) circle (  1.16);

\path[draw=drawColor,line width= 0.4pt,line join=round,line cap=round,fill=fillColor] (471.51,555.21) circle (  1.16);

\path[draw=drawColor,line width= 0.4pt,line join=round,line cap=round,fill=fillColor] (471.65,555.06) circle (  1.16);

\path[draw=drawColor,line width= 0.4pt,line join=round,line cap=round,fill=fillColor] (471.78,554.81) circle (  1.16);

\path[draw=drawColor,line width= 0.4pt,line join=round,line cap=round,fill=fillColor] (471.91,554.79) circle (  1.16);

\path[draw=drawColor,line width= 0.4pt,line join=round,line cap=round,fill=fillColor] (472.05,554.39) circle (  1.16);

\path[draw=drawColor,line width= 0.4pt,line join=round,line cap=round,fill=fillColor] (472.18,554.17) circle (  1.16);

\path[draw=drawColor,line width= 0.4pt,line join=round,line cap=round,fill=fillColor] (472.31,553.97) circle (  1.16);

\path[draw=drawColor,line width= 0.4pt,line join=round,line cap=round,fill=fillColor] (472.45,553.46) circle (  1.16);

\path[draw=drawColor,line width= 0.4pt,line join=round,line cap=round,fill=fillColor] (472.58,553.32) circle (  1.16);

\path[draw=drawColor,line width= 0.4pt,line join=round,line cap=round,fill=fillColor] (472.71,553.22) circle (  1.16);

\path[draw=drawColor,line width= 0.4pt,line join=round,line cap=round,fill=fillColor] (472.84,553.04) circle (  1.16);

\path[draw=drawColor,line width= 0.4pt,line join=round,line cap=round,fill=fillColor] (472.97,553.03) circle (  1.16);

\path[draw=drawColor,line width= 0.4pt,line join=round,line cap=round,fill=fillColor] (473.11,552.97) circle (  1.16);

\path[draw=drawColor,line width= 0.4pt,line join=round,line cap=round,fill=fillColor] (473.24,552.58) circle (  1.16);

\path[draw=drawColor,line width= 0.4pt,line join=round,line cap=round,fill=fillColor] (473.37,552.58) circle (  1.16);

\path[draw=drawColor,line width= 0.4pt,line join=round,line cap=round,fill=fillColor] (473.50,551.98) circle (  1.16);

\path[draw=drawColor,line width= 0.4pt,line join=round,line cap=round,fill=fillColor] (473.63,551.87) circle (  1.16);

\path[draw=drawColor,line width= 0.4pt,line join=round,line cap=round,fill=fillColor] (473.76,551.59) circle (  1.16);

\path[draw=drawColor,line width= 0.4pt,line join=round,line cap=round,fill=fillColor] (473.89,551.05) circle (  1.16);

\path[draw=drawColor,line width= 0.4pt,line join=round,line cap=round,fill=fillColor] (474.02,550.75) circle (  1.16);

\path[draw=drawColor,line width= 0.4pt,line join=round,line cap=round,fill=fillColor] (474.15,550.70) circle (  1.16);

\path[draw=drawColor,line width= 0.4pt,line join=round,line cap=round,fill=fillColor] (474.28,550.26) circle (  1.16);

\path[draw=drawColor,line width= 0.4pt,line join=round,line cap=round,fill=fillColor] (474.41,548.91) circle (  1.16);

\path[draw=drawColor,line width= 0.4pt,line join=round,line cap=round,fill=fillColor] (474.54,547.64) circle (  1.16);

\path[draw=drawColor,line width= 0.4pt,line join=round,line cap=round,fill=fillColor] (474.67,546.44) circle (  1.16);

\path[draw=drawColor,line width= 0.4pt,line join=round,line cap=round,fill=fillColor] (474.80,546.07) circle (  1.16);

\path[draw=drawColor,line width= 0.4pt,line join=round,line cap=round,fill=fillColor] (474.93,544.16) circle (  1.16);

\path[draw=drawColor,line width= 0.4pt,line join=round,line cap=round,fill=fillColor] (475.05,543.56) circle (  1.16);

\path[draw=drawColor,line width= 0.4pt,line join=round,line cap=round,fill=fillColor] (475.18,542.79) circle (  1.16);

\path[draw=drawColor,line width= 0.4pt,line join=round,line cap=round,fill=fillColor] (475.31,535.03) circle (  1.16);

\path[draw=drawColor,line width= 0.4pt,line join=round,line cap=round,fill=fillColor] (475.44,535.03) circle (  1.16);

\path[draw=drawColor,line width= 0.4pt,line join=round,line cap=round,fill=fillColor] (475.57,535.03) circle (  1.16);

\path[draw=drawColor,line width= 0.4pt,line join=round,line cap=round,fill=fillColor] (475.69,535.03) circle (  1.16);

\path[draw=drawColor,line width= 0.4pt,line join=round,line cap=round,fill=fillColor] (475.82,535.03) circle (  1.16);

\path[draw=drawColor,line width= 0.4pt,line join=round,line cap=round,fill=fillColor] (475.95,535.03) circle (  1.16);

\path[draw=drawColor,line width= 0.4pt,line join=round,line cap=round,fill=fillColor] (476.08,535.03) circle (  1.16);

\path[draw=drawColor,line width= 0.4pt,line join=round,line cap=round,fill=fillColor] (476.20,535.03) circle (  1.16);

\path[draw=drawColor,line width= 0.4pt,line join=round,line cap=round,fill=fillColor] (476.33,535.03) circle (  1.16);

\path[draw=drawColor,line width= 0.4pt,line join=round,line cap=round,fill=fillColor] (476.46,535.03) circle (  1.16);

\path[draw=drawColor,line width= 0.4pt,line join=round,line cap=round,fill=fillColor] (476.58,535.03) circle (  1.16);

\path[draw=drawColor,line width= 0.4pt,line join=round,line cap=round,fill=fillColor] (476.71,535.03) circle (  1.16);

\path[draw=drawColor,line width= 0.4pt,line join=round,line cap=round,fill=fillColor] (476.83,535.03) circle (  1.16);

\path[draw=drawColor,line width= 0.4pt,line join=round,line cap=round,fill=fillColor] (476.96,535.03) circle (  1.16);

\path[draw=drawColor,line width= 0.4pt,line join=round,line cap=round,fill=fillColor] (477.08,535.03) circle (  1.16);

\path[draw=drawColor,line width= 0.4pt,line join=round,line cap=round,fill=fillColor] (477.21,535.03) circle (  1.16);

\path[draw=drawColor,line width= 0.4pt,line join=round,line cap=round,fill=fillColor] (477.33,535.03) circle (  1.16);

\path[draw=drawColor,line width= 0.4pt,line join=round,line cap=round,fill=fillColor] (477.46,535.03) circle (  1.16);

\path[draw=drawColor,line width= 0.4pt,line join=round,line cap=round,fill=fillColor] (477.58,535.03) circle (  1.16);

\path[draw=drawColor,line width= 0.4pt,line join=round,line cap=round,fill=fillColor] (477.71,535.03) circle (  1.16);

\path[draw=drawColor,line width= 0.4pt,line join=round,line cap=round,fill=fillColor] (477.83,535.03) circle (  1.16);

\path[draw=drawColor,line width= 0.4pt,line join=round,line cap=round,fill=fillColor] (477.96,535.03) circle (  1.16);

\path[draw=drawColor,line width= 0.4pt,line join=round,line cap=round,fill=fillColor] (478.08,535.03) circle (  1.16);

\path[draw=drawColor,line width= 0.4pt,line join=round,line cap=round,fill=fillColor] (478.21,535.03) circle (  1.16);
\definecolor[named]{drawColor}{rgb}{1.00,0.50,0.00}
\definecolor[named]{fillColor}{rgb}{1.00,0.50,0.00}

\path[draw=drawColor,line width= 0.4pt,line join=round,line cap=round,fill=fillColor] (327.82,608.45) circle (  1.16);

\path[draw=drawColor,line width= 0.4pt,line join=round,line cap=round,fill=fillColor] (333.62,602.83) circle (  1.16);

\path[draw=drawColor,line width= 0.4pt,line join=round,line cap=round,fill=fillColor] (337.68,602.49) circle (  1.16);

\path[draw=drawColor,line width= 0.4pt,line join=round,line cap=round,fill=fillColor] (340.92,601.91) circle (  1.16);

\path[draw=drawColor,line width= 0.4pt,line join=round,line cap=round,fill=fillColor] (343.65,600.97) circle (  1.16);

\path[draw=drawColor,line width= 0.4pt,line join=round,line cap=round,fill=fillColor] (346.04,600.72) circle (  1.16);

\path[draw=drawColor,line width= 0.4pt,line join=round,line cap=round,fill=fillColor] (348.17,600.65) circle (  1.16);

\path[draw=drawColor,line width= 0.4pt,line join=round,line cap=round,fill=fillColor] (350.11,600.00) circle (  1.16);

\path[draw=drawColor,line width= 0.4pt,line join=round,line cap=round,fill=fillColor] (351.90,599.77) circle (  1.16);

\path[draw=drawColor,line width= 0.4pt,line join=round,line cap=round,fill=fillColor] (353.55,599.37) circle (  1.16);

\path[draw=drawColor,line width= 0.4pt,line join=round,line cap=round,fill=fillColor] (355.10,599.00) circle (  1.16);

\path[draw=drawColor,line width= 0.4pt,line join=round,line cap=round,fill=fillColor] (356.56,598.82) circle (  1.16);

\path[draw=drawColor,line width= 0.4pt,line join=round,line cap=round,fill=fillColor] (357.94,598.75) circle (  1.16);

\path[draw=drawColor,line width= 0.4pt,line join=round,line cap=round,fill=fillColor] (359.25,598.60) circle (  1.16);

\path[draw=drawColor,line width= 0.4pt,line join=round,line cap=round,fill=fillColor] (360.50,598.34) circle (  1.16);

\path[draw=drawColor,line width= 0.4pt,line join=round,line cap=round,fill=fillColor] (361.70,597.87) circle (  1.16);

\path[draw=drawColor,line width= 0.4pt,line join=round,line cap=round,fill=fillColor] (362.84,597.02) circle (  1.16);

\path[draw=drawColor,line width= 0.4pt,line join=round,line cap=round,fill=fillColor] (363.95,596.90) circle (  1.16);

\path[draw=drawColor,line width= 0.4pt,line join=round,line cap=round,fill=fillColor] (365.01,595.77) circle (  1.16);

\path[draw=drawColor,line width= 0.4pt,line join=round,line cap=round,fill=fillColor] (366.03,594.39) circle (  1.16);

\path[draw=drawColor,line width= 0.4pt,line join=round,line cap=round,fill=fillColor] (367.03,594.20) circle (  1.16);

\path[draw=drawColor,line width= 0.4pt,line join=round,line cap=round,fill=fillColor] (367.99,593.43) circle (  1.16);

\path[draw=drawColor,line width= 0.4pt,line join=round,line cap=round,fill=fillColor] (368.92,592.90) circle (  1.16);

\path[draw=drawColor,line width= 0.4pt,line join=round,line cap=round,fill=fillColor] (369.83,592.64) circle (  1.16);

\path[draw=drawColor,line width= 0.4pt,line join=round,line cap=round,fill=fillColor] (370.71,592.35) circle (  1.16);

\path[draw=drawColor,line width= 0.4pt,line join=round,line cap=round,fill=fillColor] (371.56,591.97) circle (  1.16);

\path[draw=drawColor,line width= 0.4pt,line join=round,line cap=round,fill=fillColor] (372.40,591.67) circle (  1.16);

\path[draw=drawColor,line width= 0.4pt,line join=round,line cap=round,fill=fillColor] (373.21,590.69) circle (  1.16);

\path[draw=drawColor,line width= 0.4pt,line join=round,line cap=round,fill=fillColor] (374.01,590.56) circle (  1.16);

\path[draw=drawColor,line width= 0.4pt,line join=round,line cap=round,fill=fillColor] (374.79,590.47) circle (  1.16);

\path[draw=drawColor,line width= 0.4pt,line join=round,line cap=round,fill=fillColor] (375.55,590.14) circle (  1.16);

\path[draw=drawColor,line width= 0.4pt,line join=round,line cap=round,fill=fillColor] (376.30,589.85) circle (  1.16);

\path[draw=drawColor,line width= 0.4pt,line join=round,line cap=round,fill=fillColor] (377.02,588.95) circle (  1.16);

\path[draw=drawColor,line width= 0.4pt,line join=round,line cap=round,fill=fillColor] (377.74,588.74) circle (  1.16);

\path[draw=drawColor,line width= 0.4pt,line join=round,line cap=round,fill=fillColor] (378.44,588.60) circle (  1.16);

\path[draw=drawColor,line width= 0.4pt,line join=round,line cap=round,fill=fillColor] (379.13,588.39) circle (  1.16);

\path[draw=drawColor,line width= 0.4pt,line join=round,line cap=round,fill=fillColor] (379.80,588.19) circle (  1.16);

\path[draw=drawColor,line width= 0.4pt,line join=round,line cap=round,fill=fillColor] (380.47,587.08) circle (  1.16);

\path[draw=drawColor,line width= 0.4pt,line join=round,line cap=round,fill=fillColor] (381.12,586.86) circle (  1.16);

\path[draw=drawColor,line width= 0.4pt,line join=round,line cap=round,fill=fillColor] (381.76,586.71) circle (  1.16);

\path[draw=drawColor,line width= 0.4pt,line join=round,line cap=round,fill=fillColor] (382.39,586.50) circle (  1.16);

\path[draw=drawColor,line width= 0.4pt,line join=round,line cap=round,fill=fillColor] (383.01,586.43) circle (  1.16);

\path[draw=drawColor,line width= 0.4pt,line join=round,line cap=round,fill=fillColor] (383.62,586.36) circle (  1.16);

\path[draw=drawColor,line width= 0.4pt,line join=round,line cap=round,fill=fillColor] (384.22,586.20) circle (  1.16);

\path[draw=drawColor,line width= 0.4pt,line join=round,line cap=round,fill=fillColor] (384.81,586.07) circle (  1.16);

\path[draw=drawColor,line width= 0.4pt,line join=round,line cap=round,fill=fillColor] (385.39,586.07) circle (  1.16);

\path[draw=drawColor,line width= 0.4pt,line join=round,line cap=round,fill=fillColor] (385.97,586.01) circle (  1.16);

\path[draw=drawColor,line width= 0.4pt,line join=round,line cap=round,fill=fillColor] (386.54,585.93) circle (  1.16);

\path[draw=drawColor,line width= 0.4pt,line join=round,line cap=round,fill=fillColor] (387.09,585.66) circle (  1.16);

\path[draw=drawColor,line width= 0.4pt,line join=round,line cap=round,fill=fillColor] (387.64,585.59) circle (  1.16);

\path[draw=drawColor,line width= 0.4pt,line join=round,line cap=round,fill=fillColor] (388.19,585.56) circle (  1.16);

\path[draw=drawColor,line width= 0.4pt,line join=round,line cap=round,fill=fillColor] (388.73,585.26) circle (  1.16);

\path[draw=drawColor,line width= 0.4pt,line join=round,line cap=round,fill=fillColor] (389.26,585.20) circle (  1.16);

\path[draw=drawColor,line width= 0.4pt,line join=round,line cap=round,fill=fillColor] (389.78,585.19) circle (  1.16);

\path[draw=drawColor,line width= 0.4pt,line join=round,line cap=round,fill=fillColor] (390.30,585.11) circle (  1.16);

\path[draw=drawColor,line width= 0.4pt,line join=round,line cap=round,fill=fillColor] (390.81,585.09) circle (  1.16);

\path[draw=drawColor,line width= 0.4pt,line join=round,line cap=round,fill=fillColor] (391.31,585.05) circle (  1.16);

\path[draw=drawColor,line width= 0.4pt,line join=round,line cap=round,fill=fillColor] (391.81,585.05) circle (  1.16);

\path[draw=drawColor,line width= 0.4pt,line join=round,line cap=round,fill=fillColor] (392.30,584.65) circle (  1.16);

\path[draw=drawColor,line width= 0.4pt,line join=round,line cap=round,fill=fillColor] (392.79,584.56) circle (  1.16);

\path[draw=drawColor,line width= 0.4pt,line join=round,line cap=round,fill=fillColor] (393.27,584.49) circle (  1.16);

\path[draw=drawColor,line width= 0.4pt,line join=round,line cap=round,fill=fillColor] (393.75,584.22) circle (  1.16);

\path[draw=drawColor,line width= 0.4pt,line join=round,line cap=round,fill=fillColor] (394.22,584.22) circle (  1.16);

\path[draw=drawColor,line width= 0.4pt,line join=round,line cap=round,fill=fillColor] (394.69,584.15) circle (  1.16);

\path[draw=drawColor,line width= 0.4pt,line join=round,line cap=round,fill=fillColor] (395.15,583.98) circle (  1.16);

\path[draw=drawColor,line width= 0.4pt,line join=round,line cap=round,fill=fillColor] (395.61,583.97) circle (  1.16);

\path[draw=drawColor,line width= 0.4pt,line join=round,line cap=round,fill=fillColor] (396.06,583.96) circle (  1.16);

\path[draw=drawColor,line width= 0.4pt,line join=round,line cap=round,fill=fillColor] (396.51,583.64) circle (  1.16);

\path[draw=drawColor,line width= 0.4pt,line join=round,line cap=round,fill=fillColor] (396.95,583.54) circle (  1.16);

\path[draw=drawColor,line width= 0.4pt,line join=round,line cap=round,fill=fillColor] (397.39,583.52) circle (  1.16);

\path[draw=drawColor,line width= 0.4pt,line join=round,line cap=round,fill=fillColor] (397.83,583.48) circle (  1.16);

\path[draw=drawColor,line width= 0.4pt,line join=round,line cap=round,fill=fillColor] (398.26,583.36) circle (  1.16);

\path[draw=drawColor,line width= 0.4pt,line join=round,line cap=round,fill=fillColor] (398.68,583.29) circle (  1.16);

\path[draw=drawColor,line width= 0.4pt,line join=round,line cap=round,fill=fillColor] (399.11,582.94) circle (  1.16);

\path[draw=drawColor,line width= 0.4pt,line join=round,line cap=round,fill=fillColor] (399.53,582.90) circle (  1.16);

\path[draw=drawColor,line width= 0.4pt,line join=round,line cap=round,fill=fillColor] (399.94,582.90) circle (  1.16);

\path[draw=drawColor,line width= 0.4pt,line join=round,line cap=round,fill=fillColor] (400.36,582.66) circle (  1.16);

\path[draw=drawColor,line width= 0.4pt,line join=round,line cap=round,fill=fillColor] (400.76,582.63) circle (  1.16);

\path[draw=drawColor,line width= 0.4pt,line join=round,line cap=round,fill=fillColor] (401.17,582.59) circle (  1.16);

\path[draw=drawColor,line width= 0.4pt,line join=round,line cap=round,fill=fillColor] (401.57,582.28) circle (  1.16);

\path[draw=drawColor,line width= 0.4pt,line join=round,line cap=round,fill=fillColor] (401.97,582.17) circle (  1.16);

\path[draw=drawColor,line width= 0.4pt,line join=round,line cap=round,fill=fillColor] (402.36,582.16) circle (  1.16);

\path[draw=drawColor,line width= 0.4pt,line join=round,line cap=round,fill=fillColor] (402.76,581.90) circle (  1.16);

\path[draw=drawColor,line width= 0.4pt,line join=round,line cap=round,fill=fillColor] (403.15,581.82) circle (  1.16);

\path[draw=drawColor,line width= 0.4pt,line join=round,line cap=round,fill=fillColor] (403.53,581.66) circle (  1.16);

\path[draw=drawColor,line width= 0.4pt,line join=round,line cap=round,fill=fillColor] (403.91,581.48) circle (  1.16);

\path[draw=drawColor,line width= 0.4pt,line join=round,line cap=round,fill=fillColor] (404.29,581.39) circle (  1.16);

\path[draw=drawColor,line width= 0.4pt,line join=round,line cap=round,fill=fillColor] (404.67,581.09) circle (  1.16);

\path[draw=drawColor,line width= 0.4pt,line join=round,line cap=round,fill=fillColor] (405.05,580.93) circle (  1.16);

\path[draw=drawColor,line width= 0.4pt,line join=round,line cap=round,fill=fillColor] (405.42,580.68) circle (  1.16);

\path[draw=drawColor,line width= 0.4pt,line join=round,line cap=round,fill=fillColor] (405.78,580.41) circle (  1.16);

\path[draw=drawColor,line width= 0.4pt,line join=round,line cap=round,fill=fillColor] (406.15,580.33) circle (  1.16);

\path[draw=drawColor,line width= 0.4pt,line join=round,line cap=round,fill=fillColor] (406.51,580.18) circle (  1.16);

\path[draw=drawColor,line width= 0.4pt,line join=round,line cap=round,fill=fillColor] (406.87,580.18) circle (  1.16);

\path[draw=drawColor,line width= 0.4pt,line join=round,line cap=round,fill=fillColor] (407.23,580.03) circle (  1.16);

\path[draw=drawColor,line width= 0.4pt,line join=round,line cap=round,fill=fillColor] (407.59,579.98) circle (  1.16);

\path[draw=drawColor,line width= 0.4pt,line join=round,line cap=round,fill=fillColor] (407.94,579.98) circle (  1.16);

\path[draw=drawColor,line width= 0.4pt,line join=round,line cap=round,fill=fillColor] (408.29,579.77) circle (  1.16);

\path[draw=drawColor,line width= 0.4pt,line join=round,line cap=round,fill=fillColor] (408.64,579.69) circle (  1.16);

\path[draw=drawColor,line width= 0.4pt,line join=round,line cap=round,fill=fillColor] (408.99,579.50) circle (  1.16);

\path[draw=drawColor,line width= 0.4pt,line join=round,line cap=round,fill=fillColor] (409.33,579.41) circle (  1.16);

\path[draw=drawColor,line width= 0.4pt,line join=round,line cap=round,fill=fillColor] (409.67,579.38) circle (  1.16);

\path[draw=drawColor,line width= 0.4pt,line join=round,line cap=round,fill=fillColor] (410.01,579.18) circle (  1.16);

\path[draw=drawColor,line width= 0.4pt,line join=round,line cap=round,fill=fillColor] (410.35,579.16) circle (  1.16);

\path[draw=drawColor,line width= 0.4pt,line join=round,line cap=round,fill=fillColor] (410.68,579.12) circle (  1.16);

\path[draw=drawColor,line width= 0.4pt,line join=round,line cap=round,fill=fillColor] (411.02,579.12) circle (  1.16);

\path[draw=drawColor,line width= 0.4pt,line join=round,line cap=round,fill=fillColor] (411.35,578.72) circle (  1.16);

\path[draw=drawColor,line width= 0.4pt,line join=round,line cap=round,fill=fillColor] (411.67,578.70) circle (  1.16);

\path[draw=drawColor,line width= 0.4pt,line join=round,line cap=round,fill=fillColor] (412.00,578.43) circle (  1.16);

\path[draw=drawColor,line width= 0.4pt,line join=round,line cap=round,fill=fillColor] (412.33,578.42) circle (  1.16);

\path[draw=drawColor,line width= 0.4pt,line join=round,line cap=round,fill=fillColor] (412.65,578.38) circle (  1.16);

\path[draw=drawColor,line width= 0.4pt,line join=round,line cap=round,fill=fillColor] (412.97,578.35) circle (  1.16);

\path[draw=drawColor,line width= 0.4pt,line join=round,line cap=round,fill=fillColor] (413.29,578.22) circle (  1.16);

\path[draw=drawColor,line width= 0.4pt,line join=round,line cap=round,fill=fillColor] (413.60,578.18) circle (  1.16);

\path[draw=drawColor,line width= 0.4pt,line join=round,line cap=round,fill=fillColor] (413.92,578.13) circle (  1.16);

\path[draw=drawColor,line width= 0.4pt,line join=round,line cap=round,fill=fillColor] (414.23,578.09) circle (  1.16);

\path[draw=drawColor,line width= 0.4pt,line join=round,line cap=round,fill=fillColor] (414.54,578.05) circle (  1.16);

\path[draw=drawColor,line width= 0.4pt,line join=round,line cap=round,fill=fillColor] (414.85,578.01) circle (  1.16);

\path[draw=drawColor,line width= 0.4pt,line join=round,line cap=round,fill=fillColor] (415.16,577.96) circle (  1.16);

\path[draw=drawColor,line width= 0.4pt,line join=round,line cap=round,fill=fillColor] (415.47,577.90) circle (  1.16);

\path[draw=drawColor,line width= 0.4pt,line join=round,line cap=round,fill=fillColor] (415.77,577.75) circle (  1.16);

\path[draw=drawColor,line width= 0.4pt,line join=round,line cap=round,fill=fillColor] (416.08,577.63) circle (  1.16);

\path[draw=drawColor,line width= 0.4pt,line join=round,line cap=round,fill=fillColor] (416.38,577.52) circle (  1.16);

\path[draw=drawColor,line width= 0.4pt,line join=round,line cap=round,fill=fillColor] (416.68,577.35) circle (  1.16);

\path[draw=drawColor,line width= 0.4pt,line join=round,line cap=round,fill=fillColor] (416.97,577.30) circle (  1.16);

\path[draw=drawColor,line width= 0.4pt,line join=round,line cap=round,fill=fillColor] (417.27,577.29) circle (  1.16);

\path[draw=drawColor,line width= 0.4pt,line join=round,line cap=round,fill=fillColor] (417.57,577.27) circle (  1.16);

\path[draw=drawColor,line width= 0.4pt,line join=round,line cap=round,fill=fillColor] (417.86,577.13) circle (  1.16);

\path[draw=drawColor,line width= 0.4pt,line join=round,line cap=round,fill=fillColor] (418.15,577.08) circle (  1.16);

\path[draw=drawColor,line width= 0.4pt,line join=round,line cap=round,fill=fillColor] (418.44,577.05) circle (  1.16);

\path[draw=drawColor,line width= 0.4pt,line join=round,line cap=round,fill=fillColor] (418.73,576.95) circle (  1.16);

\path[draw=drawColor,line width= 0.4pt,line join=round,line cap=round,fill=fillColor] (419.02,576.85) circle (  1.16);

\path[draw=drawColor,line width= 0.4pt,line join=round,line cap=round,fill=fillColor] (419.30,576.78) circle (  1.16);

\path[draw=drawColor,line width= 0.4pt,line join=round,line cap=round,fill=fillColor] (419.59,576.77) circle (  1.16);

\path[draw=drawColor,line width= 0.4pt,line join=round,line cap=round,fill=fillColor] (419.87,576.75) circle (  1.16);

\path[draw=drawColor,line width= 0.4pt,line join=round,line cap=round,fill=fillColor] (420.15,576.69) circle (  1.16);

\path[draw=drawColor,line width= 0.4pt,line join=round,line cap=round,fill=fillColor] (420.43,576.52) circle (  1.16);

\path[draw=drawColor,line width= 0.4pt,line join=round,line cap=round,fill=fillColor] (420.71,576.52) circle (  1.16);

\path[draw=drawColor,line width= 0.4pt,line join=round,line cap=round,fill=fillColor] (420.99,576.38) circle (  1.16);

\path[draw=drawColor,line width= 0.4pt,line join=round,line cap=round,fill=fillColor] (421.26,576.21) circle (  1.16);

\path[draw=drawColor,line width= 0.4pt,line join=round,line cap=round,fill=fillColor] (421.54,576.19) circle (  1.16);

\path[draw=drawColor,line width= 0.4pt,line join=round,line cap=round,fill=fillColor] (421.81,576.10) circle (  1.16);

\path[draw=drawColor,line width= 0.4pt,line join=round,line cap=round,fill=fillColor] (422.09,576.05) circle (  1.16);

\path[draw=drawColor,line width= 0.4pt,line join=round,line cap=round,fill=fillColor] (422.36,575.95) circle (  1.16);

\path[draw=drawColor,line width= 0.4pt,line join=round,line cap=round,fill=fillColor] (422.63,575.84) circle (  1.16);

\path[draw=drawColor,line width= 0.4pt,line join=round,line cap=round,fill=fillColor] (422.90,575.82) circle (  1.16);

\path[draw=drawColor,line width= 0.4pt,line join=round,line cap=round,fill=fillColor] (423.16,575.79) circle (  1.16);

\path[draw=drawColor,line width= 0.4pt,line join=round,line cap=round,fill=fillColor] (423.43,575.56) circle (  1.16);

\path[draw=drawColor,line width= 0.4pt,line join=round,line cap=round,fill=fillColor] (423.69,575.52) circle (  1.16);

\path[draw=drawColor,line width= 0.4pt,line join=round,line cap=round,fill=fillColor] (423.96,575.52) circle (  1.16);

\path[draw=drawColor,line width= 0.4pt,line join=round,line cap=round,fill=fillColor] (424.22,575.47) circle (  1.16);

\path[draw=drawColor,line width= 0.4pt,line join=round,line cap=round,fill=fillColor] (424.48,575.43) circle (  1.16);

\path[draw=drawColor,line width= 0.4pt,line join=round,line cap=round,fill=fillColor] (424.74,575.35) circle (  1.16);

\path[draw=drawColor,line width= 0.4pt,line join=round,line cap=round,fill=fillColor] (425.00,575.35) circle (  1.16);

\path[draw=drawColor,line width= 0.4pt,line join=round,line cap=round,fill=fillColor] (425.26,575.28) circle (  1.16);

\path[draw=drawColor,line width= 0.4pt,line join=round,line cap=round,fill=fillColor] (425.52,575.27) circle (  1.16);

\path[draw=drawColor,line width= 0.4pt,line join=round,line cap=round,fill=fillColor] (425.77,575.24) circle (  1.16);

\path[draw=drawColor,line width= 0.4pt,line join=round,line cap=round,fill=fillColor] (426.03,575.15) circle (  1.16);

\path[draw=drawColor,line width= 0.4pt,line join=round,line cap=round,fill=fillColor] (426.28,575.14) circle (  1.16);

\path[draw=drawColor,line width= 0.4pt,line join=round,line cap=round,fill=fillColor] (426.53,575.12) circle (  1.16);

\path[draw=drawColor,line width= 0.4pt,line join=round,line cap=round,fill=fillColor] (426.78,575.09) circle (  1.16);

\path[draw=drawColor,line width= 0.4pt,line join=round,line cap=round,fill=fillColor] (427.03,575.00) circle (  1.16);

\path[draw=drawColor,line width= 0.4pt,line join=round,line cap=round,fill=fillColor] (427.28,574.90) circle (  1.16);

\path[draw=drawColor,line width= 0.4pt,line join=round,line cap=round,fill=fillColor] (427.53,574.76) circle (  1.16);

\path[draw=drawColor,line width= 0.4pt,line join=round,line cap=round,fill=fillColor] (427.78,574.74) circle (  1.16);

\path[draw=drawColor,line width= 0.4pt,line join=round,line cap=round,fill=fillColor] (428.03,574.65) circle (  1.16);

\path[draw=drawColor,line width= 0.4pt,line join=round,line cap=round,fill=fillColor] (428.27,574.59) circle (  1.16);

\path[draw=drawColor,line width= 0.4pt,line join=round,line cap=round,fill=fillColor] (428.52,574.58) circle (  1.16);

\path[draw=drawColor,line width= 0.4pt,line join=round,line cap=round,fill=fillColor] (428.76,574.55) circle (  1.16);

\path[draw=drawColor,line width= 0.4pt,line join=round,line cap=round,fill=fillColor] (429.00,574.55) circle (  1.16);

\path[draw=drawColor,line width= 0.4pt,line join=round,line cap=round,fill=fillColor] (429.24,574.38) circle (  1.16);

\path[draw=drawColor,line width= 0.4pt,line join=round,line cap=round,fill=fillColor] (429.48,574.31) circle (  1.16);

\path[draw=drawColor,line width= 0.4pt,line join=round,line cap=round,fill=fillColor] (429.72,574.27) circle (  1.16);

\path[draw=drawColor,line width= 0.4pt,line join=round,line cap=round,fill=fillColor] (429.96,574.21) circle (  1.16);

\path[draw=drawColor,line width= 0.4pt,line join=round,line cap=round,fill=fillColor] (430.20,574.17) circle (  1.16);

\path[draw=drawColor,line width= 0.4pt,line join=round,line cap=round,fill=fillColor] (430.44,574.12) circle (  1.16);

\path[draw=drawColor,line width= 0.4pt,line join=round,line cap=round,fill=fillColor] (430.67,574.10) circle (  1.16);

\path[draw=drawColor,line width= 0.4pt,line join=round,line cap=round,fill=fillColor] (430.91,574.09) circle (  1.16);

\path[draw=drawColor,line width= 0.4pt,line join=round,line cap=round,fill=fillColor] (431.14,574.08) circle (  1.16);

\path[draw=drawColor,line width= 0.4pt,line join=round,line cap=round,fill=fillColor] (431.38,574.03) circle (  1.16);

\path[draw=drawColor,line width= 0.4pt,line join=round,line cap=round,fill=fillColor] (431.61,573.98) circle (  1.16);

\path[draw=drawColor,line width= 0.4pt,line join=round,line cap=round,fill=fillColor] (431.84,573.97) circle (  1.16);

\path[draw=drawColor,line width= 0.4pt,line join=round,line cap=round,fill=fillColor] (432.07,573.85) circle (  1.16);

\path[draw=drawColor,line width= 0.4pt,line join=round,line cap=round,fill=fillColor] (432.30,573.71) circle (  1.16);

\path[draw=drawColor,line width= 0.4pt,line join=round,line cap=round,fill=fillColor] (432.53,573.69) circle (  1.16);

\path[draw=drawColor,line width= 0.4pt,line join=round,line cap=round,fill=fillColor] (432.76,573.54) circle (  1.16);

\path[draw=drawColor,line width= 0.4pt,line join=round,line cap=round,fill=fillColor] (432.99,573.49) circle (  1.16);

\path[draw=drawColor,line width= 0.4pt,line join=round,line cap=round,fill=fillColor] (433.21,573.48) circle (  1.16);

\path[draw=drawColor,line width= 0.4pt,line join=round,line cap=round,fill=fillColor] (433.44,573.47) circle (  1.16);

\path[draw=drawColor,line width= 0.4pt,line join=round,line cap=round,fill=fillColor] (433.67,573.42) circle (  1.16);

\path[draw=drawColor,line width= 0.4pt,line join=round,line cap=round,fill=fillColor] (433.89,573.42) circle (  1.16);

\path[draw=drawColor,line width= 0.4pt,line join=round,line cap=round,fill=fillColor] (434.11,573.36) circle (  1.16);

\path[draw=drawColor,line width= 0.4pt,line join=round,line cap=round,fill=fillColor] (434.34,573.35) circle (  1.16);

\path[draw=drawColor,line width= 0.4pt,line join=round,line cap=round,fill=fillColor] (434.56,573.34) circle (  1.16);

\path[draw=drawColor,line width= 0.4pt,line join=round,line cap=round,fill=fillColor] (434.78,573.31) circle (  1.16);

\path[draw=drawColor,line width= 0.4pt,line join=round,line cap=round,fill=fillColor] (435.00,573.09) circle (  1.16);

\path[draw=drawColor,line width= 0.4pt,line join=round,line cap=round,fill=fillColor] (435.22,572.98) circle (  1.16);

\path[draw=drawColor,line width= 0.4pt,line join=round,line cap=round,fill=fillColor] (435.44,572.97) circle (  1.16);

\path[draw=drawColor,line width= 0.4pt,line join=round,line cap=round,fill=fillColor] (435.66,572.97) circle (  1.16);

\path[draw=drawColor,line width= 0.4pt,line join=round,line cap=round,fill=fillColor] (435.88,572.96) circle (  1.16);

\path[draw=drawColor,line width= 0.4pt,line join=round,line cap=round,fill=fillColor] (436.09,572.91) circle (  1.16);

\path[draw=drawColor,line width= 0.4pt,line join=round,line cap=round,fill=fillColor] (436.31,572.85) circle (  1.16);

\path[draw=drawColor,line width= 0.4pt,line join=round,line cap=round,fill=fillColor] (436.52,572.81) circle (  1.16);

\path[draw=drawColor,line width= 0.4pt,line join=round,line cap=round,fill=fillColor] (436.74,572.80) circle (  1.16);

\path[draw=drawColor,line width= 0.4pt,line join=round,line cap=round,fill=fillColor] (436.95,572.79) circle (  1.16);

\path[draw=drawColor,line width= 0.4pt,line join=round,line cap=round,fill=fillColor] (437.17,572.57) circle (  1.16);

\path[draw=drawColor,line width= 0.4pt,line join=round,line cap=round,fill=fillColor] (437.38,572.55) circle (  1.16);

\path[draw=drawColor,line width= 0.4pt,line join=round,line cap=round,fill=fillColor] (437.59,572.53) circle (  1.16);

\path[draw=drawColor,line width= 0.4pt,line join=round,line cap=round,fill=fillColor] (437.80,572.52) circle (  1.16);

\path[draw=drawColor,line width= 0.4pt,line join=round,line cap=round,fill=fillColor] (438.01,572.49) circle (  1.16);

\path[draw=drawColor,line width= 0.4pt,line join=round,line cap=round,fill=fillColor] (438.22,572.45) circle (  1.16);

\path[draw=drawColor,line width= 0.4pt,line join=round,line cap=round,fill=fillColor] (438.43,572.45) circle (  1.16);

\path[draw=drawColor,line width= 0.4pt,line join=round,line cap=round,fill=fillColor] (438.64,572.29) circle (  1.16);

\path[draw=drawColor,line width= 0.4pt,line join=round,line cap=round,fill=fillColor] (438.85,572.26) circle (  1.16);

\path[draw=drawColor,line width= 0.4pt,line join=round,line cap=round,fill=fillColor] (439.06,572.19) circle (  1.16);

\path[draw=drawColor,line width= 0.4pt,line join=round,line cap=round,fill=fillColor] (439.26,572.16) circle (  1.16);

\path[draw=drawColor,line width= 0.4pt,line join=round,line cap=round,fill=fillColor] (439.47,572.08) circle (  1.16);

\path[draw=drawColor,line width= 0.4pt,line join=round,line cap=round,fill=fillColor] (439.67,571.90) circle (  1.16);

\path[draw=drawColor,line width= 0.4pt,line join=round,line cap=round,fill=fillColor] (439.88,571.54) circle (  1.16);

\path[draw=drawColor,line width= 0.4pt,line join=round,line cap=round,fill=fillColor] (440.08,571.43) circle (  1.16);

\path[draw=drawColor,line width= 0.4pt,line join=round,line cap=round,fill=fillColor] (440.29,571.41) circle (  1.16);

\path[draw=drawColor,line width= 0.4pt,line join=round,line cap=round,fill=fillColor] (440.49,571.40) circle (  1.16);

\path[draw=drawColor,line width= 0.4pt,line join=round,line cap=round,fill=fillColor] (440.69,571.37) circle (  1.16);

\path[draw=drawColor,line width= 0.4pt,line join=round,line cap=round,fill=fillColor] (440.89,571.36) circle (  1.16);

\path[draw=drawColor,line width= 0.4pt,line join=round,line cap=round,fill=fillColor] (441.09,571.29) circle (  1.16);

\path[draw=drawColor,line width= 0.4pt,line join=round,line cap=round,fill=fillColor] (441.29,571.12) circle (  1.16);

\path[draw=drawColor,line width= 0.4pt,line join=round,line cap=round,fill=fillColor] (441.49,571.01) circle (  1.16);

\path[draw=drawColor,line width= 0.4pt,line join=round,line cap=round,fill=fillColor] (441.69,570.97) circle (  1.16);

\path[draw=drawColor,line width= 0.4pt,line join=round,line cap=round,fill=fillColor] (441.89,570.94) circle (  1.16);

\path[draw=drawColor,line width= 0.4pt,line join=round,line cap=round,fill=fillColor] (442.09,570.93) circle (  1.16);

\path[draw=drawColor,line width= 0.4pt,line join=round,line cap=round,fill=fillColor] (442.29,570.90) circle (  1.16);

\path[draw=drawColor,line width= 0.4pt,line join=round,line cap=round,fill=fillColor] (442.49,570.83) circle (  1.16);

\path[draw=drawColor,line width= 0.4pt,line join=round,line cap=round,fill=fillColor] (442.68,570.82) circle (  1.16);

\path[draw=drawColor,line width= 0.4pt,line join=round,line cap=round,fill=fillColor] (442.88,570.78) circle (  1.16);

\path[draw=drawColor,line width= 0.4pt,line join=round,line cap=round,fill=fillColor] (443.07,570.73) circle (  1.16);

\path[draw=drawColor,line width= 0.4pt,line join=round,line cap=round,fill=fillColor] (443.27,570.66) circle (  1.16);

\path[draw=drawColor,line width= 0.4pt,line join=round,line cap=round,fill=fillColor] (443.46,570.58) circle (  1.16);

\path[draw=drawColor,line width= 0.4pt,line join=round,line cap=round,fill=fillColor] (443.66,570.33) circle (  1.16);

\path[draw=drawColor,line width= 0.4pt,line join=round,line cap=round,fill=fillColor] (443.85,570.33) circle (  1.16);

\path[draw=drawColor,line width= 0.4pt,line join=round,line cap=round,fill=fillColor] (444.04,570.22) circle (  1.16);

\path[draw=drawColor,line width= 0.4pt,line join=round,line cap=round,fill=fillColor] (444.23,570.18) circle (  1.16);

\path[draw=drawColor,line width= 0.4pt,line join=round,line cap=round,fill=fillColor] (444.43,570.17) circle (  1.16);

\path[draw=drawColor,line width= 0.4pt,line join=round,line cap=round,fill=fillColor] (444.62,570.06) circle (  1.16);

\path[draw=drawColor,line width= 0.4pt,line join=round,line cap=round,fill=fillColor] (444.81,570.04) circle (  1.16);

\path[draw=drawColor,line width= 0.4pt,line join=round,line cap=round,fill=fillColor] (445.00,569.95) circle (  1.16);

\path[draw=drawColor,line width= 0.4pt,line join=round,line cap=round,fill=fillColor] (445.19,569.91) circle (  1.16);

\path[draw=drawColor,line width= 0.4pt,line join=round,line cap=round,fill=fillColor] (445.38,569.90) circle (  1.16);

\path[draw=drawColor,line width= 0.4pt,line join=round,line cap=round,fill=fillColor] (445.56,569.85) circle (  1.16);

\path[draw=drawColor,line width= 0.4pt,line join=round,line cap=round,fill=fillColor] (445.75,569.79) circle (  1.16);

\path[draw=drawColor,line width= 0.4pt,line join=round,line cap=round,fill=fillColor] (445.94,569.78) circle (  1.16);

\path[draw=drawColor,line width= 0.4pt,line join=round,line cap=round,fill=fillColor] (446.13,569.78) circle (  1.16);

\path[draw=drawColor,line width= 0.4pt,line join=round,line cap=round,fill=fillColor] (446.31,569.64) circle (  1.16);

\path[draw=drawColor,line width= 0.4pt,line join=round,line cap=round,fill=fillColor] (446.50,569.62) circle (  1.16);

\path[draw=drawColor,line width= 0.4pt,line join=round,line cap=round,fill=fillColor] (446.68,569.58) circle (  1.16);

\path[draw=drawColor,line width= 0.4pt,line join=round,line cap=round,fill=fillColor] (446.87,569.36) circle (  1.16);

\path[draw=drawColor,line width= 0.4pt,line join=round,line cap=round,fill=fillColor] (447.05,569.32) circle (  1.16);

\path[draw=drawColor,line width= 0.4pt,line join=round,line cap=round,fill=fillColor] (447.24,569.30) circle (  1.16);

\path[draw=drawColor,line width= 0.4pt,line join=round,line cap=round,fill=fillColor] (447.42,569.29) circle (  1.16);

\path[draw=drawColor,line width= 0.4pt,line join=round,line cap=round,fill=fillColor] (447.60,569.06) circle (  1.16);

\path[draw=drawColor,line width= 0.4pt,line join=round,line cap=round,fill=fillColor] (447.79,569.04) circle (  1.16);

\path[draw=drawColor,line width= 0.4pt,line join=round,line cap=round,fill=fillColor] (447.97,568.99) circle (  1.16);

\path[draw=drawColor,line width= 0.4pt,line join=round,line cap=round,fill=fillColor] (448.15,568.97) circle (  1.16);

\path[draw=drawColor,line width= 0.4pt,line join=round,line cap=round,fill=fillColor] (448.33,568.85) circle (  1.16);

\path[draw=drawColor,line width= 0.4pt,line join=round,line cap=round,fill=fillColor] (448.51,568.81) circle (  1.16);

\path[draw=drawColor,line width= 0.4pt,line join=round,line cap=round,fill=fillColor] (448.69,568.76) circle (  1.16);

\path[draw=drawColor,line width= 0.4pt,line join=round,line cap=round,fill=fillColor] (448.87,568.73) circle (  1.16);

\path[draw=drawColor,line width= 0.4pt,line join=round,line cap=round,fill=fillColor] (449.05,568.69) circle (  1.16);

\path[draw=drawColor,line width= 0.4pt,line join=round,line cap=round,fill=fillColor] (449.23,568.54) circle (  1.16);

\path[draw=drawColor,line width= 0.4pt,line join=round,line cap=round,fill=fillColor] (449.41,568.52) circle (  1.16);

\path[draw=drawColor,line width= 0.4pt,line join=round,line cap=round,fill=fillColor] (449.59,568.52) circle (  1.16);

\path[draw=drawColor,line width= 0.4pt,line join=round,line cap=round,fill=fillColor] (449.77,568.50) circle (  1.16);

\path[draw=drawColor,line width= 0.4pt,line join=round,line cap=round,fill=fillColor] (449.94,568.48) circle (  1.16);

\path[draw=drawColor,line width= 0.4pt,line join=round,line cap=round,fill=fillColor] (450.12,568.26) circle (  1.16);

\path[draw=drawColor,line width= 0.4pt,line join=round,line cap=round,fill=fillColor] (450.30,568.22) circle (  1.16);

\path[draw=drawColor,line width= 0.4pt,line join=round,line cap=round,fill=fillColor] (450.47,568.20) circle (  1.16);

\path[draw=drawColor,line width= 0.4pt,line join=round,line cap=round,fill=fillColor] (450.65,568.18) circle (  1.16);

\path[draw=drawColor,line width= 0.4pt,line join=round,line cap=round,fill=fillColor] (450.82,568.15) circle (  1.16);

\path[draw=drawColor,line width= 0.4pt,line join=round,line cap=round,fill=fillColor] (451.00,568.10) circle (  1.16);

\path[draw=drawColor,line width= 0.4pt,line join=round,line cap=round,fill=fillColor] (451.17,568.05) circle (  1.16);

\path[draw=drawColor,line width= 0.4pt,line join=round,line cap=round,fill=fillColor] (451.35,567.88) circle (  1.16);

\path[draw=drawColor,line width= 0.4pt,line join=round,line cap=round,fill=fillColor] (451.52,567.79) circle (  1.16);

\path[draw=drawColor,line width= 0.4pt,line join=round,line cap=round,fill=fillColor] (451.69,567.76) circle (  1.16);

\path[draw=drawColor,line width= 0.4pt,line join=round,line cap=round,fill=fillColor] (451.86,567.73) circle (  1.16);

\path[draw=drawColor,line width= 0.4pt,line join=round,line cap=round,fill=fillColor] (452.04,567.71) circle (  1.16);

\path[draw=drawColor,line width= 0.4pt,line join=round,line cap=round,fill=fillColor] (452.21,567.70) circle (  1.16);

\path[draw=drawColor,line width= 0.4pt,line join=round,line cap=round,fill=fillColor] (452.38,567.67) circle (  1.16);

\path[draw=drawColor,line width= 0.4pt,line join=round,line cap=round,fill=fillColor] (452.55,567.60) circle (  1.16);

\path[draw=drawColor,line width= 0.4pt,line join=round,line cap=round,fill=fillColor] (452.72,567.57) circle (  1.16);

\path[draw=drawColor,line width= 0.4pt,line join=round,line cap=round,fill=fillColor] (452.89,567.45) circle (  1.16);

\path[draw=drawColor,line width= 0.4pt,line join=round,line cap=round,fill=fillColor] (453.06,567.44) circle (  1.16);

\path[draw=drawColor,line width= 0.4pt,line join=round,line cap=round,fill=fillColor] (453.23,567.32) circle (  1.16);

\path[draw=drawColor,line width= 0.4pt,line join=round,line cap=round,fill=fillColor] (453.40,567.25) circle (  1.16);

\path[draw=drawColor,line width= 0.4pt,line join=round,line cap=round,fill=fillColor] (453.57,567.22) circle (  1.16);

\path[draw=drawColor,line width= 0.4pt,line join=round,line cap=round,fill=fillColor] (453.74,567.16) circle (  1.16);

\path[draw=drawColor,line width= 0.4pt,line join=round,line cap=round,fill=fillColor] (453.90,567.11) circle (  1.16);

\path[draw=drawColor,line width= 0.4pt,line join=round,line cap=round,fill=fillColor] (454.07,567.08) circle (  1.16);

\path[draw=drawColor,line width= 0.4pt,line join=round,line cap=round,fill=fillColor] (454.24,567.07) circle (  1.16);

\path[draw=drawColor,line width= 0.4pt,line join=round,line cap=round,fill=fillColor] (454.40,567.01) circle (  1.16);

\path[draw=drawColor,line width= 0.4pt,line join=round,line cap=round,fill=fillColor] (454.57,566.98) circle (  1.16);

\path[draw=drawColor,line width= 0.4pt,line join=round,line cap=round,fill=fillColor] (454.74,566.98) circle (  1.16);

\path[draw=drawColor,line width= 0.4pt,line join=round,line cap=round,fill=fillColor] (454.90,566.89) circle (  1.16);

\path[draw=drawColor,line width= 0.4pt,line join=round,line cap=round,fill=fillColor] (455.07,566.87) circle (  1.16);

\path[draw=drawColor,line width= 0.4pt,line join=round,line cap=round,fill=fillColor] (455.23,566.86) circle (  1.16);

\path[draw=drawColor,line width= 0.4pt,line join=round,line cap=round,fill=fillColor] (455.40,566.83) circle (  1.16);

\path[draw=drawColor,line width= 0.4pt,line join=round,line cap=round,fill=fillColor] (455.56,566.75) circle (  1.16);

\path[draw=drawColor,line width= 0.4pt,line join=round,line cap=round,fill=fillColor] (455.73,566.71) circle (  1.16);

\path[draw=drawColor,line width= 0.4pt,line join=round,line cap=round,fill=fillColor] (455.89,566.55) circle (  1.16);

\path[draw=drawColor,line width= 0.4pt,line join=round,line cap=round,fill=fillColor] (456.05,566.51) circle (  1.16);

\path[draw=drawColor,line width= 0.4pt,line join=round,line cap=round,fill=fillColor] (456.21,566.33) circle (  1.16);

\path[draw=drawColor,line width= 0.4pt,line join=round,line cap=round,fill=fillColor] (456.38,566.23) circle (  1.16);

\path[draw=drawColor,line width= 0.4pt,line join=round,line cap=round,fill=fillColor] (456.54,566.22) circle (  1.16);

\path[draw=drawColor,line width= 0.4pt,line join=round,line cap=round,fill=fillColor] (456.70,566.13) circle (  1.16);

\path[draw=drawColor,line width= 0.4pt,line join=round,line cap=round,fill=fillColor] (456.86,566.06) circle (  1.16);

\path[draw=drawColor,line width= 0.4pt,line join=round,line cap=round,fill=fillColor] (457.02,566.02) circle (  1.16);

\path[draw=drawColor,line width= 0.4pt,line join=round,line cap=round,fill=fillColor] (457.18,566.00) circle (  1.16);

\path[draw=drawColor,line width= 0.4pt,line join=round,line cap=round,fill=fillColor] (457.34,565.98) circle (  1.16);

\path[draw=drawColor,line width= 0.4pt,line join=round,line cap=round,fill=fillColor] (457.50,565.97) circle (  1.16);

\path[draw=drawColor,line width= 0.4pt,line join=round,line cap=round,fill=fillColor] (457.66,565.82) circle (  1.16);

\path[draw=drawColor,line width= 0.4pt,line join=round,line cap=round,fill=fillColor] (457.82,565.80) circle (  1.16);

\path[draw=drawColor,line width= 0.4pt,line join=round,line cap=round,fill=fillColor] (457.98,565.79) circle (  1.16);

\path[draw=drawColor,line width= 0.4pt,line join=round,line cap=round,fill=fillColor] (458.14,565.69) circle (  1.16);

\path[draw=drawColor,line width= 0.4pt,line join=round,line cap=round,fill=fillColor] (458.30,565.63) circle (  1.16);

\path[draw=drawColor,line width= 0.4pt,line join=round,line cap=round,fill=fillColor] (458.46,565.59) circle (  1.16);

\path[draw=drawColor,line width= 0.4pt,line join=round,line cap=round,fill=fillColor] (458.61,565.55) circle (  1.16);

\path[draw=drawColor,line width= 0.4pt,line join=round,line cap=round,fill=fillColor] (458.77,565.51) circle (  1.16);

\path[draw=drawColor,line width= 0.4pt,line join=round,line cap=round,fill=fillColor] (458.93,565.48) circle (  1.16);

\path[draw=drawColor,line width= 0.4pt,line join=round,line cap=round,fill=fillColor] (459.09,565.40) circle (  1.16);

\path[draw=drawColor,line width= 0.4pt,line join=round,line cap=round,fill=fillColor] (459.24,565.36) circle (  1.16);

\path[draw=drawColor,line width= 0.4pt,line join=round,line cap=round,fill=fillColor] (459.40,565.21) circle (  1.16);

\path[draw=drawColor,line width= 0.4pt,line join=round,line cap=round,fill=fillColor] (459.55,565.16) circle (  1.16);

\path[draw=drawColor,line width= 0.4pt,line join=round,line cap=round,fill=fillColor] (459.71,565.12) circle (  1.16);

\path[draw=drawColor,line width= 0.4pt,line join=round,line cap=round,fill=fillColor] (459.86,565.06) circle (  1.16);

\path[draw=drawColor,line width= 0.4pt,line join=round,line cap=round,fill=fillColor] (460.02,565.03) circle (  1.16);

\path[draw=drawColor,line width= 0.4pt,line join=round,line cap=round,fill=fillColor] (460.17,564.87) circle (  1.16);

\path[draw=drawColor,line width= 0.4pt,line join=round,line cap=round,fill=fillColor] (460.33,564.80) circle (  1.16);

\path[draw=drawColor,line width= 0.4pt,line join=round,line cap=round,fill=fillColor] (460.48,564.75) circle (  1.16);

\path[draw=drawColor,line width= 0.4pt,line join=round,line cap=round,fill=fillColor] (460.63,564.74) circle (  1.16);

\path[draw=drawColor,line width= 0.4pt,line join=round,line cap=round,fill=fillColor] (460.79,564.58) circle (  1.16);

\path[draw=drawColor,line width= 0.4pt,line join=round,line cap=round,fill=fillColor] (460.94,564.35) circle (  1.16);

\path[draw=drawColor,line width= 0.4pt,line join=round,line cap=round,fill=fillColor] (461.09,564.32) circle (  1.16);

\path[draw=drawColor,line width= 0.4pt,line join=round,line cap=round,fill=fillColor] (461.25,564.31) circle (  1.16);

\path[draw=drawColor,line width= 0.4pt,line join=round,line cap=round,fill=fillColor] (461.40,564.22) circle (  1.16);

\path[draw=drawColor,line width= 0.4pt,line join=round,line cap=round,fill=fillColor] (461.55,564.18) circle (  1.16);

\path[draw=drawColor,line width= 0.4pt,line join=round,line cap=round,fill=fillColor] (461.70,563.98) circle (  1.16);

\path[draw=drawColor,line width= 0.4pt,line join=round,line cap=round,fill=fillColor] (461.85,563.97) circle (  1.16);

\path[draw=drawColor,line width= 0.4pt,line join=round,line cap=round,fill=fillColor] (462.00,563.94) circle (  1.16);

\path[draw=drawColor,line width= 0.4pt,line join=round,line cap=round,fill=fillColor] (462.15,563.94) circle (  1.16);

\path[draw=drawColor,line width= 0.4pt,line join=round,line cap=round,fill=fillColor] (462.30,563.84) circle (  1.16);

\path[draw=drawColor,line width= 0.4pt,line join=round,line cap=round,fill=fillColor] (462.45,563.81) circle (  1.16);

\path[draw=drawColor,line width= 0.4pt,line join=round,line cap=round,fill=fillColor] (462.60,563.80) circle (  1.16);

\path[draw=drawColor,line width= 0.4pt,line join=round,line cap=round,fill=fillColor] (462.75,563.78) circle (  1.16);

\path[draw=drawColor,line width= 0.4pt,line join=round,line cap=round,fill=fillColor] (462.90,563.64) circle (  1.16);

\path[draw=drawColor,line width= 0.4pt,line join=round,line cap=round,fill=fillColor] (463.05,563.64) circle (  1.16);

\path[draw=drawColor,line width= 0.4pt,line join=round,line cap=round,fill=fillColor] (463.20,563.50) circle (  1.16);

\path[draw=drawColor,line width= 0.4pt,line join=round,line cap=round,fill=fillColor] (463.35,563.49) circle (  1.16);

\path[draw=drawColor,line width= 0.4pt,line join=round,line cap=round,fill=fillColor] (463.50,563.43) circle (  1.16);

\path[draw=drawColor,line width= 0.4pt,line join=round,line cap=round,fill=fillColor] (463.64,563.37) circle (  1.16);

\path[draw=drawColor,line width= 0.4pt,line join=round,line cap=round,fill=fillColor] (463.79,563.21) circle (  1.16);

\path[draw=drawColor,line width= 0.4pt,line join=round,line cap=round,fill=fillColor] (463.94,563.17) circle (  1.16);

\path[draw=drawColor,line width= 0.4pt,line join=round,line cap=round,fill=fillColor] (464.09,563.17) circle (  1.16);

\path[draw=drawColor,line width= 0.4pt,line join=round,line cap=round,fill=fillColor] (464.23,563.12) circle (  1.16);

\path[draw=drawColor,line width= 0.4pt,line join=round,line cap=round,fill=fillColor] (464.38,563.07) circle (  1.16);

\path[draw=drawColor,line width= 0.4pt,line join=round,line cap=round,fill=fillColor] (464.53,562.71) circle (  1.16);

\path[draw=drawColor,line width= 0.4pt,line join=round,line cap=round,fill=fillColor] (464.67,562.71) circle (  1.16);

\path[draw=drawColor,line width= 0.4pt,line join=round,line cap=round,fill=fillColor] (464.82,562.71) circle (  1.16);

\path[draw=drawColor,line width= 0.4pt,line join=round,line cap=round,fill=fillColor] (464.96,562.52) circle (  1.16);

\path[draw=drawColor,line width= 0.4pt,line join=round,line cap=round,fill=fillColor] (465.11,562.50) circle (  1.16);

\path[draw=drawColor,line width= 0.4pt,line join=round,line cap=round,fill=fillColor] (465.25,562.39) circle (  1.16);

\path[draw=drawColor,line width= 0.4pt,line join=round,line cap=round,fill=fillColor] (465.40,562.05) circle (  1.16);

\path[draw=drawColor,line width= 0.4pt,line join=round,line cap=round,fill=fillColor] (465.54,562.03) circle (  1.16);

\path[draw=drawColor,line width= 0.4pt,line join=round,line cap=round,fill=fillColor] (465.68,561.69) circle (  1.16);

\path[draw=drawColor,line width= 0.4pt,line join=round,line cap=round,fill=fillColor] (465.83,561.52) circle (  1.16);

\path[draw=drawColor,line width= 0.4pt,line join=round,line cap=round,fill=fillColor] (465.97,561.40) circle (  1.16);

\path[draw=drawColor,line width= 0.4pt,line join=round,line cap=round,fill=fillColor] (466.12,561.37) circle (  1.16);

\path[draw=drawColor,line width= 0.4pt,line join=round,line cap=round,fill=fillColor] (466.26,561.36) circle (  1.16);

\path[draw=drawColor,line width= 0.4pt,line join=round,line cap=round,fill=fillColor] (466.40,561.34) circle (  1.16);

\path[draw=drawColor,line width= 0.4pt,line join=round,line cap=round,fill=fillColor] (466.54,561.12) circle (  1.16);

\path[draw=drawColor,line width= 0.4pt,line join=round,line cap=round,fill=fillColor] (466.69,561.02) circle (  1.16);

\path[draw=drawColor,line width= 0.4pt,line join=round,line cap=round,fill=fillColor] (466.83,560.89) circle (  1.16);

\path[draw=drawColor,line width= 0.4pt,line join=round,line cap=round,fill=fillColor] (466.97,560.79) circle (  1.16);

\path[draw=drawColor,line width= 0.4pt,line join=round,line cap=round,fill=fillColor] (467.11,560.74) circle (  1.16);

\path[draw=drawColor,line width= 0.4pt,line join=round,line cap=round,fill=fillColor] (467.25,560.59) circle (  1.16);

\path[draw=drawColor,line width= 0.4pt,line join=round,line cap=round,fill=fillColor] (467.39,560.32) circle (  1.16);

\path[draw=drawColor,line width= 0.4pt,line join=round,line cap=round,fill=fillColor] (467.53,560.30) circle (  1.16);

\path[draw=drawColor,line width= 0.4pt,line join=round,line cap=round,fill=fillColor] (467.67,560.09) circle (  1.16);

\path[draw=drawColor,line width= 0.4pt,line join=round,line cap=round,fill=fillColor] (467.81,560.05) circle (  1.16);

\path[draw=drawColor,line width= 0.4pt,line join=round,line cap=round,fill=fillColor] (467.95,560.01) circle (  1.16);

\path[draw=drawColor,line width= 0.4pt,line join=round,line cap=round,fill=fillColor] (468.09,560.00) circle (  1.16);

\path[draw=drawColor,line width= 0.4pt,line join=round,line cap=round,fill=fillColor] (468.23,559.92) circle (  1.16);

\path[draw=drawColor,line width= 0.4pt,line join=round,line cap=round,fill=fillColor] (468.37,559.74) circle (  1.16);

\path[draw=drawColor,line width= 0.4pt,line join=round,line cap=round,fill=fillColor] (468.51,559.68) circle (  1.16);

\path[draw=drawColor,line width= 0.4pt,line join=round,line cap=round,fill=fillColor] (468.65,559.38) circle (  1.16);

\path[draw=drawColor,line width= 0.4pt,line join=round,line cap=round,fill=fillColor] (468.79,559.24) circle (  1.16);

\path[draw=drawColor,line width= 0.4pt,line join=round,line cap=round,fill=fillColor] (468.93,559.09) circle (  1.16);

\path[draw=drawColor,line width= 0.4pt,line join=round,line cap=round,fill=fillColor] (469.07,558.77) circle (  1.16);

\path[draw=drawColor,line width= 0.4pt,line join=round,line cap=round,fill=fillColor] (469.20,558.51) circle (  1.16);

\path[draw=drawColor,line width= 0.4pt,line join=round,line cap=round,fill=fillColor] (469.34,558.44) circle (  1.16);

\path[draw=drawColor,line width= 0.4pt,line join=round,line cap=round,fill=fillColor] (469.48,558.02) circle (  1.16);

\path[draw=drawColor,line width= 0.4pt,line join=round,line cap=round,fill=fillColor] (469.62,557.82) circle (  1.16);

\path[draw=drawColor,line width= 0.4pt,line join=round,line cap=round,fill=fillColor] (469.75,557.73) circle (  1.16);

\path[draw=drawColor,line width= 0.4pt,line join=round,line cap=round,fill=fillColor] (469.89,557.18) circle (  1.16);

\path[draw=drawColor,line width= 0.4pt,line join=round,line cap=round,fill=fillColor] (470.03,556.75) circle (  1.16);

\path[draw=drawColor,line width= 0.4pt,line join=round,line cap=round,fill=fillColor] (470.16,556.48) circle (  1.16);

\path[draw=drawColor,line width= 0.4pt,line join=round,line cap=round,fill=fillColor] (470.30,556.42) circle (  1.16);

\path[draw=drawColor,line width= 0.4pt,line join=round,line cap=round,fill=fillColor] (470.43,556.07) circle (  1.16);

\path[draw=drawColor,line width= 0.4pt,line join=round,line cap=round,fill=fillColor] (470.57,555.96) circle (  1.16);

\path[draw=drawColor,line width= 0.4pt,line join=round,line cap=round,fill=fillColor] (470.71,555.83) circle (  1.16);

\path[draw=drawColor,line width= 0.4pt,line join=round,line cap=round,fill=fillColor] (470.84,555.50) circle (  1.16);

\path[draw=drawColor,line width= 0.4pt,line join=round,line cap=round,fill=fillColor] (470.98,555.24) circle (  1.16);

\path[draw=drawColor,line width= 0.4pt,line join=round,line cap=round,fill=fillColor] (471.11,555.21) circle (  1.16);

\path[draw=drawColor,line width= 0.4pt,line join=round,line cap=round,fill=fillColor] (471.24,554.51) circle (  1.16);

\path[draw=drawColor,line width= 0.4pt,line join=round,line cap=round,fill=fillColor] (471.38,554.50) circle (  1.16);

\path[draw=drawColor,line width= 0.4pt,line join=round,line cap=round,fill=fillColor] (471.51,553.97) circle (  1.16);

\path[draw=drawColor,line width= 0.4pt,line join=round,line cap=round,fill=fillColor] (471.65,553.64) circle (  1.16);

\path[draw=drawColor,line width= 0.4pt,line join=round,line cap=round,fill=fillColor] (471.78,553.48) circle (  1.16);

\path[draw=drawColor,line width= 0.4pt,line join=round,line cap=round,fill=fillColor] (471.91,553.40) circle (  1.16);

\path[draw=drawColor,line width= 0.4pt,line join=round,line cap=round,fill=fillColor] (472.05,553.24) circle (  1.16);

\path[draw=drawColor,line width= 0.4pt,line join=round,line cap=round,fill=fillColor] (472.18,553.15) circle (  1.16);

\path[draw=drawColor,line width= 0.4pt,line join=round,line cap=round,fill=fillColor] (472.31,553.00) circle (  1.16);

\path[draw=drawColor,line width= 0.4pt,line join=round,line cap=round,fill=fillColor] (472.45,552.64) circle (  1.16);

\path[draw=drawColor,line width= 0.4pt,line join=round,line cap=round,fill=fillColor] (472.58,552.64) circle (  1.16);

\path[draw=drawColor,line width= 0.4pt,line join=round,line cap=round,fill=fillColor] (472.71,552.22) circle (  1.16);

\path[draw=drawColor,line width= 0.4pt,line join=round,line cap=round,fill=fillColor] (472.84,552.16) circle (  1.16);

\path[draw=drawColor,line width= 0.4pt,line join=round,line cap=round,fill=fillColor] (472.97,551.03) circle (  1.16);

\path[draw=drawColor,line width= 0.4pt,line join=round,line cap=round,fill=fillColor] (473.11,550.57) circle (  1.16);

\path[draw=drawColor,line width= 0.4pt,line join=round,line cap=round,fill=fillColor] (473.24,550.32) circle (  1.16);

\path[draw=drawColor,line width= 0.4pt,line join=round,line cap=round,fill=fillColor] (473.37,550.32) circle (  1.16);

\path[draw=drawColor,line width= 0.4pt,line join=round,line cap=round,fill=fillColor] (473.50,549.88) circle (  1.16);

\path[draw=drawColor,line width= 0.4pt,line join=round,line cap=round,fill=fillColor] (473.63,549.69) circle (  1.16);

\path[draw=drawColor,line width= 0.4pt,line join=round,line cap=round,fill=fillColor] (473.76,549.39) circle (  1.16);

\path[draw=drawColor,line width= 0.4pt,line join=round,line cap=round,fill=fillColor] (473.89,549.00) circle (  1.16);

\path[draw=drawColor,line width= 0.4pt,line join=round,line cap=round,fill=fillColor] (474.02,548.52) circle (  1.16);

\path[draw=drawColor,line width= 0.4pt,line join=round,line cap=round,fill=fillColor] (474.15,548.41) circle (  1.16);

\path[draw=drawColor,line width= 0.4pt,line join=round,line cap=round,fill=fillColor] (474.28,547.88) circle (  1.16);

\path[draw=drawColor,line width= 0.4pt,line join=round,line cap=round,fill=fillColor] (474.41,547.61) circle (  1.16);

\path[draw=drawColor,line width= 0.4pt,line join=round,line cap=round,fill=fillColor] (474.54,546.67) circle (  1.16);

\path[draw=drawColor,line width= 0.4pt,line join=round,line cap=round,fill=fillColor] (474.67,546.44) circle (  1.16);

\path[draw=drawColor,line width= 0.4pt,line join=round,line cap=round,fill=fillColor] (474.80,546.18) circle (  1.16);

\path[draw=drawColor,line width= 0.4pt,line join=round,line cap=round,fill=fillColor] (474.93,545.50) circle (  1.16);

\path[draw=drawColor,line width= 0.4pt,line join=round,line cap=round,fill=fillColor] (475.05,544.95) circle (  1.16);

\path[draw=drawColor,line width= 0.4pt,line join=round,line cap=round,fill=fillColor] (475.18,544.66) circle (  1.16);

\path[draw=drawColor,line width= 0.4pt,line join=round,line cap=round,fill=fillColor] (475.31,535.03) circle (  1.16);

\path[draw=drawColor,line width= 0.4pt,line join=round,line cap=round,fill=fillColor] (475.44,535.03) circle (  1.16);

\path[draw=drawColor,line width= 0.4pt,line join=round,line cap=round,fill=fillColor] (475.57,535.03) circle (  1.16);

\path[draw=drawColor,line width= 0.4pt,line join=round,line cap=round,fill=fillColor] (475.69,535.03) circle (  1.16);

\path[draw=drawColor,line width= 0.4pt,line join=round,line cap=round,fill=fillColor] (475.82,535.03) circle (  1.16);

\path[draw=drawColor,line width= 0.4pt,line join=round,line cap=round,fill=fillColor] (475.95,535.03) circle (  1.16);

\path[draw=drawColor,line width= 0.4pt,line join=round,line cap=round,fill=fillColor] (476.08,535.03) circle (  1.16);

\path[draw=drawColor,line width= 0.4pt,line join=round,line cap=round,fill=fillColor] (476.20,535.03) circle (  1.16);

\path[draw=drawColor,line width= 0.4pt,line join=round,line cap=round,fill=fillColor] (476.33,535.03) circle (  1.16);

\path[draw=drawColor,line width= 0.4pt,line join=round,line cap=round,fill=fillColor] (476.46,535.03) circle (  1.16);

\path[draw=drawColor,line width= 0.4pt,line join=round,line cap=round,fill=fillColor] (476.58,535.03) circle (  1.16);

\path[draw=drawColor,line width= 0.4pt,line join=round,line cap=round,fill=fillColor] (476.71,535.03) circle (  1.16);

\path[draw=drawColor,line width= 0.4pt,line join=round,line cap=round,fill=fillColor] (476.83,535.03) circle (  1.16);

\path[draw=drawColor,line width= 0.4pt,line join=round,line cap=round,fill=fillColor] (476.96,535.03) circle (  1.16);

\path[draw=drawColor,line width= 0.4pt,line join=round,line cap=round,fill=fillColor] (477.08,535.03) circle (  1.16);

\path[draw=drawColor,line width= 0.4pt,line join=round,line cap=round,fill=fillColor] (477.21,535.03) circle (  1.16);

\path[draw=drawColor,line width= 0.4pt,line join=round,line cap=round,fill=fillColor] (477.33,535.03) circle (  1.16);

\path[draw=drawColor,line width= 0.4pt,line join=round,line cap=round,fill=fillColor] (477.46,535.03) circle (  1.16);

\path[draw=drawColor,line width= 0.4pt,line join=round,line cap=round,fill=fillColor] (477.58,535.03) circle (  1.16);

\path[draw=drawColor,line width= 0.4pt,line join=round,line cap=round,fill=fillColor] (477.71,535.03) circle (  1.16);

\path[draw=drawColor,line width= 0.4pt,line join=round,line cap=round,fill=fillColor] (477.83,535.03) circle (  1.16);

\path[draw=drawColor,line width= 0.4pt,line join=round,line cap=round,fill=fillColor] (477.96,535.03) circle (  1.16);

\path[draw=drawColor,line width= 0.4pt,line join=round,line cap=round,fill=fillColor] (478.08,535.03) circle (  1.16);

\path[draw=drawColor,line width= 0.4pt,line join=round,line cap=round,fill=fillColor] (478.21,535.03) circle (  1.16);
\definecolor[named]{drawColor}{rgb}{0.00,0.00,0.00}
\definecolor[named]{fillColor}{rgb}{0.00,0.00,0.00}

\path[draw=drawColor,line width= 0.6pt,line join=round,fill=fillColor] (320.31,617.99) -- (485.72,617.99);

\node[text=drawColor,anchor=base east,inner sep=0pt, outer sep=0pt, scale=  0.85] at (482.22,620.15) {infeasible solutions};

\path[draw=drawColor,line width= 0.6pt,line join=round,line cap=round] (320.31,526.73) rectangle (485.72,628.97);
\end{scope}
\begin{scope}
\path[clip] (  0.00,  0.00) rectangle (505.89,650.43);
\definecolor[named]{drawColor}{rgb}{0.00,0.00,0.00}

\node[text=drawColor,anchor=base east,inner sep=0pt, outer sep=0pt, scale=  0.80] at (314.91,532.27) {0.00};

\node[text=drawColor,anchor=base east,inner sep=0pt, outer sep=0pt, scale=  0.80] at (314.91,550.15) {0.01};

\node[text=drawColor,anchor=base east,inner sep=0pt, outer sep=0pt, scale=  0.80] at (314.91,562.84) {0.05};

\node[text=drawColor,anchor=base east,inner sep=0pt, outer sep=0pt, scale=  0.80] at (314.91,570.78) {0.10};

\node[text=drawColor,anchor=base east,inner sep=0pt, outer sep=0pt, scale=  0.80] at (314.91,580.79) {0.20};

\node[text=drawColor,anchor=base east,inner sep=0pt, outer sep=0pt, scale=  0.80] at (314.91,593.40) {0.40};

\node[text=drawColor,anchor=base east,inner sep=0pt, outer sep=0pt, scale=  0.80] at (314.91,602.25) {0.60};

\node[text=drawColor,anchor=base east,inner sep=0pt, outer sep=0pt, scale=  0.80] at (314.91,609.29) {0.80};

\node[text=drawColor,anchor=base east,inner sep=0pt, outer sep=0pt, scale=  0.80] at (314.91,615.24) {1.00};
\end{scope}
\begin{scope}
\path[clip] (  0.00,  0.00) rectangle (505.89,650.43);
\definecolor[named]{drawColor}{rgb}{0.00,0.00,0.00}

\path[draw=drawColor,line width= 0.6pt,line join=round] (317.31,535.03) --
	(320.31,535.03);

\path[draw=drawColor,line width= 0.6pt,line join=round] (317.31,552.90) --
	(320.31,552.90);

\path[draw=drawColor,line width= 0.6pt,line join=round] (317.31,565.59) --
	(320.31,565.59);

\path[draw=drawColor,line width= 0.6pt,line join=round] (317.31,573.54) --
	(320.31,573.54);

\path[draw=drawColor,line width= 0.6pt,line join=round] (317.31,583.54) --
	(320.31,583.54);

\path[draw=drawColor,line width= 0.6pt,line join=round] (317.31,596.16) --
	(320.31,596.16);

\path[draw=drawColor,line width= 0.6pt,line join=round] (317.31,605.00) --
	(320.31,605.00);

\path[draw=drawColor,line width= 0.6pt,line join=round] (317.31,612.04) --
	(320.31,612.04);

\path[draw=drawColor,line width= 0.6pt,line join=round] (317.31,617.99) --
	(320.31,617.99);
\end{scope}
\begin{scope}
\path[clip] (  0.00,  0.00) rectangle (505.89,650.43);
\definecolor[named]{drawColor}{rgb}{0.00,0.00,0.00}

\path[draw=drawColor,line width= 0.6pt,line join=round] (408.99,523.73) --
	(408.99,526.73);

\path[draw=drawColor,line width= 0.6pt,line join=round] (435.88,523.73) --
	(435.88,526.73);

\path[draw=drawColor,line width= 0.6pt,line join=round] (454.74,523.73) --
	(454.74,526.73);

\path[draw=drawColor,line width= 0.6pt,line join=round] (469.75,523.73) --
	(469.75,526.73);

\path[draw=drawColor,line width= 0.6pt,line join=round] (482.43,523.73) --
	(482.43,526.73);
\end{scope}
\begin{scope}
\path[clip] (  0.00,  0.00) rectangle (505.89,650.43);
\definecolor[named]{drawColor}{rgb}{0.00,0.00,0.00}

\node[text=drawColor,rotate= 50.00,anchor=base east,inner sep=0pt, outer sep=0pt, scale=  0.80] at (413.21,517.79) {100};

\node[text=drawColor,rotate= 50.00,anchor=base east,inner sep=0pt, outer sep=0pt, scale=  0.80] at (440.10,517.79) {200};

\node[text=drawColor,rotate= 50.00,anchor=base east,inner sep=0pt, outer sep=0pt, scale=  0.80] at (458.96,517.79) {300};

\node[text=drawColor,rotate= 50.00,anchor=base east,inner sep=0pt, outer sep=0pt, scale=  0.80] at (473.97,517.79) {400};

\node[text=drawColor,rotate= 50.00,anchor=base east,inner sep=0pt, outer sep=0pt, scale=  0.80] at (486.65,517.79) {500};
\end{scope}
\begin{scope}
\path[clip] (  0.00,  0.00) rectangle (505.89,650.43);
\definecolor[named]{drawColor}{rgb}{0.00,0.00,0.00}

\node[text=drawColor,anchor=base,inner sep=0pt, outer sep=0pt, scale=  1.10] at (403.01,496.22) {\# Instances};
\end{scope}
\begin{scope}
\path[clip] (  0.00,  0.00) rectangle (505.89,650.43);
\definecolor[named]{drawColor}{rgb}{0.00,0.00,0.00}

\node[text=drawColor,rotate= 90.00,anchor=base,inner sep=0pt, outer sep=0pt, scale=  1.10] at (290.69,577.85) {1-(Best/Algorithm)};
\end{scope}
\begin{scope}
\path[clip] (  0.00,  0.00) rectangle (505.89,650.43);
\definecolor[named]{drawColor}{rgb}{0.00,0.00,0.00}

\node[text=drawColor,anchor=base,inner sep=0pt, outer sep=0pt, scale=  1.20] at (403.01,636.17) {$k=4$};
\end{scope}
\begin{scope}
\path[clip] ( 14.17,325.21) rectangle (238.78,487.82);
\definecolor[named]{drawColor}{rgb}{1.00,1.00,1.00}
\definecolor[named]{fillColor}{rgb}{1.00,1.00,1.00}

\path[draw=drawColor,line width= 0.6pt,line join=round,line cap=round,fill=fillColor] ( 14.17,325.21) rectangle (238.78,487.82);
\end{scope}
\begin{scope}
\path[clip] ( 67.36,364.12) rectangle (232.78,466.36);
\definecolor[named]{fillColor}{rgb}{1.00,1.00,1.00}

\path[fill=fillColor] ( 67.36,364.12) rectangle (232.78,466.36);
\definecolor[named]{drawColor}{rgb}{0.98,0.98,0.98}

\path[draw=drawColor,line width= 0.6pt,line join=round] ( 67.36,381.36) --
	(232.78,381.36);

\path[draw=drawColor,line width= 0.6pt,line join=round] ( 67.36,396.64) --
	(232.78,396.64);

\path[draw=drawColor,line width= 0.6pt,line join=round] ( 67.36,406.96) --
	(232.78,406.96);

\path[draw=drawColor,line width= 0.6pt,line join=round] ( 67.36,415.93) --
	(232.78,415.93);

\path[draw=drawColor,line width= 0.6pt,line join=round] ( 67.36,427.24) --
	(232.78,427.24);

\path[draw=drawColor,line width= 0.6pt,line join=round] ( 67.36,437.97) --
	(232.78,437.97);

\path[draw=drawColor,line width= 0.6pt,line join=round] ( 67.36,445.92) --
	(232.78,445.92);

\path[draw=drawColor,line width= 0.6pt,line join=round] ( 67.36,452.41) --
	(232.78,452.41);

\path[draw=drawColor,line width= 0.6pt,line join=round] ( 67.36,464.32) --
	(232.78,464.32);

\path[draw=drawColor,line width= 0.6pt,line join=round] (129.06,364.12) --
	(129.06,466.36);

\path[draw=drawColor,line width= 0.6pt,line join=round] (142.49,364.12) --
	(142.49,466.36);

\path[draw=drawColor,line width= 0.6pt,line join=round] (169.33,364.12) --
	(169.33,466.36);

\path[draw=drawColor,line width= 0.6pt,line join=round] (192.17,364.12) --
	(192.17,466.36);

\path[draw=drawColor,line width= 0.6pt,line join=round] (209.08,364.12) --
	(209.08,466.36);

\path[draw=drawColor,line width= 0.6pt,line join=round] (222.90,364.12) --
	(222.90,466.36);
\definecolor[named]{drawColor}{rgb}{0.75,0.75,0.75}

\path[draw=drawColor,line width= 0.6pt,dash pattern=on 1pt off 3pt ,line join=round] ( 67.36,372.42) --
	(232.78,372.42);

\path[draw=drawColor,line width= 0.6pt,dash pattern=on 1pt off 3pt ,line join=round] ( 67.36,390.29) --
	(232.78,390.29);

\path[draw=drawColor,line width= 0.6pt,dash pattern=on 1pt off 3pt ,line join=round] ( 67.36,402.98) --
	(232.78,402.98);

\path[draw=drawColor,line width= 0.6pt,dash pattern=on 1pt off 3pt ,line join=round] ( 67.36,410.93) --
	(232.78,410.93);

\path[draw=drawColor,line width= 0.6pt,dash pattern=on 1pt off 3pt ,line join=round] ( 67.36,420.94) --
	(232.78,420.94);

\path[draw=drawColor,line width= 0.6pt,dash pattern=on 1pt off 3pt ,line join=round] ( 67.36,433.55) --
	(232.78,433.55);

\path[draw=drawColor,line width= 0.6pt,dash pattern=on 1pt off 3pt ,line join=round] ( 67.36,442.39) --
	(232.78,442.39);

\path[draw=drawColor,line width= 0.6pt,dash pattern=on 1pt off 3pt ,line join=round] ( 67.36,449.44) --
	(232.78,449.44);

\path[draw=drawColor,line width= 0.6pt,dash pattern=on 1pt off 3pt ,line join=round] ( 67.36,455.38) --
	(232.78,455.38);

\path[draw=drawColor,line width= 0.6pt,dash pattern=on 1pt off 3pt ,line join=round] (155.91,364.12) --
	(155.91,466.36);

\path[draw=drawColor,line width= 0.6pt,dash pattern=on 1pt off 3pt ,line join=round] (182.75,364.12) --
	(182.75,466.36);

\path[draw=drawColor,line width= 0.6pt,dash pattern=on 1pt off 3pt ,line join=round] (201.58,364.12) --
	(201.58,466.36);

\path[draw=drawColor,line width= 0.6pt,dash pattern=on 1pt off 3pt ,line join=round] (216.58,364.12) --
	(216.58,466.36);

\path[draw=drawColor,line width= 0.6pt,dash pattern=on 1pt off 3pt ,line join=round] (229.23,364.12) --
	(229.23,466.36);
\definecolor[named]{drawColor}{rgb}{0.89,0.10,0.11}
\definecolor[named]{fillColor}{rgb}{0.89,0.10,0.11}

\path[draw=drawColor,line width= 0.4pt,line join=round,line cap=round,fill=fillColor] ( 74.88,429.56) circle (  1.16);

\path[draw=drawColor,line width= 0.4pt,line join=round,line cap=round,fill=fillColor] ( 80.66,426.34) circle (  1.16);

\path[draw=drawColor,line width= 0.4pt,line join=round,line cap=round,fill=fillColor] ( 84.72,426.17) circle (  1.16);

\path[draw=drawColor,line width= 0.4pt,line join=round,line cap=round,fill=fillColor] ( 87.95,425.74) circle (  1.16);

\path[draw=drawColor,line width= 0.4pt,line join=round,line cap=round,fill=fillColor] ( 90.68,424.77) circle (  1.16);

\path[draw=drawColor,line width= 0.4pt,line join=round,line cap=round,fill=fillColor] ( 93.06,424.11) circle (  1.16);

\path[draw=drawColor,line width= 0.4pt,line join=round,line cap=round,fill=fillColor] ( 95.19,423.26) circle (  1.16);

\path[draw=drawColor,line width= 0.4pt,line join=round,line cap=round,fill=fillColor] ( 97.13,422.91) circle (  1.16);

\path[draw=drawColor,line width= 0.4pt,line join=round,line cap=round,fill=fillColor] ( 98.91,421.89) circle (  1.16);

\path[draw=drawColor,line width= 0.4pt,line join=round,line cap=round,fill=fillColor] (100.57,420.03) circle (  1.16);

\path[draw=drawColor,line width= 0.4pt,line join=round,line cap=round,fill=fillColor] (102.11,418.74) circle (  1.16);

\path[draw=drawColor,line width= 0.4pt,line join=round,line cap=round,fill=fillColor] (103.57,417.86) circle (  1.16);

\path[draw=drawColor,line width= 0.4pt,line join=round,line cap=round,fill=fillColor] (104.95,417.29) circle (  1.16);

\path[draw=drawColor,line width= 0.4pt,line join=round,line cap=round,fill=fillColor] (106.26,417.23) circle (  1.16);

\path[draw=drawColor,line width= 0.4pt,line join=round,line cap=round,fill=fillColor] (107.50,417.01) circle (  1.16);

\path[draw=drawColor,line width= 0.4pt,line join=round,line cap=round,fill=fillColor] (108.70,416.86) circle (  1.16);

\path[draw=drawColor,line width= 0.4pt,line join=round,line cap=round,fill=fillColor] (109.84,416.70) circle (  1.16);

\path[draw=drawColor,line width= 0.4pt,line join=round,line cap=round,fill=fillColor] (110.94,416.52) circle (  1.16);

\path[draw=drawColor,line width= 0.4pt,line join=round,line cap=round,fill=fillColor] (112.00,416.21) circle (  1.16);

\path[draw=drawColor,line width= 0.4pt,line join=round,line cap=round,fill=fillColor] (113.03,415.29) circle (  1.16);

\path[draw=drawColor,line width= 0.4pt,line join=round,line cap=round,fill=fillColor] (114.02,414.88) circle (  1.16);

\path[draw=drawColor,line width= 0.4pt,line join=round,line cap=round,fill=fillColor] (114.98,414.22) circle (  1.16);

\path[draw=drawColor,line width= 0.4pt,line join=round,line cap=round,fill=fillColor] (115.91,413.52) circle (  1.16);

\path[draw=drawColor,line width= 0.4pt,line join=round,line cap=round,fill=fillColor] (116.81,412.53) circle (  1.16);

\path[draw=drawColor,line width= 0.4pt,line join=round,line cap=round,fill=fillColor] (117.69,411.97) circle (  1.16);

\path[draw=drawColor,line width= 0.4pt,line join=round,line cap=round,fill=fillColor] (118.55,411.97) circle (  1.16);

\path[draw=drawColor,line width= 0.4pt,line join=round,line cap=round,fill=fillColor] (119.38,411.89) circle (  1.16);

\path[draw=drawColor,line width= 0.4pt,line join=round,line cap=round,fill=fillColor] (120.20,410.72) circle (  1.16);

\path[draw=drawColor,line width= 0.4pt,line join=round,line cap=round,fill=fillColor] (120.99,410.56) circle (  1.16);

\path[draw=drawColor,line width= 0.4pt,line join=round,line cap=round,fill=fillColor] (121.77,410.51) circle (  1.16);

\path[draw=drawColor,line width= 0.4pt,line join=round,line cap=round,fill=fillColor] (122.53,410.41) circle (  1.16);

\path[draw=drawColor,line width= 0.4pt,line join=round,line cap=round,fill=fillColor] (123.27,410.38) circle (  1.16);

\path[draw=drawColor,line width= 0.4pt,line join=round,line cap=round,fill=fillColor] (124.00,409.96) circle (  1.16);

\path[draw=drawColor,line width= 0.4pt,line join=round,line cap=round,fill=fillColor] (124.71,409.95) circle (  1.16);

\path[draw=drawColor,line width= 0.4pt,line join=round,line cap=round,fill=fillColor] (125.41,409.48) circle (  1.16);

\path[draw=drawColor,line width= 0.4pt,line join=round,line cap=round,fill=fillColor] (126.10,409.40) circle (  1.16);

\path[draw=drawColor,line width= 0.4pt,line join=round,line cap=round,fill=fillColor] (126.77,409.04) circle (  1.16);

\path[draw=drawColor,line width= 0.4pt,line join=round,line cap=round,fill=fillColor] (127.44,409.01) circle (  1.16);

\path[draw=drawColor,line width= 0.4pt,line join=round,line cap=round,fill=fillColor] (128.09,408.73) circle (  1.16);

\path[draw=drawColor,line width= 0.4pt,line join=round,line cap=round,fill=fillColor] (128.73,408.67) circle (  1.16);

\path[draw=drawColor,line width= 0.4pt,line join=round,line cap=round,fill=fillColor] (129.35,408.51) circle (  1.16);

\path[draw=drawColor,line width= 0.4pt,line join=round,line cap=round,fill=fillColor] (129.97,408.50) circle (  1.16);

\path[draw=drawColor,line width= 0.4pt,line join=round,line cap=round,fill=fillColor] (130.58,408.46) circle (  1.16);

\path[draw=drawColor,line width= 0.4pt,line join=round,line cap=round,fill=fillColor] (131.18,408.44) circle (  1.16);

\path[draw=drawColor,line width= 0.4pt,line join=round,line cap=round,fill=fillColor] (131.77,408.40) circle (  1.16);

\path[draw=drawColor,line width= 0.4pt,line join=round,line cap=round,fill=fillColor] (132.35,408.29) circle (  1.16);

\path[draw=drawColor,line width= 0.4pt,line join=round,line cap=round,fill=fillColor] (132.93,408.16) circle (  1.16);

\path[draw=drawColor,line width= 0.4pt,line join=round,line cap=round,fill=fillColor] (133.49,408.10) circle (  1.16);

\path[draw=drawColor,line width= 0.4pt,line join=round,line cap=round,fill=fillColor] (134.05,407.83) circle (  1.16);

\path[draw=drawColor,line width= 0.4pt,line join=round,line cap=round,fill=fillColor] (134.60,407.78) circle (  1.16);

\path[draw=drawColor,line width= 0.4pt,line join=round,line cap=round,fill=fillColor] (135.14,407.67) circle (  1.16);

\path[draw=drawColor,line width= 0.4pt,line join=round,line cap=round,fill=fillColor] (135.68,407.60) circle (  1.16);

\path[draw=drawColor,line width= 0.4pt,line join=round,line cap=round,fill=fillColor] (136.21,407.46) circle (  1.16);

\path[draw=drawColor,line width= 0.4pt,line join=round,line cap=round,fill=fillColor] (136.73,407.42) circle (  1.16);

\path[draw=drawColor,line width= 0.4pt,line join=round,line cap=round,fill=fillColor] (137.25,407.07) circle (  1.16);

\path[draw=drawColor,line width= 0.4pt,line join=round,line cap=round,fill=fillColor] (137.76,406.94) circle (  1.16);

\path[draw=drawColor,line width= 0.4pt,line join=round,line cap=round,fill=fillColor] (138.26,406.47) circle (  1.16);

\path[draw=drawColor,line width= 0.4pt,line join=round,line cap=round,fill=fillColor] (138.76,406.40) circle (  1.16);

\path[draw=drawColor,line width= 0.4pt,line join=round,line cap=round,fill=fillColor] (139.25,406.08) circle (  1.16);

\path[draw=drawColor,line width= 0.4pt,line join=round,line cap=round,fill=fillColor] (139.74,405.90) circle (  1.16);

\path[draw=drawColor,line width= 0.4pt,line join=round,line cap=round,fill=fillColor] (140.22,405.37) circle (  1.16);

\path[draw=drawColor,line width= 0.4pt,line join=round,line cap=round,fill=fillColor] (140.70,405.36) circle (  1.16);

\path[draw=drawColor,line width= 0.4pt,line join=round,line cap=round,fill=fillColor] (141.17,405.26) circle (  1.16);

\path[draw=drawColor,line width= 0.4pt,line join=round,line cap=round,fill=fillColor] (141.63,405.19) circle (  1.16);

\path[draw=drawColor,line width= 0.4pt,line join=round,line cap=round,fill=fillColor] (142.09,405.13) circle (  1.16);

\path[draw=drawColor,line width= 0.4pt,line join=round,line cap=round,fill=fillColor] (142.55,405.12) circle (  1.16);

\path[draw=drawColor,line width= 0.4pt,line join=round,line cap=round,fill=fillColor] (143.00,405.11) circle (  1.16);

\path[draw=drawColor,line width= 0.4pt,line join=round,line cap=round,fill=fillColor] (143.45,405.08) circle (  1.16);

\path[draw=drawColor,line width= 0.4pt,line join=round,line cap=round,fill=fillColor] (143.89,405.05) circle (  1.16);

\path[draw=drawColor,line width= 0.4pt,line join=round,line cap=round,fill=fillColor] (144.33,404.92) circle (  1.16);

\path[draw=drawColor,line width= 0.4pt,line join=round,line cap=round,fill=fillColor] (144.77,404.66) circle (  1.16);

\path[draw=drawColor,line width= 0.4pt,line join=round,line cap=round,fill=fillColor] (145.20,404.59) circle (  1.16);

\path[draw=drawColor,line width= 0.4pt,line join=round,line cap=round,fill=fillColor] (145.62,404.24) circle (  1.16);

\path[draw=drawColor,line width= 0.4pt,line join=round,line cap=round,fill=fillColor] (146.05,403.98) circle (  1.16);

\path[draw=drawColor,line width= 0.4pt,line join=round,line cap=round,fill=fillColor] (146.46,403.81) circle (  1.16);

\path[draw=drawColor,line width= 0.4pt,line join=round,line cap=round,fill=fillColor] (146.88,403.79) circle (  1.16);

\path[draw=drawColor,line width= 0.4pt,line join=round,line cap=round,fill=fillColor] (147.29,403.78) circle (  1.16);

\path[draw=drawColor,line width= 0.4pt,line join=round,line cap=round,fill=fillColor] (147.70,403.77) circle (  1.16);

\path[draw=drawColor,line width= 0.4pt,line join=round,line cap=round,fill=fillColor] (148.10,403.74) circle (  1.16);

\path[draw=drawColor,line width= 0.4pt,line join=round,line cap=round,fill=fillColor] (148.50,403.69) circle (  1.16);

\path[draw=drawColor,line width= 0.4pt,line join=round,line cap=round,fill=fillColor] (148.90,403.58) circle (  1.16);

\path[draw=drawColor,line width= 0.4pt,line join=round,line cap=round,fill=fillColor] (149.30,403.45) circle (  1.16);

\path[draw=drawColor,line width= 0.4pt,line join=round,line cap=round,fill=fillColor] (149.69,403.40) circle (  1.16);

\path[draw=drawColor,line width= 0.4pt,line join=round,line cap=round,fill=fillColor] (150.08,403.39) circle (  1.16);

\path[draw=drawColor,line width= 0.4pt,line join=round,line cap=round,fill=fillColor] (150.46,403.37) circle (  1.16);

\path[draw=drawColor,line width= 0.4pt,line join=round,line cap=round,fill=fillColor] (150.84,403.33) circle (  1.16);

\path[draw=drawColor,line width= 0.4pt,line join=round,line cap=round,fill=fillColor] (151.22,403.29) circle (  1.16);

\path[draw=drawColor,line width= 0.4pt,line join=round,line cap=round,fill=fillColor] (151.60,403.09) circle (  1.16);

\path[draw=drawColor,line width= 0.4pt,line join=round,line cap=round,fill=fillColor] (151.97,403.04) circle (  1.16);

\path[draw=drawColor,line width= 0.4pt,line join=round,line cap=round,fill=fillColor] (152.34,402.92) circle (  1.16);

\path[draw=drawColor,line width= 0.4pt,line join=round,line cap=round,fill=fillColor] (152.71,402.86) circle (  1.16);

\path[draw=drawColor,line width= 0.4pt,line join=round,line cap=round,fill=fillColor] (153.08,402.79) circle (  1.16);

\path[draw=drawColor,line width= 0.4pt,line join=round,line cap=round,fill=fillColor] (153.44,402.77) circle (  1.16);

\path[draw=drawColor,line width= 0.4pt,line join=round,line cap=round,fill=fillColor] (153.80,402.68) circle (  1.16);

\path[draw=drawColor,line width= 0.4pt,line join=round,line cap=round,fill=fillColor] (154.16,402.58) circle (  1.16);

\path[draw=drawColor,line width= 0.4pt,line join=round,line cap=round,fill=fillColor] (154.51,402.56) circle (  1.16);

\path[draw=drawColor,line width= 0.4pt,line join=round,line cap=round,fill=fillColor] (154.86,402.54) circle (  1.16);

\path[draw=drawColor,line width= 0.4pt,line join=round,line cap=round,fill=fillColor] (155.22,402.52) circle (  1.16);

\path[draw=drawColor,line width= 0.4pt,line join=round,line cap=round,fill=fillColor] (155.56,402.51) circle (  1.16);

\path[draw=drawColor,line width= 0.4pt,line join=round,line cap=round,fill=fillColor] (155.91,402.50) circle (  1.16);

\path[draw=drawColor,line width= 0.4pt,line join=round,line cap=round,fill=fillColor] (156.25,402.42) circle (  1.16);

\path[draw=drawColor,line width= 0.4pt,line join=round,line cap=round,fill=fillColor] (156.59,402.40) circle (  1.16);

\path[draw=drawColor,line width= 0.4pt,line join=round,line cap=round,fill=fillColor] (156.93,402.36) circle (  1.16);

\path[draw=drawColor,line width= 0.4pt,line join=round,line cap=round,fill=fillColor] (157.27,402.20) circle (  1.16);

\path[draw=drawColor,line width= 0.4pt,line join=round,line cap=round,fill=fillColor] (157.60,402.09) circle (  1.16);

\path[draw=drawColor,line width= 0.4pt,line join=round,line cap=round,fill=fillColor] (157.93,401.93) circle (  1.16);

\path[draw=drawColor,line width= 0.4pt,line join=round,line cap=round,fill=fillColor] (158.26,401.92) circle (  1.16);

\path[draw=drawColor,line width= 0.4pt,line join=round,line cap=round,fill=fillColor] (158.59,401.92) circle (  1.16);

\path[draw=drawColor,line width= 0.4pt,line join=round,line cap=round,fill=fillColor] (158.92,401.85) circle (  1.16);

\path[draw=drawColor,line width= 0.4pt,line join=round,line cap=round,fill=fillColor] (159.24,401.79) circle (  1.16);

\path[draw=drawColor,line width= 0.4pt,line join=round,line cap=round,fill=fillColor] (159.56,401.59) circle (  1.16);

\path[draw=drawColor,line width= 0.4pt,line join=round,line cap=round,fill=fillColor] (159.88,401.58) circle (  1.16);

\path[draw=drawColor,line width= 0.4pt,line join=round,line cap=round,fill=fillColor] (160.20,401.53) circle (  1.16);

\path[draw=drawColor,line width= 0.4pt,line join=round,line cap=round,fill=fillColor] (160.52,401.52) circle (  1.16);

\path[draw=drawColor,line width= 0.4pt,line join=round,line cap=round,fill=fillColor] (160.83,401.50) circle (  1.16);

\path[draw=drawColor,line width= 0.4pt,line join=round,line cap=round,fill=fillColor] (161.15,401.50) circle (  1.16);

\path[draw=drawColor,line width= 0.4pt,line join=round,line cap=round,fill=fillColor] (161.46,401.44) circle (  1.16);

\path[draw=drawColor,line width= 0.4pt,line join=round,line cap=round,fill=fillColor] (161.77,401.40) circle (  1.16);

\path[draw=drawColor,line width= 0.4pt,line join=round,line cap=round,fill=fillColor] (162.07,401.28) circle (  1.16);

\path[draw=drawColor,line width= 0.4pt,line join=round,line cap=round,fill=fillColor] (162.38,401.08) circle (  1.16);

\path[draw=drawColor,line width= 0.4pt,line join=round,line cap=round,fill=fillColor] (162.68,400.98) circle (  1.16);

\path[draw=drawColor,line width= 0.4pt,line join=round,line cap=round,fill=fillColor] (162.99,400.96) circle (  1.16);

\path[draw=drawColor,line width= 0.4pt,line join=round,line cap=round,fill=fillColor] (163.29,400.92) circle (  1.16);

\path[draw=drawColor,line width= 0.4pt,line join=round,line cap=round,fill=fillColor] (163.59,400.83) circle (  1.16);

\path[draw=drawColor,line width= 0.4pt,line join=round,line cap=round,fill=fillColor] (163.88,400.78) circle (  1.16);

\path[draw=drawColor,line width= 0.4pt,line join=round,line cap=round,fill=fillColor] (164.18,400.78) circle (  1.16);

\path[draw=drawColor,line width= 0.4pt,line join=round,line cap=round,fill=fillColor] (164.47,400.59) circle (  1.16);

\path[draw=drawColor,line width= 0.4pt,line join=round,line cap=round,fill=fillColor] (164.77,400.20) circle (  1.16);

\path[draw=drawColor,line width= 0.4pt,line join=round,line cap=round,fill=fillColor] (165.06,400.11) circle (  1.16);

\path[draw=drawColor,line width= 0.4pt,line join=round,line cap=round,fill=fillColor] (165.35,400.04) circle (  1.16);

\path[draw=drawColor,line width= 0.4pt,line join=round,line cap=round,fill=fillColor] (165.64,400.01) circle (  1.16);

\path[draw=drawColor,line width= 0.4pt,line join=round,line cap=round,fill=fillColor] (165.92,399.96) circle (  1.16);

\path[draw=drawColor,line width= 0.4pt,line join=round,line cap=round,fill=fillColor] (166.21,399.94) circle (  1.16);

\path[draw=drawColor,line width= 0.4pt,line join=round,line cap=round,fill=fillColor] (166.49,399.87) circle (  1.16);

\path[draw=drawColor,line width= 0.4pt,line join=round,line cap=round,fill=fillColor] (166.77,399.84) circle (  1.16);

\path[draw=drawColor,line width= 0.4pt,line join=round,line cap=round,fill=fillColor] (167.06,399.79) circle (  1.16);

\path[draw=drawColor,line width= 0.4pt,line join=round,line cap=round,fill=fillColor] (167.34,399.71) circle (  1.16);

\path[draw=drawColor,line width= 0.4pt,line join=round,line cap=round,fill=fillColor] (167.61,399.64) circle (  1.16);

\path[draw=drawColor,line width= 0.4pt,line join=round,line cap=round,fill=fillColor] (167.89,399.59) circle (  1.16);

\path[draw=drawColor,line width= 0.4pt,line join=round,line cap=round,fill=fillColor] (168.17,399.49) circle (  1.16);

\path[draw=drawColor,line width= 0.4pt,line join=round,line cap=round,fill=fillColor] (168.44,399.40) circle (  1.16);

\path[draw=drawColor,line width= 0.4pt,line join=round,line cap=round,fill=fillColor] (168.71,399.39) circle (  1.16);

\path[draw=drawColor,line width= 0.4pt,line join=round,line cap=round,fill=fillColor] (168.99,399.30) circle (  1.16);

\path[draw=drawColor,line width= 0.4pt,line join=round,line cap=round,fill=fillColor] (169.26,399.27) circle (  1.16);

\path[draw=drawColor,line width= 0.4pt,line join=round,line cap=round,fill=fillColor] (169.53,399.27) circle (  1.16);

\path[draw=drawColor,line width= 0.4pt,line join=round,line cap=round,fill=fillColor] (169.79,399.25) circle (  1.16);

\path[draw=drawColor,line width= 0.4pt,line join=round,line cap=round,fill=fillColor] (170.06,399.16) circle (  1.16);

\path[draw=drawColor,line width= 0.4pt,line join=round,line cap=round,fill=fillColor] (170.33,399.11) circle (  1.16);

\path[draw=drawColor,line width= 0.4pt,line join=round,line cap=round,fill=fillColor] (170.59,399.09) circle (  1.16);

\path[draw=drawColor,line width= 0.4pt,line join=round,line cap=round,fill=fillColor] (170.85,399.06) circle (  1.16);

\path[draw=drawColor,line width= 0.4pt,line join=round,line cap=round,fill=fillColor] (171.12,399.01) circle (  1.16);

\path[draw=drawColor,line width= 0.4pt,line join=round,line cap=round,fill=fillColor] (171.38,399.00) circle (  1.16);

\path[draw=drawColor,line width= 0.4pt,line join=round,line cap=round,fill=fillColor] (171.64,398.82) circle (  1.16);

\path[draw=drawColor,line width= 0.4pt,line join=round,line cap=round,fill=fillColor] (171.90,398.74) circle (  1.16);

\path[draw=drawColor,line width= 0.4pt,line join=round,line cap=round,fill=fillColor] (172.15,398.66) circle (  1.16);

\path[draw=drawColor,line width= 0.4pt,line join=round,line cap=round,fill=fillColor] (172.41,398.58) circle (  1.16);

\path[draw=drawColor,line width= 0.4pt,line join=round,line cap=round,fill=fillColor] (172.67,398.57) circle (  1.16);

\path[draw=drawColor,line width= 0.4pt,line join=round,line cap=round,fill=fillColor] (172.92,398.55) circle (  1.16);

\path[draw=drawColor,line width= 0.4pt,line join=round,line cap=round,fill=fillColor] (173.17,398.51) circle (  1.16);

\path[draw=drawColor,line width= 0.4pt,line join=round,line cap=round,fill=fillColor] (173.43,398.50) circle (  1.16);

\path[draw=drawColor,line width= 0.4pt,line join=round,line cap=round,fill=fillColor] (173.68,398.49) circle (  1.16);

\path[draw=drawColor,line width= 0.4pt,line join=round,line cap=round,fill=fillColor] (173.93,398.43) circle (  1.16);

\path[draw=drawColor,line width= 0.4pt,line join=round,line cap=round,fill=fillColor] (174.18,398.39) circle (  1.16);

\path[draw=drawColor,line width= 0.4pt,line join=round,line cap=round,fill=fillColor] (174.42,398.35) circle (  1.16);

\path[draw=drawColor,line width= 0.4pt,line join=round,line cap=round,fill=fillColor] (174.67,398.33) circle (  1.16);

\path[draw=drawColor,line width= 0.4pt,line join=round,line cap=round,fill=fillColor] (174.92,398.30) circle (  1.16);

\path[draw=drawColor,line width= 0.4pt,line join=round,line cap=round,fill=fillColor] (175.16,398.20) circle (  1.16);

\path[draw=drawColor,line width= 0.4pt,line join=round,line cap=round,fill=fillColor] (175.41,398.20) circle (  1.16);

\path[draw=drawColor,line width= 0.4pt,line join=round,line cap=round,fill=fillColor] (175.65,398.18) circle (  1.16);

\path[draw=drawColor,line width= 0.4pt,line join=round,line cap=round,fill=fillColor] (175.89,398.06) circle (  1.16);

\path[draw=drawColor,line width= 0.4pt,line join=round,line cap=round,fill=fillColor] (176.13,398.05) circle (  1.16);

\path[draw=drawColor,line width= 0.4pt,line join=round,line cap=round,fill=fillColor] (176.37,398.02) circle (  1.16);

\path[draw=drawColor,line width= 0.4pt,line join=round,line cap=round,fill=fillColor] (176.61,397.99) circle (  1.16);

\path[draw=drawColor,line width= 0.4pt,line join=round,line cap=round,fill=fillColor] (176.85,397.88) circle (  1.16);

\path[draw=drawColor,line width= 0.4pt,line join=round,line cap=round,fill=fillColor] (177.09,397.87) circle (  1.16);

\path[draw=drawColor,line width= 0.4pt,line join=round,line cap=round,fill=fillColor] (177.32,397.60) circle (  1.16);

\path[draw=drawColor,line width= 0.4pt,line join=round,line cap=round,fill=fillColor] (177.56,397.51) circle (  1.16);

\path[draw=drawColor,line width= 0.4pt,line join=round,line cap=round,fill=fillColor] (177.80,397.50) circle (  1.16);

\path[draw=drawColor,line width= 0.4pt,line join=round,line cap=round,fill=fillColor] (178.03,397.42) circle (  1.16);

\path[draw=drawColor,line width= 0.4pt,line join=round,line cap=round,fill=fillColor] (178.26,397.41) circle (  1.16);

\path[draw=drawColor,line width= 0.4pt,line join=round,line cap=round,fill=fillColor] (178.49,397.40) circle (  1.16);

\path[draw=drawColor,line width= 0.4pt,line join=round,line cap=round,fill=fillColor] (178.73,397.40) circle (  1.16);

\path[draw=drawColor,line width= 0.4pt,line join=round,line cap=round,fill=fillColor] (178.96,397.40) circle (  1.16);

\path[draw=drawColor,line width= 0.4pt,line join=round,line cap=round,fill=fillColor] (179.19,397.20) circle (  1.16);

\path[draw=drawColor,line width= 0.4pt,line join=round,line cap=round,fill=fillColor] (179.41,397.16) circle (  1.16);

\path[draw=drawColor,line width= 0.4pt,line join=round,line cap=round,fill=fillColor] (179.64,397.10) circle (  1.16);

\path[draw=drawColor,line width= 0.4pt,line join=round,line cap=round,fill=fillColor] (179.87,397.04) circle (  1.16);

\path[draw=drawColor,line width= 0.4pt,line join=round,line cap=round,fill=fillColor] (180.10,396.94) circle (  1.16);

\path[draw=drawColor,line width= 0.4pt,line join=round,line cap=round,fill=fillColor] (180.32,396.88) circle (  1.16);

\path[draw=drawColor,line width= 0.4pt,line join=round,line cap=round,fill=fillColor] (180.55,396.86) circle (  1.16);

\path[draw=drawColor,line width= 0.4pt,line join=round,line cap=round,fill=fillColor] (180.77,396.85) circle (  1.16);

\path[draw=drawColor,line width= 0.4pt,line join=round,line cap=round,fill=fillColor] (180.99,396.72) circle (  1.16);

\path[draw=drawColor,line width= 0.4pt,line join=round,line cap=round,fill=fillColor] (181.22,396.72) circle (  1.16);

\path[draw=drawColor,line width= 0.4pt,line join=round,line cap=round,fill=fillColor] (181.44,396.71) circle (  1.16);

\path[draw=drawColor,line width= 0.4pt,line join=round,line cap=round,fill=fillColor] (181.66,396.65) circle (  1.16);

\path[draw=drawColor,line width= 0.4pt,line join=round,line cap=round,fill=fillColor] (181.88,396.55) circle (  1.16);

\path[draw=drawColor,line width= 0.4pt,line join=round,line cap=round,fill=fillColor] (182.10,396.52) circle (  1.16);

\path[draw=drawColor,line width= 0.4pt,line join=round,line cap=round,fill=fillColor] (182.32,396.43) circle (  1.16);

\path[draw=drawColor,line width= 0.4pt,line join=round,line cap=round,fill=fillColor] (182.54,396.38) circle (  1.16);

\path[draw=drawColor,line width= 0.4pt,line join=round,line cap=round,fill=fillColor] (182.75,396.36) circle (  1.16);

\path[draw=drawColor,line width= 0.4pt,line join=round,line cap=round,fill=fillColor] (182.97,396.23) circle (  1.16);

\path[draw=drawColor,line width= 0.4pt,line join=round,line cap=round,fill=fillColor] (183.19,396.22) circle (  1.16);

\path[draw=drawColor,line width= 0.4pt,line join=round,line cap=round,fill=fillColor] (183.40,396.19) circle (  1.16);

\path[draw=drawColor,line width= 0.4pt,line join=round,line cap=round,fill=fillColor] (183.61,396.18) circle (  1.16);

\path[draw=drawColor,line width= 0.4pt,line join=round,line cap=round,fill=fillColor] (183.83,396.18) circle (  1.16);

\path[draw=drawColor,line width= 0.4pt,line join=round,line cap=round,fill=fillColor] (184.04,396.14) circle (  1.16);

\path[draw=drawColor,line width= 0.4pt,line join=round,line cap=round,fill=fillColor] (184.25,396.14) circle (  1.16);

\path[draw=drawColor,line width= 0.4pt,line join=round,line cap=round,fill=fillColor] (184.47,396.06) circle (  1.16);

\path[draw=drawColor,line width= 0.4pt,line join=round,line cap=round,fill=fillColor] (184.68,395.99) circle (  1.16);

\path[draw=drawColor,line width= 0.4pt,line join=round,line cap=round,fill=fillColor] (184.89,395.97) circle (  1.16);

\path[draw=drawColor,line width= 0.4pt,line join=round,line cap=round,fill=fillColor] (185.10,395.90) circle (  1.16);

\path[draw=drawColor,line width= 0.4pt,line join=round,line cap=round,fill=fillColor] (185.31,395.89) circle (  1.16);

\path[draw=drawColor,line width= 0.4pt,line join=round,line cap=round,fill=fillColor] (185.51,395.89) circle (  1.16);

\path[draw=drawColor,line width= 0.4pt,line join=round,line cap=round,fill=fillColor] (185.72,395.87) circle (  1.16);

\path[draw=drawColor,line width= 0.4pt,line join=round,line cap=round,fill=fillColor] (185.93,395.86) circle (  1.16);

\path[draw=drawColor,line width= 0.4pt,line join=round,line cap=round,fill=fillColor] (186.13,395.86) circle (  1.16);

\path[draw=drawColor,line width= 0.4pt,line join=round,line cap=round,fill=fillColor] (186.34,395.84) circle (  1.16);

\path[draw=drawColor,line width= 0.4pt,line join=round,line cap=round,fill=fillColor] (186.55,395.82) circle (  1.16);

\path[draw=drawColor,line width= 0.4pt,line join=round,line cap=round,fill=fillColor] (186.75,395.77) circle (  1.16);

\path[draw=drawColor,line width= 0.4pt,line join=round,line cap=round,fill=fillColor] (186.95,395.71) circle (  1.16);

\path[draw=drawColor,line width= 0.4pt,line join=round,line cap=round,fill=fillColor] (187.16,395.69) circle (  1.16);

\path[draw=drawColor,line width= 0.4pt,line join=round,line cap=round,fill=fillColor] (187.36,395.66) circle (  1.16);

\path[draw=drawColor,line width= 0.4pt,line join=round,line cap=round,fill=fillColor] (187.56,395.63) circle (  1.16);

\path[draw=drawColor,line width= 0.4pt,line join=round,line cap=round,fill=fillColor] (187.76,395.60) circle (  1.16);

\path[draw=drawColor,line width= 0.4pt,line join=round,line cap=round,fill=fillColor] (187.96,395.53) circle (  1.16);

\path[draw=drawColor,line width= 0.4pt,line join=round,line cap=round,fill=fillColor] (188.16,395.50) circle (  1.16);

\path[draw=drawColor,line width= 0.4pt,line join=round,line cap=round,fill=fillColor] (188.36,395.49) circle (  1.16);

\path[draw=drawColor,line width= 0.4pt,line join=round,line cap=round,fill=fillColor] (188.56,395.49) circle (  1.16);

\path[draw=drawColor,line width= 0.4pt,line join=round,line cap=round,fill=fillColor] (188.76,395.41) circle (  1.16);

\path[draw=drawColor,line width= 0.4pt,line join=round,line cap=round,fill=fillColor] (188.96,395.36) circle (  1.16);

\path[draw=drawColor,line width= 0.4pt,line join=round,line cap=round,fill=fillColor] (189.16,395.34) circle (  1.16);

\path[draw=drawColor,line width= 0.4pt,line join=round,line cap=round,fill=fillColor] (189.35,395.23) circle (  1.16);

\path[draw=drawColor,line width= 0.4pt,line join=round,line cap=round,fill=fillColor] (189.55,395.18) circle (  1.16);

\path[draw=drawColor,line width= 0.4pt,line join=round,line cap=round,fill=fillColor] (189.74,395.17) circle (  1.16);

\path[draw=drawColor,line width= 0.4pt,line join=round,line cap=round,fill=fillColor] (189.94,394.98) circle (  1.16);

\path[draw=drawColor,line width= 0.4pt,line join=round,line cap=round,fill=fillColor] (190.13,394.98) circle (  1.16);

\path[draw=drawColor,line width= 0.4pt,line join=round,line cap=round,fill=fillColor] (190.33,394.93) circle (  1.16);

\path[draw=drawColor,line width= 0.4pt,line join=round,line cap=round,fill=fillColor] (190.52,394.77) circle (  1.16);

\path[draw=drawColor,line width= 0.4pt,line join=round,line cap=round,fill=fillColor] (190.71,394.76) circle (  1.16);

\path[draw=drawColor,line width= 0.4pt,line join=round,line cap=round,fill=fillColor] (190.91,394.72) circle (  1.16);

\path[draw=drawColor,line width= 0.4pt,line join=round,line cap=round,fill=fillColor] (191.10,394.66) circle (  1.16);

\path[draw=drawColor,line width= 0.4pt,line join=round,line cap=round,fill=fillColor] (191.29,394.59) circle (  1.16);

\path[draw=drawColor,line width= 0.4pt,line join=round,line cap=round,fill=fillColor] (191.48,394.59) circle (  1.16);

\path[draw=drawColor,line width= 0.4pt,line join=round,line cap=round,fill=fillColor] (191.67,394.54) circle (  1.16);

\path[draw=drawColor,line width= 0.4pt,line join=round,line cap=round,fill=fillColor] (191.86,394.51) circle (  1.16);

\path[draw=drawColor,line width= 0.4pt,line join=round,line cap=round,fill=fillColor] (192.05,394.43) circle (  1.16);

\path[draw=drawColor,line width= 0.4pt,line join=round,line cap=round,fill=fillColor] (192.24,394.30) circle (  1.16);

\path[draw=drawColor,line width= 0.4pt,line join=round,line cap=round,fill=fillColor] (192.43,394.25) circle (  1.16);

\path[draw=drawColor,line width= 0.4pt,line join=round,line cap=round,fill=fillColor] (192.61,394.21) circle (  1.16);

\path[draw=drawColor,line width= 0.4pt,line join=round,line cap=round,fill=fillColor] (192.80,394.13) circle (  1.16);

\path[draw=drawColor,line width= 0.4pt,line join=round,line cap=round,fill=fillColor] (192.99,394.10) circle (  1.16);

\path[draw=drawColor,line width= 0.4pt,line join=round,line cap=round,fill=fillColor] (193.17,394.02) circle (  1.16);

\path[draw=drawColor,line width= 0.4pt,line join=round,line cap=round,fill=fillColor] (193.36,393.99) circle (  1.16);

\path[draw=drawColor,line width= 0.4pt,line join=round,line cap=round,fill=fillColor] (193.54,393.90) circle (  1.16);

\path[draw=drawColor,line width= 0.4pt,line join=round,line cap=round,fill=fillColor] (193.73,393.89) circle (  1.16);

\path[draw=drawColor,line width= 0.4pt,line join=round,line cap=round,fill=fillColor] (193.91,393.84) circle (  1.16);

\path[draw=drawColor,line width= 0.4pt,line join=round,line cap=round,fill=fillColor] (194.10,393.77) circle (  1.16);

\path[draw=drawColor,line width= 0.4pt,line join=round,line cap=round,fill=fillColor] (194.28,393.76) circle (  1.16);

\path[draw=drawColor,line width= 0.4pt,line join=round,line cap=round,fill=fillColor] (194.46,393.75) circle (  1.16);

\path[draw=drawColor,line width= 0.4pt,line join=round,line cap=round,fill=fillColor] (194.65,393.59) circle (  1.16);

\path[draw=drawColor,line width= 0.4pt,line join=round,line cap=round,fill=fillColor] (194.83,393.57) circle (  1.16);

\path[draw=drawColor,line width= 0.4pt,line join=round,line cap=round,fill=fillColor] (195.01,393.48) circle (  1.16);

\path[draw=drawColor,line width= 0.4pt,line join=round,line cap=round,fill=fillColor] (195.19,393.43) circle (  1.16);

\path[draw=drawColor,line width= 0.4pt,line join=round,line cap=round,fill=fillColor] (195.37,393.37) circle (  1.16);

\path[draw=drawColor,line width= 0.4pt,line join=round,line cap=round,fill=fillColor] (195.55,393.33) circle (  1.16);

\path[draw=drawColor,line width= 0.4pt,line join=round,line cap=round,fill=fillColor] (195.73,393.31) circle (  1.16);

\path[draw=drawColor,line width= 0.4pt,line join=round,line cap=round,fill=fillColor] (195.91,393.29) circle (  1.16);

\path[draw=drawColor,line width= 0.4pt,line join=round,line cap=round,fill=fillColor] (196.09,393.21) circle (  1.16);

\path[draw=drawColor,line width= 0.4pt,line join=round,line cap=round,fill=fillColor] (196.27,393.20) circle (  1.16);

\path[draw=drawColor,line width= 0.4pt,line join=round,line cap=round,fill=fillColor] (196.44,393.15) circle (  1.16);

\path[draw=drawColor,line width= 0.4pt,line join=round,line cap=round,fill=fillColor] (196.62,393.05) circle (  1.16);

\path[draw=drawColor,line width= 0.4pt,line join=round,line cap=round,fill=fillColor] (196.80,393.02) circle (  1.16);

\path[draw=drawColor,line width= 0.4pt,line join=round,line cap=round,fill=fillColor] (196.97,392.98) circle (  1.16);

\path[draw=drawColor,line width= 0.4pt,line join=round,line cap=round,fill=fillColor] (197.15,392.95) circle (  1.16);

\path[draw=drawColor,line width= 0.4pt,line join=round,line cap=round,fill=fillColor] (197.33,392.92) circle (  1.16);

\path[draw=drawColor,line width= 0.4pt,line join=round,line cap=round,fill=fillColor] (197.50,392.88) circle (  1.16);

\path[draw=drawColor,line width= 0.4pt,line join=round,line cap=round,fill=fillColor] (197.68,392.86) circle (  1.16);

\path[draw=drawColor,line width= 0.4pt,line join=round,line cap=round,fill=fillColor] (197.85,392.85) circle (  1.16);

\path[draw=drawColor,line width= 0.4pt,line join=round,line cap=round,fill=fillColor] (198.02,392.78) circle (  1.16);

\path[draw=drawColor,line width= 0.4pt,line join=round,line cap=round,fill=fillColor] (198.20,392.76) circle (  1.16);

\path[draw=drawColor,line width= 0.4pt,line join=round,line cap=round,fill=fillColor] (198.37,392.72) circle (  1.16);

\path[draw=drawColor,line width= 0.4pt,line join=round,line cap=round,fill=fillColor] (198.54,392.69) circle (  1.16);

\path[draw=drawColor,line width= 0.4pt,line join=round,line cap=round,fill=fillColor] (198.72,392.69) circle (  1.16);

\path[draw=drawColor,line width= 0.4pt,line join=round,line cap=round,fill=fillColor] (198.89,392.68) circle (  1.16);

\path[draw=drawColor,line width= 0.4pt,line join=round,line cap=round,fill=fillColor] (199.06,392.64) circle (  1.16);

\path[draw=drawColor,line width= 0.4pt,line join=round,line cap=round,fill=fillColor] (199.23,392.40) circle (  1.16);

\path[draw=drawColor,line width= 0.4pt,line join=round,line cap=round,fill=fillColor] (199.40,392.34) circle (  1.16);

\path[draw=drawColor,line width= 0.4pt,line join=round,line cap=round,fill=fillColor] (199.57,392.31) circle (  1.16);

\path[draw=drawColor,line width= 0.4pt,line join=round,line cap=round,fill=fillColor] (199.74,392.29) circle (  1.16);

\path[draw=drawColor,line width= 0.4pt,line join=round,line cap=round,fill=fillColor] (199.91,392.21) circle (  1.16);

\path[draw=drawColor,line width= 0.4pt,line join=round,line cap=round,fill=fillColor] (200.08,392.14) circle (  1.16);

\path[draw=drawColor,line width= 0.4pt,line join=round,line cap=round,fill=fillColor] (200.25,392.08) circle (  1.16);

\path[draw=drawColor,line width= 0.4pt,line join=round,line cap=round,fill=fillColor] (200.42,392.07) circle (  1.16);

\path[draw=drawColor,line width= 0.4pt,line join=round,line cap=round,fill=fillColor] (200.58,392.05) circle (  1.16);

\path[draw=drawColor,line width= 0.4pt,line join=round,line cap=round,fill=fillColor] (200.75,392.05) circle (  1.16);

\path[draw=drawColor,line width= 0.4pt,line join=round,line cap=round,fill=fillColor] (200.92,391.97) circle (  1.16);

\path[draw=drawColor,line width= 0.4pt,line join=round,line cap=round,fill=fillColor] (201.09,391.64) circle (  1.16);

\path[draw=drawColor,line width= 0.4pt,line join=round,line cap=round,fill=fillColor] (201.25,391.53) circle (  1.16);

\path[draw=drawColor,line width= 0.4pt,line join=round,line cap=round,fill=fillColor] (201.42,391.44) circle (  1.16);

\path[draw=drawColor,line width= 0.4pt,line join=round,line cap=round,fill=fillColor] (201.58,391.44) circle (  1.16);

\path[draw=drawColor,line width= 0.4pt,line join=round,line cap=round,fill=fillColor] (201.75,391.43) circle (  1.16);

\path[draw=drawColor,line width= 0.4pt,line join=round,line cap=round,fill=fillColor] (201.91,391.37) circle (  1.16);

\path[draw=drawColor,line width= 0.4pt,line join=round,line cap=round,fill=fillColor] (202.08,391.34) circle (  1.16);

\path[draw=drawColor,line width= 0.4pt,line join=round,line cap=round,fill=fillColor] (202.24,391.25) circle (  1.16);

\path[draw=drawColor,line width= 0.4pt,line join=round,line cap=round,fill=fillColor] (202.41,391.07) circle (  1.16);

\path[draw=drawColor,line width= 0.4pt,line join=round,line cap=round,fill=fillColor] (202.57,390.94) circle (  1.16);

\path[draw=drawColor,line width= 0.4pt,line join=round,line cap=round,fill=fillColor] (202.73,390.93) circle (  1.16);

\path[draw=drawColor,line width= 0.4pt,line join=round,line cap=round,fill=fillColor] (202.90,390.86) circle (  1.16);

\path[draw=drawColor,line width= 0.4pt,line join=round,line cap=round,fill=fillColor] (203.06,390.62) circle (  1.16);

\path[draw=drawColor,line width= 0.4pt,line join=round,line cap=round,fill=fillColor] (203.22,390.52) circle (  1.16);

\path[draw=drawColor,line width= 0.4pt,line join=round,line cap=round,fill=fillColor] (203.38,390.51) circle (  1.16);

\path[draw=drawColor,line width= 0.4pt,line join=round,line cap=round,fill=fillColor] (203.54,390.50) circle (  1.16);

\path[draw=drawColor,line width= 0.4pt,line join=round,line cap=round,fill=fillColor] (203.71,390.46) circle (  1.16);

\path[draw=drawColor,line width= 0.4pt,line join=round,line cap=round,fill=fillColor] (203.87,390.45) circle (  1.16);

\path[draw=drawColor,line width= 0.4pt,line join=round,line cap=round,fill=fillColor] (204.03,390.44) circle (  1.16);

\path[draw=drawColor,line width= 0.4pt,line join=round,line cap=round,fill=fillColor] (204.19,390.44) circle (  1.16);

\path[draw=drawColor,line width= 0.4pt,line join=round,line cap=round,fill=fillColor] (204.35,390.42) circle (  1.16);

\path[draw=drawColor,line width= 0.4pt,line join=round,line cap=round,fill=fillColor] (204.51,390.38) circle (  1.16);

\path[draw=drawColor,line width= 0.4pt,line join=round,line cap=round,fill=fillColor] (204.66,390.33) circle (  1.16);

\path[draw=drawColor,line width= 0.4pt,line join=round,line cap=round,fill=fillColor] (204.82,390.31) circle (  1.16);

\path[draw=drawColor,line width= 0.4pt,line join=round,line cap=round,fill=fillColor] (204.98,390.30) circle (  1.16);

\path[draw=drawColor,line width= 0.4pt,line join=round,line cap=round,fill=fillColor] (205.14,390.29) circle (  1.16);

\path[draw=drawColor,line width= 0.4pt,line join=round,line cap=round,fill=fillColor] (205.30,390.11) circle (  1.16);

\path[draw=drawColor,line width= 0.4pt,line join=round,line cap=round,fill=fillColor] (205.45,390.09) circle (  1.16);

\path[draw=drawColor,line width= 0.4pt,line join=round,line cap=round,fill=fillColor] (205.61,390.05) circle (  1.16);

\path[draw=drawColor,line width= 0.4pt,line join=round,line cap=round,fill=fillColor] (205.77,390.00) circle (  1.16);

\path[draw=drawColor,line width= 0.4pt,line join=round,line cap=round,fill=fillColor] (205.92,389.95) circle (  1.16);

\path[draw=drawColor,line width= 0.4pt,line join=round,line cap=round,fill=fillColor] (206.08,389.94) circle (  1.16);

\path[draw=drawColor,line width= 0.4pt,line join=round,line cap=round,fill=fillColor] (206.24,389.81) circle (  1.16);

\path[draw=drawColor,line width= 0.4pt,line join=round,line cap=round,fill=fillColor] (206.39,389.79) circle (  1.16);

\path[draw=drawColor,line width= 0.4pt,line join=round,line cap=round,fill=fillColor] (206.55,389.64) circle (  1.16);

\path[draw=drawColor,line width= 0.4pt,line join=round,line cap=round,fill=fillColor] (206.70,389.63) circle (  1.16);

\path[draw=drawColor,line width= 0.4pt,line join=round,line cap=round,fill=fillColor] (206.86,389.52) circle (  1.16);

\path[draw=drawColor,line width= 0.4pt,line join=round,line cap=round,fill=fillColor] (207.01,389.48) circle (  1.16);

\path[draw=drawColor,line width= 0.4pt,line join=round,line cap=round,fill=fillColor] (207.16,389.41) circle (  1.16);

\path[draw=drawColor,line width= 0.4pt,line join=round,line cap=round,fill=fillColor] (207.32,389.32) circle (  1.16);

\path[draw=drawColor,line width= 0.4pt,line join=round,line cap=round,fill=fillColor] (207.47,389.26) circle (  1.16);

\path[draw=drawColor,line width= 0.4pt,line join=round,line cap=round,fill=fillColor] (207.62,389.25) circle (  1.16);

\path[draw=drawColor,line width= 0.4pt,line join=round,line cap=round,fill=fillColor] (207.78,389.25) circle (  1.16);

\path[draw=drawColor,line width= 0.4pt,line join=round,line cap=round,fill=fillColor] (207.93,389.19) circle (  1.16);

\path[draw=drawColor,line width= 0.4pt,line join=round,line cap=round,fill=fillColor] (208.08,389.12) circle (  1.16);

\path[draw=drawColor,line width= 0.4pt,line join=round,line cap=round,fill=fillColor] (208.23,389.06) circle (  1.16);

\path[draw=drawColor,line width= 0.4pt,line join=round,line cap=round,fill=fillColor] (208.39,389.04) circle (  1.16);

\path[draw=drawColor,line width= 0.4pt,line join=round,line cap=round,fill=fillColor] (208.54,389.01) circle (  1.16);

\path[draw=drawColor,line width= 0.4pt,line join=round,line cap=round,fill=fillColor] (208.69,388.88) circle (  1.16);

\path[draw=drawColor,line width= 0.4pt,line join=round,line cap=round,fill=fillColor] (208.84,388.80) circle (  1.16);

\path[draw=drawColor,line width= 0.4pt,line join=round,line cap=round,fill=fillColor] (208.99,388.75) circle (  1.16);

\path[draw=drawColor,line width= 0.4pt,line join=round,line cap=round,fill=fillColor] (209.14,388.54) circle (  1.16);

\path[draw=drawColor,line width= 0.4pt,line join=round,line cap=round,fill=fillColor] (209.29,388.42) circle (  1.16);

\path[draw=drawColor,line width= 0.4pt,line join=round,line cap=round,fill=fillColor] (209.44,388.33) circle (  1.16);

\path[draw=drawColor,line width= 0.4pt,line join=round,line cap=round,fill=fillColor] (209.59,388.29) circle (  1.16);

\path[draw=drawColor,line width= 0.4pt,line join=round,line cap=round,fill=fillColor] (209.74,388.28) circle (  1.16);

\path[draw=drawColor,line width= 0.4pt,line join=round,line cap=round,fill=fillColor] (209.88,388.26) circle (  1.16);

\path[draw=drawColor,line width= 0.4pt,line join=round,line cap=round,fill=fillColor] (210.03,388.24) circle (  1.16);

\path[draw=drawColor,line width= 0.4pt,line join=round,line cap=round,fill=fillColor] (210.18,388.18) circle (  1.16);

\path[draw=drawColor,line width= 0.4pt,line join=round,line cap=round,fill=fillColor] (210.33,388.13) circle (  1.16);

\path[draw=drawColor,line width= 0.4pt,line join=round,line cap=round,fill=fillColor] (210.48,387.90) circle (  1.16);

\path[draw=drawColor,line width= 0.4pt,line join=round,line cap=round,fill=fillColor] (210.62,387.78) circle (  1.16);

\path[draw=drawColor,line width= 0.4pt,line join=round,line cap=round,fill=fillColor] (210.77,387.68) circle (  1.16);

\path[draw=drawColor,line width= 0.4pt,line join=round,line cap=round,fill=fillColor] (210.92,387.68) circle (  1.16);

\path[draw=drawColor,line width= 0.4pt,line join=round,line cap=round,fill=fillColor] (211.06,387.64) circle (  1.16);

\path[draw=drawColor,line width= 0.4pt,line join=round,line cap=round,fill=fillColor] (211.21,387.63) circle (  1.16);

\path[draw=drawColor,line width= 0.4pt,line join=round,line cap=round,fill=fillColor] (211.36,387.56) circle (  1.16);

\path[draw=drawColor,line width= 0.4pt,line join=round,line cap=round,fill=fillColor] (211.50,387.42) circle (  1.16);

\path[draw=drawColor,line width= 0.4pt,line join=round,line cap=round,fill=fillColor] (211.65,387.37) circle (  1.16);

\path[draw=drawColor,line width= 0.4pt,line join=round,line cap=round,fill=fillColor] (211.79,387.35) circle (  1.16);

\path[draw=drawColor,line width= 0.4pt,line join=round,line cap=round,fill=fillColor] (211.94,387.30) circle (  1.16);

\path[draw=drawColor,line width= 0.4pt,line join=round,line cap=round,fill=fillColor] (212.08,387.16) circle (  1.16);

\path[draw=drawColor,line width= 0.4pt,line join=round,line cap=round,fill=fillColor] (212.23,387.11) circle (  1.16);

\path[draw=drawColor,line width= 0.4pt,line join=round,line cap=round,fill=fillColor] (212.37,387.10) circle (  1.16);

\path[draw=drawColor,line width= 0.4pt,line join=round,line cap=round,fill=fillColor] (212.51,387.01) circle (  1.16);

\path[draw=drawColor,line width= 0.4pt,line join=round,line cap=round,fill=fillColor] (212.66,386.94) circle (  1.16);

\path[draw=drawColor,line width= 0.4pt,line join=round,line cap=round,fill=fillColor] (212.80,386.85) circle (  1.16);

\path[draw=drawColor,line width= 0.4pt,line join=round,line cap=round,fill=fillColor] (212.94,386.71) circle (  1.16);

\path[draw=drawColor,line width= 0.4pt,line join=round,line cap=round,fill=fillColor] (213.09,386.70) circle (  1.16);

\path[draw=drawColor,line width= 0.4pt,line join=round,line cap=round,fill=fillColor] (213.23,386.66) circle (  1.16);

\path[draw=drawColor,line width= 0.4pt,line join=round,line cap=round,fill=fillColor] (213.37,386.65) circle (  1.16);

\path[draw=drawColor,line width= 0.4pt,line join=round,line cap=round,fill=fillColor] (213.51,386.62) circle (  1.16);

\path[draw=drawColor,line width= 0.4pt,line join=round,line cap=round,fill=fillColor] (213.65,386.53) circle (  1.16);

\path[draw=drawColor,line width= 0.4pt,line join=round,line cap=round,fill=fillColor] (213.80,386.24) circle (  1.16);

\path[draw=drawColor,line width= 0.4pt,line join=round,line cap=round,fill=fillColor] (213.94,386.22) circle (  1.16);

\path[draw=drawColor,line width= 0.4pt,line join=round,line cap=round,fill=fillColor] (214.08,386.08) circle (  1.16);

\path[draw=drawColor,line width= 0.4pt,line join=round,line cap=round,fill=fillColor] (214.22,386.03) circle (  1.16);

\path[draw=drawColor,line width= 0.4pt,line join=round,line cap=round,fill=fillColor] (214.36,386.02) circle (  1.16);

\path[draw=drawColor,line width= 0.4pt,line join=round,line cap=round,fill=fillColor] (214.50,385.99) circle (  1.16);

\path[draw=drawColor,line width= 0.4pt,line join=round,line cap=round,fill=fillColor] (214.64,385.96) circle (  1.16);

\path[draw=drawColor,line width= 0.4pt,line join=round,line cap=round,fill=fillColor] (214.78,385.90) circle (  1.16);

\path[draw=drawColor,line width= 0.4pt,line join=round,line cap=round,fill=fillColor] (214.92,385.82) circle (  1.16);

\path[draw=drawColor,line width= 0.4pt,line join=round,line cap=round,fill=fillColor] (215.06,385.75) circle (  1.16);

\path[draw=drawColor,line width= 0.4pt,line join=round,line cap=round,fill=fillColor] (215.20,385.65) circle (  1.16);

\path[draw=drawColor,line width= 0.4pt,line join=round,line cap=round,fill=fillColor] (215.34,385.38) circle (  1.16);

\path[draw=drawColor,line width= 0.4pt,line join=round,line cap=round,fill=fillColor] (215.47,385.30) circle (  1.16);

\path[draw=drawColor,line width= 0.4pt,line join=round,line cap=round,fill=fillColor] (215.61,385.23) circle (  1.16);

\path[draw=drawColor,line width= 0.4pt,line join=round,line cap=round,fill=fillColor] (215.75,385.16) circle (  1.16);

\path[draw=drawColor,line width= 0.4pt,line join=round,line cap=round,fill=fillColor] (215.89,385.15) circle (  1.16);

\path[draw=drawColor,line width= 0.4pt,line join=round,line cap=round,fill=fillColor] (216.03,385.14) circle (  1.16);

\path[draw=drawColor,line width= 0.4pt,line join=round,line cap=round,fill=fillColor] (216.16,384.92) circle (  1.16);

\path[draw=drawColor,line width= 0.4pt,line join=round,line cap=round,fill=fillColor] (216.30,384.92) circle (  1.16);

\path[draw=drawColor,line width= 0.4pt,line join=round,line cap=round,fill=fillColor] (216.44,384.87) circle (  1.16);

\path[draw=drawColor,line width= 0.4pt,line join=round,line cap=round,fill=fillColor] (216.58,384.81) circle (  1.16);

\path[draw=drawColor,line width= 0.4pt,line join=round,line cap=round,fill=fillColor] (216.71,384.76) circle (  1.16);

\path[draw=drawColor,line width= 0.4pt,line join=round,line cap=round,fill=fillColor] (216.85,384.74) circle (  1.16);

\path[draw=drawColor,line width= 0.4pt,line join=round,line cap=round,fill=fillColor] (216.98,384.72) circle (  1.16);

\path[draw=drawColor,line width= 0.4pt,line join=round,line cap=round,fill=fillColor] (217.12,384.65) circle (  1.16);

\path[draw=drawColor,line width= 0.4pt,line join=round,line cap=round,fill=fillColor] (217.26,384.59) circle (  1.16);

\path[draw=drawColor,line width= 0.4pt,line join=round,line cap=round,fill=fillColor] (217.39,384.43) circle (  1.16);

\path[draw=drawColor,line width= 0.4pt,line join=round,line cap=round,fill=fillColor] (217.53,384.42) circle (  1.16);

\path[draw=drawColor,line width= 0.4pt,line join=round,line cap=round,fill=fillColor] (217.66,383.60) circle (  1.16);

\path[draw=drawColor,line width= 0.4pt,line join=round,line cap=round,fill=fillColor] (217.80,383.59) circle (  1.16);

\path[draw=drawColor,line width= 0.4pt,line join=round,line cap=round,fill=fillColor] (217.93,383.54) circle (  1.16);

\path[draw=drawColor,line width= 0.4pt,line join=round,line cap=round,fill=fillColor] (218.06,383.30) circle (  1.16);

\path[draw=drawColor,line width= 0.4pt,line join=round,line cap=round,fill=fillColor] (218.20,383.15) circle (  1.16);

\path[draw=drawColor,line width= 0.4pt,line join=round,line cap=round,fill=fillColor] (218.33,382.80) circle (  1.16);

\path[draw=drawColor,line width= 0.4pt,line join=round,line cap=round,fill=fillColor] (218.47,382.80) circle (  1.16);

\path[draw=drawColor,line width= 0.4pt,line join=round,line cap=round,fill=fillColor] (218.60,382.80) circle (  1.16);

\path[draw=drawColor,line width= 0.4pt,line join=round,line cap=round,fill=fillColor] (218.73,382.58) circle (  1.16);

\path[draw=drawColor,line width= 0.4pt,line join=round,line cap=round,fill=fillColor] (218.87,382.45) circle (  1.16);

\path[draw=drawColor,line width= 0.4pt,line join=round,line cap=round,fill=fillColor] (219.00,382.32) circle (  1.16);

\path[draw=drawColor,line width= 0.4pt,line join=round,line cap=round,fill=fillColor] (219.13,382.05) circle (  1.16);

\path[draw=drawColor,line width= 0.4pt,line join=round,line cap=round,fill=fillColor] (219.26,382.00) circle (  1.16);

\path[draw=drawColor,line width= 0.4pt,line join=round,line cap=round,fill=fillColor] (219.40,381.83) circle (  1.16);

\path[draw=drawColor,line width= 0.4pt,line join=round,line cap=round,fill=fillColor] (219.53,381.48) circle (  1.16);

\path[draw=drawColor,line width= 0.4pt,line join=round,line cap=round,fill=fillColor] (219.66,381.36) circle (  1.16);

\path[draw=drawColor,line width= 0.4pt,line join=round,line cap=round,fill=fillColor] (219.79,381.32) circle (  1.16);

\path[draw=drawColor,line width= 0.4pt,line join=round,line cap=round,fill=fillColor] (219.92,381.31) circle (  1.16);

\path[draw=drawColor,line width= 0.4pt,line join=round,line cap=round,fill=fillColor] (220.05,381.30) circle (  1.16);

\path[draw=drawColor,line width= 0.4pt,line join=round,line cap=round,fill=fillColor] (220.18,380.66) circle (  1.16);

\path[draw=drawColor,line width= 0.4pt,line join=round,line cap=round,fill=fillColor] (220.31,380.66) circle (  1.16);

\path[draw=drawColor,line width= 0.4pt,line join=round,line cap=round,fill=fillColor] (220.45,380.66) circle (  1.16);

\path[draw=drawColor,line width= 0.4pt,line join=round,line cap=round,fill=fillColor] (220.58,380.52) circle (  1.16);

\path[draw=drawColor,line width= 0.4pt,line join=round,line cap=round,fill=fillColor] (220.71,379.91) circle (  1.16);

\path[draw=drawColor,line width= 0.4pt,line join=round,line cap=round,fill=fillColor] (220.84,379.22) circle (  1.16);

\path[draw=drawColor,line width= 0.4pt,line join=round,line cap=round,fill=fillColor] (220.97,379.17) circle (  1.16);

\path[draw=drawColor,line width= 0.4pt,line join=round,line cap=round,fill=fillColor] (221.09,378.96) circle (  1.16);

\path[draw=drawColor,line width= 0.4pt,line join=round,line cap=round,fill=fillColor] (221.22,378.69) circle (  1.16);

\path[draw=drawColor,line width= 0.4pt,line join=round,line cap=round,fill=fillColor] (221.35,378.27) circle (  1.16);

\path[draw=drawColor,line width= 0.4pt,line join=round,line cap=round,fill=fillColor] (221.48,377.60) circle (  1.16);

\path[draw=drawColor,line width= 0.4pt,line join=round,line cap=round,fill=fillColor] (221.61,377.46) circle (  1.16);

\path[draw=drawColor,line width= 0.4pt,line join=round,line cap=round,fill=fillColor] (221.74,372.42) circle (  1.16);

\path[draw=drawColor,line width= 0.4pt,line join=round,line cap=round,fill=fillColor] (221.87,372.42) circle (  1.16);

\path[draw=drawColor,line width= 0.4pt,line join=round,line cap=round,fill=fillColor] (222.00,372.42) circle (  1.16);

\path[draw=drawColor,line width= 0.4pt,line join=round,line cap=round,fill=fillColor] (222.12,372.42) circle (  1.16);

\path[draw=drawColor,line width= 0.4pt,line join=round,line cap=round,fill=fillColor] (222.25,372.42) circle (  1.16);

\path[draw=drawColor,line width= 0.4pt,line join=round,line cap=round,fill=fillColor] (222.38,372.42) circle (  1.16);

\path[draw=drawColor,line width= 0.4pt,line join=round,line cap=round,fill=fillColor] (222.51,372.42) circle (  1.16);

\path[draw=drawColor,line width= 0.4pt,line join=round,line cap=round,fill=fillColor] (222.63,372.42) circle (  1.16);

\path[draw=drawColor,line width= 0.4pt,line join=round,line cap=round,fill=fillColor] (222.76,372.42) circle (  1.16);

\path[draw=drawColor,line width= 0.4pt,line join=round,line cap=round,fill=fillColor] (222.89,372.42) circle (  1.16);

\path[draw=drawColor,line width= 0.4pt,line join=round,line cap=round,fill=fillColor] (223.01,372.42) circle (  1.16);

\path[draw=drawColor,line width= 0.4pt,line join=round,line cap=round,fill=fillColor] (223.14,372.42) circle (  1.16);

\path[draw=drawColor,line width= 0.4pt,line join=round,line cap=round,fill=fillColor] (223.27,372.42) circle (  1.16);

\path[draw=drawColor,line width= 0.4pt,line join=round,line cap=round,fill=fillColor] (223.39,372.42) circle (  1.16);

\path[draw=drawColor,line width= 0.4pt,line join=round,line cap=round,fill=fillColor] (223.52,372.42) circle (  1.16);

\path[draw=drawColor,line width= 0.4pt,line join=round,line cap=round,fill=fillColor] (223.64,372.42) circle (  1.16);

\path[draw=drawColor,line width= 0.4pt,line join=round,line cap=round,fill=fillColor] (223.77,372.42) circle (  1.16);

\path[draw=drawColor,line width= 0.4pt,line join=round,line cap=round,fill=fillColor] (223.89,372.42) circle (  1.16);

\path[draw=drawColor,line width= 0.4pt,line join=round,line cap=round,fill=fillColor] (224.02,372.42) circle (  1.16);

\path[draw=drawColor,line width= 0.4pt,line join=round,line cap=round,fill=fillColor] (224.14,372.42) circle (  1.16);

\path[draw=drawColor,line width= 0.4pt,line join=round,line cap=round,fill=fillColor] (224.27,372.42) circle (  1.16);

\path[draw=drawColor,line width= 0.4pt,line join=round,line cap=round,fill=fillColor] (224.39,372.42) circle (  1.16);

\path[draw=drawColor,line width= 0.4pt,line join=round,line cap=round,fill=fillColor] (224.52,372.42) circle (  1.16);

\path[draw=drawColor,line width= 0.4pt,line join=round,line cap=round,fill=fillColor] (224.64,372.42) circle (  1.16);

\path[draw=drawColor,line width= 0.4pt,line join=round,line cap=round,fill=fillColor] (224.77,372.42) circle (  1.16);

\path[draw=drawColor,line width= 0.4pt,line join=round,line cap=round,fill=fillColor] (224.89,372.42) circle (  1.16);

\path[draw=drawColor,line width= 0.4pt,line join=round,line cap=round,fill=fillColor] (225.01,372.42) circle (  1.16);

\path[draw=drawColor,line width= 0.4pt,line join=round,line cap=round,fill=fillColor] (225.14,372.42) circle (  1.16);

\path[draw=drawColor,line width= 0.4pt,line join=round,line cap=round,fill=fillColor] (225.26,372.42) circle (  1.16);
\definecolor[named]{drawColor}{rgb}{0.65,0.34,0.16}
\definecolor[named]{fillColor}{rgb}{0.65,0.34,0.16}

\path[draw=drawColor,line width= 0.4pt,line join=round,line cap=round,fill=fillColor] ( 74.88,429.34) circle (  1.16);

\path[draw=drawColor,line width= 0.4pt,line join=round,line cap=round,fill=fillColor] ( 80.66,426.33) circle (  1.16);

\path[draw=drawColor,line width= 0.4pt,line join=round,line cap=round,fill=fillColor] ( 84.72,423.91) circle (  1.16);

\path[draw=drawColor,line width= 0.4pt,line join=round,line cap=round,fill=fillColor] ( 87.95,422.81) circle (  1.16);

\path[draw=drawColor,line width= 0.4pt,line join=round,line cap=round,fill=fillColor] ( 90.68,421.84) circle (  1.16);

\path[draw=drawColor,line width= 0.4pt,line join=round,line cap=round,fill=fillColor] ( 93.06,421.61) circle (  1.16);

\path[draw=drawColor,line width= 0.4pt,line join=round,line cap=round,fill=fillColor] ( 95.19,421.40) circle (  1.16);

\path[draw=drawColor,line width= 0.4pt,line join=round,line cap=round,fill=fillColor] ( 97.13,420.17) circle (  1.16);

\path[draw=drawColor,line width= 0.4pt,line join=round,line cap=round,fill=fillColor] ( 98.91,419.77) circle (  1.16);

\path[draw=drawColor,line width= 0.4pt,line join=round,line cap=round,fill=fillColor] (100.57,418.54) circle (  1.16);

\path[draw=drawColor,line width= 0.4pt,line join=round,line cap=round,fill=fillColor] (102.11,415.61) circle (  1.16);

\path[draw=drawColor,line width= 0.4pt,line join=round,line cap=round,fill=fillColor] (103.57,414.79) circle (  1.16);

\path[draw=drawColor,line width= 0.4pt,line join=round,line cap=round,fill=fillColor] (104.95,413.22) circle (  1.16);

\path[draw=drawColor,line width= 0.4pt,line join=round,line cap=round,fill=fillColor] (106.26,411.49) circle (  1.16);

\path[draw=drawColor,line width= 0.4pt,line join=round,line cap=round,fill=fillColor] (107.50,410.90) circle (  1.16);

\path[draw=drawColor,line width= 0.4pt,line join=round,line cap=round,fill=fillColor] (108.70,409.59) circle (  1.16);

\path[draw=drawColor,line width= 0.4pt,line join=round,line cap=round,fill=fillColor] (109.84,409.48) circle (  1.16);

\path[draw=drawColor,line width= 0.4pt,line join=round,line cap=round,fill=fillColor] (110.94,408.54) circle (  1.16);

\path[draw=drawColor,line width= 0.4pt,line join=round,line cap=round,fill=fillColor] (112.00,408.42) circle (  1.16);

\path[draw=drawColor,line width= 0.4pt,line join=round,line cap=round,fill=fillColor] (113.03,407.90) circle (  1.16);

\path[draw=drawColor,line width= 0.4pt,line join=round,line cap=round,fill=fillColor] (114.02,407.89) circle (  1.16);

\path[draw=drawColor,line width= 0.4pt,line join=round,line cap=round,fill=fillColor] (114.98,407.83) circle (  1.16);

\path[draw=drawColor,line width= 0.4pt,line join=round,line cap=round,fill=fillColor] (115.91,407.73) circle (  1.16);

\path[draw=drawColor,line width= 0.4pt,line join=round,line cap=round,fill=fillColor] (116.81,407.60) circle (  1.16);

\path[draw=drawColor,line width= 0.4pt,line join=round,line cap=round,fill=fillColor] (117.69,406.96) circle (  1.16);

\path[draw=drawColor,line width= 0.4pt,line join=round,line cap=round,fill=fillColor] (118.55,406.47) circle (  1.16);

\path[draw=drawColor,line width= 0.4pt,line join=round,line cap=round,fill=fillColor] (119.38,406.45) circle (  1.16);

\path[draw=drawColor,line width= 0.4pt,line join=round,line cap=round,fill=fillColor] (120.20,405.99) circle (  1.16);

\path[draw=drawColor,line width= 0.4pt,line join=round,line cap=round,fill=fillColor] (120.99,404.80) circle (  1.16);

\path[draw=drawColor,line width= 0.4pt,line join=round,line cap=round,fill=fillColor] (121.77,403.98) circle (  1.16);

\path[draw=drawColor,line width= 0.4pt,line join=round,line cap=round,fill=fillColor] (122.53,403.32) circle (  1.16);

\path[draw=drawColor,line width= 0.4pt,line join=round,line cap=round,fill=fillColor] (123.27,403.09) circle (  1.16);

\path[draw=drawColor,line width= 0.4pt,line join=round,line cap=round,fill=fillColor] (124.00,403.04) circle (  1.16);

\path[draw=drawColor,line width= 0.4pt,line join=round,line cap=round,fill=fillColor] (124.71,402.76) circle (  1.16);

\path[draw=drawColor,line width= 0.4pt,line join=round,line cap=round,fill=fillColor] (125.41,402.57) circle (  1.16);

\path[draw=drawColor,line width= 0.4pt,line join=round,line cap=round,fill=fillColor] (126.10,402.48) circle (  1.16);

\path[draw=drawColor,line width= 0.4pt,line join=round,line cap=round,fill=fillColor] (126.77,402.32) circle (  1.16);

\path[draw=drawColor,line width= 0.4pt,line join=round,line cap=round,fill=fillColor] (127.44,402.12) circle (  1.16);

\path[draw=drawColor,line width= 0.4pt,line join=round,line cap=round,fill=fillColor] (128.09,402.10) circle (  1.16);

\path[draw=drawColor,line width= 0.4pt,line join=round,line cap=round,fill=fillColor] (128.73,402.02) circle (  1.16);

\path[draw=drawColor,line width= 0.4pt,line join=round,line cap=round,fill=fillColor] (129.35,401.36) circle (  1.16);

\path[draw=drawColor,line width= 0.4pt,line join=round,line cap=round,fill=fillColor] (129.97,401.20) circle (  1.16);

\path[draw=drawColor,line width= 0.4pt,line join=round,line cap=round,fill=fillColor] (130.58,400.91) circle (  1.16);

\path[draw=drawColor,line width= 0.4pt,line join=round,line cap=round,fill=fillColor] (131.18,400.50) circle (  1.16);

\path[draw=drawColor,line width= 0.4pt,line join=round,line cap=round,fill=fillColor] (131.77,400.15) circle (  1.16);

\path[draw=drawColor,line width= 0.4pt,line join=round,line cap=round,fill=fillColor] (132.35,400.06) circle (  1.16);

\path[draw=drawColor,line width= 0.4pt,line join=round,line cap=round,fill=fillColor] (132.93,399.95) circle (  1.16);

\path[draw=drawColor,line width= 0.4pt,line join=round,line cap=round,fill=fillColor] (133.49,399.49) circle (  1.16);

\path[draw=drawColor,line width= 0.4pt,line join=round,line cap=round,fill=fillColor] (134.05,399.40) circle (  1.16);

\path[draw=drawColor,line width= 0.4pt,line join=round,line cap=round,fill=fillColor] (134.60,399.27) circle (  1.16);

\path[draw=drawColor,line width= 0.4pt,line join=round,line cap=round,fill=fillColor] (135.14,399.00) circle (  1.16);

\path[draw=drawColor,line width= 0.4pt,line join=round,line cap=round,fill=fillColor] (135.68,398.99) circle (  1.16);

\path[draw=drawColor,line width= 0.4pt,line join=round,line cap=round,fill=fillColor] (136.21,398.99) circle (  1.16);

\path[draw=drawColor,line width= 0.4pt,line join=round,line cap=round,fill=fillColor] (136.73,398.99) circle (  1.16);

\path[draw=drawColor,line width= 0.4pt,line join=round,line cap=round,fill=fillColor] (137.25,398.94) circle (  1.16);

\path[draw=drawColor,line width= 0.4pt,line join=round,line cap=round,fill=fillColor] (137.76,398.46) circle (  1.16);

\path[draw=drawColor,line width= 0.4pt,line join=round,line cap=round,fill=fillColor] (138.26,398.30) circle (  1.16);

\path[draw=drawColor,line width= 0.4pt,line join=round,line cap=round,fill=fillColor] (138.76,397.89) circle (  1.16);

\path[draw=drawColor,line width= 0.4pt,line join=round,line cap=round,fill=fillColor] (139.25,397.87) circle (  1.16);

\path[draw=drawColor,line width= 0.4pt,line join=round,line cap=round,fill=fillColor] (139.74,397.75) circle (  1.16);

\path[draw=drawColor,line width= 0.4pt,line join=round,line cap=round,fill=fillColor] (140.22,397.54) circle (  1.16);

\path[draw=drawColor,line width= 0.4pt,line join=round,line cap=round,fill=fillColor] (140.70,397.48) circle (  1.16);

\path[draw=drawColor,line width= 0.4pt,line join=round,line cap=round,fill=fillColor] (141.17,397.32) circle (  1.16);

\path[draw=drawColor,line width= 0.4pt,line join=round,line cap=round,fill=fillColor] (141.63,397.17) circle (  1.16);

\path[draw=drawColor,line width= 0.4pt,line join=round,line cap=round,fill=fillColor] (142.09,396.97) circle (  1.16);

\path[draw=drawColor,line width= 0.4pt,line join=round,line cap=round,fill=fillColor] (142.55,396.97) circle (  1.16);

\path[draw=drawColor,line width= 0.4pt,line join=round,line cap=round,fill=fillColor] (143.00,396.14) circle (  1.16);

\path[draw=drawColor,line width= 0.4pt,line join=round,line cap=round,fill=fillColor] (143.45,396.14) circle (  1.16);

\path[draw=drawColor,line width= 0.4pt,line join=round,line cap=round,fill=fillColor] (143.89,396.09) circle (  1.16);

\path[draw=drawColor,line width= 0.4pt,line join=round,line cap=round,fill=fillColor] (144.33,396.03) circle (  1.16);

\path[draw=drawColor,line width= 0.4pt,line join=round,line cap=round,fill=fillColor] (144.77,395.81) circle (  1.16);

\path[draw=drawColor,line width= 0.4pt,line join=round,line cap=round,fill=fillColor] (145.20,395.75) circle (  1.16);

\path[draw=drawColor,line width= 0.4pt,line join=round,line cap=round,fill=fillColor] (145.62,395.71) circle (  1.16);

\path[draw=drawColor,line width= 0.4pt,line join=round,line cap=round,fill=fillColor] (146.05,395.65) circle (  1.16);

\path[draw=drawColor,line width= 0.4pt,line join=round,line cap=round,fill=fillColor] (146.46,395.54) circle (  1.16);

\path[draw=drawColor,line width= 0.4pt,line join=round,line cap=round,fill=fillColor] (146.88,395.27) circle (  1.16);

\path[draw=drawColor,line width= 0.4pt,line join=round,line cap=round,fill=fillColor] (147.29,395.21) circle (  1.16);

\path[draw=drawColor,line width= 0.4pt,line join=round,line cap=round,fill=fillColor] (147.70,394.49) circle (  1.16);

\path[draw=drawColor,line width= 0.4pt,line join=round,line cap=round,fill=fillColor] (148.10,394.32) circle (  1.16);

\path[draw=drawColor,line width= 0.4pt,line join=round,line cap=round,fill=fillColor] (148.50,394.08) circle (  1.16);

\path[draw=drawColor,line width= 0.4pt,line join=round,line cap=round,fill=fillColor] (148.90,393.83) circle (  1.16);

\path[draw=drawColor,line width= 0.4pt,line join=round,line cap=round,fill=fillColor] (149.30,393.80) circle (  1.16);

\path[draw=drawColor,line width= 0.4pt,line join=round,line cap=round,fill=fillColor] (149.69,393.67) circle (  1.16);

\path[draw=drawColor,line width= 0.4pt,line join=round,line cap=round,fill=fillColor] (150.08,393.24) circle (  1.16);

\path[draw=drawColor,line width= 0.4pt,line join=round,line cap=round,fill=fillColor] (150.46,392.88) circle (  1.16);

\path[draw=drawColor,line width= 0.4pt,line join=round,line cap=round,fill=fillColor] (150.84,392.69) circle (  1.16);

\path[draw=drawColor,line width= 0.4pt,line join=round,line cap=round,fill=fillColor] (151.22,392.58) circle (  1.16);

\path[draw=drawColor,line width= 0.4pt,line join=round,line cap=round,fill=fillColor] (151.60,392.49) circle (  1.16);

\path[draw=drawColor,line width= 0.4pt,line join=round,line cap=round,fill=fillColor] (151.97,392.44) circle (  1.16);

\path[draw=drawColor,line width= 0.4pt,line join=round,line cap=round,fill=fillColor] (152.34,392.38) circle (  1.16);

\path[draw=drawColor,line width= 0.4pt,line join=round,line cap=round,fill=fillColor] (152.71,392.11) circle (  1.16);

\path[draw=drawColor,line width= 0.4pt,line join=round,line cap=round,fill=fillColor] (153.08,392.00) circle (  1.16);

\path[draw=drawColor,line width= 0.4pt,line join=round,line cap=round,fill=fillColor] (153.44,391.92) circle (  1.16);

\path[draw=drawColor,line width= 0.4pt,line join=round,line cap=round,fill=fillColor] (153.80,391.88) circle (  1.16);

\path[draw=drawColor,line width= 0.4pt,line join=round,line cap=round,fill=fillColor] (154.16,391.71) circle (  1.16);

\path[draw=drawColor,line width= 0.4pt,line join=round,line cap=round,fill=fillColor] (154.51,391.66) circle (  1.16);

\path[draw=drawColor,line width= 0.4pt,line join=round,line cap=round,fill=fillColor] (154.86,391.37) circle (  1.16);

\path[draw=drawColor,line width= 0.4pt,line join=round,line cap=round,fill=fillColor] (155.22,391.15) circle (  1.16);

\path[draw=drawColor,line width= 0.4pt,line join=round,line cap=round,fill=fillColor] (155.56,390.90) circle (  1.16);

\path[draw=drawColor,line width= 0.4pt,line join=round,line cap=round,fill=fillColor] (155.91,390.80) circle (  1.16);

\path[draw=drawColor,line width= 0.4pt,line join=round,line cap=round,fill=fillColor] (156.25,390.56) circle (  1.16);

\path[draw=drawColor,line width= 0.4pt,line join=round,line cap=round,fill=fillColor] (156.59,390.49) circle (  1.16);

\path[draw=drawColor,line width= 0.4pt,line join=round,line cap=round,fill=fillColor] (156.93,390.48) circle (  1.16);

\path[draw=drawColor,line width= 0.4pt,line join=round,line cap=round,fill=fillColor] (157.27,390.38) circle (  1.16);

\path[draw=drawColor,line width= 0.4pt,line join=round,line cap=round,fill=fillColor] (157.60,389.96) circle (  1.16);

\path[draw=drawColor,line width= 0.4pt,line join=round,line cap=round,fill=fillColor] (157.93,389.89) circle (  1.16);

\path[draw=drawColor,line width= 0.4pt,line join=round,line cap=round,fill=fillColor] (158.26,389.80) circle (  1.16);

\path[draw=drawColor,line width= 0.4pt,line join=round,line cap=round,fill=fillColor] (158.59,389.76) circle (  1.16);

\path[draw=drawColor,line width= 0.4pt,line join=round,line cap=round,fill=fillColor] (158.92,389.66) circle (  1.16);

\path[draw=drawColor,line width= 0.4pt,line join=round,line cap=round,fill=fillColor] (159.24,389.52) circle (  1.16);

\path[draw=drawColor,line width= 0.4pt,line join=round,line cap=round,fill=fillColor] (159.56,389.34) circle (  1.16);

\path[draw=drawColor,line width= 0.4pt,line join=round,line cap=round,fill=fillColor] (159.88,389.31) circle (  1.16);

\path[draw=drawColor,line width= 0.4pt,line join=round,line cap=round,fill=fillColor] (160.20,389.26) circle (  1.16);

\path[draw=drawColor,line width= 0.4pt,line join=round,line cap=round,fill=fillColor] (160.52,388.97) circle (  1.16);

\path[draw=drawColor,line width= 0.4pt,line join=round,line cap=round,fill=fillColor] (160.83,388.85) circle (  1.16);

\path[draw=drawColor,line width= 0.4pt,line join=round,line cap=round,fill=fillColor] (161.15,388.13) circle (  1.16);

\path[draw=drawColor,line width= 0.4pt,line join=round,line cap=round,fill=fillColor] (161.46,387.94) circle (  1.16);

\path[draw=drawColor,line width= 0.4pt,line join=round,line cap=round,fill=fillColor] (161.77,387.56) circle (  1.16);

\path[draw=drawColor,line width= 0.4pt,line join=round,line cap=round,fill=fillColor] (162.07,387.09) circle (  1.16);

\path[draw=drawColor,line width= 0.4pt,line join=round,line cap=round,fill=fillColor] (162.38,386.83) circle (  1.16);

\path[draw=drawColor,line width= 0.4pt,line join=round,line cap=round,fill=fillColor] (162.68,386.83) circle (  1.16);

\path[draw=drawColor,line width= 0.4pt,line join=round,line cap=round,fill=fillColor] (162.99,386.76) circle (  1.16);

\path[draw=drawColor,line width= 0.4pt,line join=round,line cap=round,fill=fillColor] (163.29,386.38) circle (  1.16);

\path[draw=drawColor,line width= 0.4pt,line join=round,line cap=round,fill=fillColor] (163.59,386.34) circle (  1.16);

\path[draw=drawColor,line width= 0.4pt,line join=round,line cap=round,fill=fillColor] (163.88,386.30) circle (  1.16);

\path[draw=drawColor,line width= 0.4pt,line join=round,line cap=round,fill=fillColor] (164.18,386.00) circle (  1.16);

\path[draw=drawColor,line width= 0.4pt,line join=round,line cap=round,fill=fillColor] (164.47,385.74) circle (  1.16);

\path[draw=drawColor,line width= 0.4pt,line join=round,line cap=round,fill=fillColor] (164.77,385.59) circle (  1.16);

\path[draw=drawColor,line width= 0.4pt,line join=round,line cap=round,fill=fillColor] (165.06,385.47) circle (  1.16);

\path[draw=drawColor,line width= 0.4pt,line join=round,line cap=round,fill=fillColor] (165.35,384.94) circle (  1.16);

\path[draw=drawColor,line width= 0.4pt,line join=round,line cap=round,fill=fillColor] (165.64,384.72) circle (  1.16);

\path[draw=drawColor,line width= 0.4pt,line join=round,line cap=round,fill=fillColor] (165.92,384.65) circle (  1.16);

\path[draw=drawColor,line width= 0.4pt,line join=round,line cap=round,fill=fillColor] (166.21,384.59) circle (  1.16);

\path[draw=drawColor,line width= 0.4pt,line join=round,line cap=round,fill=fillColor] (166.49,383.77) circle (  1.16);

\path[draw=drawColor,line width= 0.4pt,line join=round,line cap=round,fill=fillColor] (166.77,383.56) circle (  1.16);

\path[draw=drawColor,line width= 0.4pt,line join=round,line cap=round,fill=fillColor] (167.06,383.30) circle (  1.16);

\path[draw=drawColor,line width= 0.4pt,line join=round,line cap=round,fill=fillColor] (167.34,383.15) circle (  1.16);

\path[draw=drawColor,line width= 0.4pt,line join=round,line cap=round,fill=fillColor] (167.61,382.76) circle (  1.16);

\path[draw=drawColor,line width= 0.4pt,line join=round,line cap=round,fill=fillColor] (167.89,382.72) circle (  1.16);

\path[draw=drawColor,line width= 0.4pt,line join=round,line cap=round,fill=fillColor] (168.17,382.29) circle (  1.16);

\path[draw=drawColor,line width= 0.4pt,line join=round,line cap=round,fill=fillColor] (168.44,382.16) circle (  1.16);

\path[draw=drawColor,line width= 0.4pt,line join=round,line cap=round,fill=fillColor] (168.71,382.09) circle (  1.16);

\path[draw=drawColor,line width= 0.4pt,line join=round,line cap=round,fill=fillColor] (168.99,382.08) circle (  1.16);

\path[draw=drawColor,line width= 0.4pt,line join=round,line cap=round,fill=fillColor] (169.26,381.95) circle (  1.16);

\path[draw=drawColor,line width= 0.4pt,line join=round,line cap=round,fill=fillColor] (169.53,381.84) circle (  1.16);

\path[draw=drawColor,line width= 0.4pt,line join=round,line cap=round,fill=fillColor] (169.79,381.48) circle (  1.16);

\path[draw=drawColor,line width= 0.4pt,line join=round,line cap=round,fill=fillColor] (170.06,380.54) circle (  1.16);

\path[draw=drawColor,line width= 0.4pt,line join=round,line cap=round,fill=fillColor] (170.33,380.36) circle (  1.16);

\path[draw=drawColor,line width= 0.4pt,line join=round,line cap=round,fill=fillColor] (170.59,380.13) circle (  1.16);

\path[draw=drawColor,line width= 0.4pt,line join=round,line cap=round,fill=fillColor] (170.85,379.80) circle (  1.16);

\path[draw=drawColor,line width= 0.4pt,line join=round,line cap=round,fill=fillColor] (171.12,379.75) circle (  1.16);

\path[draw=drawColor,line width= 0.4pt,line join=round,line cap=round,fill=fillColor] (171.38,379.29) circle (  1.16);

\path[draw=drawColor,line width= 0.4pt,line join=round,line cap=round,fill=fillColor] (171.64,378.82) circle (  1.16);

\path[draw=drawColor,line width= 0.4pt,line join=round,line cap=round,fill=fillColor] (171.90,378.47) circle (  1.16);

\path[draw=drawColor,line width= 0.4pt,line join=round,line cap=round,fill=fillColor] (172.15,378.36) circle (  1.16);

\path[draw=drawColor,line width= 0.4pt,line join=round,line cap=round,fill=fillColor] (172.41,376.98) circle (  1.16);

\path[draw=drawColor,line width= 0.4pt,line join=round,line cap=round,fill=fillColor] (172.67,376.79) circle (  1.16);

\path[draw=drawColor,line width= 0.4pt,line join=round,line cap=round,fill=fillColor] (172.92,376.74) circle (  1.16);

\path[draw=drawColor,line width= 0.4pt,line join=round,line cap=round,fill=fillColor] (173.17,372.42) circle (  1.16);

\path[draw=drawColor,line width= 0.4pt,line join=round,line cap=round,fill=fillColor] (173.43,372.42) circle (  1.16);

\path[draw=drawColor,line width= 0.4pt,line join=round,line cap=round,fill=fillColor] (173.68,372.42) circle (  1.16);

\path[draw=drawColor,line width= 0.4pt,line join=round,line cap=round,fill=fillColor] (173.93,372.42) circle (  1.16);

\path[draw=drawColor,line width= 0.4pt,line join=round,line cap=round,fill=fillColor] (174.18,372.42) circle (  1.16);

\path[draw=drawColor,line width= 0.4pt,line join=round,line cap=round,fill=fillColor] (174.42,372.42) circle (  1.16);

\path[draw=drawColor,line width= 0.4pt,line join=round,line cap=round,fill=fillColor] (174.67,372.42) circle (  1.16);

\path[draw=drawColor,line width= 0.4pt,line join=round,line cap=round,fill=fillColor] (174.92,372.42) circle (  1.16);

\path[draw=drawColor,line width= 0.4pt,line join=round,line cap=round,fill=fillColor] (175.16,372.42) circle (  1.16);

\path[draw=drawColor,line width= 0.4pt,line join=round,line cap=round,fill=fillColor] (175.41,372.42) circle (  1.16);

\path[draw=drawColor,line width= 0.4pt,line join=round,line cap=round,fill=fillColor] (175.65,372.42) circle (  1.16);

\path[draw=drawColor,line width= 0.4pt,line join=round,line cap=round,fill=fillColor] (175.89,372.42) circle (  1.16);

\path[draw=drawColor,line width= 0.4pt,line join=round,line cap=round,fill=fillColor] (176.13,372.42) circle (  1.16);

\path[draw=drawColor,line width= 0.4pt,line join=round,line cap=round,fill=fillColor] (176.37,372.42) circle (  1.16);

\path[draw=drawColor,line width= 0.4pt,line join=round,line cap=round,fill=fillColor] (176.61,372.42) circle (  1.16);

\path[draw=drawColor,line width= 0.4pt,line join=round,line cap=round,fill=fillColor] (176.85,372.42) circle (  1.16);

\path[draw=drawColor,line width= 0.4pt,line join=round,line cap=round,fill=fillColor] (177.09,372.42) circle (  1.16);

\path[draw=drawColor,line width= 0.4pt,line join=round,line cap=round,fill=fillColor] (177.32,372.42) circle (  1.16);

\path[draw=drawColor,line width= 0.4pt,line join=round,line cap=round,fill=fillColor] (177.56,372.42) circle (  1.16);

\path[draw=drawColor,line width= 0.4pt,line join=round,line cap=round,fill=fillColor] (177.80,372.42) circle (  1.16);

\path[draw=drawColor,line width= 0.4pt,line join=round,line cap=round,fill=fillColor] (178.03,372.42) circle (  1.16);

\path[draw=drawColor,line width= 0.4pt,line join=round,line cap=round,fill=fillColor] (178.26,372.42) circle (  1.16);

\path[draw=drawColor,line width= 0.4pt,line join=round,line cap=round,fill=fillColor] (178.49,372.42) circle (  1.16);

\path[draw=drawColor,line width= 0.4pt,line join=round,line cap=round,fill=fillColor] (178.73,372.42) circle (  1.16);

\path[draw=drawColor,line width= 0.4pt,line join=round,line cap=round,fill=fillColor] (178.96,372.42) circle (  1.16);

\path[draw=drawColor,line width= 0.4pt,line join=round,line cap=round,fill=fillColor] (179.19,372.42) circle (  1.16);

\path[draw=drawColor,line width= 0.4pt,line join=round,line cap=round,fill=fillColor] (179.41,372.42) circle (  1.16);

\path[draw=drawColor,line width= 0.4pt,line join=round,line cap=round,fill=fillColor] (179.64,372.42) circle (  1.16);

\path[draw=drawColor,line width= 0.4pt,line join=round,line cap=round,fill=fillColor] (179.87,372.42) circle (  1.16);

\path[draw=drawColor,line width= 0.4pt,line join=round,line cap=round,fill=fillColor] (180.10,372.42) circle (  1.16);

\path[draw=drawColor,line width= 0.4pt,line join=round,line cap=round,fill=fillColor] (180.32,372.42) circle (  1.16);

\path[draw=drawColor,line width= 0.4pt,line join=round,line cap=round,fill=fillColor] (180.55,372.42) circle (  1.16);

\path[draw=drawColor,line width= 0.4pt,line join=round,line cap=round,fill=fillColor] (180.77,372.42) circle (  1.16);

\path[draw=drawColor,line width= 0.4pt,line join=round,line cap=round,fill=fillColor] (180.99,372.42) circle (  1.16);

\path[draw=drawColor,line width= 0.4pt,line join=round,line cap=round,fill=fillColor] (181.22,372.42) circle (  1.16);

\path[draw=drawColor,line width= 0.4pt,line join=round,line cap=round,fill=fillColor] (181.44,372.42) circle (  1.16);

\path[draw=drawColor,line width= 0.4pt,line join=round,line cap=round,fill=fillColor] (181.66,372.42) circle (  1.16);

\path[draw=drawColor,line width= 0.4pt,line join=round,line cap=round,fill=fillColor] (181.88,372.42) circle (  1.16);

\path[draw=drawColor,line width= 0.4pt,line join=round,line cap=round,fill=fillColor] (182.10,372.42) circle (  1.16);

\path[draw=drawColor,line width= 0.4pt,line join=round,line cap=round,fill=fillColor] (182.32,372.42) circle (  1.16);

\path[draw=drawColor,line width= 0.4pt,line join=round,line cap=round,fill=fillColor] (182.54,372.42) circle (  1.16);

\path[draw=drawColor,line width= 0.4pt,line join=round,line cap=round,fill=fillColor] (182.75,372.42) circle (  1.16);

\path[draw=drawColor,line width= 0.4pt,line join=round,line cap=round,fill=fillColor] (182.97,372.42) circle (  1.16);

\path[draw=drawColor,line width= 0.4pt,line join=round,line cap=round,fill=fillColor] (183.19,372.42) circle (  1.16);

\path[draw=drawColor,line width= 0.4pt,line join=round,line cap=round,fill=fillColor] (183.40,372.42) circle (  1.16);

\path[draw=drawColor,line width= 0.4pt,line join=round,line cap=round,fill=fillColor] (183.61,372.42) circle (  1.16);

\path[draw=drawColor,line width= 0.4pt,line join=round,line cap=round,fill=fillColor] (183.83,372.42) circle (  1.16);

\path[draw=drawColor,line width= 0.4pt,line join=round,line cap=round,fill=fillColor] (184.04,372.42) circle (  1.16);

\path[draw=drawColor,line width= 0.4pt,line join=round,line cap=round,fill=fillColor] (184.25,372.42) circle (  1.16);

\path[draw=drawColor,line width= 0.4pt,line join=round,line cap=round,fill=fillColor] (184.47,372.42) circle (  1.16);

\path[draw=drawColor,line width= 0.4pt,line join=round,line cap=round,fill=fillColor] (184.68,372.42) circle (  1.16);

\path[draw=drawColor,line width= 0.4pt,line join=round,line cap=round,fill=fillColor] (184.89,372.42) circle (  1.16);

\path[draw=drawColor,line width= 0.4pt,line join=round,line cap=round,fill=fillColor] (185.10,372.42) circle (  1.16);

\path[draw=drawColor,line width= 0.4pt,line join=round,line cap=round,fill=fillColor] (185.31,372.42) circle (  1.16);

\path[draw=drawColor,line width= 0.4pt,line join=round,line cap=round,fill=fillColor] (185.51,372.42) circle (  1.16);

\path[draw=drawColor,line width= 0.4pt,line join=round,line cap=round,fill=fillColor] (185.72,372.42) circle (  1.16);

\path[draw=drawColor,line width= 0.4pt,line join=round,line cap=round,fill=fillColor] (185.93,372.42) circle (  1.16);

\path[draw=drawColor,line width= 0.4pt,line join=round,line cap=round,fill=fillColor] (186.13,372.42) circle (  1.16);

\path[draw=drawColor,line width= 0.4pt,line join=round,line cap=round,fill=fillColor] (186.34,372.42) circle (  1.16);

\path[draw=drawColor,line width= 0.4pt,line join=round,line cap=round,fill=fillColor] (186.55,372.42) circle (  1.16);

\path[draw=drawColor,line width= 0.4pt,line join=round,line cap=round,fill=fillColor] (186.75,372.42) circle (  1.16);

\path[draw=drawColor,line width= 0.4pt,line join=round,line cap=round,fill=fillColor] (186.95,372.42) circle (  1.16);

\path[draw=drawColor,line width= 0.4pt,line join=round,line cap=round,fill=fillColor] (187.16,372.42) circle (  1.16);

\path[draw=drawColor,line width= 0.4pt,line join=round,line cap=round,fill=fillColor] (187.36,372.42) circle (  1.16);

\path[draw=drawColor,line width= 0.4pt,line join=round,line cap=round,fill=fillColor] (187.56,372.42) circle (  1.16);

\path[draw=drawColor,line width= 0.4pt,line join=round,line cap=round,fill=fillColor] (187.76,372.42) circle (  1.16);

\path[draw=drawColor,line width= 0.4pt,line join=round,line cap=round,fill=fillColor] (187.96,372.42) circle (  1.16);

\path[draw=drawColor,line width= 0.4pt,line join=round,line cap=round,fill=fillColor] (188.16,372.42) circle (  1.16);

\path[draw=drawColor,line width= 0.4pt,line join=round,line cap=round,fill=fillColor] (188.36,372.42) circle (  1.16);

\path[draw=drawColor,line width= 0.4pt,line join=round,line cap=round,fill=fillColor] (188.56,372.42) circle (  1.16);

\path[draw=drawColor,line width= 0.4pt,line join=round,line cap=round,fill=fillColor] (188.76,372.42) circle (  1.16);

\path[draw=drawColor,line width= 0.4pt,line join=round,line cap=round,fill=fillColor] (188.96,372.42) circle (  1.16);

\path[draw=drawColor,line width= 0.4pt,line join=round,line cap=round,fill=fillColor] (189.16,372.42) circle (  1.16);

\path[draw=drawColor,line width= 0.4pt,line join=round,line cap=round,fill=fillColor] (189.35,372.42) circle (  1.16);

\path[draw=drawColor,line width= 0.4pt,line join=round,line cap=round,fill=fillColor] (189.55,372.42) circle (  1.16);

\path[draw=drawColor,line width= 0.4pt,line join=round,line cap=round,fill=fillColor] (189.74,372.42) circle (  1.16);

\path[draw=drawColor,line width= 0.4pt,line join=round,line cap=round,fill=fillColor] (189.94,372.42) circle (  1.16);

\path[draw=drawColor,line width= 0.4pt,line join=round,line cap=round,fill=fillColor] (190.13,372.42) circle (  1.16);

\path[draw=drawColor,line width= 0.4pt,line join=round,line cap=round,fill=fillColor] (190.33,372.42) circle (  1.16);

\path[draw=drawColor,line width= 0.4pt,line join=round,line cap=round,fill=fillColor] (190.52,372.42) circle (  1.16);

\path[draw=drawColor,line width= 0.4pt,line join=round,line cap=round,fill=fillColor] (190.71,372.42) circle (  1.16);

\path[draw=drawColor,line width= 0.4pt,line join=round,line cap=round,fill=fillColor] (190.91,372.42) circle (  1.16);

\path[draw=drawColor,line width= 0.4pt,line join=round,line cap=round,fill=fillColor] (191.10,372.42) circle (  1.16);

\path[draw=drawColor,line width= 0.4pt,line join=round,line cap=round,fill=fillColor] (191.29,372.42) circle (  1.16);

\path[draw=drawColor,line width= 0.4pt,line join=round,line cap=round,fill=fillColor] (191.48,372.42) circle (  1.16);

\path[draw=drawColor,line width= 0.4pt,line join=round,line cap=round,fill=fillColor] (191.67,372.42) circle (  1.16);

\path[draw=drawColor,line width= 0.4pt,line join=round,line cap=round,fill=fillColor] (191.86,372.42) circle (  1.16);

\path[draw=drawColor,line width= 0.4pt,line join=round,line cap=round,fill=fillColor] (192.05,372.42) circle (  1.16);

\path[draw=drawColor,line width= 0.4pt,line join=round,line cap=round,fill=fillColor] (192.24,372.42) circle (  1.16);

\path[draw=drawColor,line width= 0.4pt,line join=round,line cap=round,fill=fillColor] (192.43,372.42) circle (  1.16);

\path[draw=drawColor,line width= 0.4pt,line join=round,line cap=round,fill=fillColor] (192.61,372.42) circle (  1.16);

\path[draw=drawColor,line width= 0.4pt,line join=round,line cap=round,fill=fillColor] (192.80,372.42) circle (  1.16);

\path[draw=drawColor,line width= 0.4pt,line join=round,line cap=round,fill=fillColor] (192.99,372.42) circle (  1.16);

\path[draw=drawColor,line width= 0.4pt,line join=round,line cap=round,fill=fillColor] (193.17,372.42) circle (  1.16);

\path[draw=drawColor,line width= 0.4pt,line join=round,line cap=round,fill=fillColor] (193.36,372.42) circle (  1.16);

\path[draw=drawColor,line width= 0.4pt,line join=round,line cap=round,fill=fillColor] (193.54,372.42) circle (  1.16);

\path[draw=drawColor,line width= 0.4pt,line join=round,line cap=round,fill=fillColor] (193.73,372.42) circle (  1.16);

\path[draw=drawColor,line width= 0.4pt,line join=round,line cap=round,fill=fillColor] (193.91,372.42) circle (  1.16);

\path[draw=drawColor,line width= 0.4pt,line join=round,line cap=round,fill=fillColor] (194.10,372.42) circle (  1.16);

\path[draw=drawColor,line width= 0.4pt,line join=round,line cap=round,fill=fillColor] (194.28,372.42) circle (  1.16);

\path[draw=drawColor,line width= 0.4pt,line join=round,line cap=round,fill=fillColor] (194.46,372.42) circle (  1.16);

\path[draw=drawColor,line width= 0.4pt,line join=round,line cap=round,fill=fillColor] (194.65,372.42) circle (  1.16);

\path[draw=drawColor,line width= 0.4pt,line join=round,line cap=round,fill=fillColor] (194.83,372.42) circle (  1.16);

\path[draw=drawColor,line width= 0.4pt,line join=round,line cap=round,fill=fillColor] (195.01,372.42) circle (  1.16);

\path[draw=drawColor,line width= 0.4pt,line join=round,line cap=round,fill=fillColor] (195.19,372.42) circle (  1.16);

\path[draw=drawColor,line width= 0.4pt,line join=round,line cap=round,fill=fillColor] (195.37,372.42) circle (  1.16);

\path[draw=drawColor,line width= 0.4pt,line join=round,line cap=round,fill=fillColor] (195.55,372.42) circle (  1.16);

\path[draw=drawColor,line width= 0.4pt,line join=round,line cap=round,fill=fillColor] (195.73,372.42) circle (  1.16);

\path[draw=drawColor,line width= 0.4pt,line join=round,line cap=round,fill=fillColor] (195.91,372.42) circle (  1.16);

\path[draw=drawColor,line width= 0.4pt,line join=round,line cap=round,fill=fillColor] (196.09,372.42) circle (  1.16);

\path[draw=drawColor,line width= 0.4pt,line join=round,line cap=round,fill=fillColor] (196.27,372.42) circle (  1.16);

\path[draw=drawColor,line width= 0.4pt,line join=round,line cap=round,fill=fillColor] (196.44,372.42) circle (  1.16);

\path[draw=drawColor,line width= 0.4pt,line join=round,line cap=round,fill=fillColor] (196.62,372.42) circle (  1.16);

\path[draw=drawColor,line width= 0.4pt,line join=round,line cap=round,fill=fillColor] (196.80,372.42) circle (  1.16);

\path[draw=drawColor,line width= 0.4pt,line join=round,line cap=round,fill=fillColor] (196.97,372.42) circle (  1.16);

\path[draw=drawColor,line width= 0.4pt,line join=round,line cap=round,fill=fillColor] (197.15,372.42) circle (  1.16);

\path[draw=drawColor,line width= 0.4pt,line join=round,line cap=round,fill=fillColor] (197.33,372.42) circle (  1.16);

\path[draw=drawColor,line width= 0.4pt,line join=round,line cap=round,fill=fillColor] (197.50,372.42) circle (  1.16);

\path[draw=drawColor,line width= 0.4pt,line join=round,line cap=round,fill=fillColor] (197.68,372.42) circle (  1.16);

\path[draw=drawColor,line width= 0.4pt,line join=round,line cap=round,fill=fillColor] (197.85,372.42) circle (  1.16);

\path[draw=drawColor,line width= 0.4pt,line join=round,line cap=round,fill=fillColor] (198.02,372.42) circle (  1.16);

\path[draw=drawColor,line width= 0.4pt,line join=round,line cap=round,fill=fillColor] (198.20,372.42) circle (  1.16);

\path[draw=drawColor,line width= 0.4pt,line join=round,line cap=round,fill=fillColor] (198.37,372.42) circle (  1.16);

\path[draw=drawColor,line width= 0.4pt,line join=round,line cap=round,fill=fillColor] (198.54,372.42) circle (  1.16);

\path[draw=drawColor,line width= 0.4pt,line join=round,line cap=round,fill=fillColor] (198.72,372.42) circle (  1.16);

\path[draw=drawColor,line width= 0.4pt,line join=round,line cap=round,fill=fillColor] (198.89,372.42) circle (  1.16);

\path[draw=drawColor,line width= 0.4pt,line join=round,line cap=round,fill=fillColor] (199.06,372.42) circle (  1.16);

\path[draw=drawColor,line width= 0.4pt,line join=round,line cap=round,fill=fillColor] (199.23,372.42) circle (  1.16);

\path[draw=drawColor,line width= 0.4pt,line join=round,line cap=round,fill=fillColor] (199.40,372.42) circle (  1.16);

\path[draw=drawColor,line width= 0.4pt,line join=round,line cap=round,fill=fillColor] (199.57,372.42) circle (  1.16);

\path[draw=drawColor,line width= 0.4pt,line join=round,line cap=round,fill=fillColor] (199.74,372.42) circle (  1.16);

\path[draw=drawColor,line width= 0.4pt,line join=round,line cap=round,fill=fillColor] (199.91,372.42) circle (  1.16);

\path[draw=drawColor,line width= 0.4pt,line join=round,line cap=round,fill=fillColor] (200.08,372.42) circle (  1.16);

\path[draw=drawColor,line width= 0.4pt,line join=round,line cap=round,fill=fillColor] (200.25,372.42) circle (  1.16);

\path[draw=drawColor,line width= 0.4pt,line join=round,line cap=round,fill=fillColor] (200.42,372.42) circle (  1.16);

\path[draw=drawColor,line width= 0.4pt,line join=round,line cap=round,fill=fillColor] (200.58,372.42) circle (  1.16);

\path[draw=drawColor,line width= 0.4pt,line join=round,line cap=round,fill=fillColor] (200.75,372.42) circle (  1.16);

\path[draw=drawColor,line width= 0.4pt,line join=round,line cap=round,fill=fillColor] (200.92,372.42) circle (  1.16);

\path[draw=drawColor,line width= 0.4pt,line join=round,line cap=round,fill=fillColor] (201.09,372.42) circle (  1.16);

\path[draw=drawColor,line width= 0.4pt,line join=round,line cap=round,fill=fillColor] (201.25,372.42) circle (  1.16);

\path[draw=drawColor,line width= 0.4pt,line join=round,line cap=round,fill=fillColor] (201.42,372.42) circle (  1.16);

\path[draw=drawColor,line width= 0.4pt,line join=round,line cap=round,fill=fillColor] (201.58,372.42) circle (  1.16);

\path[draw=drawColor,line width= 0.4pt,line join=round,line cap=round,fill=fillColor] (201.75,372.42) circle (  1.16);

\path[draw=drawColor,line width= 0.4pt,line join=round,line cap=round,fill=fillColor] (201.91,372.42) circle (  1.16);

\path[draw=drawColor,line width= 0.4pt,line join=round,line cap=round,fill=fillColor] (202.08,372.42) circle (  1.16);

\path[draw=drawColor,line width= 0.4pt,line join=round,line cap=round,fill=fillColor] (202.24,372.42) circle (  1.16);

\path[draw=drawColor,line width= 0.4pt,line join=round,line cap=round,fill=fillColor] (202.41,372.42) circle (  1.16);

\path[draw=drawColor,line width= 0.4pt,line join=round,line cap=round,fill=fillColor] (202.57,372.42) circle (  1.16);

\path[draw=drawColor,line width= 0.4pt,line join=round,line cap=round,fill=fillColor] (202.73,372.42) circle (  1.16);

\path[draw=drawColor,line width= 0.4pt,line join=round,line cap=round,fill=fillColor] (202.90,372.42) circle (  1.16);

\path[draw=drawColor,line width= 0.4pt,line join=round,line cap=round,fill=fillColor] (203.06,372.42) circle (  1.16);

\path[draw=drawColor,line width= 0.4pt,line join=round,line cap=round,fill=fillColor] (203.22,372.42) circle (  1.16);

\path[draw=drawColor,line width= 0.4pt,line join=round,line cap=round,fill=fillColor] (203.38,372.42) circle (  1.16);

\path[draw=drawColor,line width= 0.4pt,line join=round,line cap=round,fill=fillColor] (203.54,372.42) circle (  1.16);

\path[draw=drawColor,line width= 0.4pt,line join=round,line cap=round,fill=fillColor] (203.71,372.42) circle (  1.16);

\path[draw=drawColor,line width= 0.4pt,line join=round,line cap=round,fill=fillColor] (203.87,372.42) circle (  1.16);

\path[draw=drawColor,line width= 0.4pt,line join=round,line cap=round,fill=fillColor] (204.03,372.42) circle (  1.16);

\path[draw=drawColor,line width= 0.4pt,line join=round,line cap=round,fill=fillColor] (204.19,372.42) circle (  1.16);

\path[draw=drawColor,line width= 0.4pt,line join=round,line cap=round,fill=fillColor] (204.35,372.42) circle (  1.16);

\path[draw=drawColor,line width= 0.4pt,line join=round,line cap=round,fill=fillColor] (204.51,372.42) circle (  1.16);

\path[draw=drawColor,line width= 0.4pt,line join=round,line cap=round,fill=fillColor] (204.66,372.42) circle (  1.16);

\path[draw=drawColor,line width= 0.4pt,line join=round,line cap=round,fill=fillColor] (204.82,372.42) circle (  1.16);

\path[draw=drawColor,line width= 0.4pt,line join=round,line cap=round,fill=fillColor] (204.98,372.42) circle (  1.16);

\path[draw=drawColor,line width= 0.4pt,line join=round,line cap=round,fill=fillColor] (205.14,372.42) circle (  1.16);

\path[draw=drawColor,line width= 0.4pt,line join=round,line cap=round,fill=fillColor] (205.30,372.42) circle (  1.16);

\path[draw=drawColor,line width= 0.4pt,line join=round,line cap=round,fill=fillColor] (205.45,372.42) circle (  1.16);

\path[draw=drawColor,line width= 0.4pt,line join=round,line cap=round,fill=fillColor] (205.61,372.42) circle (  1.16);

\path[draw=drawColor,line width= 0.4pt,line join=round,line cap=round,fill=fillColor] (205.77,372.42) circle (  1.16);

\path[draw=drawColor,line width= 0.4pt,line join=round,line cap=round,fill=fillColor] (205.92,372.42) circle (  1.16);

\path[draw=drawColor,line width= 0.4pt,line join=round,line cap=round,fill=fillColor] (206.08,372.42) circle (  1.16);

\path[draw=drawColor,line width= 0.4pt,line join=round,line cap=round,fill=fillColor] (206.24,372.42) circle (  1.16);

\path[draw=drawColor,line width= 0.4pt,line join=round,line cap=round,fill=fillColor] (206.39,372.42) circle (  1.16);

\path[draw=drawColor,line width= 0.4pt,line join=round,line cap=round,fill=fillColor] (206.55,372.42) circle (  1.16);

\path[draw=drawColor,line width= 0.4pt,line join=round,line cap=round,fill=fillColor] (206.70,372.42) circle (  1.16);

\path[draw=drawColor,line width= 0.4pt,line join=round,line cap=round,fill=fillColor] (206.86,372.42) circle (  1.16);

\path[draw=drawColor,line width= 0.4pt,line join=round,line cap=round,fill=fillColor] (207.01,372.42) circle (  1.16);

\path[draw=drawColor,line width= 0.4pt,line join=round,line cap=round,fill=fillColor] (207.16,372.42) circle (  1.16);

\path[draw=drawColor,line width= 0.4pt,line join=round,line cap=round,fill=fillColor] (207.32,372.42) circle (  1.16);

\path[draw=drawColor,line width= 0.4pt,line join=round,line cap=round,fill=fillColor] (207.47,372.42) circle (  1.16);

\path[draw=drawColor,line width= 0.4pt,line join=round,line cap=round,fill=fillColor] (207.62,372.42) circle (  1.16);

\path[draw=drawColor,line width= 0.4pt,line join=round,line cap=round,fill=fillColor] (207.78,372.42) circle (  1.16);

\path[draw=drawColor,line width= 0.4pt,line join=round,line cap=round,fill=fillColor] (207.93,372.42) circle (  1.16);

\path[draw=drawColor,line width= 0.4pt,line join=round,line cap=round,fill=fillColor] (208.08,372.42) circle (  1.16);

\path[draw=drawColor,line width= 0.4pt,line join=round,line cap=round,fill=fillColor] (208.23,372.42) circle (  1.16);

\path[draw=drawColor,line width= 0.4pt,line join=round,line cap=round,fill=fillColor] (208.39,372.42) circle (  1.16);

\path[draw=drawColor,line width= 0.4pt,line join=round,line cap=round,fill=fillColor] (208.54,372.42) circle (  1.16);

\path[draw=drawColor,line width= 0.4pt,line join=round,line cap=round,fill=fillColor] (208.69,372.42) circle (  1.16);

\path[draw=drawColor,line width= 0.4pt,line join=round,line cap=round,fill=fillColor] (208.84,372.42) circle (  1.16);

\path[draw=drawColor,line width= 0.4pt,line join=round,line cap=round,fill=fillColor] (208.99,372.42) circle (  1.16);

\path[draw=drawColor,line width= 0.4pt,line join=round,line cap=round,fill=fillColor] (209.14,372.42) circle (  1.16);

\path[draw=drawColor,line width= 0.4pt,line join=round,line cap=round,fill=fillColor] (209.29,372.42) circle (  1.16);

\path[draw=drawColor,line width= 0.4pt,line join=round,line cap=round,fill=fillColor] (209.44,372.42) circle (  1.16);

\path[draw=drawColor,line width= 0.4pt,line join=round,line cap=round,fill=fillColor] (209.59,372.42) circle (  1.16);

\path[draw=drawColor,line width= 0.4pt,line join=round,line cap=round,fill=fillColor] (209.74,372.42) circle (  1.16);

\path[draw=drawColor,line width= 0.4pt,line join=round,line cap=round,fill=fillColor] (209.88,372.42) circle (  1.16);

\path[draw=drawColor,line width= 0.4pt,line join=round,line cap=round,fill=fillColor] (210.03,372.42) circle (  1.16);

\path[draw=drawColor,line width= 0.4pt,line join=round,line cap=round,fill=fillColor] (210.18,372.42) circle (  1.16);

\path[draw=drawColor,line width= 0.4pt,line join=round,line cap=round,fill=fillColor] (210.33,372.42) circle (  1.16);

\path[draw=drawColor,line width= 0.4pt,line join=round,line cap=round,fill=fillColor] (210.48,372.42) circle (  1.16);

\path[draw=drawColor,line width= 0.4pt,line join=round,line cap=round,fill=fillColor] (210.62,372.42) circle (  1.16);

\path[draw=drawColor,line width= 0.4pt,line join=round,line cap=round,fill=fillColor] (210.77,372.42) circle (  1.16);

\path[draw=drawColor,line width= 0.4pt,line join=round,line cap=round,fill=fillColor] (210.92,372.42) circle (  1.16);

\path[draw=drawColor,line width= 0.4pt,line join=round,line cap=round,fill=fillColor] (211.06,372.42) circle (  1.16);

\path[draw=drawColor,line width= 0.4pt,line join=round,line cap=round,fill=fillColor] (211.21,372.42) circle (  1.16);

\path[draw=drawColor,line width= 0.4pt,line join=round,line cap=round,fill=fillColor] (211.36,372.42) circle (  1.16);

\path[draw=drawColor,line width= 0.4pt,line join=round,line cap=round,fill=fillColor] (211.50,372.42) circle (  1.16);

\path[draw=drawColor,line width= 0.4pt,line join=round,line cap=round,fill=fillColor] (211.65,372.42) circle (  1.16);

\path[draw=drawColor,line width= 0.4pt,line join=round,line cap=round,fill=fillColor] (211.79,372.42) circle (  1.16);

\path[draw=drawColor,line width= 0.4pt,line join=round,line cap=round,fill=fillColor] (211.94,372.42) circle (  1.16);

\path[draw=drawColor,line width= 0.4pt,line join=round,line cap=round,fill=fillColor] (212.08,372.42) circle (  1.16);

\path[draw=drawColor,line width= 0.4pt,line join=round,line cap=round,fill=fillColor] (212.23,372.42) circle (  1.16);

\path[draw=drawColor,line width= 0.4pt,line join=round,line cap=round,fill=fillColor] (212.37,372.42) circle (  1.16);

\path[draw=drawColor,line width= 0.4pt,line join=round,line cap=round,fill=fillColor] (212.51,372.42) circle (  1.16);

\path[draw=drawColor,line width= 0.4pt,line join=round,line cap=round,fill=fillColor] (212.66,372.42) circle (  1.16);

\path[draw=drawColor,line width= 0.4pt,line join=round,line cap=round,fill=fillColor] (212.80,372.42) circle (  1.16);

\path[draw=drawColor,line width= 0.4pt,line join=round,line cap=round,fill=fillColor] (212.94,372.42) circle (  1.16);

\path[draw=drawColor,line width= 0.4pt,line join=round,line cap=round,fill=fillColor] (213.09,372.42) circle (  1.16);

\path[draw=drawColor,line width= 0.4pt,line join=round,line cap=round,fill=fillColor] (213.23,372.42) circle (  1.16);

\path[draw=drawColor,line width= 0.4pt,line join=round,line cap=round,fill=fillColor] (213.37,372.42) circle (  1.16);

\path[draw=drawColor,line width= 0.4pt,line join=round,line cap=round,fill=fillColor] (213.51,372.42) circle (  1.16);

\path[draw=drawColor,line width= 0.4pt,line join=round,line cap=round,fill=fillColor] (213.65,372.42) circle (  1.16);

\path[draw=drawColor,line width= 0.4pt,line join=round,line cap=round,fill=fillColor] (213.80,372.42) circle (  1.16);

\path[draw=drawColor,line width= 0.4pt,line join=round,line cap=round,fill=fillColor] (213.94,372.42) circle (  1.16);

\path[draw=drawColor,line width= 0.4pt,line join=round,line cap=round,fill=fillColor] (214.08,372.42) circle (  1.16);

\path[draw=drawColor,line width= 0.4pt,line join=round,line cap=round,fill=fillColor] (214.22,372.42) circle (  1.16);

\path[draw=drawColor,line width= 0.4pt,line join=round,line cap=round,fill=fillColor] (214.36,372.42) circle (  1.16);

\path[draw=drawColor,line width= 0.4pt,line join=round,line cap=round,fill=fillColor] (214.50,372.42) circle (  1.16);

\path[draw=drawColor,line width= 0.4pt,line join=round,line cap=round,fill=fillColor] (214.64,372.42) circle (  1.16);

\path[draw=drawColor,line width= 0.4pt,line join=round,line cap=round,fill=fillColor] (214.78,372.42) circle (  1.16);

\path[draw=drawColor,line width= 0.4pt,line join=round,line cap=round,fill=fillColor] (214.92,372.42) circle (  1.16);

\path[draw=drawColor,line width= 0.4pt,line join=round,line cap=round,fill=fillColor] (215.06,372.42) circle (  1.16);

\path[draw=drawColor,line width= 0.4pt,line join=round,line cap=round,fill=fillColor] (215.20,372.42) circle (  1.16);

\path[draw=drawColor,line width= 0.4pt,line join=round,line cap=round,fill=fillColor] (215.34,372.42) circle (  1.16);

\path[draw=drawColor,line width= 0.4pt,line join=round,line cap=round,fill=fillColor] (215.47,372.42) circle (  1.16);

\path[draw=drawColor,line width= 0.4pt,line join=round,line cap=round,fill=fillColor] (215.61,372.42) circle (  1.16);

\path[draw=drawColor,line width= 0.4pt,line join=round,line cap=round,fill=fillColor] (215.75,372.42) circle (  1.16);

\path[draw=drawColor,line width= 0.4pt,line join=round,line cap=round,fill=fillColor] (215.89,372.42) circle (  1.16);

\path[draw=drawColor,line width= 0.4pt,line join=round,line cap=round,fill=fillColor] (216.03,372.42) circle (  1.16);

\path[draw=drawColor,line width= 0.4pt,line join=round,line cap=round,fill=fillColor] (216.16,372.42) circle (  1.16);

\path[draw=drawColor,line width= 0.4pt,line join=round,line cap=round,fill=fillColor] (216.30,372.42) circle (  1.16);

\path[draw=drawColor,line width= 0.4pt,line join=round,line cap=round,fill=fillColor] (216.44,372.42) circle (  1.16);

\path[draw=drawColor,line width= 0.4pt,line join=round,line cap=round,fill=fillColor] (216.58,372.42) circle (  1.16);

\path[draw=drawColor,line width= 0.4pt,line join=round,line cap=round,fill=fillColor] (216.71,372.42) circle (  1.16);

\path[draw=drawColor,line width= 0.4pt,line join=round,line cap=round,fill=fillColor] (216.85,372.42) circle (  1.16);

\path[draw=drawColor,line width= 0.4pt,line join=round,line cap=round,fill=fillColor] (216.98,372.42) circle (  1.16);

\path[draw=drawColor,line width= 0.4pt,line join=round,line cap=round,fill=fillColor] (217.12,372.42) circle (  1.16);

\path[draw=drawColor,line width= 0.4pt,line join=round,line cap=round,fill=fillColor] (217.26,372.42) circle (  1.16);

\path[draw=drawColor,line width= 0.4pt,line join=round,line cap=round,fill=fillColor] (217.39,372.42) circle (  1.16);

\path[draw=drawColor,line width= 0.4pt,line join=round,line cap=round,fill=fillColor] (217.53,372.42) circle (  1.16);

\path[draw=drawColor,line width= 0.4pt,line join=round,line cap=round,fill=fillColor] (217.66,372.42) circle (  1.16);

\path[draw=drawColor,line width= 0.4pt,line join=round,line cap=round,fill=fillColor] (217.80,372.42) circle (  1.16);

\path[draw=drawColor,line width= 0.4pt,line join=round,line cap=round,fill=fillColor] (217.93,372.42) circle (  1.16);

\path[draw=drawColor,line width= 0.4pt,line join=round,line cap=round,fill=fillColor] (218.06,372.42) circle (  1.16);

\path[draw=drawColor,line width= 0.4pt,line join=round,line cap=round,fill=fillColor] (218.20,372.42) circle (  1.16);

\path[draw=drawColor,line width= 0.4pt,line join=round,line cap=round,fill=fillColor] (218.33,372.42) circle (  1.16);

\path[draw=drawColor,line width= 0.4pt,line join=round,line cap=round,fill=fillColor] (218.47,372.42) circle (  1.16);

\path[draw=drawColor,line width= 0.4pt,line join=round,line cap=round,fill=fillColor] (218.60,372.42) circle (  1.16);

\path[draw=drawColor,line width= 0.4pt,line join=round,line cap=round,fill=fillColor] (218.73,372.42) circle (  1.16);

\path[draw=drawColor,line width= 0.4pt,line join=round,line cap=round,fill=fillColor] (218.87,372.42) circle (  1.16);

\path[draw=drawColor,line width= 0.4pt,line join=round,line cap=round,fill=fillColor] (219.00,372.42) circle (  1.16);

\path[draw=drawColor,line width= 0.4pt,line join=round,line cap=round,fill=fillColor] (219.13,372.42) circle (  1.16);

\path[draw=drawColor,line width= 0.4pt,line join=round,line cap=round,fill=fillColor] (219.26,372.42) circle (  1.16);

\path[draw=drawColor,line width= 0.4pt,line join=round,line cap=round,fill=fillColor] (219.40,372.42) circle (  1.16);

\path[draw=drawColor,line width= 0.4pt,line join=round,line cap=round,fill=fillColor] (219.53,372.42) circle (  1.16);

\path[draw=drawColor,line width= 0.4pt,line join=round,line cap=round,fill=fillColor] (219.66,372.42) circle (  1.16);

\path[draw=drawColor,line width= 0.4pt,line join=round,line cap=round,fill=fillColor] (219.79,372.42) circle (  1.16);

\path[draw=drawColor,line width= 0.4pt,line join=round,line cap=round,fill=fillColor] (219.92,372.42) circle (  1.16);

\path[draw=drawColor,line width= 0.4pt,line join=round,line cap=round,fill=fillColor] (220.05,372.42) circle (  1.16);

\path[draw=drawColor,line width= 0.4pt,line join=round,line cap=round,fill=fillColor] (220.18,372.42) circle (  1.16);

\path[draw=drawColor,line width= 0.4pt,line join=round,line cap=round,fill=fillColor] (220.31,372.42) circle (  1.16);

\path[draw=drawColor,line width= 0.4pt,line join=round,line cap=round,fill=fillColor] (220.45,372.42) circle (  1.16);

\path[draw=drawColor,line width= 0.4pt,line join=round,line cap=round,fill=fillColor] (220.58,372.42) circle (  1.16);

\path[draw=drawColor,line width= 0.4pt,line join=round,line cap=round,fill=fillColor] (220.71,372.42) circle (  1.16);

\path[draw=drawColor,line width= 0.4pt,line join=round,line cap=round,fill=fillColor] (220.84,372.42) circle (  1.16);

\path[draw=drawColor,line width= 0.4pt,line join=round,line cap=round,fill=fillColor] (220.97,372.42) circle (  1.16);

\path[draw=drawColor,line width= 0.4pt,line join=round,line cap=round,fill=fillColor] (221.09,372.42) circle (  1.16);

\path[draw=drawColor,line width= 0.4pt,line join=round,line cap=round,fill=fillColor] (221.22,372.42) circle (  1.16);

\path[draw=drawColor,line width= 0.4pt,line join=round,line cap=round,fill=fillColor] (221.35,372.42) circle (  1.16);

\path[draw=drawColor,line width= 0.4pt,line join=round,line cap=round,fill=fillColor] (221.48,372.42) circle (  1.16);

\path[draw=drawColor,line width= 0.4pt,line join=round,line cap=round,fill=fillColor] (221.61,372.42) circle (  1.16);

\path[draw=drawColor,line width= 0.4pt,line join=round,line cap=round,fill=fillColor] (221.74,372.42) circle (  1.16);

\path[draw=drawColor,line width= 0.4pt,line join=round,line cap=round,fill=fillColor] (221.87,372.42) circle (  1.16);

\path[draw=drawColor,line width= 0.4pt,line join=round,line cap=round,fill=fillColor] (222.00,372.42) circle (  1.16);

\path[draw=drawColor,line width= 0.4pt,line join=round,line cap=round,fill=fillColor] (222.12,372.42) circle (  1.16);

\path[draw=drawColor,line width= 0.4pt,line join=round,line cap=round,fill=fillColor] (222.25,372.42) circle (  1.16);

\path[draw=drawColor,line width= 0.4pt,line join=round,line cap=round,fill=fillColor] (222.38,372.42) circle (  1.16);

\path[draw=drawColor,line width= 0.4pt,line join=round,line cap=round,fill=fillColor] (222.51,372.42) circle (  1.16);

\path[draw=drawColor,line width= 0.4pt,line join=round,line cap=round,fill=fillColor] (222.63,372.42) circle (  1.16);

\path[draw=drawColor,line width= 0.4pt,line join=round,line cap=round,fill=fillColor] (222.76,372.42) circle (  1.16);

\path[draw=drawColor,line width= 0.4pt,line join=round,line cap=round,fill=fillColor] (222.89,372.42) circle (  1.16);

\path[draw=drawColor,line width= 0.4pt,line join=round,line cap=round,fill=fillColor] (223.01,372.42) circle (  1.16);

\path[draw=drawColor,line width= 0.4pt,line join=round,line cap=round,fill=fillColor] (223.14,372.42) circle (  1.16);

\path[draw=drawColor,line width= 0.4pt,line join=round,line cap=round,fill=fillColor] (223.27,372.42) circle (  1.16);

\path[draw=drawColor,line width= 0.4pt,line join=round,line cap=round,fill=fillColor] (223.39,372.42) circle (  1.16);

\path[draw=drawColor,line width= 0.4pt,line join=round,line cap=round,fill=fillColor] (223.52,372.42) circle (  1.16);

\path[draw=drawColor,line width= 0.4pt,line join=round,line cap=round,fill=fillColor] (223.64,372.42) circle (  1.16);

\path[draw=drawColor,line width= 0.4pt,line join=round,line cap=round,fill=fillColor] (223.77,372.42) circle (  1.16);

\path[draw=drawColor,line width= 0.4pt,line join=round,line cap=round,fill=fillColor] (223.89,372.42) circle (  1.16);

\path[draw=drawColor,line width= 0.4pt,line join=round,line cap=round,fill=fillColor] (224.02,372.42) circle (  1.16);

\path[draw=drawColor,line width= 0.4pt,line join=round,line cap=round,fill=fillColor] (224.14,372.42) circle (  1.16);

\path[draw=drawColor,line width= 0.4pt,line join=round,line cap=round,fill=fillColor] (224.27,372.42) circle (  1.16);

\path[draw=drawColor,line width= 0.4pt,line join=round,line cap=round,fill=fillColor] (224.39,372.42) circle (  1.16);

\path[draw=drawColor,line width= 0.4pt,line join=round,line cap=round,fill=fillColor] (224.52,372.42) circle (  1.16);

\path[draw=drawColor,line width= 0.4pt,line join=round,line cap=round,fill=fillColor] (224.64,372.42) circle (  1.16);

\path[draw=drawColor,line width= 0.4pt,line join=round,line cap=round,fill=fillColor] (224.77,372.42) circle (  1.16);

\path[draw=drawColor,line width= 0.4pt,line join=round,line cap=round,fill=fillColor] (224.89,372.42) circle (  1.16);

\path[draw=drawColor,line width= 0.4pt,line join=round,line cap=round,fill=fillColor] (225.01,372.42) circle (  1.16);

\path[draw=drawColor,line width= 0.4pt,line join=round,line cap=round,fill=fillColor] (225.14,372.42) circle (  1.16);

\path[draw=drawColor,line width= 0.4pt,line join=round,line cap=round,fill=fillColor] (225.26,372.42) circle (  1.16);
\definecolor[named]{drawColor}{rgb}{0.22,0.49,0.72}
\definecolor[named]{fillColor}{rgb}{0.22,0.49,0.72}

\path[draw=drawColor,line width= 0.4pt,line join=round,line cap=round,fill=fillColor] ( 74.88,455.16) circle (  1.16);

\path[draw=drawColor,line width= 0.4pt,line join=round,line cap=round,fill=fillColor] ( 80.66,454.17) circle (  1.16);

\path[draw=drawColor,line width= 0.4pt,line join=round,line cap=round,fill=fillColor] ( 84.72,453.93) circle (  1.16);

\path[draw=drawColor,line width= 0.4pt,line join=round,line cap=round,fill=fillColor] ( 87.95,453.88) circle (  1.16);

\path[draw=drawColor,line width= 0.4pt,line join=round,line cap=round,fill=fillColor] ( 90.68,453.42) circle (  1.16);

\path[draw=drawColor,line width= 0.4pt,line join=round,line cap=round,fill=fillColor] ( 93.06,453.34) circle (  1.16);

\path[draw=drawColor,line width= 0.4pt,line join=round,line cap=round,fill=fillColor] ( 95.19,453.27) circle (  1.16);

\path[draw=drawColor,line width= 0.4pt,line join=round,line cap=round,fill=fillColor] ( 97.13,453.07) circle (  1.16);

\path[draw=drawColor,line width= 0.4pt,line join=round,line cap=round,fill=fillColor] ( 98.91,453.04) circle (  1.16);

\path[draw=drawColor,line width= 0.4pt,line join=round,line cap=round,fill=fillColor] (100.57,452.59) circle (  1.16);

\path[draw=drawColor,line width= 0.4pt,line join=round,line cap=round,fill=fillColor] (102.11,450.19) circle (  1.16);

\path[draw=drawColor,line width= 0.4pt,line join=round,line cap=round,fill=fillColor] (103.57,449.57) circle (  1.16);

\path[draw=drawColor,line width= 0.4pt,line join=round,line cap=round,fill=fillColor] (104.95,449.44) circle (  1.16);

\path[draw=drawColor,line width= 0.4pt,line join=round,line cap=round,fill=fillColor] (106.26,449.42) circle (  1.16);

\path[draw=drawColor,line width= 0.4pt,line join=round,line cap=round,fill=fillColor] (107.50,449.38) circle (  1.16);

\path[draw=drawColor,line width= 0.4pt,line join=round,line cap=round,fill=fillColor] (108.70,449.36) circle (  1.16);

\path[draw=drawColor,line width= 0.4pt,line join=round,line cap=round,fill=fillColor] (109.84,449.34) circle (  1.16);

\path[draw=drawColor,line width= 0.4pt,line join=round,line cap=round,fill=fillColor] (110.94,449.32) circle (  1.16);

\path[draw=drawColor,line width= 0.4pt,line join=round,line cap=round,fill=fillColor] (112.00,448.89) circle (  1.16);

\path[draw=drawColor,line width= 0.4pt,line join=round,line cap=round,fill=fillColor] (113.03,448.60) circle (  1.16);

\path[draw=drawColor,line width= 0.4pt,line join=round,line cap=round,fill=fillColor] (114.02,448.18) circle (  1.16);

\path[draw=drawColor,line width= 0.4pt,line join=round,line cap=round,fill=fillColor] (114.98,447.73) circle (  1.16);

\path[draw=drawColor,line width= 0.4pt,line join=round,line cap=round,fill=fillColor] (115.91,447.40) circle (  1.16);

\path[draw=drawColor,line width= 0.4pt,line join=round,line cap=round,fill=fillColor] (116.81,447.00) circle (  1.16);

\path[draw=drawColor,line width= 0.4pt,line join=round,line cap=round,fill=fillColor] (117.69,445.61) circle (  1.16);

\path[draw=drawColor,line width= 0.4pt,line join=round,line cap=round,fill=fillColor] (118.55,444.93) circle (  1.16);

\path[draw=drawColor,line width= 0.4pt,line join=round,line cap=round,fill=fillColor] (119.38,442.92) circle (  1.16);

\path[draw=drawColor,line width= 0.4pt,line join=round,line cap=round,fill=fillColor] (120.20,441.71) circle (  1.16);

\path[draw=drawColor,line width= 0.4pt,line join=round,line cap=round,fill=fillColor] (120.99,441.64) circle (  1.16);

\path[draw=drawColor,line width= 0.4pt,line join=round,line cap=round,fill=fillColor] (121.77,438.59) circle (  1.16);

\path[draw=drawColor,line width= 0.4pt,line join=round,line cap=round,fill=fillColor] (122.53,438.59) circle (  1.16);

\path[draw=drawColor,line width= 0.4pt,line join=round,line cap=round,fill=fillColor] (123.27,438.04) circle (  1.16);

\path[draw=drawColor,line width= 0.4pt,line join=round,line cap=round,fill=fillColor] (124.00,437.95) circle (  1.16);

\path[draw=drawColor,line width= 0.4pt,line join=round,line cap=round,fill=fillColor] (124.71,437.78) circle (  1.16);

\path[draw=drawColor,line width= 0.4pt,line join=round,line cap=round,fill=fillColor] (125.41,434.62) circle (  1.16);

\path[draw=drawColor,line width= 0.4pt,line join=round,line cap=round,fill=fillColor] (126.10,432.36) circle (  1.16);

\path[draw=drawColor,line width= 0.4pt,line join=round,line cap=round,fill=fillColor] (126.77,432.18) circle (  1.16);

\path[draw=drawColor,line width= 0.4pt,line join=round,line cap=round,fill=fillColor] (127.44,431.74) circle (  1.16);

\path[draw=drawColor,line width= 0.4pt,line join=round,line cap=round,fill=fillColor] (128.09,431.50) circle (  1.16);

\path[draw=drawColor,line width= 0.4pt,line join=round,line cap=round,fill=fillColor] (128.73,431.29) circle (  1.16);

\path[draw=drawColor,line width= 0.4pt,line join=round,line cap=round,fill=fillColor] (129.35,430.31) circle (  1.16);

\path[draw=drawColor,line width= 0.4pt,line join=round,line cap=round,fill=fillColor] (129.97,429.31) circle (  1.16);

\path[draw=drawColor,line width= 0.4pt,line join=round,line cap=round,fill=fillColor] (130.58,428.02) circle (  1.16);

\path[draw=drawColor,line width= 0.4pt,line join=round,line cap=round,fill=fillColor] (131.18,427.85) circle (  1.16);

\path[draw=drawColor,line width= 0.4pt,line join=round,line cap=round,fill=fillColor] (131.77,425.78) circle (  1.16);

\path[draw=drawColor,line width= 0.4pt,line join=round,line cap=round,fill=fillColor] (132.35,425.29) circle (  1.16);

\path[draw=drawColor,line width= 0.4pt,line join=round,line cap=round,fill=fillColor] (132.93,424.86) circle (  1.16);

\path[draw=drawColor,line width= 0.4pt,line join=round,line cap=round,fill=fillColor] (133.49,424.85) circle (  1.16);

\path[draw=drawColor,line width= 0.4pt,line join=round,line cap=round,fill=fillColor] (134.05,424.33) circle (  1.16);

\path[draw=drawColor,line width= 0.4pt,line join=round,line cap=round,fill=fillColor] (134.60,423.67) circle (  1.16);

\path[draw=drawColor,line width= 0.4pt,line join=round,line cap=round,fill=fillColor] (135.14,422.95) circle (  1.16);

\path[draw=drawColor,line width= 0.4pt,line join=round,line cap=round,fill=fillColor] (135.68,422.71) circle (  1.16);

\path[draw=drawColor,line width= 0.4pt,line join=round,line cap=round,fill=fillColor] (136.21,422.32) circle (  1.16);

\path[draw=drawColor,line width= 0.4pt,line join=round,line cap=round,fill=fillColor] (136.73,422.14) circle (  1.16);

\path[draw=drawColor,line width= 0.4pt,line join=round,line cap=round,fill=fillColor] (137.25,421.50) circle (  1.16);

\path[draw=drawColor,line width= 0.4pt,line join=round,line cap=round,fill=fillColor] (137.76,419.91) circle (  1.16);

\path[draw=drawColor,line width= 0.4pt,line join=round,line cap=round,fill=fillColor] (138.26,419.55) circle (  1.16);

\path[draw=drawColor,line width= 0.4pt,line join=round,line cap=round,fill=fillColor] (138.76,419.01) circle (  1.16);

\path[draw=drawColor,line width= 0.4pt,line join=round,line cap=round,fill=fillColor] (139.25,418.59) circle (  1.16);

\path[draw=drawColor,line width= 0.4pt,line join=round,line cap=round,fill=fillColor] (139.74,417.75) circle (  1.16);

\path[draw=drawColor,line width= 0.4pt,line join=round,line cap=round,fill=fillColor] (140.22,417.59) circle (  1.16);

\path[draw=drawColor,line width= 0.4pt,line join=round,line cap=round,fill=fillColor] (140.70,417.46) circle (  1.16);

\path[draw=drawColor,line width= 0.4pt,line join=round,line cap=round,fill=fillColor] (141.17,417.30) circle (  1.16);

\path[draw=drawColor,line width= 0.4pt,line join=round,line cap=round,fill=fillColor] (141.63,417.22) circle (  1.16);

\path[draw=drawColor,line width= 0.4pt,line join=round,line cap=round,fill=fillColor] (142.09,417.15) circle (  1.16);

\path[draw=drawColor,line width= 0.4pt,line join=round,line cap=round,fill=fillColor] (142.55,417.09) circle (  1.16);

\path[draw=drawColor,line width= 0.4pt,line join=round,line cap=round,fill=fillColor] (143.00,417.07) circle (  1.16);

\path[draw=drawColor,line width= 0.4pt,line join=round,line cap=round,fill=fillColor] (143.45,416.96) circle (  1.16);

\path[draw=drawColor,line width= 0.4pt,line join=round,line cap=round,fill=fillColor] (143.89,416.34) circle (  1.16);

\path[draw=drawColor,line width= 0.4pt,line join=round,line cap=round,fill=fillColor] (144.33,415.79) circle (  1.16);

\path[draw=drawColor,line width= 0.4pt,line join=round,line cap=round,fill=fillColor] (144.77,415.68) circle (  1.16);

\path[draw=drawColor,line width= 0.4pt,line join=round,line cap=round,fill=fillColor] (145.20,415.28) circle (  1.16);

\path[draw=drawColor,line width= 0.4pt,line join=round,line cap=round,fill=fillColor] (145.62,415.16) circle (  1.16);

\path[draw=drawColor,line width= 0.4pt,line join=round,line cap=round,fill=fillColor] (146.05,415.16) circle (  1.16);

\path[draw=drawColor,line width= 0.4pt,line join=round,line cap=round,fill=fillColor] (146.46,414.93) circle (  1.16);

\path[draw=drawColor,line width= 0.4pt,line join=round,line cap=round,fill=fillColor] (146.88,414.91) circle (  1.16);

\path[draw=drawColor,line width= 0.4pt,line join=round,line cap=round,fill=fillColor] (147.29,414.89) circle (  1.16);

\path[draw=drawColor,line width= 0.4pt,line join=round,line cap=round,fill=fillColor] (147.70,414.88) circle (  1.16);

\path[draw=drawColor,line width= 0.4pt,line join=round,line cap=round,fill=fillColor] (148.10,414.87) circle (  1.16);

\path[draw=drawColor,line width= 0.4pt,line join=round,line cap=round,fill=fillColor] (148.50,414.87) circle (  1.16);

\path[draw=drawColor,line width= 0.4pt,line join=round,line cap=round,fill=fillColor] (148.90,414.68) circle (  1.16);

\path[draw=drawColor,line width= 0.4pt,line join=round,line cap=round,fill=fillColor] (149.30,414.30) circle (  1.16);

\path[draw=drawColor,line width= 0.4pt,line join=round,line cap=round,fill=fillColor] (149.69,414.28) circle (  1.16);

\path[draw=drawColor,line width= 0.4pt,line join=round,line cap=round,fill=fillColor] (150.08,414.10) circle (  1.16);

\path[draw=drawColor,line width= 0.4pt,line join=round,line cap=round,fill=fillColor] (150.46,414.09) circle (  1.16);

\path[draw=drawColor,line width= 0.4pt,line join=round,line cap=round,fill=fillColor] (150.84,414.08) circle (  1.16);

\path[draw=drawColor,line width= 0.4pt,line join=round,line cap=round,fill=fillColor] (151.22,413.78) circle (  1.16);

\path[draw=drawColor,line width= 0.4pt,line join=round,line cap=round,fill=fillColor] (151.60,413.60) circle (  1.16);

\path[draw=drawColor,line width= 0.4pt,line join=round,line cap=round,fill=fillColor] (151.97,413.43) circle (  1.16);

\path[draw=drawColor,line width= 0.4pt,line join=round,line cap=round,fill=fillColor] (152.34,413.10) circle (  1.16);

\path[draw=drawColor,line width= 0.4pt,line join=round,line cap=round,fill=fillColor] (152.71,412.64) circle (  1.16);

\path[draw=drawColor,line width= 0.4pt,line join=round,line cap=round,fill=fillColor] (153.08,412.57) circle (  1.16);

\path[draw=drawColor,line width= 0.4pt,line join=round,line cap=round,fill=fillColor] (153.44,412.56) circle (  1.16);

\path[draw=drawColor,line width= 0.4pt,line join=round,line cap=round,fill=fillColor] (153.80,412.54) circle (  1.16);

\path[draw=drawColor,line width= 0.4pt,line join=round,line cap=round,fill=fillColor] (154.16,412.20) circle (  1.16);

\path[draw=drawColor,line width= 0.4pt,line join=round,line cap=round,fill=fillColor] (154.51,411.69) circle (  1.16);

\path[draw=drawColor,line width= 0.4pt,line join=round,line cap=round,fill=fillColor] (154.86,411.21) circle (  1.16);

\path[draw=drawColor,line width= 0.4pt,line join=round,line cap=round,fill=fillColor] (155.22,411.19) circle (  1.16);

\path[draw=drawColor,line width= 0.4pt,line join=round,line cap=round,fill=fillColor] (155.56,411.07) circle (  1.16);

\path[draw=drawColor,line width= 0.4pt,line join=round,line cap=round,fill=fillColor] (155.91,410.99) circle (  1.16);

\path[draw=drawColor,line width= 0.4pt,line join=round,line cap=round,fill=fillColor] (156.25,410.95) circle (  1.16);

\path[draw=drawColor,line width= 0.4pt,line join=round,line cap=round,fill=fillColor] (156.59,410.92) circle (  1.16);

\path[draw=drawColor,line width= 0.4pt,line join=round,line cap=round,fill=fillColor] (156.93,410.40) circle (  1.16);

\path[draw=drawColor,line width= 0.4pt,line join=round,line cap=round,fill=fillColor] (157.27,410.39) circle (  1.16);

\path[draw=drawColor,line width= 0.4pt,line join=round,line cap=round,fill=fillColor] (157.60,410.21) circle (  1.16);

\path[draw=drawColor,line width= 0.4pt,line join=round,line cap=round,fill=fillColor] (157.93,410.20) circle (  1.16);

\path[draw=drawColor,line width= 0.4pt,line join=round,line cap=round,fill=fillColor] (158.26,409.85) circle (  1.16);

\path[draw=drawColor,line width= 0.4pt,line join=round,line cap=round,fill=fillColor] (158.59,409.68) circle (  1.16);

\path[draw=drawColor,line width= 0.4pt,line join=round,line cap=round,fill=fillColor] (158.92,409.65) circle (  1.16);

\path[draw=drawColor,line width= 0.4pt,line join=round,line cap=round,fill=fillColor] (159.24,409.64) circle (  1.16);

\path[draw=drawColor,line width= 0.4pt,line join=round,line cap=round,fill=fillColor] (159.56,409.24) circle (  1.16);

\path[draw=drawColor,line width= 0.4pt,line join=round,line cap=round,fill=fillColor] (159.88,408.90) circle (  1.16);

\path[draw=drawColor,line width= 0.4pt,line join=round,line cap=round,fill=fillColor] (160.20,408.71) circle (  1.16);

\path[draw=drawColor,line width= 0.4pt,line join=round,line cap=round,fill=fillColor] (160.52,408.67) circle (  1.16);

\path[draw=drawColor,line width= 0.4pt,line join=round,line cap=round,fill=fillColor] (160.83,408.19) circle (  1.16);

\path[draw=drawColor,line width= 0.4pt,line join=round,line cap=round,fill=fillColor] (161.15,408.04) circle (  1.16);

\path[draw=drawColor,line width= 0.4pt,line join=round,line cap=round,fill=fillColor] (161.46,407.72) circle (  1.16);

\path[draw=drawColor,line width= 0.4pt,line join=round,line cap=round,fill=fillColor] (161.77,407.60) circle (  1.16);

\path[draw=drawColor,line width= 0.4pt,line join=round,line cap=round,fill=fillColor] (162.07,407.60) circle (  1.16);

\path[draw=drawColor,line width= 0.4pt,line join=round,line cap=round,fill=fillColor] (162.38,407.40) circle (  1.16);

\path[draw=drawColor,line width= 0.4pt,line join=round,line cap=round,fill=fillColor] (162.68,407.40) circle (  1.16);

\path[draw=drawColor,line width= 0.4pt,line join=round,line cap=round,fill=fillColor] (162.99,407.40) circle (  1.16);

\path[draw=drawColor,line width= 0.4pt,line join=round,line cap=round,fill=fillColor] (163.29,407.31) circle (  1.16);

\path[draw=drawColor,line width= 0.4pt,line join=round,line cap=round,fill=fillColor] (163.59,407.23) circle (  1.16);

\path[draw=drawColor,line width= 0.4pt,line join=round,line cap=round,fill=fillColor] (163.88,407.20) circle (  1.16);

\path[draw=drawColor,line width= 0.4pt,line join=round,line cap=round,fill=fillColor] (164.18,407.20) circle (  1.16);

\path[draw=drawColor,line width= 0.4pt,line join=round,line cap=round,fill=fillColor] (164.47,407.14) circle (  1.16);

\path[draw=drawColor,line width= 0.4pt,line join=round,line cap=round,fill=fillColor] (164.77,406.86) circle (  1.16);

\path[draw=drawColor,line width= 0.4pt,line join=round,line cap=round,fill=fillColor] (165.06,406.80) circle (  1.16);

\path[draw=drawColor,line width= 0.4pt,line join=round,line cap=round,fill=fillColor] (165.35,406.70) circle (  1.16);

\path[draw=drawColor,line width= 0.4pt,line join=round,line cap=round,fill=fillColor] (165.64,406.65) circle (  1.16);

\path[draw=drawColor,line width= 0.4pt,line join=round,line cap=round,fill=fillColor] (165.92,406.24) circle (  1.16);

\path[draw=drawColor,line width= 0.4pt,line join=round,line cap=round,fill=fillColor] (166.21,406.12) circle (  1.16);

\path[draw=drawColor,line width= 0.4pt,line join=round,line cap=round,fill=fillColor] (166.49,405.97) circle (  1.16);

\path[draw=drawColor,line width= 0.4pt,line join=round,line cap=round,fill=fillColor] (166.77,405.94) circle (  1.16);

\path[draw=drawColor,line width= 0.4pt,line join=round,line cap=round,fill=fillColor] (167.06,405.90) circle (  1.16);

\path[draw=drawColor,line width= 0.4pt,line join=round,line cap=round,fill=fillColor] (167.34,405.79) circle (  1.16);

\path[draw=drawColor,line width= 0.4pt,line join=round,line cap=round,fill=fillColor] (167.61,405.73) circle (  1.16);

\path[draw=drawColor,line width= 0.4pt,line join=round,line cap=round,fill=fillColor] (167.89,405.48) circle (  1.16);

\path[draw=drawColor,line width= 0.4pt,line join=round,line cap=round,fill=fillColor] (168.17,405.45) circle (  1.16);

\path[draw=drawColor,line width= 0.4pt,line join=round,line cap=round,fill=fillColor] (168.44,405.44) circle (  1.16);

\path[draw=drawColor,line width= 0.4pt,line join=round,line cap=round,fill=fillColor] (168.71,405.33) circle (  1.16);

\path[draw=drawColor,line width= 0.4pt,line join=round,line cap=round,fill=fillColor] (168.99,405.30) circle (  1.16);

\path[draw=drawColor,line width= 0.4pt,line join=round,line cap=round,fill=fillColor] (169.26,405.27) circle (  1.16);

\path[draw=drawColor,line width= 0.4pt,line join=round,line cap=round,fill=fillColor] (169.53,405.11) circle (  1.16);

\path[draw=drawColor,line width= 0.4pt,line join=round,line cap=round,fill=fillColor] (169.79,405.07) circle (  1.16);

\path[draw=drawColor,line width= 0.4pt,line join=round,line cap=round,fill=fillColor] (170.06,405.01) circle (  1.16);

\path[draw=drawColor,line width= 0.4pt,line join=round,line cap=round,fill=fillColor] (170.33,404.97) circle (  1.16);

\path[draw=drawColor,line width= 0.4pt,line join=round,line cap=round,fill=fillColor] (170.59,404.95) circle (  1.16);

\path[draw=drawColor,line width= 0.4pt,line join=round,line cap=round,fill=fillColor] (170.85,404.81) circle (  1.16);

\path[draw=drawColor,line width= 0.4pt,line join=round,line cap=round,fill=fillColor] (171.12,404.78) circle (  1.16);

\path[draw=drawColor,line width= 0.4pt,line join=round,line cap=round,fill=fillColor] (171.38,404.74) circle (  1.16);

\path[draw=drawColor,line width= 0.4pt,line join=round,line cap=round,fill=fillColor] (171.64,404.60) circle (  1.16);

\path[draw=drawColor,line width= 0.4pt,line join=round,line cap=round,fill=fillColor] (171.90,404.57) circle (  1.16);

\path[draw=drawColor,line width= 0.4pt,line join=round,line cap=round,fill=fillColor] (172.15,404.57) circle (  1.16);

\path[draw=drawColor,line width= 0.4pt,line join=round,line cap=round,fill=fillColor] (172.41,404.47) circle (  1.16);

\path[draw=drawColor,line width= 0.4pt,line join=round,line cap=round,fill=fillColor] (172.67,404.44) circle (  1.16);

\path[draw=drawColor,line width= 0.4pt,line join=round,line cap=round,fill=fillColor] (172.92,404.39) circle (  1.16);

\path[draw=drawColor,line width= 0.4pt,line join=round,line cap=round,fill=fillColor] (173.17,404.38) circle (  1.16);

\path[draw=drawColor,line width= 0.4pt,line join=round,line cap=round,fill=fillColor] (173.43,404.31) circle (  1.16);

\path[draw=drawColor,line width= 0.4pt,line join=round,line cap=round,fill=fillColor] (173.68,404.26) circle (  1.16);

\path[draw=drawColor,line width= 0.4pt,line join=round,line cap=round,fill=fillColor] (173.93,404.02) circle (  1.16);

\path[draw=drawColor,line width= 0.4pt,line join=round,line cap=round,fill=fillColor] (174.18,404.01) circle (  1.16);

\path[draw=drawColor,line width= 0.4pt,line join=round,line cap=round,fill=fillColor] (174.42,404.00) circle (  1.16);

\path[draw=drawColor,line width= 0.4pt,line join=round,line cap=round,fill=fillColor] (174.67,403.94) circle (  1.16);

\path[draw=drawColor,line width= 0.4pt,line join=round,line cap=round,fill=fillColor] (174.92,403.86) circle (  1.16);

\path[draw=drawColor,line width= 0.4pt,line join=round,line cap=round,fill=fillColor] (175.16,403.79) circle (  1.16);

\path[draw=drawColor,line width= 0.4pt,line join=round,line cap=round,fill=fillColor] (175.41,403.64) circle (  1.16);

\path[draw=drawColor,line width= 0.4pt,line join=round,line cap=round,fill=fillColor] (175.65,403.55) circle (  1.16);

\path[draw=drawColor,line width= 0.4pt,line join=round,line cap=round,fill=fillColor] (175.89,403.50) circle (  1.16);

\path[draw=drawColor,line width= 0.4pt,line join=round,line cap=round,fill=fillColor] (176.13,403.43) circle (  1.16);

\path[draw=drawColor,line width= 0.4pt,line join=round,line cap=round,fill=fillColor] (176.37,403.30) circle (  1.16);

\path[draw=drawColor,line width= 0.4pt,line join=round,line cap=round,fill=fillColor] (176.61,403.28) circle (  1.16);

\path[draw=drawColor,line width= 0.4pt,line join=round,line cap=round,fill=fillColor] (176.85,403.28) circle (  1.16);

\path[draw=drawColor,line width= 0.4pt,line join=round,line cap=round,fill=fillColor] (177.09,403.27) circle (  1.16);

\path[draw=drawColor,line width= 0.4pt,line join=round,line cap=round,fill=fillColor] (177.32,403.25) circle (  1.16);

\path[draw=drawColor,line width= 0.4pt,line join=round,line cap=round,fill=fillColor] (177.56,403.11) circle (  1.16);

\path[draw=drawColor,line width= 0.4pt,line join=round,line cap=round,fill=fillColor] (177.80,403.00) circle (  1.16);

\path[draw=drawColor,line width= 0.4pt,line join=round,line cap=round,fill=fillColor] (178.03,402.98) circle (  1.16);

\path[draw=drawColor,line width= 0.4pt,line join=round,line cap=round,fill=fillColor] (178.26,402.95) circle (  1.16);

\path[draw=drawColor,line width= 0.4pt,line join=round,line cap=round,fill=fillColor] (178.49,402.93) circle (  1.16);

\path[draw=drawColor,line width= 0.4pt,line join=round,line cap=round,fill=fillColor] (178.73,402.91) circle (  1.16);

\path[draw=drawColor,line width= 0.4pt,line join=round,line cap=round,fill=fillColor] (178.96,402.74) circle (  1.16);

\path[draw=drawColor,line width= 0.4pt,line join=round,line cap=round,fill=fillColor] (179.19,402.62) circle (  1.16);

\path[draw=drawColor,line width= 0.4pt,line join=round,line cap=round,fill=fillColor] (179.41,402.56) circle (  1.16);

\path[draw=drawColor,line width= 0.4pt,line join=round,line cap=round,fill=fillColor] (179.64,402.52) circle (  1.16);

\path[draw=drawColor,line width= 0.4pt,line join=round,line cap=round,fill=fillColor] (179.87,402.52) circle (  1.16);

\path[draw=drawColor,line width= 0.4pt,line join=round,line cap=round,fill=fillColor] (180.10,402.44) circle (  1.16);

\path[draw=drawColor,line width= 0.4pt,line join=round,line cap=round,fill=fillColor] (180.32,402.37) circle (  1.16);

\path[draw=drawColor,line width= 0.4pt,line join=round,line cap=round,fill=fillColor] (180.55,402.33) circle (  1.16);

\path[draw=drawColor,line width= 0.4pt,line join=round,line cap=round,fill=fillColor] (180.77,402.32) circle (  1.16);

\path[draw=drawColor,line width= 0.4pt,line join=round,line cap=round,fill=fillColor] (180.99,402.12) circle (  1.16);

\path[draw=drawColor,line width= 0.4pt,line join=round,line cap=round,fill=fillColor] (181.22,402.11) circle (  1.16);

\path[draw=drawColor,line width= 0.4pt,line join=round,line cap=round,fill=fillColor] (181.44,402.08) circle (  1.16);

\path[draw=drawColor,line width= 0.4pt,line join=round,line cap=round,fill=fillColor] (181.66,402.06) circle (  1.16);

\path[draw=drawColor,line width= 0.4pt,line join=round,line cap=round,fill=fillColor] (181.88,402.05) circle (  1.16);

\path[draw=drawColor,line width= 0.4pt,line join=round,line cap=round,fill=fillColor] (182.10,401.95) circle (  1.16);

\path[draw=drawColor,line width= 0.4pt,line join=round,line cap=round,fill=fillColor] (182.32,401.91) circle (  1.16);

\path[draw=drawColor,line width= 0.4pt,line join=round,line cap=round,fill=fillColor] (182.54,401.80) circle (  1.16);

\path[draw=drawColor,line width= 0.4pt,line join=round,line cap=round,fill=fillColor] (182.75,401.76) circle (  1.16);

\path[draw=drawColor,line width= 0.4pt,line join=round,line cap=round,fill=fillColor] (182.97,401.58) circle (  1.16);

\path[draw=drawColor,line width= 0.4pt,line join=round,line cap=round,fill=fillColor] (183.19,401.56) circle (  1.16);

\path[draw=drawColor,line width= 0.4pt,line join=round,line cap=round,fill=fillColor] (183.40,401.54) circle (  1.16);

\path[draw=drawColor,line width= 0.4pt,line join=round,line cap=round,fill=fillColor] (183.61,401.41) circle (  1.16);

\path[draw=drawColor,line width= 0.4pt,line join=round,line cap=round,fill=fillColor] (183.83,401.37) circle (  1.16);

\path[draw=drawColor,line width= 0.4pt,line join=round,line cap=round,fill=fillColor] (184.04,401.26) circle (  1.16);

\path[draw=drawColor,line width= 0.4pt,line join=round,line cap=round,fill=fillColor] (184.25,401.25) circle (  1.16);

\path[draw=drawColor,line width= 0.4pt,line join=round,line cap=round,fill=fillColor] (184.47,401.23) circle (  1.16);

\path[draw=drawColor,line width= 0.4pt,line join=round,line cap=round,fill=fillColor] (184.68,401.18) circle (  1.16);

\path[draw=drawColor,line width= 0.4pt,line join=round,line cap=round,fill=fillColor] (184.89,401.15) circle (  1.16);

\path[draw=drawColor,line width= 0.4pt,line join=round,line cap=round,fill=fillColor] (185.10,401.13) circle (  1.16);

\path[draw=drawColor,line width= 0.4pt,line join=round,line cap=round,fill=fillColor] (185.31,401.11) circle (  1.16);

\path[draw=drawColor,line width= 0.4pt,line join=round,line cap=round,fill=fillColor] (185.51,401.10) circle (  1.16);

\path[draw=drawColor,line width= 0.4pt,line join=round,line cap=round,fill=fillColor] (185.72,401.09) circle (  1.16);

\path[draw=drawColor,line width= 0.4pt,line join=round,line cap=round,fill=fillColor] (185.93,401.06) circle (  1.16);

\path[draw=drawColor,line width= 0.4pt,line join=round,line cap=round,fill=fillColor] (186.13,400.98) circle (  1.16);

\path[draw=drawColor,line width= 0.4pt,line join=round,line cap=round,fill=fillColor] (186.34,400.82) circle (  1.16);

\path[draw=drawColor,line width= 0.4pt,line join=round,line cap=round,fill=fillColor] (186.55,400.70) circle (  1.16);

\path[draw=drawColor,line width= 0.4pt,line join=round,line cap=round,fill=fillColor] (186.75,400.70) circle (  1.16);

\path[draw=drawColor,line width= 0.4pt,line join=round,line cap=round,fill=fillColor] (186.95,400.69) circle (  1.16);

\path[draw=drawColor,line width= 0.4pt,line join=round,line cap=round,fill=fillColor] (187.16,400.68) circle (  1.16);

\path[draw=drawColor,line width= 0.4pt,line join=round,line cap=round,fill=fillColor] (187.36,400.58) circle (  1.16);

\path[draw=drawColor,line width= 0.4pt,line join=round,line cap=round,fill=fillColor] (187.56,400.54) circle (  1.16);

\path[draw=drawColor,line width= 0.4pt,line join=round,line cap=round,fill=fillColor] (187.76,400.52) circle (  1.16);

\path[draw=drawColor,line width= 0.4pt,line join=round,line cap=round,fill=fillColor] (187.96,400.48) circle (  1.16);

\path[draw=drawColor,line width= 0.4pt,line join=round,line cap=round,fill=fillColor] (188.16,400.46) circle (  1.16);

\path[draw=drawColor,line width= 0.4pt,line join=round,line cap=round,fill=fillColor] (188.36,400.44) circle (  1.16);

\path[draw=drawColor,line width= 0.4pt,line join=round,line cap=round,fill=fillColor] (188.56,400.41) circle (  1.16);

\path[draw=drawColor,line width= 0.4pt,line join=round,line cap=round,fill=fillColor] (188.76,400.39) circle (  1.16);

\path[draw=drawColor,line width= 0.4pt,line join=round,line cap=round,fill=fillColor] (188.96,400.33) circle (  1.16);

\path[draw=drawColor,line width= 0.4pt,line join=round,line cap=round,fill=fillColor] (189.16,400.30) circle (  1.16);

\path[draw=drawColor,line width= 0.4pt,line join=round,line cap=round,fill=fillColor] (189.35,399.90) circle (  1.16);

\path[draw=drawColor,line width= 0.4pt,line join=round,line cap=round,fill=fillColor] (189.55,399.90) circle (  1.16);

\path[draw=drawColor,line width= 0.4pt,line join=round,line cap=round,fill=fillColor] (189.74,399.78) circle (  1.16);

\path[draw=drawColor,line width= 0.4pt,line join=round,line cap=round,fill=fillColor] (189.94,399.74) circle (  1.16);

\path[draw=drawColor,line width= 0.4pt,line join=round,line cap=round,fill=fillColor] (190.13,399.65) circle (  1.16);

\path[draw=drawColor,line width= 0.4pt,line join=round,line cap=round,fill=fillColor] (190.33,399.65) circle (  1.16);

\path[draw=drawColor,line width= 0.4pt,line join=round,line cap=round,fill=fillColor] (190.52,399.65) circle (  1.16);

\path[draw=drawColor,line width= 0.4pt,line join=round,line cap=round,fill=fillColor] (190.71,399.64) circle (  1.16);

\path[draw=drawColor,line width= 0.4pt,line join=round,line cap=round,fill=fillColor] (190.91,399.62) circle (  1.16);

\path[draw=drawColor,line width= 0.4pt,line join=round,line cap=round,fill=fillColor] (191.10,399.54) circle (  1.16);

\path[draw=drawColor,line width= 0.4pt,line join=round,line cap=round,fill=fillColor] (191.29,399.51) circle (  1.16);

\path[draw=drawColor,line width= 0.4pt,line join=round,line cap=round,fill=fillColor] (191.48,399.49) circle (  1.16);

\path[draw=drawColor,line width= 0.4pt,line join=round,line cap=round,fill=fillColor] (191.67,399.42) circle (  1.16);

\path[draw=drawColor,line width= 0.4pt,line join=round,line cap=round,fill=fillColor] (191.86,399.40) circle (  1.16);

\path[draw=drawColor,line width= 0.4pt,line join=round,line cap=round,fill=fillColor] (192.05,399.39) circle (  1.16);

\path[draw=drawColor,line width= 0.4pt,line join=round,line cap=round,fill=fillColor] (192.24,399.34) circle (  1.16);

\path[draw=drawColor,line width= 0.4pt,line join=round,line cap=round,fill=fillColor] (192.43,399.31) circle (  1.16);

\path[draw=drawColor,line width= 0.4pt,line join=round,line cap=round,fill=fillColor] (192.61,399.27) circle (  1.16);

\path[draw=drawColor,line width= 0.4pt,line join=round,line cap=round,fill=fillColor] (192.80,399.18) circle (  1.16);

\path[draw=drawColor,line width= 0.4pt,line join=round,line cap=round,fill=fillColor] (192.99,399.16) circle (  1.16);

\path[draw=drawColor,line width= 0.4pt,line join=round,line cap=round,fill=fillColor] (193.17,399.16) circle (  1.16);

\path[draw=drawColor,line width= 0.4pt,line join=round,line cap=round,fill=fillColor] (193.36,399.13) circle (  1.16);

\path[draw=drawColor,line width= 0.4pt,line join=round,line cap=round,fill=fillColor] (193.54,398.88) circle (  1.16);

\path[draw=drawColor,line width= 0.4pt,line join=round,line cap=round,fill=fillColor] (193.73,398.85) circle (  1.16);

\path[draw=drawColor,line width= 0.4pt,line join=round,line cap=round,fill=fillColor] (193.91,398.76) circle (  1.16);

\path[draw=drawColor,line width= 0.4pt,line join=round,line cap=round,fill=fillColor] (194.10,398.73) circle (  1.16);

\path[draw=drawColor,line width= 0.4pt,line join=round,line cap=round,fill=fillColor] (194.28,398.73) circle (  1.16);

\path[draw=drawColor,line width= 0.4pt,line join=round,line cap=round,fill=fillColor] (194.46,398.71) circle (  1.16);

\path[draw=drawColor,line width= 0.4pt,line join=round,line cap=round,fill=fillColor] (194.65,398.66) circle (  1.16);

\path[draw=drawColor,line width= 0.4pt,line join=round,line cap=round,fill=fillColor] (194.83,398.63) circle (  1.16);

\path[draw=drawColor,line width= 0.4pt,line join=round,line cap=round,fill=fillColor] (195.01,398.58) circle (  1.16);

\path[draw=drawColor,line width= 0.4pt,line join=round,line cap=round,fill=fillColor] (195.19,398.53) circle (  1.16);

\path[draw=drawColor,line width= 0.4pt,line join=round,line cap=round,fill=fillColor] (195.37,398.52) circle (  1.16);

\path[draw=drawColor,line width= 0.4pt,line join=round,line cap=round,fill=fillColor] (195.55,398.45) circle (  1.16);

\path[draw=drawColor,line width= 0.4pt,line join=round,line cap=round,fill=fillColor] (195.73,398.42) circle (  1.16);

\path[draw=drawColor,line width= 0.4pt,line join=round,line cap=round,fill=fillColor] (195.91,398.30) circle (  1.16);

\path[draw=drawColor,line width= 0.4pt,line join=round,line cap=round,fill=fillColor] (196.09,398.27) circle (  1.16);

\path[draw=drawColor,line width= 0.4pt,line join=round,line cap=round,fill=fillColor] (196.27,398.25) circle (  1.16);

\path[draw=drawColor,line width= 0.4pt,line join=round,line cap=round,fill=fillColor] (196.44,398.18) circle (  1.16);

\path[draw=drawColor,line width= 0.4pt,line join=round,line cap=round,fill=fillColor] (196.62,398.15) circle (  1.16);

\path[draw=drawColor,line width= 0.4pt,line join=round,line cap=round,fill=fillColor] (196.80,398.04) circle (  1.16);

\path[draw=drawColor,line width= 0.4pt,line join=round,line cap=round,fill=fillColor] (196.97,398.04) circle (  1.16);

\path[draw=drawColor,line width= 0.4pt,line join=round,line cap=round,fill=fillColor] (197.15,397.99) circle (  1.16);

\path[draw=drawColor,line width= 0.4pt,line join=round,line cap=round,fill=fillColor] (197.33,397.88) circle (  1.16);

\path[draw=drawColor,line width= 0.4pt,line join=round,line cap=round,fill=fillColor] (197.50,397.85) circle (  1.16);

\path[draw=drawColor,line width= 0.4pt,line join=round,line cap=round,fill=fillColor] (197.68,397.77) circle (  1.16);

\path[draw=drawColor,line width= 0.4pt,line join=round,line cap=round,fill=fillColor] (197.85,397.73) circle (  1.16);

\path[draw=drawColor,line width= 0.4pt,line join=round,line cap=round,fill=fillColor] (198.02,397.47) circle (  1.16);

\path[draw=drawColor,line width= 0.4pt,line join=round,line cap=round,fill=fillColor] (198.20,397.45) circle (  1.16);

\path[draw=drawColor,line width= 0.4pt,line join=round,line cap=round,fill=fillColor] (198.37,397.45) circle (  1.16);

\path[draw=drawColor,line width= 0.4pt,line join=round,line cap=round,fill=fillColor] (198.54,397.44) circle (  1.16);

\path[draw=drawColor,line width= 0.4pt,line join=round,line cap=round,fill=fillColor] (198.72,397.43) circle (  1.16);

\path[draw=drawColor,line width= 0.4pt,line join=round,line cap=round,fill=fillColor] (198.89,397.42) circle (  1.16);

\path[draw=drawColor,line width= 0.4pt,line join=round,line cap=round,fill=fillColor] (199.06,397.32) circle (  1.16);

\path[draw=drawColor,line width= 0.4pt,line join=round,line cap=round,fill=fillColor] (199.23,397.23) circle (  1.16);

\path[draw=drawColor,line width= 0.4pt,line join=round,line cap=round,fill=fillColor] (199.40,397.15) circle (  1.16);

\path[draw=drawColor,line width= 0.4pt,line join=round,line cap=round,fill=fillColor] (199.57,397.06) circle (  1.16);

\path[draw=drawColor,line width= 0.4pt,line join=round,line cap=round,fill=fillColor] (199.74,397.04) circle (  1.16);

\path[draw=drawColor,line width= 0.4pt,line join=round,line cap=round,fill=fillColor] (199.91,396.98) circle (  1.16);

\path[draw=drawColor,line width= 0.4pt,line join=round,line cap=round,fill=fillColor] (200.08,396.97) circle (  1.16);

\path[draw=drawColor,line width= 0.4pt,line join=round,line cap=round,fill=fillColor] (200.25,396.97) circle (  1.16);

\path[draw=drawColor,line width= 0.4pt,line join=round,line cap=round,fill=fillColor] (200.42,396.86) circle (  1.16);

\path[draw=drawColor,line width= 0.4pt,line join=round,line cap=round,fill=fillColor] (200.58,396.83) circle (  1.16);

\path[draw=drawColor,line width= 0.4pt,line join=round,line cap=round,fill=fillColor] (200.75,396.80) circle (  1.16);

\path[draw=drawColor,line width= 0.4pt,line join=round,line cap=round,fill=fillColor] (200.92,396.75) circle (  1.16);

\path[draw=drawColor,line width= 0.4pt,line join=round,line cap=round,fill=fillColor] (201.09,396.68) circle (  1.16);

\path[draw=drawColor,line width= 0.4pt,line join=round,line cap=round,fill=fillColor] (201.25,396.57) circle (  1.16);

\path[draw=drawColor,line width= 0.4pt,line join=round,line cap=round,fill=fillColor] (201.42,396.35) circle (  1.16);

\path[draw=drawColor,line width= 0.4pt,line join=round,line cap=round,fill=fillColor] (201.58,396.29) circle (  1.16);

\path[draw=drawColor,line width= 0.4pt,line join=round,line cap=round,fill=fillColor] (201.75,396.17) circle (  1.16);

\path[draw=drawColor,line width= 0.4pt,line join=round,line cap=round,fill=fillColor] (201.91,396.15) circle (  1.16);

\path[draw=drawColor,line width= 0.4pt,line join=round,line cap=round,fill=fillColor] (202.08,396.07) circle (  1.16);

\path[draw=drawColor,line width= 0.4pt,line join=round,line cap=round,fill=fillColor] (202.24,395.87) circle (  1.16);

\path[draw=drawColor,line width= 0.4pt,line join=round,line cap=round,fill=fillColor] (202.41,395.68) circle (  1.16);

\path[draw=drawColor,line width= 0.4pt,line join=round,line cap=round,fill=fillColor] (202.57,395.58) circle (  1.16);

\path[draw=drawColor,line width= 0.4pt,line join=round,line cap=round,fill=fillColor] (202.73,395.49) circle (  1.16);

\path[draw=drawColor,line width= 0.4pt,line join=round,line cap=round,fill=fillColor] (202.90,395.48) circle (  1.16);

\path[draw=drawColor,line width= 0.4pt,line join=round,line cap=round,fill=fillColor] (203.06,395.43) circle (  1.16);

\path[draw=drawColor,line width= 0.4pt,line join=round,line cap=round,fill=fillColor] (203.22,395.31) circle (  1.16);

\path[draw=drawColor,line width= 0.4pt,line join=round,line cap=round,fill=fillColor] (203.38,395.26) circle (  1.16);

\path[draw=drawColor,line width= 0.4pt,line join=round,line cap=round,fill=fillColor] (203.54,395.23) circle (  1.16);

\path[draw=drawColor,line width= 0.4pt,line join=round,line cap=round,fill=fillColor] (203.71,395.15) circle (  1.16);

\path[draw=drawColor,line width= 0.4pt,line join=round,line cap=round,fill=fillColor] (203.87,395.09) circle (  1.16);

\path[draw=drawColor,line width= 0.4pt,line join=round,line cap=round,fill=fillColor] (204.03,395.04) circle (  1.16);

\path[draw=drawColor,line width= 0.4pt,line join=round,line cap=round,fill=fillColor] (204.19,395.04) circle (  1.16);

\path[draw=drawColor,line width= 0.4pt,line join=round,line cap=round,fill=fillColor] (204.35,394.68) circle (  1.16);

\path[draw=drawColor,line width= 0.4pt,line join=round,line cap=round,fill=fillColor] (204.51,394.60) circle (  1.16);

\path[draw=drawColor,line width= 0.4pt,line join=round,line cap=round,fill=fillColor] (204.66,394.59) circle (  1.16);

\path[draw=drawColor,line width= 0.4pt,line join=round,line cap=round,fill=fillColor] (204.82,394.56) circle (  1.16);

\path[draw=drawColor,line width= 0.4pt,line join=round,line cap=round,fill=fillColor] (204.98,394.54) circle (  1.16);

\path[draw=drawColor,line width= 0.4pt,line join=round,line cap=round,fill=fillColor] (205.14,394.47) circle (  1.16);

\path[draw=drawColor,line width= 0.4pt,line join=round,line cap=round,fill=fillColor] (205.30,394.46) circle (  1.16);

\path[draw=drawColor,line width= 0.4pt,line join=round,line cap=round,fill=fillColor] (205.45,394.43) circle (  1.16);

\path[draw=drawColor,line width= 0.4pt,line join=round,line cap=round,fill=fillColor] (205.61,394.25) circle (  1.16);

\path[draw=drawColor,line width= 0.4pt,line join=round,line cap=round,fill=fillColor] (205.77,393.90) circle (  1.16);

\path[draw=drawColor,line width= 0.4pt,line join=round,line cap=round,fill=fillColor] (205.92,393.85) circle (  1.16);

\path[draw=drawColor,line width= 0.4pt,line join=round,line cap=round,fill=fillColor] (206.08,393.85) circle (  1.16);

\path[draw=drawColor,line width= 0.4pt,line join=round,line cap=round,fill=fillColor] (206.24,393.81) circle (  1.16);

\path[draw=drawColor,line width= 0.4pt,line join=round,line cap=round,fill=fillColor] (206.39,393.75) circle (  1.16);

\path[draw=drawColor,line width= 0.4pt,line join=round,line cap=round,fill=fillColor] (206.55,393.64) circle (  1.16);

\path[draw=drawColor,line width= 0.4pt,line join=round,line cap=round,fill=fillColor] (206.70,393.61) circle (  1.16);

\path[draw=drawColor,line width= 0.4pt,line join=round,line cap=round,fill=fillColor] (206.86,393.60) circle (  1.16);

\path[draw=drawColor,line width= 0.4pt,line join=round,line cap=round,fill=fillColor] (207.01,393.56) circle (  1.16);

\path[draw=drawColor,line width= 0.4pt,line join=round,line cap=round,fill=fillColor] (207.16,393.37) circle (  1.16);

\path[draw=drawColor,line width= 0.4pt,line join=round,line cap=round,fill=fillColor] (207.32,393.27) circle (  1.16);

\path[draw=drawColor,line width= 0.4pt,line join=round,line cap=round,fill=fillColor] (207.47,392.98) circle (  1.16);

\path[draw=drawColor,line width= 0.4pt,line join=round,line cap=round,fill=fillColor] (207.62,392.92) circle (  1.16);

\path[draw=drawColor,line width= 0.4pt,line join=round,line cap=round,fill=fillColor] (207.78,392.86) circle (  1.16);

\path[draw=drawColor,line width= 0.4pt,line join=round,line cap=round,fill=fillColor] (207.93,392.74) circle (  1.16);

\path[draw=drawColor,line width= 0.4pt,line join=round,line cap=round,fill=fillColor] (208.08,392.17) circle (  1.16);

\path[draw=drawColor,line width= 0.4pt,line join=round,line cap=round,fill=fillColor] (208.23,392.09) circle (  1.16);

\path[draw=drawColor,line width= 0.4pt,line join=round,line cap=round,fill=fillColor] (208.39,391.73) circle (  1.16);

\path[draw=drawColor,line width= 0.4pt,line join=round,line cap=round,fill=fillColor] (208.54,391.62) circle (  1.16);

\path[draw=drawColor,line width= 0.4pt,line join=round,line cap=round,fill=fillColor] (208.69,391.11) circle (  1.16);

\path[draw=drawColor,line width= 0.4pt,line join=round,line cap=round,fill=fillColor] (208.84,390.98) circle (  1.16);

\path[draw=drawColor,line width= 0.4pt,line join=round,line cap=round,fill=fillColor] (208.99,390.83) circle (  1.16);

\path[draw=drawColor,line width= 0.4pt,line join=round,line cap=round,fill=fillColor] (209.14,390.61) circle (  1.16);

\path[draw=drawColor,line width= 0.4pt,line join=round,line cap=round,fill=fillColor] (209.29,390.58) circle (  1.16);

\path[draw=drawColor,line width= 0.4pt,line join=round,line cap=round,fill=fillColor] (209.44,390.48) circle (  1.16);

\path[draw=drawColor,line width= 0.4pt,line join=round,line cap=round,fill=fillColor] (209.59,390.43) circle (  1.16);

\path[draw=drawColor,line width= 0.4pt,line join=round,line cap=round,fill=fillColor] (209.74,390.32) circle (  1.16);

\path[draw=drawColor,line width= 0.4pt,line join=round,line cap=round,fill=fillColor] (209.88,389.96) circle (  1.16);

\path[draw=drawColor,line width= 0.4pt,line join=round,line cap=round,fill=fillColor] (210.03,389.89) circle (  1.16);

\path[draw=drawColor,line width= 0.4pt,line join=round,line cap=round,fill=fillColor] (210.18,389.59) circle (  1.16);

\path[draw=drawColor,line width= 0.4pt,line join=round,line cap=round,fill=fillColor] (210.33,389.44) circle (  1.16);

\path[draw=drawColor,line width= 0.4pt,line join=round,line cap=round,fill=fillColor] (210.48,389.34) circle (  1.16);

\path[draw=drawColor,line width= 0.4pt,line join=round,line cap=round,fill=fillColor] (210.62,389.19) circle (  1.16);

\path[draw=drawColor,line width= 0.4pt,line join=round,line cap=round,fill=fillColor] (210.77,388.99) circle (  1.16);

\path[draw=drawColor,line width= 0.4pt,line join=round,line cap=round,fill=fillColor] (210.92,388.56) circle (  1.16);

\path[draw=drawColor,line width= 0.4pt,line join=round,line cap=round,fill=fillColor] (211.06,388.37) circle (  1.16);

\path[draw=drawColor,line width= 0.4pt,line join=round,line cap=round,fill=fillColor] (211.21,388.37) circle (  1.16);

\path[draw=drawColor,line width= 0.4pt,line join=round,line cap=round,fill=fillColor] (211.36,388.30) circle (  1.16);

\path[draw=drawColor,line width= 0.4pt,line join=round,line cap=round,fill=fillColor] (211.50,388.25) circle (  1.16);

\path[draw=drawColor,line width= 0.4pt,line join=round,line cap=round,fill=fillColor] (211.65,387.92) circle (  1.16);

\path[draw=drawColor,line width= 0.4pt,line join=round,line cap=round,fill=fillColor] (211.79,387.69) circle (  1.16);

\path[draw=drawColor,line width= 0.4pt,line join=round,line cap=round,fill=fillColor] (211.94,387.45) circle (  1.16);

\path[draw=drawColor,line width= 0.4pt,line join=round,line cap=round,fill=fillColor] (212.08,387.45) circle (  1.16);

\path[draw=drawColor,line width= 0.4pt,line join=round,line cap=round,fill=fillColor] (212.23,387.27) circle (  1.16);

\path[draw=drawColor,line width= 0.4pt,line join=round,line cap=round,fill=fillColor] (212.37,387.09) circle (  1.16);

\path[draw=drawColor,line width= 0.4pt,line join=round,line cap=round,fill=fillColor] (212.51,387.01) circle (  1.16);

\path[draw=drawColor,line width= 0.4pt,line join=round,line cap=round,fill=fillColor] (212.66,386.79) circle (  1.16);

\path[draw=drawColor,line width= 0.4pt,line join=round,line cap=round,fill=fillColor] (212.80,386.67) circle (  1.16);

\path[draw=drawColor,line width= 0.4pt,line join=round,line cap=round,fill=fillColor] (212.94,386.65) circle (  1.16);

\path[draw=drawColor,line width= 0.4pt,line join=round,line cap=round,fill=fillColor] (213.09,386.61) circle (  1.16);

\path[draw=drawColor,line width= 0.4pt,line join=round,line cap=round,fill=fillColor] (213.23,386.54) circle (  1.16);

\path[draw=drawColor,line width= 0.4pt,line join=round,line cap=round,fill=fillColor] (213.37,386.46) circle (  1.16);

\path[draw=drawColor,line width= 0.4pt,line join=round,line cap=round,fill=fillColor] (213.51,386.46) circle (  1.16);

\path[draw=drawColor,line width= 0.4pt,line join=round,line cap=round,fill=fillColor] (213.65,386.28) circle (  1.16);

\path[draw=drawColor,line width= 0.4pt,line join=round,line cap=round,fill=fillColor] (213.80,385.98) circle (  1.16);

\path[draw=drawColor,line width= 0.4pt,line join=round,line cap=round,fill=fillColor] (213.94,385.53) circle (  1.16);

\path[draw=drawColor,line width= 0.4pt,line join=round,line cap=round,fill=fillColor] (214.08,385.22) circle (  1.16);

\path[draw=drawColor,line width= 0.4pt,line join=round,line cap=round,fill=fillColor] (214.22,385.18) circle (  1.16);

\path[draw=drawColor,line width= 0.4pt,line join=round,line cap=round,fill=fillColor] (214.36,385.03) circle (  1.16);

\path[draw=drawColor,line width= 0.4pt,line join=round,line cap=round,fill=fillColor] (214.50,384.54) circle (  1.16);

\path[draw=drawColor,line width= 0.4pt,line join=round,line cap=round,fill=fillColor] (214.64,384.41) circle (  1.16);

\path[draw=drawColor,line width= 0.4pt,line join=round,line cap=round,fill=fillColor] (214.78,384.37) circle (  1.16);

\path[draw=drawColor,line width= 0.4pt,line join=round,line cap=round,fill=fillColor] (214.92,384.28) circle (  1.16);

\path[draw=drawColor,line width= 0.4pt,line join=round,line cap=round,fill=fillColor] (215.06,383.78) circle (  1.16);

\path[draw=drawColor,line width= 0.4pt,line join=round,line cap=round,fill=fillColor] (215.20,383.38) circle (  1.16);

\path[draw=drawColor,line width= 0.4pt,line join=round,line cap=round,fill=fillColor] (215.34,383.10) circle (  1.16);

\path[draw=drawColor,line width= 0.4pt,line join=round,line cap=round,fill=fillColor] (215.47,382.90) circle (  1.16);

\path[draw=drawColor,line width= 0.4pt,line join=round,line cap=round,fill=fillColor] (215.61,382.76) circle (  1.16);

\path[draw=drawColor,line width= 0.4pt,line join=round,line cap=round,fill=fillColor] (215.75,382.49) circle (  1.16);

\path[draw=drawColor,line width= 0.4pt,line join=round,line cap=round,fill=fillColor] (215.89,381.37) circle (  1.16);

\path[draw=drawColor,line width= 0.4pt,line join=round,line cap=round,fill=fillColor] (216.03,379.22) circle (  1.16);

\path[draw=drawColor,line width= 0.4pt,line join=round,line cap=round,fill=fillColor] (216.16,379.02) circle (  1.16);

\path[draw=drawColor,line width= 0.4pt,line join=round,line cap=round,fill=fillColor] (216.30,372.42) circle (  1.16);

\path[draw=drawColor,line width= 0.4pt,line join=round,line cap=round,fill=fillColor] (216.44,372.42) circle (  1.16);

\path[draw=drawColor,line width= 0.4pt,line join=round,line cap=round,fill=fillColor] (216.58,372.42) circle (  1.16);

\path[draw=drawColor,line width= 0.4pt,line join=round,line cap=round,fill=fillColor] (216.71,372.42) circle (  1.16);

\path[draw=drawColor,line width= 0.4pt,line join=round,line cap=round,fill=fillColor] (216.85,372.42) circle (  1.16);

\path[draw=drawColor,line width= 0.4pt,line join=round,line cap=round,fill=fillColor] (216.98,372.42) circle (  1.16);

\path[draw=drawColor,line width= 0.4pt,line join=round,line cap=round,fill=fillColor] (217.12,372.42) circle (  1.16);

\path[draw=drawColor,line width= 0.4pt,line join=round,line cap=round,fill=fillColor] (217.26,372.42) circle (  1.16);

\path[draw=drawColor,line width= 0.4pt,line join=round,line cap=round,fill=fillColor] (217.39,372.42) circle (  1.16);

\path[draw=drawColor,line width= 0.4pt,line join=round,line cap=round,fill=fillColor] (217.53,372.42) circle (  1.16);

\path[draw=drawColor,line width= 0.4pt,line join=round,line cap=round,fill=fillColor] (217.66,372.42) circle (  1.16);

\path[draw=drawColor,line width= 0.4pt,line join=round,line cap=round,fill=fillColor] (217.80,372.42) circle (  1.16);

\path[draw=drawColor,line width= 0.4pt,line join=round,line cap=round,fill=fillColor] (217.93,372.42) circle (  1.16);

\path[draw=drawColor,line width= 0.4pt,line join=round,line cap=round,fill=fillColor] (218.06,372.42) circle (  1.16);

\path[draw=drawColor,line width= 0.4pt,line join=round,line cap=round,fill=fillColor] (218.20,372.42) circle (  1.16);

\path[draw=drawColor,line width= 0.4pt,line join=round,line cap=round,fill=fillColor] (218.33,372.42) circle (  1.16);

\path[draw=drawColor,line width= 0.4pt,line join=round,line cap=round,fill=fillColor] (218.47,372.42) circle (  1.16);

\path[draw=drawColor,line width= 0.4pt,line join=round,line cap=round,fill=fillColor] (218.60,372.42) circle (  1.16);

\path[draw=drawColor,line width= 0.4pt,line join=round,line cap=round,fill=fillColor] (218.73,372.42) circle (  1.16);

\path[draw=drawColor,line width= 0.4pt,line join=round,line cap=round,fill=fillColor] (218.87,372.42) circle (  1.16);

\path[draw=drawColor,line width= 0.4pt,line join=round,line cap=round,fill=fillColor] (219.00,372.42) circle (  1.16);

\path[draw=drawColor,line width= 0.4pt,line join=round,line cap=round,fill=fillColor] (219.13,372.42) circle (  1.16);

\path[draw=drawColor,line width= 0.4pt,line join=round,line cap=round,fill=fillColor] (219.26,372.42) circle (  1.16);

\path[draw=drawColor,line width= 0.4pt,line join=round,line cap=round,fill=fillColor] (219.40,372.42) circle (  1.16);

\path[draw=drawColor,line width= 0.4pt,line join=round,line cap=round,fill=fillColor] (219.53,372.42) circle (  1.16);

\path[draw=drawColor,line width= 0.4pt,line join=round,line cap=round,fill=fillColor] (219.66,372.42) circle (  1.16);

\path[draw=drawColor,line width= 0.4pt,line join=round,line cap=round,fill=fillColor] (219.79,372.42) circle (  1.16);

\path[draw=drawColor,line width= 0.4pt,line join=round,line cap=round,fill=fillColor] (219.92,372.42) circle (  1.16);

\path[draw=drawColor,line width= 0.4pt,line join=round,line cap=round,fill=fillColor] (220.05,372.42) circle (  1.16);

\path[draw=drawColor,line width= 0.4pt,line join=round,line cap=round,fill=fillColor] (220.18,372.42) circle (  1.16);

\path[draw=drawColor,line width= 0.4pt,line join=round,line cap=round,fill=fillColor] (220.31,372.42) circle (  1.16);

\path[draw=drawColor,line width= 0.4pt,line join=round,line cap=round,fill=fillColor] (220.45,372.42) circle (  1.16);

\path[draw=drawColor,line width= 0.4pt,line join=round,line cap=round,fill=fillColor] (220.58,372.42) circle (  1.16);

\path[draw=drawColor,line width= 0.4pt,line join=round,line cap=round,fill=fillColor] (220.71,372.42) circle (  1.16);

\path[draw=drawColor,line width= 0.4pt,line join=round,line cap=round,fill=fillColor] (220.84,372.42) circle (  1.16);

\path[draw=drawColor,line width= 0.4pt,line join=round,line cap=round,fill=fillColor] (220.97,372.42) circle (  1.16);

\path[draw=drawColor,line width= 0.4pt,line join=round,line cap=round,fill=fillColor] (221.09,372.42) circle (  1.16);

\path[draw=drawColor,line width= 0.4pt,line join=round,line cap=round,fill=fillColor] (221.22,372.42) circle (  1.16);

\path[draw=drawColor,line width= 0.4pt,line join=round,line cap=round,fill=fillColor] (221.35,372.42) circle (  1.16);

\path[draw=drawColor,line width= 0.4pt,line join=round,line cap=round,fill=fillColor] (221.48,372.42) circle (  1.16);

\path[draw=drawColor,line width= 0.4pt,line join=round,line cap=round,fill=fillColor] (221.61,372.42) circle (  1.16);

\path[draw=drawColor,line width= 0.4pt,line join=round,line cap=round,fill=fillColor] (221.74,372.42) circle (  1.16);

\path[draw=drawColor,line width= 0.4pt,line join=round,line cap=round,fill=fillColor] (221.87,372.42) circle (  1.16);

\path[draw=drawColor,line width= 0.4pt,line join=round,line cap=round,fill=fillColor] (222.00,372.42) circle (  1.16);

\path[draw=drawColor,line width= 0.4pt,line join=round,line cap=round,fill=fillColor] (222.12,372.42) circle (  1.16);

\path[draw=drawColor,line width= 0.4pt,line join=round,line cap=round,fill=fillColor] (222.25,372.42) circle (  1.16);

\path[draw=drawColor,line width= 0.4pt,line join=round,line cap=round,fill=fillColor] (222.38,372.42) circle (  1.16);

\path[draw=drawColor,line width= 0.4pt,line join=round,line cap=round,fill=fillColor] (222.51,372.42) circle (  1.16);

\path[draw=drawColor,line width= 0.4pt,line join=round,line cap=round,fill=fillColor] (222.63,372.42) circle (  1.16);

\path[draw=drawColor,line width= 0.4pt,line join=round,line cap=round,fill=fillColor] (222.76,372.42) circle (  1.16);

\path[draw=drawColor,line width= 0.4pt,line join=round,line cap=round,fill=fillColor] (222.89,372.42) circle (  1.16);

\path[draw=drawColor,line width= 0.4pt,line join=round,line cap=round,fill=fillColor] (223.01,372.42) circle (  1.16);

\path[draw=drawColor,line width= 0.4pt,line join=round,line cap=round,fill=fillColor] (223.14,372.42) circle (  1.16);

\path[draw=drawColor,line width= 0.4pt,line join=round,line cap=round,fill=fillColor] (223.27,372.42) circle (  1.16);

\path[draw=drawColor,line width= 0.4pt,line join=round,line cap=round,fill=fillColor] (223.39,372.42) circle (  1.16);

\path[draw=drawColor,line width= 0.4pt,line join=round,line cap=round,fill=fillColor] (223.52,372.42) circle (  1.16);

\path[draw=drawColor,line width= 0.4pt,line join=round,line cap=round,fill=fillColor] (223.64,372.42) circle (  1.16);

\path[draw=drawColor,line width= 0.4pt,line join=round,line cap=round,fill=fillColor] (223.77,372.42) circle (  1.16);

\path[draw=drawColor,line width= 0.4pt,line join=round,line cap=round,fill=fillColor] (223.89,372.42) circle (  1.16);

\path[draw=drawColor,line width= 0.4pt,line join=round,line cap=round,fill=fillColor] (224.02,372.42) circle (  1.16);

\path[draw=drawColor,line width= 0.4pt,line join=round,line cap=round,fill=fillColor] (224.14,372.42) circle (  1.16);

\path[draw=drawColor,line width= 0.4pt,line join=round,line cap=round,fill=fillColor] (224.27,372.42) circle (  1.16);

\path[draw=drawColor,line width= 0.4pt,line join=round,line cap=round,fill=fillColor] (224.39,372.42) circle (  1.16);

\path[draw=drawColor,line width= 0.4pt,line join=round,line cap=round,fill=fillColor] (224.52,372.42) circle (  1.16);

\path[draw=drawColor,line width= 0.4pt,line join=round,line cap=round,fill=fillColor] (224.64,372.42) circle (  1.16);

\path[draw=drawColor,line width= 0.4pt,line join=round,line cap=round,fill=fillColor] (224.77,372.42) circle (  1.16);

\path[draw=drawColor,line width= 0.4pt,line join=round,line cap=round,fill=fillColor] (224.89,372.42) circle (  1.16);

\path[draw=drawColor,line width= 0.4pt,line join=round,line cap=round,fill=fillColor] (225.01,372.42) circle (  1.16);

\path[draw=drawColor,line width= 0.4pt,line join=round,line cap=round,fill=fillColor] (225.14,372.42) circle (  1.16);

\path[draw=drawColor,line width= 0.4pt,line join=round,line cap=round,fill=fillColor] (225.26,372.42) circle (  1.16);
\definecolor[named]{drawColor}{rgb}{0.30,0.69,0.29}
\definecolor[named]{fillColor}{rgb}{0.30,0.69,0.29}

\path[draw=drawColor,line width= 0.4pt,line join=round,line cap=round,fill=fillColor] ( 74.88,458.06) circle (  1.16);

\path[draw=drawColor,line width= 0.4pt,line join=round,line cap=round,fill=fillColor] ( 80.66,458.06) circle (  1.16);

\path[draw=drawColor,line width= 0.4pt,line join=round,line cap=round,fill=fillColor] ( 84.72,458.06) circle (  1.16);

\path[draw=drawColor,line width= 0.4pt,line join=round,line cap=round,fill=fillColor] ( 87.95,458.06) circle (  1.16);

\path[draw=drawColor,line width= 0.4pt,line join=round,line cap=round,fill=fillColor] ( 90.68,458.06) circle (  1.16);

\path[draw=drawColor,line width= 0.4pt,line join=round,line cap=round,fill=fillColor] ( 93.06,458.06) circle (  1.16);

\path[draw=drawColor,line width= 0.4pt,line join=round,line cap=round,fill=fillColor] ( 95.19,458.06) circle (  1.16);

\path[draw=drawColor,line width= 0.4pt,line join=round,line cap=round,fill=fillColor] ( 97.13,458.06) circle (  1.16);

\path[draw=drawColor,line width= 0.4pt,line join=round,line cap=round,fill=fillColor] ( 98.91,458.06) circle (  1.16);

\path[draw=drawColor,line width= 0.4pt,line join=round,line cap=round,fill=fillColor] (100.57,458.06) circle (  1.16);

\path[draw=drawColor,line width= 0.4pt,line join=round,line cap=round,fill=fillColor] (102.11,458.06) circle (  1.16);

\path[draw=drawColor,line width= 0.4pt,line join=round,line cap=round,fill=fillColor] (103.57,458.06) circle (  1.16);

\path[draw=drawColor,line width= 0.4pt,line join=round,line cap=round,fill=fillColor] (104.95,458.06) circle (  1.16);

\path[draw=drawColor,line width= 0.4pt,line join=round,line cap=round,fill=fillColor] (106.26,458.06) circle (  1.16);

\path[draw=drawColor,line width= 0.4pt,line join=round,line cap=round,fill=fillColor] (107.50,458.06) circle (  1.16);

\path[draw=drawColor,line width= 0.4pt,line join=round,line cap=round,fill=fillColor] (108.70,458.06) circle (  1.16);

\path[draw=drawColor,line width= 0.4pt,line join=round,line cap=round,fill=fillColor] (109.84,458.06) circle (  1.16);

\path[draw=drawColor,line width= 0.4pt,line join=round,line cap=round,fill=fillColor] (110.94,458.06) circle (  1.16);

\path[draw=drawColor,line width= 0.4pt,line join=round,line cap=round,fill=fillColor] (112.00,458.06) circle (  1.16);

\path[draw=drawColor,line width= 0.4pt,line join=round,line cap=round,fill=fillColor] (113.03,458.06) circle (  1.16);

\path[draw=drawColor,line width= 0.4pt,line join=round,line cap=round,fill=fillColor] (114.02,458.06) circle (  1.16);

\path[draw=drawColor,line width= 0.4pt,line join=round,line cap=round,fill=fillColor] (114.98,458.06) circle (  1.16);

\path[draw=drawColor,line width= 0.4pt,line join=round,line cap=round,fill=fillColor] (115.91,458.06) circle (  1.16);

\path[draw=drawColor,line width= 0.4pt,line join=round,line cap=round,fill=fillColor] (116.81,458.06) circle (  1.16);

\path[draw=drawColor,line width= 0.4pt,line join=round,line cap=round,fill=fillColor] (117.69,458.06) circle (  1.16);

\path[draw=drawColor,line width= 0.4pt,line join=round,line cap=round,fill=fillColor] (118.55,458.06) circle (  1.16);

\path[draw=drawColor,line width= 0.4pt,line join=round,line cap=round,fill=fillColor] (119.38,458.06) circle (  1.16);

\path[draw=drawColor,line width= 0.4pt,line join=round,line cap=round,fill=fillColor] (120.20,458.06) circle (  1.16);

\path[draw=drawColor,line width= 0.4pt,line join=round,line cap=round,fill=fillColor] (120.99,458.06) circle (  1.16);

\path[draw=drawColor,line width= 0.4pt,line join=round,line cap=round,fill=fillColor] (121.77,458.06) circle (  1.16);

\path[draw=drawColor,line width= 0.4pt,line join=round,line cap=round,fill=fillColor] (122.53,458.06) circle (  1.16);

\path[draw=drawColor,line width= 0.4pt,line join=round,line cap=round,fill=fillColor] (123.27,458.06) circle (  1.16);

\path[draw=drawColor,line width= 0.4pt,line join=round,line cap=round,fill=fillColor] (124.00,458.06) circle (  1.16);

\path[draw=drawColor,line width= 0.4pt,line join=round,line cap=round,fill=fillColor] (124.71,458.06) circle (  1.16);

\path[draw=drawColor,line width= 0.4pt,line join=round,line cap=round,fill=fillColor] (125.41,458.06) circle (  1.16);

\path[draw=drawColor,line width= 0.4pt,line join=round,line cap=round,fill=fillColor] (126.10,458.06) circle (  1.16);

\path[draw=drawColor,line width= 0.4pt,line join=round,line cap=round,fill=fillColor] (126.77,458.06) circle (  1.16);

\path[draw=drawColor,line width= 0.4pt,line join=round,line cap=round,fill=fillColor] (127.44,458.06) circle (  1.16);

\path[draw=drawColor,line width= 0.4pt,line join=round,line cap=round,fill=fillColor] (128.09,458.06) circle (  1.16);

\path[draw=drawColor,line width= 0.4pt,line join=round,line cap=round,fill=fillColor] (128.73,458.06) circle (  1.16);

\path[draw=drawColor,line width= 0.4pt,line join=round,line cap=round,fill=fillColor] (129.35,458.06) circle (  1.16);

\path[draw=drawColor,line width= 0.4pt,line join=round,line cap=round,fill=fillColor] (129.97,458.06) circle (  1.16);

\path[draw=drawColor,line width= 0.4pt,line join=round,line cap=round,fill=fillColor] (130.58,458.06) circle (  1.16);

\path[draw=drawColor,line width= 0.4pt,line join=round,line cap=round,fill=fillColor] (131.18,458.06) circle (  1.16);

\path[draw=drawColor,line width= 0.4pt,line join=round,line cap=round,fill=fillColor] (131.77,458.06) circle (  1.16);

\path[draw=drawColor,line width= 0.4pt,line join=round,line cap=round,fill=fillColor] (132.35,458.06) circle (  1.16);

\path[draw=drawColor,line width= 0.4pt,line join=round,line cap=round,fill=fillColor] (132.93,458.06) circle (  1.16);

\path[draw=drawColor,line width= 0.4pt,line join=round,line cap=round,fill=fillColor] (133.49,458.06) circle (  1.16);

\path[draw=drawColor,line width= 0.4pt,line join=round,line cap=round,fill=fillColor] (134.05,458.06) circle (  1.16);

\path[draw=drawColor,line width= 0.4pt,line join=round,line cap=round,fill=fillColor] (134.60,458.06) circle (  1.16);

\path[draw=drawColor,line width= 0.4pt,line join=round,line cap=round,fill=fillColor] (135.14,458.06) circle (  1.16);

\path[draw=drawColor,line width= 0.4pt,line join=round,line cap=round,fill=fillColor] (135.68,458.06) circle (  1.16);

\path[draw=drawColor,line width= 0.4pt,line join=round,line cap=round,fill=fillColor] (136.21,458.06) circle (  1.16);

\path[draw=drawColor,line width= 0.4pt,line join=round,line cap=round,fill=fillColor] (136.73,458.06) circle (  1.16);

\path[draw=drawColor,line width= 0.4pt,line join=round,line cap=round,fill=fillColor] (137.25,458.06) circle (  1.16);

\path[draw=drawColor,line width= 0.4pt,line join=round,line cap=round,fill=fillColor] (137.76,458.06) circle (  1.16);

\path[draw=drawColor,line width= 0.4pt,line join=round,line cap=round,fill=fillColor] (138.26,458.06) circle (  1.16);

\path[draw=drawColor,line width= 0.4pt,line join=round,line cap=round,fill=fillColor] (138.76,458.06) circle (  1.16);

\path[draw=drawColor,line width= 0.4pt,line join=round,line cap=round,fill=fillColor] (139.25,458.06) circle (  1.16);

\path[draw=drawColor,line width= 0.4pt,line join=round,line cap=round,fill=fillColor] (139.74,458.06) circle (  1.16);

\path[draw=drawColor,line width= 0.4pt,line join=round,line cap=round,fill=fillColor] (140.22,458.06) circle (  1.16);

\path[draw=drawColor,line width= 0.4pt,line join=round,line cap=round,fill=fillColor] (140.70,458.06) circle (  1.16);

\path[draw=drawColor,line width= 0.4pt,line join=round,line cap=round,fill=fillColor] (141.17,454.11) circle (  1.16);

\path[draw=drawColor,line width= 0.4pt,line join=round,line cap=round,fill=fillColor] (141.63,453.24) circle (  1.16);

\path[draw=drawColor,line width= 0.4pt,line join=round,line cap=round,fill=fillColor] (142.09,450.91) circle (  1.16);

\path[draw=drawColor,line width= 0.4pt,line join=round,line cap=round,fill=fillColor] (142.55,450.59) circle (  1.16);

\path[draw=drawColor,line width= 0.4pt,line join=round,line cap=round,fill=fillColor] (143.00,450.17) circle (  1.16);

\path[draw=drawColor,line width= 0.4pt,line join=round,line cap=round,fill=fillColor] (143.45,450.15) circle (  1.16);

\path[draw=drawColor,line width= 0.4pt,line join=round,line cap=round,fill=fillColor] (143.89,449.11) circle (  1.16);

\path[draw=drawColor,line width= 0.4pt,line join=round,line cap=round,fill=fillColor] (144.33,448.82) circle (  1.16);

\path[draw=drawColor,line width= 0.4pt,line join=round,line cap=round,fill=fillColor] (144.77,448.74) circle (  1.16);

\path[draw=drawColor,line width= 0.4pt,line join=round,line cap=round,fill=fillColor] (145.20,447.29) circle (  1.16);

\path[draw=drawColor,line width= 0.4pt,line join=round,line cap=round,fill=fillColor] (145.62,446.63) circle (  1.16);

\path[draw=drawColor,line width= 0.4pt,line join=round,line cap=round,fill=fillColor] (146.05,444.90) circle (  1.16);

\path[draw=drawColor,line width= 0.4pt,line join=round,line cap=round,fill=fillColor] (146.46,442.50) circle (  1.16);

\path[draw=drawColor,line width= 0.4pt,line join=round,line cap=round,fill=fillColor] (146.88,442.43) circle (  1.16);

\path[draw=drawColor,line width= 0.4pt,line join=round,line cap=round,fill=fillColor] (147.29,442.32) circle (  1.16);

\path[draw=drawColor,line width= 0.4pt,line join=round,line cap=round,fill=fillColor] (147.70,442.14) circle (  1.16);

\path[draw=drawColor,line width= 0.4pt,line join=round,line cap=round,fill=fillColor] (148.10,441.55) circle (  1.16);

\path[draw=drawColor,line width= 0.4pt,line join=round,line cap=round,fill=fillColor] (148.50,441.40) circle (  1.16);

\path[draw=drawColor,line width= 0.4pt,line join=round,line cap=round,fill=fillColor] (148.90,441.13) circle (  1.16);

\path[draw=drawColor,line width= 0.4pt,line join=round,line cap=round,fill=fillColor] (149.30,440.95) circle (  1.16);

\path[draw=drawColor,line width= 0.4pt,line join=round,line cap=round,fill=fillColor] (149.69,440.88) circle (  1.16);

\path[draw=drawColor,line width= 0.4pt,line join=round,line cap=round,fill=fillColor] (150.08,440.73) circle (  1.16);

\path[draw=drawColor,line width= 0.4pt,line join=round,line cap=round,fill=fillColor] (150.46,439.83) circle (  1.16);

\path[draw=drawColor,line width= 0.4pt,line join=round,line cap=round,fill=fillColor] (150.84,439.63) circle (  1.16);

\path[draw=drawColor,line width= 0.4pt,line join=round,line cap=round,fill=fillColor] (151.22,439.25) circle (  1.16);

\path[draw=drawColor,line width= 0.4pt,line join=round,line cap=round,fill=fillColor] (151.60,439.02) circle (  1.16);

\path[draw=drawColor,line width= 0.4pt,line join=round,line cap=round,fill=fillColor] (151.97,438.33) circle (  1.16);

\path[draw=drawColor,line width= 0.4pt,line join=round,line cap=round,fill=fillColor] (152.34,438.15) circle (  1.16);

\path[draw=drawColor,line width= 0.4pt,line join=round,line cap=round,fill=fillColor] (152.71,437.45) circle (  1.16);

\path[draw=drawColor,line width= 0.4pt,line join=round,line cap=round,fill=fillColor] (153.08,436.85) circle (  1.16);

\path[draw=drawColor,line width= 0.4pt,line join=round,line cap=round,fill=fillColor] (153.44,436.32) circle (  1.16);

\path[draw=drawColor,line width= 0.4pt,line join=round,line cap=round,fill=fillColor] (153.80,435.73) circle (  1.16);

\path[draw=drawColor,line width= 0.4pt,line join=round,line cap=round,fill=fillColor] (154.16,435.54) circle (  1.16);

\path[draw=drawColor,line width= 0.4pt,line join=round,line cap=round,fill=fillColor] (154.51,435.43) circle (  1.16);

\path[draw=drawColor,line width= 0.4pt,line join=round,line cap=round,fill=fillColor] (154.86,434.29) circle (  1.16);

\path[draw=drawColor,line width= 0.4pt,line join=round,line cap=round,fill=fillColor] (155.22,434.10) circle (  1.16);

\path[draw=drawColor,line width= 0.4pt,line join=round,line cap=round,fill=fillColor] (155.56,433.65) circle (  1.16);

\path[draw=drawColor,line width= 0.4pt,line join=round,line cap=round,fill=fillColor] (155.91,433.02) circle (  1.16);

\path[draw=drawColor,line width= 0.4pt,line join=round,line cap=round,fill=fillColor] (156.25,432.72) circle (  1.16);

\path[draw=drawColor,line width= 0.4pt,line join=round,line cap=round,fill=fillColor] (156.59,432.59) circle (  1.16);

\path[draw=drawColor,line width= 0.4pt,line join=round,line cap=round,fill=fillColor] (156.93,432.36) circle (  1.16);

\path[draw=drawColor,line width= 0.4pt,line join=round,line cap=round,fill=fillColor] (157.27,432.34) circle (  1.16);

\path[draw=drawColor,line width= 0.4pt,line join=round,line cap=round,fill=fillColor] (157.60,431.08) circle (  1.16);

\path[draw=drawColor,line width= 0.4pt,line join=round,line cap=round,fill=fillColor] (157.93,430.46) circle (  1.16);

\path[draw=drawColor,line width= 0.4pt,line join=round,line cap=round,fill=fillColor] (158.26,429.64) circle (  1.16);

\path[draw=drawColor,line width= 0.4pt,line join=round,line cap=round,fill=fillColor] (158.59,427.57) circle (  1.16);

\path[draw=drawColor,line width= 0.4pt,line join=round,line cap=round,fill=fillColor] (158.92,427.45) circle (  1.16);

\path[draw=drawColor,line width= 0.4pt,line join=round,line cap=round,fill=fillColor] (159.24,427.27) circle (  1.16);

\path[draw=drawColor,line width= 0.4pt,line join=round,line cap=round,fill=fillColor] (159.56,426.82) circle (  1.16);

\path[draw=drawColor,line width= 0.4pt,line join=round,line cap=round,fill=fillColor] (159.88,425.54) circle (  1.16);

\path[draw=drawColor,line width= 0.4pt,line join=round,line cap=round,fill=fillColor] (160.20,425.16) circle (  1.16);

\path[draw=drawColor,line width= 0.4pt,line join=round,line cap=round,fill=fillColor] (160.52,424.45) circle (  1.16);

\path[draw=drawColor,line width= 0.4pt,line join=round,line cap=round,fill=fillColor] (160.83,424.42) circle (  1.16);

\path[draw=drawColor,line width= 0.4pt,line join=round,line cap=round,fill=fillColor] (161.15,424.05) circle (  1.16);

\path[draw=drawColor,line width= 0.4pt,line join=round,line cap=round,fill=fillColor] (161.46,422.54) circle (  1.16);

\path[draw=drawColor,line width= 0.4pt,line join=round,line cap=round,fill=fillColor] (161.77,422.49) circle (  1.16);

\path[draw=drawColor,line width= 0.4pt,line join=round,line cap=round,fill=fillColor] (162.07,422.22) circle (  1.16);

\path[draw=drawColor,line width= 0.4pt,line join=round,line cap=round,fill=fillColor] (162.38,421.47) circle (  1.16);

\path[draw=drawColor,line width= 0.4pt,line join=round,line cap=round,fill=fillColor] (162.68,421.45) circle (  1.16);

\path[draw=drawColor,line width= 0.4pt,line join=round,line cap=round,fill=fillColor] (162.99,421.31) circle (  1.16);

\path[draw=drawColor,line width= 0.4pt,line join=round,line cap=round,fill=fillColor] (163.29,421.07) circle (  1.16);

\path[draw=drawColor,line width= 0.4pt,line join=round,line cap=round,fill=fillColor] (163.59,419.37) circle (  1.16);

\path[draw=drawColor,line width= 0.4pt,line join=round,line cap=round,fill=fillColor] (163.88,419.32) circle (  1.16);

\path[draw=drawColor,line width= 0.4pt,line join=round,line cap=round,fill=fillColor] (164.18,419.25) circle (  1.16);

\path[draw=drawColor,line width= 0.4pt,line join=round,line cap=round,fill=fillColor] (164.47,419.02) circle (  1.16);

\path[draw=drawColor,line width= 0.4pt,line join=round,line cap=round,fill=fillColor] (164.77,418.38) circle (  1.16);

\path[draw=drawColor,line width= 0.4pt,line join=round,line cap=round,fill=fillColor] (165.06,417.73) circle (  1.16);

\path[draw=drawColor,line width= 0.4pt,line join=round,line cap=round,fill=fillColor] (165.35,417.70) circle (  1.16);

\path[draw=drawColor,line width= 0.4pt,line join=round,line cap=round,fill=fillColor] (165.64,417.46) circle (  1.16);

\path[draw=drawColor,line width= 0.4pt,line join=round,line cap=round,fill=fillColor] (165.92,417.28) circle (  1.16);

\path[draw=drawColor,line width= 0.4pt,line join=round,line cap=round,fill=fillColor] (166.21,417.23) circle (  1.16);

\path[draw=drawColor,line width= 0.4pt,line join=round,line cap=round,fill=fillColor] (166.49,417.11) circle (  1.16);

\path[draw=drawColor,line width= 0.4pt,line join=round,line cap=round,fill=fillColor] (166.77,415.86) circle (  1.16);

\path[draw=drawColor,line width= 0.4pt,line join=round,line cap=round,fill=fillColor] (167.06,415.83) circle (  1.16);

\path[draw=drawColor,line width= 0.4pt,line join=round,line cap=round,fill=fillColor] (167.34,415.65) circle (  1.16);

\path[draw=drawColor,line width= 0.4pt,line join=round,line cap=round,fill=fillColor] (167.61,415.54) circle (  1.16);

\path[draw=drawColor,line width= 0.4pt,line join=round,line cap=round,fill=fillColor] (167.89,415.06) circle (  1.16);

\path[draw=drawColor,line width= 0.4pt,line join=round,line cap=round,fill=fillColor] (168.17,414.90) circle (  1.16);

\path[draw=drawColor,line width= 0.4pt,line join=round,line cap=round,fill=fillColor] (168.44,414.88) circle (  1.16);

\path[draw=drawColor,line width= 0.4pt,line join=round,line cap=round,fill=fillColor] (168.71,414.79) circle (  1.16);

\path[draw=drawColor,line width= 0.4pt,line join=round,line cap=round,fill=fillColor] (168.99,414.70) circle (  1.16);

\path[draw=drawColor,line width= 0.4pt,line join=round,line cap=round,fill=fillColor] (169.26,414.70) circle (  1.16);

\path[draw=drawColor,line width= 0.4pt,line join=round,line cap=round,fill=fillColor] (169.53,414.44) circle (  1.16);

\path[draw=drawColor,line width= 0.4pt,line join=round,line cap=round,fill=fillColor] (169.79,414.31) circle (  1.16);

\path[draw=drawColor,line width= 0.4pt,line join=round,line cap=round,fill=fillColor] (170.06,414.22) circle (  1.16);

\path[draw=drawColor,line width= 0.4pt,line join=round,line cap=round,fill=fillColor] (170.33,413.95) circle (  1.16);

\path[draw=drawColor,line width= 0.4pt,line join=round,line cap=round,fill=fillColor] (170.59,413.66) circle (  1.16);

\path[draw=drawColor,line width= 0.4pt,line join=round,line cap=round,fill=fillColor] (170.85,413.06) circle (  1.16);

\path[draw=drawColor,line width= 0.4pt,line join=round,line cap=round,fill=fillColor] (171.12,412.68) circle (  1.16);

\path[draw=drawColor,line width= 0.4pt,line join=round,line cap=round,fill=fillColor] (171.38,412.46) circle (  1.16);

\path[draw=drawColor,line width= 0.4pt,line join=round,line cap=round,fill=fillColor] (171.64,412.33) circle (  1.16);

\path[draw=drawColor,line width= 0.4pt,line join=round,line cap=round,fill=fillColor] (171.90,411.94) circle (  1.16);

\path[draw=drawColor,line width= 0.4pt,line join=round,line cap=round,fill=fillColor] (172.15,411.86) circle (  1.16);

\path[draw=drawColor,line width= 0.4pt,line join=round,line cap=round,fill=fillColor] (172.41,411.68) circle (  1.16);

\path[draw=drawColor,line width= 0.4pt,line join=round,line cap=round,fill=fillColor] (172.67,411.42) circle (  1.16);

\path[draw=drawColor,line width= 0.4pt,line join=round,line cap=round,fill=fillColor] (172.92,411.25) circle (  1.16);

\path[draw=drawColor,line width= 0.4pt,line join=round,line cap=round,fill=fillColor] (173.17,411.05) circle (  1.16);

\path[draw=drawColor,line width= 0.4pt,line join=round,line cap=round,fill=fillColor] (173.43,410.86) circle (  1.16);

\path[draw=drawColor,line width= 0.4pt,line join=round,line cap=round,fill=fillColor] (173.68,410.80) circle (  1.16);

\path[draw=drawColor,line width= 0.4pt,line join=round,line cap=round,fill=fillColor] (173.93,410.69) circle (  1.16);

\path[draw=drawColor,line width= 0.4pt,line join=round,line cap=round,fill=fillColor] (174.18,410.50) circle (  1.16);

\path[draw=drawColor,line width= 0.4pt,line join=round,line cap=round,fill=fillColor] (174.42,409.95) circle (  1.16);

\path[draw=drawColor,line width= 0.4pt,line join=round,line cap=round,fill=fillColor] (174.67,409.84) circle (  1.16);

\path[draw=drawColor,line width= 0.4pt,line join=round,line cap=round,fill=fillColor] (174.92,409.83) circle (  1.16);

\path[draw=drawColor,line width= 0.4pt,line join=round,line cap=round,fill=fillColor] (175.16,409.82) circle (  1.16);

\path[draw=drawColor,line width= 0.4pt,line join=round,line cap=round,fill=fillColor] (175.41,409.64) circle (  1.16);

\path[draw=drawColor,line width= 0.4pt,line join=round,line cap=round,fill=fillColor] (175.65,409.58) circle (  1.16);

\path[draw=drawColor,line width= 0.4pt,line join=round,line cap=round,fill=fillColor] (175.89,409.56) circle (  1.16);

\path[draw=drawColor,line width= 0.4pt,line join=round,line cap=round,fill=fillColor] (176.13,409.45) circle (  1.16);

\path[draw=drawColor,line width= 0.4pt,line join=round,line cap=round,fill=fillColor] (176.37,409.13) circle (  1.16);

\path[draw=drawColor,line width= 0.4pt,line join=round,line cap=round,fill=fillColor] (176.61,409.03) circle (  1.16);

\path[draw=drawColor,line width= 0.4pt,line join=round,line cap=round,fill=fillColor] (176.85,408.94) circle (  1.16);

\path[draw=drawColor,line width= 0.4pt,line join=round,line cap=round,fill=fillColor] (177.09,408.78) circle (  1.16);

\path[draw=drawColor,line width= 0.4pt,line join=round,line cap=round,fill=fillColor] (177.32,408.53) circle (  1.16);

\path[draw=drawColor,line width= 0.4pt,line join=round,line cap=round,fill=fillColor] (177.56,408.14) circle (  1.16);

\path[draw=drawColor,line width= 0.4pt,line join=round,line cap=round,fill=fillColor] (177.80,407.78) circle (  1.16);

\path[draw=drawColor,line width= 0.4pt,line join=round,line cap=round,fill=fillColor] (178.03,407.58) circle (  1.16);

\path[draw=drawColor,line width= 0.4pt,line join=round,line cap=round,fill=fillColor] (178.26,407.56) circle (  1.16);

\path[draw=drawColor,line width= 0.4pt,line join=round,line cap=round,fill=fillColor] (178.49,407.30) circle (  1.16);

\path[draw=drawColor,line width= 0.4pt,line join=round,line cap=round,fill=fillColor] (178.73,407.23) circle (  1.16);

\path[draw=drawColor,line width= 0.4pt,line join=round,line cap=round,fill=fillColor] (178.96,407.14) circle (  1.16);

\path[draw=drawColor,line width= 0.4pt,line join=round,line cap=round,fill=fillColor] (179.19,407.14) circle (  1.16);

\path[draw=drawColor,line width= 0.4pt,line join=round,line cap=round,fill=fillColor] (179.41,407.07) circle (  1.16);

\path[draw=drawColor,line width= 0.4pt,line join=round,line cap=round,fill=fillColor] (179.64,407.07) circle (  1.16);

\path[draw=drawColor,line width= 0.4pt,line join=round,line cap=round,fill=fillColor] (179.87,406.88) circle (  1.16);

\path[draw=drawColor,line width= 0.4pt,line join=round,line cap=round,fill=fillColor] (180.10,406.54) circle (  1.16);

\path[draw=drawColor,line width= 0.4pt,line join=round,line cap=round,fill=fillColor] (180.32,406.40) circle (  1.16);

\path[draw=drawColor,line width= 0.4pt,line join=round,line cap=round,fill=fillColor] (180.55,406.36) circle (  1.16);

\path[draw=drawColor,line width= 0.4pt,line join=round,line cap=round,fill=fillColor] (180.77,406.13) circle (  1.16);

\path[draw=drawColor,line width= 0.4pt,line join=round,line cap=round,fill=fillColor] (180.99,406.11) circle (  1.16);

\path[draw=drawColor,line width= 0.4pt,line join=round,line cap=round,fill=fillColor] (181.22,406.05) circle (  1.16);

\path[draw=drawColor,line width= 0.4pt,line join=round,line cap=round,fill=fillColor] (181.44,405.87) circle (  1.16);

\path[draw=drawColor,line width= 0.4pt,line join=round,line cap=round,fill=fillColor] (181.66,405.85) circle (  1.16);

\path[draw=drawColor,line width= 0.4pt,line join=round,line cap=round,fill=fillColor] (181.88,405.65) circle (  1.16);

\path[draw=drawColor,line width= 0.4pt,line join=round,line cap=round,fill=fillColor] (182.10,405.42) circle (  1.16);

\path[draw=drawColor,line width= 0.4pt,line join=round,line cap=round,fill=fillColor] (182.32,405.32) circle (  1.16);

\path[draw=drawColor,line width= 0.4pt,line join=round,line cap=round,fill=fillColor] (182.54,405.20) circle (  1.16);

\path[draw=drawColor,line width= 0.4pt,line join=round,line cap=round,fill=fillColor] (182.75,404.74) circle (  1.16);

\path[draw=drawColor,line width= 0.4pt,line join=round,line cap=round,fill=fillColor] (182.97,404.72) circle (  1.16);

\path[draw=drawColor,line width= 0.4pt,line join=round,line cap=round,fill=fillColor] (183.19,404.66) circle (  1.16);

\path[draw=drawColor,line width= 0.4pt,line join=round,line cap=round,fill=fillColor] (183.40,404.54) circle (  1.16);

\path[draw=drawColor,line width= 0.4pt,line join=round,line cap=round,fill=fillColor] (183.61,404.26) circle (  1.16);

\path[draw=drawColor,line width= 0.4pt,line join=round,line cap=round,fill=fillColor] (183.83,404.15) circle (  1.16);

\path[draw=drawColor,line width= 0.4pt,line join=round,line cap=round,fill=fillColor] (184.04,404.12) circle (  1.16);

\path[draw=drawColor,line width= 0.4pt,line join=round,line cap=round,fill=fillColor] (184.25,404.02) circle (  1.16);

\path[draw=drawColor,line width= 0.4pt,line join=round,line cap=round,fill=fillColor] (184.47,403.96) circle (  1.16);

\path[draw=drawColor,line width= 0.4pt,line join=round,line cap=round,fill=fillColor] (184.68,403.95) circle (  1.16);

\path[draw=drawColor,line width= 0.4pt,line join=round,line cap=round,fill=fillColor] (184.89,403.91) circle (  1.16);

\path[draw=drawColor,line width= 0.4pt,line join=round,line cap=round,fill=fillColor] (185.10,403.86) circle (  1.16);

\path[draw=drawColor,line width= 0.4pt,line join=round,line cap=round,fill=fillColor] (185.31,403.70) circle (  1.16);

\path[draw=drawColor,line width= 0.4pt,line join=round,line cap=round,fill=fillColor] (185.51,403.54) circle (  1.16);

\path[draw=drawColor,line width= 0.4pt,line join=round,line cap=round,fill=fillColor] (185.72,403.43) circle (  1.16);

\path[draw=drawColor,line width= 0.4pt,line join=round,line cap=round,fill=fillColor] (185.93,403.40) circle (  1.16);

\path[draw=drawColor,line width= 0.4pt,line join=round,line cap=round,fill=fillColor] (186.13,403.24) circle (  1.16);

\path[draw=drawColor,line width= 0.4pt,line join=round,line cap=round,fill=fillColor] (186.34,403.18) circle (  1.16);

\path[draw=drawColor,line width= 0.4pt,line join=round,line cap=round,fill=fillColor] (186.55,403.12) circle (  1.16);

\path[draw=drawColor,line width= 0.4pt,line join=round,line cap=round,fill=fillColor] (186.75,403.07) circle (  1.16);

\path[draw=drawColor,line width= 0.4pt,line join=round,line cap=round,fill=fillColor] (186.95,403.03) circle (  1.16);

\path[draw=drawColor,line width= 0.4pt,line join=round,line cap=round,fill=fillColor] (187.16,402.95) circle (  1.16);

\path[draw=drawColor,line width= 0.4pt,line join=round,line cap=round,fill=fillColor] (187.36,402.63) circle (  1.16);

\path[draw=drawColor,line width= 0.4pt,line join=round,line cap=round,fill=fillColor] (187.56,402.57) circle (  1.16);

\path[draw=drawColor,line width= 0.4pt,line join=round,line cap=round,fill=fillColor] (187.76,402.38) circle (  1.16);

\path[draw=drawColor,line width= 0.4pt,line join=round,line cap=round,fill=fillColor] (187.96,402.38) circle (  1.16);

\path[draw=drawColor,line width= 0.4pt,line join=round,line cap=round,fill=fillColor] (188.16,402.22) circle (  1.16);

\path[draw=drawColor,line width= 0.4pt,line join=round,line cap=round,fill=fillColor] (188.36,402.02) circle (  1.16);

\path[draw=drawColor,line width= 0.4pt,line join=round,line cap=round,fill=fillColor] (188.56,402.01) circle (  1.16);

\path[draw=drawColor,line width= 0.4pt,line join=round,line cap=round,fill=fillColor] (188.76,401.92) circle (  1.16);

\path[draw=drawColor,line width= 0.4pt,line join=round,line cap=round,fill=fillColor] (188.96,401.61) circle (  1.16);

\path[draw=drawColor,line width= 0.4pt,line join=round,line cap=round,fill=fillColor] (189.16,401.51) circle (  1.16);

\path[draw=drawColor,line width= 0.4pt,line join=round,line cap=round,fill=fillColor] (189.35,401.40) circle (  1.16);

\path[draw=drawColor,line width= 0.4pt,line join=round,line cap=round,fill=fillColor] (189.55,401.38) circle (  1.16);

\path[draw=drawColor,line width= 0.4pt,line join=round,line cap=round,fill=fillColor] (189.74,401.33) circle (  1.16);

\path[draw=drawColor,line width= 0.4pt,line join=round,line cap=round,fill=fillColor] (189.94,401.32) circle (  1.16);

\path[draw=drawColor,line width= 0.4pt,line join=round,line cap=round,fill=fillColor] (190.13,401.26) circle (  1.16);

\path[draw=drawColor,line width= 0.4pt,line join=round,line cap=round,fill=fillColor] (190.33,401.25) circle (  1.16);

\path[draw=drawColor,line width= 0.4pt,line join=round,line cap=round,fill=fillColor] (190.52,401.20) circle (  1.16);

\path[draw=drawColor,line width= 0.4pt,line join=round,line cap=round,fill=fillColor] (190.71,401.13) circle (  1.16);

\path[draw=drawColor,line width= 0.4pt,line join=round,line cap=round,fill=fillColor] (190.91,401.06) circle (  1.16);

\path[draw=drawColor,line width= 0.4pt,line join=round,line cap=round,fill=fillColor] (191.10,401.04) circle (  1.16);

\path[draw=drawColor,line width= 0.4pt,line join=round,line cap=round,fill=fillColor] (191.29,400.94) circle (  1.16);

\path[draw=drawColor,line width= 0.4pt,line join=round,line cap=round,fill=fillColor] (191.48,400.93) circle (  1.16);

\path[draw=drawColor,line width= 0.4pt,line join=round,line cap=round,fill=fillColor] (191.67,400.87) circle (  1.16);

\path[draw=drawColor,line width= 0.4pt,line join=round,line cap=round,fill=fillColor] (191.86,400.85) circle (  1.16);

\path[draw=drawColor,line width= 0.4pt,line join=round,line cap=round,fill=fillColor] (192.05,400.74) circle (  1.16);

\path[draw=drawColor,line width= 0.4pt,line join=round,line cap=round,fill=fillColor] (192.24,400.65) circle (  1.16);

\path[draw=drawColor,line width= 0.4pt,line join=round,line cap=round,fill=fillColor] (192.43,400.63) circle (  1.16);

\path[draw=drawColor,line width= 0.4pt,line join=round,line cap=round,fill=fillColor] (192.61,400.62) circle (  1.16);

\path[draw=drawColor,line width= 0.4pt,line join=round,line cap=round,fill=fillColor] (192.80,400.56) circle (  1.16);

\path[draw=drawColor,line width= 0.4pt,line join=round,line cap=round,fill=fillColor] (192.99,400.45) circle (  1.16);

\path[draw=drawColor,line width= 0.4pt,line join=round,line cap=round,fill=fillColor] (193.17,400.40) circle (  1.16);

\path[draw=drawColor,line width= 0.4pt,line join=round,line cap=round,fill=fillColor] (193.36,400.30) circle (  1.16);

\path[draw=drawColor,line width= 0.4pt,line join=round,line cap=round,fill=fillColor] (193.54,400.23) circle (  1.16);

\path[draw=drawColor,line width= 0.4pt,line join=round,line cap=round,fill=fillColor] (193.73,400.17) circle (  1.16);

\path[draw=drawColor,line width= 0.4pt,line join=round,line cap=round,fill=fillColor] (193.91,400.10) circle (  1.16);

\path[draw=drawColor,line width= 0.4pt,line join=round,line cap=round,fill=fillColor] (194.10,400.06) circle (  1.16);

\path[draw=drawColor,line width= 0.4pt,line join=round,line cap=round,fill=fillColor] (194.28,399.86) circle (  1.16);

\path[draw=drawColor,line width= 0.4pt,line join=round,line cap=round,fill=fillColor] (194.46,399.76) circle (  1.16);

\path[draw=drawColor,line width= 0.4pt,line join=round,line cap=round,fill=fillColor] (194.65,399.65) circle (  1.16);

\path[draw=drawColor,line width= 0.4pt,line join=round,line cap=round,fill=fillColor] (194.83,399.61) circle (  1.16);

\path[draw=drawColor,line width= 0.4pt,line join=round,line cap=round,fill=fillColor] (195.01,399.61) circle (  1.16);

\path[draw=drawColor,line width= 0.4pt,line join=round,line cap=round,fill=fillColor] (195.19,399.53) circle (  1.16);

\path[draw=drawColor,line width= 0.4pt,line join=round,line cap=round,fill=fillColor] (195.37,399.48) circle (  1.16);

\path[draw=drawColor,line width= 0.4pt,line join=round,line cap=round,fill=fillColor] (195.55,399.42) circle (  1.16);

\path[draw=drawColor,line width= 0.4pt,line join=round,line cap=round,fill=fillColor] (195.73,399.41) circle (  1.16);

\path[draw=drawColor,line width= 0.4pt,line join=round,line cap=round,fill=fillColor] (195.91,399.16) circle (  1.16);

\path[draw=drawColor,line width= 0.4pt,line join=round,line cap=round,fill=fillColor] (196.09,399.00) circle (  1.16);

\path[draw=drawColor,line width= 0.4pt,line join=round,line cap=round,fill=fillColor] (196.27,398.99) circle (  1.16);

\path[draw=drawColor,line width= 0.4pt,line join=round,line cap=round,fill=fillColor] (196.44,398.25) circle (  1.16);

\path[draw=drawColor,line width= 0.4pt,line join=round,line cap=round,fill=fillColor] (196.62,398.11) circle (  1.16);

\path[draw=drawColor,line width= 0.4pt,line join=round,line cap=round,fill=fillColor] (196.80,398.08) circle (  1.16);

\path[draw=drawColor,line width= 0.4pt,line join=round,line cap=round,fill=fillColor] (196.97,397.98) circle (  1.16);

\path[draw=drawColor,line width= 0.4pt,line join=round,line cap=round,fill=fillColor] (197.15,397.88) circle (  1.16);

\path[draw=drawColor,line width= 0.4pt,line join=round,line cap=round,fill=fillColor] (197.33,397.76) circle (  1.16);

\path[draw=drawColor,line width= 0.4pt,line join=round,line cap=round,fill=fillColor] (197.50,397.69) circle (  1.16);

\path[draw=drawColor,line width= 0.4pt,line join=round,line cap=round,fill=fillColor] (197.68,397.53) circle (  1.16);

\path[draw=drawColor,line width= 0.4pt,line join=round,line cap=round,fill=fillColor] (197.85,397.45) circle (  1.16);

\path[draw=drawColor,line width= 0.4pt,line join=round,line cap=round,fill=fillColor] (198.02,397.29) circle (  1.16);

\path[draw=drawColor,line width= 0.4pt,line join=round,line cap=round,fill=fillColor] (198.20,397.28) circle (  1.16);

\path[draw=drawColor,line width= 0.4pt,line join=round,line cap=round,fill=fillColor] (198.37,397.22) circle (  1.16);

\path[draw=drawColor,line width= 0.4pt,line join=round,line cap=round,fill=fillColor] (198.54,397.14) circle (  1.16);

\path[draw=drawColor,line width= 0.4pt,line join=round,line cap=round,fill=fillColor] (198.72,397.00) circle (  1.16);

\path[draw=drawColor,line width= 0.4pt,line join=round,line cap=round,fill=fillColor] (198.89,396.98) circle (  1.16);

\path[draw=drawColor,line width= 0.4pt,line join=round,line cap=round,fill=fillColor] (199.06,396.89) circle (  1.16);

\path[draw=drawColor,line width= 0.4pt,line join=round,line cap=round,fill=fillColor] (199.23,396.88) circle (  1.16);

\path[draw=drawColor,line width= 0.4pt,line join=round,line cap=round,fill=fillColor] (199.40,396.77) circle (  1.16);

\path[draw=drawColor,line width= 0.4pt,line join=round,line cap=round,fill=fillColor] (199.57,396.74) circle (  1.16);

\path[draw=drawColor,line width= 0.4pt,line join=round,line cap=round,fill=fillColor] (199.74,396.61) circle (  1.16);

\path[draw=drawColor,line width= 0.4pt,line join=round,line cap=round,fill=fillColor] (199.91,396.58) circle (  1.16);

\path[draw=drawColor,line width= 0.4pt,line join=round,line cap=round,fill=fillColor] (200.08,396.57) circle (  1.16);

\path[draw=drawColor,line width= 0.4pt,line join=round,line cap=round,fill=fillColor] (200.25,396.56) circle (  1.16);

\path[draw=drawColor,line width= 0.4pt,line join=round,line cap=round,fill=fillColor] (200.42,396.55) circle (  1.16);

\path[draw=drawColor,line width= 0.4pt,line join=round,line cap=round,fill=fillColor] (200.58,396.54) circle (  1.16);

\path[draw=drawColor,line width= 0.4pt,line join=round,line cap=round,fill=fillColor] (200.75,396.25) circle (  1.16);

\path[draw=drawColor,line width= 0.4pt,line join=round,line cap=round,fill=fillColor] (200.92,396.06) circle (  1.16);

\path[draw=drawColor,line width= 0.4pt,line join=round,line cap=round,fill=fillColor] (201.09,396.05) circle (  1.16);

\path[draw=drawColor,line width= 0.4pt,line join=round,line cap=round,fill=fillColor] (201.25,396.01) circle (  1.16);

\path[draw=drawColor,line width= 0.4pt,line join=round,line cap=round,fill=fillColor] (201.42,395.98) circle (  1.16);

\path[draw=drawColor,line width= 0.4pt,line join=round,line cap=round,fill=fillColor] (201.58,395.95) circle (  1.16);

\path[draw=drawColor,line width= 0.4pt,line join=round,line cap=round,fill=fillColor] (201.75,395.90) circle (  1.16);

\path[draw=drawColor,line width= 0.4pt,line join=round,line cap=round,fill=fillColor] (201.91,395.82) circle (  1.16);

\path[draw=drawColor,line width= 0.4pt,line join=round,line cap=round,fill=fillColor] (202.08,395.78) circle (  1.16);

\path[draw=drawColor,line width= 0.4pt,line join=round,line cap=round,fill=fillColor] (202.24,395.76) circle (  1.16);

\path[draw=drawColor,line width= 0.4pt,line join=round,line cap=round,fill=fillColor] (202.41,395.75) circle (  1.16);

\path[draw=drawColor,line width= 0.4pt,line join=round,line cap=round,fill=fillColor] (202.57,395.64) circle (  1.16);

\path[draw=drawColor,line width= 0.4pt,line join=round,line cap=round,fill=fillColor] (202.73,395.31) circle (  1.16);

\path[draw=drawColor,line width= 0.4pt,line join=round,line cap=round,fill=fillColor] (202.90,395.22) circle (  1.16);

\path[draw=drawColor,line width= 0.4pt,line join=round,line cap=round,fill=fillColor] (203.06,395.15) circle (  1.16);

\path[draw=drawColor,line width= 0.4pt,line join=round,line cap=round,fill=fillColor] (203.22,394.92) circle (  1.16);

\path[draw=drawColor,line width= 0.4pt,line join=round,line cap=round,fill=fillColor] (203.38,394.87) circle (  1.16);

\path[draw=drawColor,line width= 0.4pt,line join=round,line cap=round,fill=fillColor] (203.54,394.78) circle (  1.16);

\path[draw=drawColor,line width= 0.4pt,line join=round,line cap=round,fill=fillColor] (203.71,394.75) circle (  1.16);

\path[draw=drawColor,line width= 0.4pt,line join=round,line cap=round,fill=fillColor] (203.87,394.73) circle (  1.16);

\path[draw=drawColor,line width= 0.4pt,line join=round,line cap=round,fill=fillColor] (204.03,394.65) circle (  1.16);

\path[draw=drawColor,line width= 0.4pt,line join=round,line cap=round,fill=fillColor] (204.19,394.63) circle (  1.16);

\path[draw=drawColor,line width= 0.4pt,line join=round,line cap=round,fill=fillColor] (204.35,394.63) circle (  1.16);

\path[draw=drawColor,line width= 0.4pt,line join=round,line cap=round,fill=fillColor] (204.51,394.60) circle (  1.16);

\path[draw=drawColor,line width= 0.4pt,line join=round,line cap=round,fill=fillColor] (204.66,394.55) circle (  1.16);

\path[draw=drawColor,line width= 0.4pt,line join=round,line cap=round,fill=fillColor] (204.82,394.45) circle (  1.16);

\path[draw=drawColor,line width= 0.4pt,line join=round,line cap=round,fill=fillColor] (204.98,394.41) circle (  1.16);

\path[draw=drawColor,line width= 0.4pt,line join=round,line cap=round,fill=fillColor] (205.14,394.38) circle (  1.16);

\path[draw=drawColor,line width= 0.4pt,line join=round,line cap=round,fill=fillColor] (205.30,394.22) circle (  1.16);

\path[draw=drawColor,line width= 0.4pt,line join=round,line cap=round,fill=fillColor] (205.45,394.19) circle (  1.16);

\path[draw=drawColor,line width= 0.4pt,line join=round,line cap=round,fill=fillColor] (205.61,394.14) circle (  1.16);

\path[draw=drawColor,line width= 0.4pt,line join=round,line cap=round,fill=fillColor] (205.77,393.85) circle (  1.16);

\path[draw=drawColor,line width= 0.4pt,line join=round,line cap=round,fill=fillColor] (205.92,393.71) circle (  1.16);

\path[draw=drawColor,line width= 0.4pt,line join=round,line cap=round,fill=fillColor] (206.08,393.66) circle (  1.16);

\path[draw=drawColor,line width= 0.4pt,line join=round,line cap=round,fill=fillColor] (206.24,393.57) circle (  1.16);

\path[draw=drawColor,line width= 0.4pt,line join=round,line cap=round,fill=fillColor] (206.39,393.50) circle (  1.16);

\path[draw=drawColor,line width= 0.4pt,line join=round,line cap=round,fill=fillColor] (206.55,393.49) circle (  1.16);

\path[draw=drawColor,line width= 0.4pt,line join=round,line cap=round,fill=fillColor] (206.70,393.27) circle (  1.16);

\path[draw=drawColor,line width= 0.4pt,line join=round,line cap=round,fill=fillColor] (206.86,393.15) circle (  1.16);

\path[draw=drawColor,line width= 0.4pt,line join=round,line cap=round,fill=fillColor] (207.01,393.04) circle (  1.16);

\path[draw=drawColor,line width= 0.4pt,line join=round,line cap=round,fill=fillColor] (207.16,392.96) circle (  1.16);

\path[draw=drawColor,line width= 0.4pt,line join=round,line cap=round,fill=fillColor] (207.32,392.93) circle (  1.16);

\path[draw=drawColor,line width= 0.4pt,line join=round,line cap=round,fill=fillColor] (207.47,392.83) circle (  1.16);

\path[draw=drawColor,line width= 0.4pt,line join=round,line cap=round,fill=fillColor] (207.62,392.77) circle (  1.16);

\path[draw=drawColor,line width= 0.4pt,line join=round,line cap=round,fill=fillColor] (207.78,392.63) circle (  1.16);

\path[draw=drawColor,line width= 0.4pt,line join=round,line cap=round,fill=fillColor] (207.93,392.53) circle (  1.16);

\path[draw=drawColor,line width= 0.4pt,line join=round,line cap=round,fill=fillColor] (208.08,392.49) circle (  1.16);

\path[draw=drawColor,line width= 0.4pt,line join=round,line cap=round,fill=fillColor] (208.23,392.32) circle (  1.16);

\path[draw=drawColor,line width= 0.4pt,line join=round,line cap=round,fill=fillColor] (208.39,391.87) circle (  1.16);

\path[draw=drawColor,line width= 0.4pt,line join=round,line cap=round,fill=fillColor] (208.54,391.83) circle (  1.16);

\path[draw=drawColor,line width= 0.4pt,line join=round,line cap=round,fill=fillColor] (208.69,391.79) circle (  1.16);

\path[draw=drawColor,line width= 0.4pt,line join=round,line cap=round,fill=fillColor] (208.84,391.50) circle (  1.16);

\path[draw=drawColor,line width= 0.4pt,line join=round,line cap=round,fill=fillColor] (208.99,391.45) circle (  1.16);

\path[draw=drawColor,line width= 0.4pt,line join=round,line cap=round,fill=fillColor] (209.14,391.37) circle (  1.16);

\path[draw=drawColor,line width= 0.4pt,line join=round,line cap=round,fill=fillColor] (209.29,391.30) circle (  1.16);

\path[draw=drawColor,line width= 0.4pt,line join=round,line cap=round,fill=fillColor] (209.44,391.26) circle (  1.16);

\path[draw=drawColor,line width= 0.4pt,line join=round,line cap=round,fill=fillColor] (209.59,391.24) circle (  1.16);

\path[draw=drawColor,line width= 0.4pt,line join=round,line cap=round,fill=fillColor] (209.74,391.17) circle (  1.16);

\path[draw=drawColor,line width= 0.4pt,line join=round,line cap=round,fill=fillColor] (209.88,391.17) circle (  1.16);

\path[draw=drawColor,line width= 0.4pt,line join=round,line cap=round,fill=fillColor] (210.03,391.11) circle (  1.16);

\path[draw=drawColor,line width= 0.4pt,line join=round,line cap=round,fill=fillColor] (210.18,391.07) circle (  1.16);

\path[draw=drawColor,line width= 0.4pt,line join=round,line cap=round,fill=fillColor] (210.33,391.04) circle (  1.16);

\path[draw=drawColor,line width= 0.4pt,line join=round,line cap=round,fill=fillColor] (210.48,391.02) circle (  1.16);

\path[draw=drawColor,line width= 0.4pt,line join=round,line cap=round,fill=fillColor] (210.62,390.80) circle (  1.16);

\path[draw=drawColor,line width= 0.4pt,line join=round,line cap=round,fill=fillColor] (210.77,390.69) circle (  1.16);

\path[draw=drawColor,line width= 0.4pt,line join=round,line cap=round,fill=fillColor] (210.92,390.59) circle (  1.16);

\path[draw=drawColor,line width= 0.4pt,line join=round,line cap=round,fill=fillColor] (211.06,390.54) circle (  1.16);

\path[draw=drawColor,line width= 0.4pt,line join=round,line cap=round,fill=fillColor] (211.21,390.52) circle (  1.16);

\path[draw=drawColor,line width= 0.4pt,line join=round,line cap=round,fill=fillColor] (211.36,390.46) circle (  1.16);

\path[draw=drawColor,line width= 0.4pt,line join=round,line cap=round,fill=fillColor] (211.50,390.43) circle (  1.16);

\path[draw=drawColor,line width= 0.4pt,line join=round,line cap=round,fill=fillColor] (211.65,390.31) circle (  1.16);

\path[draw=drawColor,line width= 0.4pt,line join=round,line cap=round,fill=fillColor] (211.79,390.22) circle (  1.16);

\path[draw=drawColor,line width= 0.4pt,line join=round,line cap=round,fill=fillColor] (211.94,390.19) circle (  1.16);

\path[draw=drawColor,line width= 0.4pt,line join=round,line cap=round,fill=fillColor] (212.08,390.14) circle (  1.16);

\path[draw=drawColor,line width= 0.4pt,line join=round,line cap=round,fill=fillColor] (212.23,390.11) circle (  1.16);

\path[draw=drawColor,line width= 0.4pt,line join=round,line cap=round,fill=fillColor] (212.37,390.07) circle (  1.16);

\path[draw=drawColor,line width= 0.4pt,line join=round,line cap=round,fill=fillColor] (212.51,389.98) circle (  1.16);

\path[draw=drawColor,line width= 0.4pt,line join=round,line cap=round,fill=fillColor] (212.66,389.42) circle (  1.16);

\path[draw=drawColor,line width= 0.4pt,line join=round,line cap=round,fill=fillColor] (212.80,389.23) circle (  1.16);

\path[draw=drawColor,line width= 0.4pt,line join=round,line cap=round,fill=fillColor] (212.94,389.19) circle (  1.16);

\path[draw=drawColor,line width= 0.4pt,line join=round,line cap=round,fill=fillColor] (213.09,389.13) circle (  1.16);

\path[draw=drawColor,line width= 0.4pt,line join=round,line cap=round,fill=fillColor] (213.23,388.71) circle (  1.16);

\path[draw=drawColor,line width= 0.4pt,line join=round,line cap=round,fill=fillColor] (213.37,388.66) circle (  1.16);

\path[draw=drawColor,line width= 0.4pt,line join=round,line cap=round,fill=fillColor] (213.51,388.38) circle (  1.16);

\path[draw=drawColor,line width= 0.4pt,line join=round,line cap=round,fill=fillColor] (213.65,388.25) circle (  1.16);

\path[draw=drawColor,line width= 0.4pt,line join=round,line cap=round,fill=fillColor] (213.80,388.25) circle (  1.16);

\path[draw=drawColor,line width= 0.4pt,line join=round,line cap=round,fill=fillColor] (213.94,388.23) circle (  1.16);

\path[draw=drawColor,line width= 0.4pt,line join=round,line cap=round,fill=fillColor] (214.08,388.14) circle (  1.16);

\path[draw=drawColor,line width= 0.4pt,line join=round,line cap=round,fill=fillColor] (214.22,387.94) circle (  1.16);

\path[draw=drawColor,line width= 0.4pt,line join=round,line cap=round,fill=fillColor] (214.36,387.92) circle (  1.16);

\path[draw=drawColor,line width= 0.4pt,line join=round,line cap=round,fill=fillColor] (214.50,387.53) circle (  1.16);

\path[draw=drawColor,line width= 0.4pt,line join=round,line cap=round,fill=fillColor] (214.64,387.33) circle (  1.16);

\path[draw=drawColor,line width= 0.4pt,line join=round,line cap=round,fill=fillColor] (214.78,387.18) circle (  1.16);

\path[draw=drawColor,line width= 0.4pt,line join=round,line cap=round,fill=fillColor] (214.92,387.17) circle (  1.16);

\path[draw=drawColor,line width= 0.4pt,line join=round,line cap=round,fill=fillColor] (215.06,387.13) circle (  1.16);

\path[draw=drawColor,line width= 0.4pt,line join=round,line cap=round,fill=fillColor] (215.20,386.50) circle (  1.16);

\path[draw=drawColor,line width= 0.4pt,line join=round,line cap=round,fill=fillColor] (215.34,386.43) circle (  1.16);

\path[draw=drawColor,line width= 0.4pt,line join=round,line cap=round,fill=fillColor] (215.47,386.39) circle (  1.16);

\path[draw=drawColor,line width= 0.4pt,line join=round,line cap=round,fill=fillColor] (215.61,386.21) circle (  1.16);

\path[draw=drawColor,line width= 0.4pt,line join=round,line cap=round,fill=fillColor] (215.75,385.39) circle (  1.16);

\path[draw=drawColor,line width= 0.4pt,line join=round,line cap=round,fill=fillColor] (215.89,385.30) circle (  1.16);

\path[draw=drawColor,line width= 0.4pt,line join=round,line cap=round,fill=fillColor] (216.03,384.66) circle (  1.16);

\path[draw=drawColor,line width= 0.4pt,line join=round,line cap=round,fill=fillColor] (216.16,384.13) circle (  1.16);

\path[draw=drawColor,line width= 0.4pt,line join=round,line cap=round,fill=fillColor] (216.30,383.83) circle (  1.16);

\path[draw=drawColor,line width= 0.4pt,line join=round,line cap=round,fill=fillColor] (216.44,383.82) circle (  1.16);

\path[draw=drawColor,line width= 0.4pt,line join=round,line cap=round,fill=fillColor] (216.58,383.39) circle (  1.16);

\path[draw=drawColor,line width= 0.4pt,line join=round,line cap=round,fill=fillColor] (216.71,383.28) circle (  1.16);

\path[draw=drawColor,line width= 0.4pt,line join=round,line cap=round,fill=fillColor] (216.85,381.58) circle (  1.16);

\path[draw=drawColor,line width= 0.4pt,line join=round,line cap=round,fill=fillColor] (216.98,381.32) circle (  1.16);

\path[draw=drawColor,line width= 0.4pt,line join=round,line cap=round,fill=fillColor] (217.12,381.09) circle (  1.16);

\path[draw=drawColor,line width= 0.4pt,line join=round,line cap=round,fill=fillColor] (217.26,378.34) circle (  1.16);

\path[draw=drawColor,line width= 0.4pt,line join=round,line cap=round,fill=fillColor] (217.39,372.42) circle (  1.16);

\path[draw=drawColor,line width= 0.4pt,line join=round,line cap=round,fill=fillColor] (217.53,372.42) circle (  1.16);

\path[draw=drawColor,line width= 0.4pt,line join=round,line cap=round,fill=fillColor] (217.66,372.42) circle (  1.16);

\path[draw=drawColor,line width= 0.4pt,line join=round,line cap=round,fill=fillColor] (217.80,372.42) circle (  1.16);

\path[draw=drawColor,line width= 0.4pt,line join=round,line cap=round,fill=fillColor] (217.93,372.42) circle (  1.16);

\path[draw=drawColor,line width= 0.4pt,line join=round,line cap=round,fill=fillColor] (218.06,372.42) circle (  1.16);

\path[draw=drawColor,line width= 0.4pt,line join=round,line cap=round,fill=fillColor] (218.20,372.42) circle (  1.16);

\path[draw=drawColor,line width= 0.4pt,line join=round,line cap=round,fill=fillColor] (218.33,372.42) circle (  1.16);

\path[draw=drawColor,line width= 0.4pt,line join=round,line cap=round,fill=fillColor] (218.47,372.42) circle (  1.16);

\path[draw=drawColor,line width= 0.4pt,line join=round,line cap=round,fill=fillColor] (218.60,372.42) circle (  1.16);

\path[draw=drawColor,line width= 0.4pt,line join=round,line cap=round,fill=fillColor] (218.73,372.42) circle (  1.16);

\path[draw=drawColor,line width= 0.4pt,line join=round,line cap=round,fill=fillColor] (218.87,372.42) circle (  1.16);

\path[draw=drawColor,line width= 0.4pt,line join=round,line cap=round,fill=fillColor] (219.00,372.42) circle (  1.16);

\path[draw=drawColor,line width= 0.4pt,line join=round,line cap=round,fill=fillColor] (219.13,372.42) circle (  1.16);

\path[draw=drawColor,line width= 0.4pt,line join=round,line cap=round,fill=fillColor] (219.26,372.42) circle (  1.16);

\path[draw=drawColor,line width= 0.4pt,line join=round,line cap=round,fill=fillColor] (219.40,372.42) circle (  1.16);

\path[draw=drawColor,line width= 0.4pt,line join=round,line cap=round,fill=fillColor] (219.53,372.42) circle (  1.16);

\path[draw=drawColor,line width= 0.4pt,line join=round,line cap=round,fill=fillColor] (219.66,372.42) circle (  1.16);

\path[draw=drawColor,line width= 0.4pt,line join=round,line cap=round,fill=fillColor] (219.79,372.42) circle (  1.16);

\path[draw=drawColor,line width= 0.4pt,line join=round,line cap=round,fill=fillColor] (219.92,372.42) circle (  1.16);

\path[draw=drawColor,line width= 0.4pt,line join=round,line cap=round,fill=fillColor] (220.05,372.42) circle (  1.16);

\path[draw=drawColor,line width= 0.4pt,line join=round,line cap=round,fill=fillColor] (220.18,372.42) circle (  1.16);

\path[draw=drawColor,line width= 0.4pt,line join=round,line cap=round,fill=fillColor] (220.31,372.42) circle (  1.16);

\path[draw=drawColor,line width= 0.4pt,line join=round,line cap=round,fill=fillColor] (220.45,372.42) circle (  1.16);

\path[draw=drawColor,line width= 0.4pt,line join=round,line cap=round,fill=fillColor] (220.58,372.42) circle (  1.16);

\path[draw=drawColor,line width= 0.4pt,line join=round,line cap=round,fill=fillColor] (220.71,372.42) circle (  1.16);

\path[draw=drawColor,line width= 0.4pt,line join=round,line cap=round,fill=fillColor] (220.84,372.42) circle (  1.16);

\path[draw=drawColor,line width= 0.4pt,line join=round,line cap=round,fill=fillColor] (220.97,372.42) circle (  1.16);

\path[draw=drawColor,line width= 0.4pt,line join=round,line cap=round,fill=fillColor] (221.09,372.42) circle (  1.16);

\path[draw=drawColor,line width= 0.4pt,line join=round,line cap=round,fill=fillColor] (221.22,372.42) circle (  1.16);

\path[draw=drawColor,line width= 0.4pt,line join=round,line cap=round,fill=fillColor] (221.35,372.42) circle (  1.16);

\path[draw=drawColor,line width= 0.4pt,line join=round,line cap=round,fill=fillColor] (221.48,372.42) circle (  1.16);

\path[draw=drawColor,line width= 0.4pt,line join=round,line cap=round,fill=fillColor] (221.61,372.42) circle (  1.16);

\path[draw=drawColor,line width= 0.4pt,line join=round,line cap=round,fill=fillColor] (221.74,372.42) circle (  1.16);

\path[draw=drawColor,line width= 0.4pt,line join=round,line cap=round,fill=fillColor] (221.87,372.42) circle (  1.16);

\path[draw=drawColor,line width= 0.4pt,line join=round,line cap=round,fill=fillColor] (222.00,372.42) circle (  1.16);

\path[draw=drawColor,line width= 0.4pt,line join=round,line cap=round,fill=fillColor] (222.12,372.42) circle (  1.16);

\path[draw=drawColor,line width= 0.4pt,line join=round,line cap=round,fill=fillColor] (222.25,372.42) circle (  1.16);

\path[draw=drawColor,line width= 0.4pt,line join=round,line cap=round,fill=fillColor] (222.38,372.42) circle (  1.16);

\path[draw=drawColor,line width= 0.4pt,line join=round,line cap=round,fill=fillColor] (222.51,372.42) circle (  1.16);

\path[draw=drawColor,line width= 0.4pt,line join=round,line cap=round,fill=fillColor] (222.63,372.42) circle (  1.16);

\path[draw=drawColor,line width= 0.4pt,line join=round,line cap=round,fill=fillColor] (222.76,372.42) circle (  1.16);

\path[draw=drawColor,line width= 0.4pt,line join=round,line cap=round,fill=fillColor] (222.89,372.42) circle (  1.16);

\path[draw=drawColor,line width= 0.4pt,line join=round,line cap=round,fill=fillColor] (223.01,372.42) circle (  1.16);

\path[draw=drawColor,line width= 0.4pt,line join=round,line cap=round,fill=fillColor] (223.14,372.42) circle (  1.16);

\path[draw=drawColor,line width= 0.4pt,line join=round,line cap=round,fill=fillColor] (223.27,372.42) circle (  1.16);

\path[draw=drawColor,line width= 0.4pt,line join=round,line cap=round,fill=fillColor] (223.39,372.42) circle (  1.16);

\path[draw=drawColor,line width= 0.4pt,line join=round,line cap=round,fill=fillColor] (223.52,372.42) circle (  1.16);

\path[draw=drawColor,line width= 0.4pt,line join=round,line cap=round,fill=fillColor] (223.64,372.42) circle (  1.16);

\path[draw=drawColor,line width= 0.4pt,line join=round,line cap=round,fill=fillColor] (223.77,372.42) circle (  1.16);

\path[draw=drawColor,line width= 0.4pt,line join=round,line cap=round,fill=fillColor] (223.89,372.42) circle (  1.16);

\path[draw=drawColor,line width= 0.4pt,line join=round,line cap=round,fill=fillColor] (224.02,372.42) circle (  1.16);

\path[draw=drawColor,line width= 0.4pt,line join=round,line cap=round,fill=fillColor] (224.14,372.42) circle (  1.16);

\path[draw=drawColor,line width= 0.4pt,line join=round,line cap=round,fill=fillColor] (224.27,372.42) circle (  1.16);

\path[draw=drawColor,line width= 0.4pt,line join=round,line cap=round,fill=fillColor] (224.39,372.42) circle (  1.16);

\path[draw=drawColor,line width= 0.4pt,line join=round,line cap=round,fill=fillColor] (224.52,372.42) circle (  1.16);

\path[draw=drawColor,line width= 0.4pt,line join=round,line cap=round,fill=fillColor] (224.64,372.42) circle (  1.16);

\path[draw=drawColor,line width= 0.4pt,line join=round,line cap=round,fill=fillColor] (224.77,372.42) circle (  1.16);

\path[draw=drawColor,line width= 0.4pt,line join=round,line cap=round,fill=fillColor] (224.89,372.42) circle (  1.16);

\path[draw=drawColor,line width= 0.4pt,line join=round,line cap=round,fill=fillColor] (225.01,372.42) circle (  1.16);

\path[draw=drawColor,line width= 0.4pt,line join=round,line cap=round,fill=fillColor] (225.14,372.42) circle (  1.16);

\path[draw=drawColor,line width= 0.4pt,line join=round,line cap=round,fill=fillColor] (225.26,372.42) circle (  1.16);
\definecolor[named]{drawColor}{rgb}{0.60,0.31,0.64}
\definecolor[named]{fillColor}{rgb}{0.60,0.31,0.64}

\path[draw=drawColor,line width= 0.4pt,line join=round,line cap=round,fill=fillColor] ( 74.88,442.33) circle (  1.16);

\path[draw=drawColor,line width= 0.4pt,line join=round,line cap=round,fill=fillColor] ( 80.66,442.33) circle (  1.16);

\path[draw=drawColor,line width= 0.4pt,line join=round,line cap=round,fill=fillColor] ( 84.72,440.58) circle (  1.16);

\path[draw=drawColor,line width= 0.4pt,line join=round,line cap=round,fill=fillColor] ( 87.95,440.57) circle (  1.16);

\path[draw=drawColor,line width= 0.4pt,line join=round,line cap=round,fill=fillColor] ( 90.68,439.91) circle (  1.16);

\path[draw=drawColor,line width= 0.4pt,line join=round,line cap=round,fill=fillColor] ( 93.06,439.37) circle (  1.16);

\path[draw=drawColor,line width= 0.4pt,line join=round,line cap=round,fill=fillColor] ( 95.19,439.12) circle (  1.16);

\path[draw=drawColor,line width= 0.4pt,line join=round,line cap=round,fill=fillColor] ( 97.13,436.63) circle (  1.16);

\path[draw=drawColor,line width= 0.4pt,line join=round,line cap=round,fill=fillColor] ( 98.91,435.86) circle (  1.16);

\path[draw=drawColor,line width= 0.4pt,line join=round,line cap=round,fill=fillColor] (100.57,434.97) circle (  1.16);

\path[draw=drawColor,line width= 0.4pt,line join=round,line cap=round,fill=fillColor] (102.11,433.61) circle (  1.16);

\path[draw=drawColor,line width= 0.4pt,line join=round,line cap=round,fill=fillColor] (103.57,432.83) circle (  1.16);

\path[draw=drawColor,line width= 0.4pt,line join=round,line cap=round,fill=fillColor] (104.95,432.59) circle (  1.16);

\path[draw=drawColor,line width= 0.4pt,line join=round,line cap=round,fill=fillColor] (106.26,432.52) circle (  1.16);

\path[draw=drawColor,line width= 0.4pt,line join=round,line cap=round,fill=fillColor] (107.50,432.09) circle (  1.16);

\path[draw=drawColor,line width= 0.4pt,line join=round,line cap=round,fill=fillColor] (108.70,431.67) circle (  1.16);

\path[draw=drawColor,line width= 0.4pt,line join=round,line cap=round,fill=fillColor] (109.84,431.00) circle (  1.16);

\path[draw=drawColor,line width= 0.4pt,line join=round,line cap=round,fill=fillColor] (110.94,430.82) circle (  1.16);

\path[draw=drawColor,line width= 0.4pt,line join=round,line cap=round,fill=fillColor] (112.00,430.74) circle (  1.16);

\path[draw=drawColor,line width= 0.4pt,line join=round,line cap=round,fill=fillColor] (113.03,430.66) circle (  1.16);

\path[draw=drawColor,line width= 0.4pt,line join=round,line cap=round,fill=fillColor] (114.02,430.13) circle (  1.16);

\path[draw=drawColor,line width= 0.4pt,line join=round,line cap=round,fill=fillColor] (114.98,429.29) circle (  1.16);

\path[draw=drawColor,line width= 0.4pt,line join=round,line cap=round,fill=fillColor] (115.91,429.28) circle (  1.16);

\path[draw=drawColor,line width= 0.4pt,line join=round,line cap=round,fill=fillColor] (116.81,428.94) circle (  1.16);

\path[draw=drawColor,line width= 0.4pt,line join=round,line cap=round,fill=fillColor] (117.69,428.08) circle (  1.16);

\path[draw=drawColor,line width= 0.4pt,line join=round,line cap=round,fill=fillColor] (118.55,427.97) circle (  1.16);

\path[draw=drawColor,line width= 0.4pt,line join=round,line cap=round,fill=fillColor] (119.38,427.82) circle (  1.16);

\path[draw=drawColor,line width= 0.4pt,line join=round,line cap=round,fill=fillColor] (120.20,427.30) circle (  1.16);

\path[draw=drawColor,line width= 0.4pt,line join=round,line cap=round,fill=fillColor] (120.99,426.96) circle (  1.16);

\path[draw=drawColor,line width= 0.4pt,line join=round,line cap=round,fill=fillColor] (121.77,426.22) circle (  1.16);

\path[draw=drawColor,line width= 0.4pt,line join=round,line cap=round,fill=fillColor] (122.53,426.04) circle (  1.16);

\path[draw=drawColor,line width= 0.4pt,line join=round,line cap=round,fill=fillColor] (123.27,425.95) circle (  1.16);

\path[draw=drawColor,line width= 0.4pt,line join=round,line cap=round,fill=fillColor] (124.00,425.92) circle (  1.16);

\path[draw=drawColor,line width= 0.4pt,line join=round,line cap=round,fill=fillColor] (124.71,425.78) circle (  1.16);

\path[draw=drawColor,line width= 0.4pt,line join=round,line cap=round,fill=fillColor] (125.41,425.78) circle (  1.16);

\path[draw=drawColor,line width= 0.4pt,line join=round,line cap=round,fill=fillColor] (126.10,425.21) circle (  1.16);

\path[draw=drawColor,line width= 0.4pt,line join=round,line cap=round,fill=fillColor] (126.77,425.07) circle (  1.16);

\path[draw=drawColor,line width= 0.4pt,line join=round,line cap=round,fill=fillColor] (127.44,424.48) circle (  1.16);

\path[draw=drawColor,line width= 0.4pt,line join=round,line cap=round,fill=fillColor] (128.09,424.26) circle (  1.16);

\path[draw=drawColor,line width= 0.4pt,line join=round,line cap=round,fill=fillColor] (128.73,424.10) circle (  1.16);

\path[draw=drawColor,line width= 0.4pt,line join=round,line cap=round,fill=fillColor] (129.35,424.07) circle (  1.16);

\path[draw=drawColor,line width= 0.4pt,line join=round,line cap=round,fill=fillColor] (129.97,423.46) circle (  1.16);

\path[draw=drawColor,line width= 0.4pt,line join=round,line cap=round,fill=fillColor] (130.58,423.39) circle (  1.16);

\path[draw=drawColor,line width= 0.4pt,line join=round,line cap=round,fill=fillColor] (131.18,423.33) circle (  1.16);

\path[draw=drawColor,line width= 0.4pt,line join=round,line cap=round,fill=fillColor] (131.77,423.28) circle (  1.16);

\path[draw=drawColor,line width= 0.4pt,line join=round,line cap=round,fill=fillColor] (132.35,422.95) circle (  1.16);

\path[draw=drawColor,line width= 0.4pt,line join=round,line cap=round,fill=fillColor] (132.93,422.81) circle (  1.16);

\path[draw=drawColor,line width= 0.4pt,line join=round,line cap=round,fill=fillColor] (133.49,422.60) circle (  1.16);

\path[draw=drawColor,line width= 0.4pt,line join=round,line cap=round,fill=fillColor] (134.05,422.59) circle (  1.16);

\path[draw=drawColor,line width= 0.4pt,line join=round,line cap=round,fill=fillColor] (134.60,422.58) circle (  1.16);

\path[draw=drawColor,line width= 0.4pt,line join=round,line cap=round,fill=fillColor] (135.14,422.37) circle (  1.16);

\path[draw=drawColor,line width= 0.4pt,line join=round,line cap=round,fill=fillColor] (135.68,422.09) circle (  1.16);

\path[draw=drawColor,line width= 0.4pt,line join=round,line cap=round,fill=fillColor] (136.21,421.98) circle (  1.16);

\path[draw=drawColor,line width= 0.4pt,line join=round,line cap=round,fill=fillColor] (136.73,421.95) circle (  1.16);

\path[draw=drawColor,line width= 0.4pt,line join=round,line cap=round,fill=fillColor] (137.25,421.89) circle (  1.16);

\path[draw=drawColor,line width= 0.4pt,line join=round,line cap=round,fill=fillColor] (137.76,421.81) circle (  1.16);

\path[draw=drawColor,line width= 0.4pt,line join=round,line cap=round,fill=fillColor] (138.26,421.79) circle (  1.16);

\path[draw=drawColor,line width= 0.4pt,line join=round,line cap=round,fill=fillColor] (138.76,421.72) circle (  1.16);

\path[draw=drawColor,line width= 0.4pt,line join=round,line cap=round,fill=fillColor] (139.25,421.60) circle (  1.16);

\path[draw=drawColor,line width= 0.4pt,line join=round,line cap=round,fill=fillColor] (139.74,421.56) circle (  1.16);

\path[draw=drawColor,line width= 0.4pt,line join=round,line cap=round,fill=fillColor] (140.22,421.41) circle (  1.16);

\path[draw=drawColor,line width= 0.4pt,line join=round,line cap=round,fill=fillColor] (140.70,421.29) circle (  1.16);

\path[draw=drawColor,line width= 0.4pt,line join=round,line cap=round,fill=fillColor] (141.17,421.14) circle (  1.16);

\path[draw=drawColor,line width= 0.4pt,line join=round,line cap=round,fill=fillColor] (141.63,421.04) circle (  1.16);

\path[draw=drawColor,line width= 0.4pt,line join=round,line cap=round,fill=fillColor] (142.09,420.69) circle (  1.16);

\path[draw=drawColor,line width= 0.4pt,line join=round,line cap=round,fill=fillColor] (142.55,420.67) circle (  1.16);

\path[draw=drawColor,line width= 0.4pt,line join=round,line cap=round,fill=fillColor] (143.00,420.51) circle (  1.16);

\path[draw=drawColor,line width= 0.4pt,line join=round,line cap=round,fill=fillColor] (143.45,420.22) circle (  1.16);

\path[draw=drawColor,line width= 0.4pt,line join=round,line cap=round,fill=fillColor] (143.89,420.22) circle (  1.16);

\path[draw=drawColor,line width= 0.4pt,line join=round,line cap=round,fill=fillColor] (144.33,420.14) circle (  1.16);

\path[draw=drawColor,line width= 0.4pt,line join=round,line cap=round,fill=fillColor] (144.77,420.11) circle (  1.16);

\path[draw=drawColor,line width= 0.4pt,line join=round,line cap=round,fill=fillColor] (145.20,420.10) circle (  1.16);

\path[draw=drawColor,line width= 0.4pt,line join=round,line cap=round,fill=fillColor] (145.62,419.98) circle (  1.16);

\path[draw=drawColor,line width= 0.4pt,line join=round,line cap=round,fill=fillColor] (146.05,419.93) circle (  1.16);

\path[draw=drawColor,line width= 0.4pt,line join=round,line cap=round,fill=fillColor] (146.46,419.83) circle (  1.16);

\path[draw=drawColor,line width= 0.4pt,line join=round,line cap=round,fill=fillColor] (146.88,419.67) circle (  1.16);

\path[draw=drawColor,line width= 0.4pt,line join=round,line cap=round,fill=fillColor] (147.29,419.61) circle (  1.16);

\path[draw=drawColor,line width= 0.4pt,line join=round,line cap=round,fill=fillColor] (147.70,419.56) circle (  1.16);

\path[draw=drawColor,line width= 0.4pt,line join=round,line cap=round,fill=fillColor] (148.10,419.55) circle (  1.16);

\path[draw=drawColor,line width= 0.4pt,line join=round,line cap=round,fill=fillColor] (148.50,419.42) circle (  1.16);

\path[draw=drawColor,line width= 0.4pt,line join=round,line cap=round,fill=fillColor] (148.90,419.41) circle (  1.16);

\path[draw=drawColor,line width= 0.4pt,line join=round,line cap=round,fill=fillColor] (149.30,419.24) circle (  1.16);

\path[draw=drawColor,line width= 0.4pt,line join=round,line cap=round,fill=fillColor] (149.69,419.17) circle (  1.16);

\path[draw=drawColor,line width= 0.4pt,line join=round,line cap=round,fill=fillColor] (150.08,418.98) circle (  1.16);

\path[draw=drawColor,line width= 0.4pt,line join=round,line cap=round,fill=fillColor] (150.46,418.93) circle (  1.16);

\path[draw=drawColor,line width= 0.4pt,line join=round,line cap=round,fill=fillColor] (150.84,418.73) circle (  1.16);

\path[draw=drawColor,line width= 0.4pt,line join=round,line cap=round,fill=fillColor] (151.22,418.59) circle (  1.16);

\path[draw=drawColor,line width= 0.4pt,line join=round,line cap=round,fill=fillColor] (151.60,418.58) circle (  1.16);

\path[draw=drawColor,line width= 0.4pt,line join=round,line cap=round,fill=fillColor] (151.97,418.56) circle (  1.16);

\path[draw=drawColor,line width= 0.4pt,line join=round,line cap=round,fill=fillColor] (152.34,418.56) circle (  1.16);

\path[draw=drawColor,line width= 0.4pt,line join=round,line cap=round,fill=fillColor] (152.71,418.55) circle (  1.16);

\path[draw=drawColor,line width= 0.4pt,line join=round,line cap=round,fill=fillColor] (153.08,418.45) circle (  1.16);

\path[draw=drawColor,line width= 0.4pt,line join=round,line cap=round,fill=fillColor] (153.44,418.34) circle (  1.16);

\path[draw=drawColor,line width= 0.4pt,line join=round,line cap=round,fill=fillColor] (153.80,418.31) circle (  1.16);

\path[draw=drawColor,line width= 0.4pt,line join=round,line cap=round,fill=fillColor] (154.16,418.22) circle (  1.16);

\path[draw=drawColor,line width= 0.4pt,line join=round,line cap=round,fill=fillColor] (154.51,418.21) circle (  1.16);

\path[draw=drawColor,line width= 0.4pt,line join=round,line cap=round,fill=fillColor] (154.86,418.16) circle (  1.16);

\path[draw=drawColor,line width= 0.4pt,line join=round,line cap=round,fill=fillColor] (155.22,418.15) circle (  1.16);

\path[draw=drawColor,line width= 0.4pt,line join=round,line cap=round,fill=fillColor] (155.56,418.15) circle (  1.16);

\path[draw=drawColor,line width= 0.4pt,line join=round,line cap=round,fill=fillColor] (155.91,418.11) circle (  1.16);

\path[draw=drawColor,line width= 0.4pt,line join=round,line cap=round,fill=fillColor] (156.25,418.05) circle (  1.16);

\path[draw=drawColor,line width= 0.4pt,line join=round,line cap=round,fill=fillColor] (156.59,418.04) circle (  1.16);

\path[draw=drawColor,line width= 0.4pt,line join=round,line cap=round,fill=fillColor] (156.93,417.99) circle (  1.16);

\path[draw=drawColor,line width= 0.4pt,line join=round,line cap=round,fill=fillColor] (157.27,417.80) circle (  1.16);

\path[draw=drawColor,line width= 0.4pt,line join=round,line cap=round,fill=fillColor] (157.60,417.73) circle (  1.16);

\path[draw=drawColor,line width= 0.4pt,line join=round,line cap=round,fill=fillColor] (157.93,417.67) circle (  1.16);

\path[draw=drawColor,line width= 0.4pt,line join=round,line cap=round,fill=fillColor] (158.26,417.59) circle (  1.16);

\path[draw=drawColor,line width= 0.4pt,line join=round,line cap=round,fill=fillColor] (158.59,417.56) circle (  1.16);

\path[draw=drawColor,line width= 0.4pt,line join=round,line cap=round,fill=fillColor] (158.92,417.44) circle (  1.16);

\path[draw=drawColor,line width= 0.4pt,line join=round,line cap=round,fill=fillColor] (159.24,417.31) circle (  1.16);

\path[draw=drawColor,line width= 0.4pt,line join=round,line cap=round,fill=fillColor] (159.56,417.31) circle (  1.16);

\path[draw=drawColor,line width= 0.4pt,line join=round,line cap=round,fill=fillColor] (159.88,417.26) circle (  1.16);

\path[draw=drawColor,line width= 0.4pt,line join=round,line cap=round,fill=fillColor] (160.20,417.25) circle (  1.16);

\path[draw=drawColor,line width= 0.4pt,line join=round,line cap=round,fill=fillColor] (160.52,417.04) circle (  1.16);

\path[draw=drawColor,line width= 0.4pt,line join=round,line cap=round,fill=fillColor] (160.83,416.98) circle (  1.16);

\path[draw=drawColor,line width= 0.4pt,line join=round,line cap=round,fill=fillColor] (161.15,416.95) circle (  1.16);

\path[draw=drawColor,line width= 0.4pt,line join=round,line cap=round,fill=fillColor] (161.46,416.91) circle (  1.16);

\path[draw=drawColor,line width= 0.4pt,line join=round,line cap=round,fill=fillColor] (161.77,416.85) circle (  1.16);

\path[draw=drawColor,line width= 0.4pt,line join=round,line cap=round,fill=fillColor] (162.07,416.83) circle (  1.16);

\path[draw=drawColor,line width= 0.4pt,line join=round,line cap=round,fill=fillColor] (162.38,416.82) circle (  1.16);

\path[draw=drawColor,line width= 0.4pt,line join=round,line cap=round,fill=fillColor] (162.68,416.80) circle (  1.16);

\path[draw=drawColor,line width= 0.4pt,line join=round,line cap=round,fill=fillColor] (162.99,416.73) circle (  1.16);

\path[draw=drawColor,line width= 0.4pt,line join=round,line cap=round,fill=fillColor] (163.29,416.73) circle (  1.16);

\path[draw=drawColor,line width= 0.4pt,line join=round,line cap=round,fill=fillColor] (163.59,416.69) circle (  1.16);

\path[draw=drawColor,line width= 0.4pt,line join=round,line cap=round,fill=fillColor] (163.88,416.48) circle (  1.16);

\path[draw=drawColor,line width= 0.4pt,line join=round,line cap=round,fill=fillColor] (164.18,416.36) circle (  1.16);

\path[draw=drawColor,line width= 0.4pt,line join=round,line cap=round,fill=fillColor] (164.47,416.29) circle (  1.16);

\path[draw=drawColor,line width= 0.4pt,line join=round,line cap=round,fill=fillColor] (164.77,416.29) circle (  1.16);

\path[draw=drawColor,line width= 0.4pt,line join=round,line cap=round,fill=fillColor] (165.06,416.27) circle (  1.16);

\path[draw=drawColor,line width= 0.4pt,line join=round,line cap=round,fill=fillColor] (165.35,416.23) circle (  1.16);

\path[draw=drawColor,line width= 0.4pt,line join=round,line cap=round,fill=fillColor] (165.64,416.21) circle (  1.16);

\path[draw=drawColor,line width= 0.4pt,line join=round,line cap=round,fill=fillColor] (165.92,416.20) circle (  1.16);

\path[draw=drawColor,line width= 0.4pt,line join=round,line cap=round,fill=fillColor] (166.21,416.19) circle (  1.16);

\path[draw=drawColor,line width= 0.4pt,line join=round,line cap=round,fill=fillColor] (166.49,416.10) circle (  1.16);

\path[draw=drawColor,line width= 0.4pt,line join=round,line cap=round,fill=fillColor] (166.77,416.10) circle (  1.16);

\path[draw=drawColor,line width= 0.4pt,line join=round,line cap=round,fill=fillColor] (167.06,416.08) circle (  1.16);

\path[draw=drawColor,line width= 0.4pt,line join=round,line cap=round,fill=fillColor] (167.34,416.04) circle (  1.16);

\path[draw=drawColor,line width= 0.4pt,line join=round,line cap=round,fill=fillColor] (167.61,415.97) circle (  1.16);

\path[draw=drawColor,line width= 0.4pt,line join=round,line cap=round,fill=fillColor] (167.89,415.92) circle (  1.16);

\path[draw=drawColor,line width= 0.4pt,line join=round,line cap=round,fill=fillColor] (168.17,415.88) circle (  1.16);

\path[draw=drawColor,line width= 0.4pt,line join=round,line cap=round,fill=fillColor] (168.44,415.88) circle (  1.16);

\path[draw=drawColor,line width= 0.4pt,line join=round,line cap=round,fill=fillColor] (168.71,415.83) circle (  1.16);

\path[draw=drawColor,line width= 0.4pt,line join=round,line cap=round,fill=fillColor] (168.99,415.81) circle (  1.16);

\path[draw=drawColor,line width= 0.4pt,line join=round,line cap=round,fill=fillColor] (169.26,415.81) circle (  1.16);

\path[draw=drawColor,line width= 0.4pt,line join=round,line cap=round,fill=fillColor] (169.53,415.65) circle (  1.16);

\path[draw=drawColor,line width= 0.4pt,line join=round,line cap=round,fill=fillColor] (169.79,415.63) circle (  1.16);

\path[draw=drawColor,line width= 0.4pt,line join=round,line cap=round,fill=fillColor] (170.06,415.62) circle (  1.16);

\path[draw=drawColor,line width= 0.4pt,line join=round,line cap=round,fill=fillColor] (170.33,415.60) circle (  1.16);

\path[draw=drawColor,line width= 0.4pt,line join=round,line cap=round,fill=fillColor] (170.59,415.54) circle (  1.16);

\path[draw=drawColor,line width= 0.4pt,line join=round,line cap=round,fill=fillColor] (170.85,415.51) circle (  1.16);

\path[draw=drawColor,line width= 0.4pt,line join=round,line cap=round,fill=fillColor] (171.12,415.47) circle (  1.16);

\path[draw=drawColor,line width= 0.4pt,line join=round,line cap=round,fill=fillColor] (171.38,415.39) circle (  1.16);

\path[draw=drawColor,line width= 0.4pt,line join=round,line cap=round,fill=fillColor] (171.64,415.37) circle (  1.16);

\path[draw=drawColor,line width= 0.4pt,line join=round,line cap=round,fill=fillColor] (171.90,415.32) circle (  1.16);

\path[draw=drawColor,line width= 0.4pt,line join=round,line cap=round,fill=fillColor] (172.15,415.25) circle (  1.16);

\path[draw=drawColor,line width= 0.4pt,line join=round,line cap=round,fill=fillColor] (172.41,415.23) circle (  1.16);

\path[draw=drawColor,line width= 0.4pt,line join=round,line cap=round,fill=fillColor] (172.67,415.21) circle (  1.16);

\path[draw=drawColor,line width= 0.4pt,line join=round,line cap=round,fill=fillColor] (172.92,415.15) circle (  1.16);

\path[draw=drawColor,line width= 0.4pt,line join=round,line cap=round,fill=fillColor] (173.17,415.14) circle (  1.16);

\path[draw=drawColor,line width= 0.4pt,line join=round,line cap=round,fill=fillColor] (173.43,415.12) circle (  1.16);

\path[draw=drawColor,line width= 0.4pt,line join=round,line cap=round,fill=fillColor] (173.68,415.11) circle (  1.16);

\path[draw=drawColor,line width= 0.4pt,line join=round,line cap=round,fill=fillColor] (173.93,415.06) circle (  1.16);

\path[draw=drawColor,line width= 0.4pt,line join=round,line cap=round,fill=fillColor] (174.18,415.01) circle (  1.16);

\path[draw=drawColor,line width= 0.4pt,line join=round,line cap=round,fill=fillColor] (174.42,415.01) circle (  1.16);

\path[draw=drawColor,line width= 0.4pt,line join=round,line cap=round,fill=fillColor] (174.67,414.96) circle (  1.16);

\path[draw=drawColor,line width= 0.4pt,line join=round,line cap=round,fill=fillColor] (174.92,414.95) circle (  1.16);

\path[draw=drawColor,line width= 0.4pt,line join=round,line cap=round,fill=fillColor] (175.16,414.62) circle (  1.16);

\path[draw=drawColor,line width= 0.4pt,line join=round,line cap=round,fill=fillColor] (175.41,414.61) circle (  1.16);

\path[draw=drawColor,line width= 0.4pt,line join=round,line cap=round,fill=fillColor] (175.65,414.61) circle (  1.16);

\path[draw=drawColor,line width= 0.4pt,line join=round,line cap=round,fill=fillColor] (175.89,414.60) circle (  1.16);

\path[draw=drawColor,line width= 0.4pt,line join=round,line cap=round,fill=fillColor] (176.13,414.43) circle (  1.16);

\path[draw=drawColor,line width= 0.4pt,line join=round,line cap=round,fill=fillColor] (176.37,414.43) circle (  1.16);

\path[draw=drawColor,line width= 0.4pt,line join=round,line cap=round,fill=fillColor] (176.61,414.42) circle (  1.16);

\path[draw=drawColor,line width= 0.4pt,line join=round,line cap=round,fill=fillColor] (176.85,414.39) circle (  1.16);

\path[draw=drawColor,line width= 0.4pt,line join=round,line cap=round,fill=fillColor] (177.09,414.31) circle (  1.16);

\path[draw=drawColor,line width= 0.4pt,line join=round,line cap=round,fill=fillColor] (177.32,414.30) circle (  1.16);

\path[draw=drawColor,line width= 0.4pt,line join=round,line cap=round,fill=fillColor] (177.56,414.27) circle (  1.16);

\path[draw=drawColor,line width= 0.4pt,line join=round,line cap=round,fill=fillColor] (177.80,414.24) circle (  1.16);

\path[draw=drawColor,line width= 0.4pt,line join=round,line cap=round,fill=fillColor] (178.03,414.13) circle (  1.16);

\path[draw=drawColor,line width= 0.4pt,line join=round,line cap=round,fill=fillColor] (178.26,414.00) circle (  1.16);

\path[draw=drawColor,line width= 0.4pt,line join=round,line cap=round,fill=fillColor] (178.49,413.91) circle (  1.16);

\path[draw=drawColor,line width= 0.4pt,line join=round,line cap=round,fill=fillColor] (178.73,413.84) circle (  1.16);

\path[draw=drawColor,line width= 0.4pt,line join=round,line cap=round,fill=fillColor] (178.96,413.83) circle (  1.16);

\path[draw=drawColor,line width= 0.4pt,line join=round,line cap=round,fill=fillColor] (179.19,413.73) circle (  1.16);

\path[draw=drawColor,line width= 0.4pt,line join=round,line cap=round,fill=fillColor] (179.41,413.71) circle (  1.16);

\path[draw=drawColor,line width= 0.4pt,line join=round,line cap=round,fill=fillColor] (179.64,413.69) circle (  1.16);

\path[draw=drawColor,line width= 0.4pt,line join=round,line cap=round,fill=fillColor] (179.87,413.66) circle (  1.16);

\path[draw=drawColor,line width= 0.4pt,line join=round,line cap=round,fill=fillColor] (180.10,413.60) circle (  1.16);

\path[draw=drawColor,line width= 0.4pt,line join=round,line cap=round,fill=fillColor] (180.32,413.48) circle (  1.16);

\path[draw=drawColor,line width= 0.4pt,line join=round,line cap=round,fill=fillColor] (180.55,413.45) circle (  1.16);

\path[draw=drawColor,line width= 0.4pt,line join=round,line cap=round,fill=fillColor] (180.77,413.45) circle (  1.16);

\path[draw=drawColor,line width= 0.4pt,line join=round,line cap=round,fill=fillColor] (180.99,413.45) circle (  1.16);

\path[draw=drawColor,line width= 0.4pt,line join=round,line cap=round,fill=fillColor] (181.22,413.35) circle (  1.16);

\path[draw=drawColor,line width= 0.4pt,line join=round,line cap=round,fill=fillColor] (181.44,413.30) circle (  1.16);

\path[draw=drawColor,line width= 0.4pt,line join=round,line cap=round,fill=fillColor] (181.66,413.12) circle (  1.16);

\path[draw=drawColor,line width= 0.4pt,line join=round,line cap=round,fill=fillColor] (181.88,413.09) circle (  1.16);

\path[draw=drawColor,line width= 0.4pt,line join=round,line cap=round,fill=fillColor] (182.10,413.08) circle (  1.16);

\path[draw=drawColor,line width= 0.4pt,line join=round,line cap=round,fill=fillColor] (182.32,412.98) circle (  1.16);

\path[draw=drawColor,line width= 0.4pt,line join=round,line cap=round,fill=fillColor] (182.54,412.97) circle (  1.16);

\path[draw=drawColor,line width= 0.4pt,line join=round,line cap=round,fill=fillColor] (182.75,412.90) circle (  1.16);

\path[draw=drawColor,line width= 0.4pt,line join=round,line cap=round,fill=fillColor] (182.97,412.89) circle (  1.16);

\path[draw=drawColor,line width= 0.4pt,line join=round,line cap=round,fill=fillColor] (183.19,412.87) circle (  1.16);

\path[draw=drawColor,line width= 0.4pt,line join=round,line cap=round,fill=fillColor] (183.40,412.80) circle (  1.16);

\path[draw=drawColor,line width= 0.4pt,line join=round,line cap=round,fill=fillColor] (183.61,412.71) circle (  1.16);

\path[draw=drawColor,line width= 0.4pt,line join=round,line cap=round,fill=fillColor] (183.83,412.63) circle (  1.16);

\path[draw=drawColor,line width= 0.4pt,line join=round,line cap=round,fill=fillColor] (184.04,412.61) circle (  1.16);

\path[draw=drawColor,line width= 0.4pt,line join=round,line cap=round,fill=fillColor] (184.25,412.56) circle (  1.16);

\path[draw=drawColor,line width= 0.4pt,line join=round,line cap=round,fill=fillColor] (184.47,412.52) circle (  1.16);

\path[draw=drawColor,line width= 0.4pt,line join=round,line cap=round,fill=fillColor] (184.68,412.45) circle (  1.16);

\path[draw=drawColor,line width= 0.4pt,line join=round,line cap=round,fill=fillColor] (184.89,412.41) circle (  1.16);

\path[draw=drawColor,line width= 0.4pt,line join=round,line cap=round,fill=fillColor] (185.10,412.39) circle (  1.16);

\path[draw=drawColor,line width= 0.4pt,line join=round,line cap=round,fill=fillColor] (185.31,412.29) circle (  1.16);

\path[draw=drawColor,line width= 0.4pt,line join=round,line cap=round,fill=fillColor] (185.51,412.25) circle (  1.16);

\path[draw=drawColor,line width= 0.4pt,line join=round,line cap=round,fill=fillColor] (185.72,412.25) circle (  1.16);

\path[draw=drawColor,line width= 0.4pt,line join=round,line cap=round,fill=fillColor] (185.93,412.11) circle (  1.16);

\path[draw=drawColor,line width= 0.4pt,line join=round,line cap=round,fill=fillColor] (186.13,412.06) circle (  1.16);

\path[draw=drawColor,line width= 0.4pt,line join=round,line cap=round,fill=fillColor] (186.34,412.02) circle (  1.16);

\path[draw=drawColor,line width= 0.4pt,line join=round,line cap=round,fill=fillColor] (186.55,411.99) circle (  1.16);

\path[draw=drawColor,line width= 0.4pt,line join=round,line cap=round,fill=fillColor] (186.75,411.89) circle (  1.16);

\path[draw=drawColor,line width= 0.4pt,line join=round,line cap=round,fill=fillColor] (186.95,411.85) circle (  1.16);

\path[draw=drawColor,line width= 0.4pt,line join=round,line cap=round,fill=fillColor] (187.16,411.83) circle (  1.16);

\path[draw=drawColor,line width= 0.4pt,line join=round,line cap=round,fill=fillColor] (187.36,411.80) circle (  1.16);

\path[draw=drawColor,line width= 0.4pt,line join=round,line cap=round,fill=fillColor] (187.56,411.76) circle (  1.16);

\path[draw=drawColor,line width= 0.4pt,line join=round,line cap=round,fill=fillColor] (187.76,411.75) circle (  1.16);

\path[draw=drawColor,line width= 0.4pt,line join=round,line cap=round,fill=fillColor] (187.96,411.67) circle (  1.16);

\path[draw=drawColor,line width= 0.4pt,line join=round,line cap=round,fill=fillColor] (188.16,411.60) circle (  1.16);

\path[draw=drawColor,line width= 0.4pt,line join=round,line cap=round,fill=fillColor] (188.36,411.59) circle (  1.16);

\path[draw=drawColor,line width= 0.4pt,line join=round,line cap=round,fill=fillColor] (188.56,411.51) circle (  1.16);

\path[draw=drawColor,line width= 0.4pt,line join=round,line cap=round,fill=fillColor] (188.76,411.42) circle (  1.16);

\path[draw=drawColor,line width= 0.4pt,line join=round,line cap=round,fill=fillColor] (188.96,411.28) circle (  1.16);

\path[draw=drawColor,line width= 0.4pt,line join=round,line cap=round,fill=fillColor] (189.16,411.24) circle (  1.16);

\path[draw=drawColor,line width= 0.4pt,line join=round,line cap=round,fill=fillColor] (189.35,411.21) circle (  1.16);

\path[draw=drawColor,line width= 0.4pt,line join=round,line cap=round,fill=fillColor] (189.55,411.19) circle (  1.16);

\path[draw=drawColor,line width= 0.4pt,line join=round,line cap=round,fill=fillColor] (189.74,411.13) circle (  1.16);

\path[draw=drawColor,line width= 0.4pt,line join=round,line cap=round,fill=fillColor] (189.94,411.11) circle (  1.16);

\path[draw=drawColor,line width= 0.4pt,line join=round,line cap=round,fill=fillColor] (190.13,411.08) circle (  1.16);

\path[draw=drawColor,line width= 0.4pt,line join=round,line cap=round,fill=fillColor] (190.33,411.07) circle (  1.16);

\path[draw=drawColor,line width= 0.4pt,line join=round,line cap=round,fill=fillColor] (190.52,410.96) circle (  1.16);

\path[draw=drawColor,line width= 0.4pt,line join=round,line cap=round,fill=fillColor] (190.71,410.95) circle (  1.16);

\path[draw=drawColor,line width= 0.4pt,line join=round,line cap=round,fill=fillColor] (190.91,410.92) circle (  1.16);

\path[draw=drawColor,line width= 0.4pt,line join=round,line cap=round,fill=fillColor] (191.10,410.73) circle (  1.16);

\path[draw=drawColor,line width= 0.4pt,line join=round,line cap=round,fill=fillColor] (191.29,410.71) circle (  1.16);

\path[draw=drawColor,line width= 0.4pt,line join=round,line cap=round,fill=fillColor] (191.48,410.61) circle (  1.16);

\path[draw=drawColor,line width= 0.4pt,line join=round,line cap=round,fill=fillColor] (191.67,410.52) circle (  1.16);

\path[draw=drawColor,line width= 0.4pt,line join=round,line cap=round,fill=fillColor] (191.86,410.51) circle (  1.16);

\path[draw=drawColor,line width= 0.4pt,line join=round,line cap=round,fill=fillColor] (192.05,410.49) circle (  1.16);

\path[draw=drawColor,line width= 0.4pt,line join=round,line cap=round,fill=fillColor] (192.24,410.49) circle (  1.16);

\path[draw=drawColor,line width= 0.4pt,line join=round,line cap=round,fill=fillColor] (192.43,410.48) circle (  1.16);

\path[draw=drawColor,line width= 0.4pt,line join=round,line cap=round,fill=fillColor] (192.61,410.46) circle (  1.16);

\path[draw=drawColor,line width= 0.4pt,line join=round,line cap=round,fill=fillColor] (192.80,410.40) circle (  1.16);

\path[draw=drawColor,line width= 0.4pt,line join=round,line cap=round,fill=fillColor] (192.99,410.24) circle (  1.16);

\path[draw=drawColor,line width= 0.4pt,line join=round,line cap=round,fill=fillColor] (193.17,410.21) circle (  1.16);

\path[draw=drawColor,line width= 0.4pt,line join=round,line cap=round,fill=fillColor] (193.36,410.17) circle (  1.16);

\path[draw=drawColor,line width= 0.4pt,line join=round,line cap=round,fill=fillColor] (193.54,410.16) circle (  1.16);

\path[draw=drawColor,line width= 0.4pt,line join=round,line cap=round,fill=fillColor] (193.73,410.13) circle (  1.16);

\path[draw=drawColor,line width= 0.4pt,line join=round,line cap=round,fill=fillColor] (193.91,410.11) circle (  1.16);

\path[draw=drawColor,line width= 0.4pt,line join=round,line cap=round,fill=fillColor] (194.10,410.00) circle (  1.16);

\path[draw=drawColor,line width= 0.4pt,line join=round,line cap=round,fill=fillColor] (194.28,409.96) circle (  1.16);

\path[draw=drawColor,line width= 0.4pt,line join=round,line cap=round,fill=fillColor] (194.46,409.82) circle (  1.16);

\path[draw=drawColor,line width= 0.4pt,line join=round,line cap=round,fill=fillColor] (194.65,409.77) circle (  1.16);

\path[draw=drawColor,line width= 0.4pt,line join=round,line cap=round,fill=fillColor] (194.83,409.77) circle (  1.16);

\path[draw=drawColor,line width= 0.4pt,line join=round,line cap=round,fill=fillColor] (195.01,409.68) circle (  1.16);

\path[draw=drawColor,line width= 0.4pt,line join=round,line cap=round,fill=fillColor] (195.19,409.59) circle (  1.16);

\path[draw=drawColor,line width= 0.4pt,line join=round,line cap=round,fill=fillColor] (195.37,409.54) circle (  1.16);

\path[draw=drawColor,line width= 0.4pt,line join=round,line cap=round,fill=fillColor] (195.55,409.40) circle (  1.16);

\path[draw=drawColor,line width= 0.4pt,line join=round,line cap=round,fill=fillColor] (195.73,409.36) circle (  1.16);

\path[draw=drawColor,line width= 0.4pt,line join=round,line cap=round,fill=fillColor] (195.91,409.35) circle (  1.16);

\path[draw=drawColor,line width= 0.4pt,line join=round,line cap=round,fill=fillColor] (196.09,409.33) circle (  1.16);

\path[draw=drawColor,line width= 0.4pt,line join=round,line cap=round,fill=fillColor] (196.27,409.27) circle (  1.16);

\path[draw=drawColor,line width= 0.4pt,line join=round,line cap=round,fill=fillColor] (196.44,409.24) circle (  1.16);

\path[draw=drawColor,line width= 0.4pt,line join=round,line cap=round,fill=fillColor] (196.62,409.22) circle (  1.16);

\path[draw=drawColor,line width= 0.4pt,line join=round,line cap=round,fill=fillColor] (196.80,409.21) circle (  1.16);

\path[draw=drawColor,line width= 0.4pt,line join=round,line cap=round,fill=fillColor] (196.97,409.19) circle (  1.16);

\path[draw=drawColor,line width= 0.4pt,line join=round,line cap=round,fill=fillColor] (197.15,409.16) circle (  1.16);

\path[draw=drawColor,line width= 0.4pt,line join=round,line cap=round,fill=fillColor] (197.33,409.15) circle (  1.16);

\path[draw=drawColor,line width= 0.4pt,line join=round,line cap=round,fill=fillColor] (197.50,409.12) circle (  1.16);

\path[draw=drawColor,line width= 0.4pt,line join=round,line cap=round,fill=fillColor] (197.68,409.04) circle (  1.16);

\path[draw=drawColor,line width= 0.4pt,line join=round,line cap=round,fill=fillColor] (197.85,409.04) circle (  1.16);

\path[draw=drawColor,line width= 0.4pt,line join=round,line cap=round,fill=fillColor] (198.02,409.02) circle (  1.16);

\path[draw=drawColor,line width= 0.4pt,line join=round,line cap=round,fill=fillColor] (198.20,409.00) circle (  1.16);

\path[draw=drawColor,line width= 0.4pt,line join=round,line cap=round,fill=fillColor] (198.37,408.99) circle (  1.16);

\path[draw=drawColor,line width= 0.4pt,line join=round,line cap=round,fill=fillColor] (198.54,408.98) circle (  1.16);

\path[draw=drawColor,line width= 0.4pt,line join=round,line cap=round,fill=fillColor] (198.72,408.96) circle (  1.16);

\path[draw=drawColor,line width= 0.4pt,line join=round,line cap=round,fill=fillColor] (198.89,408.95) circle (  1.16);

\path[draw=drawColor,line width= 0.4pt,line join=round,line cap=round,fill=fillColor] (199.06,408.71) circle (  1.16);

\path[draw=drawColor,line width= 0.4pt,line join=round,line cap=round,fill=fillColor] (199.23,408.60) circle (  1.16);

\path[draw=drawColor,line width= 0.4pt,line join=round,line cap=round,fill=fillColor] (199.40,408.46) circle (  1.16);

\path[draw=drawColor,line width= 0.4pt,line join=round,line cap=round,fill=fillColor] (199.57,408.41) circle (  1.16);

\path[draw=drawColor,line width= 0.4pt,line join=round,line cap=round,fill=fillColor] (199.74,408.40) circle (  1.16);

\path[draw=drawColor,line width= 0.4pt,line join=round,line cap=round,fill=fillColor] (199.91,408.34) circle (  1.16);

\path[draw=drawColor,line width= 0.4pt,line join=round,line cap=round,fill=fillColor] (200.08,408.26) circle (  1.16);

\path[draw=drawColor,line width= 0.4pt,line join=round,line cap=round,fill=fillColor] (200.25,408.11) circle (  1.16);

\path[draw=drawColor,line width= 0.4pt,line join=round,line cap=round,fill=fillColor] (200.42,408.09) circle (  1.16);

\path[draw=drawColor,line width= 0.4pt,line join=round,line cap=round,fill=fillColor] (200.58,408.05) circle (  1.16);

\path[draw=drawColor,line width= 0.4pt,line join=round,line cap=round,fill=fillColor] (200.75,408.05) circle (  1.16);

\path[draw=drawColor,line width= 0.4pt,line join=round,line cap=round,fill=fillColor] (200.92,407.86) circle (  1.16);

\path[draw=drawColor,line width= 0.4pt,line join=round,line cap=round,fill=fillColor] (201.09,407.69) circle (  1.16);

\path[draw=drawColor,line width= 0.4pt,line join=round,line cap=round,fill=fillColor] (201.25,407.66) circle (  1.16);

\path[draw=drawColor,line width= 0.4pt,line join=round,line cap=round,fill=fillColor] (201.42,407.52) circle (  1.16);

\path[draw=drawColor,line width= 0.4pt,line join=round,line cap=round,fill=fillColor] (201.58,407.48) circle (  1.16);

\path[draw=drawColor,line width= 0.4pt,line join=round,line cap=round,fill=fillColor] (201.75,407.47) circle (  1.16);

\path[draw=drawColor,line width= 0.4pt,line join=round,line cap=round,fill=fillColor] (201.91,407.44) circle (  1.16);

\path[draw=drawColor,line width= 0.4pt,line join=round,line cap=round,fill=fillColor] (202.08,407.40) circle (  1.16);

\path[draw=drawColor,line width= 0.4pt,line join=round,line cap=round,fill=fillColor] (202.24,407.30) circle (  1.16);

\path[draw=drawColor,line width= 0.4pt,line join=round,line cap=round,fill=fillColor] (202.41,407.24) circle (  1.16);

\path[draw=drawColor,line width= 0.4pt,line join=round,line cap=round,fill=fillColor] (202.57,407.21) circle (  1.16);

\path[draw=drawColor,line width= 0.4pt,line join=round,line cap=round,fill=fillColor] (202.73,407.15) circle (  1.16);

\path[draw=drawColor,line width= 0.4pt,line join=round,line cap=round,fill=fillColor] (202.90,407.09) circle (  1.16);

\path[draw=drawColor,line width= 0.4pt,line join=round,line cap=round,fill=fillColor] (203.06,407.05) circle (  1.16);

\path[draw=drawColor,line width= 0.4pt,line join=round,line cap=round,fill=fillColor] (203.22,406.91) circle (  1.16);

\path[draw=drawColor,line width= 0.4pt,line join=round,line cap=round,fill=fillColor] (203.38,406.89) circle (  1.16);

\path[draw=drawColor,line width= 0.4pt,line join=round,line cap=round,fill=fillColor] (203.54,406.88) circle (  1.16);

\path[draw=drawColor,line width= 0.4pt,line join=round,line cap=round,fill=fillColor] (203.71,406.86) circle (  1.16);

\path[draw=drawColor,line width= 0.4pt,line join=round,line cap=round,fill=fillColor] (203.87,406.85) circle (  1.16);

\path[draw=drawColor,line width= 0.4pt,line join=round,line cap=round,fill=fillColor] (204.03,406.71) circle (  1.16);

\path[draw=drawColor,line width= 0.4pt,line join=round,line cap=round,fill=fillColor] (204.19,406.60) circle (  1.16);

\path[draw=drawColor,line width= 0.4pt,line join=round,line cap=round,fill=fillColor] (204.35,406.49) circle (  1.16);

\path[draw=drawColor,line width= 0.4pt,line join=round,line cap=round,fill=fillColor] (204.51,406.44) circle (  1.16);

\path[draw=drawColor,line width= 0.4pt,line join=round,line cap=round,fill=fillColor] (204.66,406.40) circle (  1.16);

\path[draw=drawColor,line width= 0.4pt,line join=round,line cap=round,fill=fillColor] (204.82,406.35) circle (  1.16);

\path[draw=drawColor,line width= 0.4pt,line join=round,line cap=round,fill=fillColor] (204.98,406.34) circle (  1.16);

\path[draw=drawColor,line width= 0.4pt,line join=round,line cap=round,fill=fillColor] (205.14,406.30) circle (  1.16);

\path[draw=drawColor,line width= 0.4pt,line join=round,line cap=round,fill=fillColor] (205.30,406.26) circle (  1.16);

\path[draw=drawColor,line width= 0.4pt,line join=round,line cap=round,fill=fillColor] (205.45,406.15) circle (  1.16);

\path[draw=drawColor,line width= 0.4pt,line join=round,line cap=round,fill=fillColor] (205.61,406.09) circle (  1.16);

\path[draw=drawColor,line width= 0.4pt,line join=round,line cap=round,fill=fillColor] (205.77,405.97) circle (  1.16);

\path[draw=drawColor,line width= 0.4pt,line join=round,line cap=round,fill=fillColor] (205.92,405.95) circle (  1.16);

\path[draw=drawColor,line width= 0.4pt,line join=round,line cap=round,fill=fillColor] (206.08,405.93) circle (  1.16);

\path[draw=drawColor,line width= 0.4pt,line join=round,line cap=round,fill=fillColor] (206.24,405.91) circle (  1.16);

\path[draw=drawColor,line width= 0.4pt,line join=round,line cap=round,fill=fillColor] (206.39,405.79) circle (  1.16);

\path[draw=drawColor,line width= 0.4pt,line join=round,line cap=round,fill=fillColor] (206.55,405.77) circle (  1.16);

\path[draw=drawColor,line width= 0.4pt,line join=round,line cap=round,fill=fillColor] (206.70,405.64) circle (  1.16);

\path[draw=drawColor,line width= 0.4pt,line join=round,line cap=round,fill=fillColor] (206.86,405.62) circle (  1.16);

\path[draw=drawColor,line width= 0.4pt,line join=round,line cap=round,fill=fillColor] (207.01,405.60) circle (  1.16);

\path[draw=drawColor,line width= 0.4pt,line join=round,line cap=round,fill=fillColor] (207.16,405.56) circle (  1.16);

\path[draw=drawColor,line width= 0.4pt,line join=round,line cap=round,fill=fillColor] (207.32,405.45) circle (  1.16);

\path[draw=drawColor,line width= 0.4pt,line join=round,line cap=round,fill=fillColor] (207.47,405.38) circle (  1.16);

\path[draw=drawColor,line width= 0.4pt,line join=round,line cap=round,fill=fillColor] (207.62,405.29) circle (  1.16);

\path[draw=drawColor,line width= 0.4pt,line join=round,line cap=round,fill=fillColor] (207.78,405.27) circle (  1.16);

\path[draw=drawColor,line width= 0.4pt,line join=round,line cap=round,fill=fillColor] (207.93,405.26) circle (  1.16);

\path[draw=drawColor,line width= 0.4pt,line join=round,line cap=round,fill=fillColor] (208.08,405.23) circle (  1.16);

\path[draw=drawColor,line width= 0.4pt,line join=round,line cap=round,fill=fillColor] (208.23,405.21) circle (  1.16);

\path[draw=drawColor,line width= 0.4pt,line join=round,line cap=round,fill=fillColor] (208.39,405.13) circle (  1.16);

\path[draw=drawColor,line width= 0.4pt,line join=round,line cap=round,fill=fillColor] (208.54,405.10) circle (  1.16);

\path[draw=drawColor,line width= 0.4pt,line join=round,line cap=round,fill=fillColor] (208.69,405.04) circle (  1.16);

\path[draw=drawColor,line width= 0.4pt,line join=round,line cap=round,fill=fillColor] (208.84,404.98) circle (  1.16);

\path[draw=drawColor,line width= 0.4pt,line join=round,line cap=round,fill=fillColor] (208.99,404.97) circle (  1.16);

\path[draw=drawColor,line width= 0.4pt,line join=round,line cap=round,fill=fillColor] (209.14,404.97) circle (  1.16);

\path[draw=drawColor,line width= 0.4pt,line join=round,line cap=round,fill=fillColor] (209.29,404.88) circle (  1.16);

\path[draw=drawColor,line width= 0.4pt,line join=round,line cap=round,fill=fillColor] (209.44,404.74) circle (  1.16);

\path[draw=drawColor,line width= 0.4pt,line join=round,line cap=round,fill=fillColor] (209.59,404.70) circle (  1.16);

\path[draw=drawColor,line width= 0.4pt,line join=round,line cap=round,fill=fillColor] (209.74,404.65) circle (  1.16);

\path[draw=drawColor,line width= 0.4pt,line join=round,line cap=round,fill=fillColor] (209.88,404.62) circle (  1.16);

\path[draw=drawColor,line width= 0.4pt,line join=round,line cap=round,fill=fillColor] (210.03,404.59) circle (  1.16);

\path[draw=drawColor,line width= 0.4pt,line join=round,line cap=round,fill=fillColor] (210.18,404.57) circle (  1.16);

\path[draw=drawColor,line width= 0.4pt,line join=round,line cap=round,fill=fillColor] (210.33,404.49) circle (  1.16);

\path[draw=drawColor,line width= 0.4pt,line join=round,line cap=round,fill=fillColor] (210.48,404.47) circle (  1.16);

\path[draw=drawColor,line width= 0.4pt,line join=round,line cap=round,fill=fillColor] (210.62,404.33) circle (  1.16);

\path[draw=drawColor,line width= 0.4pt,line join=round,line cap=round,fill=fillColor] (210.77,404.33) circle (  1.16);

\path[draw=drawColor,line width= 0.4pt,line join=round,line cap=round,fill=fillColor] (210.92,404.14) circle (  1.16);

\path[draw=drawColor,line width= 0.4pt,line join=round,line cap=round,fill=fillColor] (211.06,404.09) circle (  1.16);

\path[draw=drawColor,line width= 0.4pt,line join=round,line cap=round,fill=fillColor] (211.21,404.00) circle (  1.16);

\path[draw=drawColor,line width= 0.4pt,line join=round,line cap=round,fill=fillColor] (211.36,403.93) circle (  1.16);

\path[draw=drawColor,line width= 0.4pt,line join=round,line cap=round,fill=fillColor] (211.50,403.87) circle (  1.16);

\path[draw=drawColor,line width= 0.4pt,line join=round,line cap=round,fill=fillColor] (211.65,403.77) circle (  1.16);

\path[draw=drawColor,line width= 0.4pt,line join=round,line cap=round,fill=fillColor] (211.79,403.76) circle (  1.16);

\path[draw=drawColor,line width= 0.4pt,line join=round,line cap=round,fill=fillColor] (211.94,403.75) circle (  1.16);

\path[draw=drawColor,line width= 0.4pt,line join=round,line cap=round,fill=fillColor] (212.08,403.54) circle (  1.16);

\path[draw=drawColor,line width= 0.4pt,line join=round,line cap=round,fill=fillColor] (212.23,403.40) circle (  1.16);

\path[draw=drawColor,line width= 0.4pt,line join=round,line cap=round,fill=fillColor] (212.37,403.36) circle (  1.16);

\path[draw=drawColor,line width= 0.4pt,line join=round,line cap=round,fill=fillColor] (212.51,403.31) circle (  1.16);

\path[draw=drawColor,line width= 0.4pt,line join=round,line cap=round,fill=fillColor] (212.66,403.20) circle (  1.16);

\path[draw=drawColor,line width= 0.4pt,line join=round,line cap=round,fill=fillColor] (212.80,402.70) circle (  1.16);

\path[draw=drawColor,line width= 0.4pt,line join=round,line cap=round,fill=fillColor] (212.94,402.56) circle (  1.16);

\path[draw=drawColor,line width= 0.4pt,line join=round,line cap=round,fill=fillColor] (213.09,402.51) circle (  1.16);

\path[draw=drawColor,line width= 0.4pt,line join=round,line cap=round,fill=fillColor] (213.23,402.46) circle (  1.16);

\path[draw=drawColor,line width= 0.4pt,line join=round,line cap=round,fill=fillColor] (213.37,402.38) circle (  1.16);

\path[draw=drawColor,line width= 0.4pt,line join=round,line cap=round,fill=fillColor] (213.51,402.28) circle (  1.16);

\path[draw=drawColor,line width= 0.4pt,line join=round,line cap=round,fill=fillColor] (213.65,402.00) circle (  1.16);

\path[draw=drawColor,line width= 0.4pt,line join=round,line cap=round,fill=fillColor] (213.80,401.52) circle (  1.16);

\path[draw=drawColor,line width= 0.4pt,line join=round,line cap=round,fill=fillColor] (213.94,401.44) circle (  1.16);

\path[draw=drawColor,line width= 0.4pt,line join=round,line cap=round,fill=fillColor] (214.08,400.97) circle (  1.16);

\path[draw=drawColor,line width= 0.4pt,line join=round,line cap=round,fill=fillColor] (214.22,400.89) circle (  1.16);

\path[draw=drawColor,line width= 0.4pt,line join=round,line cap=round,fill=fillColor] (214.36,400.79) circle (  1.16);

\path[draw=drawColor,line width= 0.4pt,line join=round,line cap=round,fill=fillColor] (214.50,400.77) circle (  1.16);

\path[draw=drawColor,line width= 0.4pt,line join=round,line cap=round,fill=fillColor] (214.64,400.58) circle (  1.16);

\path[draw=drawColor,line width= 0.4pt,line join=round,line cap=round,fill=fillColor] (214.78,400.53) circle (  1.16);

\path[draw=drawColor,line width= 0.4pt,line join=round,line cap=round,fill=fillColor] (214.92,400.45) circle (  1.16);

\path[draw=drawColor,line width= 0.4pt,line join=round,line cap=round,fill=fillColor] (215.06,400.39) circle (  1.16);

\path[draw=drawColor,line width= 0.4pt,line join=round,line cap=round,fill=fillColor] (215.20,400.22) circle (  1.16);

\path[draw=drawColor,line width= 0.4pt,line join=round,line cap=round,fill=fillColor] (215.34,400.10) circle (  1.16);

\path[draw=drawColor,line width= 0.4pt,line join=round,line cap=round,fill=fillColor] (215.47,400.00) circle (  1.16);

\path[draw=drawColor,line width= 0.4pt,line join=round,line cap=round,fill=fillColor] (215.61,399.78) circle (  1.16);

\path[draw=drawColor,line width= 0.4pt,line join=round,line cap=round,fill=fillColor] (215.75,399.58) circle (  1.16);

\path[draw=drawColor,line width= 0.4pt,line join=round,line cap=round,fill=fillColor] (215.89,399.44) circle (  1.16);

\path[draw=drawColor,line width= 0.4pt,line join=round,line cap=round,fill=fillColor] (216.03,399.35) circle (  1.16);

\path[draw=drawColor,line width= 0.4pt,line join=round,line cap=round,fill=fillColor] (216.16,399.33) circle (  1.16);

\path[draw=drawColor,line width= 0.4pt,line join=round,line cap=round,fill=fillColor] (216.30,399.10) circle (  1.16);

\path[draw=drawColor,line width= 0.4pt,line join=round,line cap=round,fill=fillColor] (216.44,399.07) circle (  1.16);

\path[draw=drawColor,line width= 0.4pt,line join=round,line cap=round,fill=fillColor] (216.58,399.01) circle (  1.16);

\path[draw=drawColor,line width= 0.4pt,line join=round,line cap=round,fill=fillColor] (216.71,398.73) circle (  1.16);

\path[draw=drawColor,line width= 0.4pt,line join=round,line cap=round,fill=fillColor] (216.85,398.56) circle (  1.16);

\path[draw=drawColor,line width= 0.4pt,line join=round,line cap=round,fill=fillColor] (216.98,398.15) circle (  1.16);

\path[draw=drawColor,line width= 0.4pt,line join=round,line cap=round,fill=fillColor] (217.12,398.09) circle (  1.16);

\path[draw=drawColor,line width= 0.4pt,line join=round,line cap=round,fill=fillColor] (217.26,398.06) circle (  1.16);

\path[draw=drawColor,line width= 0.4pt,line join=round,line cap=round,fill=fillColor] (217.39,397.84) circle (  1.16);

\path[draw=drawColor,line width= 0.4pt,line join=round,line cap=round,fill=fillColor] (217.53,397.75) circle (  1.16);

\path[draw=drawColor,line width= 0.4pt,line join=round,line cap=round,fill=fillColor] (217.66,397.71) circle (  1.16);

\path[draw=drawColor,line width= 0.4pt,line join=round,line cap=round,fill=fillColor] (217.80,397.28) circle (  1.16);

\path[draw=drawColor,line width= 0.4pt,line join=round,line cap=round,fill=fillColor] (217.93,397.17) circle (  1.16);

\path[draw=drawColor,line width= 0.4pt,line join=round,line cap=round,fill=fillColor] (218.06,397.13) circle (  1.16);

\path[draw=drawColor,line width= 0.4pt,line join=round,line cap=round,fill=fillColor] (218.20,396.88) circle (  1.16);

\path[draw=drawColor,line width= 0.4pt,line join=round,line cap=round,fill=fillColor] (218.33,396.80) circle (  1.16);

\path[draw=drawColor,line width= 0.4pt,line join=round,line cap=round,fill=fillColor] (218.47,396.74) circle (  1.16);

\path[draw=drawColor,line width= 0.4pt,line join=round,line cap=round,fill=fillColor] (218.60,396.70) circle (  1.16);

\path[draw=drawColor,line width= 0.4pt,line join=round,line cap=round,fill=fillColor] (218.73,396.65) circle (  1.16);

\path[draw=drawColor,line width= 0.4pt,line join=round,line cap=round,fill=fillColor] (218.87,396.21) circle (  1.16);

\path[draw=drawColor,line width= 0.4pt,line join=round,line cap=round,fill=fillColor] (219.00,396.19) circle (  1.16);

\path[draw=drawColor,line width= 0.4pt,line join=round,line cap=round,fill=fillColor] (219.13,396.14) circle (  1.16);

\path[draw=drawColor,line width= 0.4pt,line join=round,line cap=round,fill=fillColor] (219.26,395.80) circle (  1.16);

\path[draw=drawColor,line width= 0.4pt,line join=round,line cap=round,fill=fillColor] (219.40,395.01) circle (  1.16);

\path[draw=drawColor,line width= 0.4pt,line join=round,line cap=round,fill=fillColor] (219.53,394.91) circle (  1.16);

\path[draw=drawColor,line width= 0.4pt,line join=round,line cap=round,fill=fillColor] (219.66,394.60) circle (  1.16);

\path[draw=drawColor,line width= 0.4pt,line join=round,line cap=round,fill=fillColor] (219.79,394.48) circle (  1.16);

\path[draw=drawColor,line width= 0.4pt,line join=round,line cap=round,fill=fillColor] (219.92,394.40) circle (  1.16);

\path[draw=drawColor,line width= 0.4pt,line join=round,line cap=round,fill=fillColor] (220.05,394.21) circle (  1.16);

\path[draw=drawColor,line width= 0.4pt,line join=round,line cap=round,fill=fillColor] (220.18,393.80) circle (  1.16);

\path[draw=drawColor,line width= 0.4pt,line join=round,line cap=round,fill=fillColor] (220.31,393.37) circle (  1.16);

\path[draw=drawColor,line width= 0.4pt,line join=round,line cap=round,fill=fillColor] (220.45,393.17) circle (  1.16);

\path[draw=drawColor,line width= 0.4pt,line join=round,line cap=round,fill=fillColor] (220.58,393.11) circle (  1.16);

\path[draw=drawColor,line width= 0.4pt,line join=round,line cap=round,fill=fillColor] (220.71,393.04) circle (  1.16);

\path[draw=drawColor,line width= 0.4pt,line join=round,line cap=round,fill=fillColor] (220.84,392.98) circle (  1.16);

\path[draw=drawColor,line width= 0.4pt,line join=round,line cap=round,fill=fillColor] (220.97,392.98) circle (  1.16);

\path[draw=drawColor,line width= 0.4pt,line join=round,line cap=round,fill=fillColor] (221.09,392.79) circle (  1.16);

\path[draw=drawColor,line width= 0.4pt,line join=round,line cap=round,fill=fillColor] (221.22,392.76) circle (  1.16);

\path[draw=drawColor,line width= 0.4pt,line join=round,line cap=round,fill=fillColor] (221.35,392.62) circle (  1.16);

\path[draw=drawColor,line width= 0.4pt,line join=round,line cap=round,fill=fillColor] (221.48,392.18) circle (  1.16);

\path[draw=drawColor,line width= 0.4pt,line join=round,line cap=round,fill=fillColor] (221.61,391.57) circle (  1.16);

\path[draw=drawColor,line width= 0.4pt,line join=round,line cap=round,fill=fillColor] (221.74,391.48) circle (  1.16);

\path[draw=drawColor,line width= 0.4pt,line join=round,line cap=round,fill=fillColor] (221.87,391.21) circle (  1.16);

\path[draw=drawColor,line width= 0.4pt,line join=round,line cap=round,fill=fillColor] (222.00,390.78) circle (  1.16);

\path[draw=drawColor,line width= 0.4pt,line join=round,line cap=round,fill=fillColor] (222.12,390.28) circle (  1.16);

\path[draw=drawColor,line width= 0.4pt,line join=round,line cap=round,fill=fillColor] (222.25,389.61) circle (  1.16);

\path[draw=drawColor,line width= 0.4pt,line join=round,line cap=round,fill=fillColor] (222.38,389.35) circle (  1.16);

\path[draw=drawColor,line width= 0.4pt,line join=round,line cap=round,fill=fillColor] (222.51,389.27) circle (  1.16);

\path[draw=drawColor,line width= 0.4pt,line join=round,line cap=round,fill=fillColor] (222.63,387.15) circle (  1.16);

\path[draw=drawColor,line width= 0.4pt,line join=round,line cap=round,fill=fillColor] (222.76,385.70) circle (  1.16);

\path[draw=drawColor,line width= 0.4pt,line join=round,line cap=round,fill=fillColor] (222.89,385.34) circle (  1.16);

\path[draw=drawColor,line width= 0.4pt,line join=round,line cap=round,fill=fillColor] (223.01,382.05) circle (  1.16);

\path[draw=drawColor,line width= 0.4pt,line join=round,line cap=round,fill=fillColor] (223.14,381.48) circle (  1.16);

\path[draw=drawColor,line width= 0.4pt,line join=round,line cap=round,fill=fillColor] (223.27,372.42) circle (  1.16);

\path[draw=drawColor,line width= 0.4pt,line join=round,line cap=round,fill=fillColor] (223.39,372.42) circle (  1.16);

\path[draw=drawColor,line width= 0.4pt,line join=round,line cap=round,fill=fillColor] (223.52,372.42) circle (  1.16);

\path[draw=drawColor,line width= 0.4pt,line join=round,line cap=round,fill=fillColor] (223.64,372.42) circle (  1.16);

\path[draw=drawColor,line width= 0.4pt,line join=round,line cap=round,fill=fillColor] (223.77,372.42) circle (  1.16);

\path[draw=drawColor,line width= 0.4pt,line join=round,line cap=round,fill=fillColor] (223.89,372.42) circle (  1.16);

\path[draw=drawColor,line width= 0.4pt,line join=round,line cap=round,fill=fillColor] (224.02,372.42) circle (  1.16);

\path[draw=drawColor,line width= 0.4pt,line join=round,line cap=round,fill=fillColor] (224.14,372.42) circle (  1.16);

\path[draw=drawColor,line width= 0.4pt,line join=round,line cap=round,fill=fillColor] (224.27,372.42) circle (  1.16);

\path[draw=drawColor,line width= 0.4pt,line join=round,line cap=round,fill=fillColor] (224.39,372.42) circle (  1.16);

\path[draw=drawColor,line width= 0.4pt,line join=round,line cap=round,fill=fillColor] (224.52,372.42) circle (  1.16);

\path[draw=drawColor,line width= 0.4pt,line join=round,line cap=round,fill=fillColor] (224.64,372.42) circle (  1.16);

\path[draw=drawColor,line width= 0.4pt,line join=round,line cap=round,fill=fillColor] (224.77,372.42) circle (  1.16);

\path[draw=drawColor,line width= 0.4pt,line join=round,line cap=round,fill=fillColor] (224.89,372.42) circle (  1.16);

\path[draw=drawColor,line width= 0.4pt,line join=round,line cap=round,fill=fillColor] (225.01,372.42) circle (  1.16);

\path[draw=drawColor,line width= 0.4pt,line join=round,line cap=round,fill=fillColor] (225.14,372.42) circle (  1.16);

\path[draw=drawColor,line width= 0.4pt,line join=round,line cap=round,fill=fillColor] (225.26,372.42) circle (  1.16);
\definecolor[named]{drawColor}{rgb}{1.00,0.50,0.00}
\definecolor[named]{fillColor}{rgb}{1.00,0.50,0.00}

\path[draw=drawColor,line width= 0.4pt,line join=round,line cap=round,fill=fillColor] ( 74.88,455.36) circle (  1.16);

\path[draw=drawColor,line width= 0.4pt,line join=round,line cap=round,fill=fillColor] ( 80.66,444.58) circle (  1.16);

\path[draw=drawColor,line width= 0.4pt,line join=round,line cap=round,fill=fillColor] ( 84.72,444.58) circle (  1.16);

\path[draw=drawColor,line width= 0.4pt,line join=round,line cap=round,fill=fillColor] ( 87.95,441.71) circle (  1.16);

\path[draw=drawColor,line width= 0.4pt,line join=round,line cap=round,fill=fillColor] ( 90.68,440.91) circle (  1.16);

\path[draw=drawColor,line width= 0.4pt,line join=round,line cap=round,fill=fillColor] ( 93.06,439.01) circle (  1.16);

\path[draw=drawColor,line width= 0.4pt,line join=round,line cap=round,fill=fillColor] ( 95.19,438.76) circle (  1.16);

\path[draw=drawColor,line width= 0.4pt,line join=round,line cap=round,fill=fillColor] ( 97.13,438.62) circle (  1.16);

\path[draw=drawColor,line width= 0.4pt,line join=round,line cap=round,fill=fillColor] ( 98.91,438.48) circle (  1.16);

\path[draw=drawColor,line width= 0.4pt,line join=round,line cap=round,fill=fillColor] (100.57,438.01) circle (  1.16);

\path[draw=drawColor,line width= 0.4pt,line join=round,line cap=round,fill=fillColor] (102.11,437.33) circle (  1.16);

\path[draw=drawColor,line width= 0.4pt,line join=round,line cap=round,fill=fillColor] (103.57,436.61) circle (  1.16);

\path[draw=drawColor,line width= 0.4pt,line join=round,line cap=round,fill=fillColor] (104.95,436.30) circle (  1.16);

\path[draw=drawColor,line width= 0.4pt,line join=round,line cap=round,fill=fillColor] (106.26,435.98) circle (  1.16);

\path[draw=drawColor,line width= 0.4pt,line join=round,line cap=round,fill=fillColor] (107.50,435.58) circle (  1.16);

\path[draw=drawColor,line width= 0.4pt,line join=round,line cap=round,fill=fillColor] (108.70,435.12) circle (  1.16);

\path[draw=drawColor,line width= 0.4pt,line join=round,line cap=round,fill=fillColor] (109.84,434.85) circle (  1.16);

\path[draw=drawColor,line width= 0.4pt,line join=round,line cap=round,fill=fillColor] (110.94,431.89) circle (  1.16);

\path[draw=drawColor,line width= 0.4pt,line join=round,line cap=round,fill=fillColor] (112.00,431.72) circle (  1.16);

\path[draw=drawColor,line width= 0.4pt,line join=round,line cap=round,fill=fillColor] (113.03,431.69) circle (  1.16);

\path[draw=drawColor,line width= 0.4pt,line join=round,line cap=round,fill=fillColor] (114.02,431.16) circle (  1.16);

\path[draw=drawColor,line width= 0.4pt,line join=round,line cap=round,fill=fillColor] (114.98,430.43) circle (  1.16);

\path[draw=drawColor,line width= 0.4pt,line join=round,line cap=round,fill=fillColor] (115.91,429.65) circle (  1.16);

\path[draw=drawColor,line width= 0.4pt,line join=round,line cap=round,fill=fillColor] (116.81,428.42) circle (  1.16);

\path[draw=drawColor,line width= 0.4pt,line join=round,line cap=round,fill=fillColor] (117.69,428.20) circle (  1.16);

\path[draw=drawColor,line width= 0.4pt,line join=round,line cap=round,fill=fillColor] (118.55,428.20) circle (  1.16);

\path[draw=drawColor,line width= 0.4pt,line join=round,line cap=round,fill=fillColor] (119.38,427.75) circle (  1.16);

\path[draw=drawColor,line width= 0.4pt,line join=round,line cap=round,fill=fillColor] (120.20,427.75) circle (  1.16);

\path[draw=drawColor,line width= 0.4pt,line join=round,line cap=round,fill=fillColor] (120.99,427.64) circle (  1.16);

\path[draw=drawColor,line width= 0.4pt,line join=round,line cap=round,fill=fillColor] (121.77,427.60) circle (  1.16);

\path[draw=drawColor,line width= 0.4pt,line join=round,line cap=round,fill=fillColor] (122.53,427.09) circle (  1.16);

\path[draw=drawColor,line width= 0.4pt,line join=round,line cap=round,fill=fillColor] (123.27,426.95) circle (  1.16);

\path[draw=drawColor,line width= 0.4pt,line join=round,line cap=round,fill=fillColor] (124.00,426.91) circle (  1.16);

\path[draw=drawColor,line width= 0.4pt,line join=round,line cap=round,fill=fillColor] (124.71,426.28) circle (  1.16);

\path[draw=drawColor,line width= 0.4pt,line join=round,line cap=round,fill=fillColor] (125.41,426.16) circle (  1.16);

\path[draw=drawColor,line width= 0.4pt,line join=round,line cap=round,fill=fillColor] (126.10,425.79) circle (  1.16);

\path[draw=drawColor,line width= 0.4pt,line join=round,line cap=round,fill=fillColor] (126.77,425.54) circle (  1.16);

\path[draw=drawColor,line width= 0.4pt,line join=round,line cap=round,fill=fillColor] (127.44,425.40) circle (  1.16);

\path[draw=drawColor,line width= 0.4pt,line join=round,line cap=round,fill=fillColor] (128.09,425.38) circle (  1.16);

\path[draw=drawColor,line width= 0.4pt,line join=round,line cap=round,fill=fillColor] (128.73,425.22) circle (  1.16);

\path[draw=drawColor,line width= 0.4pt,line join=round,line cap=round,fill=fillColor] (129.35,424.93) circle (  1.16);

\path[draw=drawColor,line width= 0.4pt,line join=round,line cap=round,fill=fillColor] (129.97,424.87) circle (  1.16);

\path[draw=drawColor,line width= 0.4pt,line join=round,line cap=round,fill=fillColor] (130.58,424.61) circle (  1.16);

\path[draw=drawColor,line width= 0.4pt,line join=round,line cap=round,fill=fillColor] (131.18,424.58) circle (  1.16);

\path[draw=drawColor,line width= 0.4pt,line join=round,line cap=round,fill=fillColor] (131.77,424.37) circle (  1.16);

\path[draw=drawColor,line width= 0.4pt,line join=round,line cap=round,fill=fillColor] (132.35,424.17) circle (  1.16);

\path[draw=drawColor,line width= 0.4pt,line join=round,line cap=round,fill=fillColor] (132.93,424.10) circle (  1.16);

\path[draw=drawColor,line width= 0.4pt,line join=round,line cap=round,fill=fillColor] (133.49,424.06) circle (  1.16);

\path[draw=drawColor,line width= 0.4pt,line join=round,line cap=round,fill=fillColor] (134.05,423.91) circle (  1.16);

\path[draw=drawColor,line width= 0.4pt,line join=round,line cap=round,fill=fillColor] (134.60,423.71) circle (  1.16);

\path[draw=drawColor,line width= 0.4pt,line join=round,line cap=round,fill=fillColor] (135.14,423.68) circle (  1.16);

\path[draw=drawColor,line width= 0.4pt,line join=round,line cap=round,fill=fillColor] (135.68,423.16) circle (  1.16);

\path[draw=drawColor,line width= 0.4pt,line join=round,line cap=round,fill=fillColor] (136.21,423.16) circle (  1.16);

\path[draw=drawColor,line width= 0.4pt,line join=round,line cap=round,fill=fillColor] (136.73,423.14) circle (  1.16);

\path[draw=drawColor,line width= 0.4pt,line join=round,line cap=round,fill=fillColor] (137.25,422.71) circle (  1.16);

\path[draw=drawColor,line width= 0.4pt,line join=round,line cap=round,fill=fillColor] (137.76,422.67) circle (  1.16);

\path[draw=drawColor,line width= 0.4pt,line join=round,line cap=round,fill=fillColor] (138.26,422.59) circle (  1.16);

\path[draw=drawColor,line width= 0.4pt,line join=round,line cap=round,fill=fillColor] (138.76,422.48) circle (  1.16);

\path[draw=drawColor,line width= 0.4pt,line join=round,line cap=round,fill=fillColor] (139.25,421.90) circle (  1.16);

\path[draw=drawColor,line width= 0.4pt,line join=round,line cap=round,fill=fillColor] (139.74,421.56) circle (  1.16);

\path[draw=drawColor,line width= 0.4pt,line join=round,line cap=round,fill=fillColor] (140.22,421.47) circle (  1.16);

\path[draw=drawColor,line width= 0.4pt,line join=round,line cap=round,fill=fillColor] (140.70,421.27) circle (  1.16);

\path[draw=drawColor,line width= 0.4pt,line join=round,line cap=round,fill=fillColor] (141.17,421.27) circle (  1.16);

\path[draw=drawColor,line width= 0.4pt,line join=round,line cap=round,fill=fillColor] (141.63,421.21) circle (  1.16);

\path[draw=drawColor,line width= 0.4pt,line join=round,line cap=round,fill=fillColor] (142.09,421.10) circle (  1.16);

\path[draw=drawColor,line width= 0.4pt,line join=round,line cap=round,fill=fillColor] (142.55,420.98) circle (  1.16);

\path[draw=drawColor,line width= 0.4pt,line join=round,line cap=round,fill=fillColor] (143.00,420.98) circle (  1.16);

\path[draw=drawColor,line width= 0.4pt,line join=round,line cap=round,fill=fillColor] (143.45,420.97) circle (  1.16);

\path[draw=drawColor,line width= 0.4pt,line join=round,line cap=round,fill=fillColor] (143.89,420.92) circle (  1.16);

\path[draw=drawColor,line width= 0.4pt,line join=round,line cap=round,fill=fillColor] (144.33,420.37) circle (  1.16);

\path[draw=drawColor,line width= 0.4pt,line join=round,line cap=round,fill=fillColor] (144.77,420.31) circle (  1.16);

\path[draw=drawColor,line width= 0.4pt,line join=round,line cap=round,fill=fillColor] (145.20,420.29) circle (  1.16);

\path[draw=drawColor,line width= 0.4pt,line join=round,line cap=round,fill=fillColor] (145.62,420.29) circle (  1.16);

\path[draw=drawColor,line width= 0.4pt,line join=round,line cap=round,fill=fillColor] (146.05,420.27) circle (  1.16);

\path[draw=drawColor,line width= 0.4pt,line join=round,line cap=round,fill=fillColor] (146.46,420.23) circle (  1.16);

\path[draw=drawColor,line width= 0.4pt,line join=round,line cap=round,fill=fillColor] (146.88,420.16) circle (  1.16);

\path[draw=drawColor,line width= 0.4pt,line join=round,line cap=round,fill=fillColor] (147.29,419.98) circle (  1.16);

\path[draw=drawColor,line width= 0.4pt,line join=round,line cap=round,fill=fillColor] (147.70,419.83) circle (  1.16);

\path[draw=drawColor,line width= 0.4pt,line join=round,line cap=round,fill=fillColor] (148.10,419.78) circle (  1.16);

\path[draw=drawColor,line width= 0.4pt,line join=round,line cap=round,fill=fillColor] (148.50,419.66) circle (  1.16);

\path[draw=drawColor,line width= 0.4pt,line join=round,line cap=round,fill=fillColor] (148.90,419.65) circle (  1.16);

\path[draw=drawColor,line width= 0.4pt,line join=round,line cap=round,fill=fillColor] (149.30,419.63) circle (  1.16);

\path[draw=drawColor,line width= 0.4pt,line join=round,line cap=round,fill=fillColor] (149.69,419.58) circle (  1.16);

\path[draw=drawColor,line width= 0.4pt,line join=round,line cap=round,fill=fillColor] (150.08,419.57) circle (  1.16);

\path[draw=drawColor,line width= 0.4pt,line join=round,line cap=round,fill=fillColor] (150.46,419.34) circle (  1.16);

\path[draw=drawColor,line width= 0.4pt,line join=round,line cap=round,fill=fillColor] (150.84,419.29) circle (  1.16);

\path[draw=drawColor,line width= 0.4pt,line join=round,line cap=round,fill=fillColor] (151.22,419.21) circle (  1.16);

\path[draw=drawColor,line width= 0.4pt,line join=round,line cap=round,fill=fillColor] (151.60,419.06) circle (  1.16);

\path[draw=drawColor,line width= 0.4pt,line join=round,line cap=round,fill=fillColor] (151.97,418.96) circle (  1.16);

\path[draw=drawColor,line width= 0.4pt,line join=round,line cap=round,fill=fillColor] (152.34,418.87) circle (  1.16);

\path[draw=drawColor,line width= 0.4pt,line join=round,line cap=round,fill=fillColor] (152.71,418.85) circle (  1.16);

\path[draw=drawColor,line width= 0.4pt,line join=round,line cap=round,fill=fillColor] (153.08,418.85) circle (  1.16);

\path[draw=drawColor,line width= 0.4pt,line join=round,line cap=round,fill=fillColor] (153.44,418.84) circle (  1.16);

\path[draw=drawColor,line width= 0.4pt,line join=round,line cap=round,fill=fillColor] (153.80,418.83) circle (  1.16);

\path[draw=drawColor,line width= 0.4pt,line join=round,line cap=round,fill=fillColor] (154.16,418.79) circle (  1.16);

\path[draw=drawColor,line width= 0.4pt,line join=round,line cap=round,fill=fillColor] (154.51,418.54) circle (  1.16);

\path[draw=drawColor,line width= 0.4pt,line join=round,line cap=round,fill=fillColor] (154.86,418.53) circle (  1.16);

\path[draw=drawColor,line width= 0.4pt,line join=round,line cap=round,fill=fillColor] (155.22,418.52) circle (  1.16);

\path[draw=drawColor,line width= 0.4pt,line join=round,line cap=round,fill=fillColor] (155.56,418.42) circle (  1.16);

\path[draw=drawColor,line width= 0.4pt,line join=round,line cap=round,fill=fillColor] (155.91,418.41) circle (  1.16);

\path[draw=drawColor,line width= 0.4pt,line join=round,line cap=round,fill=fillColor] (156.25,418.40) circle (  1.16);

\path[draw=drawColor,line width= 0.4pt,line join=round,line cap=round,fill=fillColor] (156.59,418.28) circle (  1.16);

\path[draw=drawColor,line width= 0.4pt,line join=round,line cap=round,fill=fillColor] (156.93,418.20) circle (  1.16);

\path[draw=drawColor,line width= 0.4pt,line join=round,line cap=round,fill=fillColor] (157.27,418.19) circle (  1.16);

\path[draw=drawColor,line width= 0.4pt,line join=round,line cap=round,fill=fillColor] (157.60,418.10) circle (  1.16);

\path[draw=drawColor,line width= 0.4pt,line join=round,line cap=round,fill=fillColor] (157.93,418.04) circle (  1.16);

\path[draw=drawColor,line width= 0.4pt,line join=round,line cap=round,fill=fillColor] (158.26,417.98) circle (  1.16);

\path[draw=drawColor,line width= 0.4pt,line join=round,line cap=round,fill=fillColor] (158.59,417.92) circle (  1.16);

\path[draw=drawColor,line width= 0.4pt,line join=round,line cap=round,fill=fillColor] (158.92,417.85) circle (  1.16);

\path[draw=drawColor,line width= 0.4pt,line join=round,line cap=round,fill=fillColor] (159.24,417.80) circle (  1.16);

\path[draw=drawColor,line width= 0.4pt,line join=round,line cap=round,fill=fillColor] (159.56,417.74) circle (  1.16);

\path[draw=drawColor,line width= 0.4pt,line join=round,line cap=round,fill=fillColor] (159.88,417.64) circle (  1.16);

\path[draw=drawColor,line width= 0.4pt,line join=round,line cap=round,fill=fillColor] (160.20,417.62) circle (  1.16);

\path[draw=drawColor,line width= 0.4pt,line join=round,line cap=round,fill=fillColor] (160.52,417.47) circle (  1.16);

\path[draw=drawColor,line width= 0.4pt,line join=round,line cap=round,fill=fillColor] (160.83,417.43) circle (  1.16);

\path[draw=drawColor,line width= 0.4pt,line join=round,line cap=round,fill=fillColor] (161.15,417.39) circle (  1.16);

\path[draw=drawColor,line width= 0.4pt,line join=round,line cap=round,fill=fillColor] (161.46,417.32) circle (  1.16);

\path[draw=drawColor,line width= 0.4pt,line join=round,line cap=round,fill=fillColor] (161.77,417.29) circle (  1.16);

\path[draw=drawColor,line width= 0.4pt,line join=round,line cap=round,fill=fillColor] (162.07,417.17) circle (  1.16);

\path[draw=drawColor,line width= 0.4pt,line join=round,line cap=round,fill=fillColor] (162.38,417.17) circle (  1.16);

\path[draw=drawColor,line width= 0.4pt,line join=round,line cap=round,fill=fillColor] (162.68,417.08) circle (  1.16);

\path[draw=drawColor,line width= 0.4pt,line join=round,line cap=round,fill=fillColor] (162.99,417.04) circle (  1.16);

\path[draw=drawColor,line width= 0.4pt,line join=round,line cap=round,fill=fillColor] (163.29,417.04) circle (  1.16);

\path[draw=drawColor,line width= 0.4pt,line join=round,line cap=round,fill=fillColor] (163.59,417.02) circle (  1.16);

\path[draw=drawColor,line width= 0.4pt,line join=round,line cap=round,fill=fillColor] (163.88,416.99) circle (  1.16);

\path[draw=drawColor,line width= 0.4pt,line join=round,line cap=round,fill=fillColor] (164.18,416.95) circle (  1.16);

\path[draw=drawColor,line width= 0.4pt,line join=round,line cap=round,fill=fillColor] (164.47,416.93) circle (  1.16);

\path[draw=drawColor,line width= 0.4pt,line join=round,line cap=round,fill=fillColor] (164.77,416.90) circle (  1.16);

\path[draw=drawColor,line width= 0.4pt,line join=round,line cap=round,fill=fillColor] (165.06,416.86) circle (  1.16);

\path[draw=drawColor,line width= 0.4pt,line join=round,line cap=round,fill=fillColor] (165.35,416.80) circle (  1.16);

\path[draw=drawColor,line width= 0.4pt,line join=round,line cap=round,fill=fillColor] (165.64,416.74) circle (  1.16);

\path[draw=drawColor,line width= 0.4pt,line join=round,line cap=round,fill=fillColor] (165.92,416.67) circle (  1.16);

\path[draw=drawColor,line width= 0.4pt,line join=round,line cap=round,fill=fillColor] (166.21,416.66) circle (  1.16);

\path[draw=drawColor,line width= 0.4pt,line join=round,line cap=round,fill=fillColor] (166.49,416.65) circle (  1.16);

\path[draw=drawColor,line width= 0.4pt,line join=round,line cap=round,fill=fillColor] (166.77,416.57) circle (  1.16);

\path[draw=drawColor,line width= 0.4pt,line join=round,line cap=round,fill=fillColor] (167.06,416.51) circle (  1.16);

\path[draw=drawColor,line width= 0.4pt,line join=round,line cap=round,fill=fillColor] (167.34,416.50) circle (  1.16);

\path[draw=drawColor,line width= 0.4pt,line join=round,line cap=round,fill=fillColor] (167.61,416.50) circle (  1.16);

\path[draw=drawColor,line width= 0.4pt,line join=round,line cap=round,fill=fillColor] (167.89,416.49) circle (  1.16);

\path[draw=drawColor,line width= 0.4pt,line join=round,line cap=round,fill=fillColor] (168.17,416.46) circle (  1.16);

\path[draw=drawColor,line width= 0.4pt,line join=round,line cap=round,fill=fillColor] (168.44,416.42) circle (  1.16);

\path[draw=drawColor,line width= 0.4pt,line join=round,line cap=round,fill=fillColor] (168.71,416.39) circle (  1.16);

\path[draw=drawColor,line width= 0.4pt,line join=round,line cap=round,fill=fillColor] (168.99,416.38) circle (  1.16);

\path[draw=drawColor,line width= 0.4pt,line join=round,line cap=round,fill=fillColor] (169.26,416.36) circle (  1.16);

\path[draw=drawColor,line width= 0.4pt,line join=round,line cap=round,fill=fillColor] (169.53,416.34) circle (  1.16);

\path[draw=drawColor,line width= 0.4pt,line join=round,line cap=round,fill=fillColor] (169.79,416.31) circle (  1.16);

\path[draw=drawColor,line width= 0.4pt,line join=round,line cap=round,fill=fillColor] (170.06,416.23) circle (  1.16);

\path[draw=drawColor,line width= 0.4pt,line join=round,line cap=round,fill=fillColor] (170.33,416.02) circle (  1.16);

\path[draw=drawColor,line width= 0.4pt,line join=round,line cap=round,fill=fillColor] (170.59,416.00) circle (  1.16);

\path[draw=drawColor,line width= 0.4pt,line join=round,line cap=round,fill=fillColor] (170.85,415.97) circle (  1.16);

\path[draw=drawColor,line width= 0.4pt,line join=round,line cap=round,fill=fillColor] (171.12,415.89) circle (  1.16);

\path[draw=drawColor,line width= 0.4pt,line join=round,line cap=round,fill=fillColor] (171.38,415.88) circle (  1.16);

\path[draw=drawColor,line width= 0.4pt,line join=round,line cap=round,fill=fillColor] (171.64,415.88) circle (  1.16);

\path[draw=drawColor,line width= 0.4pt,line join=round,line cap=round,fill=fillColor] (171.90,415.86) circle (  1.16);

\path[draw=drawColor,line width= 0.4pt,line join=round,line cap=round,fill=fillColor] (172.15,415.85) circle (  1.16);

\path[draw=drawColor,line width= 0.4pt,line join=round,line cap=round,fill=fillColor] (172.41,415.67) circle (  1.16);

\path[draw=drawColor,line width= 0.4pt,line join=round,line cap=round,fill=fillColor] (172.67,415.56) circle (  1.16);

\path[draw=drawColor,line width= 0.4pt,line join=round,line cap=round,fill=fillColor] (172.92,415.53) circle (  1.16);

\path[draw=drawColor,line width= 0.4pt,line join=round,line cap=round,fill=fillColor] (173.17,415.47) circle (  1.16);

\path[draw=drawColor,line width= 0.4pt,line join=round,line cap=round,fill=fillColor] (173.43,415.36) circle (  1.16);

\path[draw=drawColor,line width= 0.4pt,line join=round,line cap=round,fill=fillColor] (173.68,415.29) circle (  1.16);

\path[draw=drawColor,line width= 0.4pt,line join=round,line cap=round,fill=fillColor] (173.93,415.29) circle (  1.16);

\path[draw=drawColor,line width= 0.4pt,line join=round,line cap=round,fill=fillColor] (174.18,415.09) circle (  1.16);

\path[draw=drawColor,line width= 0.4pt,line join=round,line cap=round,fill=fillColor] (174.42,415.01) circle (  1.16);

\path[draw=drawColor,line width= 0.4pt,line join=round,line cap=round,fill=fillColor] (174.67,414.86) circle (  1.16);

\path[draw=drawColor,line width= 0.4pt,line join=round,line cap=round,fill=fillColor] (174.92,414.82) circle (  1.16);

\path[draw=drawColor,line width= 0.4pt,line join=round,line cap=round,fill=fillColor] (175.16,414.79) circle (  1.16);

\path[draw=drawColor,line width= 0.4pt,line join=round,line cap=round,fill=fillColor] (175.41,414.76) circle (  1.16);

\path[draw=drawColor,line width= 0.4pt,line join=round,line cap=round,fill=fillColor] (175.65,414.71) circle (  1.16);

\path[draw=drawColor,line width= 0.4pt,line join=round,line cap=round,fill=fillColor] (175.89,414.69) circle (  1.16);

\path[draw=drawColor,line width= 0.4pt,line join=round,line cap=round,fill=fillColor] (176.13,414.65) circle (  1.16);

\path[draw=drawColor,line width= 0.4pt,line join=round,line cap=round,fill=fillColor] (176.37,414.48) circle (  1.16);

\path[draw=drawColor,line width= 0.4pt,line join=round,line cap=round,fill=fillColor] (176.61,414.45) circle (  1.16);

\path[draw=drawColor,line width= 0.4pt,line join=round,line cap=round,fill=fillColor] (176.85,414.34) circle (  1.16);

\path[draw=drawColor,line width= 0.4pt,line join=round,line cap=round,fill=fillColor] (177.09,414.28) circle (  1.16);

\path[draw=drawColor,line width= 0.4pt,line join=round,line cap=round,fill=fillColor] (177.32,414.26) circle (  1.16);

\path[draw=drawColor,line width= 0.4pt,line join=round,line cap=round,fill=fillColor] (177.56,414.22) circle (  1.16);

\path[draw=drawColor,line width= 0.4pt,line join=round,line cap=round,fill=fillColor] (177.80,414.21) circle (  1.16);

\path[draw=drawColor,line width= 0.4pt,line join=round,line cap=round,fill=fillColor] (178.03,414.19) circle (  1.16);

\path[draw=drawColor,line width= 0.4pt,line join=round,line cap=round,fill=fillColor] (178.26,414.15) circle (  1.16);

\path[draw=drawColor,line width= 0.4pt,line join=round,line cap=round,fill=fillColor] (178.49,414.10) circle (  1.16);

\path[draw=drawColor,line width= 0.4pt,line join=round,line cap=round,fill=fillColor] (178.73,414.06) circle (  1.16);

\path[draw=drawColor,line width= 0.4pt,line join=round,line cap=round,fill=fillColor] (178.96,414.01) circle (  1.16);

\path[draw=drawColor,line width= 0.4pt,line join=round,line cap=round,fill=fillColor] (179.19,414.00) circle (  1.16);

\path[draw=drawColor,line width= 0.4pt,line join=round,line cap=round,fill=fillColor] (179.41,414.00) circle (  1.16);

\path[draw=drawColor,line width= 0.4pt,line join=round,line cap=round,fill=fillColor] (179.64,413.95) circle (  1.16);

\path[draw=drawColor,line width= 0.4pt,line join=round,line cap=round,fill=fillColor] (179.87,413.82) circle (  1.16);

\path[draw=drawColor,line width= 0.4pt,line join=round,line cap=round,fill=fillColor] (180.10,413.76) circle (  1.16);

\path[draw=drawColor,line width= 0.4pt,line join=round,line cap=round,fill=fillColor] (180.32,413.74) circle (  1.16);

\path[draw=drawColor,line width= 0.4pt,line join=round,line cap=round,fill=fillColor] (180.55,413.74) circle (  1.16);

\path[draw=drawColor,line width= 0.4pt,line join=round,line cap=round,fill=fillColor] (180.77,413.69) circle (  1.16);

\path[draw=drawColor,line width= 0.4pt,line join=round,line cap=round,fill=fillColor] (180.99,413.67) circle (  1.16);

\path[draw=drawColor,line width= 0.4pt,line join=round,line cap=round,fill=fillColor] (181.22,413.66) circle (  1.16);

\path[draw=drawColor,line width= 0.4pt,line join=round,line cap=round,fill=fillColor] (181.44,413.64) circle (  1.16);

\path[draw=drawColor,line width= 0.4pt,line join=round,line cap=round,fill=fillColor] (181.66,413.60) circle (  1.16);

\path[draw=drawColor,line width= 0.4pt,line join=round,line cap=round,fill=fillColor] (181.88,413.54) circle (  1.16);

\path[draw=drawColor,line width= 0.4pt,line join=round,line cap=round,fill=fillColor] (182.10,413.37) circle (  1.16);

\path[draw=drawColor,line width= 0.4pt,line join=round,line cap=round,fill=fillColor] (182.32,413.34) circle (  1.16);

\path[draw=drawColor,line width= 0.4pt,line join=round,line cap=round,fill=fillColor] (182.54,413.23) circle (  1.16);

\path[draw=drawColor,line width= 0.4pt,line join=round,line cap=round,fill=fillColor] (182.75,413.13) circle (  1.16);

\path[draw=drawColor,line width= 0.4pt,line join=round,line cap=round,fill=fillColor] (182.97,413.12) circle (  1.16);

\path[draw=drawColor,line width= 0.4pt,line join=round,line cap=round,fill=fillColor] (183.19,413.06) circle (  1.16);

\path[draw=drawColor,line width= 0.4pt,line join=round,line cap=round,fill=fillColor] (183.40,413.00) circle (  1.16);

\path[draw=drawColor,line width= 0.4pt,line join=round,line cap=round,fill=fillColor] (183.61,412.89) circle (  1.16);

\path[draw=drawColor,line width= 0.4pt,line join=round,line cap=round,fill=fillColor] (183.83,412.88) circle (  1.16);

\path[draw=drawColor,line width= 0.4pt,line join=round,line cap=round,fill=fillColor] (184.04,412.72) circle (  1.16);

\path[draw=drawColor,line width= 0.4pt,line join=round,line cap=round,fill=fillColor] (184.25,412.65) circle (  1.16);

\path[draw=drawColor,line width= 0.4pt,line join=round,line cap=round,fill=fillColor] (184.47,412.63) circle (  1.16);

\path[draw=drawColor,line width= 0.4pt,line join=round,line cap=round,fill=fillColor] (184.68,412.59) circle (  1.16);

\path[draw=drawColor,line width= 0.4pt,line join=round,line cap=round,fill=fillColor] (184.89,412.54) circle (  1.16);

\path[draw=drawColor,line width= 0.4pt,line join=round,line cap=round,fill=fillColor] (185.10,412.53) circle (  1.16);

\path[draw=drawColor,line width= 0.4pt,line join=round,line cap=round,fill=fillColor] (185.31,412.49) circle (  1.16);

\path[draw=drawColor,line width= 0.4pt,line join=round,line cap=round,fill=fillColor] (185.51,412.42) circle (  1.16);

\path[draw=drawColor,line width= 0.4pt,line join=round,line cap=round,fill=fillColor] (185.72,412.31) circle (  1.16);

\path[draw=drawColor,line width= 0.4pt,line join=round,line cap=round,fill=fillColor] (185.93,412.27) circle (  1.16);

\path[draw=drawColor,line width= 0.4pt,line join=round,line cap=round,fill=fillColor] (186.13,412.26) circle (  1.16);

\path[draw=drawColor,line width= 0.4pt,line join=round,line cap=round,fill=fillColor] (186.34,412.23) circle (  1.16);

\path[draw=drawColor,line width= 0.4pt,line join=round,line cap=round,fill=fillColor] (186.55,412.16) circle (  1.16);

\path[draw=drawColor,line width= 0.4pt,line join=round,line cap=round,fill=fillColor] (186.75,412.14) circle (  1.16);

\path[draw=drawColor,line width= 0.4pt,line join=round,line cap=round,fill=fillColor] (186.95,412.11) circle (  1.16);

\path[draw=drawColor,line width= 0.4pt,line join=round,line cap=round,fill=fillColor] (187.16,412.04) circle (  1.16);

\path[draw=drawColor,line width= 0.4pt,line join=round,line cap=round,fill=fillColor] (187.36,411.95) circle (  1.16);

\path[draw=drawColor,line width= 0.4pt,line join=round,line cap=round,fill=fillColor] (187.56,411.90) circle (  1.16);

\path[draw=drawColor,line width= 0.4pt,line join=round,line cap=round,fill=fillColor] (187.76,411.87) circle (  1.16);

\path[draw=drawColor,line width= 0.4pt,line join=round,line cap=round,fill=fillColor] (187.96,411.81) circle (  1.16);

\path[draw=drawColor,line width= 0.4pt,line join=round,line cap=round,fill=fillColor] (188.16,411.60) circle (  1.16);

\path[draw=drawColor,line width= 0.4pt,line join=round,line cap=round,fill=fillColor] (188.36,411.60) circle (  1.16);

\path[draw=drawColor,line width= 0.4pt,line join=round,line cap=round,fill=fillColor] (188.56,411.56) circle (  1.16);

\path[draw=drawColor,line width= 0.4pt,line join=round,line cap=round,fill=fillColor] (188.76,411.53) circle (  1.16);

\path[draw=drawColor,line width= 0.4pt,line join=round,line cap=round,fill=fillColor] (188.96,411.50) circle (  1.16);

\path[draw=drawColor,line width= 0.4pt,line join=round,line cap=round,fill=fillColor] (189.16,411.49) circle (  1.16);

\path[draw=drawColor,line width= 0.4pt,line join=round,line cap=round,fill=fillColor] (189.35,411.42) circle (  1.16);

\path[draw=drawColor,line width= 0.4pt,line join=round,line cap=round,fill=fillColor] (189.55,411.41) circle (  1.16);

\path[draw=drawColor,line width= 0.4pt,line join=round,line cap=round,fill=fillColor] (189.74,411.39) circle (  1.16);

\path[draw=drawColor,line width= 0.4pt,line join=round,line cap=round,fill=fillColor] (189.94,411.33) circle (  1.16);

\path[draw=drawColor,line width= 0.4pt,line join=round,line cap=round,fill=fillColor] (190.13,411.32) circle (  1.16);

\path[draw=drawColor,line width= 0.4pt,line join=round,line cap=round,fill=fillColor] (190.33,411.22) circle (  1.16);

\path[draw=drawColor,line width= 0.4pt,line join=round,line cap=round,fill=fillColor] (190.52,411.20) circle (  1.16);

\path[draw=drawColor,line width= 0.4pt,line join=round,line cap=round,fill=fillColor] (190.71,411.03) circle (  1.16);

\path[draw=drawColor,line width= 0.4pt,line join=round,line cap=round,fill=fillColor] (190.91,411.01) circle (  1.16);

\path[draw=drawColor,line width= 0.4pt,line join=round,line cap=round,fill=fillColor] (191.10,410.98) circle (  1.16);

\path[draw=drawColor,line width= 0.4pt,line join=round,line cap=round,fill=fillColor] (191.29,410.91) circle (  1.16);

\path[draw=drawColor,line width= 0.4pt,line join=round,line cap=round,fill=fillColor] (191.48,410.89) circle (  1.16);

\path[draw=drawColor,line width= 0.4pt,line join=round,line cap=round,fill=fillColor] (191.67,410.87) circle (  1.16);

\path[draw=drawColor,line width= 0.4pt,line join=round,line cap=round,fill=fillColor] (191.86,410.87) circle (  1.16);

\path[draw=drawColor,line width= 0.4pt,line join=round,line cap=round,fill=fillColor] (192.05,410.85) circle (  1.16);

\path[draw=drawColor,line width= 0.4pt,line join=round,line cap=round,fill=fillColor] (192.24,410.79) circle (  1.16);

\path[draw=drawColor,line width= 0.4pt,line join=round,line cap=round,fill=fillColor] (192.43,410.77) circle (  1.16);

\path[draw=drawColor,line width= 0.4pt,line join=round,line cap=round,fill=fillColor] (192.61,410.76) circle (  1.16);

\path[draw=drawColor,line width= 0.4pt,line join=round,line cap=round,fill=fillColor] (192.80,410.72) circle (  1.16);

\path[draw=drawColor,line width= 0.4pt,line join=round,line cap=round,fill=fillColor] (192.99,410.72) circle (  1.16);

\path[draw=drawColor,line width= 0.4pt,line join=round,line cap=round,fill=fillColor] (193.17,410.64) circle (  1.16);

\path[draw=drawColor,line width= 0.4pt,line join=round,line cap=round,fill=fillColor] (193.36,410.48) circle (  1.16);

\path[draw=drawColor,line width= 0.4pt,line join=round,line cap=round,fill=fillColor] (193.54,410.45) circle (  1.16);

\path[draw=drawColor,line width= 0.4pt,line join=round,line cap=round,fill=fillColor] (193.73,410.40) circle (  1.16);

\path[draw=drawColor,line width= 0.4pt,line join=round,line cap=round,fill=fillColor] (193.91,410.37) circle (  1.16);

\path[draw=drawColor,line width= 0.4pt,line join=round,line cap=round,fill=fillColor] (194.10,410.24) circle (  1.16);

\path[draw=drawColor,line width= 0.4pt,line join=round,line cap=round,fill=fillColor] (194.28,410.24) circle (  1.16);

\path[draw=drawColor,line width= 0.4pt,line join=round,line cap=round,fill=fillColor] (194.46,410.14) circle (  1.16);

\path[draw=drawColor,line width= 0.4pt,line join=round,line cap=round,fill=fillColor] (194.65,410.09) circle (  1.16);

\path[draw=drawColor,line width= 0.4pt,line join=round,line cap=round,fill=fillColor] (194.83,409.99) circle (  1.16);

\path[draw=drawColor,line width= 0.4pt,line join=round,line cap=round,fill=fillColor] (195.01,409.85) circle (  1.16);

\path[draw=drawColor,line width= 0.4pt,line join=round,line cap=round,fill=fillColor] (195.19,409.84) circle (  1.16);

\path[draw=drawColor,line width= 0.4pt,line join=round,line cap=round,fill=fillColor] (195.37,409.78) circle (  1.16);

\path[draw=drawColor,line width= 0.4pt,line join=round,line cap=round,fill=fillColor] (195.55,409.72) circle (  1.16);

\path[draw=drawColor,line width= 0.4pt,line join=round,line cap=round,fill=fillColor] (195.73,409.71) circle (  1.16);

\path[draw=drawColor,line width= 0.4pt,line join=round,line cap=round,fill=fillColor] (195.91,409.45) circle (  1.16);

\path[draw=drawColor,line width= 0.4pt,line join=round,line cap=round,fill=fillColor] (196.09,409.38) circle (  1.16);

\path[draw=drawColor,line width= 0.4pt,line join=round,line cap=round,fill=fillColor] (196.27,409.36) circle (  1.16);

\path[draw=drawColor,line width= 0.4pt,line join=round,line cap=round,fill=fillColor] (196.44,409.34) circle (  1.16);

\path[draw=drawColor,line width= 0.4pt,line join=round,line cap=round,fill=fillColor] (196.62,409.33) circle (  1.16);

\path[draw=drawColor,line width= 0.4pt,line join=round,line cap=round,fill=fillColor] (196.80,409.27) circle (  1.16);

\path[draw=drawColor,line width= 0.4pt,line join=round,line cap=round,fill=fillColor] (196.97,409.05) circle (  1.16);

\path[draw=drawColor,line width= 0.4pt,line join=round,line cap=round,fill=fillColor] (197.15,408.95) circle (  1.16);

\path[draw=drawColor,line width= 0.4pt,line join=round,line cap=round,fill=fillColor] (197.33,408.94) circle (  1.16);

\path[draw=drawColor,line width= 0.4pt,line join=round,line cap=round,fill=fillColor] (197.50,408.92) circle (  1.16);

\path[draw=drawColor,line width= 0.4pt,line join=round,line cap=round,fill=fillColor] (197.68,408.89) circle (  1.16);

\path[draw=drawColor,line width= 0.4pt,line join=round,line cap=round,fill=fillColor] (197.85,408.87) circle (  1.16);

\path[draw=drawColor,line width= 0.4pt,line join=round,line cap=round,fill=fillColor] (198.02,408.77) circle (  1.16);

\path[draw=drawColor,line width= 0.4pt,line join=round,line cap=round,fill=fillColor] (198.20,408.77) circle (  1.16);

\path[draw=drawColor,line width= 0.4pt,line join=round,line cap=round,fill=fillColor] (198.37,408.73) circle (  1.16);

\path[draw=drawColor,line width= 0.4pt,line join=round,line cap=round,fill=fillColor] (198.54,408.72) circle (  1.16);

\path[draw=drawColor,line width= 0.4pt,line join=round,line cap=round,fill=fillColor] (198.72,408.68) circle (  1.16);

\path[draw=drawColor,line width= 0.4pt,line join=round,line cap=round,fill=fillColor] (198.89,408.67) circle (  1.16);

\path[draw=drawColor,line width= 0.4pt,line join=round,line cap=round,fill=fillColor] (199.06,408.62) circle (  1.16);

\path[draw=drawColor,line width= 0.4pt,line join=round,line cap=round,fill=fillColor] (199.23,408.62) circle (  1.16);

\path[draw=drawColor,line width= 0.4pt,line join=round,line cap=round,fill=fillColor] (199.40,408.61) circle (  1.16);

\path[draw=drawColor,line width= 0.4pt,line join=round,line cap=round,fill=fillColor] (199.57,408.46) circle (  1.16);

\path[draw=drawColor,line width= 0.4pt,line join=round,line cap=round,fill=fillColor] (199.74,408.43) circle (  1.16);

\path[draw=drawColor,line width= 0.4pt,line join=round,line cap=round,fill=fillColor] (199.91,408.39) circle (  1.16);

\path[draw=drawColor,line width= 0.4pt,line join=round,line cap=round,fill=fillColor] (200.08,408.31) circle (  1.16);

\path[draw=drawColor,line width= 0.4pt,line join=round,line cap=round,fill=fillColor] (200.25,408.28) circle (  1.16);

\path[draw=drawColor,line width= 0.4pt,line join=round,line cap=round,fill=fillColor] (200.42,408.28) circle (  1.16);

\path[draw=drawColor,line width= 0.4pt,line join=round,line cap=round,fill=fillColor] (200.58,408.20) circle (  1.16);

\path[draw=drawColor,line width= 0.4pt,line join=round,line cap=round,fill=fillColor] (200.75,408.18) circle (  1.16);

\path[draw=drawColor,line width= 0.4pt,line join=round,line cap=round,fill=fillColor] (200.92,408.08) circle (  1.16);

\path[draw=drawColor,line width= 0.4pt,line join=round,line cap=round,fill=fillColor] (201.09,408.03) circle (  1.16);

\path[draw=drawColor,line width= 0.4pt,line join=round,line cap=round,fill=fillColor] (201.25,407.95) circle (  1.16);

\path[draw=drawColor,line width= 0.4pt,line join=round,line cap=round,fill=fillColor] (201.42,407.86) circle (  1.16);

\path[draw=drawColor,line width= 0.4pt,line join=round,line cap=round,fill=fillColor] (201.58,407.83) circle (  1.16);

\path[draw=drawColor,line width= 0.4pt,line join=round,line cap=round,fill=fillColor] (201.75,407.78) circle (  1.16);

\path[draw=drawColor,line width= 0.4pt,line join=round,line cap=round,fill=fillColor] (201.91,407.77) circle (  1.16);

\path[draw=drawColor,line width= 0.4pt,line join=round,line cap=round,fill=fillColor] (202.08,407.68) circle (  1.16);

\path[draw=drawColor,line width= 0.4pt,line join=round,line cap=round,fill=fillColor] (202.24,407.59) circle (  1.16);

\path[draw=drawColor,line width= 0.4pt,line join=round,line cap=round,fill=fillColor] (202.41,407.55) circle (  1.16);

\path[draw=drawColor,line width= 0.4pt,line join=round,line cap=round,fill=fillColor] (202.57,407.52) circle (  1.16);

\path[draw=drawColor,line width= 0.4pt,line join=round,line cap=round,fill=fillColor] (202.73,407.47) circle (  1.16);

\path[draw=drawColor,line width= 0.4pt,line join=round,line cap=round,fill=fillColor] (202.90,407.44) circle (  1.16);

\path[draw=drawColor,line width= 0.4pt,line join=round,line cap=round,fill=fillColor] (203.06,407.22) circle (  1.16);

\path[draw=drawColor,line width= 0.4pt,line join=round,line cap=round,fill=fillColor] (203.22,407.22) circle (  1.16);

\path[draw=drawColor,line width= 0.4pt,line join=round,line cap=round,fill=fillColor] (203.38,407.19) circle (  1.16);

\path[draw=drawColor,line width= 0.4pt,line join=round,line cap=round,fill=fillColor] (203.54,407.13) circle (  1.16);

\path[draw=drawColor,line width= 0.4pt,line join=round,line cap=round,fill=fillColor] (203.71,407.01) circle (  1.16);

\path[draw=drawColor,line width= 0.4pt,line join=round,line cap=round,fill=fillColor] (203.87,407.00) circle (  1.16);

\path[draw=drawColor,line width= 0.4pt,line join=round,line cap=round,fill=fillColor] (204.03,406.99) circle (  1.16);

\path[draw=drawColor,line width= 0.4pt,line join=round,line cap=round,fill=fillColor] (204.19,406.91) circle (  1.16);

\path[draw=drawColor,line width= 0.4pt,line join=round,line cap=round,fill=fillColor] (204.35,406.90) circle (  1.16);

\path[draw=drawColor,line width= 0.4pt,line join=round,line cap=round,fill=fillColor] (204.51,406.77) circle (  1.16);

\path[draw=drawColor,line width= 0.4pt,line join=round,line cap=round,fill=fillColor] (204.66,406.77) circle (  1.16);

\path[draw=drawColor,line width= 0.4pt,line join=round,line cap=round,fill=fillColor] (204.82,406.76) circle (  1.16);

\path[draw=drawColor,line width= 0.4pt,line join=round,line cap=round,fill=fillColor] (204.98,406.71) circle (  1.16);

\path[draw=drawColor,line width= 0.4pt,line join=round,line cap=round,fill=fillColor] (205.14,406.71) circle (  1.16);

\path[draw=drawColor,line width= 0.4pt,line join=round,line cap=round,fill=fillColor] (205.30,406.69) circle (  1.16);

\path[draw=drawColor,line width= 0.4pt,line join=round,line cap=round,fill=fillColor] (205.45,406.68) circle (  1.16);

\path[draw=drawColor,line width= 0.4pt,line join=round,line cap=round,fill=fillColor] (205.61,406.57) circle (  1.16);

\path[draw=drawColor,line width= 0.4pt,line join=round,line cap=round,fill=fillColor] (205.77,406.47) circle (  1.16);

\path[draw=drawColor,line width= 0.4pt,line join=round,line cap=round,fill=fillColor] (205.92,406.47) circle (  1.16);

\path[draw=drawColor,line width= 0.4pt,line join=round,line cap=round,fill=fillColor] (206.08,406.42) circle (  1.16);

\path[draw=drawColor,line width= 0.4pt,line join=round,line cap=round,fill=fillColor] (206.24,406.37) circle (  1.16);

\path[draw=drawColor,line width= 0.4pt,line join=round,line cap=round,fill=fillColor] (206.39,406.27) circle (  1.16);

\path[draw=drawColor,line width= 0.4pt,line join=round,line cap=round,fill=fillColor] (206.55,406.21) circle (  1.16);

\path[draw=drawColor,line width= 0.4pt,line join=round,line cap=round,fill=fillColor] (206.70,406.20) circle (  1.16);

\path[draw=drawColor,line width= 0.4pt,line join=round,line cap=round,fill=fillColor] (206.86,406.09) circle (  1.16);

\path[draw=drawColor,line width= 0.4pt,line join=round,line cap=round,fill=fillColor] (207.01,406.02) circle (  1.16);

\path[draw=drawColor,line width= 0.4pt,line join=round,line cap=round,fill=fillColor] (207.16,405.98) circle (  1.16);

\path[draw=drawColor,line width= 0.4pt,line join=round,line cap=round,fill=fillColor] (207.32,405.89) circle (  1.16);

\path[draw=drawColor,line width= 0.4pt,line join=round,line cap=round,fill=fillColor] (207.47,405.87) circle (  1.16);

\path[draw=drawColor,line width= 0.4pt,line join=round,line cap=round,fill=fillColor] (207.62,405.57) circle (  1.16);

\path[draw=drawColor,line width= 0.4pt,line join=round,line cap=round,fill=fillColor] (207.78,405.50) circle (  1.16);

\path[draw=drawColor,line width= 0.4pt,line join=round,line cap=round,fill=fillColor] (207.93,405.46) circle (  1.16);

\path[draw=drawColor,line width= 0.4pt,line join=round,line cap=round,fill=fillColor] (208.08,405.38) circle (  1.16);

\path[draw=drawColor,line width= 0.4pt,line join=round,line cap=round,fill=fillColor] (208.23,405.32) circle (  1.16);

\path[draw=drawColor,line width= 0.4pt,line join=round,line cap=round,fill=fillColor] (208.39,405.26) circle (  1.16);

\path[draw=drawColor,line width= 0.4pt,line join=round,line cap=round,fill=fillColor] (208.54,405.24) circle (  1.16);

\path[draw=drawColor,line width= 0.4pt,line join=round,line cap=round,fill=fillColor] (208.69,405.21) circle (  1.16);

\path[draw=drawColor,line width= 0.4pt,line join=round,line cap=round,fill=fillColor] (208.84,405.19) circle (  1.16);

\path[draw=drawColor,line width= 0.4pt,line join=round,line cap=round,fill=fillColor] (208.99,405.16) circle (  1.16);

\path[draw=drawColor,line width= 0.4pt,line join=round,line cap=round,fill=fillColor] (209.14,405.15) circle (  1.16);

\path[draw=drawColor,line width= 0.4pt,line join=round,line cap=round,fill=fillColor] (209.29,405.10) circle (  1.16);

\path[draw=drawColor,line width= 0.4pt,line join=round,line cap=round,fill=fillColor] (209.44,405.02) circle (  1.16);

\path[draw=drawColor,line width= 0.4pt,line join=round,line cap=round,fill=fillColor] (209.59,405.01) circle (  1.16);

\path[draw=drawColor,line width= 0.4pt,line join=round,line cap=round,fill=fillColor] (209.74,404.90) circle (  1.16);

\path[draw=drawColor,line width= 0.4pt,line join=round,line cap=round,fill=fillColor] (209.88,404.88) circle (  1.16);

\path[draw=drawColor,line width= 0.4pt,line join=round,line cap=round,fill=fillColor] (210.03,404.81) circle (  1.16);

\path[draw=drawColor,line width= 0.4pt,line join=round,line cap=round,fill=fillColor] (210.18,404.80) circle (  1.16);

\path[draw=drawColor,line width= 0.4pt,line join=round,line cap=round,fill=fillColor] (210.33,404.80) circle (  1.16);

\path[draw=drawColor,line width= 0.4pt,line join=round,line cap=round,fill=fillColor] (210.48,404.75) circle (  1.16);

\path[draw=drawColor,line width= 0.4pt,line join=round,line cap=round,fill=fillColor] (210.62,404.68) circle (  1.16);

\path[draw=drawColor,line width= 0.4pt,line join=round,line cap=round,fill=fillColor] (210.77,404.65) circle (  1.16);

\path[draw=drawColor,line width= 0.4pt,line join=round,line cap=round,fill=fillColor] (210.92,404.65) circle (  1.16);

\path[draw=drawColor,line width= 0.4pt,line join=round,line cap=round,fill=fillColor] (211.06,404.62) circle (  1.16);

\path[draw=drawColor,line width= 0.4pt,line join=round,line cap=round,fill=fillColor] (211.21,404.49) circle (  1.16);

\path[draw=drawColor,line width= 0.4pt,line join=round,line cap=round,fill=fillColor] (211.36,404.48) circle (  1.16);

\path[draw=drawColor,line width= 0.4pt,line join=round,line cap=round,fill=fillColor] (211.50,404.48) circle (  1.16);

\path[draw=drawColor,line width= 0.4pt,line join=round,line cap=round,fill=fillColor] (211.65,404.39) circle (  1.16);

\path[draw=drawColor,line width= 0.4pt,line join=round,line cap=round,fill=fillColor] (211.79,404.37) circle (  1.16);

\path[draw=drawColor,line width= 0.4pt,line join=round,line cap=round,fill=fillColor] (211.94,404.34) circle (  1.16);

\path[draw=drawColor,line width= 0.4pt,line join=round,line cap=round,fill=fillColor] (212.08,404.29) circle (  1.16);

\path[draw=drawColor,line width= 0.4pt,line join=round,line cap=round,fill=fillColor] (212.23,404.23) circle (  1.16);

\path[draw=drawColor,line width= 0.4pt,line join=round,line cap=round,fill=fillColor] (212.37,404.14) circle (  1.16);

\path[draw=drawColor,line width= 0.4pt,line join=round,line cap=round,fill=fillColor] (212.51,403.94) circle (  1.16);

\path[draw=drawColor,line width= 0.4pt,line join=round,line cap=round,fill=fillColor] (212.66,403.91) circle (  1.16);

\path[draw=drawColor,line width= 0.4pt,line join=round,line cap=round,fill=fillColor] (212.80,403.85) circle (  1.16);

\path[draw=drawColor,line width= 0.4pt,line join=round,line cap=round,fill=fillColor] (212.94,403.78) circle (  1.16);

\path[draw=drawColor,line width= 0.4pt,line join=round,line cap=round,fill=fillColor] (213.09,403.78) circle (  1.16);

\path[draw=drawColor,line width= 0.4pt,line join=round,line cap=round,fill=fillColor] (213.23,403.65) circle (  1.16);

\path[draw=drawColor,line width= 0.4pt,line join=round,line cap=round,fill=fillColor] (213.37,403.64) circle (  1.16);

\path[draw=drawColor,line width= 0.4pt,line join=round,line cap=round,fill=fillColor] (213.51,403.63) circle (  1.16);

\path[draw=drawColor,line width= 0.4pt,line join=round,line cap=round,fill=fillColor] (213.65,403.44) circle (  1.16);

\path[draw=drawColor,line width= 0.4pt,line join=round,line cap=round,fill=fillColor] (213.80,403.43) circle (  1.16);

\path[draw=drawColor,line width= 0.4pt,line join=round,line cap=round,fill=fillColor] (213.94,403.35) circle (  1.16);

\path[draw=drawColor,line width= 0.4pt,line join=round,line cap=round,fill=fillColor] (214.08,403.24) circle (  1.16);

\path[draw=drawColor,line width= 0.4pt,line join=round,line cap=round,fill=fillColor] (214.22,403.19) circle (  1.16);

\path[draw=drawColor,line width= 0.4pt,line join=round,line cap=round,fill=fillColor] (214.36,403.04) circle (  1.16);

\path[draw=drawColor,line width= 0.4pt,line join=round,line cap=round,fill=fillColor] (214.50,402.82) circle (  1.16);

\path[draw=drawColor,line width= 0.4pt,line join=round,line cap=round,fill=fillColor] (214.64,402.80) circle (  1.16);

\path[draw=drawColor,line width= 0.4pt,line join=round,line cap=round,fill=fillColor] (214.78,402.68) circle (  1.16);

\path[draw=drawColor,line width= 0.4pt,line join=round,line cap=round,fill=fillColor] (214.92,402.63) circle (  1.16);

\path[draw=drawColor,line width= 0.4pt,line join=round,line cap=round,fill=fillColor] (215.06,402.50) circle (  1.16);

\path[draw=drawColor,line width= 0.4pt,line join=round,line cap=round,fill=fillColor] (215.20,402.18) circle (  1.16);

\path[draw=drawColor,line width= 0.4pt,line join=round,line cap=round,fill=fillColor] (215.34,402.01) circle (  1.16);

\path[draw=drawColor,line width= 0.4pt,line join=round,line cap=round,fill=fillColor] (215.47,401.96) circle (  1.16);

\path[draw=drawColor,line width= 0.4pt,line join=round,line cap=round,fill=fillColor] (215.61,401.96) circle (  1.16);

\path[draw=drawColor,line width= 0.4pt,line join=round,line cap=round,fill=fillColor] (215.75,401.94) circle (  1.16);

\path[draw=drawColor,line width= 0.4pt,line join=round,line cap=round,fill=fillColor] (215.89,401.78) circle (  1.16);

\path[draw=drawColor,line width= 0.4pt,line join=round,line cap=round,fill=fillColor] (216.03,401.73) circle (  1.16);

\path[draw=drawColor,line width= 0.4pt,line join=round,line cap=round,fill=fillColor] (216.16,401.72) circle (  1.16);

\path[draw=drawColor,line width= 0.4pt,line join=round,line cap=round,fill=fillColor] (216.30,401.25) circle (  1.16);

\path[draw=drawColor,line width= 0.4pt,line join=round,line cap=round,fill=fillColor] (216.44,401.19) circle (  1.16);

\path[draw=drawColor,line width= 0.4pt,line join=round,line cap=round,fill=fillColor] (216.58,401.11) circle (  1.16);

\path[draw=drawColor,line width= 0.4pt,line join=round,line cap=round,fill=fillColor] (216.71,401.09) circle (  1.16);

\path[draw=drawColor,line width= 0.4pt,line join=round,line cap=round,fill=fillColor] (216.85,400.96) circle (  1.16);

\path[draw=drawColor,line width= 0.4pt,line join=round,line cap=round,fill=fillColor] (216.98,400.76) circle (  1.16);

\path[draw=drawColor,line width= 0.4pt,line join=round,line cap=round,fill=fillColor] (217.12,400.37) circle (  1.16);

\path[draw=drawColor,line width= 0.4pt,line join=round,line cap=round,fill=fillColor] (217.26,400.26) circle (  1.16);

\path[draw=drawColor,line width= 0.4pt,line join=round,line cap=round,fill=fillColor] (217.39,399.92) circle (  1.16);

\path[draw=drawColor,line width= 0.4pt,line join=round,line cap=round,fill=fillColor] (217.53,399.74) circle (  1.16);

\path[draw=drawColor,line width= 0.4pt,line join=round,line cap=round,fill=fillColor] (217.66,399.50) circle (  1.16);

\path[draw=drawColor,line width= 0.4pt,line join=round,line cap=round,fill=fillColor] (217.80,399.28) circle (  1.16);

\path[draw=drawColor,line width= 0.4pt,line join=round,line cap=round,fill=fillColor] (217.93,399.26) circle (  1.16);

\path[draw=drawColor,line width= 0.4pt,line join=round,line cap=round,fill=fillColor] (218.06,398.97) circle (  1.16);

\path[draw=drawColor,line width= 0.4pt,line join=round,line cap=round,fill=fillColor] (218.20,398.72) circle (  1.16);

\path[draw=drawColor,line width= 0.4pt,line join=round,line cap=round,fill=fillColor] (218.33,398.39) circle (  1.16);

\path[draw=drawColor,line width= 0.4pt,line join=round,line cap=round,fill=fillColor] (218.47,398.32) circle (  1.16);

\path[draw=drawColor,line width= 0.4pt,line join=round,line cap=round,fill=fillColor] (218.60,398.28) circle (  1.16);

\path[draw=drawColor,line width= 0.4pt,line join=round,line cap=round,fill=fillColor] (218.73,397.73) circle (  1.16);

\path[draw=drawColor,line width= 0.4pt,line join=round,line cap=round,fill=fillColor] (218.87,397.68) circle (  1.16);

\path[draw=drawColor,line width= 0.4pt,line join=round,line cap=round,fill=fillColor] (219.00,397.33) circle (  1.16);

\path[draw=drawColor,line width= 0.4pt,line join=round,line cap=round,fill=fillColor] (219.13,397.20) circle (  1.16);

\path[draw=drawColor,line width= 0.4pt,line join=round,line cap=round,fill=fillColor] (219.26,397.04) circle (  1.16);

\path[draw=drawColor,line width= 0.4pt,line join=round,line cap=round,fill=fillColor] (219.40,396.98) circle (  1.16);

\path[draw=drawColor,line width= 0.4pt,line join=round,line cap=round,fill=fillColor] (219.53,396.83) circle (  1.16);

\path[draw=drawColor,line width= 0.4pt,line join=round,line cap=round,fill=fillColor] (219.66,396.83) circle (  1.16);

\path[draw=drawColor,line width= 0.4pt,line join=round,line cap=round,fill=fillColor] (219.79,396.68) circle (  1.16);

\path[draw=drawColor,line width= 0.4pt,line join=round,line cap=round,fill=fillColor] (219.92,396.63) circle (  1.16);

\path[draw=drawColor,line width= 0.4pt,line join=round,line cap=round,fill=fillColor] (220.05,396.52) circle (  1.16);

\path[draw=drawColor,line width= 0.4pt,line join=round,line cap=round,fill=fillColor] (220.18,396.50) circle (  1.16);

\path[draw=drawColor,line width= 0.4pt,line join=round,line cap=round,fill=fillColor] (220.31,396.29) circle (  1.16);

\path[draw=drawColor,line width= 0.4pt,line join=round,line cap=round,fill=fillColor] (220.45,395.99) circle (  1.16);

\path[draw=drawColor,line width= 0.4pt,line join=round,line cap=round,fill=fillColor] (220.58,395.11) circle (  1.16);

\path[draw=drawColor,line width= 0.4pt,line join=round,line cap=round,fill=fillColor] (220.71,394.85) circle (  1.16);

\path[draw=drawColor,line width= 0.4pt,line join=round,line cap=round,fill=fillColor] (220.84,394.67) circle (  1.16);

\path[draw=drawColor,line width= 0.4pt,line join=round,line cap=round,fill=fillColor] (220.97,394.25) circle (  1.16);

\path[draw=drawColor,line width= 0.4pt,line join=round,line cap=round,fill=fillColor] (221.09,393.59) circle (  1.16);

\path[draw=drawColor,line width= 0.4pt,line join=round,line cap=round,fill=fillColor] (221.22,393.53) circle (  1.16);

\path[draw=drawColor,line width= 0.4pt,line join=round,line cap=round,fill=fillColor] (221.35,393.53) circle (  1.16);

\path[draw=drawColor,line width= 0.4pt,line join=round,line cap=round,fill=fillColor] (221.48,393.39) circle (  1.16);

\path[draw=drawColor,line width= 0.4pt,line join=round,line cap=round,fill=fillColor] (221.61,393.16) circle (  1.16);

\path[draw=drawColor,line width= 0.4pt,line join=round,line cap=round,fill=fillColor] (221.74,393.14) circle (  1.16);

\path[draw=drawColor,line width= 0.4pt,line join=round,line cap=round,fill=fillColor] (221.87,393.07) circle (  1.16);

\path[draw=drawColor,line width= 0.4pt,line join=round,line cap=round,fill=fillColor] (222.00,392.46) circle (  1.16);

\path[draw=drawColor,line width= 0.4pt,line join=round,line cap=round,fill=fillColor] (222.12,391.64) circle (  1.16);

\path[draw=drawColor,line width= 0.4pt,line join=round,line cap=round,fill=fillColor] (222.25,390.90) circle (  1.16);

\path[draw=drawColor,line width= 0.4pt,line join=round,line cap=round,fill=fillColor] (222.38,390.39) circle (  1.16);

\path[draw=drawColor,line width= 0.4pt,line join=round,line cap=round,fill=fillColor] (222.51,390.01) circle (  1.16);

\path[draw=drawColor,line width= 0.4pt,line join=round,line cap=round,fill=fillColor] (222.63,389.96) circle (  1.16);

\path[draw=drawColor,line width= 0.4pt,line join=round,line cap=round,fill=fillColor] (222.76,389.71) circle (  1.16);

\path[draw=drawColor,line width= 0.4pt,line join=round,line cap=round,fill=fillColor] (222.89,388.95) circle (  1.16);

\path[draw=drawColor,line width= 0.4pt,line join=round,line cap=round,fill=fillColor] (223.01,388.82) circle (  1.16);

\path[draw=drawColor,line width= 0.4pt,line join=round,line cap=round,fill=fillColor] (223.14,387.87) circle (  1.16);

\path[draw=drawColor,line width= 0.4pt,line join=round,line cap=round,fill=fillColor] (223.27,387.72) circle (  1.16);

\path[draw=drawColor,line width= 0.4pt,line join=round,line cap=round,fill=fillColor] (223.39,387.19) circle (  1.16);

\path[draw=drawColor,line width= 0.4pt,line join=round,line cap=round,fill=fillColor] (223.52,386.40) circle (  1.16);

\path[draw=drawColor,line width= 0.4pt,line join=round,line cap=round,fill=fillColor] (223.64,385.50) circle (  1.16);

\path[draw=drawColor,line width= 0.4pt,line join=round,line cap=round,fill=fillColor] (223.77,385.03) circle (  1.16);

\path[draw=drawColor,line width= 0.4pt,line join=round,line cap=round,fill=fillColor] (223.89,384.72) circle (  1.16);

\path[draw=drawColor,line width= 0.4pt,line join=round,line cap=round,fill=fillColor] (224.02,381.48) circle (  1.16);

\path[draw=drawColor,line width= 0.4pt,line join=round,line cap=round,fill=fillColor] (224.14,380.13) circle (  1.16);

\path[draw=drawColor,line width= 0.4pt,line join=round,line cap=round,fill=fillColor] (224.27,372.42) circle (  1.16);

\path[draw=drawColor,line width= 0.4pt,line join=round,line cap=round,fill=fillColor] (224.39,372.42) circle (  1.16);

\path[draw=drawColor,line width= 0.4pt,line join=round,line cap=round,fill=fillColor] (224.52,372.42) circle (  1.16);

\path[draw=drawColor,line width= 0.4pt,line join=round,line cap=round,fill=fillColor] (224.64,372.42) circle (  1.16);

\path[draw=drawColor,line width= 0.4pt,line join=round,line cap=round,fill=fillColor] (224.77,372.42) circle (  1.16);

\path[draw=drawColor,line width= 0.4pt,line join=round,line cap=round,fill=fillColor] (224.89,372.42) circle (  1.16);

\path[draw=drawColor,line width= 0.4pt,line join=round,line cap=round,fill=fillColor] (225.01,372.42) circle (  1.16);

\path[draw=drawColor,line width= 0.4pt,line join=round,line cap=round,fill=fillColor] (225.14,372.42) circle (  1.16);

\path[draw=drawColor,line width= 0.4pt,line join=round,line cap=round,fill=fillColor] (225.26,372.42) circle (  1.16);
\definecolor[named]{drawColor}{rgb}{0.00,0.00,0.00}
\definecolor[named]{fillColor}{rgb}{0.00,0.00,0.00}

\path[draw=drawColor,line width= 0.6pt,line join=round,fill=fillColor] ( 67.36,455.38) -- (232.78,455.38);

\node[text=drawColor,anchor=base east,inner sep=0pt, outer sep=0pt, scale=  0.85] at (229.28,457.54) {infeasible solutions};

\path[draw=drawColor,line width= 0.6pt,line join=round,line cap=round] ( 67.36,364.12) rectangle (232.78,466.36);
\end{scope}
\begin{scope}
\path[clip] (  0.00,  0.00) rectangle (505.89,650.43);
\definecolor[named]{drawColor}{rgb}{0.00,0.00,0.00}

\node[text=drawColor,anchor=base east,inner sep=0pt, outer sep=0pt, scale=  0.80] at ( 61.96,369.66) {0.00};

\node[text=drawColor,anchor=base east,inner sep=0pt, outer sep=0pt, scale=  0.80] at ( 61.96,387.54) {0.01};

\node[text=drawColor,anchor=base east,inner sep=0pt, outer sep=0pt, scale=  0.80] at ( 61.96,400.23) {0.05};

\node[text=drawColor,anchor=base east,inner sep=0pt, outer sep=0pt, scale=  0.80] at ( 61.96,408.17) {0.10};

\node[text=drawColor,anchor=base east,inner sep=0pt, outer sep=0pt, scale=  0.80] at ( 61.96,418.18) {0.20};

\node[text=drawColor,anchor=base east,inner sep=0pt, outer sep=0pt, scale=  0.80] at ( 61.96,430.79) {0.40};

\node[text=drawColor,anchor=base east,inner sep=0pt, outer sep=0pt, scale=  0.80] at ( 61.96,439.64) {0.60};

\node[text=drawColor,anchor=base east,inner sep=0pt, outer sep=0pt, scale=  0.80] at ( 61.96,446.68) {0.80};

\node[text=drawColor,anchor=base east,inner sep=0pt, outer sep=0pt, scale=  0.80] at ( 61.96,452.63) {1.00};
\end{scope}
\begin{scope}
\path[clip] (  0.00,  0.00) rectangle (505.89,650.43);
\definecolor[named]{drawColor}{rgb}{0.00,0.00,0.00}

\path[draw=drawColor,line width= 0.6pt,line join=round] ( 64.36,372.42) --
	( 67.36,372.42);

\path[draw=drawColor,line width= 0.6pt,line join=round] ( 64.36,390.29) --
	( 67.36,390.29);

\path[draw=drawColor,line width= 0.6pt,line join=round] ( 64.36,402.98) --
	( 67.36,402.98);

\path[draw=drawColor,line width= 0.6pt,line join=round] ( 64.36,410.93) --
	( 67.36,410.93);

\path[draw=drawColor,line width= 0.6pt,line join=round] ( 64.36,420.94) --
	( 67.36,420.94);

\path[draw=drawColor,line width= 0.6pt,line join=round] ( 64.36,433.55) --
	( 67.36,433.55);

\path[draw=drawColor,line width= 0.6pt,line join=round] ( 64.36,442.39) --
	( 67.36,442.39);

\path[draw=drawColor,line width= 0.6pt,line join=round] ( 64.36,449.44) --
	( 67.36,449.44);

\path[draw=drawColor,line width= 0.6pt,line join=round] ( 64.36,455.38) --
	( 67.36,455.38);
\end{scope}
\begin{scope}
\path[clip] (  0.00,  0.00) rectangle (505.89,650.43);
\definecolor[named]{drawColor}{rgb}{0.00,0.00,0.00}

\path[draw=drawColor,line width= 0.6pt,line join=round] (155.91,361.12) --
	(155.91,364.12);

\path[draw=drawColor,line width= 0.6pt,line join=round] (182.75,361.12) --
	(182.75,364.12);

\path[draw=drawColor,line width= 0.6pt,line join=round] (201.58,361.12) --
	(201.58,364.12);

\path[draw=drawColor,line width= 0.6pt,line join=round] (216.58,361.12) --
	(216.58,364.12);

\path[draw=drawColor,line width= 0.6pt,line join=round] (229.23,361.12) --
	(229.23,364.12);
\end{scope}
\begin{scope}
\path[clip] (  0.00,  0.00) rectangle (505.89,650.43);
\definecolor[named]{drawColor}{rgb}{0.00,0.00,0.00}

\node[text=drawColor,rotate= 50.00,anchor=base east,inner sep=0pt, outer sep=0pt, scale=  0.80] at (160.13,355.18) {100};

\node[text=drawColor,rotate= 50.00,anchor=base east,inner sep=0pt, outer sep=0pt, scale=  0.80] at (186.97,355.18) {200};

\node[text=drawColor,rotate= 50.00,anchor=base east,inner sep=0pt, outer sep=0pt, scale=  0.80] at (205.80,355.18) {300};

\node[text=drawColor,rotate= 50.00,anchor=base east,inner sep=0pt, outer sep=0pt, scale=  0.80] at (220.80,355.18) {400};

\node[text=drawColor,rotate= 50.00,anchor=base east,inner sep=0pt, outer sep=0pt, scale=  0.80] at (233.46,355.18) {500};
\end{scope}
\begin{scope}
\path[clip] (  0.00,  0.00) rectangle (505.89,650.43);
\definecolor[named]{drawColor}{rgb}{0.00,0.00,0.00}

\node[text=drawColor,anchor=base,inner sep=0pt, outer sep=0pt, scale=  1.10] at (150.07,333.61) {\# Instances};
\end{scope}
\begin{scope}
\path[clip] (  0.00,  0.00) rectangle (505.89,650.43);
\definecolor[named]{drawColor}{rgb}{0.00,0.00,0.00}

\node[text=drawColor,rotate= 90.00,anchor=base,inner sep=0pt, outer sep=0pt, scale=  1.10] at ( 37.74,415.24) {1-(Best/Algorithm)};
\end{scope}
\begin{scope}
\path[clip] (  0.00,  0.00) rectangle (505.89,650.43);
\definecolor[named]{drawColor}{rgb}{0.00,0.00,0.00}

\node[text=drawColor,anchor=base,inner sep=0pt, outer sep=0pt, scale=  1.20] at (150.07,473.56) {$k=8$};
\end{scope}
\begin{scope}
\path[clip] (267.11,325.21) rectangle (491.72,487.82);
\definecolor[named]{drawColor}{rgb}{1.00,1.00,1.00}
\definecolor[named]{fillColor}{rgb}{1.00,1.00,1.00}

\path[draw=drawColor,line width= 0.6pt,line join=round,line cap=round,fill=fillColor] (267.11,325.21) rectangle (491.72,487.82);
\end{scope}
\begin{scope}
\path[clip] (320.31,364.12) rectangle (485.72,466.36);
\definecolor[named]{fillColor}{rgb}{1.00,1.00,1.00}

\path[fill=fillColor] (320.31,364.12) rectangle (485.72,466.36);
\definecolor[named]{drawColor}{rgb}{0.98,0.98,0.98}

\path[draw=drawColor,line width= 0.6pt,line join=round] (320.31,381.36) --
	(485.72,381.36);

\path[draw=drawColor,line width= 0.6pt,line join=round] (320.31,396.64) --
	(485.72,396.64);

\path[draw=drawColor,line width= 0.6pt,line join=round] (320.31,406.96) --
	(485.72,406.96);

\path[draw=drawColor,line width= 0.6pt,line join=round] (320.31,415.93) --
	(485.72,415.93);

\path[draw=drawColor,line width= 0.6pt,line join=round] (320.31,427.24) --
	(485.72,427.24);

\path[draw=drawColor,line width= 0.6pt,line join=round] (320.31,437.97) --
	(485.72,437.97);

\path[draw=drawColor,line width= 0.6pt,line join=round] (320.31,445.92) --
	(485.72,445.92);

\path[draw=drawColor,line width= 0.6pt,line join=round] (320.31,452.41) --
	(485.72,452.41);

\path[draw=drawColor,line width= 0.6pt,line join=round] (320.31,464.32) --
	(485.72,464.32);

\path[draw=drawColor,line width= 0.6pt,line join=round] (382.41,364.12) --
	(382.41,466.36);

\path[draw=drawColor,line width= 0.6pt,line join=round] (395.94,364.12) --
	(395.94,466.36);

\path[draw=drawColor,line width= 0.6pt,line join=round] (422.98,364.12) --
	(422.98,466.36);

\path[draw=drawColor,line width= 0.6pt,line join=round] (445.99,364.12) --
	(445.99,466.36);

\path[draw=drawColor,line width= 0.6pt,line join=round] (463.03,364.12) --
	(463.03,466.36);

\path[draw=drawColor,line width= 0.6pt,line join=round] (476.96,364.12) --
	(476.96,466.36);
\definecolor[named]{drawColor}{rgb}{0.75,0.75,0.75}

\path[draw=drawColor,line width= 0.6pt,dash pattern=on 1pt off 3pt ,line join=round] (320.31,372.42) --
	(485.72,372.42);

\path[draw=drawColor,line width= 0.6pt,dash pattern=on 1pt off 3pt ,line join=round] (320.31,390.29) --
	(485.72,390.29);

\path[draw=drawColor,line width= 0.6pt,dash pattern=on 1pt off 3pt ,line join=round] (320.31,402.98) --
	(485.72,402.98);

\path[draw=drawColor,line width= 0.6pt,dash pattern=on 1pt off 3pt ,line join=round] (320.31,410.93) --
	(485.72,410.93);

\path[draw=drawColor,line width= 0.6pt,dash pattern=on 1pt off 3pt ,line join=round] (320.31,420.94) --
	(485.72,420.94);

\path[draw=drawColor,line width= 0.6pt,dash pattern=on 1pt off 3pt ,line join=round] (320.31,433.55) --
	(485.72,433.55);

\path[draw=drawColor,line width= 0.6pt,dash pattern=on 1pt off 3pt ,line join=round] (320.31,442.39) --
	(485.72,442.39);

\path[draw=drawColor,line width= 0.6pt,dash pattern=on 1pt off 3pt ,line join=round] (320.31,449.44) --
	(485.72,449.44);

\path[draw=drawColor,line width= 0.6pt,dash pattern=on 1pt off 3pt ,line join=round] (320.31,455.38) --
	(485.72,455.38);

\path[draw=drawColor,line width= 0.6pt,dash pattern=on 1pt off 3pt ,line join=round] (409.46,364.12) --
	(409.46,466.36);

\path[draw=drawColor,line width= 0.6pt,dash pattern=on 1pt off 3pt ,line join=round] (436.50,364.12) --
	(436.50,466.36);

\path[draw=drawColor,line width= 0.6pt,dash pattern=on 1pt off 3pt ,line join=round] (455.48,364.12) --
	(455.48,466.36);

\path[draw=drawColor,line width= 0.6pt,dash pattern=on 1pt off 3pt ,line join=round] (470.58,364.12) --
	(470.58,466.36);

\path[draw=drawColor,line width= 0.6pt,dash pattern=on 1pt off 3pt ,line join=round] (483.33,364.12) --
	(483.33,466.36);
\definecolor[named]{drawColor}{rgb}{0.89,0.10,0.11}
\definecolor[named]{fillColor}{rgb}{0.89,0.10,0.11}

\path[draw=drawColor,line width= 0.4pt,line join=round,line cap=round,fill=fillColor] (327.82,429.82) circle (  1.16);

\path[draw=drawColor,line width= 0.4pt,line join=round,line cap=round,fill=fillColor] (333.65,426.80) circle (  1.16);

\path[draw=drawColor,line width= 0.4pt,line join=round,line cap=round,fill=fillColor] (337.74,420.80) circle (  1.16);

\path[draw=drawColor,line width= 0.4pt,line join=round,line cap=round,fill=fillColor] (340.99,419.52) circle (  1.16);

\path[draw=drawColor,line width= 0.4pt,line join=round,line cap=round,fill=fillColor] (343.74,419.19) circle (  1.16);

\path[draw=drawColor,line width= 0.4pt,line join=round,line cap=round,fill=fillColor] (346.14,417.60) circle (  1.16);

\path[draw=drawColor,line width= 0.4pt,line join=round,line cap=round,fill=fillColor] (348.29,417.05) circle (  1.16);

\path[draw=drawColor,line width= 0.4pt,line join=round,line cap=round,fill=fillColor] (350.24,416.90) circle (  1.16);

\path[draw=drawColor,line width= 0.4pt,line join=round,line cap=round,fill=fillColor] (352.04,416.75) circle (  1.16);

\path[draw=drawColor,line width= 0.4pt,line join=round,line cap=round,fill=fillColor] (353.70,416.32) circle (  1.16);

\path[draw=drawColor,line width= 0.4pt,line join=round,line cap=round,fill=fillColor] (355.26,414.66) circle (  1.16);

\path[draw=drawColor,line width= 0.4pt,line join=round,line cap=round,fill=fillColor] (356.73,414.02) circle (  1.16);

\path[draw=drawColor,line width= 0.4pt,line join=round,line cap=round,fill=fillColor] (358.12,412.56) circle (  1.16);

\path[draw=drawColor,line width= 0.4pt,line join=round,line cap=round,fill=fillColor] (359.44,412.34) circle (  1.16);

\path[draw=drawColor,line width= 0.4pt,line join=round,line cap=round,fill=fillColor] (360.69,412.31) circle (  1.16);

\path[draw=drawColor,line width= 0.4pt,line join=round,line cap=round,fill=fillColor] (361.90,412.14) circle (  1.16);

\path[draw=drawColor,line width= 0.4pt,line join=round,line cap=round,fill=fillColor] (363.05,412.00) circle (  1.16);

\path[draw=drawColor,line width= 0.4pt,line join=round,line cap=round,fill=fillColor] (364.16,411.68) circle (  1.16);

\path[draw=drawColor,line width= 0.4pt,line join=round,line cap=round,fill=fillColor] (365.23,411.65) circle (  1.16);

\path[draw=drawColor,line width= 0.4pt,line join=round,line cap=round,fill=fillColor] (366.26,411.35) circle (  1.16);

\path[draw=drawColor,line width= 0.4pt,line join=round,line cap=round,fill=fillColor] (367.25,411.30) circle (  1.16);

\path[draw=drawColor,line width= 0.4pt,line join=round,line cap=round,fill=fillColor] (368.22,411.24) circle (  1.16);

\path[draw=drawColor,line width= 0.4pt,line join=round,line cap=round,fill=fillColor] (369.16,410.98) circle (  1.16);

\path[draw=drawColor,line width= 0.4pt,line join=round,line cap=round,fill=fillColor] (370.07,410.49) circle (  1.16);

\path[draw=drawColor,line width= 0.4pt,line join=round,line cap=round,fill=fillColor] (370.96,410.49) circle (  1.16);

\path[draw=drawColor,line width= 0.4pt,line join=round,line cap=round,fill=fillColor] (371.82,410.41) circle (  1.16);

\path[draw=drawColor,line width= 0.4pt,line join=round,line cap=round,fill=fillColor] (372.66,410.21) circle (  1.16);

\path[draw=drawColor,line width= 0.4pt,line join=round,line cap=round,fill=fillColor] (373.48,409.91) circle (  1.16);

\path[draw=drawColor,line width= 0.4pt,line join=round,line cap=round,fill=fillColor] (374.28,409.89) circle (  1.16);

\path[draw=drawColor,line width= 0.4pt,line join=round,line cap=round,fill=fillColor] (375.06,409.77) circle (  1.16);

\path[draw=drawColor,line width= 0.4pt,line join=round,line cap=round,fill=fillColor] (375.83,409.77) circle (  1.16);

\path[draw=drawColor,line width= 0.4pt,line join=round,line cap=round,fill=fillColor] (376.58,409.64) circle (  1.16);

\path[draw=drawColor,line width= 0.4pt,line join=round,line cap=round,fill=fillColor] (377.31,409.40) circle (  1.16);

\path[draw=drawColor,line width= 0.4pt,line join=round,line cap=round,fill=fillColor] (378.03,408.47) circle (  1.16);

\path[draw=drawColor,line width= 0.4pt,line join=round,line cap=round,fill=fillColor] (378.74,408.47) circle (  1.16);

\path[draw=drawColor,line width= 0.4pt,line join=round,line cap=round,fill=fillColor] (379.43,408.38) circle (  1.16);

\path[draw=drawColor,line width= 0.4pt,line join=round,line cap=round,fill=fillColor] (380.11,408.29) circle (  1.16);

\path[draw=drawColor,line width= 0.4pt,line join=round,line cap=round,fill=fillColor] (380.77,408.20) circle (  1.16);

\path[draw=drawColor,line width= 0.4pt,line join=round,line cap=round,fill=fillColor] (381.43,408.15) circle (  1.16);

\path[draw=drawColor,line width= 0.4pt,line join=round,line cap=round,fill=fillColor] (382.07,407.50) circle (  1.16);

\path[draw=drawColor,line width= 0.4pt,line join=round,line cap=round,fill=fillColor] (382.71,407.26) circle (  1.16);

\path[draw=drawColor,line width= 0.4pt,line join=round,line cap=round,fill=fillColor] (383.33,407.03) circle (  1.16);

\path[draw=drawColor,line width= 0.4pt,line join=round,line cap=round,fill=fillColor] (383.94,406.97) circle (  1.16);

\path[draw=drawColor,line width= 0.4pt,line join=round,line cap=round,fill=fillColor] (384.55,406.91) circle (  1.16);

\path[draw=drawColor,line width= 0.4pt,line join=round,line cap=round,fill=fillColor] (385.14,406.87) circle (  1.16);

\path[draw=drawColor,line width= 0.4pt,line join=round,line cap=round,fill=fillColor] (385.73,406.74) circle (  1.16);

\path[draw=drawColor,line width= 0.4pt,line join=round,line cap=round,fill=fillColor] (386.31,406.73) circle (  1.16);

\path[draw=drawColor,line width= 0.4pt,line join=round,line cap=round,fill=fillColor] (386.88,406.44) circle (  1.16);

\path[draw=drawColor,line width= 0.4pt,line join=round,line cap=round,fill=fillColor] (387.44,406.23) circle (  1.16);

\path[draw=drawColor,line width= 0.4pt,line join=round,line cap=round,fill=fillColor] (387.99,406.06) circle (  1.16);

\path[draw=drawColor,line width= 0.4pt,line join=round,line cap=round,fill=fillColor] (388.54,406.04) circle (  1.16);

\path[draw=drawColor,line width= 0.4pt,line join=round,line cap=round,fill=fillColor] (389.08,406.03) circle (  1.16);

\path[draw=drawColor,line width= 0.4pt,line join=round,line cap=round,fill=fillColor] (389.61,406.01) circle (  1.16);

\path[draw=drawColor,line width= 0.4pt,line join=round,line cap=round,fill=fillColor] (390.14,405.91) circle (  1.16);

\path[draw=drawColor,line width= 0.4pt,line join=round,line cap=round,fill=fillColor] (390.66,405.86) circle (  1.16);

\path[draw=drawColor,line width= 0.4pt,line join=round,line cap=round,fill=fillColor] (391.17,405.67) circle (  1.16);

\path[draw=drawColor,line width= 0.4pt,line join=round,line cap=round,fill=fillColor] (391.68,405.60) circle (  1.16);

\path[draw=drawColor,line width= 0.4pt,line join=round,line cap=round,fill=fillColor] (392.18,405.44) circle (  1.16);

\path[draw=drawColor,line width= 0.4pt,line join=round,line cap=round,fill=fillColor] (392.68,405.36) circle (  1.16);

\path[draw=drawColor,line width= 0.4pt,line join=round,line cap=round,fill=fillColor] (393.17,405.32) circle (  1.16);

\path[draw=drawColor,line width= 0.4pt,line join=round,line cap=round,fill=fillColor] (393.65,405.30) circle (  1.16);

\path[draw=drawColor,line width= 0.4pt,line join=round,line cap=round,fill=fillColor] (394.13,404.98) circle (  1.16);

\path[draw=drawColor,line width= 0.4pt,line join=round,line cap=round,fill=fillColor] (394.61,404.94) circle (  1.16);

\path[draw=drawColor,line width= 0.4pt,line join=round,line cap=round,fill=fillColor] (395.08,404.86) circle (  1.16);

\path[draw=drawColor,line width= 0.4pt,line join=round,line cap=round,fill=fillColor] (395.54,404.75) circle (  1.16);

\path[draw=drawColor,line width= 0.4pt,line join=round,line cap=round,fill=fillColor] (396.00,404.37) circle (  1.16);

\path[draw=drawColor,line width= 0.4pt,line join=round,line cap=round,fill=fillColor] (396.46,404.26) circle (  1.16);

\path[draw=drawColor,line width= 0.4pt,line join=round,line cap=round,fill=fillColor] (396.91,404.12) circle (  1.16);

\path[draw=drawColor,line width= 0.4pt,line join=round,line cap=round,fill=fillColor] (397.35,403.72) circle (  1.16);

\path[draw=drawColor,line width= 0.4pt,line join=round,line cap=round,fill=fillColor] (397.80,403.67) circle (  1.16);

\path[draw=drawColor,line width= 0.4pt,line join=round,line cap=round,fill=fillColor] (398.23,403.52) circle (  1.16);

\path[draw=drawColor,line width= 0.4pt,line join=round,line cap=round,fill=fillColor] (398.67,403.51) circle (  1.16);

\path[draw=drawColor,line width= 0.4pt,line join=round,line cap=round,fill=fillColor] (399.10,403.37) circle (  1.16);

\path[draw=drawColor,line width= 0.4pt,line join=round,line cap=round,fill=fillColor] (399.52,403.37) circle (  1.16);

\path[draw=drawColor,line width= 0.4pt,line join=round,line cap=round,fill=fillColor] (399.94,403.36) circle (  1.16);

\path[draw=drawColor,line width= 0.4pt,line join=round,line cap=round,fill=fillColor] (400.36,403.34) circle (  1.16);

\path[draw=drawColor,line width= 0.4pt,line join=round,line cap=round,fill=fillColor] (400.78,403.23) circle (  1.16);

\path[draw=drawColor,line width= 0.4pt,line join=round,line cap=round,fill=fillColor] (401.19,403.12) circle (  1.16);

\path[draw=drawColor,line width= 0.4pt,line join=round,line cap=round,fill=fillColor] (401.60,403.02) circle (  1.16);

\path[draw=drawColor,line width= 0.4pt,line join=round,line cap=round,fill=fillColor] (402.00,402.64) circle (  1.16);

\path[draw=drawColor,line width= 0.4pt,line join=round,line cap=round,fill=fillColor] (402.40,402.64) circle (  1.16);

\path[draw=drawColor,line width= 0.4pt,line join=round,line cap=round,fill=fillColor] (402.80,402.62) circle (  1.16);

\path[draw=drawColor,line width= 0.4pt,line join=round,line cap=round,fill=fillColor] (403.19,402.59) circle (  1.16);

\path[draw=drawColor,line width= 0.4pt,line join=round,line cap=round,fill=fillColor] (403.58,402.58) circle (  1.16);

\path[draw=drawColor,line width= 0.4pt,line join=round,line cap=round,fill=fillColor] (403.97,402.53) circle (  1.16);

\path[draw=drawColor,line width= 0.4pt,line join=round,line cap=round,fill=fillColor] (404.36,402.25) circle (  1.16);

\path[draw=drawColor,line width= 0.4pt,line join=round,line cap=round,fill=fillColor] (404.74,402.04) circle (  1.16);

\path[draw=drawColor,line width= 0.4pt,line join=round,line cap=round,fill=fillColor] (405.12,401.97) circle (  1.16);

\path[draw=drawColor,line width= 0.4pt,line join=round,line cap=round,fill=fillColor] (405.49,401.94) circle (  1.16);

\path[draw=drawColor,line width= 0.4pt,line join=round,line cap=round,fill=fillColor] (405.87,401.67) circle (  1.16);

\path[draw=drawColor,line width= 0.4pt,line join=round,line cap=round,fill=fillColor] (406.24,401.53) circle (  1.16);

\path[draw=drawColor,line width= 0.4pt,line join=round,line cap=round,fill=fillColor] (406.61,401.49) circle (  1.16);

\path[draw=drawColor,line width= 0.4pt,line join=round,line cap=round,fill=fillColor] (406.97,401.49) circle (  1.16);

\path[draw=drawColor,line width= 0.4pt,line join=round,line cap=round,fill=fillColor] (407.34,401.40) circle (  1.16);

\path[draw=drawColor,line width= 0.4pt,line join=round,line cap=round,fill=fillColor] (407.70,401.37) circle (  1.16);

\path[draw=drawColor,line width= 0.4pt,line join=round,line cap=round,fill=fillColor] (408.05,401.35) circle (  1.16);

\path[draw=drawColor,line width= 0.4pt,line join=round,line cap=round,fill=fillColor] (408.41,401.32) circle (  1.16);

\path[draw=drawColor,line width= 0.4pt,line join=round,line cap=round,fill=fillColor] (408.76,401.23) circle (  1.16);

\path[draw=drawColor,line width= 0.4pt,line join=round,line cap=round,fill=fillColor] (409.11,401.16) circle (  1.16);

\path[draw=drawColor,line width= 0.4pt,line join=round,line cap=round,fill=fillColor] (409.46,401.11) circle (  1.16);

\path[draw=drawColor,line width= 0.4pt,line join=round,line cap=round,fill=fillColor] (409.80,401.05) circle (  1.16);

\path[draw=drawColor,line width= 0.4pt,line join=round,line cap=round,fill=fillColor] (410.15,400.98) circle (  1.16);

\path[draw=drawColor,line width= 0.4pt,line join=round,line cap=round,fill=fillColor] (410.49,400.87) circle (  1.16);

\path[draw=drawColor,line width= 0.4pt,line join=round,line cap=round,fill=fillColor] (410.83,400.80) circle (  1.16);

\path[draw=drawColor,line width= 0.4pt,line join=round,line cap=round,fill=fillColor] (411.17,400.77) circle (  1.16);

\path[draw=drawColor,line width= 0.4pt,line join=round,line cap=round,fill=fillColor] (411.50,400.73) circle (  1.16);

\path[draw=drawColor,line width= 0.4pt,line join=round,line cap=round,fill=fillColor] (411.83,400.71) circle (  1.16);

\path[draw=drawColor,line width= 0.4pt,line join=round,line cap=round,fill=fillColor] (412.16,400.66) circle (  1.16);

\path[draw=drawColor,line width= 0.4pt,line join=round,line cap=round,fill=fillColor] (412.49,400.56) circle (  1.16);

\path[draw=drawColor,line width= 0.4pt,line join=round,line cap=round,fill=fillColor] (412.82,400.51) circle (  1.16);

\path[draw=drawColor,line width= 0.4pt,line join=round,line cap=round,fill=fillColor] (413.14,400.49) circle (  1.16);

\path[draw=drawColor,line width= 0.4pt,line join=round,line cap=round,fill=fillColor] (413.47,400.45) circle (  1.16);

\path[draw=drawColor,line width= 0.4pt,line join=round,line cap=round,fill=fillColor] (413.79,400.45) circle (  1.16);

\path[draw=drawColor,line width= 0.4pt,line join=round,line cap=round,fill=fillColor] (414.10,400.37) circle (  1.16);

\path[draw=drawColor,line width= 0.4pt,line join=round,line cap=round,fill=fillColor] (414.42,400.33) circle (  1.16);

\path[draw=drawColor,line width= 0.4pt,line join=round,line cap=round,fill=fillColor] (414.74,400.26) circle (  1.16);

\path[draw=drawColor,line width= 0.4pt,line join=round,line cap=round,fill=fillColor] (415.05,400.13) circle (  1.16);

\path[draw=drawColor,line width= 0.4pt,line join=round,line cap=round,fill=fillColor] (415.36,400.11) circle (  1.16);

\path[draw=drawColor,line width= 0.4pt,line join=round,line cap=round,fill=fillColor] (415.67,400.05) circle (  1.16);

\path[draw=drawColor,line width= 0.4pt,line join=round,line cap=round,fill=fillColor] (415.98,400.00) circle (  1.16);

\path[draw=drawColor,line width= 0.4pt,line join=round,line cap=round,fill=fillColor] (416.29,399.85) circle (  1.16);

\path[draw=drawColor,line width= 0.4pt,line join=round,line cap=round,fill=fillColor] (416.59,399.83) circle (  1.16);

\path[draw=drawColor,line width= 0.4pt,line join=round,line cap=round,fill=fillColor] (416.89,399.77) circle (  1.16);

\path[draw=drawColor,line width= 0.4pt,line join=round,line cap=round,fill=fillColor] (417.19,399.72) circle (  1.16);

\path[draw=drawColor,line width= 0.4pt,line join=round,line cap=round,fill=fillColor] (417.49,399.63) circle (  1.16);

\path[draw=drawColor,line width= 0.4pt,line join=round,line cap=round,fill=fillColor] (417.79,399.61) circle (  1.16);

\path[draw=drawColor,line width= 0.4pt,line join=round,line cap=round,fill=fillColor] (418.09,399.61) circle (  1.16);

\path[draw=drawColor,line width= 0.4pt,line join=round,line cap=round,fill=fillColor] (418.38,399.51) circle (  1.16);

\path[draw=drawColor,line width= 0.4pt,line join=round,line cap=round,fill=fillColor] (418.68,399.50) circle (  1.16);

\path[draw=drawColor,line width= 0.4pt,line join=round,line cap=round,fill=fillColor] (418.97,399.33) circle (  1.16);

\path[draw=drawColor,line width= 0.4pt,line join=round,line cap=round,fill=fillColor] (419.26,399.32) circle (  1.16);

\path[draw=drawColor,line width= 0.4pt,line join=round,line cap=round,fill=fillColor] (419.55,399.24) circle (  1.16);

\path[draw=drawColor,line width= 0.4pt,line join=round,line cap=round,fill=fillColor] (419.84,399.24) circle (  1.16);

\path[draw=drawColor,line width= 0.4pt,line join=round,line cap=round,fill=fillColor] (420.12,399.23) circle (  1.16);

\path[draw=drawColor,line width= 0.4pt,line join=round,line cap=round,fill=fillColor] (420.41,399.14) circle (  1.16);

\path[draw=drawColor,line width= 0.4pt,line join=round,line cap=round,fill=fillColor] (420.69,399.14) circle (  1.16);

\path[draw=drawColor,line width= 0.4pt,line join=round,line cap=round,fill=fillColor] (420.97,399.13) circle (  1.16);

\path[draw=drawColor,line width= 0.4pt,line join=round,line cap=round,fill=fillColor] (421.25,399.13) circle (  1.16);

\path[draw=drawColor,line width= 0.4pt,line join=round,line cap=round,fill=fillColor] (421.53,399.08) circle (  1.16);

\path[draw=drawColor,line width= 0.4pt,line join=round,line cap=round,fill=fillColor] (421.81,399.01) circle (  1.16);

\path[draw=drawColor,line width= 0.4pt,line join=round,line cap=round,fill=fillColor] (422.09,399.00) circle (  1.16);

\path[draw=drawColor,line width= 0.4pt,line join=round,line cap=round,fill=fillColor] (422.36,398.98) circle (  1.16);

\path[draw=drawColor,line width= 0.4pt,line join=round,line cap=round,fill=fillColor] (422.63,398.92) circle (  1.16);

\path[draw=drawColor,line width= 0.4pt,line join=round,line cap=round,fill=fillColor] (422.91,398.87) circle (  1.16);

\path[draw=drawColor,line width= 0.4pt,line join=round,line cap=round,fill=fillColor] (423.18,398.85) circle (  1.16);

\path[draw=drawColor,line width= 0.4pt,line join=round,line cap=round,fill=fillColor] (423.45,398.76) circle (  1.16);

\path[draw=drawColor,line width= 0.4pt,line join=round,line cap=round,fill=fillColor] (423.72,398.75) circle (  1.16);

\path[draw=drawColor,line width= 0.4pt,line join=round,line cap=round,fill=fillColor] (423.99,398.75) circle (  1.16);

\path[draw=drawColor,line width= 0.4pt,line join=round,line cap=round,fill=fillColor] (424.25,398.74) circle (  1.16);

\path[draw=drawColor,line width= 0.4pt,line join=round,line cap=round,fill=fillColor] (424.52,398.74) circle (  1.16);

\path[draw=drawColor,line width= 0.4pt,line join=round,line cap=round,fill=fillColor] (424.78,398.73) circle (  1.16);

\path[draw=drawColor,line width= 0.4pt,line join=round,line cap=round,fill=fillColor] (425.04,398.68) circle (  1.16);

\path[draw=drawColor,line width= 0.4pt,line join=round,line cap=round,fill=fillColor] (425.31,398.60) circle (  1.16);

\path[draw=drawColor,line width= 0.4pt,line join=round,line cap=round,fill=fillColor] (425.57,398.59) circle (  1.16);

\path[draw=drawColor,line width= 0.4pt,line join=round,line cap=round,fill=fillColor] (425.83,398.56) circle (  1.16);

\path[draw=drawColor,line width= 0.4pt,line join=round,line cap=round,fill=fillColor] (426.08,398.53) circle (  1.16);

\path[draw=drawColor,line width= 0.4pt,line join=round,line cap=round,fill=fillColor] (426.34,398.50) circle (  1.16);

\path[draw=drawColor,line width= 0.4pt,line join=round,line cap=round,fill=fillColor] (426.60,398.47) circle (  1.16);

\path[draw=drawColor,line width= 0.4pt,line join=round,line cap=round,fill=fillColor] (426.85,398.36) circle (  1.16);

\path[draw=drawColor,line width= 0.4pt,line join=round,line cap=round,fill=fillColor] (427.11,398.32) circle (  1.16);

\path[draw=drawColor,line width= 0.4pt,line join=round,line cap=round,fill=fillColor] (427.36,398.27) circle (  1.16);

\path[draw=drawColor,line width= 0.4pt,line join=round,line cap=round,fill=fillColor] (427.61,398.20) circle (  1.16);

\path[draw=drawColor,line width= 0.4pt,line join=round,line cap=round,fill=fillColor] (427.86,398.18) circle (  1.16);

\path[draw=drawColor,line width= 0.4pt,line join=round,line cap=round,fill=fillColor] (428.11,398.06) circle (  1.16);

\path[draw=drawColor,line width= 0.4pt,line join=round,line cap=round,fill=fillColor] (428.36,398.01) circle (  1.16);

\path[draw=drawColor,line width= 0.4pt,line join=round,line cap=round,fill=fillColor] (428.61,397.88) circle (  1.16);

\path[draw=drawColor,line width= 0.4pt,line join=round,line cap=round,fill=fillColor] (428.86,397.84) circle (  1.16);

\path[draw=drawColor,line width= 0.4pt,line join=round,line cap=round,fill=fillColor] (429.10,397.79) circle (  1.16);

\path[draw=drawColor,line width= 0.4pt,line join=round,line cap=round,fill=fillColor] (429.35,397.79) circle (  1.16);

\path[draw=drawColor,line width= 0.4pt,line join=round,line cap=round,fill=fillColor] (429.59,397.76) circle (  1.16);

\path[draw=drawColor,line width= 0.4pt,line join=round,line cap=round,fill=fillColor] (429.83,397.70) circle (  1.16);

\path[draw=drawColor,line width= 0.4pt,line join=round,line cap=round,fill=fillColor] (430.08,397.65) circle (  1.16);

\path[draw=drawColor,line width= 0.4pt,line join=round,line cap=round,fill=fillColor] (430.32,397.61) circle (  1.16);

\path[draw=drawColor,line width= 0.4pt,line join=round,line cap=round,fill=fillColor] (430.56,397.60) circle (  1.16);

\path[draw=drawColor,line width= 0.4pt,line join=round,line cap=round,fill=fillColor] (430.80,397.60) circle (  1.16);

\path[draw=drawColor,line width= 0.4pt,line join=round,line cap=round,fill=fillColor] (431.04,397.51) circle (  1.16);

\path[draw=drawColor,line width= 0.4pt,line join=round,line cap=round,fill=fillColor] (431.27,397.49) circle (  1.16);

\path[draw=drawColor,line width= 0.4pt,line join=round,line cap=round,fill=fillColor] (431.51,397.44) circle (  1.16);

\path[draw=drawColor,line width= 0.4pt,line join=round,line cap=round,fill=fillColor] (431.75,397.40) circle (  1.16);

\path[draw=drawColor,line width= 0.4pt,line join=round,line cap=round,fill=fillColor] (431.98,397.38) circle (  1.16);

\path[draw=drawColor,line width= 0.4pt,line join=round,line cap=round,fill=fillColor] (432.21,397.38) circle (  1.16);

\path[draw=drawColor,line width= 0.4pt,line join=round,line cap=round,fill=fillColor] (432.45,397.37) circle (  1.16);

\path[draw=drawColor,line width= 0.4pt,line join=round,line cap=round,fill=fillColor] (432.68,397.32) circle (  1.16);

\path[draw=drawColor,line width= 0.4pt,line join=round,line cap=round,fill=fillColor] (432.91,397.14) circle (  1.16);

\path[draw=drawColor,line width= 0.4pt,line join=round,line cap=round,fill=fillColor] (433.14,397.09) circle (  1.16);

\path[draw=drawColor,line width= 0.4pt,line join=round,line cap=round,fill=fillColor] (433.37,397.02) circle (  1.16);

\path[draw=drawColor,line width= 0.4pt,line join=round,line cap=round,fill=fillColor] (433.60,397.01) circle (  1.16);

\path[draw=drawColor,line width= 0.4pt,line join=round,line cap=round,fill=fillColor] (433.83,396.95) circle (  1.16);

\path[draw=drawColor,line width= 0.4pt,line join=round,line cap=round,fill=fillColor] (434.06,396.92) circle (  1.16);

\path[draw=drawColor,line width= 0.4pt,line join=round,line cap=round,fill=fillColor] (434.28,396.91) circle (  1.16);

\path[draw=drawColor,line width= 0.4pt,line join=round,line cap=round,fill=fillColor] (434.51,396.83) circle (  1.16);

\path[draw=drawColor,line width= 0.4pt,line join=round,line cap=round,fill=fillColor] (434.73,396.63) circle (  1.16);

\path[draw=drawColor,line width= 0.4pt,line join=round,line cap=round,fill=fillColor] (434.96,396.54) circle (  1.16);

\path[draw=drawColor,line width= 0.4pt,line join=round,line cap=round,fill=fillColor] (435.18,396.47) circle (  1.16);

\path[draw=drawColor,line width= 0.4pt,line join=round,line cap=round,fill=fillColor] (435.40,396.41) circle (  1.16);

\path[draw=drawColor,line width= 0.4pt,line join=round,line cap=round,fill=fillColor] (435.62,396.40) circle (  1.16);

\path[draw=drawColor,line width= 0.4pt,line join=round,line cap=round,fill=fillColor] (435.85,396.36) circle (  1.16);

\path[draw=drawColor,line width= 0.4pt,line join=round,line cap=round,fill=fillColor] (436.07,396.15) circle (  1.16);

\path[draw=drawColor,line width= 0.4pt,line join=round,line cap=round,fill=fillColor] (436.29,396.09) circle (  1.16);

\path[draw=drawColor,line width= 0.4pt,line join=round,line cap=round,fill=fillColor] (436.50,396.05) circle (  1.16);

\path[draw=drawColor,line width= 0.4pt,line join=round,line cap=round,fill=fillColor] (436.72,396.04) circle (  1.16);

\path[draw=drawColor,line width= 0.4pt,line join=round,line cap=round,fill=fillColor] (436.94,395.95) circle (  1.16);

\path[draw=drawColor,line width= 0.4pt,line join=round,line cap=round,fill=fillColor] (437.16,395.91) circle (  1.16);

\path[draw=drawColor,line width= 0.4pt,line join=round,line cap=round,fill=fillColor] (437.37,395.90) circle (  1.16);

\path[draw=drawColor,line width= 0.4pt,line join=round,line cap=round,fill=fillColor] (437.59,395.88) circle (  1.16);

\path[draw=drawColor,line width= 0.4pt,line join=round,line cap=round,fill=fillColor] (437.80,395.87) circle (  1.16);

\path[draw=drawColor,line width= 0.4pt,line join=round,line cap=round,fill=fillColor] (438.02,395.87) circle (  1.16);

\path[draw=drawColor,line width= 0.4pt,line join=round,line cap=round,fill=fillColor] (438.23,395.84) circle (  1.16);

\path[draw=drawColor,line width= 0.4pt,line join=round,line cap=round,fill=fillColor] (438.44,395.84) circle (  1.16);

\path[draw=drawColor,line width= 0.4pt,line join=round,line cap=round,fill=fillColor] (438.65,395.83) circle (  1.16);

\path[draw=drawColor,line width= 0.4pt,line join=round,line cap=round,fill=fillColor] (438.87,395.82) circle (  1.16);

\path[draw=drawColor,line width= 0.4pt,line join=round,line cap=round,fill=fillColor] (439.08,395.80) circle (  1.16);

\path[draw=drawColor,line width= 0.4pt,line join=round,line cap=round,fill=fillColor] (439.29,395.65) circle (  1.16);

\path[draw=drawColor,line width= 0.4pt,line join=round,line cap=round,fill=fillColor] (439.49,395.64) circle (  1.16);

\path[draw=drawColor,line width= 0.4pt,line join=round,line cap=round,fill=fillColor] (439.70,395.61) circle (  1.16);

\path[draw=drawColor,line width= 0.4pt,line join=round,line cap=round,fill=fillColor] (439.91,395.56) circle (  1.16);

\path[draw=drawColor,line width= 0.4pt,line join=round,line cap=round,fill=fillColor] (440.12,395.52) circle (  1.16);

\path[draw=drawColor,line width= 0.4pt,line join=round,line cap=round,fill=fillColor] (440.33,395.51) circle (  1.16);

\path[draw=drawColor,line width= 0.4pt,line join=round,line cap=round,fill=fillColor] (440.53,395.39) circle (  1.16);

\path[draw=drawColor,line width= 0.4pt,line join=round,line cap=round,fill=fillColor] (440.74,395.38) circle (  1.16);

\path[draw=drawColor,line width= 0.4pt,line join=round,line cap=round,fill=fillColor] (440.94,395.34) circle (  1.16);

\path[draw=drawColor,line width= 0.4pt,line join=round,line cap=round,fill=fillColor] (441.15,395.32) circle (  1.16);

\path[draw=drawColor,line width= 0.4pt,line join=round,line cap=round,fill=fillColor] (441.35,395.15) circle (  1.16);

\path[draw=drawColor,line width= 0.4pt,line join=round,line cap=round,fill=fillColor] (441.55,395.01) circle (  1.16);

\path[draw=drawColor,line width= 0.4pt,line join=round,line cap=round,fill=fillColor] (441.75,395.00) circle (  1.16);

\path[draw=drawColor,line width= 0.4pt,line join=round,line cap=round,fill=fillColor] (441.96,394.99) circle (  1.16);

\path[draw=drawColor,line width= 0.4pt,line join=round,line cap=round,fill=fillColor] (442.16,394.96) circle (  1.16);

\path[draw=drawColor,line width= 0.4pt,line join=round,line cap=round,fill=fillColor] (442.36,394.83) circle (  1.16);

\path[draw=drawColor,line width= 0.4pt,line join=round,line cap=round,fill=fillColor] (442.56,394.78) circle (  1.16);

\path[draw=drawColor,line width= 0.4pt,line join=round,line cap=round,fill=fillColor] (442.76,394.74) circle (  1.16);

\path[draw=drawColor,line width= 0.4pt,line join=round,line cap=round,fill=fillColor] (442.96,394.68) circle (  1.16);

\path[draw=drawColor,line width= 0.4pt,line join=round,line cap=round,fill=fillColor] (443.15,394.66) circle (  1.16);

\path[draw=drawColor,line width= 0.4pt,line join=round,line cap=round,fill=fillColor] (443.35,394.61) circle (  1.16);

\path[draw=drawColor,line width= 0.4pt,line join=round,line cap=round,fill=fillColor] (443.55,394.59) circle (  1.16);

\path[draw=drawColor,line width= 0.4pt,line join=round,line cap=round,fill=fillColor] (443.74,394.56) circle (  1.16);

\path[draw=drawColor,line width= 0.4pt,line join=round,line cap=round,fill=fillColor] (443.94,394.54) circle (  1.16);

\path[draw=drawColor,line width= 0.4pt,line join=round,line cap=round,fill=fillColor] (444.14,394.52) circle (  1.16);

\path[draw=drawColor,line width= 0.4pt,line join=round,line cap=round,fill=fillColor] (444.33,394.44) circle (  1.16);

\path[draw=drawColor,line width= 0.4pt,line join=round,line cap=round,fill=fillColor] (444.53,394.44) circle (  1.16);

\path[draw=drawColor,line width= 0.4pt,line join=round,line cap=round,fill=fillColor] (444.72,394.40) circle (  1.16);

\path[draw=drawColor,line width= 0.4pt,line join=round,line cap=round,fill=fillColor] (444.91,394.37) circle (  1.16);

\path[draw=drawColor,line width= 0.4pt,line join=round,line cap=round,fill=fillColor] (445.10,394.29) circle (  1.16);

\path[draw=drawColor,line width= 0.4pt,line join=round,line cap=round,fill=fillColor] (445.30,394.19) circle (  1.16);

\path[draw=drawColor,line width= 0.4pt,line join=round,line cap=round,fill=fillColor] (445.49,394.19) circle (  1.16);

\path[draw=drawColor,line width= 0.4pt,line join=round,line cap=round,fill=fillColor] (445.68,394.13) circle (  1.16);

\path[draw=drawColor,line width= 0.4pt,line join=round,line cap=round,fill=fillColor] (445.87,394.12) circle (  1.16);

\path[draw=drawColor,line width= 0.4pt,line join=round,line cap=round,fill=fillColor] (446.06,394.10) circle (  1.16);

\path[draw=drawColor,line width= 0.4pt,line join=round,line cap=round,fill=fillColor] (446.25,393.98) circle (  1.16);

\path[draw=drawColor,line width= 0.4pt,line join=round,line cap=round,fill=fillColor] (446.44,393.96) circle (  1.16);

\path[draw=drawColor,line width= 0.4pt,line join=round,line cap=round,fill=fillColor] (446.63,393.95) circle (  1.16);

\path[draw=drawColor,line width= 0.4pt,line join=round,line cap=round,fill=fillColor] (446.82,393.92) circle (  1.16);

\path[draw=drawColor,line width= 0.4pt,line join=round,line cap=round,fill=fillColor] (447.00,393.88) circle (  1.16);

\path[draw=drawColor,line width= 0.4pt,line join=round,line cap=round,fill=fillColor] (447.19,393.85) circle (  1.16);

\path[draw=drawColor,line width= 0.4pt,line join=round,line cap=round,fill=fillColor] (447.38,393.85) circle (  1.16);

\path[draw=drawColor,line width= 0.4pt,line join=round,line cap=round,fill=fillColor] (447.56,393.81) circle (  1.16);

\path[draw=drawColor,line width= 0.4pt,line join=round,line cap=round,fill=fillColor] (447.75,393.74) circle (  1.16);

\path[draw=drawColor,line width= 0.4pt,line join=round,line cap=round,fill=fillColor] (447.93,393.74) circle (  1.16);

\path[draw=drawColor,line width= 0.4pt,line join=round,line cap=round,fill=fillColor] (448.12,393.73) circle (  1.16);

\path[draw=drawColor,line width= 0.4pt,line join=round,line cap=round,fill=fillColor] (448.30,393.73) circle (  1.16);

\path[draw=drawColor,line width= 0.4pt,line join=round,line cap=round,fill=fillColor] (448.49,393.71) circle (  1.16);

\path[draw=drawColor,line width= 0.4pt,line join=round,line cap=round,fill=fillColor] (448.67,393.55) circle (  1.16);

\path[draw=drawColor,line width= 0.4pt,line join=round,line cap=round,fill=fillColor] (448.85,393.32) circle (  1.16);

\path[draw=drawColor,line width= 0.4pt,line join=round,line cap=round,fill=fillColor] (449.03,393.28) circle (  1.16);

\path[draw=drawColor,line width= 0.4pt,line join=round,line cap=round,fill=fillColor] (449.22,393.27) circle (  1.16);

\path[draw=drawColor,line width= 0.4pt,line join=round,line cap=round,fill=fillColor] (449.40,393.23) circle (  1.16);

\path[draw=drawColor,line width= 0.4pt,line join=round,line cap=round,fill=fillColor] (449.58,393.20) circle (  1.16);

\path[draw=drawColor,line width= 0.4pt,line join=round,line cap=round,fill=fillColor] (449.76,393.04) circle (  1.16);

\path[draw=drawColor,line width= 0.4pt,line join=round,line cap=round,fill=fillColor] (449.94,393.00) circle (  1.16);

\path[draw=drawColor,line width= 0.4pt,line join=round,line cap=round,fill=fillColor] (450.12,392.99) circle (  1.16);

\path[draw=drawColor,line width= 0.4pt,line join=round,line cap=round,fill=fillColor] (450.30,392.95) circle (  1.16);

\path[draw=drawColor,line width= 0.4pt,line join=round,line cap=round,fill=fillColor] (450.48,392.91) circle (  1.16);

\path[draw=drawColor,line width= 0.4pt,line join=round,line cap=round,fill=fillColor] (450.65,392.88) circle (  1.16);

\path[draw=drawColor,line width= 0.4pt,line join=round,line cap=round,fill=fillColor] (450.83,392.77) circle (  1.16);

\path[draw=drawColor,line width= 0.4pt,line join=round,line cap=round,fill=fillColor] (451.01,392.76) circle (  1.16);

\path[draw=drawColor,line width= 0.4pt,line join=round,line cap=round,fill=fillColor] (451.19,392.70) circle (  1.16);

\path[draw=drawColor,line width= 0.4pt,line join=round,line cap=round,fill=fillColor] (451.36,392.69) circle (  1.16);

\path[draw=drawColor,line width= 0.4pt,line join=round,line cap=round,fill=fillColor] (451.54,392.67) circle (  1.16);

\path[draw=drawColor,line width= 0.4pt,line join=round,line cap=round,fill=fillColor] (451.71,392.56) circle (  1.16);

\path[draw=drawColor,line width= 0.4pt,line join=round,line cap=round,fill=fillColor] (451.89,392.46) circle (  1.16);

\path[draw=drawColor,line width= 0.4pt,line join=round,line cap=round,fill=fillColor] (452.06,392.45) circle (  1.16);

\path[draw=drawColor,line width= 0.4pt,line join=round,line cap=round,fill=fillColor] (452.24,392.44) circle (  1.16);

\path[draw=drawColor,line width= 0.4pt,line join=round,line cap=round,fill=fillColor] (452.41,392.35) circle (  1.16);

\path[draw=drawColor,line width= 0.4pt,line join=round,line cap=round,fill=fillColor] (452.59,392.25) circle (  1.16);

\path[draw=drawColor,line width= 0.4pt,line join=round,line cap=round,fill=fillColor] (452.76,392.23) circle (  1.16);

\path[draw=drawColor,line width= 0.4pt,line join=round,line cap=round,fill=fillColor] (452.93,392.22) circle (  1.16);

\path[draw=drawColor,line width= 0.4pt,line join=round,line cap=round,fill=fillColor] (453.10,392.07) circle (  1.16);

\path[draw=drawColor,line width= 0.4pt,line join=round,line cap=round,fill=fillColor] (453.28,392.07) circle (  1.16);

\path[draw=drawColor,line width= 0.4pt,line join=round,line cap=round,fill=fillColor] (453.45,391.83) circle (  1.16);

\path[draw=drawColor,line width= 0.4pt,line join=round,line cap=round,fill=fillColor] (453.62,391.82) circle (  1.16);

\path[draw=drawColor,line width= 0.4pt,line join=round,line cap=round,fill=fillColor] (453.79,391.81) circle (  1.16);

\path[draw=drawColor,line width= 0.4pt,line join=round,line cap=round,fill=fillColor] (453.96,391.79) circle (  1.16);

\path[draw=drawColor,line width= 0.4pt,line join=round,line cap=round,fill=fillColor] (454.13,391.75) circle (  1.16);

\path[draw=drawColor,line width= 0.4pt,line join=round,line cap=round,fill=fillColor] (454.30,391.62) circle (  1.16);

\path[draw=drawColor,line width= 0.4pt,line join=round,line cap=round,fill=fillColor] (454.47,391.35) circle (  1.16);

\path[draw=drawColor,line width= 0.4pt,line join=round,line cap=round,fill=fillColor] (454.64,391.26) circle (  1.16);

\path[draw=drawColor,line width= 0.4pt,line join=round,line cap=round,fill=fillColor] (454.81,391.20) circle (  1.16);

\path[draw=drawColor,line width= 0.4pt,line join=round,line cap=round,fill=fillColor] (454.97,391.18) circle (  1.16);

\path[draw=drawColor,line width= 0.4pt,line join=round,line cap=round,fill=fillColor] (455.14,391.16) circle (  1.16);

\path[draw=drawColor,line width= 0.4pt,line join=round,line cap=round,fill=fillColor] (455.31,391.16) circle (  1.16);

\path[draw=drawColor,line width= 0.4pt,line join=round,line cap=round,fill=fillColor] (455.48,391.12) circle (  1.16);

\path[draw=drawColor,line width= 0.4pt,line join=round,line cap=round,fill=fillColor] (455.64,391.02) circle (  1.16);

\path[draw=drawColor,line width= 0.4pt,line join=round,line cap=round,fill=fillColor] (455.81,390.95) circle (  1.16);

\path[draw=drawColor,line width= 0.4pt,line join=round,line cap=round,fill=fillColor] (455.97,390.92) circle (  1.16);

\path[draw=drawColor,line width= 0.4pt,line join=round,line cap=round,fill=fillColor] (456.14,390.90) circle (  1.16);

\path[draw=drawColor,line width= 0.4pt,line join=round,line cap=round,fill=fillColor] (456.31,390.83) circle (  1.16);

\path[draw=drawColor,line width= 0.4pt,line join=round,line cap=round,fill=fillColor] (456.47,390.81) circle (  1.16);

\path[draw=drawColor,line width= 0.4pt,line join=round,line cap=round,fill=fillColor] (456.63,390.80) circle (  1.16);

\path[draw=drawColor,line width= 0.4pt,line join=round,line cap=round,fill=fillColor] (456.80,390.78) circle (  1.16);

\path[draw=drawColor,line width= 0.4pt,line join=round,line cap=round,fill=fillColor] (456.96,390.75) circle (  1.16);

\path[draw=drawColor,line width= 0.4pt,line join=round,line cap=round,fill=fillColor] (457.13,390.71) circle (  1.16);

\path[draw=drawColor,line width= 0.4pt,line join=round,line cap=round,fill=fillColor] (457.29,390.66) circle (  1.16);

\path[draw=drawColor,line width= 0.4pt,line join=round,line cap=round,fill=fillColor] (457.45,390.65) circle (  1.16);

\path[draw=drawColor,line width= 0.4pt,line join=round,line cap=round,fill=fillColor] (457.61,390.64) circle (  1.16);

\path[draw=drawColor,line width= 0.4pt,line join=round,line cap=round,fill=fillColor] (457.78,390.62) circle (  1.16);

\path[draw=drawColor,line width= 0.4pt,line join=round,line cap=round,fill=fillColor] (457.94,390.61) circle (  1.16);

\path[draw=drawColor,line width= 0.4pt,line join=round,line cap=round,fill=fillColor] (458.10,390.59) circle (  1.16);

\path[draw=drawColor,line width= 0.4pt,line join=round,line cap=round,fill=fillColor] (458.26,390.46) circle (  1.16);

\path[draw=drawColor,line width= 0.4pt,line join=round,line cap=round,fill=fillColor] (458.42,390.45) circle (  1.16);

\path[draw=drawColor,line width= 0.4pt,line join=round,line cap=round,fill=fillColor] (458.58,390.42) circle (  1.16);

\path[draw=drawColor,line width= 0.4pt,line join=round,line cap=round,fill=fillColor] (458.74,390.41) circle (  1.16);

\path[draw=drawColor,line width= 0.4pt,line join=round,line cap=round,fill=fillColor] (458.90,390.35) circle (  1.16);

\path[draw=drawColor,line width= 0.4pt,line join=round,line cap=round,fill=fillColor] (459.06,390.34) circle (  1.16);

\path[draw=drawColor,line width= 0.4pt,line join=round,line cap=round,fill=fillColor] (459.22,390.32) circle (  1.16);

\path[draw=drawColor,line width= 0.4pt,line join=round,line cap=round,fill=fillColor] (459.38,390.31) circle (  1.16);

\path[draw=drawColor,line width= 0.4pt,line join=round,line cap=round,fill=fillColor] (459.53,390.27) circle (  1.16);

\path[draw=drawColor,line width= 0.4pt,line join=round,line cap=round,fill=fillColor] (459.69,390.26) circle (  1.16);

\path[draw=drawColor,line width= 0.4pt,line join=round,line cap=round,fill=fillColor] (459.85,390.21) circle (  1.16);

\path[draw=drawColor,line width= 0.4pt,line join=round,line cap=round,fill=fillColor] (460.01,390.11) circle (  1.16);

\path[draw=drawColor,line width= 0.4pt,line join=round,line cap=round,fill=fillColor] (460.16,390.10) circle (  1.16);

\path[draw=drawColor,line width= 0.4pt,line join=round,line cap=round,fill=fillColor] (460.32,389.94) circle (  1.16);

\path[draw=drawColor,line width= 0.4pt,line join=round,line cap=round,fill=fillColor] (460.48,389.88) circle (  1.16);

\path[draw=drawColor,line width= 0.4pt,line join=round,line cap=round,fill=fillColor] (460.63,389.79) circle (  1.16);

\path[draw=drawColor,line width= 0.4pt,line join=round,line cap=round,fill=fillColor] (460.79,389.59) circle (  1.16);

\path[draw=drawColor,line width= 0.4pt,line join=round,line cap=round,fill=fillColor] (460.94,389.58) circle (  1.16);

\path[draw=drawColor,line width= 0.4pt,line join=round,line cap=round,fill=fillColor] (461.10,389.53) circle (  1.16);

\path[draw=drawColor,line width= 0.4pt,line join=round,line cap=round,fill=fillColor] (461.25,389.41) circle (  1.16);

\path[draw=drawColor,line width= 0.4pt,line join=round,line cap=round,fill=fillColor] (461.41,389.38) circle (  1.16);

\path[draw=drawColor,line width= 0.4pt,line join=round,line cap=round,fill=fillColor] (461.56,389.36) circle (  1.16);

\path[draw=drawColor,line width= 0.4pt,line join=round,line cap=round,fill=fillColor] (461.72,389.34) circle (  1.16);

\path[draw=drawColor,line width= 0.4pt,line join=round,line cap=round,fill=fillColor] (461.87,389.34) circle (  1.16);

\path[draw=drawColor,line width= 0.4pt,line join=round,line cap=round,fill=fillColor] (462.02,389.31) circle (  1.16);

\path[draw=drawColor,line width= 0.4pt,line join=round,line cap=round,fill=fillColor] (462.18,389.13) circle (  1.16);

\path[draw=drawColor,line width= 0.4pt,line join=round,line cap=round,fill=fillColor] (462.33,389.12) circle (  1.16);

\path[draw=drawColor,line width= 0.4pt,line join=round,line cap=round,fill=fillColor] (462.48,389.00) circle (  1.16);

\path[draw=drawColor,line width= 0.4pt,line join=round,line cap=round,fill=fillColor] (462.63,388.94) circle (  1.16);

\path[draw=drawColor,line width= 0.4pt,line join=round,line cap=round,fill=fillColor] (462.78,388.92) circle (  1.16);

\path[draw=drawColor,line width= 0.4pt,line join=round,line cap=round,fill=fillColor] (462.94,388.84) circle (  1.16);

\path[draw=drawColor,line width= 0.4pt,line join=round,line cap=round,fill=fillColor] (463.09,388.76) circle (  1.16);

\path[draw=drawColor,line width= 0.4pt,line join=round,line cap=round,fill=fillColor] (463.24,388.75) circle (  1.16);

\path[draw=drawColor,line width= 0.4pt,line join=round,line cap=round,fill=fillColor] (463.39,388.62) circle (  1.16);

\path[draw=drawColor,line width= 0.4pt,line join=round,line cap=round,fill=fillColor] (463.54,388.61) circle (  1.16);

\path[draw=drawColor,line width= 0.4pt,line join=round,line cap=round,fill=fillColor] (463.69,388.30) circle (  1.16);

\path[draw=drawColor,line width= 0.4pt,line join=round,line cap=round,fill=fillColor] (463.84,388.16) circle (  1.16);

\path[draw=drawColor,line width= 0.4pt,line join=round,line cap=round,fill=fillColor] (463.99,388.14) circle (  1.16);

\path[draw=drawColor,line width= 0.4pt,line join=round,line cap=round,fill=fillColor] (464.14,388.08) circle (  1.16);

\path[draw=drawColor,line width= 0.4pt,line join=round,line cap=round,fill=fillColor] (464.29,388.08) circle (  1.16);

\path[draw=drawColor,line width= 0.4pt,line join=round,line cap=round,fill=fillColor] (464.44,388.08) circle (  1.16);

\path[draw=drawColor,line width= 0.4pt,line join=round,line cap=round,fill=fillColor] (464.58,388.03) circle (  1.16);

\path[draw=drawColor,line width= 0.4pt,line join=round,line cap=round,fill=fillColor] (464.73,388.00) circle (  1.16);

\path[draw=drawColor,line width= 0.4pt,line join=round,line cap=round,fill=fillColor] (464.88,387.91) circle (  1.16);

\path[draw=drawColor,line width= 0.4pt,line join=round,line cap=round,fill=fillColor] (465.03,387.81) circle (  1.16);

\path[draw=drawColor,line width= 0.4pt,line join=round,line cap=round,fill=fillColor] (465.17,387.56) circle (  1.16);

\path[draw=drawColor,line width= 0.4pt,line join=round,line cap=round,fill=fillColor] (465.32,387.52) circle (  1.16);

\path[draw=drawColor,line width= 0.4pt,line join=round,line cap=round,fill=fillColor] (465.47,387.48) circle (  1.16);

\path[draw=drawColor,line width= 0.4pt,line join=round,line cap=round,fill=fillColor] (465.61,387.47) circle (  1.16);

\path[draw=drawColor,line width= 0.4pt,line join=round,line cap=round,fill=fillColor] (465.76,387.41) circle (  1.16);

\path[draw=drawColor,line width= 0.4pt,line join=round,line cap=round,fill=fillColor] (465.91,387.30) circle (  1.16);

\path[draw=drawColor,line width= 0.4pt,line join=round,line cap=round,fill=fillColor] (466.05,387.26) circle (  1.16);

\path[draw=drawColor,line width= 0.4pt,line join=round,line cap=round,fill=fillColor] (466.20,387.24) circle (  1.16);

\path[draw=drawColor,line width= 0.4pt,line join=round,line cap=round,fill=fillColor] (466.34,387.23) circle (  1.16);

\path[draw=drawColor,line width= 0.4pt,line join=round,line cap=round,fill=fillColor] (466.49,387.17) circle (  1.16);

\path[draw=drawColor,line width= 0.4pt,line join=round,line cap=round,fill=fillColor] (466.63,387.15) circle (  1.16);

\path[draw=drawColor,line width= 0.4pt,line join=round,line cap=round,fill=fillColor] (466.78,387.14) circle (  1.16);

\path[draw=drawColor,line width= 0.4pt,line join=round,line cap=round,fill=fillColor] (466.92,387.12) circle (  1.16);

\path[draw=drawColor,line width= 0.4pt,line join=round,line cap=round,fill=fillColor] (467.06,386.91) circle (  1.16);

\path[draw=drawColor,line width= 0.4pt,line join=round,line cap=round,fill=fillColor] (467.21,386.83) circle (  1.16);

\path[draw=drawColor,line width= 0.4pt,line join=round,line cap=round,fill=fillColor] (467.35,386.82) circle (  1.16);

\path[draw=drawColor,line width= 0.4pt,line join=round,line cap=round,fill=fillColor] (467.49,386.76) circle (  1.16);

\path[draw=drawColor,line width= 0.4pt,line join=round,line cap=round,fill=fillColor] (467.64,386.73) circle (  1.16);

\path[draw=drawColor,line width= 0.4pt,line join=round,line cap=round,fill=fillColor] (467.78,386.68) circle (  1.16);

\path[draw=drawColor,line width= 0.4pt,line join=round,line cap=round,fill=fillColor] (467.92,386.66) circle (  1.16);

\path[draw=drawColor,line width= 0.4pt,line join=round,line cap=round,fill=fillColor] (468.06,386.64) circle (  1.16);

\path[draw=drawColor,line width= 0.4pt,line join=round,line cap=round,fill=fillColor] (468.21,386.58) circle (  1.16);

\path[draw=drawColor,line width= 0.4pt,line join=round,line cap=round,fill=fillColor] (468.35,386.48) circle (  1.16);

\path[draw=drawColor,line width= 0.4pt,line join=round,line cap=round,fill=fillColor] (468.49,386.43) circle (  1.16);

\path[draw=drawColor,line width= 0.4pt,line join=round,line cap=round,fill=fillColor] (468.63,386.31) circle (  1.16);

\path[draw=drawColor,line width= 0.4pt,line join=round,line cap=round,fill=fillColor] (468.77,386.28) circle (  1.16);

\path[draw=drawColor,line width= 0.4pt,line join=round,line cap=round,fill=fillColor] (468.91,386.22) circle (  1.16);

\path[draw=drawColor,line width= 0.4pt,line join=round,line cap=round,fill=fillColor] (469.05,386.19) circle (  1.16);

\path[draw=drawColor,line width= 0.4pt,line join=round,line cap=round,fill=fillColor] (469.19,386.01) circle (  1.16);

\path[draw=drawColor,line width= 0.4pt,line join=round,line cap=round,fill=fillColor] (469.33,385.84) circle (  1.16);

\path[draw=drawColor,line width= 0.4pt,line join=round,line cap=round,fill=fillColor] (469.47,385.79) circle (  1.16);

\path[draw=drawColor,line width= 0.4pt,line join=round,line cap=round,fill=fillColor] (469.61,385.71) circle (  1.16);

\path[draw=drawColor,line width= 0.4pt,line join=round,line cap=round,fill=fillColor] (469.75,385.70) circle (  1.16);

\path[draw=drawColor,line width= 0.4pt,line join=round,line cap=round,fill=fillColor] (469.89,385.67) circle (  1.16);

\path[draw=drawColor,line width= 0.4pt,line join=round,line cap=round,fill=fillColor] (470.03,385.42) circle (  1.16);

\path[draw=drawColor,line width= 0.4pt,line join=round,line cap=round,fill=fillColor] (470.17,385.33) circle (  1.16);

\path[draw=drawColor,line width= 0.4pt,line join=round,line cap=round,fill=fillColor] (470.30,385.26) circle (  1.16);

\path[draw=drawColor,line width= 0.4pt,line join=round,line cap=round,fill=fillColor] (470.44,385.10) circle (  1.16);

\path[draw=drawColor,line width= 0.4pt,line join=round,line cap=round,fill=fillColor] (470.58,384.88) circle (  1.16);

\path[draw=drawColor,line width= 0.4pt,line join=round,line cap=round,fill=fillColor] (470.72,384.71) circle (  1.16);

\path[draw=drawColor,line width= 0.4pt,line join=round,line cap=round,fill=fillColor] (470.85,384.63) circle (  1.16);

\path[draw=drawColor,line width= 0.4pt,line join=round,line cap=round,fill=fillColor] (470.99,384.62) circle (  1.16);

\path[draw=drawColor,line width= 0.4pt,line join=round,line cap=round,fill=fillColor] (471.13,384.60) circle (  1.16);

\path[draw=drawColor,line width= 0.4pt,line join=round,line cap=round,fill=fillColor] (471.27,384.52) circle (  1.16);

\path[draw=drawColor,line width= 0.4pt,line join=round,line cap=round,fill=fillColor] (471.40,384.45) circle (  1.16);

\path[draw=drawColor,line width= 0.4pt,line join=round,line cap=round,fill=fillColor] (471.54,384.42) circle (  1.16);

\path[draw=drawColor,line width= 0.4pt,line join=round,line cap=round,fill=fillColor] (471.67,384.41) circle (  1.16);

\path[draw=drawColor,line width= 0.4pt,line join=round,line cap=round,fill=fillColor] (471.81,384.05) circle (  1.16);

\path[draw=drawColor,line width= 0.4pt,line join=round,line cap=round,fill=fillColor] (471.94,383.97) circle (  1.16);

\path[draw=drawColor,line width= 0.4pt,line join=round,line cap=round,fill=fillColor] (472.08,383.93) circle (  1.16);

\path[draw=drawColor,line width= 0.4pt,line join=round,line cap=round,fill=fillColor] (472.22,383.76) circle (  1.16);

\path[draw=drawColor,line width= 0.4pt,line join=round,line cap=round,fill=fillColor] (472.35,383.53) circle (  1.16);

\path[draw=drawColor,line width= 0.4pt,line join=round,line cap=round,fill=fillColor] (472.48,383.53) circle (  1.16);

\path[draw=drawColor,line width= 0.4pt,line join=round,line cap=round,fill=fillColor] (472.62,383.22) circle (  1.16);

\path[draw=drawColor,line width= 0.4pt,line join=round,line cap=round,fill=fillColor] (472.75,382.97) circle (  1.16);

\path[draw=drawColor,line width= 0.4pt,line join=round,line cap=round,fill=fillColor] (472.89,382.83) circle (  1.16);

\path[draw=drawColor,line width= 0.4pt,line join=round,line cap=round,fill=fillColor] (473.02,382.82) circle (  1.16);

\path[draw=drawColor,line width= 0.4pt,line join=round,line cap=round,fill=fillColor] (473.15,382.82) circle (  1.16);

\path[draw=drawColor,line width= 0.4pt,line join=round,line cap=round,fill=fillColor] (473.29,382.19) circle (  1.16);

\path[draw=drawColor,line width= 0.4pt,line join=round,line cap=round,fill=fillColor] (473.42,382.09) circle (  1.16);

\path[draw=drawColor,line width= 0.4pt,line join=round,line cap=round,fill=fillColor] (473.55,382.00) circle (  1.16);

\path[draw=drawColor,line width= 0.4pt,line join=round,line cap=round,fill=fillColor] (473.69,381.94) circle (  1.16);

\path[draw=drawColor,line width= 0.4pt,line join=round,line cap=round,fill=fillColor] (473.82,381.34) circle (  1.16);

\path[draw=drawColor,line width= 0.4pt,line join=round,line cap=round,fill=fillColor] (473.95,380.10) circle (  1.16);

\path[draw=drawColor,line width= 0.4pt,line join=round,line cap=round,fill=fillColor] (474.08,380.10) circle (  1.16);

\path[draw=drawColor,line width= 0.4pt,line join=round,line cap=round,fill=fillColor] (474.22,379.62) circle (  1.16);

\path[draw=drawColor,line width= 0.4pt,line join=round,line cap=round,fill=fillColor] (474.35,379.61) circle (  1.16);

\path[draw=drawColor,line width= 0.4pt,line join=round,line cap=round,fill=fillColor] (474.48,379.61) circle (  1.16);

\path[draw=drawColor,line width= 0.4pt,line join=round,line cap=round,fill=fillColor] (474.61,379.30) circle (  1.16);

\path[draw=drawColor,line width= 0.4pt,line join=round,line cap=round,fill=fillColor] (474.74,379.21) circle (  1.16);

\path[draw=drawColor,line width= 0.4pt,line join=round,line cap=round,fill=fillColor] (474.87,378.81) circle (  1.16);

\path[draw=drawColor,line width= 0.4pt,line join=round,line cap=round,fill=fillColor] (475.00,378.75) circle (  1.16);

\path[draw=drawColor,line width= 0.4pt,line join=round,line cap=round,fill=fillColor] (475.13,377.25) circle (  1.16);

\path[draw=drawColor,line width= 0.4pt,line join=round,line cap=round,fill=fillColor] (475.26,372.42) circle (  1.16);

\path[draw=drawColor,line width= 0.4pt,line join=round,line cap=round,fill=fillColor] (475.39,372.42) circle (  1.16);

\path[draw=drawColor,line width= 0.4pt,line join=round,line cap=round,fill=fillColor] (475.52,372.42) circle (  1.16);

\path[draw=drawColor,line width= 0.4pt,line join=round,line cap=round,fill=fillColor] (475.65,372.42) circle (  1.16);

\path[draw=drawColor,line width= 0.4pt,line join=round,line cap=round,fill=fillColor] (475.78,372.42) circle (  1.16);

\path[draw=drawColor,line width= 0.4pt,line join=round,line cap=round,fill=fillColor] (475.91,372.42) circle (  1.16);

\path[draw=drawColor,line width= 0.4pt,line join=round,line cap=round,fill=fillColor] (476.04,372.42) circle (  1.16);

\path[draw=drawColor,line width= 0.4pt,line join=round,line cap=round,fill=fillColor] (476.17,372.42) circle (  1.16);

\path[draw=drawColor,line width= 0.4pt,line join=round,line cap=round,fill=fillColor] (476.30,372.42) circle (  1.16);

\path[draw=drawColor,line width= 0.4pt,line join=round,line cap=round,fill=fillColor] (476.43,372.42) circle (  1.16);

\path[draw=drawColor,line width= 0.4pt,line join=round,line cap=round,fill=fillColor] (476.56,372.42) circle (  1.16);

\path[draw=drawColor,line width= 0.4pt,line join=round,line cap=round,fill=fillColor] (476.68,372.42) circle (  1.16);

\path[draw=drawColor,line width= 0.4pt,line join=round,line cap=round,fill=fillColor] (476.81,372.42) circle (  1.16);

\path[draw=drawColor,line width= 0.4pt,line join=round,line cap=round,fill=fillColor] (476.94,372.42) circle (  1.16);

\path[draw=drawColor,line width= 0.4pt,line join=round,line cap=round,fill=fillColor] (477.07,372.42) circle (  1.16);

\path[draw=drawColor,line width= 0.4pt,line join=round,line cap=round,fill=fillColor] (477.19,372.42) circle (  1.16);

\path[draw=drawColor,line width= 0.4pt,line join=round,line cap=round,fill=fillColor] (477.32,372.42) circle (  1.16);

\path[draw=drawColor,line width= 0.4pt,line join=round,line cap=round,fill=fillColor] (477.45,372.42) circle (  1.16);

\path[draw=drawColor,line width= 0.4pt,line join=round,line cap=round,fill=fillColor] (477.57,372.42) circle (  1.16);

\path[draw=drawColor,line width= 0.4pt,line join=round,line cap=round,fill=fillColor] (477.70,372.42) circle (  1.16);

\path[draw=drawColor,line width= 0.4pt,line join=round,line cap=round,fill=fillColor] (477.83,372.42) circle (  1.16);

\path[draw=drawColor,line width= 0.4pt,line join=round,line cap=round,fill=fillColor] (477.95,372.42) circle (  1.16);

\path[draw=drawColor,line width= 0.4pt,line join=round,line cap=round,fill=fillColor] (478.08,372.42) circle (  1.16);

\path[draw=drawColor,line width= 0.4pt,line join=round,line cap=round,fill=fillColor] (478.21,372.42) circle (  1.16);
\definecolor[named]{drawColor}{rgb}{0.65,0.34,0.16}
\definecolor[named]{fillColor}{rgb}{0.65,0.34,0.16}

\path[draw=drawColor,line width= 0.4pt,line join=round,line cap=round,fill=fillColor] (327.82,428.32) circle (  1.16);

\path[draw=drawColor,line width= 0.4pt,line join=round,line cap=round,fill=fillColor] (333.65,418.68) circle (  1.16);

\path[draw=drawColor,line width= 0.4pt,line join=round,line cap=round,fill=fillColor] (337.74,418.65) circle (  1.16);

\path[draw=drawColor,line width= 0.4pt,line join=round,line cap=round,fill=fillColor] (340.99,417.73) circle (  1.16);

\path[draw=drawColor,line width= 0.4pt,line join=round,line cap=round,fill=fillColor] (343.74,414.02) circle (  1.16);

\path[draw=drawColor,line width= 0.4pt,line join=round,line cap=round,fill=fillColor] (346.14,412.86) circle (  1.16);

\path[draw=drawColor,line width= 0.4pt,line join=round,line cap=round,fill=fillColor] (348.29,411.91) circle (  1.16);

\path[draw=drawColor,line width= 0.4pt,line join=round,line cap=round,fill=fillColor] (350.24,411.63) circle (  1.16);

\path[draw=drawColor,line width= 0.4pt,line join=round,line cap=round,fill=fillColor] (352.04,411.31) circle (  1.16);

\path[draw=drawColor,line width= 0.4pt,line join=round,line cap=round,fill=fillColor] (353.70,410.93) circle (  1.16);

\path[draw=drawColor,line width= 0.4pt,line join=round,line cap=round,fill=fillColor] (355.26,410.82) circle (  1.16);

\path[draw=drawColor,line width= 0.4pt,line join=round,line cap=round,fill=fillColor] (356.73,409.06) circle (  1.16);

\path[draw=drawColor,line width= 0.4pt,line join=round,line cap=round,fill=fillColor] (358.12,408.62) circle (  1.16);

\path[draw=drawColor,line width= 0.4pt,line join=round,line cap=round,fill=fillColor] (359.44,408.47) circle (  1.16);

\path[draw=drawColor,line width= 0.4pt,line join=round,line cap=round,fill=fillColor] (360.69,408.39) circle (  1.16);

\path[draw=drawColor,line width= 0.4pt,line join=round,line cap=round,fill=fillColor] (361.90,408.04) circle (  1.16);

\path[draw=drawColor,line width= 0.4pt,line join=round,line cap=round,fill=fillColor] (363.05,407.57) circle (  1.16);

\path[draw=drawColor,line width= 0.4pt,line join=round,line cap=round,fill=fillColor] (364.16,406.38) circle (  1.16);

\path[draw=drawColor,line width= 0.4pt,line join=round,line cap=round,fill=fillColor] (365.23,406.23) circle (  1.16);

\path[draw=drawColor,line width= 0.4pt,line join=round,line cap=round,fill=fillColor] (366.26,406.06) circle (  1.16);

\path[draw=drawColor,line width= 0.4pt,line join=round,line cap=round,fill=fillColor] (367.25,405.88) circle (  1.16);

\path[draw=drawColor,line width= 0.4pt,line join=round,line cap=round,fill=fillColor] (368.22,405.45) circle (  1.16);

\path[draw=drawColor,line width= 0.4pt,line join=round,line cap=round,fill=fillColor] (369.16,405.19) circle (  1.16);

\path[draw=drawColor,line width= 0.4pt,line join=round,line cap=round,fill=fillColor] (370.07,404.32) circle (  1.16);

\path[draw=drawColor,line width= 0.4pt,line join=round,line cap=round,fill=fillColor] (370.96,403.91) circle (  1.16);

\path[draw=drawColor,line width= 0.4pt,line join=round,line cap=round,fill=fillColor] (371.82,403.75) circle (  1.16);

\path[draw=drawColor,line width= 0.4pt,line join=round,line cap=round,fill=fillColor] (372.66,403.00) circle (  1.16);

\path[draw=drawColor,line width= 0.4pt,line join=round,line cap=round,fill=fillColor] (373.48,402.35) circle (  1.16);

\path[draw=drawColor,line width= 0.4pt,line join=round,line cap=round,fill=fillColor] (374.28,402.35) circle (  1.16);

\path[draw=drawColor,line width= 0.4pt,line join=round,line cap=round,fill=fillColor] (375.06,401.66) circle (  1.16);

\path[draw=drawColor,line width= 0.4pt,line join=round,line cap=round,fill=fillColor] (375.83,401.42) circle (  1.16);

\path[draw=drawColor,line width= 0.4pt,line join=round,line cap=round,fill=fillColor] (376.58,401.08) circle (  1.16);

\path[draw=drawColor,line width= 0.4pt,line join=round,line cap=round,fill=fillColor] (377.31,401.07) circle (  1.16);

\path[draw=drawColor,line width= 0.4pt,line join=round,line cap=round,fill=fillColor] (378.03,400.97) circle (  1.16);

\path[draw=drawColor,line width= 0.4pt,line join=round,line cap=round,fill=fillColor] (378.74,400.54) circle (  1.16);

\path[draw=drawColor,line width= 0.4pt,line join=round,line cap=round,fill=fillColor] (379.43,399.74) circle (  1.16);

\path[draw=drawColor,line width= 0.4pt,line join=round,line cap=round,fill=fillColor] (380.11,399.16) circle (  1.16);

\path[draw=drawColor,line width= 0.4pt,line join=round,line cap=round,fill=fillColor] (380.77,399.14) circle (  1.16);

\path[draw=drawColor,line width= 0.4pt,line join=round,line cap=round,fill=fillColor] (381.43,398.95) circle (  1.16);

\path[draw=drawColor,line width= 0.4pt,line join=round,line cap=round,fill=fillColor] (382.07,398.23) circle (  1.16);

\path[draw=drawColor,line width= 0.4pt,line join=round,line cap=round,fill=fillColor] (382.71,397.92) circle (  1.16);

\path[draw=drawColor,line width= 0.4pt,line join=round,line cap=round,fill=fillColor] (383.33,397.76) circle (  1.16);

\path[draw=drawColor,line width= 0.4pt,line join=round,line cap=round,fill=fillColor] (383.94,397.74) circle (  1.16);

\path[draw=drawColor,line width= 0.4pt,line join=round,line cap=round,fill=fillColor] (384.55,397.60) circle (  1.16);

\path[draw=drawColor,line width= 0.4pt,line join=round,line cap=round,fill=fillColor] (385.14,397.27) circle (  1.16);

\path[draw=drawColor,line width= 0.4pt,line join=round,line cap=round,fill=fillColor] (385.73,397.23) circle (  1.16);

\path[draw=drawColor,line width= 0.4pt,line join=round,line cap=round,fill=fillColor] (386.31,396.73) circle (  1.16);

\path[draw=drawColor,line width= 0.4pt,line join=round,line cap=round,fill=fillColor] (386.88,396.69) circle (  1.16);

\path[draw=drawColor,line width= 0.4pt,line join=round,line cap=round,fill=fillColor] (387.44,396.56) circle (  1.16);

\path[draw=drawColor,line width= 0.4pt,line join=round,line cap=round,fill=fillColor] (387.99,396.56) circle (  1.16);

\path[draw=drawColor,line width= 0.4pt,line join=round,line cap=round,fill=fillColor] (388.54,396.41) circle (  1.16);

\path[draw=drawColor,line width= 0.4pt,line join=round,line cap=round,fill=fillColor] (389.08,396.17) circle (  1.16);

\path[draw=drawColor,line width= 0.4pt,line join=round,line cap=round,fill=fillColor] (389.61,396.03) circle (  1.16);

\path[draw=drawColor,line width= 0.4pt,line join=round,line cap=round,fill=fillColor] (390.14,395.54) circle (  1.16);

\path[draw=drawColor,line width= 0.4pt,line join=round,line cap=round,fill=fillColor] (390.66,395.53) circle (  1.16);

\path[draw=drawColor,line width= 0.4pt,line join=round,line cap=round,fill=fillColor] (391.17,395.45) circle (  1.16);

\path[draw=drawColor,line width= 0.4pt,line join=round,line cap=round,fill=fillColor] (391.68,395.40) circle (  1.16);

\path[draw=drawColor,line width= 0.4pt,line join=round,line cap=round,fill=fillColor] (392.18,394.49) circle (  1.16);

\path[draw=drawColor,line width= 0.4pt,line join=round,line cap=round,fill=fillColor] (392.68,394.39) circle (  1.16);

\path[draw=drawColor,line width= 0.4pt,line join=round,line cap=round,fill=fillColor] (393.17,394.23) circle (  1.16);

\path[draw=drawColor,line width= 0.4pt,line join=round,line cap=round,fill=fillColor] (393.65,394.13) circle (  1.16);

\path[draw=drawColor,line width= 0.4pt,line join=round,line cap=round,fill=fillColor] (394.13,394.06) circle (  1.16);

\path[draw=drawColor,line width= 0.4pt,line join=round,line cap=round,fill=fillColor] (394.61,393.82) circle (  1.16);

\path[draw=drawColor,line width= 0.4pt,line join=round,line cap=round,fill=fillColor] (395.08,393.71) circle (  1.16);

\path[draw=drawColor,line width= 0.4pt,line join=round,line cap=round,fill=fillColor] (395.54,393.51) circle (  1.16);

\path[draw=drawColor,line width= 0.4pt,line join=round,line cap=round,fill=fillColor] (396.00,393.32) circle (  1.16);

\path[draw=drawColor,line width= 0.4pt,line join=round,line cap=round,fill=fillColor] (396.46,393.10) circle (  1.16);

\path[draw=drawColor,line width= 0.4pt,line join=round,line cap=round,fill=fillColor] (396.91,392.47) circle (  1.16);

\path[draw=drawColor,line width= 0.4pt,line join=round,line cap=round,fill=fillColor] (397.35,391.85) circle (  1.16);

\path[draw=drawColor,line width= 0.4pt,line join=round,line cap=round,fill=fillColor] (397.80,391.69) circle (  1.16);

\path[draw=drawColor,line width= 0.4pt,line join=round,line cap=round,fill=fillColor] (398.23,391.26) circle (  1.16);

\path[draw=drawColor,line width= 0.4pt,line join=round,line cap=round,fill=fillColor] (398.67,391.11) circle (  1.16);

\path[draw=drawColor,line width= 0.4pt,line join=round,line cap=round,fill=fillColor] (399.10,391.07) circle (  1.16);

\path[draw=drawColor,line width= 0.4pt,line join=round,line cap=round,fill=fillColor] (399.52,391.05) circle (  1.16);

\path[draw=drawColor,line width= 0.4pt,line join=round,line cap=round,fill=fillColor] (399.94,390.52) circle (  1.16);

\path[draw=drawColor,line width= 0.4pt,line join=round,line cap=round,fill=fillColor] (400.36,390.51) circle (  1.16);

\path[draw=drawColor,line width= 0.4pt,line join=round,line cap=round,fill=fillColor] (400.78,390.37) circle (  1.16);

\path[draw=drawColor,line width= 0.4pt,line join=round,line cap=round,fill=fillColor] (401.19,390.22) circle (  1.16);

\path[draw=drawColor,line width= 0.4pt,line join=round,line cap=round,fill=fillColor] (401.60,390.14) circle (  1.16);

\path[draw=drawColor,line width= 0.4pt,line join=round,line cap=round,fill=fillColor] (402.00,390.13) circle (  1.16);

\path[draw=drawColor,line width= 0.4pt,line join=round,line cap=round,fill=fillColor] (402.40,390.06) circle (  1.16);

\path[draw=drawColor,line width= 0.4pt,line join=round,line cap=round,fill=fillColor] (402.80,390.00) circle (  1.16);

\path[draw=drawColor,line width= 0.4pt,line join=round,line cap=round,fill=fillColor] (403.19,389.82) circle (  1.16);

\path[draw=drawColor,line width= 0.4pt,line join=round,line cap=round,fill=fillColor] (403.58,389.53) circle (  1.16);

\path[draw=drawColor,line width= 0.4pt,line join=round,line cap=round,fill=fillColor] (403.97,389.19) circle (  1.16);

\path[draw=drawColor,line width= 0.4pt,line join=round,line cap=round,fill=fillColor] (404.36,388.94) circle (  1.16);

\path[draw=drawColor,line width= 0.4pt,line join=round,line cap=round,fill=fillColor] (404.74,388.66) circle (  1.16);

\path[draw=drawColor,line width= 0.4pt,line join=round,line cap=round,fill=fillColor] (405.12,388.65) circle (  1.16);

\path[draw=drawColor,line width= 0.4pt,line join=round,line cap=round,fill=fillColor] (405.49,388.53) circle (  1.16);

\path[draw=drawColor,line width= 0.4pt,line join=round,line cap=round,fill=fillColor] (405.87,388.36) circle (  1.16);

\path[draw=drawColor,line width= 0.4pt,line join=round,line cap=round,fill=fillColor] (406.24,388.15) circle (  1.16);

\path[draw=drawColor,line width= 0.4pt,line join=round,line cap=round,fill=fillColor] (406.61,388.10) circle (  1.16);

\path[draw=drawColor,line width= 0.4pt,line join=round,line cap=round,fill=fillColor] (406.97,387.94) circle (  1.16);

\path[draw=drawColor,line width= 0.4pt,line join=round,line cap=round,fill=fillColor] (407.34,387.90) circle (  1.16);

\path[draw=drawColor,line width= 0.4pt,line join=round,line cap=round,fill=fillColor] (407.70,387.80) circle (  1.16);

\path[draw=drawColor,line width= 0.4pt,line join=round,line cap=round,fill=fillColor] (408.05,387.79) circle (  1.16);

\path[draw=drawColor,line width= 0.4pt,line join=round,line cap=round,fill=fillColor] (408.41,387.76) circle (  1.16);

\path[draw=drawColor,line width= 0.4pt,line join=round,line cap=round,fill=fillColor] (408.76,387.17) circle (  1.16);

\path[draw=drawColor,line width= 0.4pt,line join=round,line cap=round,fill=fillColor] (409.11,387.05) circle (  1.16);

\path[draw=drawColor,line width= 0.4pt,line join=round,line cap=round,fill=fillColor] (409.46,386.80) circle (  1.16);

\path[draw=drawColor,line width= 0.4pt,line join=round,line cap=round,fill=fillColor] (409.80,386.73) circle (  1.16);

\path[draw=drawColor,line width= 0.4pt,line join=round,line cap=round,fill=fillColor] (410.15,386.72) circle (  1.16);

\path[draw=drawColor,line width= 0.4pt,line join=round,line cap=round,fill=fillColor] (410.49,386.71) circle (  1.16);

\path[draw=drawColor,line width= 0.4pt,line join=round,line cap=round,fill=fillColor] (410.83,386.44) circle (  1.16);

\path[draw=drawColor,line width= 0.4pt,line join=round,line cap=round,fill=fillColor] (411.17,386.24) circle (  1.16);

\path[draw=drawColor,line width= 0.4pt,line join=round,line cap=round,fill=fillColor] (411.50,385.98) circle (  1.16);

\path[draw=drawColor,line width= 0.4pt,line join=round,line cap=round,fill=fillColor] (411.83,385.68) circle (  1.16);

\path[draw=drawColor,line width= 0.4pt,line join=round,line cap=round,fill=fillColor] (412.16,385.16) circle (  1.16);

\path[draw=drawColor,line width= 0.4pt,line join=round,line cap=round,fill=fillColor] (412.49,385.12) circle (  1.16);

\path[draw=drawColor,line width= 0.4pt,line join=round,line cap=round,fill=fillColor] (412.82,385.03) circle (  1.16);

\path[draw=drawColor,line width= 0.4pt,line join=round,line cap=round,fill=fillColor] (413.14,384.98) circle (  1.16);

\path[draw=drawColor,line width= 0.4pt,line join=round,line cap=round,fill=fillColor] (413.47,384.79) circle (  1.16);

\path[draw=drawColor,line width= 0.4pt,line join=round,line cap=round,fill=fillColor] (413.79,384.45) circle (  1.16);

\path[draw=drawColor,line width= 0.4pt,line join=round,line cap=round,fill=fillColor] (414.10,384.39) circle (  1.16);

\path[draw=drawColor,line width= 0.4pt,line join=round,line cap=round,fill=fillColor] (414.42,384.33) circle (  1.16);

\path[draw=drawColor,line width= 0.4pt,line join=round,line cap=round,fill=fillColor] (414.74,383.77) circle (  1.16);

\path[draw=drawColor,line width= 0.4pt,line join=round,line cap=round,fill=fillColor] (415.05,383.73) circle (  1.16);

\path[draw=drawColor,line width= 0.4pt,line join=round,line cap=round,fill=fillColor] (415.36,383.64) circle (  1.16);

\path[draw=drawColor,line width= 0.4pt,line join=round,line cap=round,fill=fillColor] (415.67,383.42) circle (  1.16);

\path[draw=drawColor,line width= 0.4pt,line join=round,line cap=round,fill=fillColor] (415.98,383.19) circle (  1.16);

\path[draw=drawColor,line width= 0.4pt,line join=round,line cap=round,fill=fillColor] (416.29,383.15) circle (  1.16);

\path[draw=drawColor,line width= 0.4pt,line join=round,line cap=round,fill=fillColor] (416.59,383.13) circle (  1.16);

\path[draw=drawColor,line width= 0.4pt,line join=round,line cap=round,fill=fillColor] (416.89,383.13) circle (  1.16);

\path[draw=drawColor,line width= 0.4pt,line join=round,line cap=round,fill=fillColor] (417.19,382.44) circle (  1.16);

\path[draw=drawColor,line width= 0.4pt,line join=round,line cap=round,fill=fillColor] (417.49,381.76) circle (  1.16);

\path[draw=drawColor,line width= 0.4pt,line join=round,line cap=round,fill=fillColor] (417.79,380.98) circle (  1.16);

\path[draw=drawColor,line width= 0.4pt,line join=round,line cap=round,fill=fillColor] (418.09,380.96) circle (  1.16);

\path[draw=drawColor,line width= 0.4pt,line join=round,line cap=round,fill=fillColor] (418.38,380.89) circle (  1.16);

\path[draw=drawColor,line width= 0.4pt,line join=round,line cap=round,fill=fillColor] (418.68,380.81) circle (  1.16);

\path[draw=drawColor,line width= 0.4pt,line join=round,line cap=round,fill=fillColor] (418.97,380.54) circle (  1.16);

\path[draw=drawColor,line width= 0.4pt,line join=round,line cap=round,fill=fillColor] (419.26,380.14) circle (  1.16);

\path[draw=drawColor,line width= 0.4pt,line join=round,line cap=round,fill=fillColor] (419.55,379.89) circle (  1.16);

\path[draw=drawColor,line width= 0.4pt,line join=round,line cap=round,fill=fillColor] (419.84,379.39) circle (  1.16);

\path[draw=drawColor,line width= 0.4pt,line join=round,line cap=round,fill=fillColor] (420.12,378.20) circle (  1.16);

\path[draw=drawColor,line width= 0.4pt,line join=round,line cap=round,fill=fillColor] (420.41,377.71) circle (  1.16);

\path[draw=drawColor,line width= 0.4pt,line join=round,line cap=round,fill=fillColor] (420.69,377.61) circle (  1.16);

\path[draw=drawColor,line width= 0.4pt,line join=round,line cap=round,fill=fillColor] (420.97,377.19) circle (  1.16);

\path[draw=drawColor,line width= 0.4pt,line join=round,line cap=round,fill=fillColor] (421.25,372.42) circle (  1.16);

\path[draw=drawColor,line width= 0.4pt,line join=round,line cap=round,fill=fillColor] (421.53,372.42) circle (  1.16);

\path[draw=drawColor,line width= 0.4pt,line join=round,line cap=round,fill=fillColor] (421.81,372.42) circle (  1.16);

\path[draw=drawColor,line width= 0.4pt,line join=round,line cap=round,fill=fillColor] (422.09,372.42) circle (  1.16);

\path[draw=drawColor,line width= 0.4pt,line join=round,line cap=round,fill=fillColor] (422.36,372.42) circle (  1.16);

\path[draw=drawColor,line width= 0.4pt,line join=round,line cap=round,fill=fillColor] (422.63,372.42) circle (  1.16);

\path[draw=drawColor,line width= 0.4pt,line join=round,line cap=round,fill=fillColor] (422.91,372.42) circle (  1.16);

\path[draw=drawColor,line width= 0.4pt,line join=round,line cap=round,fill=fillColor] (423.18,372.42) circle (  1.16);

\path[draw=drawColor,line width= 0.4pt,line join=round,line cap=round,fill=fillColor] (423.45,372.42) circle (  1.16);

\path[draw=drawColor,line width= 0.4pt,line join=round,line cap=round,fill=fillColor] (423.72,372.42) circle (  1.16);

\path[draw=drawColor,line width= 0.4pt,line join=round,line cap=round,fill=fillColor] (423.99,372.42) circle (  1.16);

\path[draw=drawColor,line width= 0.4pt,line join=round,line cap=round,fill=fillColor] (424.25,372.42) circle (  1.16);

\path[draw=drawColor,line width= 0.4pt,line join=round,line cap=round,fill=fillColor] (424.52,372.42) circle (  1.16);

\path[draw=drawColor,line width= 0.4pt,line join=round,line cap=round,fill=fillColor] (424.78,372.42) circle (  1.16);

\path[draw=drawColor,line width= 0.4pt,line join=round,line cap=round,fill=fillColor] (425.04,372.42) circle (  1.16);

\path[draw=drawColor,line width= 0.4pt,line join=round,line cap=round,fill=fillColor] (425.31,372.42) circle (  1.16);

\path[draw=drawColor,line width= 0.4pt,line join=round,line cap=round,fill=fillColor] (425.57,372.42) circle (  1.16);

\path[draw=drawColor,line width= 0.4pt,line join=round,line cap=round,fill=fillColor] (425.83,372.42) circle (  1.16);

\path[draw=drawColor,line width= 0.4pt,line join=round,line cap=round,fill=fillColor] (426.08,372.42) circle (  1.16);

\path[draw=drawColor,line width= 0.4pt,line join=round,line cap=round,fill=fillColor] (426.34,372.42) circle (  1.16);

\path[draw=drawColor,line width= 0.4pt,line join=round,line cap=round,fill=fillColor] (426.60,372.42) circle (  1.16);

\path[draw=drawColor,line width= 0.4pt,line join=round,line cap=round,fill=fillColor] (426.85,372.42) circle (  1.16);

\path[draw=drawColor,line width= 0.4pt,line join=round,line cap=round,fill=fillColor] (427.11,372.42) circle (  1.16);

\path[draw=drawColor,line width= 0.4pt,line join=round,line cap=round,fill=fillColor] (427.36,372.42) circle (  1.16);

\path[draw=drawColor,line width= 0.4pt,line join=round,line cap=round,fill=fillColor] (427.61,372.42) circle (  1.16);

\path[draw=drawColor,line width= 0.4pt,line join=round,line cap=round,fill=fillColor] (427.86,372.42) circle (  1.16);

\path[draw=drawColor,line width= 0.4pt,line join=round,line cap=round,fill=fillColor] (428.11,372.42) circle (  1.16);

\path[draw=drawColor,line width= 0.4pt,line join=round,line cap=round,fill=fillColor] (428.36,372.42) circle (  1.16);

\path[draw=drawColor,line width= 0.4pt,line join=round,line cap=round,fill=fillColor] (428.61,372.42) circle (  1.16);

\path[draw=drawColor,line width= 0.4pt,line join=round,line cap=round,fill=fillColor] (428.86,372.42) circle (  1.16);

\path[draw=drawColor,line width= 0.4pt,line join=round,line cap=round,fill=fillColor] (429.10,372.42) circle (  1.16);

\path[draw=drawColor,line width= 0.4pt,line join=round,line cap=round,fill=fillColor] (429.35,372.42) circle (  1.16);

\path[draw=drawColor,line width= 0.4pt,line join=round,line cap=round,fill=fillColor] (429.59,372.42) circle (  1.16);

\path[draw=drawColor,line width= 0.4pt,line join=round,line cap=round,fill=fillColor] (429.83,372.42) circle (  1.16);

\path[draw=drawColor,line width= 0.4pt,line join=round,line cap=round,fill=fillColor] (430.08,372.42) circle (  1.16);

\path[draw=drawColor,line width= 0.4pt,line join=round,line cap=round,fill=fillColor] (430.32,372.42) circle (  1.16);

\path[draw=drawColor,line width= 0.4pt,line join=round,line cap=round,fill=fillColor] (430.56,372.42) circle (  1.16);

\path[draw=drawColor,line width= 0.4pt,line join=round,line cap=round,fill=fillColor] (430.80,372.42) circle (  1.16);

\path[draw=drawColor,line width= 0.4pt,line join=round,line cap=round,fill=fillColor] (431.04,372.42) circle (  1.16);

\path[draw=drawColor,line width= 0.4pt,line join=round,line cap=round,fill=fillColor] (431.27,372.42) circle (  1.16);

\path[draw=drawColor,line width= 0.4pt,line join=round,line cap=round,fill=fillColor] (431.51,372.42) circle (  1.16);

\path[draw=drawColor,line width= 0.4pt,line join=round,line cap=round,fill=fillColor] (431.75,372.42) circle (  1.16);

\path[draw=drawColor,line width= 0.4pt,line join=round,line cap=round,fill=fillColor] (431.98,372.42) circle (  1.16);

\path[draw=drawColor,line width= 0.4pt,line join=round,line cap=round,fill=fillColor] (432.21,372.42) circle (  1.16);

\path[draw=drawColor,line width= 0.4pt,line join=round,line cap=round,fill=fillColor] (432.45,372.42) circle (  1.16);

\path[draw=drawColor,line width= 0.4pt,line join=round,line cap=round,fill=fillColor] (432.68,372.42) circle (  1.16);

\path[draw=drawColor,line width= 0.4pt,line join=round,line cap=round,fill=fillColor] (432.91,372.42) circle (  1.16);

\path[draw=drawColor,line width= 0.4pt,line join=round,line cap=round,fill=fillColor] (433.14,372.42) circle (  1.16);

\path[draw=drawColor,line width= 0.4pt,line join=round,line cap=round,fill=fillColor] (433.37,372.42) circle (  1.16);

\path[draw=drawColor,line width= 0.4pt,line join=round,line cap=round,fill=fillColor] (433.60,372.42) circle (  1.16);

\path[draw=drawColor,line width= 0.4pt,line join=round,line cap=round,fill=fillColor] (433.83,372.42) circle (  1.16);

\path[draw=drawColor,line width= 0.4pt,line join=round,line cap=round,fill=fillColor] (434.06,372.42) circle (  1.16);

\path[draw=drawColor,line width= 0.4pt,line join=round,line cap=round,fill=fillColor] (434.28,372.42) circle (  1.16);

\path[draw=drawColor,line width= 0.4pt,line join=round,line cap=round,fill=fillColor] (434.51,372.42) circle (  1.16);

\path[draw=drawColor,line width= 0.4pt,line join=round,line cap=round,fill=fillColor] (434.73,372.42) circle (  1.16);

\path[draw=drawColor,line width= 0.4pt,line join=round,line cap=round,fill=fillColor] (434.96,372.42) circle (  1.16);

\path[draw=drawColor,line width= 0.4pt,line join=round,line cap=round,fill=fillColor] (435.18,372.42) circle (  1.16);

\path[draw=drawColor,line width= 0.4pt,line join=round,line cap=round,fill=fillColor] (435.40,372.42) circle (  1.16);

\path[draw=drawColor,line width= 0.4pt,line join=round,line cap=round,fill=fillColor] (435.62,372.42) circle (  1.16);

\path[draw=drawColor,line width= 0.4pt,line join=round,line cap=round,fill=fillColor] (435.85,372.42) circle (  1.16);

\path[draw=drawColor,line width= 0.4pt,line join=round,line cap=round,fill=fillColor] (436.07,372.42) circle (  1.16);

\path[draw=drawColor,line width= 0.4pt,line join=round,line cap=round,fill=fillColor] (436.29,372.42) circle (  1.16);

\path[draw=drawColor,line width= 0.4pt,line join=round,line cap=round,fill=fillColor] (436.50,372.42) circle (  1.16);

\path[draw=drawColor,line width= 0.4pt,line join=round,line cap=round,fill=fillColor] (436.72,372.42) circle (  1.16);

\path[draw=drawColor,line width= 0.4pt,line join=round,line cap=round,fill=fillColor] (436.94,372.42) circle (  1.16);

\path[draw=drawColor,line width= 0.4pt,line join=round,line cap=round,fill=fillColor] (437.16,372.42) circle (  1.16);

\path[draw=drawColor,line width= 0.4pt,line join=round,line cap=round,fill=fillColor] (437.37,372.42) circle (  1.16);

\path[draw=drawColor,line width= 0.4pt,line join=round,line cap=round,fill=fillColor] (437.59,372.42) circle (  1.16);

\path[draw=drawColor,line width= 0.4pt,line join=round,line cap=round,fill=fillColor] (437.80,372.42) circle (  1.16);

\path[draw=drawColor,line width= 0.4pt,line join=round,line cap=round,fill=fillColor] (438.02,372.42) circle (  1.16);

\path[draw=drawColor,line width= 0.4pt,line join=round,line cap=round,fill=fillColor] (438.23,372.42) circle (  1.16);

\path[draw=drawColor,line width= 0.4pt,line join=round,line cap=round,fill=fillColor] (438.44,372.42) circle (  1.16);

\path[draw=drawColor,line width= 0.4pt,line join=round,line cap=round,fill=fillColor] (438.65,372.42) circle (  1.16);

\path[draw=drawColor,line width= 0.4pt,line join=round,line cap=round,fill=fillColor] (438.87,372.42) circle (  1.16);

\path[draw=drawColor,line width= 0.4pt,line join=round,line cap=round,fill=fillColor] (439.08,372.42) circle (  1.16);

\path[draw=drawColor,line width= 0.4pt,line join=round,line cap=round,fill=fillColor] (439.29,372.42) circle (  1.16);

\path[draw=drawColor,line width= 0.4pt,line join=round,line cap=round,fill=fillColor] (439.49,372.42) circle (  1.16);

\path[draw=drawColor,line width= 0.4pt,line join=round,line cap=round,fill=fillColor] (439.70,372.42) circle (  1.16);

\path[draw=drawColor,line width= 0.4pt,line join=round,line cap=round,fill=fillColor] (439.91,372.42) circle (  1.16);

\path[draw=drawColor,line width= 0.4pt,line join=round,line cap=round,fill=fillColor] (440.12,372.42) circle (  1.16);

\path[draw=drawColor,line width= 0.4pt,line join=round,line cap=round,fill=fillColor] (440.33,372.42) circle (  1.16);

\path[draw=drawColor,line width= 0.4pt,line join=round,line cap=round,fill=fillColor] (440.53,372.42) circle (  1.16);

\path[draw=drawColor,line width= 0.4pt,line join=round,line cap=round,fill=fillColor] (440.74,372.42) circle (  1.16);

\path[draw=drawColor,line width= 0.4pt,line join=round,line cap=round,fill=fillColor] (440.94,372.42) circle (  1.16);

\path[draw=drawColor,line width= 0.4pt,line join=round,line cap=round,fill=fillColor] (441.15,372.42) circle (  1.16);

\path[draw=drawColor,line width= 0.4pt,line join=round,line cap=round,fill=fillColor] (441.35,372.42) circle (  1.16);

\path[draw=drawColor,line width= 0.4pt,line join=round,line cap=round,fill=fillColor] (441.55,372.42) circle (  1.16);

\path[draw=drawColor,line width= 0.4pt,line join=round,line cap=round,fill=fillColor] (441.75,372.42) circle (  1.16);

\path[draw=drawColor,line width= 0.4pt,line join=round,line cap=round,fill=fillColor] (441.96,372.42) circle (  1.16);

\path[draw=drawColor,line width= 0.4pt,line join=round,line cap=round,fill=fillColor] (442.16,372.42) circle (  1.16);

\path[draw=drawColor,line width= 0.4pt,line join=round,line cap=round,fill=fillColor] (442.36,372.42) circle (  1.16);

\path[draw=drawColor,line width= 0.4pt,line join=round,line cap=round,fill=fillColor] (442.56,372.42) circle (  1.16);

\path[draw=drawColor,line width= 0.4pt,line join=round,line cap=round,fill=fillColor] (442.76,372.42) circle (  1.16);

\path[draw=drawColor,line width= 0.4pt,line join=round,line cap=round,fill=fillColor] (442.96,372.42) circle (  1.16);

\path[draw=drawColor,line width= 0.4pt,line join=round,line cap=round,fill=fillColor] (443.15,372.42) circle (  1.16);

\path[draw=drawColor,line width= 0.4pt,line join=round,line cap=round,fill=fillColor] (443.35,372.42) circle (  1.16);

\path[draw=drawColor,line width= 0.4pt,line join=round,line cap=round,fill=fillColor] (443.55,372.42) circle (  1.16);

\path[draw=drawColor,line width= 0.4pt,line join=round,line cap=round,fill=fillColor] (443.74,372.42) circle (  1.16);

\path[draw=drawColor,line width= 0.4pt,line join=round,line cap=round,fill=fillColor] (443.94,372.42) circle (  1.16);

\path[draw=drawColor,line width= 0.4pt,line join=round,line cap=round,fill=fillColor] (444.14,372.42) circle (  1.16);

\path[draw=drawColor,line width= 0.4pt,line join=round,line cap=round,fill=fillColor] (444.33,372.42) circle (  1.16);

\path[draw=drawColor,line width= 0.4pt,line join=round,line cap=round,fill=fillColor] (444.53,372.42) circle (  1.16);

\path[draw=drawColor,line width= 0.4pt,line join=round,line cap=round,fill=fillColor] (444.72,372.42) circle (  1.16);

\path[draw=drawColor,line width= 0.4pt,line join=round,line cap=round,fill=fillColor] (444.91,372.42) circle (  1.16);

\path[draw=drawColor,line width= 0.4pt,line join=round,line cap=round,fill=fillColor] (445.10,372.42) circle (  1.16);

\path[draw=drawColor,line width= 0.4pt,line join=round,line cap=round,fill=fillColor] (445.30,372.42) circle (  1.16);

\path[draw=drawColor,line width= 0.4pt,line join=round,line cap=round,fill=fillColor] (445.49,372.42) circle (  1.16);

\path[draw=drawColor,line width= 0.4pt,line join=round,line cap=round,fill=fillColor] (445.68,372.42) circle (  1.16);

\path[draw=drawColor,line width= 0.4pt,line join=round,line cap=round,fill=fillColor] (445.87,372.42) circle (  1.16);

\path[draw=drawColor,line width= 0.4pt,line join=round,line cap=round,fill=fillColor] (446.06,372.42) circle (  1.16);

\path[draw=drawColor,line width= 0.4pt,line join=round,line cap=round,fill=fillColor] (446.25,372.42) circle (  1.16);

\path[draw=drawColor,line width= 0.4pt,line join=round,line cap=round,fill=fillColor] (446.44,372.42) circle (  1.16);

\path[draw=drawColor,line width= 0.4pt,line join=round,line cap=round,fill=fillColor] (446.63,372.42) circle (  1.16);

\path[draw=drawColor,line width= 0.4pt,line join=round,line cap=round,fill=fillColor] (446.82,372.42) circle (  1.16);

\path[draw=drawColor,line width= 0.4pt,line join=round,line cap=round,fill=fillColor] (447.00,372.42) circle (  1.16);

\path[draw=drawColor,line width= 0.4pt,line join=round,line cap=round,fill=fillColor] (447.19,372.42) circle (  1.16);

\path[draw=drawColor,line width= 0.4pt,line join=round,line cap=round,fill=fillColor] (447.38,372.42) circle (  1.16);

\path[draw=drawColor,line width= 0.4pt,line join=round,line cap=round,fill=fillColor] (447.56,372.42) circle (  1.16);

\path[draw=drawColor,line width= 0.4pt,line join=round,line cap=round,fill=fillColor] (447.75,372.42) circle (  1.16);

\path[draw=drawColor,line width= 0.4pt,line join=round,line cap=round,fill=fillColor] (447.93,372.42) circle (  1.16);

\path[draw=drawColor,line width= 0.4pt,line join=round,line cap=round,fill=fillColor] (448.12,372.42) circle (  1.16);

\path[draw=drawColor,line width= 0.4pt,line join=round,line cap=round,fill=fillColor] (448.30,372.42) circle (  1.16);

\path[draw=drawColor,line width= 0.4pt,line join=round,line cap=round,fill=fillColor] (448.49,372.42) circle (  1.16);

\path[draw=drawColor,line width= 0.4pt,line join=round,line cap=round,fill=fillColor] (448.67,372.42) circle (  1.16);

\path[draw=drawColor,line width= 0.4pt,line join=round,line cap=round,fill=fillColor] (448.85,372.42) circle (  1.16);

\path[draw=drawColor,line width= 0.4pt,line join=round,line cap=round,fill=fillColor] (449.03,372.42) circle (  1.16);

\path[draw=drawColor,line width= 0.4pt,line join=round,line cap=round,fill=fillColor] (449.22,372.42) circle (  1.16);

\path[draw=drawColor,line width= 0.4pt,line join=round,line cap=round,fill=fillColor] (449.40,372.42) circle (  1.16);

\path[draw=drawColor,line width= 0.4pt,line join=round,line cap=round,fill=fillColor] (449.58,372.42) circle (  1.16);

\path[draw=drawColor,line width= 0.4pt,line join=round,line cap=round,fill=fillColor] (449.76,372.42) circle (  1.16);

\path[draw=drawColor,line width= 0.4pt,line join=round,line cap=round,fill=fillColor] (449.94,372.42) circle (  1.16);

\path[draw=drawColor,line width= 0.4pt,line join=round,line cap=round,fill=fillColor] (450.12,372.42) circle (  1.16);

\path[draw=drawColor,line width= 0.4pt,line join=round,line cap=round,fill=fillColor] (450.30,372.42) circle (  1.16);

\path[draw=drawColor,line width= 0.4pt,line join=round,line cap=round,fill=fillColor] (450.48,372.42) circle (  1.16);

\path[draw=drawColor,line width= 0.4pt,line join=round,line cap=round,fill=fillColor] (450.65,372.42) circle (  1.16);

\path[draw=drawColor,line width= 0.4pt,line join=round,line cap=round,fill=fillColor] (450.83,372.42) circle (  1.16);

\path[draw=drawColor,line width= 0.4pt,line join=round,line cap=round,fill=fillColor] (451.01,372.42) circle (  1.16);

\path[draw=drawColor,line width= 0.4pt,line join=round,line cap=round,fill=fillColor] (451.19,372.42) circle (  1.16);

\path[draw=drawColor,line width= 0.4pt,line join=round,line cap=round,fill=fillColor] (451.36,372.42) circle (  1.16);

\path[draw=drawColor,line width= 0.4pt,line join=round,line cap=round,fill=fillColor] (451.54,372.42) circle (  1.16);

\path[draw=drawColor,line width= 0.4pt,line join=round,line cap=round,fill=fillColor] (451.71,372.42) circle (  1.16);

\path[draw=drawColor,line width= 0.4pt,line join=round,line cap=round,fill=fillColor] (451.89,372.42) circle (  1.16);

\path[draw=drawColor,line width= 0.4pt,line join=round,line cap=round,fill=fillColor] (452.06,372.42) circle (  1.16);

\path[draw=drawColor,line width= 0.4pt,line join=round,line cap=round,fill=fillColor] (452.24,372.42) circle (  1.16);

\path[draw=drawColor,line width= 0.4pt,line join=round,line cap=round,fill=fillColor] (452.41,372.42) circle (  1.16);

\path[draw=drawColor,line width= 0.4pt,line join=round,line cap=round,fill=fillColor] (452.59,372.42) circle (  1.16);

\path[draw=drawColor,line width= 0.4pt,line join=round,line cap=round,fill=fillColor] (452.76,372.42) circle (  1.16);

\path[draw=drawColor,line width= 0.4pt,line join=round,line cap=round,fill=fillColor] (452.93,372.42) circle (  1.16);

\path[draw=drawColor,line width= 0.4pt,line join=round,line cap=round,fill=fillColor] (453.10,372.42) circle (  1.16);

\path[draw=drawColor,line width= 0.4pt,line join=round,line cap=round,fill=fillColor] (453.28,372.42) circle (  1.16);

\path[draw=drawColor,line width= 0.4pt,line join=round,line cap=round,fill=fillColor] (453.45,372.42) circle (  1.16);

\path[draw=drawColor,line width= 0.4pt,line join=round,line cap=round,fill=fillColor] (453.62,372.42) circle (  1.16);

\path[draw=drawColor,line width= 0.4pt,line join=round,line cap=round,fill=fillColor] (453.79,372.42) circle (  1.16);

\path[draw=drawColor,line width= 0.4pt,line join=round,line cap=round,fill=fillColor] (453.96,372.42) circle (  1.16);

\path[draw=drawColor,line width= 0.4pt,line join=round,line cap=round,fill=fillColor] (454.13,372.42) circle (  1.16);

\path[draw=drawColor,line width= 0.4pt,line join=round,line cap=round,fill=fillColor] (454.30,372.42) circle (  1.16);

\path[draw=drawColor,line width= 0.4pt,line join=round,line cap=round,fill=fillColor] (454.47,372.42) circle (  1.16);

\path[draw=drawColor,line width= 0.4pt,line join=round,line cap=round,fill=fillColor] (454.64,372.42) circle (  1.16);

\path[draw=drawColor,line width= 0.4pt,line join=round,line cap=round,fill=fillColor] (454.81,372.42) circle (  1.16);

\path[draw=drawColor,line width= 0.4pt,line join=round,line cap=round,fill=fillColor] (454.97,372.42) circle (  1.16);

\path[draw=drawColor,line width= 0.4pt,line join=round,line cap=round,fill=fillColor] (455.14,372.42) circle (  1.16);

\path[draw=drawColor,line width= 0.4pt,line join=round,line cap=round,fill=fillColor] (455.31,372.42) circle (  1.16);

\path[draw=drawColor,line width= 0.4pt,line join=round,line cap=round,fill=fillColor] (455.48,372.42) circle (  1.16);

\path[draw=drawColor,line width= 0.4pt,line join=round,line cap=round,fill=fillColor] (455.64,372.42) circle (  1.16);

\path[draw=drawColor,line width= 0.4pt,line join=round,line cap=round,fill=fillColor] (455.81,372.42) circle (  1.16);

\path[draw=drawColor,line width= 0.4pt,line join=round,line cap=round,fill=fillColor] (455.97,372.42) circle (  1.16);

\path[draw=drawColor,line width= 0.4pt,line join=round,line cap=round,fill=fillColor] (456.14,372.42) circle (  1.16);

\path[draw=drawColor,line width= 0.4pt,line join=round,line cap=round,fill=fillColor] (456.31,372.42) circle (  1.16);

\path[draw=drawColor,line width= 0.4pt,line join=round,line cap=round,fill=fillColor] (456.47,372.42) circle (  1.16);

\path[draw=drawColor,line width= 0.4pt,line join=round,line cap=round,fill=fillColor] (456.63,372.42) circle (  1.16);

\path[draw=drawColor,line width= 0.4pt,line join=round,line cap=round,fill=fillColor] (456.80,372.42) circle (  1.16);

\path[draw=drawColor,line width= 0.4pt,line join=round,line cap=round,fill=fillColor] (456.96,372.42) circle (  1.16);

\path[draw=drawColor,line width= 0.4pt,line join=round,line cap=round,fill=fillColor] (457.13,372.42) circle (  1.16);

\path[draw=drawColor,line width= 0.4pt,line join=round,line cap=round,fill=fillColor] (457.29,372.42) circle (  1.16);

\path[draw=drawColor,line width= 0.4pt,line join=round,line cap=round,fill=fillColor] (457.45,372.42) circle (  1.16);

\path[draw=drawColor,line width= 0.4pt,line join=round,line cap=round,fill=fillColor] (457.61,372.42) circle (  1.16);

\path[draw=drawColor,line width= 0.4pt,line join=round,line cap=round,fill=fillColor] (457.78,372.42) circle (  1.16);

\path[draw=drawColor,line width= 0.4pt,line join=round,line cap=round,fill=fillColor] (457.94,372.42) circle (  1.16);

\path[draw=drawColor,line width= 0.4pt,line join=round,line cap=round,fill=fillColor] (458.10,372.42) circle (  1.16);

\path[draw=drawColor,line width= 0.4pt,line join=round,line cap=round,fill=fillColor] (458.26,372.42) circle (  1.16);

\path[draw=drawColor,line width= 0.4pt,line join=round,line cap=round,fill=fillColor] (458.42,372.42) circle (  1.16);

\path[draw=drawColor,line width= 0.4pt,line join=round,line cap=round,fill=fillColor] (458.58,372.42) circle (  1.16);

\path[draw=drawColor,line width= 0.4pt,line join=round,line cap=round,fill=fillColor] (458.74,372.42) circle (  1.16);

\path[draw=drawColor,line width= 0.4pt,line join=round,line cap=round,fill=fillColor] (458.90,372.42) circle (  1.16);

\path[draw=drawColor,line width= 0.4pt,line join=round,line cap=round,fill=fillColor] (459.06,372.42) circle (  1.16);

\path[draw=drawColor,line width= 0.4pt,line join=round,line cap=round,fill=fillColor] (459.22,372.42) circle (  1.16);

\path[draw=drawColor,line width= 0.4pt,line join=round,line cap=round,fill=fillColor] (459.38,372.42) circle (  1.16);

\path[draw=drawColor,line width= 0.4pt,line join=round,line cap=round,fill=fillColor] (459.53,372.42) circle (  1.16);

\path[draw=drawColor,line width= 0.4pt,line join=round,line cap=round,fill=fillColor] (459.69,372.42) circle (  1.16);

\path[draw=drawColor,line width= 0.4pt,line join=round,line cap=round,fill=fillColor] (459.85,372.42) circle (  1.16);

\path[draw=drawColor,line width= 0.4pt,line join=round,line cap=round,fill=fillColor] (460.01,372.42) circle (  1.16);

\path[draw=drawColor,line width= 0.4pt,line join=round,line cap=round,fill=fillColor] (460.16,372.42) circle (  1.16);

\path[draw=drawColor,line width= 0.4pt,line join=round,line cap=round,fill=fillColor] (460.32,372.42) circle (  1.16);

\path[draw=drawColor,line width= 0.4pt,line join=round,line cap=round,fill=fillColor] (460.48,372.42) circle (  1.16);

\path[draw=drawColor,line width= 0.4pt,line join=round,line cap=round,fill=fillColor] (460.63,372.42) circle (  1.16);

\path[draw=drawColor,line width= 0.4pt,line join=round,line cap=round,fill=fillColor] (460.79,372.42) circle (  1.16);

\path[draw=drawColor,line width= 0.4pt,line join=round,line cap=round,fill=fillColor] (460.94,372.42) circle (  1.16);

\path[draw=drawColor,line width= 0.4pt,line join=round,line cap=round,fill=fillColor] (461.10,372.42) circle (  1.16);

\path[draw=drawColor,line width= 0.4pt,line join=round,line cap=round,fill=fillColor] (461.25,372.42) circle (  1.16);

\path[draw=drawColor,line width= 0.4pt,line join=round,line cap=round,fill=fillColor] (461.41,372.42) circle (  1.16);

\path[draw=drawColor,line width= 0.4pt,line join=round,line cap=round,fill=fillColor] (461.56,372.42) circle (  1.16);

\path[draw=drawColor,line width= 0.4pt,line join=round,line cap=round,fill=fillColor] (461.72,372.42) circle (  1.16);

\path[draw=drawColor,line width= 0.4pt,line join=round,line cap=round,fill=fillColor] (461.87,372.42) circle (  1.16);

\path[draw=drawColor,line width= 0.4pt,line join=round,line cap=round,fill=fillColor] (462.02,372.42) circle (  1.16);

\path[draw=drawColor,line width= 0.4pt,line join=round,line cap=round,fill=fillColor] (462.18,372.42) circle (  1.16);

\path[draw=drawColor,line width= 0.4pt,line join=round,line cap=round,fill=fillColor] (462.33,372.42) circle (  1.16);

\path[draw=drawColor,line width= 0.4pt,line join=round,line cap=round,fill=fillColor] (462.48,372.42) circle (  1.16);

\path[draw=drawColor,line width= 0.4pt,line join=round,line cap=round,fill=fillColor] (462.63,372.42) circle (  1.16);

\path[draw=drawColor,line width= 0.4pt,line join=round,line cap=round,fill=fillColor] (462.78,372.42) circle (  1.16);

\path[draw=drawColor,line width= 0.4pt,line join=round,line cap=round,fill=fillColor] (462.94,372.42) circle (  1.16);

\path[draw=drawColor,line width= 0.4pt,line join=round,line cap=round,fill=fillColor] (463.09,372.42) circle (  1.16);

\path[draw=drawColor,line width= 0.4pt,line join=round,line cap=round,fill=fillColor] (463.24,372.42) circle (  1.16);

\path[draw=drawColor,line width= 0.4pt,line join=round,line cap=round,fill=fillColor] (463.39,372.42) circle (  1.16);

\path[draw=drawColor,line width= 0.4pt,line join=round,line cap=round,fill=fillColor] (463.54,372.42) circle (  1.16);

\path[draw=drawColor,line width= 0.4pt,line join=round,line cap=round,fill=fillColor] (463.69,372.42) circle (  1.16);

\path[draw=drawColor,line width= 0.4pt,line join=round,line cap=round,fill=fillColor] (463.84,372.42) circle (  1.16);

\path[draw=drawColor,line width= 0.4pt,line join=round,line cap=round,fill=fillColor] (463.99,372.42) circle (  1.16);

\path[draw=drawColor,line width= 0.4pt,line join=round,line cap=round,fill=fillColor] (464.14,372.42) circle (  1.16);

\path[draw=drawColor,line width= 0.4pt,line join=round,line cap=round,fill=fillColor] (464.29,372.42) circle (  1.16);

\path[draw=drawColor,line width= 0.4pt,line join=round,line cap=round,fill=fillColor] (464.44,372.42) circle (  1.16);

\path[draw=drawColor,line width= 0.4pt,line join=round,line cap=round,fill=fillColor] (464.58,372.42) circle (  1.16);

\path[draw=drawColor,line width= 0.4pt,line join=round,line cap=round,fill=fillColor] (464.73,372.42) circle (  1.16);

\path[draw=drawColor,line width= 0.4pt,line join=round,line cap=round,fill=fillColor] (464.88,372.42) circle (  1.16);

\path[draw=drawColor,line width= 0.4pt,line join=round,line cap=round,fill=fillColor] (465.03,372.42) circle (  1.16);

\path[draw=drawColor,line width= 0.4pt,line join=round,line cap=round,fill=fillColor] (465.17,372.42) circle (  1.16);

\path[draw=drawColor,line width= 0.4pt,line join=round,line cap=round,fill=fillColor] (465.32,372.42) circle (  1.16);

\path[draw=drawColor,line width= 0.4pt,line join=round,line cap=round,fill=fillColor] (465.47,372.42) circle (  1.16);

\path[draw=drawColor,line width= 0.4pt,line join=round,line cap=round,fill=fillColor] (465.61,372.42) circle (  1.16);

\path[draw=drawColor,line width= 0.4pt,line join=round,line cap=round,fill=fillColor] (465.76,372.42) circle (  1.16);

\path[draw=drawColor,line width= 0.4pt,line join=round,line cap=round,fill=fillColor] (465.91,372.42) circle (  1.16);

\path[draw=drawColor,line width= 0.4pt,line join=round,line cap=round,fill=fillColor] (466.05,372.42) circle (  1.16);

\path[draw=drawColor,line width= 0.4pt,line join=round,line cap=round,fill=fillColor] (466.20,372.42) circle (  1.16);

\path[draw=drawColor,line width= 0.4pt,line join=round,line cap=round,fill=fillColor] (466.34,372.42) circle (  1.16);

\path[draw=drawColor,line width= 0.4pt,line join=round,line cap=round,fill=fillColor] (466.49,372.42) circle (  1.16);

\path[draw=drawColor,line width= 0.4pt,line join=round,line cap=round,fill=fillColor] (466.63,372.42) circle (  1.16);

\path[draw=drawColor,line width= 0.4pt,line join=round,line cap=round,fill=fillColor] (466.78,372.42) circle (  1.16);

\path[draw=drawColor,line width= 0.4pt,line join=round,line cap=round,fill=fillColor] (466.92,372.42) circle (  1.16);

\path[draw=drawColor,line width= 0.4pt,line join=round,line cap=round,fill=fillColor] (467.06,372.42) circle (  1.16);

\path[draw=drawColor,line width= 0.4pt,line join=round,line cap=round,fill=fillColor] (467.21,372.42) circle (  1.16);

\path[draw=drawColor,line width= 0.4pt,line join=round,line cap=round,fill=fillColor] (467.35,372.42) circle (  1.16);

\path[draw=drawColor,line width= 0.4pt,line join=round,line cap=round,fill=fillColor] (467.49,372.42) circle (  1.16);

\path[draw=drawColor,line width= 0.4pt,line join=round,line cap=round,fill=fillColor] (467.64,372.42) circle (  1.16);

\path[draw=drawColor,line width= 0.4pt,line join=round,line cap=round,fill=fillColor] (467.78,372.42) circle (  1.16);

\path[draw=drawColor,line width= 0.4pt,line join=round,line cap=round,fill=fillColor] (467.92,372.42) circle (  1.16);

\path[draw=drawColor,line width= 0.4pt,line join=round,line cap=round,fill=fillColor] (468.06,372.42) circle (  1.16);

\path[draw=drawColor,line width= 0.4pt,line join=round,line cap=round,fill=fillColor] (468.21,372.42) circle (  1.16);

\path[draw=drawColor,line width= 0.4pt,line join=round,line cap=round,fill=fillColor] (468.35,372.42) circle (  1.16);

\path[draw=drawColor,line width= 0.4pt,line join=round,line cap=round,fill=fillColor] (468.49,372.42) circle (  1.16);

\path[draw=drawColor,line width= 0.4pt,line join=round,line cap=round,fill=fillColor] (468.63,372.42) circle (  1.16);

\path[draw=drawColor,line width= 0.4pt,line join=round,line cap=round,fill=fillColor] (468.77,372.42) circle (  1.16);

\path[draw=drawColor,line width= 0.4pt,line join=round,line cap=round,fill=fillColor] (468.91,372.42) circle (  1.16);

\path[draw=drawColor,line width= 0.4pt,line join=round,line cap=round,fill=fillColor] (469.05,372.42) circle (  1.16);

\path[draw=drawColor,line width= 0.4pt,line join=round,line cap=round,fill=fillColor] (469.19,372.42) circle (  1.16);

\path[draw=drawColor,line width= 0.4pt,line join=round,line cap=round,fill=fillColor] (469.33,372.42) circle (  1.16);

\path[draw=drawColor,line width= 0.4pt,line join=round,line cap=round,fill=fillColor] (469.47,372.42) circle (  1.16);

\path[draw=drawColor,line width= 0.4pt,line join=round,line cap=round,fill=fillColor] (469.61,372.42) circle (  1.16);

\path[draw=drawColor,line width= 0.4pt,line join=round,line cap=round,fill=fillColor] (469.75,372.42) circle (  1.16);

\path[draw=drawColor,line width= 0.4pt,line join=round,line cap=round,fill=fillColor] (469.89,372.42) circle (  1.16);

\path[draw=drawColor,line width= 0.4pt,line join=round,line cap=round,fill=fillColor] (470.03,372.42) circle (  1.16);

\path[draw=drawColor,line width= 0.4pt,line join=round,line cap=round,fill=fillColor] (470.17,372.42) circle (  1.16);

\path[draw=drawColor,line width= 0.4pt,line join=round,line cap=round,fill=fillColor] (470.30,372.42) circle (  1.16);

\path[draw=drawColor,line width= 0.4pt,line join=round,line cap=round,fill=fillColor] (470.44,372.42) circle (  1.16);

\path[draw=drawColor,line width= 0.4pt,line join=round,line cap=round,fill=fillColor] (470.58,372.42) circle (  1.16);

\path[draw=drawColor,line width= 0.4pt,line join=round,line cap=round,fill=fillColor] (470.72,372.42) circle (  1.16);

\path[draw=drawColor,line width= 0.4pt,line join=round,line cap=round,fill=fillColor] (470.85,372.42) circle (  1.16);

\path[draw=drawColor,line width= 0.4pt,line join=round,line cap=round,fill=fillColor] (470.99,372.42) circle (  1.16);

\path[draw=drawColor,line width= 0.4pt,line join=round,line cap=round,fill=fillColor] (471.13,372.42) circle (  1.16);

\path[draw=drawColor,line width= 0.4pt,line join=round,line cap=round,fill=fillColor] (471.27,372.42) circle (  1.16);

\path[draw=drawColor,line width= 0.4pt,line join=round,line cap=round,fill=fillColor] (471.40,372.42) circle (  1.16);

\path[draw=drawColor,line width= 0.4pt,line join=round,line cap=round,fill=fillColor] (471.54,372.42) circle (  1.16);

\path[draw=drawColor,line width= 0.4pt,line join=round,line cap=round,fill=fillColor] (471.67,372.42) circle (  1.16);

\path[draw=drawColor,line width= 0.4pt,line join=round,line cap=round,fill=fillColor] (471.81,372.42) circle (  1.16);

\path[draw=drawColor,line width= 0.4pt,line join=round,line cap=round,fill=fillColor] (471.94,372.42) circle (  1.16);

\path[draw=drawColor,line width= 0.4pt,line join=round,line cap=round,fill=fillColor] (472.08,372.42) circle (  1.16);

\path[draw=drawColor,line width= 0.4pt,line join=round,line cap=round,fill=fillColor] (472.22,372.42) circle (  1.16);

\path[draw=drawColor,line width= 0.4pt,line join=round,line cap=round,fill=fillColor] (472.35,372.42) circle (  1.16);

\path[draw=drawColor,line width= 0.4pt,line join=round,line cap=round,fill=fillColor] (472.48,372.42) circle (  1.16);

\path[draw=drawColor,line width= 0.4pt,line join=round,line cap=round,fill=fillColor] (472.62,372.42) circle (  1.16);

\path[draw=drawColor,line width= 0.4pt,line join=round,line cap=round,fill=fillColor] (472.75,372.42) circle (  1.16);

\path[draw=drawColor,line width= 0.4pt,line join=round,line cap=round,fill=fillColor] (472.89,372.42) circle (  1.16);

\path[draw=drawColor,line width= 0.4pt,line join=round,line cap=round,fill=fillColor] (473.02,372.42) circle (  1.16);

\path[draw=drawColor,line width= 0.4pt,line join=round,line cap=round,fill=fillColor] (473.15,372.42) circle (  1.16);

\path[draw=drawColor,line width= 0.4pt,line join=round,line cap=round,fill=fillColor] (473.29,372.42) circle (  1.16);

\path[draw=drawColor,line width= 0.4pt,line join=round,line cap=round,fill=fillColor] (473.42,372.42) circle (  1.16);

\path[draw=drawColor,line width= 0.4pt,line join=round,line cap=round,fill=fillColor] (473.55,372.42) circle (  1.16);

\path[draw=drawColor,line width= 0.4pt,line join=round,line cap=round,fill=fillColor] (473.69,372.42) circle (  1.16);

\path[draw=drawColor,line width= 0.4pt,line join=round,line cap=round,fill=fillColor] (473.82,372.42) circle (  1.16);

\path[draw=drawColor,line width= 0.4pt,line join=round,line cap=round,fill=fillColor] (473.95,372.42) circle (  1.16);

\path[draw=drawColor,line width= 0.4pt,line join=round,line cap=round,fill=fillColor] (474.08,372.42) circle (  1.16);

\path[draw=drawColor,line width= 0.4pt,line join=round,line cap=round,fill=fillColor] (474.22,372.42) circle (  1.16);

\path[draw=drawColor,line width= 0.4pt,line join=round,line cap=round,fill=fillColor] (474.35,372.42) circle (  1.16);

\path[draw=drawColor,line width= 0.4pt,line join=round,line cap=round,fill=fillColor] (474.48,372.42) circle (  1.16);

\path[draw=drawColor,line width= 0.4pt,line join=round,line cap=round,fill=fillColor] (474.61,372.42) circle (  1.16);

\path[draw=drawColor,line width= 0.4pt,line join=round,line cap=round,fill=fillColor] (474.74,372.42) circle (  1.16);

\path[draw=drawColor,line width= 0.4pt,line join=round,line cap=round,fill=fillColor] (474.87,372.42) circle (  1.16);

\path[draw=drawColor,line width= 0.4pt,line join=round,line cap=round,fill=fillColor] (475.00,372.42) circle (  1.16);

\path[draw=drawColor,line width= 0.4pt,line join=round,line cap=round,fill=fillColor] (475.13,372.42) circle (  1.16);

\path[draw=drawColor,line width= 0.4pt,line join=round,line cap=round,fill=fillColor] (475.26,372.42) circle (  1.16);

\path[draw=drawColor,line width= 0.4pt,line join=round,line cap=round,fill=fillColor] (475.39,372.42) circle (  1.16);

\path[draw=drawColor,line width= 0.4pt,line join=round,line cap=round,fill=fillColor] (475.52,372.42) circle (  1.16);

\path[draw=drawColor,line width= 0.4pt,line join=round,line cap=round,fill=fillColor] (475.65,372.42) circle (  1.16);

\path[draw=drawColor,line width= 0.4pt,line join=round,line cap=round,fill=fillColor] (475.78,372.42) circle (  1.16);

\path[draw=drawColor,line width= 0.4pt,line join=round,line cap=round,fill=fillColor] (475.91,372.42) circle (  1.16);

\path[draw=drawColor,line width= 0.4pt,line join=round,line cap=round,fill=fillColor] (476.04,372.42) circle (  1.16);

\path[draw=drawColor,line width= 0.4pt,line join=round,line cap=round,fill=fillColor] (476.17,372.42) circle (  1.16);

\path[draw=drawColor,line width= 0.4pt,line join=round,line cap=round,fill=fillColor] (476.30,372.42) circle (  1.16);

\path[draw=drawColor,line width= 0.4pt,line join=round,line cap=round,fill=fillColor] (476.43,372.42) circle (  1.16);

\path[draw=drawColor,line width= 0.4pt,line join=round,line cap=round,fill=fillColor] (476.56,372.42) circle (  1.16);

\path[draw=drawColor,line width= 0.4pt,line join=round,line cap=round,fill=fillColor] (476.68,372.42) circle (  1.16);

\path[draw=drawColor,line width= 0.4pt,line join=round,line cap=round,fill=fillColor] (476.81,372.42) circle (  1.16);

\path[draw=drawColor,line width= 0.4pt,line join=round,line cap=round,fill=fillColor] (476.94,372.42) circle (  1.16);

\path[draw=drawColor,line width= 0.4pt,line join=round,line cap=round,fill=fillColor] (477.07,372.42) circle (  1.16);

\path[draw=drawColor,line width= 0.4pt,line join=round,line cap=round,fill=fillColor] (477.19,372.42) circle (  1.16);

\path[draw=drawColor,line width= 0.4pt,line join=round,line cap=round,fill=fillColor] (477.32,372.42) circle (  1.16);

\path[draw=drawColor,line width= 0.4pt,line join=round,line cap=round,fill=fillColor] (477.45,372.42) circle (  1.16);

\path[draw=drawColor,line width= 0.4pt,line join=round,line cap=round,fill=fillColor] (477.57,372.42) circle (  1.16);

\path[draw=drawColor,line width= 0.4pt,line join=round,line cap=round,fill=fillColor] (477.70,372.42) circle (  1.16);

\path[draw=drawColor,line width= 0.4pt,line join=round,line cap=round,fill=fillColor] (477.83,372.42) circle (  1.16);

\path[draw=drawColor,line width= 0.4pt,line join=round,line cap=round,fill=fillColor] (477.95,372.42) circle (  1.16);

\path[draw=drawColor,line width= 0.4pt,line join=round,line cap=round,fill=fillColor] (478.08,372.42) circle (  1.16);

\path[draw=drawColor,line width= 0.4pt,line join=round,line cap=round,fill=fillColor] (478.21,372.42) circle (  1.16);
\definecolor[named]{drawColor}{rgb}{0.22,0.49,0.72}
\definecolor[named]{fillColor}{rgb}{0.22,0.49,0.72}

\path[draw=drawColor,line width= 0.4pt,line join=round,line cap=round,fill=fillColor] (327.82,455.17) circle (  1.16);

\path[draw=drawColor,line width= 0.4pt,line join=round,line cap=round,fill=fillColor] (333.65,453.66) circle (  1.16);

\path[draw=drawColor,line width= 0.4pt,line join=round,line cap=round,fill=fillColor] (337.74,453.34) circle (  1.16);

\path[draw=drawColor,line width= 0.4pt,line join=round,line cap=round,fill=fillColor] (340.99,453.10) circle (  1.16);

\path[draw=drawColor,line width= 0.4pt,line join=round,line cap=round,fill=fillColor] (343.74,452.77) circle (  1.16);

\path[draw=drawColor,line width= 0.4pt,line join=round,line cap=round,fill=fillColor] (346.14,452.68) circle (  1.16);

\path[draw=drawColor,line width= 0.4pt,line join=round,line cap=round,fill=fillColor] (348.29,452.59) circle (  1.16);

\path[draw=drawColor,line width= 0.4pt,line join=round,line cap=round,fill=fillColor] (350.24,452.56) circle (  1.16);

\path[draw=drawColor,line width= 0.4pt,line join=round,line cap=round,fill=fillColor] (352.04,451.73) circle (  1.16);

\path[draw=drawColor,line width= 0.4pt,line join=round,line cap=round,fill=fillColor] (353.70,451.46) circle (  1.16);

\path[draw=drawColor,line width= 0.4pt,line join=round,line cap=round,fill=fillColor] (355.26,450.50) circle (  1.16);

\path[draw=drawColor,line width= 0.4pt,line join=round,line cap=round,fill=fillColor] (356.73,450.14) circle (  1.16);

\path[draw=drawColor,line width= 0.4pt,line join=round,line cap=round,fill=fillColor] (358.12,448.30) circle (  1.16);

\path[draw=drawColor,line width= 0.4pt,line join=round,line cap=round,fill=fillColor] (359.44,448.21) circle (  1.16);

\path[draw=drawColor,line width= 0.4pt,line join=round,line cap=round,fill=fillColor] (360.69,447.62) circle (  1.16);

\path[draw=drawColor,line width= 0.4pt,line join=round,line cap=round,fill=fillColor] (361.90,447.41) circle (  1.16);

\path[draw=drawColor,line width= 0.4pt,line join=round,line cap=round,fill=fillColor] (363.05,446.60) circle (  1.16);

\path[draw=drawColor,line width= 0.4pt,line join=round,line cap=round,fill=fillColor] (364.16,445.52) circle (  1.16);

\path[draw=drawColor,line width= 0.4pt,line join=round,line cap=round,fill=fillColor] (365.23,445.05) circle (  1.16);

\path[draw=drawColor,line width= 0.4pt,line join=round,line cap=round,fill=fillColor] (366.26,444.46) circle (  1.16);

\path[draw=drawColor,line width= 0.4pt,line join=round,line cap=round,fill=fillColor] (367.25,444.33) circle (  1.16);

\path[draw=drawColor,line width= 0.4pt,line join=round,line cap=round,fill=fillColor] (368.22,443.57) circle (  1.16);

\path[draw=drawColor,line width= 0.4pt,line join=round,line cap=round,fill=fillColor] (369.16,442.38) circle (  1.16);

\path[draw=drawColor,line width= 0.4pt,line join=round,line cap=round,fill=fillColor] (370.07,440.61) circle (  1.16);

\path[draw=drawColor,line width= 0.4pt,line join=round,line cap=round,fill=fillColor] (370.96,440.58) circle (  1.16);

\path[draw=drawColor,line width= 0.4pt,line join=round,line cap=round,fill=fillColor] (371.82,438.78) circle (  1.16);

\path[draw=drawColor,line width= 0.4pt,line join=round,line cap=round,fill=fillColor] (372.66,438.24) circle (  1.16);

\path[draw=drawColor,line width= 0.4pt,line join=round,line cap=round,fill=fillColor] (373.48,437.56) circle (  1.16);

\path[draw=drawColor,line width= 0.4pt,line join=round,line cap=round,fill=fillColor] (374.28,436.96) circle (  1.16);

\path[draw=drawColor,line width= 0.4pt,line join=round,line cap=round,fill=fillColor] (375.06,435.51) circle (  1.16);

\path[draw=drawColor,line width= 0.4pt,line join=round,line cap=round,fill=fillColor] (375.83,433.44) circle (  1.16);

\path[draw=drawColor,line width= 0.4pt,line join=round,line cap=round,fill=fillColor] (376.58,428.89) circle (  1.16);

\path[draw=drawColor,line width= 0.4pt,line join=round,line cap=round,fill=fillColor] (377.31,428.81) circle (  1.16);

\path[draw=drawColor,line width= 0.4pt,line join=round,line cap=round,fill=fillColor] (378.03,428.31) circle (  1.16);

\path[draw=drawColor,line width= 0.4pt,line join=round,line cap=round,fill=fillColor] (378.74,428.22) circle (  1.16);

\path[draw=drawColor,line width= 0.4pt,line join=round,line cap=round,fill=fillColor] (379.43,427.98) circle (  1.16);

\path[draw=drawColor,line width= 0.4pt,line join=round,line cap=round,fill=fillColor] (380.11,427.85) circle (  1.16);

\path[draw=drawColor,line width= 0.4pt,line join=round,line cap=round,fill=fillColor] (380.77,426.01) circle (  1.16);

\path[draw=drawColor,line width= 0.4pt,line join=round,line cap=round,fill=fillColor] (381.43,425.85) circle (  1.16);

\path[draw=drawColor,line width= 0.4pt,line join=round,line cap=round,fill=fillColor] (382.07,425.13) circle (  1.16);

\path[draw=drawColor,line width= 0.4pt,line join=round,line cap=round,fill=fillColor] (382.71,423.83) circle (  1.16);

\path[draw=drawColor,line width= 0.4pt,line join=round,line cap=round,fill=fillColor] (383.33,423.49) circle (  1.16);

\path[draw=drawColor,line width= 0.4pt,line join=round,line cap=round,fill=fillColor] (383.94,423.25) circle (  1.16);

\path[draw=drawColor,line width= 0.4pt,line join=round,line cap=round,fill=fillColor] (384.55,423.15) circle (  1.16);

\path[draw=drawColor,line width= 0.4pt,line join=round,line cap=round,fill=fillColor] (385.14,422.83) circle (  1.16);

\path[draw=drawColor,line width= 0.4pt,line join=round,line cap=round,fill=fillColor] (385.73,422.63) circle (  1.16);

\path[draw=drawColor,line width= 0.4pt,line join=round,line cap=round,fill=fillColor] (386.31,421.87) circle (  1.16);

\path[draw=drawColor,line width= 0.4pt,line join=round,line cap=round,fill=fillColor] (386.88,421.26) circle (  1.16);

\path[draw=drawColor,line width= 0.4pt,line join=round,line cap=round,fill=fillColor] (387.44,420.36) circle (  1.16);

\path[draw=drawColor,line width= 0.4pt,line join=round,line cap=round,fill=fillColor] (387.99,419.82) circle (  1.16);

\path[draw=drawColor,line width= 0.4pt,line join=round,line cap=round,fill=fillColor] (388.54,419.78) circle (  1.16);

\path[draw=drawColor,line width= 0.4pt,line join=round,line cap=round,fill=fillColor] (389.08,419.57) circle (  1.16);

\path[draw=drawColor,line width= 0.4pt,line join=round,line cap=round,fill=fillColor] (389.61,419.00) circle (  1.16);

\path[draw=drawColor,line width= 0.4pt,line join=round,line cap=round,fill=fillColor] (390.14,418.94) circle (  1.16);

\path[draw=drawColor,line width= 0.4pt,line join=round,line cap=round,fill=fillColor] (390.66,418.94) circle (  1.16);

\path[draw=drawColor,line width= 0.4pt,line join=round,line cap=round,fill=fillColor] (391.17,418.91) circle (  1.16);

\path[draw=drawColor,line width= 0.4pt,line join=round,line cap=round,fill=fillColor] (391.68,418.40) circle (  1.16);

\path[draw=drawColor,line width= 0.4pt,line join=round,line cap=round,fill=fillColor] (392.18,418.28) circle (  1.16);

\path[draw=drawColor,line width= 0.4pt,line join=round,line cap=round,fill=fillColor] (392.68,418.26) circle (  1.16);

\path[draw=drawColor,line width= 0.4pt,line join=round,line cap=round,fill=fillColor] (393.17,418.22) circle (  1.16);

\path[draw=drawColor,line width= 0.4pt,line join=round,line cap=round,fill=fillColor] (393.65,417.86) circle (  1.16);

\path[draw=drawColor,line width= 0.4pt,line join=round,line cap=round,fill=fillColor] (394.13,417.64) circle (  1.16);

\path[draw=drawColor,line width= 0.4pt,line join=round,line cap=round,fill=fillColor] (394.61,417.52) circle (  1.16);

\path[draw=drawColor,line width= 0.4pt,line join=round,line cap=round,fill=fillColor] (395.08,417.40) circle (  1.16);

\path[draw=drawColor,line width= 0.4pt,line join=round,line cap=round,fill=fillColor] (395.54,417.30) circle (  1.16);

\path[draw=drawColor,line width= 0.4pt,line join=round,line cap=round,fill=fillColor] (396.00,417.13) circle (  1.16);

\path[draw=drawColor,line width= 0.4pt,line join=round,line cap=round,fill=fillColor] (396.46,417.05) circle (  1.16);

\path[draw=drawColor,line width= 0.4pt,line join=round,line cap=round,fill=fillColor] (396.91,417.02) circle (  1.16);

\path[draw=drawColor,line width= 0.4pt,line join=round,line cap=round,fill=fillColor] (397.35,416.96) circle (  1.16);

\path[draw=drawColor,line width= 0.4pt,line join=round,line cap=round,fill=fillColor] (397.80,416.89) circle (  1.16);

\path[draw=drawColor,line width= 0.4pt,line join=round,line cap=round,fill=fillColor] (398.23,416.64) circle (  1.16);

\path[draw=drawColor,line width= 0.4pt,line join=round,line cap=round,fill=fillColor] (398.67,416.59) circle (  1.16);

\path[draw=drawColor,line width= 0.4pt,line join=round,line cap=round,fill=fillColor] (399.10,416.49) circle (  1.16);

\path[draw=drawColor,line width= 0.4pt,line join=round,line cap=round,fill=fillColor] (399.52,416.33) circle (  1.16);

\path[draw=drawColor,line width= 0.4pt,line join=round,line cap=round,fill=fillColor] (399.94,416.24) circle (  1.16);

\path[draw=drawColor,line width= 0.4pt,line join=round,line cap=round,fill=fillColor] (400.36,416.16) circle (  1.16);

\path[draw=drawColor,line width= 0.4pt,line join=round,line cap=round,fill=fillColor] (400.78,415.86) circle (  1.16);

\path[draw=drawColor,line width= 0.4pt,line join=round,line cap=round,fill=fillColor] (401.19,415.62) circle (  1.16);

\path[draw=drawColor,line width= 0.4pt,line join=round,line cap=round,fill=fillColor] (401.60,415.52) circle (  1.16);

\path[draw=drawColor,line width= 0.4pt,line join=round,line cap=round,fill=fillColor] (402.00,415.20) circle (  1.16);

\path[draw=drawColor,line width= 0.4pt,line join=round,line cap=round,fill=fillColor] (402.40,414.69) circle (  1.16);

\path[draw=drawColor,line width= 0.4pt,line join=round,line cap=round,fill=fillColor] (402.80,414.66) circle (  1.16);

\path[draw=drawColor,line width= 0.4pt,line join=round,line cap=round,fill=fillColor] (403.19,414.27) circle (  1.16);

\path[draw=drawColor,line width= 0.4pt,line join=round,line cap=round,fill=fillColor] (403.58,413.80) circle (  1.16);

\path[draw=drawColor,line width= 0.4pt,line join=round,line cap=round,fill=fillColor] (403.97,413.46) circle (  1.16);

\path[draw=drawColor,line width= 0.4pt,line join=round,line cap=round,fill=fillColor] (404.36,413.41) circle (  1.16);

\path[draw=drawColor,line width= 0.4pt,line join=round,line cap=round,fill=fillColor] (404.74,413.39) circle (  1.16);

\path[draw=drawColor,line width= 0.4pt,line join=round,line cap=round,fill=fillColor] (405.12,413.18) circle (  1.16);

\path[draw=drawColor,line width= 0.4pt,line join=round,line cap=round,fill=fillColor] (405.49,413.02) circle (  1.16);

\path[draw=drawColor,line width= 0.4pt,line join=round,line cap=round,fill=fillColor] (405.87,412.84) circle (  1.16);

\path[draw=drawColor,line width= 0.4pt,line join=round,line cap=round,fill=fillColor] (406.24,412.68) circle (  1.16);

\path[draw=drawColor,line width= 0.4pt,line join=round,line cap=round,fill=fillColor] (406.61,412.13) circle (  1.16);

\path[draw=drawColor,line width= 0.4pt,line join=round,line cap=round,fill=fillColor] (406.97,412.04) circle (  1.16);

\path[draw=drawColor,line width= 0.4pt,line join=round,line cap=round,fill=fillColor] (407.34,411.93) circle (  1.16);

\path[draw=drawColor,line width= 0.4pt,line join=round,line cap=round,fill=fillColor] (407.70,411.61) circle (  1.16);

\path[draw=drawColor,line width= 0.4pt,line join=round,line cap=round,fill=fillColor] (408.05,411.53) circle (  1.16);

\path[draw=drawColor,line width= 0.4pt,line join=round,line cap=round,fill=fillColor] (408.41,411.39) circle (  1.16);

\path[draw=drawColor,line width= 0.4pt,line join=round,line cap=round,fill=fillColor] (408.76,411.26) circle (  1.16);

\path[draw=drawColor,line width= 0.4pt,line join=round,line cap=round,fill=fillColor] (409.11,411.14) circle (  1.16);

\path[draw=drawColor,line width= 0.4pt,line join=round,line cap=round,fill=fillColor] (409.46,410.98) circle (  1.16);

\path[draw=drawColor,line width= 0.4pt,line join=round,line cap=round,fill=fillColor] (409.80,410.69) circle (  1.16);

\path[draw=drawColor,line width= 0.4pt,line join=round,line cap=round,fill=fillColor] (410.15,410.56) circle (  1.16);

\path[draw=drawColor,line width= 0.4pt,line join=round,line cap=round,fill=fillColor] (410.49,410.40) circle (  1.16);

\path[draw=drawColor,line width= 0.4pt,line join=round,line cap=round,fill=fillColor] (410.83,410.26) circle (  1.16);

\path[draw=drawColor,line width= 0.4pt,line join=round,line cap=round,fill=fillColor] (411.17,410.06) circle (  1.16);

\path[draw=drawColor,line width= 0.4pt,line join=round,line cap=round,fill=fillColor] (411.50,410.03) circle (  1.16);

\path[draw=drawColor,line width= 0.4pt,line join=round,line cap=round,fill=fillColor] (411.83,410.02) circle (  1.16);

\path[draw=drawColor,line width= 0.4pt,line join=round,line cap=round,fill=fillColor] (412.16,409.68) circle (  1.16);

\path[draw=drawColor,line width= 0.4pt,line join=round,line cap=round,fill=fillColor] (412.49,409.39) circle (  1.16);

\path[draw=drawColor,line width= 0.4pt,line join=round,line cap=round,fill=fillColor] (412.82,409.20) circle (  1.16);

\path[draw=drawColor,line width= 0.4pt,line join=round,line cap=round,fill=fillColor] (413.14,409.19) circle (  1.16);

\path[draw=drawColor,line width= 0.4pt,line join=round,line cap=round,fill=fillColor] (413.47,409.19) circle (  1.16);

\path[draw=drawColor,line width= 0.4pt,line join=round,line cap=round,fill=fillColor] (413.79,408.85) circle (  1.16);

\path[draw=drawColor,line width= 0.4pt,line join=round,line cap=round,fill=fillColor] (414.10,408.78) circle (  1.16);

\path[draw=drawColor,line width= 0.4pt,line join=round,line cap=round,fill=fillColor] (414.42,408.60) circle (  1.16);

\path[draw=drawColor,line width= 0.4pt,line join=round,line cap=round,fill=fillColor] (414.74,408.41) circle (  1.16);

\path[draw=drawColor,line width= 0.4pt,line join=round,line cap=round,fill=fillColor] (415.05,408.28) circle (  1.16);

\path[draw=drawColor,line width= 0.4pt,line join=round,line cap=round,fill=fillColor] (415.36,408.23) circle (  1.16);

\path[draw=drawColor,line width= 0.4pt,line join=round,line cap=round,fill=fillColor] (415.67,408.18) circle (  1.16);

\path[draw=drawColor,line width= 0.4pt,line join=round,line cap=round,fill=fillColor] (415.98,408.16) circle (  1.16);

\path[draw=drawColor,line width= 0.4pt,line join=round,line cap=round,fill=fillColor] (416.29,408.05) circle (  1.16);

\path[draw=drawColor,line width= 0.4pt,line join=round,line cap=round,fill=fillColor] (416.59,407.98) circle (  1.16);

\path[draw=drawColor,line width= 0.4pt,line join=round,line cap=round,fill=fillColor] (416.89,407.95) circle (  1.16);

\path[draw=drawColor,line width= 0.4pt,line join=round,line cap=round,fill=fillColor] (417.19,407.70) circle (  1.16);

\path[draw=drawColor,line width= 0.4pt,line join=round,line cap=round,fill=fillColor] (417.49,407.51) circle (  1.16);

\path[draw=drawColor,line width= 0.4pt,line join=round,line cap=round,fill=fillColor] (417.79,407.50) circle (  1.16);

\path[draw=drawColor,line width= 0.4pt,line join=round,line cap=round,fill=fillColor] (418.09,407.40) circle (  1.16);

\path[draw=drawColor,line width= 0.4pt,line join=round,line cap=round,fill=fillColor] (418.38,407.37) circle (  1.16);

\path[draw=drawColor,line width= 0.4pt,line join=round,line cap=round,fill=fillColor] (418.68,407.35) circle (  1.16);

\path[draw=drawColor,line width= 0.4pt,line join=round,line cap=round,fill=fillColor] (418.97,407.16) circle (  1.16);

\path[draw=drawColor,line width= 0.4pt,line join=round,line cap=round,fill=fillColor] (419.26,407.02) circle (  1.16);

\path[draw=drawColor,line width= 0.4pt,line join=round,line cap=round,fill=fillColor] (419.55,407.00) circle (  1.16);

\path[draw=drawColor,line width= 0.4pt,line join=round,line cap=round,fill=fillColor] (419.84,407.00) circle (  1.16);

\path[draw=drawColor,line width= 0.4pt,line join=round,line cap=round,fill=fillColor] (420.12,406.91) circle (  1.16);

\path[draw=drawColor,line width= 0.4pt,line join=round,line cap=round,fill=fillColor] (420.41,406.91) circle (  1.16);

\path[draw=drawColor,line width= 0.4pt,line join=round,line cap=round,fill=fillColor] (420.69,406.81) circle (  1.16);

\path[draw=drawColor,line width= 0.4pt,line join=round,line cap=round,fill=fillColor] (420.97,406.79) circle (  1.16);

\path[draw=drawColor,line width= 0.4pt,line join=round,line cap=round,fill=fillColor] (421.25,406.76) circle (  1.16);

\path[draw=drawColor,line width= 0.4pt,line join=round,line cap=round,fill=fillColor] (421.53,406.69) circle (  1.16);

\path[draw=drawColor,line width= 0.4pt,line join=round,line cap=round,fill=fillColor] (421.81,406.65) circle (  1.16);

\path[draw=drawColor,line width= 0.4pt,line join=round,line cap=round,fill=fillColor] (422.09,406.62) circle (  1.16);

\path[draw=drawColor,line width= 0.4pt,line join=round,line cap=round,fill=fillColor] (422.36,406.31) circle (  1.16);

\path[draw=drawColor,line width= 0.4pt,line join=round,line cap=round,fill=fillColor] (422.63,406.31) circle (  1.16);

\path[draw=drawColor,line width= 0.4pt,line join=round,line cap=round,fill=fillColor] (422.91,406.30) circle (  1.16);

\path[draw=drawColor,line width= 0.4pt,line join=round,line cap=round,fill=fillColor] (423.18,406.22) circle (  1.16);

\path[draw=drawColor,line width= 0.4pt,line join=round,line cap=round,fill=fillColor] (423.45,406.11) circle (  1.16);

\path[draw=drawColor,line width= 0.4pt,line join=round,line cap=round,fill=fillColor] (423.72,405.99) circle (  1.16);

\path[draw=drawColor,line width= 0.4pt,line join=round,line cap=round,fill=fillColor] (423.99,405.91) circle (  1.16);

\path[draw=drawColor,line width= 0.4pt,line join=round,line cap=round,fill=fillColor] (424.25,405.84) circle (  1.16);

\path[draw=drawColor,line width= 0.4pt,line join=round,line cap=round,fill=fillColor] (424.52,405.73) circle (  1.16);

\path[draw=drawColor,line width= 0.4pt,line join=round,line cap=round,fill=fillColor] (424.78,405.62) circle (  1.16);

\path[draw=drawColor,line width= 0.4pt,line join=round,line cap=round,fill=fillColor] (425.04,405.59) circle (  1.16);

\path[draw=drawColor,line width= 0.4pt,line join=round,line cap=round,fill=fillColor] (425.31,405.49) circle (  1.16);

\path[draw=drawColor,line width= 0.4pt,line join=round,line cap=round,fill=fillColor] (425.57,405.21) circle (  1.16);

\path[draw=drawColor,line width= 0.4pt,line join=round,line cap=round,fill=fillColor] (425.83,405.15) circle (  1.16);

\path[draw=drawColor,line width= 0.4pt,line join=round,line cap=round,fill=fillColor] (426.08,405.06) circle (  1.16);

\path[draw=drawColor,line width= 0.4pt,line join=round,line cap=round,fill=fillColor] (426.34,405.05) circle (  1.16);

\path[draw=drawColor,line width= 0.4pt,line join=round,line cap=round,fill=fillColor] (426.60,405.00) circle (  1.16);

\path[draw=drawColor,line width= 0.4pt,line join=round,line cap=round,fill=fillColor] (426.85,404.92) circle (  1.16);

\path[draw=drawColor,line width= 0.4pt,line join=round,line cap=round,fill=fillColor] (427.11,404.88) circle (  1.16);

\path[draw=drawColor,line width= 0.4pt,line join=round,line cap=round,fill=fillColor] (427.36,404.84) circle (  1.16);

\path[draw=drawColor,line width= 0.4pt,line join=round,line cap=round,fill=fillColor] (427.61,404.78) circle (  1.16);

\path[draw=drawColor,line width= 0.4pt,line join=round,line cap=round,fill=fillColor] (427.86,404.76) circle (  1.16);

\path[draw=drawColor,line width= 0.4pt,line join=round,line cap=round,fill=fillColor] (428.11,404.71) circle (  1.16);

\path[draw=drawColor,line width= 0.4pt,line join=round,line cap=round,fill=fillColor] (428.36,404.70) circle (  1.16);

\path[draw=drawColor,line width= 0.4pt,line join=round,line cap=round,fill=fillColor] (428.61,404.47) circle (  1.16);

\path[draw=drawColor,line width= 0.4pt,line join=round,line cap=round,fill=fillColor] (428.86,404.39) circle (  1.16);

\path[draw=drawColor,line width= 0.4pt,line join=round,line cap=round,fill=fillColor] (429.10,404.36) circle (  1.16);

\path[draw=drawColor,line width= 0.4pt,line join=round,line cap=round,fill=fillColor] (429.35,404.20) circle (  1.16);

\path[draw=drawColor,line width= 0.4pt,line join=round,line cap=round,fill=fillColor] (429.59,404.11) circle (  1.16);

\path[draw=drawColor,line width= 0.4pt,line join=round,line cap=round,fill=fillColor] (429.83,404.06) circle (  1.16);

\path[draw=drawColor,line width= 0.4pt,line join=round,line cap=round,fill=fillColor] (430.08,404.02) circle (  1.16);

\path[draw=drawColor,line width= 0.4pt,line join=round,line cap=round,fill=fillColor] (430.32,403.98) circle (  1.16);

\path[draw=drawColor,line width= 0.4pt,line join=round,line cap=round,fill=fillColor] (430.56,403.91) circle (  1.16);

\path[draw=drawColor,line width= 0.4pt,line join=round,line cap=round,fill=fillColor] (430.80,403.83) circle (  1.16);

\path[draw=drawColor,line width= 0.4pt,line join=round,line cap=round,fill=fillColor] (431.04,403.81) circle (  1.16);

\path[draw=drawColor,line width= 0.4pt,line join=round,line cap=round,fill=fillColor] (431.27,403.69) circle (  1.16);

\path[draw=drawColor,line width= 0.4pt,line join=round,line cap=round,fill=fillColor] (431.51,403.58) circle (  1.16);

\path[draw=drawColor,line width= 0.4pt,line join=round,line cap=round,fill=fillColor] (431.75,403.57) circle (  1.16);

\path[draw=drawColor,line width= 0.4pt,line join=round,line cap=round,fill=fillColor] (431.98,403.56) circle (  1.16);

\path[draw=drawColor,line width= 0.4pt,line join=round,line cap=round,fill=fillColor] (432.21,403.56) circle (  1.16);

\path[draw=drawColor,line width= 0.4pt,line join=round,line cap=round,fill=fillColor] (432.45,403.53) circle (  1.16);

\path[draw=drawColor,line width= 0.4pt,line join=round,line cap=round,fill=fillColor] (432.68,403.36) circle (  1.16);

\path[draw=drawColor,line width= 0.4pt,line join=round,line cap=round,fill=fillColor] (432.91,403.31) circle (  1.16);

\path[draw=drawColor,line width= 0.4pt,line join=round,line cap=round,fill=fillColor] (433.14,403.28) circle (  1.16);

\path[draw=drawColor,line width= 0.4pt,line join=round,line cap=round,fill=fillColor] (433.37,403.16) circle (  1.16);

\path[draw=drawColor,line width= 0.4pt,line join=round,line cap=round,fill=fillColor] (433.60,403.12) circle (  1.16);

\path[draw=drawColor,line width= 0.4pt,line join=round,line cap=round,fill=fillColor] (433.83,403.09) circle (  1.16);

\path[draw=drawColor,line width= 0.4pt,line join=round,line cap=round,fill=fillColor] (434.06,403.01) circle (  1.16);

\path[draw=drawColor,line width= 0.4pt,line join=round,line cap=round,fill=fillColor] (434.28,402.96) circle (  1.16);

\path[draw=drawColor,line width= 0.4pt,line join=round,line cap=round,fill=fillColor] (434.51,402.92) circle (  1.16);

\path[draw=drawColor,line width= 0.4pt,line join=round,line cap=round,fill=fillColor] (434.73,402.91) circle (  1.16);

\path[draw=drawColor,line width= 0.4pt,line join=round,line cap=round,fill=fillColor] (434.96,402.90) circle (  1.16);

\path[draw=drawColor,line width= 0.4pt,line join=round,line cap=round,fill=fillColor] (435.18,402.89) circle (  1.16);

\path[draw=drawColor,line width= 0.4pt,line join=round,line cap=round,fill=fillColor] (435.40,402.85) circle (  1.16);

\path[draw=drawColor,line width= 0.4pt,line join=round,line cap=round,fill=fillColor] (435.62,402.82) circle (  1.16);

\path[draw=drawColor,line width= 0.4pt,line join=round,line cap=round,fill=fillColor] (435.85,402.80) circle (  1.16);

\path[draw=drawColor,line width= 0.4pt,line join=round,line cap=round,fill=fillColor] (436.07,402.59) circle (  1.16);

\path[draw=drawColor,line width= 0.4pt,line join=round,line cap=round,fill=fillColor] (436.29,402.58) circle (  1.16);

\path[draw=drawColor,line width= 0.4pt,line join=round,line cap=round,fill=fillColor] (436.50,402.42) circle (  1.16);

\path[draw=drawColor,line width= 0.4pt,line join=round,line cap=round,fill=fillColor] (436.72,402.40) circle (  1.16);

\path[draw=drawColor,line width= 0.4pt,line join=round,line cap=round,fill=fillColor] (436.94,402.31) circle (  1.16);

\path[draw=drawColor,line width= 0.4pt,line join=round,line cap=round,fill=fillColor] (437.16,402.22) circle (  1.16);

\path[draw=drawColor,line width= 0.4pt,line join=round,line cap=round,fill=fillColor] (437.37,402.13) circle (  1.16);

\path[draw=drawColor,line width= 0.4pt,line join=round,line cap=round,fill=fillColor] (437.59,402.11) circle (  1.16);

\path[draw=drawColor,line width= 0.4pt,line join=round,line cap=round,fill=fillColor] (437.80,402.10) circle (  1.16);

\path[draw=drawColor,line width= 0.4pt,line join=round,line cap=round,fill=fillColor] (438.02,402.00) circle (  1.16);

\path[draw=drawColor,line width= 0.4pt,line join=round,line cap=round,fill=fillColor] (438.23,401.99) circle (  1.16);

\path[draw=drawColor,line width= 0.4pt,line join=round,line cap=round,fill=fillColor] (438.44,401.94) circle (  1.16);

\path[draw=drawColor,line width= 0.4pt,line join=round,line cap=round,fill=fillColor] (438.65,401.90) circle (  1.16);

\path[draw=drawColor,line width= 0.4pt,line join=round,line cap=round,fill=fillColor] (438.87,401.77) circle (  1.16);

\path[draw=drawColor,line width= 0.4pt,line join=round,line cap=round,fill=fillColor] (439.08,401.68) circle (  1.16);

\path[draw=drawColor,line width= 0.4pt,line join=round,line cap=round,fill=fillColor] (439.29,401.63) circle (  1.16);

\path[draw=drawColor,line width= 0.4pt,line join=round,line cap=round,fill=fillColor] (439.49,401.55) circle (  1.16);

\path[draw=drawColor,line width= 0.4pt,line join=round,line cap=round,fill=fillColor] (439.70,401.54) circle (  1.16);

\path[draw=drawColor,line width= 0.4pt,line join=round,line cap=round,fill=fillColor] (439.91,401.50) circle (  1.16);

\path[draw=drawColor,line width= 0.4pt,line join=round,line cap=round,fill=fillColor] (440.12,401.43) circle (  1.16);

\path[draw=drawColor,line width= 0.4pt,line join=round,line cap=round,fill=fillColor] (440.33,401.34) circle (  1.16);

\path[draw=drawColor,line width= 0.4pt,line join=round,line cap=round,fill=fillColor] (440.53,401.08) circle (  1.16);

\path[draw=drawColor,line width= 0.4pt,line join=round,line cap=round,fill=fillColor] (440.74,401.07) circle (  1.16);

\path[draw=drawColor,line width= 0.4pt,line join=round,line cap=round,fill=fillColor] (440.94,400.92) circle (  1.16);

\path[draw=drawColor,line width= 0.4pt,line join=round,line cap=round,fill=fillColor] (441.15,400.87) circle (  1.16);

\path[draw=drawColor,line width= 0.4pt,line join=round,line cap=round,fill=fillColor] (441.35,400.86) circle (  1.16);

\path[draw=drawColor,line width= 0.4pt,line join=round,line cap=round,fill=fillColor] (441.55,400.78) circle (  1.16);

\path[draw=drawColor,line width= 0.4pt,line join=round,line cap=round,fill=fillColor] (441.75,400.72) circle (  1.16);

\path[draw=drawColor,line width= 0.4pt,line join=round,line cap=round,fill=fillColor] (441.96,400.58) circle (  1.16);

\path[draw=drawColor,line width= 0.4pt,line join=round,line cap=round,fill=fillColor] (442.16,400.55) circle (  1.16);

\path[draw=drawColor,line width= 0.4pt,line join=round,line cap=round,fill=fillColor] (442.36,400.54) circle (  1.16);

\path[draw=drawColor,line width= 0.4pt,line join=round,line cap=round,fill=fillColor] (442.56,400.53) circle (  1.16);

\path[draw=drawColor,line width= 0.4pt,line join=round,line cap=round,fill=fillColor] (442.76,400.49) circle (  1.16);

\path[draw=drawColor,line width= 0.4pt,line join=round,line cap=round,fill=fillColor] (442.96,400.37) circle (  1.16);

\path[draw=drawColor,line width= 0.4pt,line join=round,line cap=round,fill=fillColor] (443.15,400.37) circle (  1.16);

\path[draw=drawColor,line width= 0.4pt,line join=round,line cap=round,fill=fillColor] (443.35,400.32) circle (  1.16);

\path[draw=drawColor,line width= 0.4pt,line join=round,line cap=round,fill=fillColor] (443.55,400.19) circle (  1.16);

\path[draw=drawColor,line width= 0.4pt,line join=round,line cap=round,fill=fillColor] (443.74,400.18) circle (  1.16);

\path[draw=drawColor,line width= 0.4pt,line join=round,line cap=round,fill=fillColor] (443.94,400.08) circle (  1.16);

\path[draw=drawColor,line width= 0.4pt,line join=round,line cap=round,fill=fillColor] (444.14,400.08) circle (  1.16);

\path[draw=drawColor,line width= 0.4pt,line join=round,line cap=round,fill=fillColor] (444.33,400.06) circle (  1.16);

\path[draw=drawColor,line width= 0.4pt,line join=round,line cap=round,fill=fillColor] (444.53,400.03) circle (  1.16);

\path[draw=drawColor,line width= 0.4pt,line join=round,line cap=round,fill=fillColor] (444.72,400.00) circle (  1.16);

\path[draw=drawColor,line width= 0.4pt,line join=round,line cap=round,fill=fillColor] (444.91,400.00) circle (  1.16);

\path[draw=drawColor,line width= 0.4pt,line join=round,line cap=round,fill=fillColor] (445.10,399.96) circle (  1.16);

\path[draw=drawColor,line width= 0.4pt,line join=round,line cap=round,fill=fillColor] (445.30,399.95) circle (  1.16);

\path[draw=drawColor,line width= 0.4pt,line join=round,line cap=round,fill=fillColor] (445.49,399.77) circle (  1.16);

\path[draw=drawColor,line width= 0.4pt,line join=round,line cap=round,fill=fillColor] (445.68,399.58) circle (  1.16);

\path[draw=drawColor,line width= 0.4pt,line join=round,line cap=round,fill=fillColor] (445.87,399.50) circle (  1.16);

\path[draw=drawColor,line width= 0.4pt,line join=round,line cap=round,fill=fillColor] (446.06,399.40) circle (  1.16);

\path[draw=drawColor,line width= 0.4pt,line join=round,line cap=round,fill=fillColor] (446.25,399.40) circle (  1.16);

\path[draw=drawColor,line width= 0.4pt,line join=round,line cap=round,fill=fillColor] (446.44,399.38) circle (  1.16);

\path[draw=drawColor,line width= 0.4pt,line join=round,line cap=round,fill=fillColor] (446.63,399.37) circle (  1.16);

\path[draw=drawColor,line width= 0.4pt,line join=round,line cap=round,fill=fillColor] (446.82,399.34) circle (  1.16);

\path[draw=drawColor,line width= 0.4pt,line join=round,line cap=round,fill=fillColor] (447.00,399.23) circle (  1.16);

\path[draw=drawColor,line width= 0.4pt,line join=round,line cap=round,fill=fillColor] (447.19,399.17) circle (  1.16);

\path[draw=drawColor,line width= 0.4pt,line join=round,line cap=round,fill=fillColor] (447.38,399.06) circle (  1.16);

\path[draw=drawColor,line width= 0.4pt,line join=round,line cap=round,fill=fillColor] (447.56,399.06) circle (  1.16);

\path[draw=drawColor,line width= 0.4pt,line join=round,line cap=round,fill=fillColor] (447.75,399.02) circle (  1.16);

\path[draw=drawColor,line width= 0.4pt,line join=round,line cap=round,fill=fillColor] (447.93,398.99) circle (  1.16);

\path[draw=drawColor,line width= 0.4pt,line join=round,line cap=round,fill=fillColor] (448.12,398.96) circle (  1.16);

\path[draw=drawColor,line width= 0.4pt,line join=round,line cap=round,fill=fillColor] (448.30,398.91) circle (  1.16);

\path[draw=drawColor,line width= 0.4pt,line join=round,line cap=round,fill=fillColor] (448.49,398.89) circle (  1.16);

\path[draw=drawColor,line width= 0.4pt,line join=round,line cap=round,fill=fillColor] (448.67,398.82) circle (  1.16);

\path[draw=drawColor,line width= 0.4pt,line join=round,line cap=round,fill=fillColor] (448.85,398.76) circle (  1.16);

\path[draw=drawColor,line width= 0.4pt,line join=round,line cap=round,fill=fillColor] (449.03,398.70) circle (  1.16);

\path[draw=drawColor,line width= 0.4pt,line join=round,line cap=round,fill=fillColor] (449.22,398.63) circle (  1.16);

\path[draw=drawColor,line width= 0.4pt,line join=round,line cap=round,fill=fillColor] (449.40,398.60) circle (  1.16);

\path[draw=drawColor,line width= 0.4pt,line join=round,line cap=round,fill=fillColor] (449.58,398.58) circle (  1.16);

\path[draw=drawColor,line width= 0.4pt,line join=round,line cap=round,fill=fillColor] (449.76,398.50) circle (  1.16);

\path[draw=drawColor,line width= 0.4pt,line join=round,line cap=round,fill=fillColor] (449.94,398.38) circle (  1.16);

\path[draw=drawColor,line width= 0.4pt,line join=round,line cap=round,fill=fillColor] (450.12,398.37) circle (  1.16);

\path[draw=drawColor,line width= 0.4pt,line join=round,line cap=round,fill=fillColor] (450.30,398.27) circle (  1.16);

\path[draw=drawColor,line width= 0.4pt,line join=round,line cap=round,fill=fillColor] (450.48,398.25) circle (  1.16);

\path[draw=drawColor,line width= 0.4pt,line join=round,line cap=round,fill=fillColor] (450.65,398.21) circle (  1.16);

\path[draw=drawColor,line width= 0.4pt,line join=round,line cap=round,fill=fillColor] (450.83,398.16) circle (  1.16);

\path[draw=drawColor,line width= 0.4pt,line join=round,line cap=round,fill=fillColor] (451.01,397.94) circle (  1.16);

\path[draw=drawColor,line width= 0.4pt,line join=round,line cap=round,fill=fillColor] (451.19,397.80) circle (  1.16);

\path[draw=drawColor,line width= 0.4pt,line join=round,line cap=round,fill=fillColor] (451.36,397.72) circle (  1.16);

\path[draw=drawColor,line width= 0.4pt,line join=round,line cap=round,fill=fillColor] (451.54,397.72) circle (  1.16);

\path[draw=drawColor,line width= 0.4pt,line join=round,line cap=round,fill=fillColor] (451.71,397.71) circle (  1.16);

\path[draw=drawColor,line width= 0.4pt,line join=round,line cap=round,fill=fillColor] (451.89,397.61) circle (  1.16);

\path[draw=drawColor,line width= 0.4pt,line join=round,line cap=round,fill=fillColor] (452.06,397.59) circle (  1.16);

\path[draw=drawColor,line width= 0.4pt,line join=round,line cap=round,fill=fillColor] (452.24,397.51) circle (  1.16);

\path[draw=drawColor,line width= 0.4pt,line join=round,line cap=round,fill=fillColor] (452.41,397.48) circle (  1.16);

\path[draw=drawColor,line width= 0.4pt,line join=round,line cap=round,fill=fillColor] (452.59,397.46) circle (  1.16);

\path[draw=drawColor,line width= 0.4pt,line join=round,line cap=round,fill=fillColor] (452.76,397.34) circle (  1.16);

\path[draw=drawColor,line width= 0.4pt,line join=round,line cap=round,fill=fillColor] (452.93,397.27) circle (  1.16);

\path[draw=drawColor,line width= 0.4pt,line join=round,line cap=round,fill=fillColor] (453.10,397.27) circle (  1.16);

\path[draw=drawColor,line width= 0.4pt,line join=round,line cap=round,fill=fillColor] (453.28,397.26) circle (  1.16);

\path[draw=drawColor,line width= 0.4pt,line join=round,line cap=round,fill=fillColor] (453.45,397.22) circle (  1.16);

\path[draw=drawColor,line width= 0.4pt,line join=round,line cap=round,fill=fillColor] (453.62,397.20) circle (  1.16);

\path[draw=drawColor,line width= 0.4pt,line join=round,line cap=round,fill=fillColor] (453.79,397.03) circle (  1.16);

\path[draw=drawColor,line width= 0.4pt,line join=round,line cap=round,fill=fillColor] (453.96,396.86) circle (  1.16);

\path[draw=drawColor,line width= 0.4pt,line join=round,line cap=round,fill=fillColor] (454.13,396.84) circle (  1.16);

\path[draw=drawColor,line width= 0.4pt,line join=round,line cap=round,fill=fillColor] (454.30,396.76) circle (  1.16);

\path[draw=drawColor,line width= 0.4pt,line join=round,line cap=round,fill=fillColor] (454.47,396.67) circle (  1.16);

\path[draw=drawColor,line width= 0.4pt,line join=round,line cap=round,fill=fillColor] (454.64,396.64) circle (  1.16);

\path[draw=drawColor,line width= 0.4pt,line join=round,line cap=round,fill=fillColor] (454.81,396.63) circle (  1.16);

\path[draw=drawColor,line width= 0.4pt,line join=round,line cap=round,fill=fillColor] (454.97,396.47) circle (  1.16);

\path[draw=drawColor,line width= 0.4pt,line join=round,line cap=round,fill=fillColor] (455.14,396.46) circle (  1.16);

\path[draw=drawColor,line width= 0.4pt,line join=round,line cap=round,fill=fillColor] (455.31,396.43) circle (  1.16);

\path[draw=drawColor,line width= 0.4pt,line join=round,line cap=round,fill=fillColor] (455.48,396.41) circle (  1.16);

\path[draw=drawColor,line width= 0.4pt,line join=round,line cap=round,fill=fillColor] (455.64,396.37) circle (  1.16);

\path[draw=drawColor,line width= 0.4pt,line join=round,line cap=round,fill=fillColor] (455.81,396.32) circle (  1.16);

\path[draw=drawColor,line width= 0.4pt,line join=round,line cap=round,fill=fillColor] (455.97,396.31) circle (  1.16);

\path[draw=drawColor,line width= 0.4pt,line join=round,line cap=round,fill=fillColor] (456.14,396.28) circle (  1.16);

\path[draw=drawColor,line width= 0.4pt,line join=round,line cap=round,fill=fillColor] (456.31,396.24) circle (  1.16);

\path[draw=drawColor,line width= 0.4pt,line join=round,line cap=round,fill=fillColor] (456.47,396.22) circle (  1.16);

\path[draw=drawColor,line width= 0.4pt,line join=round,line cap=round,fill=fillColor] (456.63,396.17) circle (  1.16);

\path[draw=drawColor,line width= 0.4pt,line join=round,line cap=round,fill=fillColor] (456.80,396.13) circle (  1.16);

\path[draw=drawColor,line width= 0.4pt,line join=round,line cap=round,fill=fillColor] (456.96,396.12) circle (  1.16);

\path[draw=drawColor,line width= 0.4pt,line join=round,line cap=round,fill=fillColor] (457.13,396.06) circle (  1.16);

\path[draw=drawColor,line width= 0.4pt,line join=round,line cap=round,fill=fillColor] (457.29,396.01) circle (  1.16);

\path[draw=drawColor,line width= 0.4pt,line join=round,line cap=round,fill=fillColor] (457.45,395.96) circle (  1.16);

\path[draw=drawColor,line width= 0.4pt,line join=round,line cap=round,fill=fillColor] (457.61,395.85) circle (  1.16);

\path[draw=drawColor,line width= 0.4pt,line join=round,line cap=round,fill=fillColor] (457.78,395.80) circle (  1.16);

\path[draw=drawColor,line width= 0.4pt,line join=round,line cap=round,fill=fillColor] (457.94,395.79) circle (  1.16);

\path[draw=drawColor,line width= 0.4pt,line join=round,line cap=round,fill=fillColor] (458.10,395.70) circle (  1.16);

\path[draw=drawColor,line width= 0.4pt,line join=round,line cap=round,fill=fillColor] (458.26,395.67) circle (  1.16);

\path[draw=drawColor,line width= 0.4pt,line join=round,line cap=round,fill=fillColor] (458.42,395.63) circle (  1.16);

\path[draw=drawColor,line width= 0.4pt,line join=round,line cap=round,fill=fillColor] (458.58,395.61) circle (  1.16);

\path[draw=drawColor,line width= 0.4pt,line join=round,line cap=round,fill=fillColor] (458.74,395.47) circle (  1.16);

\path[draw=drawColor,line width= 0.4pt,line join=round,line cap=round,fill=fillColor] (458.90,395.28) circle (  1.16);

\path[draw=drawColor,line width= 0.4pt,line join=round,line cap=round,fill=fillColor] (459.06,395.27) circle (  1.16);

\path[draw=drawColor,line width= 0.4pt,line join=round,line cap=round,fill=fillColor] (459.22,395.20) circle (  1.16);

\path[draw=drawColor,line width= 0.4pt,line join=round,line cap=round,fill=fillColor] (459.38,395.20) circle (  1.16);

\path[draw=drawColor,line width= 0.4pt,line join=round,line cap=round,fill=fillColor] (459.53,395.02) circle (  1.16);

\path[draw=drawColor,line width= 0.4pt,line join=round,line cap=round,fill=fillColor] (459.69,394.98) circle (  1.16);

\path[draw=drawColor,line width= 0.4pt,line join=round,line cap=round,fill=fillColor] (459.85,394.96) circle (  1.16);

\path[draw=drawColor,line width= 0.4pt,line join=round,line cap=round,fill=fillColor] (460.01,394.95) circle (  1.16);

\path[draw=drawColor,line width= 0.4pt,line join=round,line cap=round,fill=fillColor] (460.16,394.56) circle (  1.16);

\path[draw=drawColor,line width= 0.4pt,line join=round,line cap=round,fill=fillColor] (460.32,394.55) circle (  1.16);

\path[draw=drawColor,line width= 0.4pt,line join=round,line cap=round,fill=fillColor] (460.48,394.47) circle (  1.16);

\path[draw=drawColor,line width= 0.4pt,line join=round,line cap=round,fill=fillColor] (460.63,394.46) circle (  1.16);

\path[draw=drawColor,line width= 0.4pt,line join=round,line cap=round,fill=fillColor] (460.79,394.43) circle (  1.16);

\path[draw=drawColor,line width= 0.4pt,line join=round,line cap=round,fill=fillColor] (460.94,394.31) circle (  1.16);

\path[draw=drawColor,line width= 0.4pt,line join=round,line cap=round,fill=fillColor] (461.10,394.16) circle (  1.16);

\path[draw=drawColor,line width= 0.4pt,line join=round,line cap=round,fill=fillColor] (461.25,393.86) circle (  1.16);

\path[draw=drawColor,line width= 0.4pt,line join=round,line cap=round,fill=fillColor] (461.41,393.85) circle (  1.16);

\path[draw=drawColor,line width= 0.4pt,line join=round,line cap=round,fill=fillColor] (461.56,393.66) circle (  1.16);

\path[draw=drawColor,line width= 0.4pt,line join=round,line cap=round,fill=fillColor] (461.72,393.24) circle (  1.16);

\path[draw=drawColor,line width= 0.4pt,line join=round,line cap=round,fill=fillColor] (461.87,393.01) circle (  1.16);

\path[draw=drawColor,line width= 0.4pt,line join=round,line cap=round,fill=fillColor] (462.02,392.72) circle (  1.16);

\path[draw=drawColor,line width= 0.4pt,line join=round,line cap=round,fill=fillColor] (462.18,392.66) circle (  1.16);

\path[draw=drawColor,line width= 0.4pt,line join=round,line cap=round,fill=fillColor] (462.33,392.47) circle (  1.16);

\path[draw=drawColor,line width= 0.4pt,line join=round,line cap=round,fill=fillColor] (462.48,392.41) circle (  1.16);

\path[draw=drawColor,line width= 0.4pt,line join=round,line cap=round,fill=fillColor] (462.63,392.26) circle (  1.16);

\path[draw=drawColor,line width= 0.4pt,line join=round,line cap=round,fill=fillColor] (462.78,392.26) circle (  1.16);

\path[draw=drawColor,line width= 0.4pt,line join=round,line cap=round,fill=fillColor] (462.94,392.21) circle (  1.16);

\path[draw=drawColor,line width= 0.4pt,line join=round,line cap=round,fill=fillColor] (463.09,392.06) circle (  1.16);

\path[draw=drawColor,line width= 0.4pt,line join=round,line cap=round,fill=fillColor] (463.24,391.91) circle (  1.16);

\path[draw=drawColor,line width= 0.4pt,line join=round,line cap=round,fill=fillColor] (463.39,391.68) circle (  1.16);

\path[draw=drawColor,line width= 0.4pt,line join=round,line cap=round,fill=fillColor] (463.54,391.67) circle (  1.16);

\path[draw=drawColor,line width= 0.4pt,line join=round,line cap=round,fill=fillColor] (463.69,391.47) circle (  1.16);

\path[draw=drawColor,line width= 0.4pt,line join=round,line cap=round,fill=fillColor] (463.84,391.17) circle (  1.16);

\path[draw=drawColor,line width= 0.4pt,line join=round,line cap=round,fill=fillColor] (463.99,391.12) circle (  1.16);

\path[draw=drawColor,line width= 0.4pt,line join=round,line cap=round,fill=fillColor] (464.14,391.03) circle (  1.16);

\path[draw=drawColor,line width= 0.4pt,line join=round,line cap=round,fill=fillColor] (464.29,391.03) circle (  1.16);

\path[draw=drawColor,line width= 0.4pt,line join=round,line cap=round,fill=fillColor] (464.44,390.92) circle (  1.16);

\path[draw=drawColor,line width= 0.4pt,line join=round,line cap=round,fill=fillColor] (464.58,390.92) circle (  1.16);

\path[draw=drawColor,line width= 0.4pt,line join=round,line cap=round,fill=fillColor] (464.73,390.52) circle (  1.16);

\path[draw=drawColor,line width= 0.4pt,line join=round,line cap=round,fill=fillColor] (464.88,390.41) circle (  1.16);

\path[draw=drawColor,line width= 0.4pt,line join=round,line cap=round,fill=fillColor] (465.03,390.23) circle (  1.16);

\path[draw=drawColor,line width= 0.4pt,line join=round,line cap=round,fill=fillColor] (465.17,390.04) circle (  1.16);

\path[draw=drawColor,line width= 0.4pt,line join=round,line cap=round,fill=fillColor] (465.32,389.83) circle (  1.16);

\path[draw=drawColor,line width= 0.4pt,line join=round,line cap=round,fill=fillColor] (465.47,389.83) circle (  1.16);

\path[draw=drawColor,line width= 0.4pt,line join=round,line cap=round,fill=fillColor] (465.61,389.56) circle (  1.16);

\path[draw=drawColor,line width= 0.4pt,line join=round,line cap=round,fill=fillColor] (465.76,389.13) circle (  1.16);

\path[draw=drawColor,line width= 0.4pt,line join=round,line cap=round,fill=fillColor] (465.91,389.04) circle (  1.16);

\path[draw=drawColor,line width= 0.4pt,line join=round,line cap=round,fill=fillColor] (466.05,388.42) circle (  1.16);

\path[draw=drawColor,line width= 0.4pt,line join=round,line cap=round,fill=fillColor] (466.20,388.21) circle (  1.16);

\path[draw=drawColor,line width= 0.4pt,line join=round,line cap=round,fill=fillColor] (466.34,388.15) circle (  1.16);

\path[draw=drawColor,line width= 0.4pt,line join=round,line cap=round,fill=fillColor] (466.49,387.55) circle (  1.16);

\path[draw=drawColor,line width= 0.4pt,line join=round,line cap=round,fill=fillColor] (466.63,387.49) circle (  1.16);

\path[draw=drawColor,line width= 0.4pt,line join=round,line cap=round,fill=fillColor] (466.78,387.27) circle (  1.16);

\path[draw=drawColor,line width= 0.4pt,line join=round,line cap=round,fill=fillColor] (466.92,387.27) circle (  1.16);

\path[draw=drawColor,line width= 0.4pt,line join=round,line cap=round,fill=fillColor] (467.06,386.33) circle (  1.16);

\path[draw=drawColor,line width= 0.4pt,line join=round,line cap=round,fill=fillColor] (467.21,386.18) circle (  1.16);

\path[draw=drawColor,line width= 0.4pt,line join=round,line cap=round,fill=fillColor] (467.35,386.15) circle (  1.16);

\path[draw=drawColor,line width= 0.4pt,line join=round,line cap=round,fill=fillColor] (467.49,386.05) circle (  1.16);

\path[draw=drawColor,line width= 0.4pt,line join=round,line cap=round,fill=fillColor] (467.64,386.05) circle (  1.16);

\path[draw=drawColor,line width= 0.4pt,line join=round,line cap=round,fill=fillColor] (467.78,385.81) circle (  1.16);

\path[draw=drawColor,line width= 0.4pt,line join=round,line cap=round,fill=fillColor] (467.92,385.72) circle (  1.16);

\path[draw=drawColor,line width= 0.4pt,line join=round,line cap=round,fill=fillColor] (468.06,385.47) circle (  1.16);

\path[draw=drawColor,line width= 0.4pt,line join=round,line cap=round,fill=fillColor] (468.21,385.01) circle (  1.16);

\path[draw=drawColor,line width= 0.4pt,line join=round,line cap=round,fill=fillColor] (468.35,384.87) circle (  1.16);

\path[draw=drawColor,line width= 0.4pt,line join=round,line cap=round,fill=fillColor] (468.49,384.62) circle (  1.16);

\path[draw=drawColor,line width= 0.4pt,line join=round,line cap=round,fill=fillColor] (468.63,384.57) circle (  1.16);

\path[draw=drawColor,line width= 0.4pt,line join=round,line cap=round,fill=fillColor] (468.77,384.10) circle (  1.16);

\path[draw=drawColor,line width= 0.4pt,line join=round,line cap=round,fill=fillColor] (468.91,384.05) circle (  1.16);

\path[draw=drawColor,line width= 0.4pt,line join=round,line cap=round,fill=fillColor] (469.05,383.57) circle (  1.16);

\path[draw=drawColor,line width= 0.4pt,line join=round,line cap=round,fill=fillColor] (469.19,383.55) circle (  1.16);

\path[draw=drawColor,line width= 0.4pt,line join=round,line cap=round,fill=fillColor] (469.33,383.24) circle (  1.16);

\path[draw=drawColor,line width= 0.4pt,line join=round,line cap=round,fill=fillColor] (469.47,382.92) circle (  1.16);

\path[draw=drawColor,line width= 0.4pt,line join=round,line cap=round,fill=fillColor] (469.61,381.72) circle (  1.16);

\path[draw=drawColor,line width= 0.4pt,line join=round,line cap=round,fill=fillColor] (469.75,380.09) circle (  1.16);

\path[draw=drawColor,line width= 0.4pt,line join=round,line cap=round,fill=fillColor] (469.89,379.13) circle (  1.16);

\path[draw=drawColor,line width= 0.4pt,line join=round,line cap=round,fill=fillColor] (470.03,377.99) circle (  1.16);

\path[draw=drawColor,line width= 0.4pt,line join=round,line cap=round,fill=fillColor] (470.17,377.84) circle (  1.16);

\path[draw=drawColor,line width= 0.4pt,line join=round,line cap=round,fill=fillColor] (470.30,377.80) circle (  1.16);

\path[draw=drawColor,line width= 0.4pt,line join=round,line cap=round,fill=fillColor] (470.44,372.42) circle (  1.16);

\path[draw=drawColor,line width= 0.4pt,line join=round,line cap=round,fill=fillColor] (470.58,372.42) circle (  1.16);

\path[draw=drawColor,line width= 0.4pt,line join=round,line cap=round,fill=fillColor] (470.72,372.42) circle (  1.16);

\path[draw=drawColor,line width= 0.4pt,line join=round,line cap=round,fill=fillColor] (470.85,372.42) circle (  1.16);

\path[draw=drawColor,line width= 0.4pt,line join=round,line cap=round,fill=fillColor] (470.99,372.42) circle (  1.16);

\path[draw=drawColor,line width= 0.4pt,line join=round,line cap=round,fill=fillColor] (471.13,372.42) circle (  1.16);

\path[draw=drawColor,line width= 0.4pt,line join=round,line cap=round,fill=fillColor] (471.27,372.42) circle (  1.16);

\path[draw=drawColor,line width= 0.4pt,line join=round,line cap=round,fill=fillColor] (471.40,372.42) circle (  1.16);

\path[draw=drawColor,line width= 0.4pt,line join=round,line cap=round,fill=fillColor] (471.54,372.42) circle (  1.16);

\path[draw=drawColor,line width= 0.4pt,line join=round,line cap=round,fill=fillColor] (471.67,372.42) circle (  1.16);

\path[draw=drawColor,line width= 0.4pt,line join=round,line cap=round,fill=fillColor] (471.81,372.42) circle (  1.16);

\path[draw=drawColor,line width= 0.4pt,line join=round,line cap=round,fill=fillColor] (471.94,372.42) circle (  1.16);

\path[draw=drawColor,line width= 0.4pt,line join=round,line cap=round,fill=fillColor] (472.08,372.42) circle (  1.16);

\path[draw=drawColor,line width= 0.4pt,line join=round,line cap=round,fill=fillColor] (472.22,372.42) circle (  1.16);

\path[draw=drawColor,line width= 0.4pt,line join=round,line cap=round,fill=fillColor] (472.35,372.42) circle (  1.16);

\path[draw=drawColor,line width= 0.4pt,line join=round,line cap=round,fill=fillColor] (472.48,372.42) circle (  1.16);

\path[draw=drawColor,line width= 0.4pt,line join=round,line cap=round,fill=fillColor] (472.62,372.42) circle (  1.16);

\path[draw=drawColor,line width= 0.4pt,line join=round,line cap=round,fill=fillColor] (472.75,372.42) circle (  1.16);

\path[draw=drawColor,line width= 0.4pt,line join=round,line cap=round,fill=fillColor] (472.89,372.42) circle (  1.16);

\path[draw=drawColor,line width= 0.4pt,line join=round,line cap=round,fill=fillColor] (473.02,372.42) circle (  1.16);

\path[draw=drawColor,line width= 0.4pt,line join=round,line cap=round,fill=fillColor] (473.15,372.42) circle (  1.16);

\path[draw=drawColor,line width= 0.4pt,line join=round,line cap=round,fill=fillColor] (473.29,372.42) circle (  1.16);

\path[draw=drawColor,line width= 0.4pt,line join=round,line cap=round,fill=fillColor] (473.42,372.42) circle (  1.16);

\path[draw=drawColor,line width= 0.4pt,line join=round,line cap=round,fill=fillColor] (473.55,372.42) circle (  1.16);

\path[draw=drawColor,line width= 0.4pt,line join=round,line cap=round,fill=fillColor] (473.69,372.42) circle (  1.16);

\path[draw=drawColor,line width= 0.4pt,line join=round,line cap=round,fill=fillColor] (473.82,372.42) circle (  1.16);

\path[draw=drawColor,line width= 0.4pt,line join=round,line cap=round,fill=fillColor] (473.95,372.42) circle (  1.16);

\path[draw=drawColor,line width= 0.4pt,line join=round,line cap=round,fill=fillColor] (474.08,372.42) circle (  1.16);

\path[draw=drawColor,line width= 0.4pt,line join=round,line cap=round,fill=fillColor] (474.22,372.42) circle (  1.16);

\path[draw=drawColor,line width= 0.4pt,line join=round,line cap=round,fill=fillColor] (474.35,372.42) circle (  1.16);

\path[draw=drawColor,line width= 0.4pt,line join=round,line cap=round,fill=fillColor] (474.48,372.42) circle (  1.16);

\path[draw=drawColor,line width= 0.4pt,line join=round,line cap=round,fill=fillColor] (474.61,372.42) circle (  1.16);

\path[draw=drawColor,line width= 0.4pt,line join=round,line cap=round,fill=fillColor] (474.74,372.42) circle (  1.16);

\path[draw=drawColor,line width= 0.4pt,line join=round,line cap=round,fill=fillColor] (474.87,372.42) circle (  1.16);

\path[draw=drawColor,line width= 0.4pt,line join=round,line cap=round,fill=fillColor] (475.00,372.42) circle (  1.16);

\path[draw=drawColor,line width= 0.4pt,line join=round,line cap=round,fill=fillColor] (475.13,372.42) circle (  1.16);

\path[draw=drawColor,line width= 0.4pt,line join=round,line cap=round,fill=fillColor] (475.26,372.42) circle (  1.16);

\path[draw=drawColor,line width= 0.4pt,line join=round,line cap=round,fill=fillColor] (475.39,372.42) circle (  1.16);

\path[draw=drawColor,line width= 0.4pt,line join=round,line cap=round,fill=fillColor] (475.52,372.42) circle (  1.16);

\path[draw=drawColor,line width= 0.4pt,line join=round,line cap=round,fill=fillColor] (475.65,372.42) circle (  1.16);

\path[draw=drawColor,line width= 0.4pt,line join=round,line cap=round,fill=fillColor] (475.78,372.42) circle (  1.16);

\path[draw=drawColor,line width= 0.4pt,line join=round,line cap=round,fill=fillColor] (475.91,372.42) circle (  1.16);

\path[draw=drawColor,line width= 0.4pt,line join=round,line cap=round,fill=fillColor] (476.04,372.42) circle (  1.16);

\path[draw=drawColor,line width= 0.4pt,line join=round,line cap=round,fill=fillColor] (476.17,372.42) circle (  1.16);

\path[draw=drawColor,line width= 0.4pt,line join=round,line cap=round,fill=fillColor] (476.30,372.42) circle (  1.16);

\path[draw=drawColor,line width= 0.4pt,line join=round,line cap=round,fill=fillColor] (476.43,372.42) circle (  1.16);

\path[draw=drawColor,line width= 0.4pt,line join=round,line cap=round,fill=fillColor] (476.56,372.42) circle (  1.16);

\path[draw=drawColor,line width= 0.4pt,line join=round,line cap=round,fill=fillColor] (476.68,372.42) circle (  1.16);

\path[draw=drawColor,line width= 0.4pt,line join=round,line cap=round,fill=fillColor] (476.81,372.42) circle (  1.16);

\path[draw=drawColor,line width= 0.4pt,line join=round,line cap=round,fill=fillColor] (476.94,372.42) circle (  1.16);

\path[draw=drawColor,line width= 0.4pt,line join=round,line cap=round,fill=fillColor] (477.07,372.42) circle (  1.16);

\path[draw=drawColor,line width= 0.4pt,line join=round,line cap=round,fill=fillColor] (477.19,372.42) circle (  1.16);

\path[draw=drawColor,line width= 0.4pt,line join=round,line cap=round,fill=fillColor] (477.32,372.42) circle (  1.16);

\path[draw=drawColor,line width= 0.4pt,line join=round,line cap=round,fill=fillColor] (477.45,372.42) circle (  1.16);

\path[draw=drawColor,line width= 0.4pt,line join=round,line cap=round,fill=fillColor] (477.57,372.42) circle (  1.16);

\path[draw=drawColor,line width= 0.4pt,line join=round,line cap=round,fill=fillColor] (477.70,372.42) circle (  1.16);

\path[draw=drawColor,line width= 0.4pt,line join=round,line cap=round,fill=fillColor] (477.83,372.42) circle (  1.16);

\path[draw=drawColor,line width= 0.4pt,line join=round,line cap=round,fill=fillColor] (477.95,372.42) circle (  1.16);

\path[draw=drawColor,line width= 0.4pt,line join=round,line cap=round,fill=fillColor] (478.08,372.42) circle (  1.16);

\path[draw=drawColor,line width= 0.4pt,line join=round,line cap=round,fill=fillColor] (478.21,372.42) circle (  1.16);
\definecolor[named]{drawColor}{rgb}{0.30,0.69,0.29}
\definecolor[named]{fillColor}{rgb}{0.30,0.69,0.29}

\path[draw=drawColor,line width= 0.4pt,line join=round,line cap=round,fill=fillColor] (327.82,458.06) circle (  1.16);

\path[draw=drawColor,line width= 0.4pt,line join=round,line cap=round,fill=fillColor] (333.65,458.06) circle (  1.16);

\path[draw=drawColor,line width= 0.4pt,line join=round,line cap=round,fill=fillColor] (337.74,458.06) circle (  1.16);

\path[draw=drawColor,line width= 0.4pt,line join=round,line cap=round,fill=fillColor] (340.99,458.06) circle (  1.16);

\path[draw=drawColor,line width= 0.4pt,line join=round,line cap=round,fill=fillColor] (343.74,458.06) circle (  1.16);

\path[draw=drawColor,line width= 0.4pt,line join=round,line cap=round,fill=fillColor] (346.14,458.06) circle (  1.16);

\path[draw=drawColor,line width= 0.4pt,line join=round,line cap=round,fill=fillColor] (348.29,458.06) circle (  1.16);

\path[draw=drawColor,line width= 0.4pt,line join=round,line cap=round,fill=fillColor] (350.24,458.06) circle (  1.16);

\path[draw=drawColor,line width= 0.4pt,line join=round,line cap=round,fill=fillColor] (352.04,458.06) circle (  1.16);

\path[draw=drawColor,line width= 0.4pt,line join=round,line cap=round,fill=fillColor] (353.70,458.06) circle (  1.16);

\path[draw=drawColor,line width= 0.4pt,line join=round,line cap=round,fill=fillColor] (355.26,458.06) circle (  1.16);

\path[draw=drawColor,line width= 0.4pt,line join=round,line cap=round,fill=fillColor] (356.73,458.06) circle (  1.16);

\path[draw=drawColor,line width= 0.4pt,line join=round,line cap=round,fill=fillColor] (358.12,458.06) circle (  1.16);

\path[draw=drawColor,line width= 0.4pt,line join=round,line cap=round,fill=fillColor] (359.44,458.06) circle (  1.16);

\path[draw=drawColor,line width= 0.4pt,line join=round,line cap=round,fill=fillColor] (360.69,458.06) circle (  1.16);

\path[draw=drawColor,line width= 0.4pt,line join=round,line cap=round,fill=fillColor] (361.90,458.06) circle (  1.16);

\path[draw=drawColor,line width= 0.4pt,line join=round,line cap=round,fill=fillColor] (363.05,458.06) circle (  1.16);

\path[draw=drawColor,line width= 0.4pt,line join=round,line cap=round,fill=fillColor] (364.16,458.06) circle (  1.16);

\path[draw=drawColor,line width= 0.4pt,line join=round,line cap=round,fill=fillColor] (365.23,458.06) circle (  1.16);

\path[draw=drawColor,line width= 0.4pt,line join=round,line cap=round,fill=fillColor] (366.26,458.06) circle (  1.16);

\path[draw=drawColor,line width= 0.4pt,line join=round,line cap=round,fill=fillColor] (367.25,458.06) circle (  1.16);

\path[draw=drawColor,line width= 0.4pt,line join=round,line cap=round,fill=fillColor] (368.22,458.06) circle (  1.16);

\path[draw=drawColor,line width= 0.4pt,line join=round,line cap=round,fill=fillColor] (369.16,458.06) circle (  1.16);

\path[draw=drawColor,line width= 0.4pt,line join=round,line cap=round,fill=fillColor] (370.07,458.06) circle (  1.16);

\path[draw=drawColor,line width= 0.4pt,line join=round,line cap=round,fill=fillColor] (370.96,458.06) circle (  1.16);

\path[draw=drawColor,line width= 0.4pt,line join=round,line cap=round,fill=fillColor] (371.82,458.06) circle (  1.16);

\path[draw=drawColor,line width= 0.4pt,line join=round,line cap=round,fill=fillColor] (372.66,458.06) circle (  1.16);

\path[draw=drawColor,line width= 0.4pt,line join=round,line cap=round,fill=fillColor] (373.48,458.06) circle (  1.16);

\path[draw=drawColor,line width= 0.4pt,line join=round,line cap=round,fill=fillColor] (374.28,458.06) circle (  1.16);

\path[draw=drawColor,line width= 0.4pt,line join=round,line cap=round,fill=fillColor] (375.06,458.06) circle (  1.16);

\path[draw=drawColor,line width= 0.4pt,line join=round,line cap=round,fill=fillColor] (375.83,458.06) circle (  1.16);

\path[draw=drawColor,line width= 0.4pt,line join=round,line cap=round,fill=fillColor] (376.58,458.06) circle (  1.16);

\path[draw=drawColor,line width= 0.4pt,line join=round,line cap=round,fill=fillColor] (377.31,458.06) circle (  1.16);

\path[draw=drawColor,line width= 0.4pt,line join=round,line cap=round,fill=fillColor] (378.03,458.06) circle (  1.16);

\path[draw=drawColor,line width= 0.4pt,line join=round,line cap=round,fill=fillColor] (378.74,458.06) circle (  1.16);

\path[draw=drawColor,line width= 0.4pt,line join=round,line cap=round,fill=fillColor] (379.43,458.06) circle (  1.16);

\path[draw=drawColor,line width= 0.4pt,line join=round,line cap=round,fill=fillColor] (380.11,458.06) circle (  1.16);

\path[draw=drawColor,line width= 0.4pt,line join=round,line cap=round,fill=fillColor] (380.77,458.06) circle (  1.16);

\path[draw=drawColor,line width= 0.4pt,line join=round,line cap=round,fill=fillColor] (381.43,458.06) circle (  1.16);

\path[draw=drawColor,line width= 0.4pt,line join=round,line cap=round,fill=fillColor] (382.07,458.06) circle (  1.16);

\path[draw=drawColor,line width= 0.4pt,line join=round,line cap=round,fill=fillColor] (382.71,458.06) circle (  1.16);

\path[draw=drawColor,line width= 0.4pt,line join=round,line cap=round,fill=fillColor] (383.33,458.06) circle (  1.16);

\path[draw=drawColor,line width= 0.4pt,line join=round,line cap=round,fill=fillColor] (383.94,458.06) circle (  1.16);

\path[draw=drawColor,line width= 0.4pt,line join=round,line cap=round,fill=fillColor] (384.55,458.06) circle (  1.16);

\path[draw=drawColor,line width= 0.4pt,line join=round,line cap=round,fill=fillColor] (385.14,458.06) circle (  1.16);

\path[draw=drawColor,line width= 0.4pt,line join=round,line cap=round,fill=fillColor] (385.73,458.06) circle (  1.16);

\path[draw=drawColor,line width= 0.4pt,line join=round,line cap=round,fill=fillColor] (386.31,458.06) circle (  1.16);

\path[draw=drawColor,line width= 0.4pt,line join=round,line cap=round,fill=fillColor] (386.88,458.06) circle (  1.16);

\path[draw=drawColor,line width= 0.4pt,line join=round,line cap=round,fill=fillColor] (387.44,458.06) circle (  1.16);

\path[draw=drawColor,line width= 0.4pt,line join=round,line cap=round,fill=fillColor] (387.99,458.06) circle (  1.16);

\path[draw=drawColor,line width= 0.4pt,line join=round,line cap=round,fill=fillColor] (388.54,458.06) circle (  1.16);

\path[draw=drawColor,line width= 0.4pt,line join=round,line cap=round,fill=fillColor] (389.08,458.06) circle (  1.16);

\path[draw=drawColor,line width= 0.4pt,line join=round,line cap=round,fill=fillColor] (389.61,458.06) circle (  1.16);

\path[draw=drawColor,line width= 0.4pt,line join=round,line cap=round,fill=fillColor] (390.14,458.06) circle (  1.16);

\path[draw=drawColor,line width= 0.4pt,line join=round,line cap=round,fill=fillColor] (390.66,458.06) circle (  1.16);

\path[draw=drawColor,line width= 0.4pt,line join=round,line cap=round,fill=fillColor] (391.17,458.06) circle (  1.16);

\path[draw=drawColor,line width= 0.4pt,line join=round,line cap=round,fill=fillColor] (391.68,458.06) circle (  1.16);

\path[draw=drawColor,line width= 0.4pt,line join=round,line cap=round,fill=fillColor] (392.18,458.06) circle (  1.16);

\path[draw=drawColor,line width= 0.4pt,line join=round,line cap=round,fill=fillColor] (392.68,458.06) circle (  1.16);

\path[draw=drawColor,line width= 0.4pt,line join=round,line cap=round,fill=fillColor] (393.17,458.06) circle (  1.16);

\path[draw=drawColor,line width= 0.4pt,line join=round,line cap=round,fill=fillColor] (393.65,458.06) circle (  1.16);

\path[draw=drawColor,line width= 0.4pt,line join=round,line cap=round,fill=fillColor] (394.13,458.06) circle (  1.16);

\path[draw=drawColor,line width= 0.4pt,line join=round,line cap=round,fill=fillColor] (394.61,458.06) circle (  1.16);

\path[draw=drawColor,line width= 0.4pt,line join=round,line cap=round,fill=fillColor] (395.08,458.06) circle (  1.16);

\path[draw=drawColor,line width= 0.4pt,line join=round,line cap=round,fill=fillColor] (395.54,458.06) circle (  1.16);

\path[draw=drawColor,line width= 0.4pt,line join=round,line cap=round,fill=fillColor] (396.00,458.06) circle (  1.16);

\path[draw=drawColor,line width= 0.4pt,line join=round,line cap=round,fill=fillColor] (396.46,458.06) circle (  1.16);

\path[draw=drawColor,line width= 0.4pt,line join=round,line cap=round,fill=fillColor] (396.91,458.06) circle (  1.16);

\path[draw=drawColor,line width= 0.4pt,line join=round,line cap=round,fill=fillColor] (397.35,458.06) circle (  1.16);

\path[draw=drawColor,line width= 0.4pt,line join=round,line cap=round,fill=fillColor] (397.80,458.06) circle (  1.16);

\path[draw=drawColor,line width= 0.4pt,line join=round,line cap=round,fill=fillColor] (398.23,458.06) circle (  1.16);

\path[draw=drawColor,line width= 0.4pt,line join=round,line cap=round,fill=fillColor] (398.67,458.06) circle (  1.16);

\path[draw=drawColor,line width= 0.4pt,line join=round,line cap=round,fill=fillColor] (399.10,458.06) circle (  1.16);

\path[draw=drawColor,line width= 0.4pt,line join=round,line cap=round,fill=fillColor] (399.52,458.06) circle (  1.16);

\path[draw=drawColor,line width= 0.4pt,line join=round,line cap=round,fill=fillColor] (399.94,458.06) circle (  1.16);

\path[draw=drawColor,line width= 0.4pt,line join=round,line cap=round,fill=fillColor] (400.36,458.06) circle (  1.16);

\path[draw=drawColor,line width= 0.4pt,line join=round,line cap=round,fill=fillColor] (400.78,458.06) circle (  1.16);

\path[draw=drawColor,line width= 0.4pt,line join=round,line cap=round,fill=fillColor] (401.19,458.06) circle (  1.16);

\path[draw=drawColor,line width= 0.4pt,line join=round,line cap=round,fill=fillColor] (401.60,458.06) circle (  1.16);

\path[draw=drawColor,line width= 0.4pt,line join=round,line cap=round,fill=fillColor] (402.00,458.06) circle (  1.16);

\path[draw=drawColor,line width= 0.4pt,line join=round,line cap=round,fill=fillColor] (402.40,458.06) circle (  1.16);

\path[draw=drawColor,line width= 0.4pt,line join=round,line cap=round,fill=fillColor] (402.80,458.06) circle (  1.16);

\path[draw=drawColor,line width= 0.4pt,line join=round,line cap=round,fill=fillColor] (403.19,458.06) circle (  1.16);

\path[draw=drawColor,line width= 0.4pt,line join=round,line cap=round,fill=fillColor] (403.58,458.06) circle (  1.16);

\path[draw=drawColor,line width= 0.4pt,line join=round,line cap=round,fill=fillColor] (403.97,458.06) circle (  1.16);

\path[draw=drawColor,line width= 0.4pt,line join=round,line cap=round,fill=fillColor] (404.36,458.06) circle (  1.16);

\path[draw=drawColor,line width= 0.4pt,line join=round,line cap=round,fill=fillColor] (404.74,458.06) circle (  1.16);

\path[draw=drawColor,line width= 0.4pt,line join=round,line cap=round,fill=fillColor] (405.12,458.06) circle (  1.16);

\path[draw=drawColor,line width= 0.4pt,line join=round,line cap=round,fill=fillColor] (405.49,458.06) circle (  1.16);

\path[draw=drawColor,line width= 0.4pt,line join=round,line cap=round,fill=fillColor] (405.87,458.06) circle (  1.16);

\path[draw=drawColor,line width= 0.4pt,line join=round,line cap=round,fill=fillColor] (406.24,458.06) circle (  1.16);

\path[draw=drawColor,line width= 0.4pt,line join=round,line cap=round,fill=fillColor] (406.61,458.06) circle (  1.16);

\path[draw=drawColor,line width= 0.4pt,line join=round,line cap=round,fill=fillColor] (406.97,458.06) circle (  1.16);

\path[draw=drawColor,line width= 0.4pt,line join=round,line cap=round,fill=fillColor] (407.34,458.06) circle (  1.16);

\path[draw=drawColor,line width= 0.4pt,line join=round,line cap=round,fill=fillColor] (407.70,458.06) circle (  1.16);

\path[draw=drawColor,line width= 0.4pt,line join=round,line cap=round,fill=fillColor] (408.05,458.06) circle (  1.16);

\path[draw=drawColor,line width= 0.4pt,line join=round,line cap=round,fill=fillColor] (408.41,458.06) circle (  1.16);

\path[draw=drawColor,line width= 0.4pt,line join=round,line cap=round,fill=fillColor] (408.76,458.06) circle (  1.16);

\path[draw=drawColor,line width= 0.4pt,line join=round,line cap=round,fill=fillColor] (409.11,458.06) circle (  1.16);

\path[draw=drawColor,line width= 0.4pt,line join=round,line cap=round,fill=fillColor] (409.46,458.06) circle (  1.16);

\path[draw=drawColor,line width= 0.4pt,line join=round,line cap=round,fill=fillColor] (409.80,458.06) circle (  1.16);

\path[draw=drawColor,line width= 0.4pt,line join=round,line cap=round,fill=fillColor] (410.15,458.06) circle (  1.16);

\path[draw=drawColor,line width= 0.4pt,line join=round,line cap=round,fill=fillColor] (410.49,458.06) circle (  1.16);

\path[draw=drawColor,line width= 0.4pt,line join=round,line cap=round,fill=fillColor] (410.83,458.06) circle (  1.16);

\path[draw=drawColor,line width= 0.4pt,line join=round,line cap=round,fill=fillColor] (411.17,458.06) circle (  1.16);

\path[draw=drawColor,line width= 0.4pt,line join=round,line cap=round,fill=fillColor] (411.50,458.06) circle (  1.16);

\path[draw=drawColor,line width= 0.4pt,line join=round,line cap=round,fill=fillColor] (411.83,458.06) circle (  1.16);

\path[draw=drawColor,line width= 0.4pt,line join=round,line cap=round,fill=fillColor] (412.16,458.06) circle (  1.16);

\path[draw=drawColor,line width= 0.4pt,line join=round,line cap=round,fill=fillColor] (412.49,458.06) circle (  1.16);

\path[draw=drawColor,line width= 0.4pt,line join=round,line cap=round,fill=fillColor] (412.82,458.06) circle (  1.16);

\path[draw=drawColor,line width= 0.4pt,line join=round,line cap=round,fill=fillColor] (413.14,458.06) circle (  1.16);

\path[draw=drawColor,line width= 0.4pt,line join=round,line cap=round,fill=fillColor] (413.47,458.06) circle (  1.16);

\path[draw=drawColor,line width= 0.4pt,line join=round,line cap=round,fill=fillColor] (413.79,458.06) circle (  1.16);

\path[draw=drawColor,line width= 0.4pt,line join=round,line cap=round,fill=fillColor] (414.10,458.06) circle (  1.16);

\path[draw=drawColor,line width= 0.4pt,line join=round,line cap=round,fill=fillColor] (414.42,458.06) circle (  1.16);

\path[draw=drawColor,line width= 0.4pt,line join=round,line cap=round,fill=fillColor] (414.74,458.06) circle (  1.16);

\path[draw=drawColor,line width= 0.4pt,line join=round,line cap=round,fill=fillColor] (415.05,458.06) circle (  1.16);

\path[draw=drawColor,line width= 0.4pt,line join=round,line cap=round,fill=fillColor] (415.36,458.06) circle (  1.16);

\path[draw=drawColor,line width= 0.4pt,line join=round,line cap=round,fill=fillColor] (415.67,458.06) circle (  1.16);

\path[draw=drawColor,line width= 0.4pt,line join=round,line cap=round,fill=fillColor] (415.98,458.06) circle (  1.16);

\path[draw=drawColor,line width= 0.4pt,line join=round,line cap=round,fill=fillColor] (416.29,458.06) circle (  1.16);

\path[draw=drawColor,line width= 0.4pt,line join=round,line cap=round,fill=fillColor] (416.59,458.06) circle (  1.16);

\path[draw=drawColor,line width= 0.4pt,line join=round,line cap=round,fill=fillColor] (416.89,458.06) circle (  1.16);

\path[draw=drawColor,line width= 0.4pt,line join=round,line cap=round,fill=fillColor] (417.19,458.06) circle (  1.16);

\path[draw=drawColor,line width= 0.4pt,line join=round,line cap=round,fill=fillColor] (417.49,458.06) circle (  1.16);

\path[draw=drawColor,line width= 0.4pt,line join=round,line cap=round,fill=fillColor] (417.79,458.06) circle (  1.16);

\path[draw=drawColor,line width= 0.4pt,line join=round,line cap=round,fill=fillColor] (418.09,458.06) circle (  1.16);

\path[draw=drawColor,line width= 0.4pt,line join=round,line cap=round,fill=fillColor] (418.38,458.06) circle (  1.16);

\path[draw=drawColor,line width= 0.4pt,line join=round,line cap=round,fill=fillColor] (418.68,458.06) circle (  1.16);

\path[draw=drawColor,line width= 0.4pt,line join=round,line cap=round,fill=fillColor] (418.97,458.06) circle (  1.16);

\path[draw=drawColor,line width= 0.4pt,line join=round,line cap=round,fill=fillColor] (419.26,458.06) circle (  1.16);

\path[draw=drawColor,line width= 0.4pt,line join=round,line cap=round,fill=fillColor] (419.55,458.06) circle (  1.16);

\path[draw=drawColor,line width= 0.4pt,line join=round,line cap=round,fill=fillColor] (419.84,458.06) circle (  1.16);

\path[draw=drawColor,line width= 0.4pt,line join=round,line cap=round,fill=fillColor] (420.12,458.06) circle (  1.16);

\path[draw=drawColor,line width= 0.4pt,line join=round,line cap=round,fill=fillColor] (420.41,458.06) circle (  1.16);

\path[draw=drawColor,line width= 0.4pt,line join=round,line cap=round,fill=fillColor] (420.69,458.06) circle (  1.16);

\path[draw=drawColor,line width= 0.4pt,line join=round,line cap=round,fill=fillColor] (420.97,458.06) circle (  1.16);

\path[draw=drawColor,line width= 0.4pt,line join=round,line cap=round,fill=fillColor] (421.25,458.06) circle (  1.16);

\path[draw=drawColor,line width= 0.4pt,line join=round,line cap=round,fill=fillColor] (421.53,458.06) circle (  1.16);

\path[draw=drawColor,line width= 0.4pt,line join=round,line cap=round,fill=fillColor] (421.81,458.06) circle (  1.16);

\path[draw=drawColor,line width= 0.4pt,line join=round,line cap=round,fill=fillColor] (422.09,458.06) circle (  1.16);

\path[draw=drawColor,line width= 0.4pt,line join=round,line cap=round,fill=fillColor] (422.36,458.06) circle (  1.16);

\path[draw=drawColor,line width= 0.4pt,line join=round,line cap=round,fill=fillColor] (422.63,458.06) circle (  1.16);

\path[draw=drawColor,line width= 0.4pt,line join=round,line cap=round,fill=fillColor] (422.91,458.06) circle (  1.16);

\path[draw=drawColor,line width= 0.4pt,line join=round,line cap=round,fill=fillColor] (423.18,458.06) circle (  1.16);

\path[draw=drawColor,line width= 0.4pt,line join=round,line cap=round,fill=fillColor] (423.45,458.06) circle (  1.16);

\path[draw=drawColor,line width= 0.4pt,line join=round,line cap=round,fill=fillColor] (423.72,455.38) circle (  1.16);

\path[draw=drawColor,line width= 0.4pt,line join=round,line cap=round,fill=fillColor] (423.99,454.82) circle (  1.16);

\path[draw=drawColor,line width= 0.4pt,line join=round,line cap=round,fill=fillColor] (424.25,451.74) circle (  1.16);

\path[draw=drawColor,line width= 0.4pt,line join=round,line cap=round,fill=fillColor] (424.52,451.08) circle (  1.16);

\path[draw=drawColor,line width= 0.4pt,line join=round,line cap=round,fill=fillColor] (424.78,450.23) circle (  1.16);

\path[draw=drawColor,line width= 0.4pt,line join=round,line cap=round,fill=fillColor] (425.04,449.99) circle (  1.16);

\path[draw=drawColor,line width= 0.4pt,line join=round,line cap=round,fill=fillColor] (425.31,448.12) circle (  1.16);

\path[draw=drawColor,line width= 0.4pt,line join=round,line cap=round,fill=fillColor] (425.57,448.05) circle (  1.16);

\path[draw=drawColor,line width= 0.4pt,line join=round,line cap=round,fill=fillColor] (425.83,446.25) circle (  1.16);

\path[draw=drawColor,line width= 0.4pt,line join=round,line cap=round,fill=fillColor] (426.08,446.24) circle (  1.16);

\path[draw=drawColor,line width= 0.4pt,line join=round,line cap=round,fill=fillColor] (426.34,443.40) circle (  1.16);

\path[draw=drawColor,line width= 0.4pt,line join=round,line cap=round,fill=fillColor] (426.60,440.08) circle (  1.16);

\path[draw=drawColor,line width= 0.4pt,line join=round,line cap=round,fill=fillColor] (426.85,440.06) circle (  1.16);

\path[draw=drawColor,line width= 0.4pt,line join=round,line cap=round,fill=fillColor] (427.11,440.01) circle (  1.16);

\path[draw=drawColor,line width= 0.4pt,line join=round,line cap=round,fill=fillColor] (427.36,439.81) circle (  1.16);

\path[draw=drawColor,line width= 0.4pt,line join=round,line cap=round,fill=fillColor] (427.61,439.46) circle (  1.16);

\path[draw=drawColor,line width= 0.4pt,line join=round,line cap=round,fill=fillColor] (427.86,439.32) circle (  1.16);

\path[draw=drawColor,line width= 0.4pt,line join=round,line cap=round,fill=fillColor] (428.11,439.28) circle (  1.16);

\path[draw=drawColor,line width= 0.4pt,line join=round,line cap=round,fill=fillColor] (428.36,439.15) circle (  1.16);

\path[draw=drawColor,line width= 0.4pt,line join=round,line cap=round,fill=fillColor] (428.61,439.04) circle (  1.16);

\path[draw=drawColor,line width= 0.4pt,line join=round,line cap=round,fill=fillColor] (428.86,438.62) circle (  1.16);

\path[draw=drawColor,line width= 0.4pt,line join=round,line cap=round,fill=fillColor] (429.10,438.58) circle (  1.16);

\path[draw=drawColor,line width= 0.4pt,line join=round,line cap=round,fill=fillColor] (429.35,437.14) circle (  1.16);

\path[draw=drawColor,line width= 0.4pt,line join=round,line cap=round,fill=fillColor] (429.59,436.29) circle (  1.16);

\path[draw=drawColor,line width= 0.4pt,line join=round,line cap=round,fill=fillColor] (429.83,436.00) circle (  1.16);

\path[draw=drawColor,line width= 0.4pt,line join=round,line cap=round,fill=fillColor] (430.08,432.72) circle (  1.16);

\path[draw=drawColor,line width= 0.4pt,line join=round,line cap=round,fill=fillColor] (430.32,432.54) circle (  1.16);

\path[draw=drawColor,line width= 0.4pt,line join=round,line cap=round,fill=fillColor] (430.56,432.37) circle (  1.16);

\path[draw=drawColor,line width= 0.4pt,line join=round,line cap=round,fill=fillColor] (430.80,431.76) circle (  1.16);

\path[draw=drawColor,line width= 0.4pt,line join=round,line cap=round,fill=fillColor] (431.04,431.70) circle (  1.16);

\path[draw=drawColor,line width= 0.4pt,line join=round,line cap=round,fill=fillColor] (431.27,430.63) circle (  1.16);

\path[draw=drawColor,line width= 0.4pt,line join=round,line cap=round,fill=fillColor] (431.51,429.84) circle (  1.16);

\path[draw=drawColor,line width= 0.4pt,line join=round,line cap=round,fill=fillColor] (431.75,429.49) circle (  1.16);

\path[draw=drawColor,line width= 0.4pt,line join=round,line cap=round,fill=fillColor] (431.98,429.40) circle (  1.16);

\path[draw=drawColor,line width= 0.4pt,line join=round,line cap=round,fill=fillColor] (432.21,426.12) circle (  1.16);

\path[draw=drawColor,line width= 0.4pt,line join=round,line cap=round,fill=fillColor] (432.45,426.11) circle (  1.16);

\path[draw=drawColor,line width= 0.4pt,line join=round,line cap=round,fill=fillColor] (432.68,425.86) circle (  1.16);

\path[draw=drawColor,line width= 0.4pt,line join=round,line cap=round,fill=fillColor] (432.91,425.35) circle (  1.16);

\path[draw=drawColor,line width= 0.4pt,line join=round,line cap=round,fill=fillColor] (433.14,424.82) circle (  1.16);

\path[draw=drawColor,line width= 0.4pt,line join=round,line cap=round,fill=fillColor] (433.37,424.45) circle (  1.16);

\path[draw=drawColor,line width= 0.4pt,line join=round,line cap=round,fill=fillColor] (433.60,424.19) circle (  1.16);

\path[draw=drawColor,line width= 0.4pt,line join=round,line cap=round,fill=fillColor] (433.83,423.69) circle (  1.16);

\path[draw=drawColor,line width= 0.4pt,line join=round,line cap=round,fill=fillColor] (434.06,422.88) circle (  1.16);

\path[draw=drawColor,line width= 0.4pt,line join=round,line cap=round,fill=fillColor] (434.28,422.34) circle (  1.16);

\path[draw=drawColor,line width= 0.4pt,line join=round,line cap=round,fill=fillColor] (434.51,421.68) circle (  1.16);

\path[draw=drawColor,line width= 0.4pt,line join=round,line cap=round,fill=fillColor] (434.73,420.90) circle (  1.16);

\path[draw=drawColor,line width= 0.4pt,line join=round,line cap=round,fill=fillColor] (434.96,420.69) circle (  1.16);

\path[draw=drawColor,line width= 0.4pt,line join=round,line cap=round,fill=fillColor] (435.18,420.65) circle (  1.16);

\path[draw=drawColor,line width= 0.4pt,line join=round,line cap=round,fill=fillColor] (435.40,419.87) circle (  1.16);

\path[draw=drawColor,line width= 0.4pt,line join=round,line cap=round,fill=fillColor] (435.62,419.82) circle (  1.16);

\path[draw=drawColor,line width= 0.4pt,line join=round,line cap=round,fill=fillColor] (435.85,419.61) circle (  1.16);

\path[draw=drawColor,line width= 0.4pt,line join=round,line cap=round,fill=fillColor] (436.07,419.32) circle (  1.16);

\path[draw=drawColor,line width= 0.4pt,line join=round,line cap=round,fill=fillColor] (436.29,419.15) circle (  1.16);

\path[draw=drawColor,line width= 0.4pt,line join=round,line cap=round,fill=fillColor] (436.50,419.08) circle (  1.16);

\path[draw=drawColor,line width= 0.4pt,line join=round,line cap=round,fill=fillColor] (436.72,418.69) circle (  1.16);

\path[draw=drawColor,line width= 0.4pt,line join=round,line cap=round,fill=fillColor] (436.94,418.37) circle (  1.16);

\path[draw=drawColor,line width= 0.4pt,line join=round,line cap=round,fill=fillColor] (437.16,418.32) circle (  1.16);

\path[draw=drawColor,line width= 0.4pt,line join=round,line cap=round,fill=fillColor] (437.37,417.21) circle (  1.16);

\path[draw=drawColor,line width= 0.4pt,line join=round,line cap=round,fill=fillColor] (437.59,416.83) circle (  1.16);

\path[draw=drawColor,line width= 0.4pt,line join=round,line cap=round,fill=fillColor] (437.80,416.83) circle (  1.16);

\path[draw=drawColor,line width= 0.4pt,line join=round,line cap=round,fill=fillColor] (438.02,416.81) circle (  1.16);

\path[draw=drawColor,line width= 0.4pt,line join=round,line cap=round,fill=fillColor] (438.23,416.74) circle (  1.16);

\path[draw=drawColor,line width= 0.4pt,line join=round,line cap=round,fill=fillColor] (438.44,416.65) circle (  1.16);

\path[draw=drawColor,line width= 0.4pt,line join=round,line cap=round,fill=fillColor] (438.65,415.90) circle (  1.16);

\path[draw=drawColor,line width= 0.4pt,line join=round,line cap=round,fill=fillColor] (438.87,415.74) circle (  1.16);

\path[draw=drawColor,line width= 0.4pt,line join=round,line cap=round,fill=fillColor] (439.08,415.23) circle (  1.16);

\path[draw=drawColor,line width= 0.4pt,line join=round,line cap=round,fill=fillColor] (439.29,415.20) circle (  1.16);

\path[draw=drawColor,line width= 0.4pt,line join=round,line cap=round,fill=fillColor] (439.49,413.67) circle (  1.16);

\path[draw=drawColor,line width= 0.4pt,line join=round,line cap=round,fill=fillColor] (439.70,413.33) circle (  1.16);

\path[draw=drawColor,line width= 0.4pt,line join=round,line cap=round,fill=fillColor] (439.91,413.20) circle (  1.16);

\path[draw=drawColor,line width= 0.4pt,line join=round,line cap=round,fill=fillColor] (440.12,412.92) circle (  1.16);

\path[draw=drawColor,line width= 0.4pt,line join=round,line cap=round,fill=fillColor] (440.33,412.90) circle (  1.16);

\path[draw=drawColor,line width= 0.4pt,line join=round,line cap=round,fill=fillColor] (440.53,412.80) circle (  1.16);

\path[draw=drawColor,line width= 0.4pt,line join=round,line cap=round,fill=fillColor] (440.74,412.79) circle (  1.16);

\path[draw=drawColor,line width= 0.4pt,line join=round,line cap=round,fill=fillColor] (440.94,412.70) circle (  1.16);

\path[draw=drawColor,line width= 0.4pt,line join=round,line cap=round,fill=fillColor] (441.15,412.16) circle (  1.16);

\path[draw=drawColor,line width= 0.4pt,line join=round,line cap=round,fill=fillColor] (441.35,411.91) circle (  1.16);

\path[draw=drawColor,line width= 0.4pt,line join=round,line cap=round,fill=fillColor] (441.55,411.88) circle (  1.16);

\path[draw=drawColor,line width= 0.4pt,line join=round,line cap=round,fill=fillColor] (441.75,411.62) circle (  1.16);

\path[draw=drawColor,line width= 0.4pt,line join=round,line cap=round,fill=fillColor] (441.96,411.33) circle (  1.16);

\path[draw=drawColor,line width= 0.4pt,line join=round,line cap=round,fill=fillColor] (442.16,411.33) circle (  1.16);

\path[draw=drawColor,line width= 0.4pt,line join=round,line cap=round,fill=fillColor] (442.36,410.61) circle (  1.16);

\path[draw=drawColor,line width= 0.4pt,line join=round,line cap=round,fill=fillColor] (442.56,410.59) circle (  1.16);

\path[draw=drawColor,line width= 0.4pt,line join=round,line cap=round,fill=fillColor] (442.76,410.59) circle (  1.16);

\path[draw=drawColor,line width= 0.4pt,line join=round,line cap=round,fill=fillColor] (442.96,410.36) circle (  1.16);

\path[draw=drawColor,line width= 0.4pt,line join=round,line cap=round,fill=fillColor] (443.15,410.18) circle (  1.16);

\path[draw=drawColor,line width= 0.4pt,line join=round,line cap=round,fill=fillColor] (443.35,410.07) circle (  1.16);

\path[draw=drawColor,line width= 0.4pt,line join=round,line cap=round,fill=fillColor] (443.55,409.95) circle (  1.16);

\path[draw=drawColor,line width= 0.4pt,line join=round,line cap=round,fill=fillColor] (443.74,409.72) circle (  1.16);

\path[draw=drawColor,line width= 0.4pt,line join=round,line cap=round,fill=fillColor] (443.94,409.72) circle (  1.16);

\path[draw=drawColor,line width= 0.4pt,line join=round,line cap=round,fill=fillColor] (444.14,409.50) circle (  1.16);

\path[draw=drawColor,line width= 0.4pt,line join=round,line cap=round,fill=fillColor] (444.33,409.44) circle (  1.16);

\path[draw=drawColor,line width= 0.4pt,line join=round,line cap=round,fill=fillColor] (444.53,409.18) circle (  1.16);

\path[draw=drawColor,line width= 0.4pt,line join=round,line cap=round,fill=fillColor] (444.72,409.17) circle (  1.16);

\path[draw=drawColor,line width= 0.4pt,line join=round,line cap=round,fill=fillColor] (444.91,409.02) circle (  1.16);

\path[draw=drawColor,line width= 0.4pt,line join=round,line cap=round,fill=fillColor] (445.10,408.82) circle (  1.16);

\path[draw=drawColor,line width= 0.4pt,line join=round,line cap=round,fill=fillColor] (445.30,408.79) circle (  1.16);

\path[draw=drawColor,line width= 0.4pt,line join=round,line cap=round,fill=fillColor] (445.49,408.42) circle (  1.16);

\path[draw=drawColor,line width= 0.4pt,line join=round,line cap=round,fill=fillColor] (445.68,408.41) circle (  1.16);

\path[draw=drawColor,line width= 0.4pt,line join=round,line cap=round,fill=fillColor] (445.87,408.40) circle (  1.16);

\path[draw=drawColor,line width= 0.4pt,line join=round,line cap=round,fill=fillColor] (446.06,408.15) circle (  1.16);

\path[draw=drawColor,line width= 0.4pt,line join=round,line cap=round,fill=fillColor] (446.25,408.02) circle (  1.16);

\path[draw=drawColor,line width= 0.4pt,line join=round,line cap=round,fill=fillColor] (446.44,408.01) circle (  1.16);

\path[draw=drawColor,line width= 0.4pt,line join=round,line cap=round,fill=fillColor] (446.63,407.90) circle (  1.16);

\path[draw=drawColor,line width= 0.4pt,line join=round,line cap=round,fill=fillColor] (446.82,407.82) circle (  1.16);

\path[draw=drawColor,line width= 0.4pt,line join=round,line cap=round,fill=fillColor] (447.00,407.79) circle (  1.16);

\path[draw=drawColor,line width= 0.4pt,line join=round,line cap=round,fill=fillColor] (447.19,407.62) circle (  1.16);

\path[draw=drawColor,line width= 0.4pt,line join=round,line cap=round,fill=fillColor] (447.38,407.44) circle (  1.16);

\path[draw=drawColor,line width= 0.4pt,line join=round,line cap=round,fill=fillColor] (447.56,407.38) circle (  1.16);

\path[draw=drawColor,line width= 0.4pt,line join=round,line cap=round,fill=fillColor] (447.75,407.14) circle (  1.16);

\path[draw=drawColor,line width= 0.4pt,line join=round,line cap=round,fill=fillColor] (447.93,407.09) circle (  1.16);

\path[draw=drawColor,line width= 0.4pt,line join=round,line cap=round,fill=fillColor] (448.12,406.94) circle (  1.16);

\path[draw=drawColor,line width= 0.4pt,line join=round,line cap=round,fill=fillColor] (448.30,406.86) circle (  1.16);

\path[draw=drawColor,line width= 0.4pt,line join=round,line cap=round,fill=fillColor] (448.49,406.79) circle (  1.16);

\path[draw=drawColor,line width= 0.4pt,line join=round,line cap=round,fill=fillColor] (448.67,406.71) circle (  1.16);

\path[draw=drawColor,line width= 0.4pt,line join=round,line cap=round,fill=fillColor] (448.85,406.68) circle (  1.16);

\path[draw=drawColor,line width= 0.4pt,line join=round,line cap=round,fill=fillColor] (449.03,406.66) circle (  1.16);

\path[draw=drawColor,line width= 0.4pt,line join=round,line cap=round,fill=fillColor] (449.22,406.61) circle (  1.16);

\path[draw=drawColor,line width= 0.4pt,line join=round,line cap=round,fill=fillColor] (449.40,406.60) circle (  1.16);

\path[draw=drawColor,line width= 0.4pt,line join=round,line cap=round,fill=fillColor] (449.58,406.48) circle (  1.16);

\path[draw=drawColor,line width= 0.4pt,line join=round,line cap=round,fill=fillColor] (449.76,406.39) circle (  1.16);

\path[draw=drawColor,line width= 0.4pt,line join=round,line cap=round,fill=fillColor] (449.94,406.00) circle (  1.16);

\path[draw=drawColor,line width= 0.4pt,line join=round,line cap=round,fill=fillColor] (450.12,405.68) circle (  1.16);

\path[draw=drawColor,line width= 0.4pt,line join=round,line cap=round,fill=fillColor] (450.30,405.60) circle (  1.16);

\path[draw=drawColor,line width= 0.4pt,line join=round,line cap=round,fill=fillColor] (450.48,405.52) circle (  1.16);

\path[draw=drawColor,line width= 0.4pt,line join=round,line cap=round,fill=fillColor] (450.65,405.37) circle (  1.16);

\path[draw=drawColor,line width= 0.4pt,line join=round,line cap=round,fill=fillColor] (450.83,405.37) circle (  1.16);

\path[draw=drawColor,line width= 0.4pt,line join=round,line cap=round,fill=fillColor] (451.01,405.15) circle (  1.16);

\path[draw=drawColor,line width= 0.4pt,line join=round,line cap=round,fill=fillColor] (451.19,404.94) circle (  1.16);

\path[draw=drawColor,line width= 0.4pt,line join=round,line cap=round,fill=fillColor] (451.36,404.86) circle (  1.16);

\path[draw=drawColor,line width= 0.4pt,line join=round,line cap=round,fill=fillColor] (451.54,404.65) circle (  1.16);

\path[draw=drawColor,line width= 0.4pt,line join=round,line cap=round,fill=fillColor] (451.71,404.64) circle (  1.16);

\path[draw=drawColor,line width= 0.4pt,line join=round,line cap=round,fill=fillColor] (451.89,404.55) circle (  1.16);

\path[draw=drawColor,line width= 0.4pt,line join=round,line cap=round,fill=fillColor] (452.06,404.37) circle (  1.16);

\path[draw=drawColor,line width= 0.4pt,line join=round,line cap=round,fill=fillColor] (452.24,404.37) circle (  1.16);

\path[draw=drawColor,line width= 0.4pt,line join=round,line cap=round,fill=fillColor] (452.41,404.36) circle (  1.16);

\path[draw=drawColor,line width= 0.4pt,line join=round,line cap=round,fill=fillColor] (452.59,404.20) circle (  1.16);

\path[draw=drawColor,line width= 0.4pt,line join=round,line cap=round,fill=fillColor] (452.76,404.09) circle (  1.16);

\path[draw=drawColor,line width= 0.4pt,line join=round,line cap=round,fill=fillColor] (452.93,403.76) circle (  1.16);

\path[draw=drawColor,line width= 0.4pt,line join=round,line cap=round,fill=fillColor] (453.10,403.70) circle (  1.16);

\path[draw=drawColor,line width= 0.4pt,line join=round,line cap=round,fill=fillColor] (453.28,403.63) circle (  1.16);

\path[draw=drawColor,line width= 0.4pt,line join=round,line cap=round,fill=fillColor] (453.45,403.62) circle (  1.16);

\path[draw=drawColor,line width= 0.4pt,line join=round,line cap=round,fill=fillColor] (453.62,403.47) circle (  1.16);

\path[draw=drawColor,line width= 0.4pt,line join=round,line cap=round,fill=fillColor] (453.79,403.46) circle (  1.16);

\path[draw=drawColor,line width= 0.4pt,line join=round,line cap=round,fill=fillColor] (453.96,403.41) circle (  1.16);

\path[draw=drawColor,line width= 0.4pt,line join=round,line cap=round,fill=fillColor] (454.13,403.41) circle (  1.16);

\path[draw=drawColor,line width= 0.4pt,line join=round,line cap=round,fill=fillColor] (454.30,403.24) circle (  1.16);

\path[draw=drawColor,line width= 0.4pt,line join=round,line cap=round,fill=fillColor] (454.47,403.24) circle (  1.16);

\path[draw=drawColor,line width= 0.4pt,line join=round,line cap=round,fill=fillColor] (454.64,403.19) circle (  1.16);

\path[draw=drawColor,line width= 0.4pt,line join=round,line cap=round,fill=fillColor] (454.81,403.18) circle (  1.16);

\path[draw=drawColor,line width= 0.4pt,line join=round,line cap=round,fill=fillColor] (454.97,403.17) circle (  1.16);

\path[draw=drawColor,line width= 0.4pt,line join=round,line cap=round,fill=fillColor] (455.14,403.16) circle (  1.16);

\path[draw=drawColor,line width= 0.4pt,line join=round,line cap=round,fill=fillColor] (455.31,403.16) circle (  1.16);

\path[draw=drawColor,line width= 0.4pt,line join=round,line cap=round,fill=fillColor] (455.48,403.14) circle (  1.16);

\path[draw=drawColor,line width= 0.4pt,line join=round,line cap=round,fill=fillColor] (455.64,403.01) circle (  1.16);

\path[draw=drawColor,line width= 0.4pt,line join=round,line cap=round,fill=fillColor] (455.81,402.82) circle (  1.16);

\path[draw=drawColor,line width= 0.4pt,line join=round,line cap=round,fill=fillColor] (455.97,402.71) circle (  1.16);

\path[draw=drawColor,line width= 0.4pt,line join=round,line cap=round,fill=fillColor] (456.14,402.41) circle (  1.16);

\path[draw=drawColor,line width= 0.4pt,line join=round,line cap=round,fill=fillColor] (456.31,402.21) circle (  1.16);

\path[draw=drawColor,line width= 0.4pt,line join=round,line cap=round,fill=fillColor] (456.47,402.10) circle (  1.16);

\path[draw=drawColor,line width= 0.4pt,line join=round,line cap=round,fill=fillColor] (456.63,402.03) circle (  1.16);

\path[draw=drawColor,line width= 0.4pt,line join=round,line cap=round,fill=fillColor] (456.80,401.93) circle (  1.16);

\path[draw=drawColor,line width= 0.4pt,line join=round,line cap=round,fill=fillColor] (456.96,401.92) circle (  1.16);

\path[draw=drawColor,line width= 0.4pt,line join=round,line cap=round,fill=fillColor] (457.13,401.86) circle (  1.16);

\path[draw=drawColor,line width= 0.4pt,line join=round,line cap=round,fill=fillColor] (457.29,401.82) circle (  1.16);

\path[draw=drawColor,line width= 0.4pt,line join=round,line cap=round,fill=fillColor] (457.45,401.81) circle (  1.16);

\path[draw=drawColor,line width= 0.4pt,line join=round,line cap=round,fill=fillColor] (457.61,401.75) circle (  1.16);

\path[draw=drawColor,line width= 0.4pt,line join=round,line cap=round,fill=fillColor] (457.78,401.53) circle (  1.16);

\path[draw=drawColor,line width= 0.4pt,line join=round,line cap=round,fill=fillColor] (457.94,401.24) circle (  1.16);

\path[draw=drawColor,line width= 0.4pt,line join=round,line cap=round,fill=fillColor] (458.10,400.86) circle (  1.16);

\path[draw=drawColor,line width= 0.4pt,line join=round,line cap=round,fill=fillColor] (458.26,400.81) circle (  1.16);

\path[draw=drawColor,line width= 0.4pt,line join=round,line cap=round,fill=fillColor] (458.42,400.80) circle (  1.16);

\path[draw=drawColor,line width= 0.4pt,line join=round,line cap=round,fill=fillColor] (458.58,400.68) circle (  1.16);

\path[draw=drawColor,line width= 0.4pt,line join=round,line cap=round,fill=fillColor] (458.74,400.48) circle (  1.16);

\path[draw=drawColor,line width= 0.4pt,line join=round,line cap=round,fill=fillColor] (458.90,400.38) circle (  1.16);

\path[draw=drawColor,line width= 0.4pt,line join=round,line cap=round,fill=fillColor] (459.06,400.31) circle (  1.16);

\path[draw=drawColor,line width= 0.4pt,line join=round,line cap=round,fill=fillColor] (459.22,400.31) circle (  1.16);

\path[draw=drawColor,line width= 0.4pt,line join=round,line cap=round,fill=fillColor] (459.38,400.16) circle (  1.16);

\path[draw=drawColor,line width= 0.4pt,line join=round,line cap=round,fill=fillColor] (459.53,400.00) circle (  1.16);

\path[draw=drawColor,line width= 0.4pt,line join=round,line cap=round,fill=fillColor] (459.69,399.97) circle (  1.16);

\path[draw=drawColor,line width= 0.4pt,line join=round,line cap=round,fill=fillColor] (459.85,399.92) circle (  1.16);

\path[draw=drawColor,line width= 0.4pt,line join=round,line cap=round,fill=fillColor] (460.01,399.87) circle (  1.16);

\path[draw=drawColor,line width= 0.4pt,line join=round,line cap=round,fill=fillColor] (460.16,399.76) circle (  1.16);

\path[draw=drawColor,line width= 0.4pt,line join=round,line cap=round,fill=fillColor] (460.32,399.70) circle (  1.16);

\path[draw=drawColor,line width= 0.4pt,line join=round,line cap=round,fill=fillColor] (460.48,399.62) circle (  1.16);

\path[draw=drawColor,line width= 0.4pt,line join=round,line cap=round,fill=fillColor] (460.63,399.59) circle (  1.16);

\path[draw=drawColor,line width= 0.4pt,line join=round,line cap=round,fill=fillColor] (460.79,399.33) circle (  1.16);

\path[draw=drawColor,line width= 0.4pt,line join=round,line cap=round,fill=fillColor] (460.94,399.28) circle (  1.16);

\path[draw=drawColor,line width= 0.4pt,line join=round,line cap=round,fill=fillColor] (461.10,399.27) circle (  1.16);

\path[draw=drawColor,line width= 0.4pt,line join=round,line cap=round,fill=fillColor] (461.25,399.26) circle (  1.16);

\path[draw=drawColor,line width= 0.4pt,line join=round,line cap=round,fill=fillColor] (461.41,399.19) circle (  1.16);

\path[draw=drawColor,line width= 0.4pt,line join=round,line cap=round,fill=fillColor] (461.56,399.14) circle (  1.16);

\path[draw=drawColor,line width= 0.4pt,line join=round,line cap=round,fill=fillColor] (461.72,398.98) circle (  1.16);

\path[draw=drawColor,line width= 0.4pt,line join=round,line cap=round,fill=fillColor] (461.87,398.85) circle (  1.16);

\path[draw=drawColor,line width= 0.4pt,line join=round,line cap=round,fill=fillColor] (462.02,398.78) circle (  1.16);

\path[draw=drawColor,line width= 0.4pt,line join=round,line cap=round,fill=fillColor] (462.18,398.77) circle (  1.16);

\path[draw=drawColor,line width= 0.4pt,line join=round,line cap=round,fill=fillColor] (462.33,398.31) circle (  1.16);

\path[draw=drawColor,line width= 0.4pt,line join=round,line cap=round,fill=fillColor] (462.48,398.23) circle (  1.16);

\path[draw=drawColor,line width= 0.4pt,line join=round,line cap=round,fill=fillColor] (462.63,398.06) circle (  1.16);

\path[draw=drawColor,line width= 0.4pt,line join=round,line cap=round,fill=fillColor] (462.78,397.96) circle (  1.16);

\path[draw=drawColor,line width= 0.4pt,line join=round,line cap=round,fill=fillColor] (462.94,397.91) circle (  1.16);

\path[draw=drawColor,line width= 0.4pt,line join=round,line cap=round,fill=fillColor] (463.09,397.89) circle (  1.16);

\path[draw=drawColor,line width= 0.4pt,line join=round,line cap=round,fill=fillColor] (463.24,397.88) circle (  1.16);

\path[draw=drawColor,line width= 0.4pt,line join=round,line cap=round,fill=fillColor] (463.39,397.81) circle (  1.16);

\path[draw=drawColor,line width= 0.4pt,line join=round,line cap=round,fill=fillColor] (463.54,397.78) circle (  1.16);

\path[draw=drawColor,line width= 0.4pt,line join=round,line cap=round,fill=fillColor] (463.69,397.54) circle (  1.16);

\path[draw=drawColor,line width= 0.4pt,line join=round,line cap=round,fill=fillColor] (463.84,397.35) circle (  1.16);

\path[draw=drawColor,line width= 0.4pt,line join=round,line cap=round,fill=fillColor] (463.99,397.24) circle (  1.16);

\path[draw=drawColor,line width= 0.4pt,line join=round,line cap=round,fill=fillColor] (464.14,397.22) circle (  1.16);

\path[draw=drawColor,line width= 0.4pt,line join=round,line cap=round,fill=fillColor] (464.29,397.12) circle (  1.16);

\path[draw=drawColor,line width= 0.4pt,line join=round,line cap=round,fill=fillColor] (464.44,397.10) circle (  1.16);

\path[draw=drawColor,line width= 0.4pt,line join=round,line cap=round,fill=fillColor] (464.58,397.09) circle (  1.16);

\path[draw=drawColor,line width= 0.4pt,line join=round,line cap=round,fill=fillColor] (464.73,396.97) circle (  1.16);

\path[draw=drawColor,line width= 0.4pt,line join=round,line cap=round,fill=fillColor] (464.88,396.90) circle (  1.16);

\path[draw=drawColor,line width= 0.4pt,line join=round,line cap=round,fill=fillColor] (465.03,396.45) circle (  1.16);

\path[draw=drawColor,line width= 0.4pt,line join=round,line cap=round,fill=fillColor] (465.17,396.38) circle (  1.16);

\path[draw=drawColor,line width= 0.4pt,line join=round,line cap=round,fill=fillColor] (465.32,396.14) circle (  1.16);

\path[draw=drawColor,line width= 0.4pt,line join=round,line cap=round,fill=fillColor] (465.47,395.96) circle (  1.16);

\path[draw=drawColor,line width= 0.4pt,line join=round,line cap=round,fill=fillColor] (465.61,395.94) circle (  1.16);

\path[draw=drawColor,line width= 0.4pt,line join=round,line cap=round,fill=fillColor] (465.76,395.86) circle (  1.16);

\path[draw=drawColor,line width= 0.4pt,line join=round,line cap=round,fill=fillColor] (465.91,395.78) circle (  1.16);

\path[draw=drawColor,line width= 0.4pt,line join=round,line cap=round,fill=fillColor] (466.05,395.75) circle (  1.16);

\path[draw=drawColor,line width= 0.4pt,line join=round,line cap=round,fill=fillColor] (466.20,395.63) circle (  1.16);

\path[draw=drawColor,line width= 0.4pt,line join=round,line cap=round,fill=fillColor] (466.34,395.57) circle (  1.16);

\path[draw=drawColor,line width= 0.4pt,line join=round,line cap=round,fill=fillColor] (466.49,395.34) circle (  1.16);

\path[draw=drawColor,line width= 0.4pt,line join=round,line cap=round,fill=fillColor] (466.63,395.29) circle (  1.16);

\path[draw=drawColor,line width= 0.4pt,line join=round,line cap=round,fill=fillColor] (466.78,395.22) circle (  1.16);

\path[draw=drawColor,line width= 0.4pt,line join=round,line cap=round,fill=fillColor] (466.92,394.90) circle (  1.16);

\path[draw=drawColor,line width= 0.4pt,line join=round,line cap=round,fill=fillColor] (467.06,394.74) circle (  1.16);

\path[draw=drawColor,line width= 0.4pt,line join=round,line cap=round,fill=fillColor] (467.21,394.63) circle (  1.16);

\path[draw=drawColor,line width= 0.4pt,line join=round,line cap=round,fill=fillColor] (467.35,394.52) circle (  1.16);

\path[draw=drawColor,line width= 0.4pt,line join=round,line cap=round,fill=fillColor] (467.49,394.45) circle (  1.16);

\path[draw=drawColor,line width= 0.4pt,line join=round,line cap=round,fill=fillColor] (467.64,394.42) circle (  1.16);

\path[draw=drawColor,line width= 0.4pt,line join=round,line cap=round,fill=fillColor] (467.78,394.33) circle (  1.16);

\path[draw=drawColor,line width= 0.4pt,line join=round,line cap=round,fill=fillColor] (467.92,394.29) circle (  1.16);

\path[draw=drawColor,line width= 0.4pt,line join=round,line cap=round,fill=fillColor] (468.06,394.22) circle (  1.16);

\path[draw=drawColor,line width= 0.4pt,line join=round,line cap=round,fill=fillColor] (468.21,394.09) circle (  1.16);

\path[draw=drawColor,line width= 0.4pt,line join=round,line cap=round,fill=fillColor] (468.35,393.98) circle (  1.16);

\path[draw=drawColor,line width= 0.4pt,line join=round,line cap=round,fill=fillColor] (468.49,393.95) circle (  1.16);

\path[draw=drawColor,line width= 0.4pt,line join=round,line cap=round,fill=fillColor] (468.63,393.82) circle (  1.16);

\path[draw=drawColor,line width= 0.4pt,line join=round,line cap=round,fill=fillColor] (468.77,393.81) circle (  1.16);

\path[draw=drawColor,line width= 0.4pt,line join=round,line cap=round,fill=fillColor] (468.91,393.54) circle (  1.16);

\path[draw=drawColor,line width= 0.4pt,line join=round,line cap=round,fill=fillColor] (469.05,393.29) circle (  1.16);

\path[draw=drawColor,line width= 0.4pt,line join=round,line cap=round,fill=fillColor] (469.19,393.08) circle (  1.16);

\path[draw=drawColor,line width= 0.4pt,line join=round,line cap=round,fill=fillColor] (469.33,392.90) circle (  1.16);

\path[draw=drawColor,line width= 0.4pt,line join=round,line cap=round,fill=fillColor] (469.47,392.48) circle (  1.16);

\path[draw=drawColor,line width= 0.4pt,line join=round,line cap=round,fill=fillColor] (469.61,392.42) circle (  1.16);

\path[draw=drawColor,line width= 0.4pt,line join=round,line cap=round,fill=fillColor] (469.75,392.35) circle (  1.16);

\path[draw=drawColor,line width= 0.4pt,line join=round,line cap=round,fill=fillColor] (469.89,392.25) circle (  1.16);

\path[draw=drawColor,line width= 0.4pt,line join=round,line cap=round,fill=fillColor] (470.03,392.00) circle (  1.16);

\path[draw=drawColor,line width= 0.4pt,line join=round,line cap=round,fill=fillColor] (470.17,391.88) circle (  1.16);

\path[draw=drawColor,line width= 0.4pt,line join=round,line cap=round,fill=fillColor] (470.30,391.88) circle (  1.16);

\path[draw=drawColor,line width= 0.4pt,line join=round,line cap=round,fill=fillColor] (470.44,391.81) circle (  1.16);

\path[draw=drawColor,line width= 0.4pt,line join=round,line cap=round,fill=fillColor] (470.58,391.67) circle (  1.16);

\path[draw=drawColor,line width= 0.4pt,line join=round,line cap=round,fill=fillColor] (470.72,391.26) circle (  1.16);

\path[draw=drawColor,line width= 0.4pt,line join=round,line cap=round,fill=fillColor] (470.85,391.22) circle (  1.16);

\path[draw=drawColor,line width= 0.4pt,line join=round,line cap=round,fill=fillColor] (470.99,391.20) circle (  1.16);

\path[draw=drawColor,line width= 0.4pt,line join=round,line cap=round,fill=fillColor] (471.13,390.70) circle (  1.16);

\path[draw=drawColor,line width= 0.4pt,line join=round,line cap=round,fill=fillColor] (471.27,390.18) circle (  1.16);

\path[draw=drawColor,line width= 0.4pt,line join=round,line cap=round,fill=fillColor] (471.40,389.96) circle (  1.16);

\path[draw=drawColor,line width= 0.4pt,line join=round,line cap=round,fill=fillColor] (471.54,389.94) circle (  1.16);

\path[draw=drawColor,line width= 0.4pt,line join=round,line cap=round,fill=fillColor] (471.67,389.81) circle (  1.16);

\path[draw=drawColor,line width= 0.4pt,line join=round,line cap=round,fill=fillColor] (471.81,389.80) circle (  1.16);

\path[draw=drawColor,line width= 0.4pt,line join=round,line cap=round,fill=fillColor] (471.94,389.34) circle (  1.16);

\path[draw=drawColor,line width= 0.4pt,line join=round,line cap=round,fill=fillColor] (472.08,389.25) circle (  1.16);

\path[draw=drawColor,line width= 0.4pt,line join=round,line cap=round,fill=fillColor] (472.22,388.94) circle (  1.16);

\path[draw=drawColor,line width= 0.4pt,line join=round,line cap=round,fill=fillColor] (472.35,388.89) circle (  1.16);

\path[draw=drawColor,line width= 0.4pt,line join=round,line cap=round,fill=fillColor] (472.48,388.82) circle (  1.16);

\path[draw=drawColor,line width= 0.4pt,line join=round,line cap=round,fill=fillColor] (472.62,388.75) circle (  1.16);

\path[draw=drawColor,line width= 0.4pt,line join=round,line cap=round,fill=fillColor] (472.75,388.62) circle (  1.16);

\path[draw=drawColor,line width= 0.4pt,line join=round,line cap=round,fill=fillColor] (472.89,388.62) circle (  1.16);

\path[draw=drawColor,line width= 0.4pt,line join=round,line cap=round,fill=fillColor] (473.02,387.73) circle (  1.16);

\path[draw=drawColor,line width= 0.4pt,line join=round,line cap=round,fill=fillColor] (473.15,386.82) circle (  1.16);

\path[draw=drawColor,line width= 0.4pt,line join=round,line cap=round,fill=fillColor] (473.29,386.80) circle (  1.16);

\path[draw=drawColor,line width= 0.4pt,line join=round,line cap=round,fill=fillColor] (473.42,386.48) circle (  1.16);

\path[draw=drawColor,line width= 0.4pt,line join=round,line cap=round,fill=fillColor] (473.55,383.88) circle (  1.16);

\path[draw=drawColor,line width= 0.4pt,line join=round,line cap=round,fill=fillColor] (473.69,383.88) circle (  1.16);

\path[draw=drawColor,line width= 0.4pt,line join=round,line cap=round,fill=fillColor] (473.82,383.76) circle (  1.16);

\path[draw=drawColor,line width= 0.4pt,line join=round,line cap=round,fill=fillColor] (473.95,383.68) circle (  1.16);

\path[draw=drawColor,line width= 0.4pt,line join=round,line cap=round,fill=fillColor] (474.08,380.11) circle (  1.16);

\path[draw=drawColor,line width= 0.4pt,line join=round,line cap=round,fill=fillColor] (474.22,379.72) circle (  1.16);

\path[draw=drawColor,line width= 0.4pt,line join=round,line cap=round,fill=fillColor] (474.35,379.33) circle (  1.16);

\path[draw=drawColor,line width= 0.4pt,line join=round,line cap=round,fill=fillColor] (474.48,376.62) circle (  1.16);

\path[draw=drawColor,line width= 0.4pt,line join=round,line cap=round,fill=fillColor] (474.61,372.42) circle (  1.16);

\path[draw=drawColor,line width= 0.4pt,line join=round,line cap=round,fill=fillColor] (474.74,372.42) circle (  1.16);

\path[draw=drawColor,line width= 0.4pt,line join=round,line cap=round,fill=fillColor] (474.87,372.42) circle (  1.16);

\path[draw=drawColor,line width= 0.4pt,line join=round,line cap=round,fill=fillColor] (475.00,372.42) circle (  1.16);

\path[draw=drawColor,line width= 0.4pt,line join=round,line cap=round,fill=fillColor] (475.13,372.42) circle (  1.16);

\path[draw=drawColor,line width= 0.4pt,line join=round,line cap=round,fill=fillColor] (475.26,372.42) circle (  1.16);

\path[draw=drawColor,line width= 0.4pt,line join=round,line cap=round,fill=fillColor] (475.39,372.42) circle (  1.16);

\path[draw=drawColor,line width= 0.4pt,line join=round,line cap=round,fill=fillColor] (475.52,372.42) circle (  1.16);

\path[draw=drawColor,line width= 0.4pt,line join=round,line cap=round,fill=fillColor] (475.65,372.42) circle (  1.16);

\path[draw=drawColor,line width= 0.4pt,line join=round,line cap=round,fill=fillColor] (475.78,372.42) circle (  1.16);

\path[draw=drawColor,line width= 0.4pt,line join=round,line cap=round,fill=fillColor] (475.91,372.42) circle (  1.16);

\path[draw=drawColor,line width= 0.4pt,line join=round,line cap=round,fill=fillColor] (476.04,372.42) circle (  1.16);

\path[draw=drawColor,line width= 0.4pt,line join=round,line cap=round,fill=fillColor] (476.17,372.42) circle (  1.16);

\path[draw=drawColor,line width= 0.4pt,line join=round,line cap=round,fill=fillColor] (476.30,372.42) circle (  1.16);

\path[draw=drawColor,line width= 0.4pt,line join=round,line cap=round,fill=fillColor] (476.43,372.42) circle (  1.16);

\path[draw=drawColor,line width= 0.4pt,line join=round,line cap=round,fill=fillColor] (476.56,372.42) circle (  1.16);

\path[draw=drawColor,line width= 0.4pt,line join=round,line cap=round,fill=fillColor] (476.68,372.42) circle (  1.16);

\path[draw=drawColor,line width= 0.4pt,line join=round,line cap=round,fill=fillColor] (476.81,372.42) circle (  1.16);

\path[draw=drawColor,line width= 0.4pt,line join=round,line cap=round,fill=fillColor] (476.94,372.42) circle (  1.16);

\path[draw=drawColor,line width= 0.4pt,line join=round,line cap=round,fill=fillColor] (477.07,372.42) circle (  1.16);

\path[draw=drawColor,line width= 0.4pt,line join=round,line cap=round,fill=fillColor] (477.19,372.42) circle (  1.16);

\path[draw=drawColor,line width= 0.4pt,line join=round,line cap=round,fill=fillColor] (477.32,372.42) circle (  1.16);

\path[draw=drawColor,line width= 0.4pt,line join=round,line cap=round,fill=fillColor] (477.45,372.42) circle (  1.16);

\path[draw=drawColor,line width= 0.4pt,line join=round,line cap=round,fill=fillColor] (477.57,372.42) circle (  1.16);

\path[draw=drawColor,line width= 0.4pt,line join=round,line cap=round,fill=fillColor] (477.70,372.42) circle (  1.16);

\path[draw=drawColor,line width= 0.4pt,line join=round,line cap=round,fill=fillColor] (477.83,372.42) circle (  1.16);

\path[draw=drawColor,line width= 0.4pt,line join=round,line cap=round,fill=fillColor] (477.95,372.42) circle (  1.16);

\path[draw=drawColor,line width= 0.4pt,line join=round,line cap=round,fill=fillColor] (478.08,372.42) circle (  1.16);

\path[draw=drawColor,line width= 0.4pt,line join=round,line cap=round,fill=fillColor] (478.21,372.42) circle (  1.16);
\definecolor[named]{drawColor}{rgb}{0.60,0.31,0.64}
\definecolor[named]{fillColor}{rgb}{0.60,0.31,0.64}

\path[draw=drawColor,line width= 0.4pt,line join=round,line cap=round,fill=fillColor] (327.82,440.00) circle (  1.16);

\path[draw=drawColor,line width= 0.4pt,line join=round,line cap=round,fill=fillColor] (333.65,436.82) circle (  1.16);

\path[draw=drawColor,line width= 0.4pt,line join=round,line cap=round,fill=fillColor] (337.74,436.35) circle (  1.16);

\path[draw=drawColor,line width= 0.4pt,line join=round,line cap=round,fill=fillColor] (340.99,436.10) circle (  1.16);

\path[draw=drawColor,line width= 0.4pt,line join=round,line cap=round,fill=fillColor] (343.74,434.74) circle (  1.16);

\path[draw=drawColor,line width= 0.4pt,line join=round,line cap=round,fill=fillColor] (346.14,433.59) circle (  1.16);

\path[draw=drawColor,line width= 0.4pt,line join=round,line cap=round,fill=fillColor] (348.29,433.40) circle (  1.16);

\path[draw=drawColor,line width= 0.4pt,line join=round,line cap=round,fill=fillColor] (350.24,432.90) circle (  1.16);

\path[draw=drawColor,line width= 0.4pt,line join=round,line cap=round,fill=fillColor] (352.04,432.80) circle (  1.16);

\path[draw=drawColor,line width= 0.4pt,line join=round,line cap=round,fill=fillColor] (353.70,432.66) circle (  1.16);

\path[draw=drawColor,line width= 0.4pt,line join=round,line cap=round,fill=fillColor] (355.26,432.48) circle (  1.16);

\path[draw=drawColor,line width= 0.4pt,line join=round,line cap=round,fill=fillColor] (356.73,432.17) circle (  1.16);

\path[draw=drawColor,line width= 0.4pt,line join=round,line cap=round,fill=fillColor] (358.12,431.90) circle (  1.16);

\path[draw=drawColor,line width= 0.4pt,line join=round,line cap=round,fill=fillColor] (359.44,431.60) circle (  1.16);

\path[draw=drawColor,line width= 0.4pt,line join=round,line cap=round,fill=fillColor] (360.69,431.28) circle (  1.16);

\path[draw=drawColor,line width= 0.4pt,line join=round,line cap=round,fill=fillColor] (361.90,430.78) circle (  1.16);

\path[draw=drawColor,line width= 0.4pt,line join=round,line cap=round,fill=fillColor] (363.05,430.52) circle (  1.16);

\path[draw=drawColor,line width= 0.4pt,line join=round,line cap=round,fill=fillColor] (364.16,430.45) circle (  1.16);

\path[draw=drawColor,line width= 0.4pt,line join=round,line cap=round,fill=fillColor] (365.23,430.33) circle (  1.16);

\path[draw=drawColor,line width= 0.4pt,line join=round,line cap=round,fill=fillColor] (366.26,430.31) circle (  1.16);

\path[draw=drawColor,line width= 0.4pt,line join=round,line cap=round,fill=fillColor] (367.25,430.18) circle (  1.16);

\path[draw=drawColor,line width= 0.4pt,line join=round,line cap=round,fill=fillColor] (368.22,430.13) circle (  1.16);

\path[draw=drawColor,line width= 0.4pt,line join=round,line cap=round,fill=fillColor] (369.16,429.67) circle (  1.16);

\path[draw=drawColor,line width= 0.4pt,line join=round,line cap=round,fill=fillColor] (370.07,429.21) circle (  1.16);

\path[draw=drawColor,line width= 0.4pt,line join=round,line cap=round,fill=fillColor] (370.96,429.04) circle (  1.16);

\path[draw=drawColor,line width= 0.4pt,line join=round,line cap=round,fill=fillColor] (371.82,429.04) circle (  1.16);

\path[draw=drawColor,line width= 0.4pt,line join=round,line cap=round,fill=fillColor] (372.66,428.89) circle (  1.16);

\path[draw=drawColor,line width= 0.4pt,line join=round,line cap=round,fill=fillColor] (373.48,428.78) circle (  1.16);

\path[draw=drawColor,line width= 0.4pt,line join=round,line cap=round,fill=fillColor] (374.28,428.66) circle (  1.16);

\path[draw=drawColor,line width= 0.4pt,line join=round,line cap=round,fill=fillColor] (375.06,428.55) circle (  1.16);

\path[draw=drawColor,line width= 0.4pt,line join=round,line cap=round,fill=fillColor] (375.83,428.27) circle (  1.16);

\path[draw=drawColor,line width= 0.4pt,line join=round,line cap=round,fill=fillColor] (376.58,427.89) circle (  1.16);

\path[draw=drawColor,line width= 0.4pt,line join=round,line cap=round,fill=fillColor] (377.31,427.66) circle (  1.16);

\path[draw=drawColor,line width= 0.4pt,line join=round,line cap=round,fill=fillColor] (378.03,427.62) circle (  1.16);

\path[draw=drawColor,line width= 0.4pt,line join=round,line cap=round,fill=fillColor] (378.74,427.55) circle (  1.16);

\path[draw=drawColor,line width= 0.4pt,line join=round,line cap=round,fill=fillColor] (379.43,426.96) circle (  1.16);

\path[draw=drawColor,line width= 0.4pt,line join=round,line cap=round,fill=fillColor] (380.11,426.92) circle (  1.16);

\path[draw=drawColor,line width= 0.4pt,line join=round,line cap=round,fill=fillColor] (380.77,426.88) circle (  1.16);

\path[draw=drawColor,line width= 0.4pt,line join=round,line cap=round,fill=fillColor] (381.43,426.72) circle (  1.16);

\path[draw=drawColor,line width= 0.4pt,line join=round,line cap=round,fill=fillColor] (382.07,426.56) circle (  1.16);

\path[draw=drawColor,line width= 0.4pt,line join=round,line cap=round,fill=fillColor] (382.71,426.55) circle (  1.16);

\path[draw=drawColor,line width= 0.4pt,line join=round,line cap=round,fill=fillColor] (383.33,426.26) circle (  1.16);

\path[draw=drawColor,line width= 0.4pt,line join=round,line cap=round,fill=fillColor] (383.94,426.12) circle (  1.16);

\path[draw=drawColor,line width= 0.4pt,line join=round,line cap=round,fill=fillColor] (384.55,426.02) circle (  1.16);

\path[draw=drawColor,line width= 0.4pt,line join=round,line cap=round,fill=fillColor] (385.14,425.64) circle (  1.16);

\path[draw=drawColor,line width= 0.4pt,line join=round,line cap=round,fill=fillColor] (385.73,425.52) circle (  1.16);

\path[draw=drawColor,line width= 0.4pt,line join=round,line cap=round,fill=fillColor] (386.31,425.47) circle (  1.16);

\path[draw=drawColor,line width= 0.4pt,line join=round,line cap=round,fill=fillColor] (386.88,425.38) circle (  1.16);

\path[draw=drawColor,line width= 0.4pt,line join=round,line cap=round,fill=fillColor] (387.44,425.14) circle (  1.16);

\path[draw=drawColor,line width= 0.4pt,line join=round,line cap=round,fill=fillColor] (387.99,425.12) circle (  1.16);

\path[draw=drawColor,line width= 0.4pt,line join=round,line cap=round,fill=fillColor] (388.54,424.75) circle (  1.16);

\path[draw=drawColor,line width= 0.4pt,line join=round,line cap=round,fill=fillColor] (389.08,424.64) circle (  1.16);

\path[draw=drawColor,line width= 0.4pt,line join=round,line cap=round,fill=fillColor] (389.61,423.81) circle (  1.16);

\path[draw=drawColor,line width= 0.4pt,line join=round,line cap=round,fill=fillColor] (390.14,423.45) circle (  1.16);

\path[draw=drawColor,line width= 0.4pt,line join=round,line cap=round,fill=fillColor] (390.66,422.96) circle (  1.16);

\path[draw=drawColor,line width= 0.4pt,line join=round,line cap=round,fill=fillColor] (391.17,422.66) circle (  1.16);

\path[draw=drawColor,line width= 0.4pt,line join=round,line cap=round,fill=fillColor] (391.68,421.60) circle (  1.16);

\path[draw=drawColor,line width= 0.4pt,line join=round,line cap=round,fill=fillColor] (392.18,421.51) circle (  1.16);

\path[draw=drawColor,line width= 0.4pt,line join=round,line cap=round,fill=fillColor] (392.68,421.38) circle (  1.16);

\path[draw=drawColor,line width= 0.4pt,line join=round,line cap=round,fill=fillColor] (393.17,421.37) circle (  1.16);

\path[draw=drawColor,line width= 0.4pt,line join=round,line cap=round,fill=fillColor] (393.65,421.35) circle (  1.16);

\path[draw=drawColor,line width= 0.4pt,line join=round,line cap=round,fill=fillColor] (394.13,421.06) circle (  1.16);

\path[draw=drawColor,line width= 0.4pt,line join=round,line cap=round,fill=fillColor] (394.61,420.67) circle (  1.16);

\path[draw=drawColor,line width= 0.4pt,line join=round,line cap=round,fill=fillColor] (395.08,420.64) circle (  1.16);

\path[draw=drawColor,line width= 0.4pt,line join=round,line cap=round,fill=fillColor] (395.54,420.64) circle (  1.16);

\path[draw=drawColor,line width= 0.4pt,line join=round,line cap=round,fill=fillColor] (396.00,420.60) circle (  1.16);

\path[draw=drawColor,line width= 0.4pt,line join=round,line cap=round,fill=fillColor] (396.46,420.38) circle (  1.16);

\path[draw=drawColor,line width= 0.4pt,line join=round,line cap=round,fill=fillColor] (396.91,420.35) circle (  1.16);

\path[draw=drawColor,line width= 0.4pt,line join=round,line cap=round,fill=fillColor] (397.35,420.30) circle (  1.16);

\path[draw=drawColor,line width= 0.4pt,line join=round,line cap=round,fill=fillColor] (397.80,419.87) circle (  1.16);

\path[draw=drawColor,line width= 0.4pt,line join=round,line cap=round,fill=fillColor] (398.23,419.86) circle (  1.16);

\path[draw=drawColor,line width= 0.4pt,line join=round,line cap=round,fill=fillColor] (398.67,419.76) circle (  1.16);

\path[draw=drawColor,line width= 0.4pt,line join=round,line cap=round,fill=fillColor] (399.10,419.69) circle (  1.16);

\path[draw=drawColor,line width= 0.4pt,line join=round,line cap=round,fill=fillColor] (399.52,419.69) circle (  1.16);

\path[draw=drawColor,line width= 0.4pt,line join=round,line cap=round,fill=fillColor] (399.94,419.67) circle (  1.16);

\path[draw=drawColor,line width= 0.4pt,line join=round,line cap=round,fill=fillColor] (400.36,419.44) circle (  1.16);

\path[draw=drawColor,line width= 0.4pt,line join=round,line cap=round,fill=fillColor] (400.78,419.43) circle (  1.16);

\path[draw=drawColor,line width= 0.4pt,line join=round,line cap=round,fill=fillColor] (401.19,419.30) circle (  1.16);

\path[draw=drawColor,line width= 0.4pt,line join=round,line cap=round,fill=fillColor] (401.60,419.26) circle (  1.16);

\path[draw=drawColor,line width= 0.4pt,line join=round,line cap=round,fill=fillColor] (402.00,418.98) circle (  1.16);

\path[draw=drawColor,line width= 0.4pt,line join=round,line cap=round,fill=fillColor] (402.40,418.95) circle (  1.16);

\path[draw=drawColor,line width= 0.4pt,line join=round,line cap=round,fill=fillColor] (402.80,418.79) circle (  1.16);

\path[draw=drawColor,line width= 0.4pt,line join=round,line cap=round,fill=fillColor] (403.19,418.64) circle (  1.16);

\path[draw=drawColor,line width= 0.4pt,line join=round,line cap=round,fill=fillColor] (403.58,418.61) circle (  1.16);

\path[draw=drawColor,line width= 0.4pt,line join=round,line cap=round,fill=fillColor] (403.97,418.59) circle (  1.16);

\path[draw=drawColor,line width= 0.4pt,line join=round,line cap=round,fill=fillColor] (404.36,418.50) circle (  1.16);

\path[draw=drawColor,line width= 0.4pt,line join=round,line cap=round,fill=fillColor] (404.74,418.38) circle (  1.16);

\path[draw=drawColor,line width= 0.4pt,line join=round,line cap=round,fill=fillColor] (405.12,418.26) circle (  1.16);

\path[draw=drawColor,line width= 0.4pt,line join=round,line cap=round,fill=fillColor] (405.49,418.20) circle (  1.16);

\path[draw=drawColor,line width= 0.4pt,line join=round,line cap=round,fill=fillColor] (405.87,418.17) circle (  1.16);

\path[draw=drawColor,line width= 0.4pt,line join=round,line cap=round,fill=fillColor] (406.24,418.10) circle (  1.16);

\path[draw=drawColor,line width= 0.4pt,line join=round,line cap=round,fill=fillColor] (406.61,417.85) circle (  1.16);

\path[draw=drawColor,line width= 0.4pt,line join=round,line cap=round,fill=fillColor] (406.97,417.81) circle (  1.16);

\path[draw=drawColor,line width= 0.4pt,line join=round,line cap=round,fill=fillColor] (407.34,417.45) circle (  1.16);

\path[draw=drawColor,line width= 0.4pt,line join=round,line cap=round,fill=fillColor] (407.70,417.43) circle (  1.16);

\path[draw=drawColor,line width= 0.4pt,line join=round,line cap=round,fill=fillColor] (408.05,417.37) circle (  1.16);

\path[draw=drawColor,line width= 0.4pt,line join=round,line cap=round,fill=fillColor] (408.41,417.29) circle (  1.16);

\path[draw=drawColor,line width= 0.4pt,line join=round,line cap=round,fill=fillColor] (408.76,417.28) circle (  1.16);

\path[draw=drawColor,line width= 0.4pt,line join=round,line cap=round,fill=fillColor] (409.11,417.22) circle (  1.16);

\path[draw=drawColor,line width= 0.4pt,line join=round,line cap=round,fill=fillColor] (409.46,417.13) circle (  1.16);

\path[draw=drawColor,line width= 0.4pt,line join=round,line cap=round,fill=fillColor] (409.80,417.04) circle (  1.16);

\path[draw=drawColor,line width= 0.4pt,line join=round,line cap=round,fill=fillColor] (410.15,416.96) circle (  1.16);

\path[draw=drawColor,line width= 0.4pt,line join=round,line cap=round,fill=fillColor] (410.49,416.96) circle (  1.16);

\path[draw=drawColor,line width= 0.4pt,line join=round,line cap=round,fill=fillColor] (410.83,416.79) circle (  1.16);

\path[draw=drawColor,line width= 0.4pt,line join=round,line cap=round,fill=fillColor] (411.17,416.75) circle (  1.16);

\path[draw=drawColor,line width= 0.4pt,line join=round,line cap=round,fill=fillColor] (411.50,416.62) circle (  1.16);

\path[draw=drawColor,line width= 0.4pt,line join=round,line cap=round,fill=fillColor] (411.83,416.50) circle (  1.16);

\path[draw=drawColor,line width= 0.4pt,line join=round,line cap=round,fill=fillColor] (412.16,416.39) circle (  1.16);

\path[draw=drawColor,line width= 0.4pt,line join=round,line cap=round,fill=fillColor] (412.49,416.36) circle (  1.16);

\path[draw=drawColor,line width= 0.4pt,line join=round,line cap=round,fill=fillColor] (412.82,416.36) circle (  1.16);

\path[draw=drawColor,line width= 0.4pt,line join=round,line cap=round,fill=fillColor] (413.14,416.13) circle (  1.16);

\path[draw=drawColor,line width= 0.4pt,line join=round,line cap=round,fill=fillColor] (413.47,416.09) circle (  1.16);

\path[draw=drawColor,line width= 0.4pt,line join=round,line cap=round,fill=fillColor] (413.79,415.94) circle (  1.16);

\path[draw=drawColor,line width= 0.4pt,line join=round,line cap=round,fill=fillColor] (414.10,415.83) circle (  1.16);

\path[draw=drawColor,line width= 0.4pt,line join=round,line cap=round,fill=fillColor] (414.42,415.77) circle (  1.16);

\path[draw=drawColor,line width= 0.4pt,line join=round,line cap=round,fill=fillColor] (414.74,415.75) circle (  1.16);

\path[draw=drawColor,line width= 0.4pt,line join=round,line cap=round,fill=fillColor] (415.05,415.62) circle (  1.16);

\path[draw=drawColor,line width= 0.4pt,line join=round,line cap=round,fill=fillColor] (415.36,415.61) circle (  1.16);

\path[draw=drawColor,line width= 0.4pt,line join=round,line cap=round,fill=fillColor] (415.67,415.61) circle (  1.16);

\path[draw=drawColor,line width= 0.4pt,line join=round,line cap=round,fill=fillColor] (415.98,415.56) circle (  1.16);

\path[draw=drawColor,line width= 0.4pt,line join=round,line cap=round,fill=fillColor] (416.29,415.54) circle (  1.16);

\path[draw=drawColor,line width= 0.4pt,line join=round,line cap=round,fill=fillColor] (416.59,415.49) circle (  1.16);

\path[draw=drawColor,line width= 0.4pt,line join=round,line cap=round,fill=fillColor] (416.89,415.42) circle (  1.16);

\path[draw=drawColor,line width= 0.4pt,line join=round,line cap=round,fill=fillColor] (417.19,415.36) circle (  1.16);

\path[draw=drawColor,line width= 0.4pt,line join=round,line cap=round,fill=fillColor] (417.49,415.31) circle (  1.16);

\path[draw=drawColor,line width= 0.4pt,line join=round,line cap=round,fill=fillColor] (417.79,415.30) circle (  1.16);

\path[draw=drawColor,line width= 0.4pt,line join=round,line cap=round,fill=fillColor] (418.09,415.23) circle (  1.16);

\path[draw=drawColor,line width= 0.4pt,line join=round,line cap=round,fill=fillColor] (418.38,415.22) circle (  1.16);

\path[draw=drawColor,line width= 0.4pt,line join=round,line cap=round,fill=fillColor] (418.68,415.15) circle (  1.16);

\path[draw=drawColor,line width= 0.4pt,line join=round,line cap=round,fill=fillColor] (418.97,415.09) circle (  1.16);

\path[draw=drawColor,line width= 0.4pt,line join=round,line cap=round,fill=fillColor] (419.26,415.04) circle (  1.16);

\path[draw=drawColor,line width= 0.4pt,line join=round,line cap=round,fill=fillColor] (419.55,414.86) circle (  1.16);

\path[draw=drawColor,line width= 0.4pt,line join=round,line cap=round,fill=fillColor] (419.84,414.82) circle (  1.16);

\path[draw=drawColor,line width= 0.4pt,line join=round,line cap=round,fill=fillColor] (420.12,414.78) circle (  1.16);

\path[draw=drawColor,line width= 0.4pt,line join=round,line cap=round,fill=fillColor] (420.41,414.76) circle (  1.16);

\path[draw=drawColor,line width= 0.4pt,line join=round,line cap=round,fill=fillColor] (420.69,414.63) circle (  1.16);

\path[draw=drawColor,line width= 0.4pt,line join=round,line cap=round,fill=fillColor] (420.97,414.62) circle (  1.16);

\path[draw=drawColor,line width= 0.4pt,line join=round,line cap=round,fill=fillColor] (421.25,414.49) circle (  1.16);

\path[draw=drawColor,line width= 0.4pt,line join=round,line cap=round,fill=fillColor] (421.53,414.46) circle (  1.16);

\path[draw=drawColor,line width= 0.4pt,line join=round,line cap=round,fill=fillColor] (421.81,414.41) circle (  1.16);

\path[draw=drawColor,line width= 0.4pt,line join=round,line cap=round,fill=fillColor] (422.09,414.37) circle (  1.16);

\path[draw=drawColor,line width= 0.4pt,line join=round,line cap=round,fill=fillColor] (422.36,414.34) circle (  1.16);

\path[draw=drawColor,line width= 0.4pt,line join=round,line cap=round,fill=fillColor] (422.63,414.33) circle (  1.16);

\path[draw=drawColor,line width= 0.4pt,line join=round,line cap=round,fill=fillColor] (422.91,414.30) circle (  1.16);

\path[draw=drawColor,line width= 0.4pt,line join=round,line cap=round,fill=fillColor] (423.18,414.26) circle (  1.16);

\path[draw=drawColor,line width= 0.4pt,line join=round,line cap=round,fill=fillColor] (423.45,414.21) circle (  1.16);

\path[draw=drawColor,line width= 0.4pt,line join=round,line cap=round,fill=fillColor] (423.72,414.09) circle (  1.16);

\path[draw=drawColor,line width= 0.4pt,line join=round,line cap=round,fill=fillColor] (423.99,414.07) circle (  1.16);

\path[draw=drawColor,line width= 0.4pt,line join=round,line cap=round,fill=fillColor] (424.25,413.97) circle (  1.16);

\path[draw=drawColor,line width= 0.4pt,line join=round,line cap=round,fill=fillColor] (424.52,413.95) circle (  1.16);

\path[draw=drawColor,line width= 0.4pt,line join=round,line cap=round,fill=fillColor] (424.78,413.84) circle (  1.16);

\path[draw=drawColor,line width= 0.4pt,line join=round,line cap=round,fill=fillColor] (425.04,413.79) circle (  1.16);

\path[draw=drawColor,line width= 0.4pt,line join=round,line cap=round,fill=fillColor] (425.31,413.75) circle (  1.16);

\path[draw=drawColor,line width= 0.4pt,line join=round,line cap=round,fill=fillColor] (425.57,413.74) circle (  1.16);

\path[draw=drawColor,line width= 0.4pt,line join=round,line cap=round,fill=fillColor] (425.83,413.68) circle (  1.16);

\path[draw=drawColor,line width= 0.4pt,line join=round,line cap=round,fill=fillColor] (426.08,413.67) circle (  1.16);

\path[draw=drawColor,line width= 0.4pt,line join=round,line cap=round,fill=fillColor] (426.34,413.66) circle (  1.16);

\path[draw=drawColor,line width= 0.4pt,line join=round,line cap=round,fill=fillColor] (426.60,413.60) circle (  1.16);

\path[draw=drawColor,line width= 0.4pt,line join=round,line cap=round,fill=fillColor] (426.85,413.59) circle (  1.16);

\path[draw=drawColor,line width= 0.4pt,line join=round,line cap=round,fill=fillColor] (427.11,413.54) circle (  1.16);

\path[draw=drawColor,line width= 0.4pt,line join=round,line cap=round,fill=fillColor] (427.36,413.50) circle (  1.16);

\path[draw=drawColor,line width= 0.4pt,line join=round,line cap=round,fill=fillColor] (427.61,413.46) circle (  1.16);

\path[draw=drawColor,line width= 0.4pt,line join=round,line cap=round,fill=fillColor] (427.86,413.44) circle (  1.16);

\path[draw=drawColor,line width= 0.4pt,line join=round,line cap=round,fill=fillColor] (428.11,413.43) circle (  1.16);

\path[draw=drawColor,line width= 0.4pt,line join=round,line cap=round,fill=fillColor] (428.36,413.34) circle (  1.16);

\path[draw=drawColor,line width= 0.4pt,line join=round,line cap=round,fill=fillColor] (428.61,413.34) circle (  1.16);

\path[draw=drawColor,line width= 0.4pt,line join=round,line cap=round,fill=fillColor] (428.86,413.30) circle (  1.16);

\path[draw=drawColor,line width= 0.4pt,line join=round,line cap=round,fill=fillColor] (429.10,413.14) circle (  1.16);

\path[draw=drawColor,line width= 0.4pt,line join=round,line cap=round,fill=fillColor] (429.35,412.91) circle (  1.16);

\path[draw=drawColor,line width= 0.4pt,line join=round,line cap=round,fill=fillColor] (429.59,412.90) circle (  1.16);

\path[draw=drawColor,line width= 0.4pt,line join=round,line cap=round,fill=fillColor] (429.83,412.86) circle (  1.16);

\path[draw=drawColor,line width= 0.4pt,line join=round,line cap=round,fill=fillColor] (430.08,412.84) circle (  1.16);

\path[draw=drawColor,line width= 0.4pt,line join=round,line cap=round,fill=fillColor] (430.32,412.80) circle (  1.16);

\path[draw=drawColor,line width= 0.4pt,line join=round,line cap=round,fill=fillColor] (430.56,412.78) circle (  1.16);

\path[draw=drawColor,line width= 0.4pt,line join=round,line cap=round,fill=fillColor] (430.80,412.77) circle (  1.16);

\path[draw=drawColor,line width= 0.4pt,line join=round,line cap=round,fill=fillColor] (431.04,412.68) circle (  1.16);

\path[draw=drawColor,line width= 0.4pt,line join=round,line cap=round,fill=fillColor] (431.27,412.65) circle (  1.16);

\path[draw=drawColor,line width= 0.4pt,line join=round,line cap=round,fill=fillColor] (431.51,412.61) circle (  1.16);

\path[draw=drawColor,line width= 0.4pt,line join=round,line cap=round,fill=fillColor] (431.75,412.49) circle (  1.16);

\path[draw=drawColor,line width= 0.4pt,line join=round,line cap=round,fill=fillColor] (431.98,412.45) circle (  1.16);

\path[draw=drawColor,line width= 0.4pt,line join=round,line cap=round,fill=fillColor] (432.21,412.42) circle (  1.16);

\path[draw=drawColor,line width= 0.4pt,line join=round,line cap=round,fill=fillColor] (432.45,412.42) circle (  1.16);

\path[draw=drawColor,line width= 0.4pt,line join=round,line cap=round,fill=fillColor] (432.68,412.31) circle (  1.16);

\path[draw=drawColor,line width= 0.4pt,line join=round,line cap=round,fill=fillColor] (432.91,412.20) circle (  1.16);

\path[draw=drawColor,line width= 0.4pt,line join=round,line cap=round,fill=fillColor] (433.14,412.20) circle (  1.16);

\path[draw=drawColor,line width= 0.4pt,line join=round,line cap=round,fill=fillColor] (433.37,412.19) circle (  1.16);

\path[draw=drawColor,line width= 0.4pt,line join=round,line cap=round,fill=fillColor] (433.60,412.07) circle (  1.16);

\path[draw=drawColor,line width= 0.4pt,line join=round,line cap=round,fill=fillColor] (433.83,412.02) circle (  1.16);

\path[draw=drawColor,line width= 0.4pt,line join=round,line cap=round,fill=fillColor] (434.06,412.01) circle (  1.16);

\path[draw=drawColor,line width= 0.4pt,line join=round,line cap=round,fill=fillColor] (434.28,411.95) circle (  1.16);

\path[draw=drawColor,line width= 0.4pt,line join=round,line cap=round,fill=fillColor] (434.51,411.90) circle (  1.16);

\path[draw=drawColor,line width= 0.4pt,line join=round,line cap=round,fill=fillColor] (434.73,411.87) circle (  1.16);

\path[draw=drawColor,line width= 0.4pt,line join=round,line cap=round,fill=fillColor] (434.96,411.79) circle (  1.16);

\path[draw=drawColor,line width= 0.4pt,line join=round,line cap=round,fill=fillColor] (435.18,411.59) circle (  1.16);

\path[draw=drawColor,line width= 0.4pt,line join=round,line cap=round,fill=fillColor] (435.40,411.46) circle (  1.16);

\path[draw=drawColor,line width= 0.4pt,line join=round,line cap=round,fill=fillColor] (435.62,411.35) circle (  1.16);

\path[draw=drawColor,line width= 0.4pt,line join=round,line cap=round,fill=fillColor] (435.85,411.29) circle (  1.16);

\path[draw=drawColor,line width= 0.4pt,line join=round,line cap=round,fill=fillColor] (436.07,411.28) circle (  1.16);

\path[draw=drawColor,line width= 0.4pt,line join=round,line cap=round,fill=fillColor] (436.29,411.23) circle (  1.16);

\path[draw=drawColor,line width= 0.4pt,line join=round,line cap=round,fill=fillColor] (436.50,411.18) circle (  1.16);

\path[draw=drawColor,line width= 0.4pt,line join=round,line cap=round,fill=fillColor] (436.72,411.10) circle (  1.16);

\path[draw=drawColor,line width= 0.4pt,line join=round,line cap=round,fill=fillColor] (436.94,411.08) circle (  1.16);

\path[draw=drawColor,line width= 0.4pt,line join=round,line cap=round,fill=fillColor] (437.16,411.00) circle (  1.16);

\path[draw=drawColor,line width= 0.4pt,line join=round,line cap=round,fill=fillColor] (437.37,410.98) circle (  1.16);

\path[draw=drawColor,line width= 0.4pt,line join=round,line cap=round,fill=fillColor] (437.59,410.98) circle (  1.16);

\path[draw=drawColor,line width= 0.4pt,line join=round,line cap=round,fill=fillColor] (437.80,410.98) circle (  1.16);

\path[draw=drawColor,line width= 0.4pt,line join=round,line cap=round,fill=fillColor] (438.02,410.94) circle (  1.16);

\path[draw=drawColor,line width= 0.4pt,line join=round,line cap=round,fill=fillColor] (438.23,410.94) circle (  1.16);

\path[draw=drawColor,line width= 0.4pt,line join=round,line cap=round,fill=fillColor] (438.44,410.93) circle (  1.16);

\path[draw=drawColor,line width= 0.4pt,line join=round,line cap=round,fill=fillColor] (438.65,410.91) circle (  1.16);

\path[draw=drawColor,line width= 0.4pt,line join=round,line cap=round,fill=fillColor] (438.87,410.88) circle (  1.16);

\path[draw=drawColor,line width= 0.4pt,line join=round,line cap=round,fill=fillColor] (439.08,410.87) circle (  1.16);

\path[draw=drawColor,line width= 0.4pt,line join=round,line cap=round,fill=fillColor] (439.29,410.86) circle (  1.16);

\path[draw=drawColor,line width= 0.4pt,line join=round,line cap=round,fill=fillColor] (439.49,410.71) circle (  1.16);

\path[draw=drawColor,line width= 0.4pt,line join=round,line cap=round,fill=fillColor] (439.70,410.70) circle (  1.16);

\path[draw=drawColor,line width= 0.4pt,line join=round,line cap=round,fill=fillColor] (439.91,410.69) circle (  1.16);

\path[draw=drawColor,line width= 0.4pt,line join=round,line cap=round,fill=fillColor] (440.12,410.69) circle (  1.16);

\path[draw=drawColor,line width= 0.4pt,line join=round,line cap=round,fill=fillColor] (440.33,410.67) circle (  1.16);

\path[draw=drawColor,line width= 0.4pt,line join=round,line cap=round,fill=fillColor] (440.53,410.64) circle (  1.16);

\path[draw=drawColor,line width= 0.4pt,line join=round,line cap=round,fill=fillColor] (440.74,410.61) circle (  1.16);

\path[draw=drawColor,line width= 0.4pt,line join=round,line cap=round,fill=fillColor] (440.94,410.52) circle (  1.16);

\path[draw=drawColor,line width= 0.4pt,line join=round,line cap=round,fill=fillColor] (441.15,410.50) circle (  1.16);

\path[draw=drawColor,line width= 0.4pt,line join=round,line cap=round,fill=fillColor] (441.35,410.46) circle (  1.16);

\path[draw=drawColor,line width= 0.4pt,line join=round,line cap=round,fill=fillColor] (441.55,410.42) circle (  1.16);

\path[draw=drawColor,line width= 0.4pt,line join=round,line cap=round,fill=fillColor] (441.75,410.33) circle (  1.16);

\path[draw=drawColor,line width= 0.4pt,line join=round,line cap=round,fill=fillColor] (441.96,410.32) circle (  1.16);

\path[draw=drawColor,line width= 0.4pt,line join=round,line cap=round,fill=fillColor] (442.16,410.27) circle (  1.16);

\path[draw=drawColor,line width= 0.4pt,line join=round,line cap=round,fill=fillColor] (442.36,410.27) circle (  1.16);

\path[draw=drawColor,line width= 0.4pt,line join=round,line cap=round,fill=fillColor] (442.56,410.27) circle (  1.16);

\path[draw=drawColor,line width= 0.4pt,line join=round,line cap=round,fill=fillColor] (442.76,410.23) circle (  1.16);

\path[draw=drawColor,line width= 0.4pt,line join=round,line cap=round,fill=fillColor] (442.96,410.22) circle (  1.16);

\path[draw=drawColor,line width= 0.4pt,line join=round,line cap=round,fill=fillColor] (443.15,410.21) circle (  1.16);

\path[draw=drawColor,line width= 0.4pt,line join=round,line cap=round,fill=fillColor] (443.35,410.16) circle (  1.16);

\path[draw=drawColor,line width= 0.4pt,line join=round,line cap=round,fill=fillColor] (443.55,410.08) circle (  1.16);

\path[draw=drawColor,line width= 0.4pt,line join=round,line cap=round,fill=fillColor] (443.74,410.08) circle (  1.16);

\path[draw=drawColor,line width= 0.4pt,line join=round,line cap=round,fill=fillColor] (443.94,410.06) circle (  1.16);

\path[draw=drawColor,line width= 0.4pt,line join=round,line cap=round,fill=fillColor] (444.14,410.01) circle (  1.16);

\path[draw=drawColor,line width= 0.4pt,line join=round,line cap=round,fill=fillColor] (444.33,410.01) circle (  1.16);

\path[draw=drawColor,line width= 0.4pt,line join=round,line cap=round,fill=fillColor] (444.53,410.01) circle (  1.16);

\path[draw=drawColor,line width= 0.4pt,line join=round,line cap=round,fill=fillColor] (444.72,409.97) circle (  1.16);

\path[draw=drawColor,line width= 0.4pt,line join=round,line cap=round,fill=fillColor] (444.91,409.90) circle (  1.16);

\path[draw=drawColor,line width= 0.4pt,line join=round,line cap=round,fill=fillColor] (445.10,409.90) circle (  1.16);

\path[draw=drawColor,line width= 0.4pt,line join=round,line cap=round,fill=fillColor] (445.30,409.89) circle (  1.16);

\path[draw=drawColor,line width= 0.4pt,line join=round,line cap=round,fill=fillColor] (445.49,409.86) circle (  1.16);

\path[draw=drawColor,line width= 0.4pt,line join=round,line cap=round,fill=fillColor] (445.68,409.77) circle (  1.16);

\path[draw=drawColor,line width= 0.4pt,line join=round,line cap=round,fill=fillColor] (445.87,409.77) circle (  1.16);

\path[draw=drawColor,line width= 0.4pt,line join=round,line cap=round,fill=fillColor] (446.06,409.76) circle (  1.16);

\path[draw=drawColor,line width= 0.4pt,line join=round,line cap=round,fill=fillColor] (446.25,409.74) circle (  1.16);

\path[draw=drawColor,line width= 0.4pt,line join=round,line cap=round,fill=fillColor] (446.44,409.71) circle (  1.16);

\path[draw=drawColor,line width= 0.4pt,line join=round,line cap=round,fill=fillColor] (446.63,409.64) circle (  1.16);

\path[draw=drawColor,line width= 0.4pt,line join=round,line cap=round,fill=fillColor] (446.82,409.49) circle (  1.16);

\path[draw=drawColor,line width= 0.4pt,line join=round,line cap=round,fill=fillColor] (447.00,409.46) circle (  1.16);

\path[draw=drawColor,line width= 0.4pt,line join=round,line cap=round,fill=fillColor] (447.19,409.39) circle (  1.16);

\path[draw=drawColor,line width= 0.4pt,line join=round,line cap=round,fill=fillColor] (447.38,409.32) circle (  1.16);

\path[draw=drawColor,line width= 0.4pt,line join=round,line cap=round,fill=fillColor] (447.56,409.31) circle (  1.16);

\path[draw=drawColor,line width= 0.4pt,line join=round,line cap=round,fill=fillColor] (447.75,409.25) circle (  1.16);

\path[draw=drawColor,line width= 0.4pt,line join=round,line cap=round,fill=fillColor] (447.93,409.24) circle (  1.16);

\path[draw=drawColor,line width= 0.4pt,line join=round,line cap=round,fill=fillColor] (448.12,409.17) circle (  1.16);

\path[draw=drawColor,line width= 0.4pt,line join=round,line cap=round,fill=fillColor] (448.30,409.14) circle (  1.16);

\path[draw=drawColor,line width= 0.4pt,line join=round,line cap=round,fill=fillColor] (448.49,409.03) circle (  1.16);

\path[draw=drawColor,line width= 0.4pt,line join=round,line cap=round,fill=fillColor] (448.67,408.98) circle (  1.16);

\path[draw=drawColor,line width= 0.4pt,line join=round,line cap=round,fill=fillColor] (448.85,408.90) circle (  1.16);

\path[draw=drawColor,line width= 0.4pt,line join=round,line cap=round,fill=fillColor] (449.03,408.79) circle (  1.16);

\path[draw=drawColor,line width= 0.4pt,line join=round,line cap=round,fill=fillColor] (449.22,408.75) circle (  1.16);

\path[draw=drawColor,line width= 0.4pt,line join=round,line cap=round,fill=fillColor] (449.40,408.64) circle (  1.16);

\path[draw=drawColor,line width= 0.4pt,line join=round,line cap=round,fill=fillColor] (449.58,408.52) circle (  1.16);

\path[draw=drawColor,line width= 0.4pt,line join=round,line cap=round,fill=fillColor] (449.76,408.45) circle (  1.16);

\path[draw=drawColor,line width= 0.4pt,line join=round,line cap=round,fill=fillColor] (449.94,408.37) circle (  1.16);

\path[draw=drawColor,line width= 0.4pt,line join=round,line cap=round,fill=fillColor] (450.12,408.35) circle (  1.16);

\path[draw=drawColor,line width= 0.4pt,line join=round,line cap=round,fill=fillColor] (450.30,408.32) circle (  1.16);

\path[draw=drawColor,line width= 0.4pt,line join=round,line cap=round,fill=fillColor] (450.48,408.29) circle (  1.16);

\path[draw=drawColor,line width= 0.4pt,line join=round,line cap=round,fill=fillColor] (450.65,408.27) circle (  1.16);

\path[draw=drawColor,line width= 0.4pt,line join=round,line cap=round,fill=fillColor] (450.83,408.27) circle (  1.16);

\path[draw=drawColor,line width= 0.4pt,line join=round,line cap=round,fill=fillColor] (451.01,408.23) circle (  1.16);

\path[draw=drawColor,line width= 0.4pt,line join=round,line cap=round,fill=fillColor] (451.19,408.18) circle (  1.16);

\path[draw=drawColor,line width= 0.4pt,line join=round,line cap=round,fill=fillColor] (451.36,408.15) circle (  1.16);

\path[draw=drawColor,line width= 0.4pt,line join=round,line cap=round,fill=fillColor] (451.54,408.08) circle (  1.16);

\path[draw=drawColor,line width= 0.4pt,line join=round,line cap=round,fill=fillColor] (451.71,408.06) circle (  1.16);

\path[draw=drawColor,line width= 0.4pt,line join=round,line cap=round,fill=fillColor] (451.89,407.96) circle (  1.16);

\path[draw=drawColor,line width= 0.4pt,line join=round,line cap=round,fill=fillColor] (452.06,407.92) circle (  1.16);

\path[draw=drawColor,line width= 0.4pt,line join=round,line cap=round,fill=fillColor] (452.24,407.89) circle (  1.16);

\path[draw=drawColor,line width= 0.4pt,line join=round,line cap=round,fill=fillColor] (452.41,407.86) circle (  1.16);

\path[draw=drawColor,line width= 0.4pt,line join=round,line cap=round,fill=fillColor] (452.59,407.76) circle (  1.16);

\path[draw=drawColor,line width= 0.4pt,line join=round,line cap=round,fill=fillColor] (452.76,407.75) circle (  1.16);

\path[draw=drawColor,line width= 0.4pt,line join=round,line cap=round,fill=fillColor] (452.93,407.73) circle (  1.16);

\path[draw=drawColor,line width= 0.4pt,line join=round,line cap=round,fill=fillColor] (453.10,407.62) circle (  1.16);

\path[draw=drawColor,line width= 0.4pt,line join=round,line cap=round,fill=fillColor] (453.28,407.57) circle (  1.16);

\path[draw=drawColor,line width= 0.4pt,line join=round,line cap=round,fill=fillColor] (453.45,407.56) circle (  1.16);

\path[draw=drawColor,line width= 0.4pt,line join=round,line cap=round,fill=fillColor] (453.62,407.52) circle (  1.16);

\path[draw=drawColor,line width= 0.4pt,line join=round,line cap=round,fill=fillColor] (453.79,407.38) circle (  1.16);

\path[draw=drawColor,line width= 0.4pt,line join=round,line cap=round,fill=fillColor] (453.96,407.35) circle (  1.16);

\path[draw=drawColor,line width= 0.4pt,line join=round,line cap=round,fill=fillColor] (454.13,407.33) circle (  1.16);

\path[draw=drawColor,line width= 0.4pt,line join=round,line cap=round,fill=fillColor] (454.30,407.31) circle (  1.16);

\path[draw=drawColor,line width= 0.4pt,line join=round,line cap=round,fill=fillColor] (454.47,407.31) circle (  1.16);

\path[draw=drawColor,line width= 0.4pt,line join=round,line cap=round,fill=fillColor] (454.64,407.31) circle (  1.16);

\path[draw=drawColor,line width= 0.4pt,line join=round,line cap=round,fill=fillColor] (454.81,407.27) circle (  1.16);

\path[draw=drawColor,line width= 0.4pt,line join=round,line cap=round,fill=fillColor] (454.97,407.25) circle (  1.16);

\path[draw=drawColor,line width= 0.4pt,line join=round,line cap=round,fill=fillColor] (455.14,407.19) circle (  1.16);

\path[draw=drawColor,line width= 0.4pt,line join=round,line cap=round,fill=fillColor] (455.31,407.11) circle (  1.16);

\path[draw=drawColor,line width= 0.4pt,line join=round,line cap=round,fill=fillColor] (455.48,406.98) circle (  1.16);

\path[draw=drawColor,line width= 0.4pt,line join=round,line cap=round,fill=fillColor] (455.64,406.94) circle (  1.16);

\path[draw=drawColor,line width= 0.4pt,line join=round,line cap=round,fill=fillColor] (455.81,406.94) circle (  1.16);

\path[draw=drawColor,line width= 0.4pt,line join=round,line cap=round,fill=fillColor] (455.97,406.87) circle (  1.16);

\path[draw=drawColor,line width= 0.4pt,line join=round,line cap=round,fill=fillColor] (456.14,406.83) circle (  1.16);

\path[draw=drawColor,line width= 0.4pt,line join=round,line cap=round,fill=fillColor] (456.31,406.81) circle (  1.16);

\path[draw=drawColor,line width= 0.4pt,line join=round,line cap=round,fill=fillColor] (456.47,406.80) circle (  1.16);

\path[draw=drawColor,line width= 0.4pt,line join=round,line cap=round,fill=fillColor] (456.63,406.64) circle (  1.16);

\path[draw=drawColor,line width= 0.4pt,line join=round,line cap=round,fill=fillColor] (456.80,406.59) circle (  1.16);

\path[draw=drawColor,line width= 0.4pt,line join=round,line cap=round,fill=fillColor] (456.96,406.55) circle (  1.16);

\path[draw=drawColor,line width= 0.4pt,line join=round,line cap=round,fill=fillColor] (457.13,406.52) circle (  1.16);

\path[draw=drawColor,line width= 0.4pt,line join=round,line cap=round,fill=fillColor] (457.29,406.49) circle (  1.16);

\path[draw=drawColor,line width= 0.4pt,line join=round,line cap=round,fill=fillColor] (457.45,406.43) circle (  1.16);

\path[draw=drawColor,line width= 0.4pt,line join=round,line cap=round,fill=fillColor] (457.61,406.40) circle (  1.16);

\path[draw=drawColor,line width= 0.4pt,line join=round,line cap=round,fill=fillColor] (457.78,406.33) circle (  1.16);

\path[draw=drawColor,line width= 0.4pt,line join=round,line cap=round,fill=fillColor] (457.94,406.32) circle (  1.16);

\path[draw=drawColor,line width= 0.4pt,line join=round,line cap=round,fill=fillColor] (458.10,406.28) circle (  1.16);

\path[draw=drawColor,line width= 0.4pt,line join=round,line cap=round,fill=fillColor] (458.26,406.27) circle (  1.16);

\path[draw=drawColor,line width= 0.4pt,line join=round,line cap=round,fill=fillColor] (458.42,406.22) circle (  1.16);

\path[draw=drawColor,line width= 0.4pt,line join=round,line cap=round,fill=fillColor] (458.58,406.14) circle (  1.16);

\path[draw=drawColor,line width= 0.4pt,line join=round,line cap=round,fill=fillColor] (458.74,406.13) circle (  1.16);

\path[draw=drawColor,line width= 0.4pt,line join=round,line cap=round,fill=fillColor] (458.90,406.05) circle (  1.16);

\path[draw=drawColor,line width= 0.4pt,line join=round,line cap=round,fill=fillColor] (459.06,406.04) circle (  1.16);

\path[draw=drawColor,line width= 0.4pt,line join=round,line cap=round,fill=fillColor] (459.22,405.96) circle (  1.16);

\path[draw=drawColor,line width= 0.4pt,line join=round,line cap=round,fill=fillColor] (459.38,405.92) circle (  1.16);

\path[draw=drawColor,line width= 0.4pt,line join=round,line cap=round,fill=fillColor] (459.53,405.89) circle (  1.16);

\path[draw=drawColor,line width= 0.4pt,line join=round,line cap=round,fill=fillColor] (459.69,405.88) circle (  1.16);

\path[draw=drawColor,line width= 0.4pt,line join=round,line cap=round,fill=fillColor] (459.85,405.85) circle (  1.16);

\path[draw=drawColor,line width= 0.4pt,line join=round,line cap=round,fill=fillColor] (460.01,405.84) circle (  1.16);

\path[draw=drawColor,line width= 0.4pt,line join=round,line cap=round,fill=fillColor] (460.16,405.75) circle (  1.16);

\path[draw=drawColor,line width= 0.4pt,line join=round,line cap=round,fill=fillColor] (460.32,405.73) circle (  1.16);

\path[draw=drawColor,line width= 0.4pt,line join=round,line cap=round,fill=fillColor] (460.48,405.72) circle (  1.16);

\path[draw=drawColor,line width= 0.4pt,line join=round,line cap=round,fill=fillColor] (460.63,405.72) circle (  1.16);

\path[draw=drawColor,line width= 0.4pt,line join=round,line cap=round,fill=fillColor] (460.79,405.70) circle (  1.16);

\path[draw=drawColor,line width= 0.4pt,line join=round,line cap=round,fill=fillColor] (460.94,405.66) circle (  1.16);

\path[draw=drawColor,line width= 0.4pt,line join=round,line cap=round,fill=fillColor] (461.10,405.66) circle (  1.16);

\path[draw=drawColor,line width= 0.4pt,line join=round,line cap=round,fill=fillColor] (461.25,405.64) circle (  1.16);

\path[draw=drawColor,line width= 0.4pt,line join=round,line cap=round,fill=fillColor] (461.41,405.61) circle (  1.16);

\path[draw=drawColor,line width= 0.4pt,line join=round,line cap=round,fill=fillColor] (461.56,405.57) circle (  1.16);

\path[draw=drawColor,line width= 0.4pt,line join=round,line cap=round,fill=fillColor] (461.72,405.43) circle (  1.16);

\path[draw=drawColor,line width= 0.4pt,line join=round,line cap=round,fill=fillColor] (461.87,405.42) circle (  1.16);

\path[draw=drawColor,line width= 0.4pt,line join=round,line cap=round,fill=fillColor] (462.02,405.41) circle (  1.16);

\path[draw=drawColor,line width= 0.4pt,line join=round,line cap=round,fill=fillColor] (462.18,405.11) circle (  1.16);

\path[draw=drawColor,line width= 0.4pt,line join=round,line cap=round,fill=fillColor] (462.33,404.86) circle (  1.16);

\path[draw=drawColor,line width= 0.4pt,line join=round,line cap=round,fill=fillColor] (462.48,404.85) circle (  1.16);

\path[draw=drawColor,line width= 0.4pt,line join=round,line cap=round,fill=fillColor] (462.63,404.80) circle (  1.16);

\path[draw=drawColor,line width= 0.4pt,line join=round,line cap=round,fill=fillColor] (462.78,404.77) circle (  1.16);

\path[draw=drawColor,line width= 0.4pt,line join=round,line cap=round,fill=fillColor] (462.94,404.66) circle (  1.16);

\path[draw=drawColor,line width= 0.4pt,line join=round,line cap=round,fill=fillColor] (463.09,404.53) circle (  1.16);

\path[draw=drawColor,line width= 0.4pt,line join=round,line cap=round,fill=fillColor] (463.24,404.44) circle (  1.16);

\path[draw=drawColor,line width= 0.4pt,line join=round,line cap=round,fill=fillColor] (463.39,404.38) circle (  1.16);

\path[draw=drawColor,line width= 0.4pt,line join=round,line cap=round,fill=fillColor] (463.54,404.26) circle (  1.16);

\path[draw=drawColor,line width= 0.4pt,line join=round,line cap=round,fill=fillColor] (463.69,404.24) circle (  1.16);

\path[draw=drawColor,line width= 0.4pt,line join=round,line cap=round,fill=fillColor] (463.84,404.20) circle (  1.16);

\path[draw=drawColor,line width= 0.4pt,line join=round,line cap=round,fill=fillColor] (463.99,404.18) circle (  1.16);

\path[draw=drawColor,line width= 0.4pt,line join=round,line cap=round,fill=fillColor] (464.14,404.13) circle (  1.16);

\path[draw=drawColor,line width= 0.4pt,line join=round,line cap=round,fill=fillColor] (464.29,404.12) circle (  1.16);

\path[draw=drawColor,line width= 0.4pt,line join=round,line cap=round,fill=fillColor] (464.44,404.12) circle (  1.16);

\path[draw=drawColor,line width= 0.4pt,line join=round,line cap=round,fill=fillColor] (464.58,404.09) circle (  1.16);

\path[draw=drawColor,line width= 0.4pt,line join=round,line cap=round,fill=fillColor] (464.73,404.05) circle (  1.16);

\path[draw=drawColor,line width= 0.4pt,line join=round,line cap=round,fill=fillColor] (464.88,404.02) circle (  1.16);

\path[draw=drawColor,line width= 0.4pt,line join=round,line cap=round,fill=fillColor] (465.03,404.01) circle (  1.16);

\path[draw=drawColor,line width= 0.4pt,line join=round,line cap=round,fill=fillColor] (465.17,403.95) circle (  1.16);

\path[draw=drawColor,line width= 0.4pt,line join=round,line cap=round,fill=fillColor] (465.32,403.52) circle (  1.16);

\path[draw=drawColor,line width= 0.4pt,line join=round,line cap=round,fill=fillColor] (465.47,403.49) circle (  1.16);

\path[draw=drawColor,line width= 0.4pt,line join=round,line cap=round,fill=fillColor] (465.61,403.39) circle (  1.16);

\path[draw=drawColor,line width= 0.4pt,line join=round,line cap=round,fill=fillColor] (465.76,403.39) circle (  1.16);

\path[draw=drawColor,line width= 0.4pt,line join=round,line cap=round,fill=fillColor] (465.91,403.35) circle (  1.16);

\path[draw=drawColor,line width= 0.4pt,line join=round,line cap=round,fill=fillColor] (466.05,403.22) circle (  1.16);

\path[draw=drawColor,line width= 0.4pt,line join=round,line cap=round,fill=fillColor] (466.20,403.15) circle (  1.16);

\path[draw=drawColor,line width= 0.4pt,line join=round,line cap=round,fill=fillColor] (466.34,403.12) circle (  1.16);

\path[draw=drawColor,line width= 0.4pt,line join=round,line cap=round,fill=fillColor] (466.49,403.11) circle (  1.16);

\path[draw=drawColor,line width= 0.4pt,line join=round,line cap=round,fill=fillColor] (466.63,403.10) circle (  1.16);

\path[draw=drawColor,line width= 0.4pt,line join=round,line cap=round,fill=fillColor] (466.78,402.84) circle (  1.16);

\path[draw=drawColor,line width= 0.4pt,line join=round,line cap=round,fill=fillColor] (466.92,402.73) circle (  1.16);

\path[draw=drawColor,line width= 0.4pt,line join=round,line cap=round,fill=fillColor] (467.06,402.72) circle (  1.16);

\path[draw=drawColor,line width= 0.4pt,line join=round,line cap=round,fill=fillColor] (467.21,402.66) circle (  1.16);

\path[draw=drawColor,line width= 0.4pt,line join=round,line cap=round,fill=fillColor] (467.35,402.65) circle (  1.16);

\path[draw=drawColor,line width= 0.4pt,line join=round,line cap=round,fill=fillColor] (467.49,402.59) circle (  1.16);

\path[draw=drawColor,line width= 0.4pt,line join=round,line cap=round,fill=fillColor] (467.64,402.47) circle (  1.16);

\path[draw=drawColor,line width= 0.4pt,line join=round,line cap=round,fill=fillColor] (467.78,402.46) circle (  1.16);

\path[draw=drawColor,line width= 0.4pt,line join=round,line cap=round,fill=fillColor] (467.92,402.40) circle (  1.16);

\path[draw=drawColor,line width= 0.4pt,line join=round,line cap=round,fill=fillColor] (468.06,402.19) circle (  1.16);

\path[draw=drawColor,line width= 0.4pt,line join=round,line cap=round,fill=fillColor] (468.21,402.13) circle (  1.16);

\path[draw=drawColor,line width= 0.4pt,line join=round,line cap=round,fill=fillColor] (468.35,401.87) circle (  1.16);

\path[draw=drawColor,line width= 0.4pt,line join=round,line cap=round,fill=fillColor] (468.49,401.80) circle (  1.16);

\path[draw=drawColor,line width= 0.4pt,line join=round,line cap=round,fill=fillColor] (468.63,401.58) circle (  1.16);

\path[draw=drawColor,line width= 0.4pt,line join=round,line cap=round,fill=fillColor] (468.77,401.56) circle (  1.16);

\path[draw=drawColor,line width= 0.4pt,line join=round,line cap=round,fill=fillColor] (468.91,401.54) circle (  1.16);

\path[draw=drawColor,line width= 0.4pt,line join=round,line cap=round,fill=fillColor] (469.05,401.43) circle (  1.16);

\path[draw=drawColor,line width= 0.4pt,line join=round,line cap=round,fill=fillColor] (469.19,401.22) circle (  1.16);

\path[draw=drawColor,line width= 0.4pt,line join=round,line cap=round,fill=fillColor] (469.33,401.18) circle (  1.16);

\path[draw=drawColor,line width= 0.4pt,line join=round,line cap=round,fill=fillColor] (469.47,401.16) circle (  1.16);

\path[draw=drawColor,line width= 0.4pt,line join=round,line cap=round,fill=fillColor] (469.61,401.09) circle (  1.16);

\path[draw=drawColor,line width= 0.4pt,line join=round,line cap=round,fill=fillColor] (469.75,400.97) circle (  1.16);

\path[draw=drawColor,line width= 0.4pt,line join=round,line cap=round,fill=fillColor] (469.89,400.92) circle (  1.16);

\path[draw=drawColor,line width= 0.4pt,line join=round,line cap=round,fill=fillColor] (470.03,400.78) circle (  1.16);

\path[draw=drawColor,line width= 0.4pt,line join=round,line cap=round,fill=fillColor] (470.17,400.68) circle (  1.16);

\path[draw=drawColor,line width= 0.4pt,line join=round,line cap=round,fill=fillColor] (470.30,400.67) circle (  1.16);

\path[draw=drawColor,line width= 0.4pt,line join=round,line cap=round,fill=fillColor] (470.44,400.59) circle (  1.16);

\path[draw=drawColor,line width= 0.4pt,line join=round,line cap=round,fill=fillColor] (470.58,400.51) circle (  1.16);

\path[draw=drawColor,line width= 0.4pt,line join=round,line cap=round,fill=fillColor] (470.72,400.39) circle (  1.16);

\path[draw=drawColor,line width= 0.4pt,line join=round,line cap=round,fill=fillColor] (470.85,400.25) circle (  1.16);

\path[draw=drawColor,line width= 0.4pt,line join=round,line cap=round,fill=fillColor] (470.99,400.16) circle (  1.16);

\path[draw=drawColor,line width= 0.4pt,line join=round,line cap=round,fill=fillColor] (471.13,400.08) circle (  1.16);

\path[draw=drawColor,line width= 0.4pt,line join=round,line cap=round,fill=fillColor] (471.27,399.56) circle (  1.16);

\path[draw=drawColor,line width= 0.4pt,line join=round,line cap=round,fill=fillColor] (471.40,399.23) circle (  1.16);

\path[draw=drawColor,line width= 0.4pt,line join=round,line cap=round,fill=fillColor] (471.54,398.77) circle (  1.16);

\path[draw=drawColor,line width= 0.4pt,line join=round,line cap=round,fill=fillColor] (471.67,398.69) circle (  1.16);

\path[draw=drawColor,line width= 0.4pt,line join=round,line cap=round,fill=fillColor] (471.81,398.52) circle (  1.16);

\path[draw=drawColor,line width= 0.4pt,line join=round,line cap=round,fill=fillColor] (471.94,398.52) circle (  1.16);

\path[draw=drawColor,line width= 0.4pt,line join=round,line cap=round,fill=fillColor] (472.08,398.45) circle (  1.16);

\path[draw=drawColor,line width= 0.4pt,line join=round,line cap=round,fill=fillColor] (472.22,398.16) circle (  1.16);

\path[draw=drawColor,line width= 0.4pt,line join=round,line cap=round,fill=fillColor] (472.35,397.87) circle (  1.16);

\path[draw=drawColor,line width= 0.4pt,line join=round,line cap=round,fill=fillColor] (472.48,397.79) circle (  1.16);

\path[draw=drawColor,line width= 0.4pt,line join=round,line cap=round,fill=fillColor] (472.62,397.67) circle (  1.16);

\path[draw=drawColor,line width= 0.4pt,line join=round,line cap=round,fill=fillColor] (472.75,397.61) circle (  1.16);

\path[draw=drawColor,line width= 0.4pt,line join=round,line cap=round,fill=fillColor] (472.89,397.32) circle (  1.16);

\path[draw=drawColor,line width= 0.4pt,line join=round,line cap=round,fill=fillColor] (473.02,397.25) circle (  1.16);

\path[draw=drawColor,line width= 0.4pt,line join=round,line cap=round,fill=fillColor] (473.15,397.04) circle (  1.16);

\path[draw=drawColor,line width= 0.4pt,line join=round,line cap=round,fill=fillColor] (473.29,396.90) circle (  1.16);

\path[draw=drawColor,line width= 0.4pt,line join=round,line cap=round,fill=fillColor] (473.42,396.51) circle (  1.16);

\path[draw=drawColor,line width= 0.4pt,line join=round,line cap=round,fill=fillColor] (473.55,396.36) circle (  1.16);

\path[draw=drawColor,line width= 0.4pt,line join=round,line cap=round,fill=fillColor] (473.69,396.15) circle (  1.16);

\path[draw=drawColor,line width= 0.4pt,line join=round,line cap=round,fill=fillColor] (473.82,395.88) circle (  1.16);

\path[draw=drawColor,line width= 0.4pt,line join=round,line cap=round,fill=fillColor] (473.95,395.85) circle (  1.16);

\path[draw=drawColor,line width= 0.4pt,line join=round,line cap=round,fill=fillColor] (474.08,395.49) circle (  1.16);

\path[draw=drawColor,line width= 0.4pt,line join=round,line cap=round,fill=fillColor] (474.22,395.17) circle (  1.16);

\path[draw=drawColor,line width= 0.4pt,line join=round,line cap=round,fill=fillColor] (474.35,393.26) circle (  1.16);

\path[draw=drawColor,line width= 0.4pt,line join=round,line cap=round,fill=fillColor] (474.48,393.26) circle (  1.16);

\path[draw=drawColor,line width= 0.4pt,line join=round,line cap=round,fill=fillColor] (474.61,392.53) circle (  1.16);

\path[draw=drawColor,line width= 0.4pt,line join=round,line cap=round,fill=fillColor] (474.74,392.22) circle (  1.16);

\path[draw=drawColor,line width= 0.4pt,line join=round,line cap=round,fill=fillColor] (474.87,392.05) circle (  1.16);

\path[draw=drawColor,line width= 0.4pt,line join=round,line cap=round,fill=fillColor] (475.00,391.72) circle (  1.16);

\path[draw=drawColor,line width= 0.4pt,line join=round,line cap=round,fill=fillColor] (475.13,391.60) circle (  1.16);

\path[draw=drawColor,line width= 0.4pt,line join=round,line cap=round,fill=fillColor] (475.26,391.33) circle (  1.16);

\path[draw=drawColor,line width= 0.4pt,line join=round,line cap=round,fill=fillColor] (475.39,390.89) circle (  1.16);

\path[draw=drawColor,line width= 0.4pt,line join=round,line cap=round,fill=fillColor] (475.52,390.53) circle (  1.16);

\path[draw=drawColor,line width= 0.4pt,line join=round,line cap=round,fill=fillColor] (475.65,390.47) circle (  1.16);

\path[draw=drawColor,line width= 0.4pt,line join=round,line cap=round,fill=fillColor] (475.78,390.41) circle (  1.16);

\path[draw=drawColor,line width= 0.4pt,line join=round,line cap=round,fill=fillColor] (475.91,389.90) circle (  1.16);

\path[draw=drawColor,line width= 0.4pt,line join=round,line cap=round,fill=fillColor] (476.04,388.82) circle (  1.16);

\path[draw=drawColor,line width= 0.4pt,line join=round,line cap=round,fill=fillColor] (476.17,388.39) circle (  1.16);

\path[draw=drawColor,line width= 0.4pt,line join=round,line cap=round,fill=fillColor] (476.30,387.87) circle (  1.16);

\path[draw=drawColor,line width= 0.4pt,line join=round,line cap=round,fill=fillColor] (476.43,385.68) circle (  1.16);

\path[draw=drawColor,line width= 0.4pt,line join=round,line cap=round,fill=fillColor] (476.56,384.53) circle (  1.16);

\path[draw=drawColor,line width= 0.4pt,line join=round,line cap=round,fill=fillColor] (476.68,384.50) circle (  1.16);

\path[draw=drawColor,line width= 0.4pt,line join=round,line cap=round,fill=fillColor] (476.81,384.33) circle (  1.16);

\path[draw=drawColor,line width= 0.4pt,line join=round,line cap=round,fill=fillColor] (476.94,382.91) circle (  1.16);

\path[draw=drawColor,line width= 0.4pt,line join=round,line cap=round,fill=fillColor] (477.07,380.04) circle (  1.16);

\path[draw=drawColor,line width= 0.4pt,line join=round,line cap=round,fill=fillColor] (477.19,372.42) circle (  1.16);

\path[draw=drawColor,line width= 0.4pt,line join=round,line cap=round,fill=fillColor] (477.32,372.42) circle (  1.16);

\path[draw=drawColor,line width= 0.4pt,line join=round,line cap=round,fill=fillColor] (477.45,372.42) circle (  1.16);

\path[draw=drawColor,line width= 0.4pt,line join=round,line cap=round,fill=fillColor] (477.57,372.42) circle (  1.16);

\path[draw=drawColor,line width= 0.4pt,line join=round,line cap=round,fill=fillColor] (477.70,372.42) circle (  1.16);

\path[draw=drawColor,line width= 0.4pt,line join=round,line cap=round,fill=fillColor] (477.83,372.42) circle (  1.16);

\path[draw=drawColor,line width= 0.4pt,line join=round,line cap=round,fill=fillColor] (477.95,372.42) circle (  1.16);

\path[draw=drawColor,line width= 0.4pt,line join=round,line cap=round,fill=fillColor] (478.08,372.42) circle (  1.16);

\path[draw=drawColor,line width= 0.4pt,line join=round,line cap=round,fill=fillColor] (478.21,372.42) circle (  1.16);
\definecolor[named]{drawColor}{rgb}{1.00,0.50,0.00}
\definecolor[named]{fillColor}{rgb}{1.00,0.50,0.00}

\path[draw=drawColor,line width= 0.4pt,line join=round,line cap=round,fill=fillColor] (327.82,455.36) circle (  1.16);

\path[draw=drawColor,line width= 0.4pt,line join=round,line cap=round,fill=fillColor] (333.65,442.46) circle (  1.16);

\path[draw=drawColor,line width= 0.4pt,line join=round,line cap=round,fill=fillColor] (337.74,439.94) circle (  1.16);

\path[draw=drawColor,line width= 0.4pt,line join=round,line cap=round,fill=fillColor] (340.99,437.94) circle (  1.16);

\path[draw=drawColor,line width= 0.4pt,line join=round,line cap=round,fill=fillColor] (343.74,437.40) circle (  1.16);

\path[draw=drawColor,line width= 0.4pt,line join=round,line cap=round,fill=fillColor] (346.14,437.20) circle (  1.16);

\path[draw=drawColor,line width= 0.4pt,line join=round,line cap=round,fill=fillColor] (348.29,436.08) circle (  1.16);

\path[draw=drawColor,line width= 0.4pt,line join=round,line cap=round,fill=fillColor] (350.24,435.99) circle (  1.16);

\path[draw=drawColor,line width= 0.4pt,line join=round,line cap=round,fill=fillColor] (352.04,435.56) circle (  1.16);

\path[draw=drawColor,line width= 0.4pt,line join=round,line cap=round,fill=fillColor] (353.70,435.13) circle (  1.16);

\path[draw=drawColor,line width= 0.4pt,line join=round,line cap=round,fill=fillColor] (355.26,435.13) circle (  1.16);

\path[draw=drawColor,line width= 0.4pt,line join=round,line cap=round,fill=fillColor] (356.73,434.11) circle (  1.16);

\path[draw=drawColor,line width= 0.4pt,line join=round,line cap=round,fill=fillColor] (358.12,433.93) circle (  1.16);

\path[draw=drawColor,line width= 0.4pt,line join=round,line cap=round,fill=fillColor] (359.44,433.79) circle (  1.16);

\path[draw=drawColor,line width= 0.4pt,line join=round,line cap=round,fill=fillColor] (360.69,433.70) circle (  1.16);

\path[draw=drawColor,line width= 0.4pt,line join=round,line cap=round,fill=fillColor] (361.90,433.40) circle (  1.16);

\path[draw=drawColor,line width= 0.4pt,line join=round,line cap=round,fill=fillColor] (363.05,433.30) circle (  1.16);

\path[draw=drawColor,line width= 0.4pt,line join=round,line cap=round,fill=fillColor] (364.16,432.32) circle (  1.16);

\path[draw=drawColor,line width= 0.4pt,line join=round,line cap=round,fill=fillColor] (365.23,432.30) circle (  1.16);

\path[draw=drawColor,line width= 0.4pt,line join=round,line cap=round,fill=fillColor] (366.26,431.13) circle (  1.16);

\path[draw=drawColor,line width= 0.4pt,line join=round,line cap=round,fill=fillColor] (367.25,430.38) circle (  1.16);

\path[draw=drawColor,line width= 0.4pt,line join=round,line cap=round,fill=fillColor] (368.22,430.30) circle (  1.16);

\path[draw=drawColor,line width= 0.4pt,line join=round,line cap=round,fill=fillColor] (369.16,430.24) circle (  1.16);

\path[draw=drawColor,line width= 0.4pt,line join=round,line cap=round,fill=fillColor] (370.07,430.18) circle (  1.16);

\path[draw=drawColor,line width= 0.4pt,line join=round,line cap=round,fill=fillColor] (370.96,430.15) circle (  1.16);

\path[draw=drawColor,line width= 0.4pt,line join=round,line cap=round,fill=fillColor] (371.82,430.13) circle (  1.16);

\path[draw=drawColor,line width= 0.4pt,line join=round,line cap=round,fill=fillColor] (372.66,430.11) circle (  1.16);

\path[draw=drawColor,line width= 0.4pt,line join=round,line cap=round,fill=fillColor] (373.48,430.06) circle (  1.16);

\path[draw=drawColor,line width= 0.4pt,line join=round,line cap=round,fill=fillColor] (374.28,429.96) circle (  1.16);

\path[draw=drawColor,line width= 0.4pt,line join=round,line cap=round,fill=fillColor] (375.06,429.96) circle (  1.16);

\path[draw=drawColor,line width= 0.4pt,line join=round,line cap=round,fill=fillColor] (375.83,429.91) circle (  1.16);

\path[draw=drawColor,line width= 0.4pt,line join=round,line cap=round,fill=fillColor] (376.58,429.53) circle (  1.16);

\path[draw=drawColor,line width= 0.4pt,line join=round,line cap=round,fill=fillColor] (377.31,429.51) circle (  1.16);

\path[draw=drawColor,line width= 0.4pt,line join=round,line cap=round,fill=fillColor] (378.03,429.46) circle (  1.16);

\path[draw=drawColor,line width= 0.4pt,line join=round,line cap=round,fill=fillColor] (378.74,429.37) circle (  1.16);

\path[draw=drawColor,line width= 0.4pt,line join=round,line cap=round,fill=fillColor] (379.43,428.46) circle (  1.16);

\path[draw=drawColor,line width= 0.4pt,line join=round,line cap=round,fill=fillColor] (380.11,428.35) circle (  1.16);

\path[draw=drawColor,line width= 0.4pt,line join=round,line cap=round,fill=fillColor] (380.77,428.00) circle (  1.16);

\path[draw=drawColor,line width= 0.4pt,line join=round,line cap=round,fill=fillColor] (381.43,427.83) circle (  1.16);

\path[draw=drawColor,line width= 0.4pt,line join=round,line cap=round,fill=fillColor] (382.07,427.70) circle (  1.16);

\path[draw=drawColor,line width= 0.4pt,line join=round,line cap=round,fill=fillColor] (382.71,427.68) circle (  1.16);

\path[draw=drawColor,line width= 0.4pt,line join=round,line cap=round,fill=fillColor] (383.33,427.64) circle (  1.16);

\path[draw=drawColor,line width= 0.4pt,line join=round,line cap=round,fill=fillColor] (383.94,427.33) circle (  1.16);

\path[draw=drawColor,line width= 0.4pt,line join=round,line cap=round,fill=fillColor] (384.55,426.75) circle (  1.16);

\path[draw=drawColor,line width= 0.4pt,line join=round,line cap=round,fill=fillColor] (385.14,426.72) circle (  1.16);

\path[draw=drawColor,line width= 0.4pt,line join=round,line cap=round,fill=fillColor] (385.73,426.63) circle (  1.16);

\path[draw=drawColor,line width= 0.4pt,line join=round,line cap=round,fill=fillColor] (386.31,426.48) circle (  1.16);

\path[draw=drawColor,line width= 0.4pt,line join=round,line cap=round,fill=fillColor] (386.88,426.41) circle (  1.16);

\path[draw=drawColor,line width= 0.4pt,line join=round,line cap=round,fill=fillColor] (387.44,426.38) circle (  1.16);

\path[draw=drawColor,line width= 0.4pt,line join=round,line cap=round,fill=fillColor] (387.99,426.38) circle (  1.16);

\path[draw=drawColor,line width= 0.4pt,line join=round,line cap=round,fill=fillColor] (388.54,426.26) circle (  1.16);

\path[draw=drawColor,line width= 0.4pt,line join=round,line cap=round,fill=fillColor] (389.08,425.98) circle (  1.16);

\path[draw=drawColor,line width= 0.4pt,line join=round,line cap=round,fill=fillColor] (389.61,425.94) circle (  1.16);

\path[draw=drawColor,line width= 0.4pt,line join=round,line cap=round,fill=fillColor] (390.14,425.90) circle (  1.16);

\path[draw=drawColor,line width= 0.4pt,line join=round,line cap=round,fill=fillColor] (390.66,425.76) circle (  1.16);

\path[draw=drawColor,line width= 0.4pt,line join=round,line cap=round,fill=fillColor] (391.17,425.50) circle (  1.16);

\path[draw=drawColor,line width= 0.4pt,line join=round,line cap=round,fill=fillColor] (391.68,425.25) circle (  1.16);

\path[draw=drawColor,line width= 0.4pt,line join=round,line cap=round,fill=fillColor] (392.18,425.23) circle (  1.16);

\path[draw=drawColor,line width= 0.4pt,line join=round,line cap=round,fill=fillColor] (392.68,425.21) circle (  1.16);

\path[draw=drawColor,line width= 0.4pt,line join=round,line cap=round,fill=fillColor] (393.17,425.14) circle (  1.16);

\path[draw=drawColor,line width= 0.4pt,line join=round,line cap=round,fill=fillColor] (393.65,425.05) circle (  1.16);

\path[draw=drawColor,line width= 0.4pt,line join=round,line cap=round,fill=fillColor] (394.13,425.01) circle (  1.16);

\path[draw=drawColor,line width= 0.4pt,line join=round,line cap=round,fill=fillColor] (394.61,424.89) circle (  1.16);

\path[draw=drawColor,line width= 0.4pt,line join=round,line cap=round,fill=fillColor] (395.08,424.88) circle (  1.16);

\path[draw=drawColor,line width= 0.4pt,line join=round,line cap=round,fill=fillColor] (395.54,424.47) circle (  1.16);

\path[draw=drawColor,line width= 0.4pt,line join=round,line cap=round,fill=fillColor] (396.00,424.08) circle (  1.16);

\path[draw=drawColor,line width= 0.4pt,line join=round,line cap=round,fill=fillColor] (396.46,423.94) circle (  1.16);

\path[draw=drawColor,line width= 0.4pt,line join=round,line cap=round,fill=fillColor] (396.91,423.72) circle (  1.16);

\path[draw=drawColor,line width= 0.4pt,line join=round,line cap=round,fill=fillColor] (397.35,422.70) circle (  1.16);

\path[draw=drawColor,line width= 0.4pt,line join=round,line cap=round,fill=fillColor] (397.80,422.39) circle (  1.16);

\path[draw=drawColor,line width= 0.4pt,line join=round,line cap=round,fill=fillColor] (398.23,422.00) circle (  1.16);

\path[draw=drawColor,line width= 0.4pt,line join=round,line cap=round,fill=fillColor] (398.67,422.00) circle (  1.16);

\path[draw=drawColor,line width= 0.4pt,line join=round,line cap=round,fill=fillColor] (399.10,421.85) circle (  1.16);

\path[draw=drawColor,line width= 0.4pt,line join=round,line cap=round,fill=fillColor] (399.52,421.63) circle (  1.16);

\path[draw=drawColor,line width= 0.4pt,line join=round,line cap=round,fill=fillColor] (399.94,421.42) circle (  1.16);

\path[draw=drawColor,line width= 0.4pt,line join=round,line cap=round,fill=fillColor] (400.36,421.32) circle (  1.16);

\path[draw=drawColor,line width= 0.4pt,line join=round,line cap=round,fill=fillColor] (400.78,421.28) circle (  1.16);

\path[draw=drawColor,line width= 0.4pt,line join=round,line cap=round,fill=fillColor] (401.19,420.80) circle (  1.16);

\path[draw=drawColor,line width= 0.4pt,line join=round,line cap=round,fill=fillColor] (401.60,420.77) circle (  1.16);

\path[draw=drawColor,line width= 0.4pt,line join=round,line cap=round,fill=fillColor] (402.00,420.75) circle (  1.16);

\path[draw=drawColor,line width= 0.4pt,line join=round,line cap=round,fill=fillColor] (402.40,420.56) circle (  1.16);

\path[draw=drawColor,line width= 0.4pt,line join=round,line cap=round,fill=fillColor] (402.80,420.54) circle (  1.16);

\path[draw=drawColor,line width= 0.4pt,line join=round,line cap=round,fill=fillColor] (403.19,420.51) circle (  1.16);

\path[draw=drawColor,line width= 0.4pt,line join=round,line cap=round,fill=fillColor] (403.58,420.45) circle (  1.16);

\path[draw=drawColor,line width= 0.4pt,line join=round,line cap=round,fill=fillColor] (403.97,420.35) circle (  1.16);

\path[draw=drawColor,line width= 0.4pt,line join=round,line cap=round,fill=fillColor] (404.36,420.34) circle (  1.16);

\path[draw=drawColor,line width= 0.4pt,line join=round,line cap=round,fill=fillColor] (404.74,420.23) circle (  1.16);

\path[draw=drawColor,line width= 0.4pt,line join=round,line cap=round,fill=fillColor] (405.12,420.22) circle (  1.16);

\path[draw=drawColor,line width= 0.4pt,line join=round,line cap=round,fill=fillColor] (405.49,419.36) circle (  1.16);

\path[draw=drawColor,line width= 0.4pt,line join=round,line cap=round,fill=fillColor] (405.87,419.36) circle (  1.16);

\path[draw=drawColor,line width= 0.4pt,line join=round,line cap=round,fill=fillColor] (406.24,419.32) circle (  1.16);

\path[draw=drawColor,line width= 0.4pt,line join=round,line cap=round,fill=fillColor] (406.61,419.27) circle (  1.16);

\path[draw=drawColor,line width= 0.4pt,line join=round,line cap=round,fill=fillColor] (406.97,419.09) circle (  1.16);

\path[draw=drawColor,line width= 0.4pt,line join=round,line cap=round,fill=fillColor] (407.34,419.05) circle (  1.16);

\path[draw=drawColor,line width= 0.4pt,line join=round,line cap=round,fill=fillColor] (407.70,418.95) circle (  1.16);

\path[draw=drawColor,line width= 0.4pt,line join=round,line cap=round,fill=fillColor] (408.05,418.94) circle (  1.16);

\path[draw=drawColor,line width= 0.4pt,line join=round,line cap=round,fill=fillColor] (408.41,418.93) circle (  1.16);

\path[draw=drawColor,line width= 0.4pt,line join=round,line cap=round,fill=fillColor] (408.76,418.90) circle (  1.16);

\path[draw=drawColor,line width= 0.4pt,line join=round,line cap=round,fill=fillColor] (409.11,418.90) circle (  1.16);

\path[draw=drawColor,line width= 0.4pt,line join=round,line cap=round,fill=fillColor] (409.46,418.83) circle (  1.16);

\path[draw=drawColor,line width= 0.4pt,line join=round,line cap=round,fill=fillColor] (409.80,418.77) circle (  1.16);

\path[draw=drawColor,line width= 0.4pt,line join=round,line cap=round,fill=fillColor] (410.15,418.51) circle (  1.16);

\path[draw=drawColor,line width= 0.4pt,line join=round,line cap=round,fill=fillColor] (410.49,418.50) circle (  1.16);

\path[draw=drawColor,line width= 0.4pt,line join=round,line cap=round,fill=fillColor] (410.83,418.47) circle (  1.16);

\path[draw=drawColor,line width= 0.4pt,line join=round,line cap=round,fill=fillColor] (411.17,418.27) circle (  1.16);

\path[draw=drawColor,line width= 0.4pt,line join=round,line cap=round,fill=fillColor] (411.50,418.11) circle (  1.16);

\path[draw=drawColor,line width= 0.4pt,line join=round,line cap=round,fill=fillColor] (411.83,417.98) circle (  1.16);

\path[draw=drawColor,line width= 0.4pt,line join=round,line cap=round,fill=fillColor] (412.16,417.83) circle (  1.16);

\path[draw=drawColor,line width= 0.4pt,line join=round,line cap=round,fill=fillColor] (412.49,417.72) circle (  1.16);

\path[draw=drawColor,line width= 0.4pt,line join=round,line cap=round,fill=fillColor] (412.82,417.57) circle (  1.16);

\path[draw=drawColor,line width= 0.4pt,line join=round,line cap=round,fill=fillColor] (413.14,417.52) circle (  1.16);

\path[draw=drawColor,line width= 0.4pt,line join=round,line cap=round,fill=fillColor] (413.47,417.26) circle (  1.16);

\path[draw=drawColor,line width= 0.4pt,line join=round,line cap=round,fill=fillColor] (413.79,417.25) circle (  1.16);

\path[draw=drawColor,line width= 0.4pt,line join=round,line cap=round,fill=fillColor] (414.10,417.16) circle (  1.16);

\path[draw=drawColor,line width= 0.4pt,line join=round,line cap=round,fill=fillColor] (414.42,417.15) circle (  1.16);

\path[draw=drawColor,line width= 0.4pt,line join=round,line cap=round,fill=fillColor] (414.74,417.11) circle (  1.16);

\path[draw=drawColor,line width= 0.4pt,line join=round,line cap=round,fill=fillColor] (415.05,417.06) circle (  1.16);

\path[draw=drawColor,line width= 0.4pt,line join=round,line cap=round,fill=fillColor] (415.36,416.97) circle (  1.16);

\path[draw=drawColor,line width= 0.4pt,line join=round,line cap=round,fill=fillColor] (415.67,416.94) circle (  1.16);

\path[draw=drawColor,line width= 0.4pt,line join=round,line cap=round,fill=fillColor] (415.98,416.87) circle (  1.16);

\path[draw=drawColor,line width= 0.4pt,line join=round,line cap=round,fill=fillColor] (416.29,416.86) circle (  1.16);

\path[draw=drawColor,line width= 0.4pt,line join=round,line cap=round,fill=fillColor] (416.59,416.80) circle (  1.16);

\path[draw=drawColor,line width= 0.4pt,line join=round,line cap=round,fill=fillColor] (416.89,416.75) circle (  1.16);

\path[draw=drawColor,line width= 0.4pt,line join=round,line cap=round,fill=fillColor] (417.19,416.72) circle (  1.16);

\path[draw=drawColor,line width= 0.4pt,line join=round,line cap=round,fill=fillColor] (417.49,416.71) circle (  1.16);

\path[draw=drawColor,line width= 0.4pt,line join=round,line cap=round,fill=fillColor] (417.79,416.67) circle (  1.16);

\path[draw=drawColor,line width= 0.4pt,line join=round,line cap=round,fill=fillColor] (418.09,416.57) circle (  1.16);

\path[draw=drawColor,line width= 0.4pt,line join=round,line cap=round,fill=fillColor] (418.38,416.55) circle (  1.16);

\path[draw=drawColor,line width= 0.4pt,line join=round,line cap=round,fill=fillColor] (418.68,416.46) circle (  1.16);

\path[draw=drawColor,line width= 0.4pt,line join=round,line cap=round,fill=fillColor] (418.97,416.39) circle (  1.16);

\path[draw=drawColor,line width= 0.4pt,line join=round,line cap=round,fill=fillColor] (419.26,416.36) circle (  1.16);

\path[draw=drawColor,line width= 0.4pt,line join=round,line cap=round,fill=fillColor] (419.55,416.34) circle (  1.16);

\path[draw=drawColor,line width= 0.4pt,line join=round,line cap=round,fill=fillColor] (419.84,416.31) circle (  1.16);

\path[draw=drawColor,line width= 0.4pt,line join=round,line cap=round,fill=fillColor] (420.12,416.26) circle (  1.16);

\path[draw=drawColor,line width= 0.4pt,line join=round,line cap=round,fill=fillColor] (420.41,416.23) circle (  1.16);

\path[draw=drawColor,line width= 0.4pt,line join=round,line cap=round,fill=fillColor] (420.69,416.23) circle (  1.16);

\path[draw=drawColor,line width= 0.4pt,line join=round,line cap=round,fill=fillColor] (420.97,416.20) circle (  1.16);

\path[draw=drawColor,line width= 0.4pt,line join=round,line cap=round,fill=fillColor] (421.25,416.18) circle (  1.16);

\path[draw=drawColor,line width= 0.4pt,line join=round,line cap=round,fill=fillColor] (421.53,416.16) circle (  1.16);

\path[draw=drawColor,line width= 0.4pt,line join=round,line cap=round,fill=fillColor] (421.81,416.07) circle (  1.16);

\path[draw=drawColor,line width= 0.4pt,line join=round,line cap=round,fill=fillColor] (422.09,415.86) circle (  1.16);

\path[draw=drawColor,line width= 0.4pt,line join=round,line cap=round,fill=fillColor] (422.36,415.86) circle (  1.16);

\path[draw=drawColor,line width= 0.4pt,line join=round,line cap=round,fill=fillColor] (422.63,415.85) circle (  1.16);

\path[draw=drawColor,line width= 0.4pt,line join=round,line cap=round,fill=fillColor] (422.91,415.83) circle (  1.16);

\path[draw=drawColor,line width= 0.4pt,line join=round,line cap=round,fill=fillColor] (423.18,415.75) circle (  1.16);

\path[draw=drawColor,line width= 0.4pt,line join=round,line cap=round,fill=fillColor] (423.45,415.70) circle (  1.16);

\path[draw=drawColor,line width= 0.4pt,line join=round,line cap=round,fill=fillColor] (423.72,415.66) circle (  1.16);

\path[draw=drawColor,line width= 0.4pt,line join=round,line cap=round,fill=fillColor] (423.99,415.58) circle (  1.16);

\path[draw=drawColor,line width= 0.4pt,line join=round,line cap=round,fill=fillColor] (424.25,415.55) circle (  1.16);

\path[draw=drawColor,line width= 0.4pt,line join=round,line cap=round,fill=fillColor] (424.52,415.53) circle (  1.16);

\path[draw=drawColor,line width= 0.4pt,line join=round,line cap=round,fill=fillColor] (424.78,415.51) circle (  1.16);

\path[draw=drawColor,line width= 0.4pt,line join=round,line cap=round,fill=fillColor] (425.04,415.39) circle (  1.16);

\path[draw=drawColor,line width= 0.4pt,line join=round,line cap=round,fill=fillColor] (425.31,415.37) circle (  1.16);

\path[draw=drawColor,line width= 0.4pt,line join=round,line cap=round,fill=fillColor] (425.57,415.34) circle (  1.16);

\path[draw=drawColor,line width= 0.4pt,line join=round,line cap=round,fill=fillColor] (425.83,415.34) circle (  1.16);

\path[draw=drawColor,line width= 0.4pt,line join=round,line cap=round,fill=fillColor] (426.08,415.28) circle (  1.16);

\path[draw=drawColor,line width= 0.4pt,line join=round,line cap=round,fill=fillColor] (426.34,415.19) circle (  1.16);

\path[draw=drawColor,line width= 0.4pt,line join=round,line cap=round,fill=fillColor] (426.60,415.14) circle (  1.16);

\path[draw=drawColor,line width= 0.4pt,line join=round,line cap=round,fill=fillColor] (426.85,414.94) circle (  1.16);

\path[draw=drawColor,line width= 0.4pt,line join=round,line cap=round,fill=fillColor] (427.11,414.94) circle (  1.16);

\path[draw=drawColor,line width= 0.4pt,line join=round,line cap=round,fill=fillColor] (427.36,414.93) circle (  1.16);

\path[draw=drawColor,line width= 0.4pt,line join=round,line cap=round,fill=fillColor] (427.61,414.86) circle (  1.16);

\path[draw=drawColor,line width= 0.4pt,line join=round,line cap=round,fill=fillColor] (427.86,414.80) circle (  1.16);

\path[draw=drawColor,line width= 0.4pt,line join=round,line cap=round,fill=fillColor] (428.11,414.75) circle (  1.16);

\path[draw=drawColor,line width= 0.4pt,line join=round,line cap=round,fill=fillColor] (428.36,414.66) circle (  1.16);

\path[draw=drawColor,line width= 0.4pt,line join=round,line cap=round,fill=fillColor] (428.61,414.61) circle (  1.16);

\path[draw=drawColor,line width= 0.4pt,line join=round,line cap=round,fill=fillColor] (428.86,414.46) circle (  1.16);

\path[draw=drawColor,line width= 0.4pt,line join=round,line cap=round,fill=fillColor] (429.10,414.40) circle (  1.16);

\path[draw=drawColor,line width= 0.4pt,line join=round,line cap=round,fill=fillColor] (429.35,414.30) circle (  1.16);

\path[draw=drawColor,line width= 0.4pt,line join=round,line cap=round,fill=fillColor] (429.59,414.30) circle (  1.16);

\path[draw=drawColor,line width= 0.4pt,line join=round,line cap=round,fill=fillColor] (429.83,414.28) circle (  1.16);

\path[draw=drawColor,line width= 0.4pt,line join=round,line cap=round,fill=fillColor] (430.08,414.25) circle (  1.16);

\path[draw=drawColor,line width= 0.4pt,line join=round,line cap=round,fill=fillColor] (430.32,414.22) circle (  1.16);

\path[draw=drawColor,line width= 0.4pt,line join=round,line cap=round,fill=fillColor] (430.56,413.98) circle (  1.16);

\path[draw=drawColor,line width= 0.4pt,line join=round,line cap=round,fill=fillColor] (430.80,413.95) circle (  1.16);

\path[draw=drawColor,line width= 0.4pt,line join=round,line cap=round,fill=fillColor] (431.04,413.95) circle (  1.16);

\path[draw=drawColor,line width= 0.4pt,line join=round,line cap=round,fill=fillColor] (431.27,413.88) circle (  1.16);

\path[draw=drawColor,line width= 0.4pt,line join=round,line cap=round,fill=fillColor] (431.51,413.84) circle (  1.16);

\path[draw=drawColor,line width= 0.4pt,line join=round,line cap=round,fill=fillColor] (431.75,413.80) circle (  1.16);

\path[draw=drawColor,line width= 0.4pt,line join=round,line cap=round,fill=fillColor] (431.98,413.79) circle (  1.16);

\path[draw=drawColor,line width= 0.4pt,line join=round,line cap=round,fill=fillColor] (432.21,413.78) circle (  1.16);

\path[draw=drawColor,line width= 0.4pt,line join=round,line cap=round,fill=fillColor] (432.45,413.64) circle (  1.16);

\path[draw=drawColor,line width= 0.4pt,line join=round,line cap=round,fill=fillColor] (432.68,413.51) circle (  1.16);

\path[draw=drawColor,line width= 0.4pt,line join=round,line cap=round,fill=fillColor] (432.91,413.49) circle (  1.16);

\path[draw=drawColor,line width= 0.4pt,line join=round,line cap=round,fill=fillColor] (433.14,413.48) circle (  1.16);

\path[draw=drawColor,line width= 0.4pt,line join=round,line cap=round,fill=fillColor] (433.37,413.46) circle (  1.16);

\path[draw=drawColor,line width= 0.4pt,line join=round,line cap=round,fill=fillColor] (433.60,413.41) circle (  1.16);

\path[draw=drawColor,line width= 0.4pt,line join=round,line cap=round,fill=fillColor] (433.83,413.28) circle (  1.16);

\path[draw=drawColor,line width= 0.4pt,line join=round,line cap=round,fill=fillColor] (434.06,413.23) circle (  1.16);

\path[draw=drawColor,line width= 0.4pt,line join=round,line cap=round,fill=fillColor] (434.28,413.14) circle (  1.16);

\path[draw=drawColor,line width= 0.4pt,line join=round,line cap=round,fill=fillColor] (434.51,413.13) circle (  1.16);

\path[draw=drawColor,line width= 0.4pt,line join=round,line cap=round,fill=fillColor] (434.73,413.11) circle (  1.16);

\path[draw=drawColor,line width= 0.4pt,line join=round,line cap=round,fill=fillColor] (434.96,413.09) circle (  1.16);

\path[draw=drawColor,line width= 0.4pt,line join=round,line cap=round,fill=fillColor] (435.18,413.03) circle (  1.16);

\path[draw=drawColor,line width= 0.4pt,line join=round,line cap=round,fill=fillColor] (435.40,413.00) circle (  1.16);

\path[draw=drawColor,line width= 0.4pt,line join=round,line cap=round,fill=fillColor] (435.62,412.98) circle (  1.16);

\path[draw=drawColor,line width= 0.4pt,line join=round,line cap=round,fill=fillColor] (435.85,412.96) circle (  1.16);

\path[draw=drawColor,line width= 0.4pt,line join=round,line cap=round,fill=fillColor] (436.07,412.92) circle (  1.16);

\path[draw=drawColor,line width= 0.4pt,line join=round,line cap=round,fill=fillColor] (436.29,412.92) circle (  1.16);

\path[draw=drawColor,line width= 0.4pt,line join=round,line cap=round,fill=fillColor] (436.50,412.88) circle (  1.16);

\path[draw=drawColor,line width= 0.4pt,line join=round,line cap=round,fill=fillColor] (436.72,412.87) circle (  1.16);

\path[draw=drawColor,line width= 0.4pt,line join=round,line cap=round,fill=fillColor] (436.94,412.85) circle (  1.16);

\path[draw=drawColor,line width= 0.4pt,line join=round,line cap=round,fill=fillColor] (437.16,412.83) circle (  1.16);

\path[draw=drawColor,line width= 0.4pt,line join=round,line cap=round,fill=fillColor] (437.37,412.82) circle (  1.16);

\path[draw=drawColor,line width= 0.4pt,line join=round,line cap=round,fill=fillColor] (437.59,412.78) circle (  1.16);

\path[draw=drawColor,line width= 0.4pt,line join=round,line cap=round,fill=fillColor] (437.80,412.75) circle (  1.16);

\path[draw=drawColor,line width= 0.4pt,line join=round,line cap=round,fill=fillColor] (438.02,412.69) circle (  1.16);

\path[draw=drawColor,line width= 0.4pt,line join=round,line cap=round,fill=fillColor] (438.23,412.67) circle (  1.16);

\path[draw=drawColor,line width= 0.4pt,line join=round,line cap=round,fill=fillColor] (438.44,412.53) circle (  1.16);

\path[draw=drawColor,line width= 0.4pt,line join=round,line cap=round,fill=fillColor] (438.65,412.52) circle (  1.16);

\path[draw=drawColor,line width= 0.4pt,line join=round,line cap=round,fill=fillColor] (438.87,412.49) circle (  1.16);

\path[draw=drawColor,line width= 0.4pt,line join=round,line cap=round,fill=fillColor] (439.08,412.45) circle (  1.16);

\path[draw=drawColor,line width= 0.4pt,line join=round,line cap=round,fill=fillColor] (439.29,412.34) circle (  1.16);

\path[draw=drawColor,line width= 0.4pt,line join=round,line cap=round,fill=fillColor] (439.49,412.29) circle (  1.16);

\path[draw=drawColor,line width= 0.4pt,line join=round,line cap=round,fill=fillColor] (439.70,412.26) circle (  1.16);

\path[draw=drawColor,line width= 0.4pt,line join=round,line cap=round,fill=fillColor] (439.91,412.21) circle (  1.16);

\path[draw=drawColor,line width= 0.4pt,line join=round,line cap=round,fill=fillColor] (440.12,412.15) circle (  1.16);

\path[draw=drawColor,line width= 0.4pt,line join=round,line cap=round,fill=fillColor] (440.33,412.15) circle (  1.16);

\path[draw=drawColor,line width= 0.4pt,line join=round,line cap=round,fill=fillColor] (440.53,412.02) circle (  1.16);

\path[draw=drawColor,line width= 0.4pt,line join=round,line cap=round,fill=fillColor] (440.74,412.00) circle (  1.16);

\path[draw=drawColor,line width= 0.4pt,line join=round,line cap=round,fill=fillColor] (440.94,411.98) circle (  1.16);

\path[draw=drawColor,line width= 0.4pt,line join=round,line cap=round,fill=fillColor] (441.15,411.93) circle (  1.16);

\path[draw=drawColor,line width= 0.4pt,line join=round,line cap=round,fill=fillColor] (441.35,411.91) circle (  1.16);

\path[draw=drawColor,line width= 0.4pt,line join=round,line cap=round,fill=fillColor] (441.55,411.85) circle (  1.16);

\path[draw=drawColor,line width= 0.4pt,line join=round,line cap=round,fill=fillColor] (441.75,411.78) circle (  1.16);

\path[draw=drawColor,line width= 0.4pt,line join=round,line cap=round,fill=fillColor] (441.96,411.78) circle (  1.16);

\path[draw=drawColor,line width= 0.4pt,line join=round,line cap=round,fill=fillColor] (442.16,411.74) circle (  1.16);

\path[draw=drawColor,line width= 0.4pt,line join=round,line cap=round,fill=fillColor] (442.36,411.65) circle (  1.16);

\path[draw=drawColor,line width= 0.4pt,line join=round,line cap=round,fill=fillColor] (442.56,411.65) circle (  1.16);

\path[draw=drawColor,line width= 0.4pt,line join=round,line cap=round,fill=fillColor] (442.76,411.61) circle (  1.16);

\path[draw=drawColor,line width= 0.4pt,line join=round,line cap=round,fill=fillColor] (442.96,411.56) circle (  1.16);

\path[draw=drawColor,line width= 0.4pt,line join=round,line cap=round,fill=fillColor] (443.15,411.53) circle (  1.16);

\path[draw=drawColor,line width= 0.4pt,line join=round,line cap=round,fill=fillColor] (443.35,411.51) circle (  1.16);

\path[draw=drawColor,line width= 0.4pt,line join=round,line cap=round,fill=fillColor] (443.55,411.45) circle (  1.16);

\path[draw=drawColor,line width= 0.4pt,line join=round,line cap=round,fill=fillColor] (443.74,411.36) circle (  1.16);

\path[draw=drawColor,line width= 0.4pt,line join=round,line cap=round,fill=fillColor] (443.94,411.30) circle (  1.16);

\path[draw=drawColor,line width= 0.4pt,line join=round,line cap=round,fill=fillColor] (444.14,411.29) circle (  1.16);

\path[draw=drawColor,line width= 0.4pt,line join=round,line cap=round,fill=fillColor] (444.33,411.21) circle (  1.16);

\path[draw=drawColor,line width= 0.4pt,line join=round,line cap=round,fill=fillColor] (444.53,411.20) circle (  1.16);

\path[draw=drawColor,line width= 0.4pt,line join=round,line cap=round,fill=fillColor] (444.72,411.19) circle (  1.16);

\path[draw=drawColor,line width= 0.4pt,line join=round,line cap=round,fill=fillColor] (444.91,411.16) circle (  1.16);

\path[draw=drawColor,line width= 0.4pt,line join=round,line cap=round,fill=fillColor] (445.10,411.10) circle (  1.16);

\path[draw=drawColor,line width= 0.4pt,line join=round,line cap=round,fill=fillColor] (445.30,411.02) circle (  1.16);

\path[draw=drawColor,line width= 0.4pt,line join=round,line cap=round,fill=fillColor] (445.49,410.96) circle (  1.16);

\path[draw=drawColor,line width= 0.4pt,line join=round,line cap=round,fill=fillColor] (445.68,410.96) circle (  1.16);

\path[draw=drawColor,line width= 0.4pt,line join=round,line cap=round,fill=fillColor] (445.87,410.96) circle (  1.16);

\path[draw=drawColor,line width= 0.4pt,line join=round,line cap=round,fill=fillColor] (446.06,410.94) circle (  1.16);

\path[draw=drawColor,line width= 0.4pt,line join=round,line cap=round,fill=fillColor] (446.25,410.93) circle (  1.16);

\path[draw=drawColor,line width= 0.4pt,line join=round,line cap=round,fill=fillColor] (446.44,410.91) circle (  1.16);

\path[draw=drawColor,line width= 0.4pt,line join=round,line cap=round,fill=fillColor] (446.63,410.89) circle (  1.16);

\path[draw=drawColor,line width= 0.4pt,line join=round,line cap=round,fill=fillColor] (446.82,410.78) circle (  1.16);

\path[draw=drawColor,line width= 0.4pt,line join=round,line cap=round,fill=fillColor] (447.00,410.72) circle (  1.16);

\path[draw=drawColor,line width= 0.4pt,line join=round,line cap=round,fill=fillColor] (447.19,410.67) circle (  1.16);

\path[draw=drawColor,line width= 0.4pt,line join=round,line cap=round,fill=fillColor] (447.38,410.64) circle (  1.16);

\path[draw=drawColor,line width= 0.4pt,line join=round,line cap=round,fill=fillColor] (447.56,410.62) circle (  1.16);

\path[draw=drawColor,line width= 0.4pt,line join=round,line cap=round,fill=fillColor] (447.75,410.61) circle (  1.16);

\path[draw=drawColor,line width= 0.4pt,line join=round,line cap=round,fill=fillColor] (447.93,410.60) circle (  1.16);

\path[draw=drawColor,line width= 0.4pt,line join=round,line cap=round,fill=fillColor] (448.12,410.47) circle (  1.16);

\path[draw=drawColor,line width= 0.4pt,line join=round,line cap=round,fill=fillColor] (448.30,410.34) circle (  1.16);

\path[draw=drawColor,line width= 0.4pt,line join=round,line cap=round,fill=fillColor] (448.49,410.29) circle (  1.16);

\path[draw=drawColor,line width= 0.4pt,line join=round,line cap=round,fill=fillColor] (448.67,410.26) circle (  1.16);

\path[draw=drawColor,line width= 0.4pt,line join=round,line cap=round,fill=fillColor] (448.85,410.16) circle (  1.16);

\path[draw=drawColor,line width= 0.4pt,line join=round,line cap=round,fill=fillColor] (449.03,410.06) circle (  1.16);

\path[draw=drawColor,line width= 0.4pt,line join=round,line cap=round,fill=fillColor] (449.22,410.04) circle (  1.16);

\path[draw=drawColor,line width= 0.4pt,line join=round,line cap=round,fill=fillColor] (449.40,410.01) circle (  1.16);

\path[draw=drawColor,line width= 0.4pt,line join=round,line cap=round,fill=fillColor] (449.58,410.00) circle (  1.16);

\path[draw=drawColor,line width= 0.4pt,line join=round,line cap=round,fill=fillColor] (449.76,409.96) circle (  1.16);

\path[draw=drawColor,line width= 0.4pt,line join=round,line cap=round,fill=fillColor] (449.94,409.93) circle (  1.16);

\path[draw=drawColor,line width= 0.4pt,line join=round,line cap=round,fill=fillColor] (450.12,409.93) circle (  1.16);

\path[draw=drawColor,line width= 0.4pt,line join=round,line cap=round,fill=fillColor] (450.30,409.88) circle (  1.16);

\path[draw=drawColor,line width= 0.4pt,line join=round,line cap=round,fill=fillColor] (450.48,409.87) circle (  1.16);

\path[draw=drawColor,line width= 0.4pt,line join=round,line cap=round,fill=fillColor] (450.65,409.86) circle (  1.16);

\path[draw=drawColor,line width= 0.4pt,line join=round,line cap=round,fill=fillColor] (450.83,409.82) circle (  1.16);

\path[draw=drawColor,line width= 0.4pt,line join=round,line cap=round,fill=fillColor] (451.01,409.81) circle (  1.16);

\path[draw=drawColor,line width= 0.4pt,line join=round,line cap=round,fill=fillColor] (451.19,409.71) circle (  1.16);

\path[draw=drawColor,line width= 0.4pt,line join=round,line cap=round,fill=fillColor] (451.36,409.68) circle (  1.16);

\path[draw=drawColor,line width= 0.4pt,line join=round,line cap=round,fill=fillColor] (451.54,409.61) circle (  1.16);

\path[draw=drawColor,line width= 0.4pt,line join=round,line cap=round,fill=fillColor] (451.71,409.61) circle (  1.16);

\path[draw=drawColor,line width= 0.4pt,line join=round,line cap=round,fill=fillColor] (451.89,409.54) circle (  1.16);

\path[draw=drawColor,line width= 0.4pt,line join=round,line cap=round,fill=fillColor] (452.06,409.53) circle (  1.16);

\path[draw=drawColor,line width= 0.4pt,line join=round,line cap=round,fill=fillColor] (452.24,409.53) circle (  1.16);

\path[draw=drawColor,line width= 0.4pt,line join=round,line cap=round,fill=fillColor] (452.41,409.50) circle (  1.16);

\path[draw=drawColor,line width= 0.4pt,line join=round,line cap=round,fill=fillColor] (452.59,409.49) circle (  1.16);

\path[draw=drawColor,line width= 0.4pt,line join=round,line cap=round,fill=fillColor] (452.76,409.45) circle (  1.16);

\path[draw=drawColor,line width= 0.4pt,line join=round,line cap=round,fill=fillColor] (452.93,409.34) circle (  1.16);

\path[draw=drawColor,line width= 0.4pt,line join=round,line cap=round,fill=fillColor] (453.10,409.33) circle (  1.16);

\path[draw=drawColor,line width= 0.4pt,line join=round,line cap=round,fill=fillColor] (453.28,409.25) circle (  1.16);

\path[draw=drawColor,line width= 0.4pt,line join=round,line cap=round,fill=fillColor] (453.45,409.24) circle (  1.16);

\path[draw=drawColor,line width= 0.4pt,line join=round,line cap=round,fill=fillColor] (453.62,409.14) circle (  1.16);

\path[draw=drawColor,line width= 0.4pt,line join=round,line cap=round,fill=fillColor] (453.79,409.14) circle (  1.16);

\path[draw=drawColor,line width= 0.4pt,line join=round,line cap=round,fill=fillColor] (453.96,408.86) circle (  1.16);

\path[draw=drawColor,line width= 0.4pt,line join=round,line cap=round,fill=fillColor] (454.13,408.75) circle (  1.16);

\path[draw=drawColor,line width= 0.4pt,line join=round,line cap=round,fill=fillColor] (454.30,408.75) circle (  1.16);

\path[draw=drawColor,line width= 0.4pt,line join=round,line cap=round,fill=fillColor] (454.47,408.69) circle (  1.16);

\path[draw=drawColor,line width= 0.4pt,line join=round,line cap=round,fill=fillColor] (454.64,408.66) circle (  1.16);

\path[draw=drawColor,line width= 0.4pt,line join=round,line cap=round,fill=fillColor] (454.81,408.60) circle (  1.16);

\path[draw=drawColor,line width= 0.4pt,line join=round,line cap=round,fill=fillColor] (454.97,408.54) circle (  1.16);

\path[draw=drawColor,line width= 0.4pt,line join=round,line cap=round,fill=fillColor] (455.14,408.53) circle (  1.16);

\path[draw=drawColor,line width= 0.4pt,line join=round,line cap=round,fill=fillColor] (455.31,408.46) circle (  1.16);

\path[draw=drawColor,line width= 0.4pt,line join=round,line cap=round,fill=fillColor] (455.48,408.44) circle (  1.16);

\path[draw=drawColor,line width= 0.4pt,line join=round,line cap=round,fill=fillColor] (455.64,408.44) circle (  1.16);

\path[draw=drawColor,line width= 0.4pt,line join=round,line cap=round,fill=fillColor] (455.81,408.43) circle (  1.16);

\path[draw=drawColor,line width= 0.4pt,line join=round,line cap=round,fill=fillColor] (455.97,408.30) circle (  1.16);

\path[draw=drawColor,line width= 0.4pt,line join=round,line cap=round,fill=fillColor] (456.14,408.24) circle (  1.16);

\path[draw=drawColor,line width= 0.4pt,line join=round,line cap=round,fill=fillColor] (456.31,408.18) circle (  1.16);

\path[draw=drawColor,line width= 0.4pt,line join=round,line cap=round,fill=fillColor] (456.47,408.06) circle (  1.16);

\path[draw=drawColor,line width= 0.4pt,line join=round,line cap=round,fill=fillColor] (456.63,408.05) circle (  1.16);

\path[draw=drawColor,line width= 0.4pt,line join=round,line cap=round,fill=fillColor] (456.80,408.05) circle (  1.16);

\path[draw=drawColor,line width= 0.4pt,line join=round,line cap=round,fill=fillColor] (456.96,407.96) circle (  1.16);

\path[draw=drawColor,line width= 0.4pt,line join=round,line cap=round,fill=fillColor] (457.13,407.93) circle (  1.16);

\path[draw=drawColor,line width= 0.4pt,line join=round,line cap=round,fill=fillColor] (457.29,407.87) circle (  1.16);

\path[draw=drawColor,line width= 0.4pt,line join=round,line cap=round,fill=fillColor] (457.45,407.83) circle (  1.16);

\path[draw=drawColor,line width= 0.4pt,line join=round,line cap=round,fill=fillColor] (457.61,407.78) circle (  1.16);

\path[draw=drawColor,line width= 0.4pt,line join=round,line cap=round,fill=fillColor] (457.78,407.76) circle (  1.16);

\path[draw=drawColor,line width= 0.4pt,line join=round,line cap=round,fill=fillColor] (457.94,407.70) circle (  1.16);

\path[draw=drawColor,line width= 0.4pt,line join=round,line cap=round,fill=fillColor] (458.10,407.66) circle (  1.16);

\path[draw=drawColor,line width= 0.4pt,line join=round,line cap=round,fill=fillColor] (458.26,407.51) circle (  1.16);

\path[draw=drawColor,line width= 0.4pt,line join=round,line cap=round,fill=fillColor] (458.42,407.48) circle (  1.16);

\path[draw=drawColor,line width= 0.4pt,line join=round,line cap=round,fill=fillColor] (458.58,407.44) circle (  1.16);

\path[draw=drawColor,line width= 0.4pt,line join=round,line cap=round,fill=fillColor] (458.74,407.43) circle (  1.16);

\path[draw=drawColor,line width= 0.4pt,line join=round,line cap=round,fill=fillColor] (458.90,407.41) circle (  1.16);

\path[draw=drawColor,line width= 0.4pt,line join=round,line cap=round,fill=fillColor] (459.06,407.36) circle (  1.16);

\path[draw=drawColor,line width= 0.4pt,line join=round,line cap=round,fill=fillColor] (459.22,407.35) circle (  1.16);

\path[draw=drawColor,line width= 0.4pt,line join=round,line cap=round,fill=fillColor] (459.38,407.28) circle (  1.16);

\path[draw=drawColor,line width= 0.4pt,line join=round,line cap=round,fill=fillColor] (459.53,407.28) circle (  1.16);

\path[draw=drawColor,line width= 0.4pt,line join=round,line cap=round,fill=fillColor] (459.69,407.26) circle (  1.16);

\path[draw=drawColor,line width= 0.4pt,line join=round,line cap=round,fill=fillColor] (459.85,407.21) circle (  1.16);

\path[draw=drawColor,line width= 0.4pt,line join=round,line cap=round,fill=fillColor] (460.01,407.15) circle (  1.16);

\path[draw=drawColor,line width= 0.4pt,line join=round,line cap=round,fill=fillColor] (460.16,407.10) circle (  1.16);

\path[draw=drawColor,line width= 0.4pt,line join=round,line cap=round,fill=fillColor] (460.32,407.06) circle (  1.16);

\path[draw=drawColor,line width= 0.4pt,line join=round,line cap=round,fill=fillColor] (460.48,407.06) circle (  1.16);

\path[draw=drawColor,line width= 0.4pt,line join=round,line cap=round,fill=fillColor] (460.63,407.06) circle (  1.16);

\path[draw=drawColor,line width= 0.4pt,line join=round,line cap=round,fill=fillColor] (460.79,406.95) circle (  1.16);

\path[draw=drawColor,line width= 0.4pt,line join=round,line cap=round,fill=fillColor] (460.94,406.88) circle (  1.16);

\path[draw=drawColor,line width= 0.4pt,line join=round,line cap=round,fill=fillColor] (461.10,406.82) circle (  1.16);

\path[draw=drawColor,line width= 0.4pt,line join=round,line cap=round,fill=fillColor] (461.25,406.82) circle (  1.16);

\path[draw=drawColor,line width= 0.4pt,line join=round,line cap=round,fill=fillColor] (461.41,406.80) circle (  1.16);

\path[draw=drawColor,line width= 0.4pt,line join=round,line cap=round,fill=fillColor] (461.56,406.77) circle (  1.16);

\path[draw=drawColor,line width= 0.4pt,line join=round,line cap=round,fill=fillColor] (461.72,406.74) circle (  1.16);

\path[draw=drawColor,line width= 0.4pt,line join=round,line cap=round,fill=fillColor] (461.87,406.74) circle (  1.16);

\path[draw=drawColor,line width= 0.4pt,line join=round,line cap=round,fill=fillColor] (462.02,406.72) circle (  1.16);

\path[draw=drawColor,line width= 0.4pt,line join=round,line cap=round,fill=fillColor] (462.18,406.67) circle (  1.16);

\path[draw=drawColor,line width= 0.4pt,line join=round,line cap=round,fill=fillColor] (462.33,406.64) circle (  1.16);

\path[draw=drawColor,line width= 0.4pt,line join=round,line cap=round,fill=fillColor] (462.48,406.51) circle (  1.16);

\path[draw=drawColor,line width= 0.4pt,line join=round,line cap=round,fill=fillColor] (462.63,406.48) circle (  1.16);

\path[draw=drawColor,line width= 0.4pt,line join=round,line cap=round,fill=fillColor] (462.78,406.46) circle (  1.16);

\path[draw=drawColor,line width= 0.4pt,line join=round,line cap=round,fill=fillColor] (462.94,406.41) circle (  1.16);

\path[draw=drawColor,line width= 0.4pt,line join=round,line cap=round,fill=fillColor] (463.09,406.37) circle (  1.16);

\path[draw=drawColor,line width= 0.4pt,line join=round,line cap=round,fill=fillColor] (463.24,406.23) circle (  1.16);

\path[draw=drawColor,line width= 0.4pt,line join=round,line cap=round,fill=fillColor] (463.39,406.20) circle (  1.16);

\path[draw=drawColor,line width= 0.4pt,line join=round,line cap=round,fill=fillColor] (463.54,406.18) circle (  1.16);

\path[draw=drawColor,line width= 0.4pt,line join=round,line cap=round,fill=fillColor] (463.69,406.10) circle (  1.16);

\path[draw=drawColor,line width= 0.4pt,line join=round,line cap=round,fill=fillColor] (463.84,406.02) circle (  1.16);

\path[draw=drawColor,line width= 0.4pt,line join=round,line cap=round,fill=fillColor] (463.99,405.99) circle (  1.16);

\path[draw=drawColor,line width= 0.4pt,line join=round,line cap=round,fill=fillColor] (464.14,405.99) circle (  1.16);

\path[draw=drawColor,line width= 0.4pt,line join=round,line cap=round,fill=fillColor] (464.29,405.89) circle (  1.16);

\path[draw=drawColor,line width= 0.4pt,line join=round,line cap=round,fill=fillColor] (464.44,405.84) circle (  1.16);

\path[draw=drawColor,line width= 0.4pt,line join=round,line cap=round,fill=fillColor] (464.58,405.65) circle (  1.16);

\path[draw=drawColor,line width= 0.4pt,line join=round,line cap=round,fill=fillColor] (464.73,405.64) circle (  1.16);

\path[draw=drawColor,line width= 0.4pt,line join=round,line cap=round,fill=fillColor] (464.88,405.64) circle (  1.16);

\path[draw=drawColor,line width= 0.4pt,line join=round,line cap=round,fill=fillColor] (465.03,405.58) circle (  1.16);

\path[draw=drawColor,line width= 0.4pt,line join=round,line cap=round,fill=fillColor] (465.17,405.57) circle (  1.16);

\path[draw=drawColor,line width= 0.4pt,line join=round,line cap=round,fill=fillColor] (465.32,405.55) circle (  1.16);

\path[draw=drawColor,line width= 0.4pt,line join=round,line cap=round,fill=fillColor] (465.47,405.51) circle (  1.16);

\path[draw=drawColor,line width= 0.4pt,line join=round,line cap=round,fill=fillColor] (465.61,405.48) circle (  1.16);

\path[draw=drawColor,line width= 0.4pt,line join=round,line cap=round,fill=fillColor] (465.76,405.47) circle (  1.16);

\path[draw=drawColor,line width= 0.4pt,line join=round,line cap=round,fill=fillColor] (465.91,405.46) circle (  1.16);

\path[draw=drawColor,line width= 0.4pt,line join=round,line cap=round,fill=fillColor] (466.05,405.39) circle (  1.16);

\path[draw=drawColor,line width= 0.4pt,line join=round,line cap=round,fill=fillColor] (466.20,405.39) circle (  1.16);

\path[draw=drawColor,line width= 0.4pt,line join=round,line cap=round,fill=fillColor] (466.34,405.32) circle (  1.16);

\path[draw=drawColor,line width= 0.4pt,line join=round,line cap=round,fill=fillColor] (466.49,405.16) circle (  1.16);

\path[draw=drawColor,line width= 0.4pt,line join=round,line cap=round,fill=fillColor] (466.63,405.16) circle (  1.16);

\path[draw=drawColor,line width= 0.4pt,line join=round,line cap=round,fill=fillColor] (466.78,405.03) circle (  1.16);

\path[draw=drawColor,line width= 0.4pt,line join=round,line cap=round,fill=fillColor] (466.92,404.97) circle (  1.16);

\path[draw=drawColor,line width= 0.4pt,line join=round,line cap=round,fill=fillColor] (467.06,404.83) circle (  1.16);

\path[draw=drawColor,line width= 0.4pt,line join=round,line cap=round,fill=fillColor] (467.21,404.83) circle (  1.16);

\path[draw=drawColor,line width= 0.4pt,line join=round,line cap=round,fill=fillColor] (467.35,404.80) circle (  1.16);

\path[draw=drawColor,line width= 0.4pt,line join=round,line cap=round,fill=fillColor] (467.49,404.68) circle (  1.16);

\path[draw=drawColor,line width= 0.4pt,line join=round,line cap=round,fill=fillColor] (467.64,404.63) circle (  1.16);

\path[draw=drawColor,line width= 0.4pt,line join=round,line cap=round,fill=fillColor] (467.78,404.58) circle (  1.16);

\path[draw=drawColor,line width= 0.4pt,line join=round,line cap=round,fill=fillColor] (467.92,404.53) circle (  1.16);

\path[draw=drawColor,line width= 0.4pt,line join=round,line cap=round,fill=fillColor] (468.06,404.50) circle (  1.16);

\path[draw=drawColor,line width= 0.4pt,line join=round,line cap=round,fill=fillColor] (468.21,404.48) circle (  1.16);

\path[draw=drawColor,line width= 0.4pt,line join=round,line cap=round,fill=fillColor] (468.35,404.24) circle (  1.16);

\path[draw=drawColor,line width= 0.4pt,line join=round,line cap=round,fill=fillColor] (468.49,404.20) circle (  1.16);

\path[draw=drawColor,line width= 0.4pt,line join=round,line cap=round,fill=fillColor] (468.63,404.13) circle (  1.16);

\path[draw=drawColor,line width= 0.4pt,line join=round,line cap=round,fill=fillColor] (468.77,404.08) circle (  1.16);

\path[draw=drawColor,line width= 0.4pt,line join=round,line cap=round,fill=fillColor] (468.91,404.08) circle (  1.16);

\path[draw=drawColor,line width= 0.4pt,line join=round,line cap=round,fill=fillColor] (469.05,403.84) circle (  1.16);

\path[draw=drawColor,line width= 0.4pt,line join=round,line cap=round,fill=fillColor] (469.19,403.76) circle (  1.16);

\path[draw=drawColor,line width= 0.4pt,line join=round,line cap=round,fill=fillColor] (469.33,403.64) circle (  1.16);

\path[draw=drawColor,line width= 0.4pt,line join=round,line cap=round,fill=fillColor] (469.47,403.64) circle (  1.16);

\path[draw=drawColor,line width= 0.4pt,line join=round,line cap=round,fill=fillColor] (469.61,403.47) circle (  1.16);

\path[draw=drawColor,line width= 0.4pt,line join=round,line cap=round,fill=fillColor] (469.75,403.46) circle (  1.16);

\path[draw=drawColor,line width= 0.4pt,line join=round,line cap=round,fill=fillColor] (469.89,403.39) circle (  1.16);

\path[draw=drawColor,line width= 0.4pt,line join=round,line cap=round,fill=fillColor] (470.03,403.38) circle (  1.16);

\path[draw=drawColor,line width= 0.4pt,line join=round,line cap=round,fill=fillColor] (470.17,403.32) circle (  1.16);

\path[draw=drawColor,line width= 0.4pt,line join=round,line cap=round,fill=fillColor] (470.30,403.31) circle (  1.16);

\path[draw=drawColor,line width= 0.4pt,line join=round,line cap=round,fill=fillColor] (470.44,403.30) circle (  1.16);

\path[draw=drawColor,line width= 0.4pt,line join=round,line cap=round,fill=fillColor] (470.58,402.60) circle (  1.16);

\path[draw=drawColor,line width= 0.4pt,line join=round,line cap=round,fill=fillColor] (470.72,402.59) circle (  1.16);

\path[draw=drawColor,line width= 0.4pt,line join=round,line cap=round,fill=fillColor] (470.85,402.35) circle (  1.16);

\path[draw=drawColor,line width= 0.4pt,line join=round,line cap=round,fill=fillColor] (470.99,402.34) circle (  1.16);

\path[draw=drawColor,line width= 0.4pt,line join=round,line cap=round,fill=fillColor] (471.13,402.23) circle (  1.16);

\path[draw=drawColor,line width= 0.4pt,line join=round,line cap=round,fill=fillColor] (471.27,402.23) circle (  1.16);

\path[draw=drawColor,line width= 0.4pt,line join=round,line cap=round,fill=fillColor] (471.40,402.10) circle (  1.16);

\path[draw=drawColor,line width= 0.4pt,line join=round,line cap=round,fill=fillColor] (471.54,401.63) circle (  1.16);

\path[draw=drawColor,line width= 0.4pt,line join=round,line cap=round,fill=fillColor] (471.67,401.28) circle (  1.16);

\path[draw=drawColor,line width= 0.4pt,line join=round,line cap=round,fill=fillColor] (471.81,401.08) circle (  1.16);

\path[draw=drawColor,line width= 0.4pt,line join=round,line cap=round,fill=fillColor] (471.94,401.07) circle (  1.16);

\path[draw=drawColor,line width= 0.4pt,line join=round,line cap=round,fill=fillColor] (472.08,400.97) circle (  1.16);

\path[draw=drawColor,line width= 0.4pt,line join=round,line cap=round,fill=fillColor] (472.22,400.93) circle (  1.16);

\path[draw=drawColor,line width= 0.4pt,line join=round,line cap=round,fill=fillColor] (472.35,400.72) circle (  1.16);

\path[draw=drawColor,line width= 0.4pt,line join=round,line cap=round,fill=fillColor] (472.48,400.46) circle (  1.16);

\path[draw=drawColor,line width= 0.4pt,line join=round,line cap=round,fill=fillColor] (472.62,400.42) circle (  1.16);

\path[draw=drawColor,line width= 0.4pt,line join=round,line cap=round,fill=fillColor] (472.75,400.15) circle (  1.16);

\path[draw=drawColor,line width= 0.4pt,line join=round,line cap=round,fill=fillColor] (472.89,399.34) circle (  1.16);

\path[draw=drawColor,line width= 0.4pt,line join=round,line cap=round,fill=fillColor] (473.02,399.28) circle (  1.16);

\path[draw=drawColor,line width= 0.4pt,line join=round,line cap=round,fill=fillColor] (473.15,399.16) circle (  1.16);

\path[draw=drawColor,line width= 0.4pt,line join=round,line cap=round,fill=fillColor] (473.29,399.08) circle (  1.16);

\path[draw=drawColor,line width= 0.4pt,line join=round,line cap=round,fill=fillColor] (473.42,398.97) circle (  1.16);

\path[draw=drawColor,line width= 0.4pt,line join=round,line cap=round,fill=fillColor] (473.55,398.73) circle (  1.16);

\path[draw=drawColor,line width= 0.4pt,line join=round,line cap=round,fill=fillColor] (473.69,398.69) circle (  1.16);

\path[draw=drawColor,line width= 0.4pt,line join=round,line cap=round,fill=fillColor] (473.82,398.04) circle (  1.16);

\path[draw=drawColor,line width= 0.4pt,line join=round,line cap=round,fill=fillColor] (473.95,397.72) circle (  1.16);

\path[draw=drawColor,line width= 0.4pt,line join=round,line cap=round,fill=fillColor] (474.08,397.61) circle (  1.16);

\path[draw=drawColor,line width= 0.4pt,line join=round,line cap=round,fill=fillColor] (474.22,397.60) circle (  1.16);

\path[draw=drawColor,line width= 0.4pt,line join=round,line cap=round,fill=fillColor] (474.35,397.52) circle (  1.16);

\path[draw=drawColor,line width= 0.4pt,line join=round,line cap=round,fill=fillColor] (474.48,397.43) circle (  1.16);

\path[draw=drawColor,line width= 0.4pt,line join=round,line cap=round,fill=fillColor] (474.61,397.32) circle (  1.16);

\path[draw=drawColor,line width= 0.4pt,line join=round,line cap=round,fill=fillColor] (474.74,397.22) circle (  1.16);

\path[draw=drawColor,line width= 0.4pt,line join=round,line cap=round,fill=fillColor] (474.87,397.20) circle (  1.16);

\path[draw=drawColor,line width= 0.4pt,line join=round,line cap=round,fill=fillColor] (475.00,397.04) circle (  1.16);

\path[draw=drawColor,line width= 0.4pt,line join=round,line cap=round,fill=fillColor] (475.13,396.42) circle (  1.16);

\path[draw=drawColor,line width= 0.4pt,line join=round,line cap=round,fill=fillColor] (475.26,396.42) circle (  1.16);

\path[draw=drawColor,line width= 0.4pt,line join=round,line cap=round,fill=fillColor] (475.39,395.79) circle (  1.16);

\path[draw=drawColor,line width= 0.4pt,line join=round,line cap=round,fill=fillColor] (475.52,395.62) circle (  1.16);

\path[draw=drawColor,line width= 0.4pt,line join=round,line cap=round,fill=fillColor] (475.65,395.48) circle (  1.16);

\path[draw=drawColor,line width= 0.4pt,line join=round,line cap=round,fill=fillColor] (475.78,395.26) circle (  1.16);

\path[draw=drawColor,line width= 0.4pt,line join=round,line cap=round,fill=fillColor] (475.91,395.05) circle (  1.16);

\path[draw=drawColor,line width= 0.4pt,line join=round,line cap=round,fill=fillColor] (476.04,392.71) circle (  1.16);

\path[draw=drawColor,line width= 0.4pt,line join=round,line cap=round,fill=fillColor] (476.17,392.09) circle (  1.16);

\path[draw=drawColor,line width= 0.4pt,line join=round,line cap=round,fill=fillColor] (476.30,391.77) circle (  1.16);

\path[draw=drawColor,line width= 0.4pt,line join=round,line cap=round,fill=fillColor] (476.43,391.62) circle (  1.16);

\path[draw=drawColor,line width= 0.4pt,line join=round,line cap=round,fill=fillColor] (476.56,391.51) circle (  1.16);

\path[draw=drawColor,line width= 0.4pt,line join=round,line cap=round,fill=fillColor] (476.68,390.42) circle (  1.16);

\path[draw=drawColor,line width= 0.4pt,line join=round,line cap=round,fill=fillColor] (476.81,388.59) circle (  1.16);

\path[draw=drawColor,line width= 0.4pt,line join=round,line cap=round,fill=fillColor] (476.94,385.94) circle (  1.16);

\path[draw=drawColor,line width= 0.4pt,line join=round,line cap=round,fill=fillColor] (477.07,382.15) circle (  1.16);

\path[draw=drawColor,line width= 0.4pt,line join=round,line cap=round,fill=fillColor] (477.19,381.07) circle (  1.16);

\path[draw=drawColor,line width= 0.4pt,line join=round,line cap=round,fill=fillColor] (477.32,372.42) circle (  1.16);

\path[draw=drawColor,line width= 0.4pt,line join=round,line cap=round,fill=fillColor] (477.45,372.42) circle (  1.16);

\path[draw=drawColor,line width= 0.4pt,line join=round,line cap=round,fill=fillColor] (477.57,372.42) circle (  1.16);

\path[draw=drawColor,line width= 0.4pt,line join=round,line cap=round,fill=fillColor] (477.70,372.42) circle (  1.16);

\path[draw=drawColor,line width= 0.4pt,line join=round,line cap=round,fill=fillColor] (477.83,372.42) circle (  1.16);

\path[draw=drawColor,line width= 0.4pt,line join=round,line cap=round,fill=fillColor] (477.95,372.42) circle (  1.16);

\path[draw=drawColor,line width= 0.4pt,line join=round,line cap=round,fill=fillColor] (478.08,372.42) circle (  1.16);

\path[draw=drawColor,line width= 0.4pt,line join=round,line cap=round,fill=fillColor] (478.21,372.42) circle (  1.16);
\definecolor[named]{drawColor}{rgb}{0.00,0.00,0.00}
\definecolor[named]{fillColor}{rgb}{0.00,0.00,0.00}

\path[draw=drawColor,line width= 0.6pt,line join=round,fill=fillColor] (320.31,455.38) -- (485.72,455.38);

\node[text=drawColor,anchor=base east,inner sep=0pt, outer sep=0pt, scale=  0.85] at (482.22,457.54) {infeasible solutions};

\path[draw=drawColor,line width= 0.6pt,line join=round,line cap=round] (320.31,364.12) rectangle (485.72,466.36);
\end{scope}
\begin{scope}
\path[clip] (  0.00,  0.00) rectangle (505.89,650.43);
\definecolor[named]{drawColor}{rgb}{0.00,0.00,0.00}

\node[text=drawColor,anchor=base east,inner sep=0pt, outer sep=0pt, scale=  0.80] at (314.91,369.66) {0.00};

\node[text=drawColor,anchor=base east,inner sep=0pt, outer sep=0pt, scale=  0.80] at (314.91,387.54) {0.01};

\node[text=drawColor,anchor=base east,inner sep=0pt, outer sep=0pt, scale=  0.80] at (314.91,400.23) {0.05};

\node[text=drawColor,anchor=base east,inner sep=0pt, outer sep=0pt, scale=  0.80] at (314.91,408.17) {0.10};

\node[text=drawColor,anchor=base east,inner sep=0pt, outer sep=0pt, scale=  0.80] at (314.91,418.18) {0.20};

\node[text=drawColor,anchor=base east,inner sep=0pt, outer sep=0pt, scale=  0.80] at (314.91,430.79) {0.40};

\node[text=drawColor,anchor=base east,inner sep=0pt, outer sep=0pt, scale=  0.80] at (314.91,439.64) {0.60};

\node[text=drawColor,anchor=base east,inner sep=0pt, outer sep=0pt, scale=  0.80] at (314.91,446.68) {0.80};

\node[text=drawColor,anchor=base east,inner sep=0pt, outer sep=0pt, scale=  0.80] at (314.91,452.63) {1.00};
\end{scope}
\begin{scope}
\path[clip] (  0.00,  0.00) rectangle (505.89,650.43);
\definecolor[named]{drawColor}{rgb}{0.00,0.00,0.00}

\path[draw=drawColor,line width= 0.6pt,line join=round] (317.31,372.42) --
	(320.31,372.42);

\path[draw=drawColor,line width= 0.6pt,line join=round] (317.31,390.29) --
	(320.31,390.29);

\path[draw=drawColor,line width= 0.6pt,line join=round] (317.31,402.98) --
	(320.31,402.98);

\path[draw=drawColor,line width= 0.6pt,line join=round] (317.31,410.93) --
	(320.31,410.93);

\path[draw=drawColor,line width= 0.6pt,line join=round] (317.31,420.94) --
	(320.31,420.94);

\path[draw=drawColor,line width= 0.6pt,line join=round] (317.31,433.55) --
	(320.31,433.55);

\path[draw=drawColor,line width= 0.6pt,line join=round] (317.31,442.39) --
	(320.31,442.39);

\path[draw=drawColor,line width= 0.6pt,line join=round] (317.31,449.44) --
	(320.31,449.44);

\path[draw=drawColor,line width= 0.6pt,line join=round] (317.31,455.38) --
	(320.31,455.38);
\end{scope}
\begin{scope}
\path[clip] (  0.00,  0.00) rectangle (505.89,650.43);
\definecolor[named]{drawColor}{rgb}{0.00,0.00,0.00}

\path[draw=drawColor,line width= 0.6pt,line join=round] (409.46,361.12) --
	(409.46,364.12);

\path[draw=drawColor,line width= 0.6pt,line join=round] (436.50,361.12) --
	(436.50,364.12);

\path[draw=drawColor,line width= 0.6pt,line join=round] (455.48,361.12) --
	(455.48,364.12);

\path[draw=drawColor,line width= 0.6pt,line join=round] (470.58,361.12) --
	(470.58,364.12);

\path[draw=drawColor,line width= 0.6pt,line join=round] (483.33,361.12) --
	(483.33,364.12);
\end{scope}
\begin{scope}
\path[clip] (  0.00,  0.00) rectangle (505.89,650.43);
\definecolor[named]{drawColor}{rgb}{0.00,0.00,0.00}

\node[text=drawColor,rotate= 50.00,anchor=base east,inner sep=0pt, outer sep=0pt, scale=  0.80] at (413.68,355.18) {100};

\node[text=drawColor,rotate= 50.00,anchor=base east,inner sep=0pt, outer sep=0pt, scale=  0.80] at (440.73,355.18) {200};

\node[text=drawColor,rotate= 50.00,anchor=base east,inner sep=0pt, outer sep=0pt, scale=  0.80] at (459.70,355.18) {300};

\node[text=drawColor,rotate= 50.00,anchor=base east,inner sep=0pt, outer sep=0pt, scale=  0.80] at (474.80,355.18) {400};

\node[text=drawColor,rotate= 50.00,anchor=base east,inner sep=0pt, outer sep=0pt, scale=  0.80] at (487.55,355.18) {500};
\end{scope}
\begin{scope}
\path[clip] (  0.00,  0.00) rectangle (505.89,650.43);
\definecolor[named]{drawColor}{rgb}{0.00,0.00,0.00}

\node[text=drawColor,anchor=base,inner sep=0pt, outer sep=0pt, scale=  1.10] at (403.01,333.61) {\# Instances};
\end{scope}
\begin{scope}
\path[clip] (  0.00,  0.00) rectangle (505.89,650.43);
\definecolor[named]{drawColor}{rgb}{0.00,0.00,0.00}

\node[text=drawColor,rotate= 90.00,anchor=base,inner sep=0pt, outer sep=0pt, scale=  1.10] at (290.69,415.24) {1-(Best/Algorithm)};
\end{scope}
\begin{scope}
\path[clip] (  0.00,  0.00) rectangle (505.89,650.43);
\definecolor[named]{drawColor}{rgb}{0.00,0.00,0.00}

\node[text=drawColor,anchor=base,inner sep=0pt, outer sep=0pt, scale=  1.20] at (403.01,473.56) {$k=16$};
\end{scope}
\begin{scope}
\path[clip] ( 14.17,162.61) rectangle (238.78,325.21);
\definecolor[named]{drawColor}{rgb}{1.00,1.00,1.00}
\definecolor[named]{fillColor}{rgb}{1.00,1.00,1.00}

\path[draw=drawColor,line width= 0.6pt,line join=round,line cap=round,fill=fillColor] ( 14.17,162.61) rectangle (238.78,325.21);
\end{scope}
\begin{scope}
\path[clip] ( 67.36,201.52) rectangle (232.78,303.75);
\definecolor[named]{fillColor}{rgb}{1.00,1.00,1.00}

\path[fill=fillColor] ( 67.36,201.52) rectangle (232.78,303.75);
\definecolor[named]{drawColor}{rgb}{0.98,0.98,0.98}

\path[draw=drawColor,line width= 0.6pt,line join=round] ( 67.36,218.75) --
	(232.78,218.75);

\path[draw=drawColor,line width= 0.6pt,line join=round] ( 67.36,234.03) --
	(232.78,234.03);

\path[draw=drawColor,line width= 0.6pt,line join=round] ( 67.36,244.35) --
	(232.78,244.35);

\path[draw=drawColor,line width= 0.6pt,line join=round] ( 67.36,253.32) --
	(232.78,253.32);

\path[draw=drawColor,line width= 0.6pt,line join=round] ( 67.36,264.63) --
	(232.78,264.63);

\path[draw=drawColor,line width= 0.6pt,line join=round] ( 67.36,275.36) --
	(232.78,275.36);

\path[draw=drawColor,line width= 0.6pt,line join=round] ( 67.36,283.31) --
	(232.78,283.31);

\path[draw=drawColor,line width= 0.6pt,line join=round] ( 67.36,289.80) --
	(232.78,289.80);

\path[draw=drawColor,line width= 0.6pt,line join=round] ( 67.36,301.71) --
	(232.78,301.71);

\path[draw=drawColor,line width= 0.6pt,line join=round] (129.75,201.52) --
	(129.75,303.75);

\path[draw=drawColor,line width= 0.6pt,line join=round] (143.34,201.52) --
	(143.34,303.75);

\path[draw=drawColor,line width= 0.6pt,line join=round] (170.52,201.52) --
	(170.52,303.75);

\path[draw=drawColor,line width= 0.6pt,line join=round] (193.64,201.52) --
	(193.64,303.75);

\path[draw=drawColor,line width= 0.6pt,line join=round] (210.77,201.52) --
	(210.77,303.75);

\path[draw=drawColor,line width= 0.6pt,line join=round] (224.77,201.52) --
	(224.77,303.75);
\definecolor[named]{drawColor}{rgb}{0.75,0.75,0.75}

\path[draw=drawColor,line width= 0.6pt,dash pattern=on 1pt off 3pt ,line join=round] ( 67.36,209.81) --
	(232.78,209.81);

\path[draw=drawColor,line width= 0.6pt,dash pattern=on 1pt off 3pt ,line join=round] ( 67.36,227.69) --
	(232.78,227.69);

\path[draw=drawColor,line width= 0.6pt,dash pattern=on 1pt off 3pt ,line join=round] ( 67.36,240.38) --
	(232.78,240.38);

\path[draw=drawColor,line width= 0.6pt,dash pattern=on 1pt off 3pt ,line join=round] ( 67.36,248.32) --
	(232.78,248.32);

\path[draw=drawColor,line width= 0.6pt,dash pattern=on 1pt off 3pt ,line join=round] ( 67.36,258.33) --
	(232.78,258.33);

\path[draw=drawColor,line width= 0.6pt,dash pattern=on 1pt off 3pt ,line join=round] ( 67.36,270.94) --
	(232.78,270.94);

\path[draw=drawColor,line width= 0.6pt,dash pattern=on 1pt off 3pt ,line join=round] ( 67.36,279.79) --
	(232.78,279.79);

\path[draw=drawColor,line width= 0.6pt,dash pattern=on 1pt off 3pt ,line join=round] ( 67.36,286.83) --
	(232.78,286.83);

\path[draw=drawColor,line width= 0.6pt,dash pattern=on 1pt off 3pt ,line join=round] ( 67.36,292.78) --
	(232.78,292.78);

\path[draw=drawColor,line width= 0.6pt,dash pattern=on 1pt off 3pt ,line join=round] (156.93,201.52) --
	(156.93,303.75);

\path[draw=drawColor,line width= 0.6pt,dash pattern=on 1pt off 3pt ,line join=round] (184.11,201.52) --
	(184.11,303.75);

\path[draw=drawColor,line width= 0.6pt,dash pattern=on 1pt off 3pt ,line join=round] (203.18,201.52) --
	(203.18,303.75);

\path[draw=drawColor,line width= 0.6pt,dash pattern=on 1pt off 3pt ,line join=round] (218.36,201.52) --
	(218.36,303.75);

\path[draw=drawColor,line width= 0.6pt,dash pattern=on 1pt off 3pt ,line join=round] (231.18,201.52) --
	(231.18,303.75);
\definecolor[named]{drawColor}{rgb}{0.89,0.10,0.11}
\definecolor[named]{fillColor}{rgb}{0.89,0.10,0.11}

\path[draw=drawColor,line width= 0.4pt,line join=round,line cap=round,fill=fillColor] ( 74.88,262.55) circle (  1.16);

\path[draw=drawColor,line width= 0.4pt,line join=round,line cap=round,fill=fillColor] ( 80.74,261.89) circle (  1.16);

\path[draw=drawColor,line width= 0.4pt,line join=round,line cap=round,fill=fillColor] ( 84.84,259.66) circle (  1.16);

\path[draw=drawColor,line width= 0.4pt,line join=round,line cap=round,fill=fillColor] ( 88.11,255.67) circle (  1.16);

\path[draw=drawColor,line width= 0.4pt,line join=round,line cap=round,fill=fillColor] ( 90.88,253.84) circle (  1.16);

\path[draw=drawColor,line width= 0.4pt,line join=round,line cap=round,fill=fillColor] ( 93.29,253.64) circle (  1.16);

\path[draw=drawColor,line width= 0.4pt,line join=round,line cap=round,fill=fillColor] ( 95.45,252.95) circle (  1.16);

\path[draw=drawColor,line width= 0.4pt,line join=round,line cap=round,fill=fillColor] ( 97.41,251.33) circle (  1.16);

\path[draw=drawColor,line width= 0.4pt,line join=round,line cap=round,fill=fillColor] ( 99.21,251.32) circle (  1.16);

\path[draw=drawColor,line width= 0.4pt,line join=round,line cap=round,fill=fillColor] (100.89,250.86) circle (  1.16);

\path[draw=drawColor,line width= 0.4pt,line join=round,line cap=round,fill=fillColor] (102.46,250.73) circle (  1.16);

\path[draw=drawColor,line width= 0.4pt,line join=round,line cap=round,fill=fillColor] (103.93,250.07) circle (  1.16);

\path[draw=drawColor,line width= 0.4pt,line join=round,line cap=round,fill=fillColor] (105.33,248.26) circle (  1.16);

\path[draw=drawColor,line width= 0.4pt,line join=round,line cap=round,fill=fillColor] (106.65,247.86) circle (  1.16);

\path[draw=drawColor,line width= 0.4pt,line join=round,line cap=round,fill=fillColor] (107.91,247.71) circle (  1.16);

\path[draw=drawColor,line width= 0.4pt,line join=round,line cap=round,fill=fillColor] (109.12,247.08) circle (  1.16);

\path[draw=drawColor,line width= 0.4pt,line join=round,line cap=round,fill=fillColor] (110.28,247.05) circle (  1.16);

\path[draw=drawColor,line width= 0.4pt,line join=round,line cap=round,fill=fillColor] (111.40,246.82) circle (  1.16);

\path[draw=drawColor,line width= 0.4pt,line join=round,line cap=round,fill=fillColor] (112.47,246.65) circle (  1.16);

\path[draw=drawColor,line width= 0.4pt,line join=round,line cap=round,fill=fillColor] (113.51,246.55) circle (  1.16);

\path[draw=drawColor,line width= 0.4pt,line join=round,line cap=round,fill=fillColor] (114.51,246.38) circle (  1.16);

\path[draw=drawColor,line width= 0.4pt,line join=round,line cap=round,fill=fillColor] (115.48,246.02) circle (  1.16);

\path[draw=drawColor,line width= 0.4pt,line join=round,line cap=round,fill=fillColor] (116.42,245.96) circle (  1.16);

\path[draw=drawColor,line width= 0.4pt,line join=round,line cap=round,fill=fillColor] (117.34,245.37) circle (  1.16);

\path[draw=drawColor,line width= 0.4pt,line join=round,line cap=round,fill=fillColor] (118.23,245.23) circle (  1.16);

\path[draw=drawColor,line width= 0.4pt,line join=round,line cap=round,fill=fillColor] (119.10,245.20) circle (  1.16);

\path[draw=drawColor,line width= 0.4pt,line join=round,line cap=round,fill=fillColor] (119.94,244.81) circle (  1.16);

\path[draw=drawColor,line width= 0.4pt,line join=round,line cap=round,fill=fillColor] (120.77,244.32) circle (  1.16);

\path[draw=drawColor,line width= 0.4pt,line join=round,line cap=round,fill=fillColor] (121.57,243.93) circle (  1.16);

\path[draw=drawColor,line width= 0.4pt,line join=round,line cap=round,fill=fillColor] (122.36,243.93) circle (  1.16);

\path[draw=drawColor,line width= 0.4pt,line join=round,line cap=round,fill=fillColor] (123.13,243.87) circle (  1.16);

\path[draw=drawColor,line width= 0.4pt,line join=round,line cap=round,fill=fillColor] (123.88,243.78) circle (  1.16);

\path[draw=drawColor,line width= 0.4pt,line join=round,line cap=round,fill=fillColor] (124.62,243.77) circle (  1.16);

\path[draw=drawColor,line width= 0.4pt,line join=round,line cap=round,fill=fillColor] (125.34,243.68) circle (  1.16);

\path[draw=drawColor,line width= 0.4pt,line join=round,line cap=round,fill=fillColor] (126.05,243.48) circle (  1.16);

\path[draw=drawColor,line width= 0.4pt,line join=round,line cap=round,fill=fillColor] (126.74,243.35) circle (  1.16);

\path[draw=drawColor,line width= 0.4pt,line join=round,line cap=round,fill=fillColor] (127.43,243.19) circle (  1.16);

\path[draw=drawColor,line width= 0.4pt,line join=round,line cap=round,fill=fillColor] (128.10,242.85) circle (  1.16);

\path[draw=drawColor,line width= 0.4pt,line join=round,line cap=round,fill=fillColor] (128.76,242.79) circle (  1.16);

\path[draw=drawColor,line width= 0.4pt,line join=round,line cap=round,fill=fillColor] (129.40,242.72) circle (  1.16);

\path[draw=drawColor,line width= 0.4pt,line join=round,line cap=round,fill=fillColor] (130.04,242.34) circle (  1.16);

\path[draw=drawColor,line width= 0.4pt,line join=round,line cap=round,fill=fillColor] (130.67,242.34) circle (  1.16);

\path[draw=drawColor,line width= 0.4pt,line join=round,line cap=round,fill=fillColor] (131.28,242.29) circle (  1.16);

\path[draw=drawColor,line width= 0.4pt,line join=round,line cap=round,fill=fillColor] (131.89,242.20) circle (  1.16);

\path[draw=drawColor,line width= 0.4pt,line join=round,line cap=round,fill=fillColor] (132.49,242.12) circle (  1.16);

\path[draw=drawColor,line width= 0.4pt,line join=round,line cap=round,fill=fillColor] (133.08,242.07) circle (  1.16);

\path[draw=drawColor,line width= 0.4pt,line join=round,line cap=round,fill=fillColor] (133.66,242.02) circle (  1.16);

\path[draw=drawColor,line width= 0.4pt,line join=round,line cap=round,fill=fillColor] (134.23,241.98) circle (  1.16);

\path[draw=drawColor,line width= 0.4pt,line join=round,line cap=round,fill=fillColor] (134.80,241.86) circle (  1.16);

\path[draw=drawColor,line width= 0.4pt,line join=round,line cap=round,fill=fillColor] (135.35,241.81) circle (  1.16);

\path[draw=drawColor,line width= 0.4pt,line join=round,line cap=round,fill=fillColor] (135.90,241.74) circle (  1.16);

\path[draw=drawColor,line width= 0.4pt,line join=round,line cap=round,fill=fillColor] (136.45,241.71) circle (  1.16);

\path[draw=drawColor,line width= 0.4pt,line join=round,line cap=round,fill=fillColor] (136.98,241.63) circle (  1.16);

\path[draw=drawColor,line width= 0.4pt,line join=round,line cap=round,fill=fillColor] (137.51,241.39) circle (  1.16);

\path[draw=drawColor,line width= 0.4pt,line join=round,line cap=round,fill=fillColor] (138.03,241.21) circle (  1.16);

\path[draw=drawColor,line width= 0.4pt,line join=round,line cap=round,fill=fillColor] (138.55,241.14) circle (  1.16);

\path[draw=drawColor,line width= 0.4pt,line join=round,line cap=round,fill=fillColor] (139.06,241.09) circle (  1.16);

\path[draw=drawColor,line width= 0.4pt,line join=round,line cap=round,fill=fillColor] (139.56,241.03) circle (  1.16);

\path[draw=drawColor,line width= 0.4pt,line join=round,line cap=round,fill=fillColor] (140.06,240.97) circle (  1.16);

\path[draw=drawColor,line width= 0.4pt,line join=round,line cap=round,fill=fillColor] (140.55,240.92) circle (  1.16);

\path[draw=drawColor,line width= 0.4pt,line join=round,line cap=round,fill=fillColor] (141.04,240.90) circle (  1.16);

\path[draw=drawColor,line width= 0.4pt,line join=round,line cap=round,fill=fillColor] (141.52,240.89) circle (  1.16);

\path[draw=drawColor,line width= 0.4pt,line join=round,line cap=round,fill=fillColor] (142.00,240.58) circle (  1.16);

\path[draw=drawColor,line width= 0.4pt,line join=round,line cap=round,fill=fillColor] (142.47,240.46) circle (  1.16);

\path[draw=drawColor,line width= 0.4pt,line join=round,line cap=round,fill=fillColor] (142.94,240.46) circle (  1.16);

\path[draw=drawColor,line width= 0.4pt,line join=round,line cap=round,fill=fillColor] (143.40,240.29) circle (  1.16);

\path[draw=drawColor,line width= 0.4pt,line join=round,line cap=round,fill=fillColor] (143.86,240.14) circle (  1.16);

\path[draw=drawColor,line width= 0.4pt,line join=round,line cap=round,fill=fillColor] (144.31,240.04) circle (  1.16);

\path[draw=drawColor,line width= 0.4pt,line join=round,line cap=round,fill=fillColor] (144.76,240.01) circle (  1.16);

\path[draw=drawColor,line width= 0.4pt,line join=round,line cap=round,fill=fillColor] (145.20,240.00) circle (  1.16);

\path[draw=drawColor,line width= 0.4pt,line join=round,line cap=round,fill=fillColor] (145.64,239.90) circle (  1.16);

\path[draw=drawColor,line width= 0.4pt,line join=round,line cap=round,fill=fillColor] (146.08,239.70) circle (  1.16);

\path[draw=drawColor,line width= 0.4pt,line join=round,line cap=round,fill=fillColor] (146.51,239.67) circle (  1.16);

\path[draw=drawColor,line width= 0.4pt,line join=round,line cap=round,fill=fillColor] (146.94,239.64) circle (  1.16);

\path[draw=drawColor,line width= 0.4pt,line join=round,line cap=round,fill=fillColor] (147.36,239.58) circle (  1.16);

\path[draw=drawColor,line width= 0.4pt,line join=round,line cap=round,fill=fillColor] (147.79,239.58) circle (  1.16);

\path[draw=drawColor,line width= 0.4pt,line join=round,line cap=round,fill=fillColor] (148.20,239.55) circle (  1.16);

\path[draw=drawColor,line width= 0.4pt,line join=round,line cap=round,fill=fillColor] (148.62,239.28) circle (  1.16);

\path[draw=drawColor,line width= 0.4pt,line join=round,line cap=round,fill=fillColor] (149.03,239.10) circle (  1.16);

\path[draw=drawColor,line width= 0.4pt,line join=round,line cap=round,fill=fillColor] (149.43,239.07) circle (  1.16);

\path[draw=drawColor,line width= 0.4pt,line join=round,line cap=round,fill=fillColor] (149.83,239.03) circle (  1.16);

\path[draw=drawColor,line width= 0.4pt,line join=round,line cap=round,fill=fillColor] (150.23,238.97) circle (  1.16);

\path[draw=drawColor,line width= 0.4pt,line join=round,line cap=round,fill=fillColor] (150.63,238.90) circle (  1.16);

\path[draw=drawColor,line width= 0.4pt,line join=round,line cap=round,fill=fillColor] (151.02,238.88) circle (  1.16);

\path[draw=drawColor,line width= 0.4pt,line join=round,line cap=round,fill=fillColor] (151.41,238.84) circle (  1.16);

\path[draw=drawColor,line width= 0.4pt,line join=round,line cap=round,fill=fillColor] (151.80,238.84) circle (  1.16);

\path[draw=drawColor,line width= 0.4pt,line join=round,line cap=round,fill=fillColor] (152.18,238.77) circle (  1.16);

\path[draw=drawColor,line width= 0.4pt,line join=round,line cap=round,fill=fillColor] (152.57,238.61) circle (  1.16);

\path[draw=drawColor,line width= 0.4pt,line join=round,line cap=round,fill=fillColor] (152.94,238.60) circle (  1.16);

\path[draw=drawColor,line width= 0.4pt,line join=round,line cap=round,fill=fillColor] (153.32,238.40) circle (  1.16);

\path[draw=drawColor,line width= 0.4pt,line join=round,line cap=round,fill=fillColor] (153.69,238.40) circle (  1.16);

\path[draw=drawColor,line width= 0.4pt,line join=round,line cap=round,fill=fillColor] (154.06,238.35) circle (  1.16);

\path[draw=drawColor,line width= 0.4pt,line join=round,line cap=round,fill=fillColor] (154.43,238.21) circle (  1.16);

\path[draw=drawColor,line width= 0.4pt,line join=round,line cap=round,fill=fillColor] (154.79,238.06) circle (  1.16);

\path[draw=drawColor,line width= 0.4pt,line join=round,line cap=round,fill=fillColor] (155.15,238.00) circle (  1.16);

\path[draw=drawColor,line width= 0.4pt,line join=round,line cap=round,fill=fillColor] (155.51,237.94) circle (  1.16);

\path[draw=drawColor,line width= 0.4pt,line join=round,line cap=round,fill=fillColor] (155.87,237.75) circle (  1.16);

\path[draw=drawColor,line width= 0.4pt,line join=round,line cap=round,fill=fillColor] (156.23,237.64) circle (  1.16);

\path[draw=drawColor,line width= 0.4pt,line join=round,line cap=round,fill=fillColor] (156.58,237.62) circle (  1.16);

\path[draw=drawColor,line width= 0.4pt,line join=round,line cap=round,fill=fillColor] (156.93,237.61) circle (  1.16);

\path[draw=drawColor,line width= 0.4pt,line join=round,line cap=round,fill=fillColor] (157.28,237.58) circle (  1.16);

\path[draw=drawColor,line width= 0.4pt,line join=round,line cap=round,fill=fillColor] (157.62,237.38) circle (  1.16);

\path[draw=drawColor,line width= 0.4pt,line join=round,line cap=round,fill=fillColor] (157.96,237.30) circle (  1.16);

\path[draw=drawColor,line width= 0.4pt,line join=round,line cap=round,fill=fillColor] (158.30,237.29) circle (  1.16);

\path[draw=drawColor,line width= 0.4pt,line join=round,line cap=round,fill=fillColor] (158.64,237.28) circle (  1.16);

\path[draw=drawColor,line width= 0.4pt,line join=round,line cap=round,fill=fillColor] (158.98,237.10) circle (  1.16);

\path[draw=drawColor,line width= 0.4pt,line join=round,line cap=round,fill=fillColor] (159.31,237.09) circle (  1.16);

\path[draw=drawColor,line width= 0.4pt,line join=round,line cap=round,fill=fillColor] (159.65,237.00) circle (  1.16);

\path[draw=drawColor,line width= 0.4pt,line join=round,line cap=round,fill=fillColor] (159.98,236.82) circle (  1.16);

\path[draw=drawColor,line width= 0.4pt,line join=round,line cap=round,fill=fillColor] (160.30,236.80) circle (  1.16);

\path[draw=drawColor,line width= 0.4pt,line join=round,line cap=round,fill=fillColor] (160.63,236.74) circle (  1.16);

\path[draw=drawColor,line width= 0.4pt,line join=round,line cap=round,fill=fillColor] (160.95,236.63) circle (  1.16);

\path[draw=drawColor,line width= 0.4pt,line join=round,line cap=round,fill=fillColor] (161.28,236.63) circle (  1.16);

\path[draw=drawColor,line width= 0.4pt,line join=round,line cap=round,fill=fillColor] (161.60,236.63) circle (  1.16);

\path[draw=drawColor,line width= 0.4pt,line join=round,line cap=round,fill=fillColor] (161.92,236.52) circle (  1.16);

\path[draw=drawColor,line width= 0.4pt,line join=round,line cap=round,fill=fillColor] (162.23,236.40) circle (  1.16);

\path[draw=drawColor,line width= 0.4pt,line join=round,line cap=round,fill=fillColor] (162.55,236.40) circle (  1.16);

\path[draw=drawColor,line width= 0.4pt,line join=round,line cap=round,fill=fillColor] (162.86,236.35) circle (  1.16);

\path[draw=drawColor,line width= 0.4pt,line join=round,line cap=round,fill=fillColor] (163.17,236.30) circle (  1.16);

\path[draw=drawColor,line width= 0.4pt,line join=round,line cap=round,fill=fillColor] (163.48,236.24) circle (  1.16);

\path[draw=drawColor,line width= 0.4pt,line join=round,line cap=round,fill=fillColor] (163.79,236.22) circle (  1.16);

\path[draw=drawColor,line width= 0.4pt,line join=round,line cap=round,fill=fillColor] (164.09,236.20) circle (  1.16);

\path[draw=drawColor,line width= 0.4pt,line join=round,line cap=round,fill=fillColor] (164.40,236.15) circle (  1.16);

\path[draw=drawColor,line width= 0.4pt,line join=round,line cap=round,fill=fillColor] (164.70,236.02) circle (  1.16);

\path[draw=drawColor,line width= 0.4pt,line join=round,line cap=round,fill=fillColor] (165.00,236.01) circle (  1.16);

\path[draw=drawColor,line width= 0.4pt,line join=round,line cap=round,fill=fillColor] (165.30,235.96) circle (  1.16);

\path[draw=drawColor,line width= 0.4pt,line join=round,line cap=round,fill=fillColor] (165.60,235.94) circle (  1.16);

\path[draw=drawColor,line width= 0.4pt,line join=round,line cap=round,fill=fillColor] (165.90,235.90) circle (  1.16);

\path[draw=drawColor,line width= 0.4pt,line join=round,line cap=round,fill=fillColor] (166.19,235.87) circle (  1.16);

\path[draw=drawColor,line width= 0.4pt,line join=round,line cap=round,fill=fillColor] (166.49,235.79) circle (  1.16);

\path[draw=drawColor,line width= 0.4pt,line join=round,line cap=round,fill=fillColor] (166.78,235.77) circle (  1.16);

\path[draw=drawColor,line width= 0.4pt,line join=round,line cap=round,fill=fillColor] (167.07,235.67) circle (  1.16);

\path[draw=drawColor,line width= 0.4pt,line join=round,line cap=round,fill=fillColor] (167.36,235.64) circle (  1.16);

\path[draw=drawColor,line width= 0.4pt,line join=round,line cap=round,fill=fillColor] (167.64,235.57) circle (  1.16);

\path[draw=drawColor,line width= 0.4pt,line join=round,line cap=round,fill=fillColor] (167.93,235.55) circle (  1.16);

\path[draw=drawColor,line width= 0.4pt,line join=round,line cap=round,fill=fillColor] (168.22,235.49) circle (  1.16);

\path[draw=drawColor,line width= 0.4pt,line join=round,line cap=round,fill=fillColor] (168.50,235.47) circle (  1.16);

\path[draw=drawColor,line width= 0.4pt,line join=round,line cap=round,fill=fillColor] (168.78,235.45) circle (  1.16);

\path[draw=drawColor,line width= 0.4pt,line join=round,line cap=round,fill=fillColor] (169.06,235.44) circle (  1.16);

\path[draw=drawColor,line width= 0.4pt,line join=round,line cap=round,fill=fillColor] (169.34,235.38) circle (  1.16);

\path[draw=drawColor,line width= 0.4pt,line join=round,line cap=round,fill=fillColor] (169.62,235.30) circle (  1.16);

\path[draw=drawColor,line width= 0.4pt,line join=round,line cap=round,fill=fillColor] (169.89,235.27) circle (  1.16);

\path[draw=drawColor,line width= 0.4pt,line join=round,line cap=round,fill=fillColor] (170.17,235.21) circle (  1.16);

\path[draw=drawColor,line width= 0.4pt,line join=round,line cap=round,fill=fillColor] (170.44,235.19) circle (  1.16);

\path[draw=drawColor,line width= 0.4pt,line join=round,line cap=round,fill=fillColor] (170.72,235.18) circle (  1.16);

\path[draw=drawColor,line width= 0.4pt,line join=round,line cap=round,fill=fillColor] (170.99,235.16) circle (  1.16);

\path[draw=drawColor,line width= 0.4pt,line join=round,line cap=round,fill=fillColor] (171.26,235.13) circle (  1.16);

\path[draw=drawColor,line width= 0.4pt,line join=round,line cap=round,fill=fillColor] (171.53,235.09) circle (  1.16);

\path[draw=drawColor,line width= 0.4pt,line join=round,line cap=round,fill=fillColor] (171.80,235.01) circle (  1.16);

\path[draw=drawColor,line width= 0.4pt,line join=round,line cap=round,fill=fillColor] (172.06,234.98) circle (  1.16);

\path[draw=drawColor,line width= 0.4pt,line join=round,line cap=round,fill=fillColor] (172.33,234.94) circle (  1.16);

\path[draw=drawColor,line width= 0.4pt,line join=round,line cap=round,fill=fillColor] (172.59,234.92) circle (  1.16);

\path[draw=drawColor,line width= 0.4pt,line join=round,line cap=round,fill=fillColor] (172.85,234.91) circle (  1.16);

\path[draw=drawColor,line width= 0.4pt,line join=round,line cap=round,fill=fillColor] (173.12,234.80) circle (  1.16);

\path[draw=drawColor,line width= 0.4pt,line join=round,line cap=round,fill=fillColor] (173.38,234.71) circle (  1.16);

\path[draw=drawColor,line width= 0.4pt,line join=round,line cap=round,fill=fillColor] (173.64,234.50) circle (  1.16);

\path[draw=drawColor,line width= 0.4pt,line join=round,line cap=round,fill=fillColor] (173.90,234.48) circle (  1.16);

\path[draw=drawColor,line width= 0.4pt,line join=round,line cap=round,fill=fillColor] (174.15,234.46) circle (  1.16);

\path[draw=drawColor,line width= 0.4pt,line join=round,line cap=round,fill=fillColor] (174.41,234.45) circle (  1.16);

\path[draw=drawColor,line width= 0.4pt,line join=round,line cap=round,fill=fillColor] (174.67,234.39) circle (  1.16);

\path[draw=drawColor,line width= 0.4pt,line join=round,line cap=round,fill=fillColor] (174.92,234.23) circle (  1.16);

\path[draw=drawColor,line width= 0.4pt,line join=round,line cap=round,fill=fillColor] (175.17,234.08) circle (  1.16);

\path[draw=drawColor,line width= 0.4pt,line join=round,line cap=round,fill=fillColor] (175.42,234.06) circle (  1.16);

\path[draw=drawColor,line width= 0.4pt,line join=round,line cap=round,fill=fillColor] (175.68,233.97) circle (  1.16);

\path[draw=drawColor,line width= 0.4pt,line join=round,line cap=round,fill=fillColor] (175.93,233.97) circle (  1.16);

\path[draw=drawColor,line width= 0.4pt,line join=round,line cap=round,fill=fillColor] (176.18,233.94) circle (  1.16);

\path[draw=drawColor,line width= 0.4pt,line join=round,line cap=round,fill=fillColor] (176.42,233.86) circle (  1.16);

\path[draw=drawColor,line width= 0.4pt,line join=round,line cap=round,fill=fillColor] (176.67,233.86) circle (  1.16);

\path[draw=drawColor,line width= 0.4pt,line join=round,line cap=round,fill=fillColor] (176.92,233.78) circle (  1.16);

\path[draw=drawColor,line width= 0.4pt,line join=round,line cap=round,fill=fillColor] (177.16,233.75) circle (  1.16);

\path[draw=drawColor,line width= 0.4pt,line join=round,line cap=round,fill=fillColor] (177.41,233.74) circle (  1.16);

\path[draw=drawColor,line width= 0.4pt,line join=round,line cap=round,fill=fillColor] (177.65,233.65) circle (  1.16);

\path[draw=drawColor,line width= 0.4pt,line join=round,line cap=round,fill=fillColor] (177.89,233.63) circle (  1.16);

\path[draw=drawColor,line width= 0.4pt,line join=round,line cap=round,fill=fillColor] (178.13,233.63) circle (  1.16);

\path[draw=drawColor,line width= 0.4pt,line join=round,line cap=round,fill=fillColor] (178.37,233.58) circle (  1.16);

\path[draw=drawColor,line width= 0.4pt,line join=round,line cap=round,fill=fillColor] (178.61,233.53) circle (  1.16);

\path[draw=drawColor,line width= 0.4pt,line join=round,line cap=round,fill=fillColor] (178.85,233.34) circle (  1.16);

\path[draw=drawColor,line width= 0.4pt,line join=round,line cap=round,fill=fillColor] (179.09,233.25) circle (  1.16);

\path[draw=drawColor,line width= 0.4pt,line join=round,line cap=round,fill=fillColor] (179.33,233.21) circle (  1.16);

\path[draw=drawColor,line width= 0.4pt,line join=round,line cap=round,fill=fillColor] (179.56,233.21) circle (  1.16);

\path[draw=drawColor,line width= 0.4pt,line join=round,line cap=round,fill=fillColor] (179.80,233.17) circle (  1.16);

\path[draw=drawColor,line width= 0.4pt,line join=round,line cap=round,fill=fillColor] (180.03,233.13) circle (  1.16);

\path[draw=drawColor,line width= 0.4pt,line join=round,line cap=round,fill=fillColor] (180.27,233.12) circle (  1.16);

\path[draw=drawColor,line width= 0.4pt,line join=round,line cap=round,fill=fillColor] (180.50,233.11) circle (  1.16);

\path[draw=drawColor,line width= 0.4pt,line join=round,line cap=round,fill=fillColor] (180.73,233.04) circle (  1.16);

\path[draw=drawColor,line width= 0.4pt,line join=round,line cap=round,fill=fillColor] (180.96,232.91) circle (  1.16);

\path[draw=drawColor,line width= 0.4pt,line join=round,line cap=round,fill=fillColor] (181.19,232.90) circle (  1.16);

\path[draw=drawColor,line width= 0.4pt,line join=round,line cap=round,fill=fillColor] (181.42,232.87) circle (  1.16);

\path[draw=drawColor,line width= 0.4pt,line join=round,line cap=round,fill=fillColor] (181.65,232.87) circle (  1.16);

\path[draw=drawColor,line width= 0.4pt,line join=round,line cap=round,fill=fillColor] (181.88,232.84) circle (  1.16);

\path[draw=drawColor,line width= 0.4pt,line join=round,line cap=round,fill=fillColor] (182.10,232.83) circle (  1.16);

\path[draw=drawColor,line width= 0.4pt,line join=round,line cap=round,fill=fillColor] (182.33,232.76) circle (  1.16);

\path[draw=drawColor,line width= 0.4pt,line join=round,line cap=round,fill=fillColor] (182.55,232.75) circle (  1.16);

\path[draw=drawColor,line width= 0.4pt,line join=round,line cap=round,fill=fillColor] (182.78,232.73) circle (  1.16);

\path[draw=drawColor,line width= 0.4pt,line join=round,line cap=round,fill=fillColor] (183.00,232.71) circle (  1.16);

\path[draw=drawColor,line width= 0.4pt,line join=round,line cap=round,fill=fillColor] (183.23,232.68) circle (  1.16);

\path[draw=drawColor,line width= 0.4pt,line join=round,line cap=round,fill=fillColor] (183.45,232.67) circle (  1.16);

\path[draw=drawColor,line width= 0.4pt,line join=round,line cap=round,fill=fillColor] (183.67,232.67) circle (  1.16);

\path[draw=drawColor,line width= 0.4pt,line join=round,line cap=round,fill=fillColor] (183.89,232.63) circle (  1.16);

\path[draw=drawColor,line width= 0.4pt,line join=round,line cap=round,fill=fillColor] (184.11,232.50) circle (  1.16);

\path[draw=drawColor,line width= 0.4pt,line join=round,line cap=round,fill=fillColor] (184.33,232.41) circle (  1.16);

\path[draw=drawColor,line width= 0.4pt,line join=round,line cap=round,fill=fillColor] (184.55,232.40) circle (  1.16);

\path[draw=drawColor,line width= 0.4pt,line join=round,line cap=round,fill=fillColor] (184.77,232.39) circle (  1.16);

\path[draw=drawColor,line width= 0.4pt,line join=round,line cap=round,fill=fillColor] (184.98,232.35) circle (  1.16);

\path[draw=drawColor,line width= 0.4pt,line join=round,line cap=round,fill=fillColor] (185.20,232.35) circle (  1.16);

\path[draw=drawColor,line width= 0.4pt,line join=round,line cap=round,fill=fillColor] (185.41,232.33) circle (  1.16);

\path[draw=drawColor,line width= 0.4pt,line join=round,line cap=round,fill=fillColor] (185.63,232.24) circle (  1.16);

\path[draw=drawColor,line width= 0.4pt,line join=round,line cap=round,fill=fillColor] (185.84,232.22) circle (  1.16);

\path[draw=drawColor,line width= 0.4pt,line join=round,line cap=round,fill=fillColor] (186.06,232.16) circle (  1.16);

\path[draw=drawColor,line width= 0.4pt,line join=round,line cap=round,fill=fillColor] (186.27,232.16) circle (  1.16);

\path[draw=drawColor,line width= 0.4pt,line join=round,line cap=round,fill=fillColor] (186.48,232.04) circle (  1.16);

\path[draw=drawColor,line width= 0.4pt,line join=round,line cap=round,fill=fillColor] (186.69,232.02) circle (  1.16);

\path[draw=drawColor,line width= 0.4pt,line join=round,line cap=round,fill=fillColor] (186.91,231.98) circle (  1.16);

\path[draw=drawColor,line width= 0.4pt,line join=round,line cap=round,fill=fillColor] (187.12,231.95) circle (  1.16);

\path[draw=drawColor,line width= 0.4pt,line join=round,line cap=round,fill=fillColor] (187.33,231.93) circle (  1.16);

\path[draw=drawColor,line width= 0.4pt,line join=round,line cap=round,fill=fillColor] (187.53,231.89) circle (  1.16);

\path[draw=drawColor,line width= 0.4pt,line join=round,line cap=round,fill=fillColor] (187.74,231.89) circle (  1.16);

\path[draw=drawColor,line width= 0.4pt,line join=round,line cap=round,fill=fillColor] (187.95,231.88) circle (  1.16);

\path[draw=drawColor,line width= 0.4pt,line join=round,line cap=round,fill=fillColor] (188.16,231.82) circle (  1.16);

\path[draw=drawColor,line width= 0.4pt,line join=round,line cap=round,fill=fillColor] (188.36,231.70) circle (  1.16);

\path[draw=drawColor,line width= 0.4pt,line join=round,line cap=round,fill=fillColor] (188.57,231.62) circle (  1.16);

\path[draw=drawColor,line width= 0.4pt,line join=round,line cap=round,fill=fillColor] (188.77,231.56) circle (  1.16);

\path[draw=drawColor,line width= 0.4pt,line join=round,line cap=round,fill=fillColor] (188.98,231.44) circle (  1.16);

\path[draw=drawColor,line width= 0.4pt,line join=round,line cap=round,fill=fillColor] (189.18,231.42) circle (  1.16);

\path[draw=drawColor,line width= 0.4pt,line join=round,line cap=round,fill=fillColor] (189.39,231.39) circle (  1.16);

\path[draw=drawColor,line width= 0.4pt,line join=round,line cap=round,fill=fillColor] (189.59,231.38) circle (  1.16);

\path[draw=drawColor,line width= 0.4pt,line join=round,line cap=round,fill=fillColor] (189.79,231.36) circle (  1.16);

\path[draw=drawColor,line width= 0.4pt,line join=round,line cap=round,fill=fillColor] (189.99,231.34) circle (  1.16);

\path[draw=drawColor,line width= 0.4pt,line join=round,line cap=round,fill=fillColor] (190.19,231.20) circle (  1.16);

\path[draw=drawColor,line width= 0.4pt,line join=round,line cap=round,fill=fillColor] (190.39,231.16) circle (  1.16);

\path[draw=drawColor,line width= 0.4pt,line join=round,line cap=round,fill=fillColor] (190.59,231.10) circle (  1.16);

\path[draw=drawColor,line width= 0.4pt,line join=round,line cap=round,fill=fillColor] (190.79,231.09) circle (  1.16);

\path[draw=drawColor,line width= 0.4pt,line join=round,line cap=round,fill=fillColor] (190.99,231.07) circle (  1.16);

\path[draw=drawColor,line width= 0.4pt,line join=round,line cap=round,fill=fillColor] (191.19,231.05) circle (  1.16);

\path[draw=drawColor,line width= 0.4pt,line join=round,line cap=round,fill=fillColor] (191.39,230.97) circle (  1.16);

\path[draw=drawColor,line width= 0.4pt,line join=round,line cap=round,fill=fillColor] (191.58,230.78) circle (  1.16);

\path[draw=drawColor,line width= 0.4pt,line join=round,line cap=round,fill=fillColor] (191.78,230.74) circle (  1.16);

\path[draw=drawColor,line width= 0.4pt,line join=round,line cap=round,fill=fillColor] (191.98,230.72) circle (  1.16);

\path[draw=drawColor,line width= 0.4pt,line join=round,line cap=round,fill=fillColor] (192.17,230.71) circle (  1.16);

\path[draw=drawColor,line width= 0.4pt,line join=round,line cap=round,fill=fillColor] (192.37,230.67) circle (  1.16);

\path[draw=drawColor,line width= 0.4pt,line join=round,line cap=round,fill=fillColor] (192.56,230.64) circle (  1.16);

\path[draw=drawColor,line width= 0.4pt,line join=round,line cap=round,fill=fillColor] (192.75,230.62) circle (  1.16);

\path[draw=drawColor,line width= 0.4pt,line join=round,line cap=round,fill=fillColor] (192.95,230.62) circle (  1.16);

\path[draw=drawColor,line width= 0.4pt,line join=round,line cap=round,fill=fillColor] (193.14,230.57) circle (  1.16);

\path[draw=drawColor,line width= 0.4pt,line join=round,line cap=round,fill=fillColor] (193.33,230.56) circle (  1.16);

\path[draw=drawColor,line width= 0.4pt,line join=round,line cap=round,fill=fillColor] (193.52,230.52) circle (  1.16);

\path[draw=drawColor,line width= 0.4pt,line join=round,line cap=round,fill=fillColor] (193.71,230.52) circle (  1.16);

\path[draw=drawColor,line width= 0.4pt,line join=round,line cap=round,fill=fillColor] (193.90,230.43) circle (  1.16);

\path[draw=drawColor,line width= 0.4pt,line join=round,line cap=round,fill=fillColor] (194.09,230.39) circle (  1.16);

\path[draw=drawColor,line width= 0.4pt,line join=round,line cap=round,fill=fillColor] (194.28,230.33) circle (  1.16);

\path[draw=drawColor,line width= 0.4pt,line join=round,line cap=round,fill=fillColor] (194.47,230.33) circle (  1.16);

\path[draw=drawColor,line width= 0.4pt,line join=round,line cap=round,fill=fillColor] (194.66,230.31) circle (  1.16);

\path[draw=drawColor,line width= 0.4pt,line join=round,line cap=round,fill=fillColor] (194.85,230.31) circle (  1.16);

\path[draw=drawColor,line width= 0.4pt,line join=round,line cap=round,fill=fillColor] (195.04,230.31) circle (  1.16);

\path[draw=drawColor,line width= 0.4pt,line join=round,line cap=round,fill=fillColor] (195.22,230.21) circle (  1.16);

\path[draw=drawColor,line width= 0.4pt,line join=round,line cap=round,fill=fillColor] (195.41,230.13) circle (  1.16);

\path[draw=drawColor,line width= 0.4pt,line join=round,line cap=round,fill=fillColor] (195.60,230.11) circle (  1.16);

\path[draw=drawColor,line width= 0.4pt,line join=round,line cap=round,fill=fillColor] (195.78,230.04) circle (  1.16);

\path[draw=drawColor,line width= 0.4pt,line join=round,line cap=round,fill=fillColor] (195.97,230.04) circle (  1.16);

\path[draw=drawColor,line width= 0.4pt,line join=round,line cap=round,fill=fillColor] (196.15,229.94) circle (  1.16);

\path[draw=drawColor,line width= 0.4pt,line join=round,line cap=round,fill=fillColor] (196.34,229.90) circle (  1.16);

\path[draw=drawColor,line width= 0.4pt,line join=round,line cap=round,fill=fillColor] (196.52,229.88) circle (  1.16);

\path[draw=drawColor,line width= 0.4pt,line join=round,line cap=round,fill=fillColor] (196.70,229.82) circle (  1.16);

\path[draw=drawColor,line width= 0.4pt,line join=round,line cap=round,fill=fillColor] (196.89,229.80) circle (  1.16);

\path[draw=drawColor,line width= 0.4pt,line join=round,line cap=round,fill=fillColor] (197.07,229.74) circle (  1.16);

\path[draw=drawColor,line width= 0.4pt,line join=round,line cap=round,fill=fillColor] (197.25,229.73) circle (  1.16);

\path[draw=drawColor,line width= 0.4pt,line join=round,line cap=round,fill=fillColor] (197.43,229.73) circle (  1.16);

\path[draw=drawColor,line width= 0.4pt,line join=round,line cap=round,fill=fillColor] (197.61,229.72) circle (  1.16);

\path[draw=drawColor,line width= 0.4pt,line join=round,line cap=round,fill=fillColor] (197.79,229.61) circle (  1.16);

\path[draw=drawColor,line width= 0.4pt,line join=round,line cap=round,fill=fillColor] (197.97,229.60) circle (  1.16);

\path[draw=drawColor,line width= 0.4pt,line join=round,line cap=round,fill=fillColor] (198.15,229.50) circle (  1.16);

\path[draw=drawColor,line width= 0.4pt,line join=round,line cap=round,fill=fillColor] (198.33,229.48) circle (  1.16);

\path[draw=drawColor,line width= 0.4pt,line join=round,line cap=round,fill=fillColor] (198.51,229.45) circle (  1.16);

\path[draw=drawColor,line width= 0.4pt,line join=round,line cap=round,fill=fillColor] (198.69,229.43) circle (  1.16);

\path[draw=drawColor,line width= 0.4pt,line join=round,line cap=round,fill=fillColor] (198.87,229.42) circle (  1.16);

\path[draw=drawColor,line width= 0.4pt,line join=round,line cap=round,fill=fillColor] (199.04,229.37) circle (  1.16);

\path[draw=drawColor,line width= 0.4pt,line join=round,line cap=round,fill=fillColor] (199.22,229.37) circle (  1.16);

\path[draw=drawColor,line width= 0.4pt,line join=round,line cap=round,fill=fillColor] (199.40,229.37) circle (  1.16);

\path[draw=drawColor,line width= 0.4pt,line join=round,line cap=round,fill=fillColor] (199.57,229.35) circle (  1.16);

\path[draw=drawColor,line width= 0.4pt,line join=round,line cap=round,fill=fillColor] (199.75,229.34) circle (  1.16);

\path[draw=drawColor,line width= 0.4pt,line join=round,line cap=round,fill=fillColor] (199.92,229.34) circle (  1.16);

\path[draw=drawColor,line width= 0.4pt,line join=round,line cap=round,fill=fillColor] (200.10,229.33) circle (  1.16);

\path[draw=drawColor,line width= 0.4pt,line join=round,line cap=round,fill=fillColor] (200.27,229.23) circle (  1.16);

\path[draw=drawColor,line width= 0.4pt,line join=round,line cap=round,fill=fillColor] (200.45,229.19) circle (  1.16);

\path[draw=drawColor,line width= 0.4pt,line join=round,line cap=round,fill=fillColor] (200.62,229.18) circle (  1.16);

\path[draw=drawColor,line width= 0.4pt,line join=round,line cap=round,fill=fillColor] (200.79,229.16) circle (  1.16);

\path[draw=drawColor,line width= 0.4pt,line join=round,line cap=round,fill=fillColor] (200.97,229.16) circle (  1.16);

\path[draw=drawColor,line width= 0.4pt,line join=round,line cap=round,fill=fillColor] (201.14,229.09) circle (  1.16);

\path[draw=drawColor,line width= 0.4pt,line join=round,line cap=round,fill=fillColor] (201.31,229.07) circle (  1.16);

\path[draw=drawColor,line width= 0.4pt,line join=round,line cap=round,fill=fillColor] (201.48,229.05) circle (  1.16);

\path[draw=drawColor,line width= 0.4pt,line join=round,line cap=round,fill=fillColor] (201.65,229.00) circle (  1.16);

\path[draw=drawColor,line width= 0.4pt,line join=round,line cap=round,fill=fillColor] (201.83,228.95) circle (  1.16);

\path[draw=drawColor,line width= 0.4pt,line join=round,line cap=round,fill=fillColor] (202.00,228.95) circle (  1.16);

\path[draw=drawColor,line width= 0.4pt,line join=round,line cap=round,fill=fillColor] (202.17,228.94) circle (  1.16);

\path[draw=drawColor,line width= 0.4pt,line join=round,line cap=round,fill=fillColor] (202.34,228.94) circle (  1.16);

\path[draw=drawColor,line width= 0.4pt,line join=round,line cap=round,fill=fillColor] (202.50,228.91) circle (  1.16);

\path[draw=drawColor,line width= 0.4pt,line join=round,line cap=round,fill=fillColor] (202.67,228.89) circle (  1.16);

\path[draw=drawColor,line width= 0.4pt,line join=round,line cap=round,fill=fillColor] (202.84,228.83) circle (  1.16);

\path[draw=drawColor,line width= 0.4pt,line join=round,line cap=round,fill=fillColor] (203.01,228.83) circle (  1.16);

\path[draw=drawColor,line width= 0.4pt,line join=round,line cap=round,fill=fillColor] (203.18,228.79) circle (  1.16);

\path[draw=drawColor,line width= 0.4pt,line join=round,line cap=round,fill=fillColor] (203.35,228.74) circle (  1.16);

\path[draw=drawColor,line width= 0.4pt,line join=round,line cap=round,fill=fillColor] (203.51,228.73) circle (  1.16);

\path[draw=drawColor,line width= 0.4pt,line join=round,line cap=round,fill=fillColor] (203.68,228.72) circle (  1.16);

\path[draw=drawColor,line width= 0.4pt,line join=round,line cap=round,fill=fillColor] (203.85,228.71) circle (  1.16);

\path[draw=drawColor,line width= 0.4pt,line join=round,line cap=round,fill=fillColor] (204.01,228.69) circle (  1.16);

\path[draw=drawColor,line width= 0.4pt,line join=round,line cap=round,fill=fillColor] (204.18,228.68) circle (  1.16);

\path[draw=drawColor,line width= 0.4pt,line join=round,line cap=round,fill=fillColor] (204.34,228.68) circle (  1.16);

\path[draw=drawColor,line width= 0.4pt,line join=round,line cap=round,fill=fillColor] (204.51,228.67) circle (  1.16);

\path[draw=drawColor,line width= 0.4pt,line join=round,line cap=round,fill=fillColor] (204.67,228.63) circle (  1.16);

\path[draw=drawColor,line width= 0.4pt,line join=round,line cap=round,fill=fillColor] (204.84,228.57) circle (  1.16);

\path[draw=drawColor,line width= 0.4pt,line join=round,line cap=round,fill=fillColor] (205.00,228.52) circle (  1.16);

\path[draw=drawColor,line width= 0.4pt,line join=round,line cap=round,fill=fillColor] (205.16,228.51) circle (  1.16);

\path[draw=drawColor,line width= 0.4pt,line join=round,line cap=round,fill=fillColor] (205.33,228.48) circle (  1.16);

\path[draw=drawColor,line width= 0.4pt,line join=round,line cap=round,fill=fillColor] (205.49,228.43) circle (  1.16);

\path[draw=drawColor,line width= 0.4pt,line join=round,line cap=round,fill=fillColor] (205.65,228.34) circle (  1.16);

\path[draw=drawColor,line width= 0.4pt,line join=round,line cap=round,fill=fillColor] (205.81,228.24) circle (  1.16);

\path[draw=drawColor,line width= 0.4pt,line join=round,line cap=round,fill=fillColor] (205.97,228.22) circle (  1.16);

\path[draw=drawColor,line width= 0.4pt,line join=round,line cap=round,fill=fillColor] (206.14,228.20) circle (  1.16);

\path[draw=drawColor,line width= 0.4pt,line join=round,line cap=round,fill=fillColor] (206.30,228.11) circle (  1.16);

\path[draw=drawColor,line width= 0.4pt,line join=round,line cap=round,fill=fillColor] (206.46,228.10) circle (  1.16);

\path[draw=drawColor,line width= 0.4pt,line join=round,line cap=round,fill=fillColor] (206.62,228.10) circle (  1.16);

\path[draw=drawColor,line width= 0.4pt,line join=round,line cap=round,fill=fillColor] (206.78,228.00) circle (  1.16);

\path[draw=drawColor,line width= 0.4pt,line join=round,line cap=round,fill=fillColor] (206.94,227.98) circle (  1.16);

\path[draw=drawColor,line width= 0.4pt,line join=round,line cap=round,fill=fillColor] (207.10,227.96) circle (  1.16);

\path[draw=drawColor,line width= 0.4pt,line join=round,line cap=round,fill=fillColor] (207.26,227.88) circle (  1.16);

\path[draw=drawColor,line width= 0.4pt,line join=round,line cap=round,fill=fillColor] (207.42,227.87) circle (  1.16);

\path[draw=drawColor,line width= 0.4pt,line join=round,line cap=round,fill=fillColor] (207.57,227.85) circle (  1.16);

\path[draw=drawColor,line width= 0.4pt,line join=round,line cap=round,fill=fillColor] (207.73,227.81) circle (  1.16);

\path[draw=drawColor,line width= 0.4pt,line join=round,line cap=round,fill=fillColor] (207.89,227.76) circle (  1.16);

\path[draw=drawColor,line width= 0.4pt,line join=round,line cap=round,fill=fillColor] (208.05,227.68) circle (  1.16);

\path[draw=drawColor,line width= 0.4pt,line join=round,line cap=round,fill=fillColor] (208.20,227.55) circle (  1.16);

\path[draw=drawColor,line width= 0.4pt,line join=round,line cap=round,fill=fillColor] (208.36,227.29) circle (  1.16);

\path[draw=drawColor,line width= 0.4pt,line join=round,line cap=round,fill=fillColor] (208.52,227.27) circle (  1.16);

\path[draw=drawColor,line width= 0.4pt,line join=round,line cap=round,fill=fillColor] (208.67,227.24) circle (  1.16);

\path[draw=drawColor,line width= 0.4pt,line join=round,line cap=round,fill=fillColor] (208.83,227.19) circle (  1.16);

\path[draw=drawColor,line width= 0.4pt,line join=round,line cap=round,fill=fillColor] (208.98,227.14) circle (  1.16);

\path[draw=drawColor,line width= 0.4pt,line join=round,line cap=round,fill=fillColor] (209.14,227.10) circle (  1.16);

\path[draw=drawColor,line width= 0.4pt,line join=round,line cap=round,fill=fillColor] (209.29,227.10) circle (  1.16);

\path[draw=drawColor,line width= 0.4pt,line join=round,line cap=round,fill=fillColor] (209.45,227.04) circle (  1.16);

\path[draw=drawColor,line width= 0.4pt,line join=round,line cap=round,fill=fillColor] (209.60,227.00) circle (  1.16);

\path[draw=drawColor,line width= 0.4pt,line join=round,line cap=round,fill=fillColor] (209.76,226.95) circle (  1.16);

\path[draw=drawColor,line width= 0.4pt,line join=round,line cap=round,fill=fillColor] (209.91,226.87) circle (  1.16);

\path[draw=drawColor,line width= 0.4pt,line join=round,line cap=round,fill=fillColor] (210.07,226.87) circle (  1.16);

\path[draw=drawColor,line width= 0.4pt,line join=round,line cap=round,fill=fillColor] (210.22,226.74) circle (  1.16);

\path[draw=drawColor,line width= 0.4pt,line join=round,line cap=round,fill=fillColor] (210.37,226.74) circle (  1.16);

\path[draw=drawColor,line width= 0.4pt,line join=round,line cap=round,fill=fillColor] (210.52,226.70) circle (  1.16);

\path[draw=drawColor,line width= 0.4pt,line join=round,line cap=round,fill=fillColor] (210.68,226.68) circle (  1.16);

\path[draw=drawColor,line width= 0.4pt,line join=round,line cap=round,fill=fillColor] (210.83,226.67) circle (  1.16);

\path[draw=drawColor,line width= 0.4pt,line join=round,line cap=round,fill=fillColor] (210.98,226.66) circle (  1.16);

\path[draw=drawColor,line width= 0.4pt,line join=round,line cap=round,fill=fillColor] (211.13,226.64) circle (  1.16);

\path[draw=drawColor,line width= 0.4pt,line join=round,line cap=round,fill=fillColor] (211.28,226.63) circle (  1.16);

\path[draw=drawColor,line width= 0.4pt,line join=round,line cap=round,fill=fillColor] (211.43,226.63) circle (  1.16);

\path[draw=drawColor,line width= 0.4pt,line join=round,line cap=round,fill=fillColor] (211.58,226.54) circle (  1.16);

\path[draw=drawColor,line width= 0.4pt,line join=round,line cap=round,fill=fillColor] (211.73,226.45) circle (  1.16);

\path[draw=drawColor,line width= 0.4pt,line join=round,line cap=round,fill=fillColor] (211.88,226.36) circle (  1.16);

\path[draw=drawColor,line width= 0.4pt,line join=round,line cap=round,fill=fillColor] (212.03,226.27) circle (  1.16);

\path[draw=drawColor,line width= 0.4pt,line join=round,line cap=round,fill=fillColor] (212.18,226.22) circle (  1.16);

\path[draw=drawColor,line width= 0.4pt,line join=round,line cap=round,fill=fillColor] (212.33,226.16) circle (  1.16);

\path[draw=drawColor,line width= 0.4pt,line join=round,line cap=round,fill=fillColor] (212.48,226.15) circle (  1.16);

\path[draw=drawColor,line width= 0.4pt,line join=round,line cap=round,fill=fillColor] (212.63,226.11) circle (  1.16);

\path[draw=drawColor,line width= 0.4pt,line join=round,line cap=round,fill=fillColor] (212.78,226.11) circle (  1.16);

\path[draw=drawColor,line width= 0.4pt,line join=round,line cap=round,fill=fillColor] (212.93,226.11) circle (  1.16);

\path[draw=drawColor,line width= 0.4pt,line join=round,line cap=round,fill=fillColor] (213.07,226.06) circle (  1.16);

\path[draw=drawColor,line width= 0.4pt,line join=round,line cap=round,fill=fillColor] (213.22,225.90) circle (  1.16);

\path[draw=drawColor,line width= 0.4pt,line join=round,line cap=round,fill=fillColor] (213.37,225.80) circle (  1.16);

\path[draw=drawColor,line width= 0.4pt,line join=round,line cap=round,fill=fillColor] (213.51,225.74) circle (  1.16);

\path[draw=drawColor,line width= 0.4pt,line join=round,line cap=round,fill=fillColor] (213.66,225.67) circle (  1.16);

\path[draw=drawColor,line width= 0.4pt,line join=round,line cap=round,fill=fillColor] (213.81,225.61) circle (  1.16);

\path[draw=drawColor,line width= 0.4pt,line join=round,line cap=round,fill=fillColor] (213.95,225.59) circle (  1.16);

\path[draw=drawColor,line width= 0.4pt,line join=round,line cap=round,fill=fillColor] (214.10,225.54) circle (  1.16);

\path[draw=drawColor,line width= 0.4pt,line join=round,line cap=round,fill=fillColor] (214.24,225.43) circle (  1.16);

\path[draw=drawColor,line width= 0.4pt,line join=round,line cap=round,fill=fillColor] (214.39,225.34) circle (  1.16);

\path[draw=drawColor,line width= 0.4pt,line join=round,line cap=round,fill=fillColor] (214.54,225.32) circle (  1.16);

\path[draw=drawColor,line width= 0.4pt,line join=round,line cap=round,fill=fillColor] (214.68,225.17) circle (  1.16);

\path[draw=drawColor,line width= 0.4pt,line join=round,line cap=round,fill=fillColor] (214.82,225.14) circle (  1.16);

\path[draw=drawColor,line width= 0.4pt,line join=round,line cap=round,fill=fillColor] (214.97,225.10) circle (  1.16);

\path[draw=drawColor,line width= 0.4pt,line join=round,line cap=round,fill=fillColor] (215.11,225.01) circle (  1.16);

\path[draw=drawColor,line width= 0.4pt,line join=round,line cap=round,fill=fillColor] (215.26,224.82) circle (  1.16);

\path[draw=drawColor,line width= 0.4pt,line join=round,line cap=round,fill=fillColor] (215.40,224.79) circle (  1.16);

\path[draw=drawColor,line width= 0.4pt,line join=round,line cap=round,fill=fillColor] (215.54,224.49) circle (  1.16);

\path[draw=drawColor,line width= 0.4pt,line join=round,line cap=round,fill=fillColor] (215.69,224.49) circle (  1.16);

\path[draw=drawColor,line width= 0.4pt,line join=round,line cap=round,fill=fillColor] (215.83,224.48) circle (  1.16);

\path[draw=drawColor,line width= 0.4pt,line join=round,line cap=round,fill=fillColor] (215.97,224.44) circle (  1.16);

\path[draw=drawColor,line width= 0.4pt,line join=round,line cap=round,fill=fillColor] (216.11,224.24) circle (  1.16);

\path[draw=drawColor,line width= 0.4pt,line join=round,line cap=round,fill=fillColor] (216.26,224.11) circle (  1.16);

\path[draw=drawColor,line width= 0.4pt,line join=round,line cap=round,fill=fillColor] (216.40,223.97) circle (  1.16);

\path[draw=drawColor,line width= 0.4pt,line join=round,line cap=round,fill=fillColor] (216.54,223.78) circle (  1.16);

\path[draw=drawColor,line width= 0.4pt,line join=round,line cap=round,fill=fillColor] (216.68,223.77) circle (  1.16);

\path[draw=drawColor,line width= 0.4pt,line join=round,line cap=round,fill=fillColor] (216.82,223.60) circle (  1.16);

\path[draw=drawColor,line width= 0.4pt,line join=round,line cap=round,fill=fillColor] (216.96,223.52) circle (  1.16);

\path[draw=drawColor,line width= 0.4pt,line join=round,line cap=round,fill=fillColor] (217.10,223.42) circle (  1.16);

\path[draw=drawColor,line width= 0.4pt,line join=round,line cap=round,fill=fillColor] (217.24,223.22) circle (  1.16);

\path[draw=drawColor,line width= 0.4pt,line join=round,line cap=round,fill=fillColor] (217.38,223.17) circle (  1.16);

\path[draw=drawColor,line width= 0.4pt,line join=round,line cap=round,fill=fillColor] (217.52,223.06) circle (  1.16);

\path[draw=drawColor,line width= 0.4pt,line join=round,line cap=round,fill=fillColor] (217.66,222.69) circle (  1.16);

\path[draw=drawColor,line width= 0.4pt,line join=round,line cap=round,fill=fillColor] (217.80,222.60) circle (  1.16);

\path[draw=drawColor,line width= 0.4pt,line join=round,line cap=round,fill=fillColor] (217.94,222.56) circle (  1.16);

\path[draw=drawColor,line width= 0.4pt,line join=round,line cap=round,fill=fillColor] (218.08,222.52) circle (  1.16);

\path[draw=drawColor,line width= 0.4pt,line join=round,line cap=round,fill=fillColor] (218.22,222.45) circle (  1.16);

\path[draw=drawColor,line width= 0.4pt,line join=round,line cap=round,fill=fillColor] (218.36,222.32) circle (  1.16);

\path[draw=drawColor,line width= 0.4pt,line join=round,line cap=round,fill=fillColor] (218.50,221.94) circle (  1.16);

\path[draw=drawColor,line width= 0.4pt,line join=round,line cap=round,fill=fillColor] (218.63,221.72) circle (  1.16);

\path[draw=drawColor,line width= 0.4pt,line join=round,line cap=round,fill=fillColor] (218.77,221.62) circle (  1.16);

\path[draw=drawColor,line width= 0.4pt,line join=round,line cap=round,fill=fillColor] (218.91,221.42) circle (  1.16);

\path[draw=drawColor,line width= 0.4pt,line join=round,line cap=round,fill=fillColor] (219.05,221.39) circle (  1.16);

\path[draw=drawColor,line width= 0.4pt,line join=round,line cap=round,fill=fillColor] (219.18,221.24) circle (  1.16);

\path[draw=drawColor,line width= 0.4pt,line join=round,line cap=round,fill=fillColor] (219.32,221.19) circle (  1.16);

\path[draw=drawColor,line width= 0.4pt,line join=round,line cap=round,fill=fillColor] (219.46,221.09) circle (  1.16);

\path[draw=drawColor,line width= 0.4pt,line join=round,line cap=round,fill=fillColor] (219.59,221.00) circle (  1.16);

\path[draw=drawColor,line width= 0.4pt,line join=round,line cap=round,fill=fillColor] (219.73,220.85) circle (  1.16);

\path[draw=drawColor,line width= 0.4pt,line join=round,line cap=round,fill=fillColor] (219.87,220.78) circle (  1.16);

\path[draw=drawColor,line width= 0.4pt,line join=round,line cap=round,fill=fillColor] (220.00,220.75) circle (  1.16);

\path[draw=drawColor,line width= 0.4pt,line join=round,line cap=round,fill=fillColor] (220.14,220.59) circle (  1.16);

\path[draw=drawColor,line width= 0.4pt,line join=round,line cap=round,fill=fillColor] (220.27,220.46) circle (  1.16);

\path[draw=drawColor,line width= 0.4pt,line join=round,line cap=round,fill=fillColor] (220.41,220.42) circle (  1.16);

\path[draw=drawColor,line width= 0.4pt,line join=round,line cap=round,fill=fillColor] (220.54,220.00) circle (  1.16);

\path[draw=drawColor,line width= 0.4pt,line join=round,line cap=round,fill=fillColor] (220.68,219.90) circle (  1.16);

\path[draw=drawColor,line width= 0.4pt,line join=round,line cap=round,fill=fillColor] (220.81,219.82) circle (  1.16);

\path[draw=drawColor,line width= 0.4pt,line join=round,line cap=round,fill=fillColor] (220.95,219.68) circle (  1.16);

\path[draw=drawColor,line width= 0.4pt,line join=round,line cap=round,fill=fillColor] (221.08,219.42) circle (  1.16);

\path[draw=drawColor,line width= 0.4pt,line join=round,line cap=round,fill=fillColor] (221.21,219.39) circle (  1.16);

\path[draw=drawColor,line width= 0.4pt,line join=round,line cap=round,fill=fillColor] (221.35,219.11) circle (  1.16);

\path[draw=drawColor,line width= 0.4pt,line join=round,line cap=round,fill=fillColor] (221.48,218.94) circle (  1.16);

\path[draw=drawColor,line width= 0.4pt,line join=round,line cap=round,fill=fillColor] (221.61,218.84) circle (  1.16);

\path[draw=drawColor,line width= 0.4pt,line join=round,line cap=round,fill=fillColor] (221.75,218.78) circle (  1.16);

\path[draw=drawColor,line width= 0.4pt,line join=round,line cap=round,fill=fillColor] (221.88,218.06) circle (  1.16);

\path[draw=drawColor,line width= 0.4pt,line join=round,line cap=round,fill=fillColor] (222.01,217.84) circle (  1.16);

\path[draw=drawColor,line width= 0.4pt,line join=round,line cap=round,fill=fillColor] (222.14,216.03) circle (  1.16);

\path[draw=drawColor,line width= 0.4pt,line join=round,line cap=round,fill=fillColor] (222.28,216.01) circle (  1.16);

\path[draw=drawColor,line width= 0.4pt,line join=round,line cap=round,fill=fillColor] (222.41,215.04) circle (  1.16);

\path[draw=drawColor,line width= 0.4pt,line join=round,line cap=round,fill=fillColor] (222.54,214.79) circle (  1.16);

\path[draw=drawColor,line width= 0.4pt,line join=round,line cap=round,fill=fillColor] (222.67,209.81) circle (  1.16);

\path[draw=drawColor,line width= 0.4pt,line join=round,line cap=round,fill=fillColor] (222.80,209.81) circle (  1.16);

\path[draw=drawColor,line width= 0.4pt,line join=round,line cap=round,fill=fillColor] (222.93,209.81) circle (  1.16);

\path[draw=drawColor,line width= 0.4pt,line join=round,line cap=round,fill=fillColor] (223.07,209.81) circle (  1.16);

\path[draw=drawColor,line width= 0.4pt,line join=round,line cap=round,fill=fillColor] (223.20,209.81) circle (  1.16);

\path[draw=drawColor,line width= 0.4pt,line join=round,line cap=round,fill=fillColor] (223.33,209.81) circle (  1.16);

\path[draw=drawColor,line width= 0.4pt,line join=round,line cap=round,fill=fillColor] (223.46,209.81) circle (  1.16);

\path[draw=drawColor,line width= 0.4pt,line join=round,line cap=round,fill=fillColor] (223.59,209.81) circle (  1.16);

\path[draw=drawColor,line width= 0.4pt,line join=round,line cap=round,fill=fillColor] (223.72,209.81) circle (  1.16);

\path[draw=drawColor,line width= 0.4pt,line join=round,line cap=round,fill=fillColor] (223.85,209.81) circle (  1.16);

\path[draw=drawColor,line width= 0.4pt,line join=round,line cap=round,fill=fillColor] (223.98,209.81) circle (  1.16);

\path[draw=drawColor,line width= 0.4pt,line join=round,line cap=round,fill=fillColor] (224.11,209.81) circle (  1.16);

\path[draw=drawColor,line width= 0.4pt,line join=round,line cap=round,fill=fillColor] (224.23,209.81) circle (  1.16);

\path[draw=drawColor,line width= 0.4pt,line join=round,line cap=round,fill=fillColor] (224.36,209.81) circle (  1.16);

\path[draw=drawColor,line width= 0.4pt,line join=round,line cap=round,fill=fillColor] (224.49,209.81) circle (  1.16);

\path[draw=drawColor,line width= 0.4pt,line join=round,line cap=round,fill=fillColor] (224.62,209.81) circle (  1.16);

\path[draw=drawColor,line width= 0.4pt,line join=round,line cap=round,fill=fillColor] (224.75,209.81) circle (  1.16);

\path[draw=drawColor,line width= 0.4pt,line join=round,line cap=round,fill=fillColor] (224.88,209.81) circle (  1.16);

\path[draw=drawColor,line width= 0.4pt,line join=round,line cap=round,fill=fillColor] (225.01,209.81) circle (  1.16);

\path[draw=drawColor,line width= 0.4pt,line join=round,line cap=round,fill=fillColor] (225.13,209.81) circle (  1.16);

\path[draw=drawColor,line width= 0.4pt,line join=round,line cap=round,fill=fillColor] (225.26,209.81) circle (  1.16);
\definecolor[named]{drawColor}{rgb}{0.65,0.34,0.16}
\definecolor[named]{fillColor}{rgb}{0.65,0.34,0.16}

\path[draw=drawColor,line width= 0.4pt,line join=round,line cap=round,fill=fillColor] ( 74.88,259.00) circle (  1.16);

\path[draw=drawColor,line width= 0.4pt,line join=round,line cap=round,fill=fillColor] ( 80.74,258.40) circle (  1.16);

\path[draw=drawColor,line width= 0.4pt,line join=round,line cap=round,fill=fillColor] ( 84.84,258.03) circle (  1.16);

\path[draw=drawColor,line width= 0.4pt,line join=round,line cap=round,fill=fillColor] ( 88.11,254.56) circle (  1.16);

\path[draw=drawColor,line width= 0.4pt,line join=round,line cap=round,fill=fillColor] ( 90.88,251.51) circle (  1.16);

\path[draw=drawColor,line width= 0.4pt,line join=round,line cap=round,fill=fillColor] ( 93.29,250.35) circle (  1.16);

\path[draw=drawColor,line width= 0.4pt,line join=round,line cap=round,fill=fillColor] ( 95.45,247.58) circle (  1.16);

\path[draw=drawColor,line width= 0.4pt,line join=round,line cap=round,fill=fillColor] ( 97.41,246.39) circle (  1.16);

\path[draw=drawColor,line width= 0.4pt,line join=round,line cap=round,fill=fillColor] ( 99.21,245.67) circle (  1.16);

\path[draw=drawColor,line width= 0.4pt,line join=round,line cap=round,fill=fillColor] (100.89,245.64) circle (  1.16);

\path[draw=drawColor,line width= 0.4pt,line join=round,line cap=round,fill=fillColor] (102.46,244.88) circle (  1.16);

\path[draw=drawColor,line width= 0.4pt,line join=round,line cap=round,fill=fillColor] (103.93,243.55) circle (  1.16);

\path[draw=drawColor,line width= 0.4pt,line join=round,line cap=round,fill=fillColor] (105.33,242.82) circle (  1.16);

\path[draw=drawColor,line width= 0.4pt,line join=round,line cap=round,fill=fillColor] (106.65,242.60) circle (  1.16);

\path[draw=drawColor,line width= 0.4pt,line join=round,line cap=round,fill=fillColor] (107.91,242.02) circle (  1.16);

\path[draw=drawColor,line width= 0.4pt,line join=round,line cap=round,fill=fillColor] (109.12,241.45) circle (  1.16);

\path[draw=drawColor,line width= 0.4pt,line join=round,line cap=round,fill=fillColor] (110.28,240.91) circle (  1.16);

\path[draw=drawColor,line width= 0.4pt,line join=round,line cap=round,fill=fillColor] (111.40,240.12) circle (  1.16);

\path[draw=drawColor,line width= 0.4pt,line join=round,line cap=round,fill=fillColor] (112.47,239.53) circle (  1.16);

\path[draw=drawColor,line width= 0.4pt,line join=round,line cap=round,fill=fillColor] (113.51,239.16) circle (  1.16);

\path[draw=drawColor,line width= 0.4pt,line join=round,line cap=round,fill=fillColor] (114.51,238.11) circle (  1.16);

\path[draw=drawColor,line width= 0.4pt,line join=round,line cap=round,fill=fillColor] (115.48,237.62) circle (  1.16);

\path[draw=drawColor,line width= 0.4pt,line join=round,line cap=round,fill=fillColor] (116.42,237.58) circle (  1.16);

\path[draw=drawColor,line width= 0.4pt,line join=round,line cap=round,fill=fillColor] (117.34,237.38) circle (  1.16);

\path[draw=drawColor,line width= 0.4pt,line join=round,line cap=round,fill=fillColor] (118.23,236.77) circle (  1.16);

\path[draw=drawColor,line width= 0.4pt,line join=round,line cap=round,fill=fillColor] (119.10,236.60) circle (  1.16);

\path[draw=drawColor,line width= 0.4pt,line join=round,line cap=round,fill=fillColor] (119.94,236.58) circle (  1.16);

\path[draw=drawColor,line width= 0.4pt,line join=round,line cap=round,fill=fillColor] (120.77,236.47) circle (  1.16);

\path[draw=drawColor,line width= 0.4pt,line join=round,line cap=round,fill=fillColor] (121.57,236.43) circle (  1.16);

\path[draw=drawColor,line width= 0.4pt,line join=round,line cap=round,fill=fillColor] (122.36,236.40) circle (  1.16);

\path[draw=drawColor,line width= 0.4pt,line join=round,line cap=round,fill=fillColor] (123.13,236.22) circle (  1.16);

\path[draw=drawColor,line width= 0.4pt,line join=round,line cap=round,fill=fillColor] (123.88,236.18) circle (  1.16);

\path[draw=drawColor,line width= 0.4pt,line join=round,line cap=round,fill=fillColor] (124.62,235.88) circle (  1.16);

\path[draw=drawColor,line width= 0.4pt,line join=round,line cap=round,fill=fillColor] (125.34,235.27) circle (  1.16);

\path[draw=drawColor,line width= 0.4pt,line join=round,line cap=round,fill=fillColor] (126.05,235.21) circle (  1.16);

\path[draw=drawColor,line width= 0.4pt,line join=round,line cap=round,fill=fillColor] (126.74,235.10) circle (  1.16);

\path[draw=drawColor,line width= 0.4pt,line join=round,line cap=round,fill=fillColor] (127.43,234.88) circle (  1.16);

\path[draw=drawColor,line width= 0.4pt,line join=round,line cap=round,fill=fillColor] (128.10,234.67) circle (  1.16);

\path[draw=drawColor,line width= 0.4pt,line join=round,line cap=round,fill=fillColor] (128.76,233.05) circle (  1.16);

\path[draw=drawColor,line width= 0.4pt,line join=round,line cap=round,fill=fillColor] (129.40,232.88) circle (  1.16);

\path[draw=drawColor,line width= 0.4pt,line join=round,line cap=round,fill=fillColor] (130.04,232.43) circle (  1.16);

\path[draw=drawColor,line width= 0.4pt,line join=round,line cap=round,fill=fillColor] (130.67,232.40) circle (  1.16);

\path[draw=drawColor,line width= 0.4pt,line join=round,line cap=round,fill=fillColor] (131.28,232.23) circle (  1.16);

\path[draw=drawColor,line width= 0.4pt,line join=round,line cap=round,fill=fillColor] (131.89,232.05) circle (  1.16);

\path[draw=drawColor,line width= 0.4pt,line join=round,line cap=round,fill=fillColor] (132.49,232.02) circle (  1.16);

\path[draw=drawColor,line width= 0.4pt,line join=round,line cap=round,fill=fillColor] (133.08,231.55) circle (  1.16);

\path[draw=drawColor,line width= 0.4pt,line join=round,line cap=round,fill=fillColor] (133.66,231.47) circle (  1.16);

\path[draw=drawColor,line width= 0.4pt,line join=round,line cap=round,fill=fillColor] (134.23,231.35) circle (  1.16);

\path[draw=drawColor,line width= 0.4pt,line join=round,line cap=round,fill=fillColor] (134.80,230.91) circle (  1.16);

\path[draw=drawColor,line width= 0.4pt,line join=round,line cap=round,fill=fillColor] (135.35,230.53) circle (  1.16);

\path[draw=drawColor,line width= 0.4pt,line join=round,line cap=round,fill=fillColor] (135.90,230.22) circle (  1.16);

\path[draw=drawColor,line width= 0.4pt,line join=round,line cap=round,fill=fillColor] (136.45,229.53) circle (  1.16);

\path[draw=drawColor,line width= 0.4pt,line join=round,line cap=round,fill=fillColor] (136.98,229.46) circle (  1.16);

\path[draw=drawColor,line width= 0.4pt,line join=round,line cap=round,fill=fillColor] (137.51,229.38) circle (  1.16);

\path[draw=drawColor,line width= 0.4pt,line join=round,line cap=round,fill=fillColor] (138.03,229.15) circle (  1.16);

\path[draw=drawColor,line width= 0.4pt,line join=round,line cap=round,fill=fillColor] (138.55,229.06) circle (  1.16);

\path[draw=drawColor,line width= 0.4pt,line join=round,line cap=round,fill=fillColor] (139.06,229.00) circle (  1.16);

\path[draw=drawColor,line width= 0.4pt,line join=round,line cap=round,fill=fillColor] (139.56,228.95) circle (  1.16);

\path[draw=drawColor,line width= 0.4pt,line join=round,line cap=round,fill=fillColor] (140.06,228.82) circle (  1.16);

\path[draw=drawColor,line width= 0.4pt,line join=round,line cap=round,fill=fillColor] (140.55,228.82) circle (  1.16);

\path[draw=drawColor,line width= 0.4pt,line join=round,line cap=round,fill=fillColor] (141.04,228.26) circle (  1.16);

\path[draw=drawColor,line width= 0.4pt,line join=round,line cap=round,fill=fillColor] (141.52,228.05) circle (  1.16);

\path[draw=drawColor,line width= 0.4pt,line join=round,line cap=round,fill=fillColor] (142.00,227.66) circle (  1.16);

\path[draw=drawColor,line width= 0.4pt,line join=round,line cap=round,fill=fillColor] (142.47,227.64) circle (  1.16);

\path[draw=drawColor,line width= 0.4pt,line join=round,line cap=round,fill=fillColor] (142.94,227.62) circle (  1.16);

\path[draw=drawColor,line width= 0.4pt,line join=round,line cap=round,fill=fillColor] (143.40,227.61) circle (  1.16);

\path[draw=drawColor,line width= 0.4pt,line join=round,line cap=round,fill=fillColor] (143.86,227.50) circle (  1.16);

\path[draw=drawColor,line width= 0.4pt,line join=round,line cap=round,fill=fillColor] (144.31,227.28) circle (  1.16);

\path[draw=drawColor,line width= 0.4pt,line join=round,line cap=round,fill=fillColor] (144.76,226.95) circle (  1.16);

\path[draw=drawColor,line width= 0.4pt,line join=round,line cap=round,fill=fillColor] (145.20,226.59) circle (  1.16);

\path[draw=drawColor,line width= 0.4pt,line join=round,line cap=round,fill=fillColor] (145.64,226.44) circle (  1.16);

\path[draw=drawColor,line width= 0.4pt,line join=round,line cap=round,fill=fillColor] (146.08,226.17) circle (  1.16);

\path[draw=drawColor,line width= 0.4pt,line join=round,line cap=round,fill=fillColor] (146.51,226.16) circle (  1.16);

\path[draw=drawColor,line width= 0.4pt,line join=round,line cap=round,fill=fillColor] (146.94,225.89) circle (  1.16);

\path[draw=drawColor,line width= 0.4pt,line join=round,line cap=round,fill=fillColor] (147.36,225.69) circle (  1.16);

\path[draw=drawColor,line width= 0.4pt,line join=round,line cap=round,fill=fillColor] (147.79,225.47) circle (  1.16);

\path[draw=drawColor,line width= 0.4pt,line join=round,line cap=round,fill=fillColor] (148.20,225.41) circle (  1.16);

\path[draw=drawColor,line width= 0.4pt,line join=round,line cap=round,fill=fillColor] (148.62,225.37) circle (  1.16);

\path[draw=drawColor,line width= 0.4pt,line join=round,line cap=round,fill=fillColor] (149.03,224.95) circle (  1.16);

\path[draw=drawColor,line width= 0.4pt,line join=round,line cap=round,fill=fillColor] (149.43,224.82) circle (  1.16);

\path[draw=drawColor,line width= 0.4pt,line join=round,line cap=round,fill=fillColor] (149.83,224.73) circle (  1.16);

\path[draw=drawColor,line width= 0.4pt,line join=round,line cap=round,fill=fillColor] (150.23,224.34) circle (  1.16);

\path[draw=drawColor,line width= 0.4pt,line join=round,line cap=round,fill=fillColor] (150.63,224.25) circle (  1.16);

\path[draw=drawColor,line width= 0.4pt,line join=round,line cap=round,fill=fillColor] (151.02,224.21) circle (  1.16);

\path[draw=drawColor,line width= 0.4pt,line join=round,line cap=round,fill=fillColor] (151.41,224.04) circle (  1.16);

\path[draw=drawColor,line width= 0.4pt,line join=round,line cap=round,fill=fillColor] (151.80,223.73) circle (  1.16);

\path[draw=drawColor,line width= 0.4pt,line join=round,line cap=round,fill=fillColor] (152.18,223.72) circle (  1.16);

\path[draw=drawColor,line width= 0.4pt,line join=round,line cap=round,fill=fillColor] (152.57,223.69) circle (  1.16);

\path[draw=drawColor,line width= 0.4pt,line join=round,line cap=round,fill=fillColor] (152.94,223.66) circle (  1.16);

\path[draw=drawColor,line width= 0.4pt,line join=round,line cap=round,fill=fillColor] (153.32,223.53) circle (  1.16);

\path[draw=drawColor,line width= 0.4pt,line join=round,line cap=round,fill=fillColor] (153.69,222.66) circle (  1.16);

\path[draw=drawColor,line width= 0.4pt,line join=round,line cap=round,fill=fillColor] (154.06,222.62) circle (  1.16);

\path[draw=drawColor,line width= 0.4pt,line join=round,line cap=round,fill=fillColor] (154.43,222.50) circle (  1.16);

\path[draw=drawColor,line width= 0.4pt,line join=round,line cap=round,fill=fillColor] (154.79,222.47) circle (  1.16);

\path[draw=drawColor,line width= 0.4pt,line join=round,line cap=round,fill=fillColor] (155.15,222.37) circle (  1.16);

\path[draw=drawColor,line width= 0.4pt,line join=round,line cap=round,fill=fillColor] (155.51,222.24) circle (  1.16);

\path[draw=drawColor,line width= 0.4pt,line join=round,line cap=round,fill=fillColor] (155.87,221.92) circle (  1.16);

\path[draw=drawColor,line width= 0.4pt,line join=round,line cap=round,fill=fillColor] (156.23,221.57) circle (  1.16);

\path[draw=drawColor,line width= 0.4pt,line join=round,line cap=round,fill=fillColor] (156.58,221.19) circle (  1.16);

\path[draw=drawColor,line width= 0.4pt,line join=round,line cap=round,fill=fillColor] (156.93,221.07) circle (  1.16);

\path[draw=drawColor,line width= 0.4pt,line join=round,line cap=round,fill=fillColor] (157.28,220.72) circle (  1.16);

\path[draw=drawColor,line width= 0.4pt,line join=round,line cap=round,fill=fillColor] (157.62,219.77) circle (  1.16);

\path[draw=drawColor,line width= 0.4pt,line join=round,line cap=round,fill=fillColor] (157.96,219.72) circle (  1.16);

\path[draw=drawColor,line width= 0.4pt,line join=round,line cap=round,fill=fillColor] (158.30,219.65) circle (  1.16);

\path[draw=drawColor,line width= 0.4pt,line join=round,line cap=round,fill=fillColor] (158.64,219.48) circle (  1.16);

\path[draw=drawColor,line width= 0.4pt,line join=round,line cap=round,fill=fillColor] (158.98,219.47) circle (  1.16);

\path[draw=drawColor,line width= 0.4pt,line join=round,line cap=round,fill=fillColor] (159.31,216.22) circle (  1.16);

\path[draw=drawColor,line width= 0.4pt,line join=round,line cap=round,fill=fillColor] (159.65,212.56) circle (  1.16);

\path[draw=drawColor,line width= 0.4pt,line join=round,line cap=round,fill=fillColor] (159.98,209.81) circle (  1.16);

\path[draw=drawColor,line width= 0.4pt,line join=round,line cap=round,fill=fillColor] (160.30,209.81) circle (  1.16);

\path[draw=drawColor,line width= 0.4pt,line join=round,line cap=round,fill=fillColor] (160.63,209.81) circle (  1.16);

\path[draw=drawColor,line width= 0.4pt,line join=round,line cap=round,fill=fillColor] (160.95,209.81) circle (  1.16);

\path[draw=drawColor,line width= 0.4pt,line join=round,line cap=round,fill=fillColor] (161.28,209.81) circle (  1.16);

\path[draw=drawColor,line width= 0.4pt,line join=round,line cap=round,fill=fillColor] (161.60,209.81) circle (  1.16);

\path[draw=drawColor,line width= 0.4pt,line join=round,line cap=round,fill=fillColor] (161.92,209.81) circle (  1.16);

\path[draw=drawColor,line width= 0.4pt,line join=round,line cap=round,fill=fillColor] (162.23,209.81) circle (  1.16);

\path[draw=drawColor,line width= 0.4pt,line join=round,line cap=round,fill=fillColor] (162.55,209.81) circle (  1.16);

\path[draw=drawColor,line width= 0.4pt,line join=round,line cap=round,fill=fillColor] (162.86,209.81) circle (  1.16);

\path[draw=drawColor,line width= 0.4pt,line join=round,line cap=round,fill=fillColor] (163.17,209.81) circle (  1.16);

\path[draw=drawColor,line width= 0.4pt,line join=round,line cap=round,fill=fillColor] (163.48,209.81) circle (  1.16);

\path[draw=drawColor,line width= 0.4pt,line join=round,line cap=round,fill=fillColor] (163.79,209.81) circle (  1.16);

\path[draw=drawColor,line width= 0.4pt,line join=round,line cap=round,fill=fillColor] (164.09,209.81) circle (  1.16);

\path[draw=drawColor,line width= 0.4pt,line join=round,line cap=round,fill=fillColor] (164.40,209.81) circle (  1.16);

\path[draw=drawColor,line width= 0.4pt,line join=round,line cap=round,fill=fillColor] (164.70,209.81) circle (  1.16);

\path[draw=drawColor,line width= 0.4pt,line join=round,line cap=round,fill=fillColor] (165.00,209.81) circle (  1.16);

\path[draw=drawColor,line width= 0.4pt,line join=round,line cap=round,fill=fillColor] (165.30,209.81) circle (  1.16);

\path[draw=drawColor,line width= 0.4pt,line join=round,line cap=round,fill=fillColor] (165.60,209.81) circle (  1.16);

\path[draw=drawColor,line width= 0.4pt,line join=round,line cap=round,fill=fillColor] (165.90,209.81) circle (  1.16);

\path[draw=drawColor,line width= 0.4pt,line join=round,line cap=round,fill=fillColor] (166.19,209.81) circle (  1.16);

\path[draw=drawColor,line width= 0.4pt,line join=round,line cap=round,fill=fillColor] (166.49,209.81) circle (  1.16);

\path[draw=drawColor,line width= 0.4pt,line join=round,line cap=round,fill=fillColor] (166.78,209.81) circle (  1.16);

\path[draw=drawColor,line width= 0.4pt,line join=round,line cap=round,fill=fillColor] (167.07,209.81) circle (  1.16);

\path[draw=drawColor,line width= 0.4pt,line join=round,line cap=round,fill=fillColor] (167.36,209.81) circle (  1.16);

\path[draw=drawColor,line width= 0.4pt,line join=round,line cap=round,fill=fillColor] (167.64,209.81) circle (  1.16);

\path[draw=drawColor,line width= 0.4pt,line join=round,line cap=round,fill=fillColor] (167.93,209.81) circle (  1.16);

\path[draw=drawColor,line width= 0.4pt,line join=round,line cap=round,fill=fillColor] (168.22,209.81) circle (  1.16);

\path[draw=drawColor,line width= 0.4pt,line join=round,line cap=round,fill=fillColor] (168.50,209.81) circle (  1.16);

\path[draw=drawColor,line width= 0.4pt,line join=round,line cap=round,fill=fillColor] (168.78,209.81) circle (  1.16);

\path[draw=drawColor,line width= 0.4pt,line join=round,line cap=round,fill=fillColor] (169.06,209.81) circle (  1.16);

\path[draw=drawColor,line width= 0.4pt,line join=round,line cap=round,fill=fillColor] (169.34,209.81) circle (  1.16);

\path[draw=drawColor,line width= 0.4pt,line join=round,line cap=round,fill=fillColor] (169.62,209.81) circle (  1.16);

\path[draw=drawColor,line width= 0.4pt,line join=round,line cap=round,fill=fillColor] (169.89,209.81) circle (  1.16);

\path[draw=drawColor,line width= 0.4pt,line join=round,line cap=round,fill=fillColor] (170.17,209.81) circle (  1.16);

\path[draw=drawColor,line width= 0.4pt,line join=round,line cap=round,fill=fillColor] (170.44,209.81) circle (  1.16);

\path[draw=drawColor,line width= 0.4pt,line join=round,line cap=round,fill=fillColor] (170.72,209.81) circle (  1.16);

\path[draw=drawColor,line width= 0.4pt,line join=round,line cap=round,fill=fillColor] (170.99,209.81) circle (  1.16);

\path[draw=drawColor,line width= 0.4pt,line join=round,line cap=round,fill=fillColor] (171.26,209.81) circle (  1.16);

\path[draw=drawColor,line width= 0.4pt,line join=round,line cap=round,fill=fillColor] (171.53,209.81) circle (  1.16);

\path[draw=drawColor,line width= 0.4pt,line join=round,line cap=round,fill=fillColor] (171.80,209.81) circle (  1.16);

\path[draw=drawColor,line width= 0.4pt,line join=round,line cap=round,fill=fillColor] (172.06,209.81) circle (  1.16);

\path[draw=drawColor,line width= 0.4pt,line join=round,line cap=round,fill=fillColor] (172.33,209.81) circle (  1.16);

\path[draw=drawColor,line width= 0.4pt,line join=round,line cap=round,fill=fillColor] (172.59,209.81) circle (  1.16);

\path[draw=drawColor,line width= 0.4pt,line join=round,line cap=round,fill=fillColor] (172.85,209.81) circle (  1.16);

\path[draw=drawColor,line width= 0.4pt,line join=round,line cap=round,fill=fillColor] (173.12,209.81) circle (  1.16);

\path[draw=drawColor,line width= 0.4pt,line join=round,line cap=round,fill=fillColor] (173.38,209.81) circle (  1.16);

\path[draw=drawColor,line width= 0.4pt,line join=round,line cap=round,fill=fillColor] (173.64,209.81) circle (  1.16);

\path[draw=drawColor,line width= 0.4pt,line join=round,line cap=round,fill=fillColor] (173.90,209.81) circle (  1.16);

\path[draw=drawColor,line width= 0.4pt,line join=round,line cap=round,fill=fillColor] (174.15,209.81) circle (  1.16);

\path[draw=drawColor,line width= 0.4pt,line join=round,line cap=round,fill=fillColor] (174.41,209.81) circle (  1.16);

\path[draw=drawColor,line width= 0.4pt,line join=round,line cap=round,fill=fillColor] (174.67,209.81) circle (  1.16);

\path[draw=drawColor,line width= 0.4pt,line join=round,line cap=round,fill=fillColor] (174.92,209.81) circle (  1.16);

\path[draw=drawColor,line width= 0.4pt,line join=round,line cap=round,fill=fillColor] (175.17,209.81) circle (  1.16);

\path[draw=drawColor,line width= 0.4pt,line join=round,line cap=round,fill=fillColor] (175.42,209.81) circle (  1.16);

\path[draw=drawColor,line width= 0.4pt,line join=round,line cap=round,fill=fillColor] (175.68,209.81) circle (  1.16);

\path[draw=drawColor,line width= 0.4pt,line join=round,line cap=round,fill=fillColor] (175.93,209.81) circle (  1.16);

\path[draw=drawColor,line width= 0.4pt,line join=round,line cap=round,fill=fillColor] (176.18,209.81) circle (  1.16);

\path[draw=drawColor,line width= 0.4pt,line join=round,line cap=round,fill=fillColor] (176.42,209.81) circle (  1.16);

\path[draw=drawColor,line width= 0.4pt,line join=round,line cap=round,fill=fillColor] (176.67,209.81) circle (  1.16);

\path[draw=drawColor,line width= 0.4pt,line join=round,line cap=round,fill=fillColor] (176.92,209.81) circle (  1.16);

\path[draw=drawColor,line width= 0.4pt,line join=round,line cap=round,fill=fillColor] (177.16,209.81) circle (  1.16);

\path[draw=drawColor,line width= 0.4pt,line join=round,line cap=round,fill=fillColor] (177.41,209.81) circle (  1.16);

\path[draw=drawColor,line width= 0.4pt,line join=round,line cap=round,fill=fillColor] (177.65,209.81) circle (  1.16);

\path[draw=drawColor,line width= 0.4pt,line join=round,line cap=round,fill=fillColor] (177.89,209.81) circle (  1.16);

\path[draw=drawColor,line width= 0.4pt,line join=round,line cap=round,fill=fillColor] (178.13,209.81) circle (  1.16);

\path[draw=drawColor,line width= 0.4pt,line join=round,line cap=round,fill=fillColor] (178.37,209.81) circle (  1.16);

\path[draw=drawColor,line width= 0.4pt,line join=round,line cap=round,fill=fillColor] (178.61,209.81) circle (  1.16);

\path[draw=drawColor,line width= 0.4pt,line join=round,line cap=round,fill=fillColor] (178.85,209.81) circle (  1.16);

\path[draw=drawColor,line width= 0.4pt,line join=round,line cap=round,fill=fillColor] (179.09,209.81) circle (  1.16);

\path[draw=drawColor,line width= 0.4pt,line join=round,line cap=round,fill=fillColor] (179.33,209.81) circle (  1.16);

\path[draw=drawColor,line width= 0.4pt,line join=round,line cap=round,fill=fillColor] (179.56,209.81) circle (  1.16);

\path[draw=drawColor,line width= 0.4pt,line join=round,line cap=round,fill=fillColor] (179.80,209.81) circle (  1.16);

\path[draw=drawColor,line width= 0.4pt,line join=round,line cap=round,fill=fillColor] (180.03,209.81) circle (  1.16);

\path[draw=drawColor,line width= 0.4pt,line join=round,line cap=round,fill=fillColor] (180.27,209.81) circle (  1.16);

\path[draw=drawColor,line width= 0.4pt,line join=round,line cap=round,fill=fillColor] (180.50,209.81) circle (  1.16);

\path[draw=drawColor,line width= 0.4pt,line join=round,line cap=round,fill=fillColor] (180.73,209.81) circle (  1.16);

\path[draw=drawColor,line width= 0.4pt,line join=round,line cap=round,fill=fillColor] (180.96,209.81) circle (  1.16);

\path[draw=drawColor,line width= 0.4pt,line join=round,line cap=round,fill=fillColor] (181.19,209.81) circle (  1.16);

\path[draw=drawColor,line width= 0.4pt,line join=round,line cap=round,fill=fillColor] (181.42,209.81) circle (  1.16);

\path[draw=drawColor,line width= 0.4pt,line join=round,line cap=round,fill=fillColor] (181.65,209.81) circle (  1.16);

\path[draw=drawColor,line width= 0.4pt,line join=round,line cap=round,fill=fillColor] (181.88,209.81) circle (  1.16);

\path[draw=drawColor,line width= 0.4pt,line join=round,line cap=round,fill=fillColor] (182.10,209.81) circle (  1.16);

\path[draw=drawColor,line width= 0.4pt,line join=round,line cap=round,fill=fillColor] (182.33,209.81) circle (  1.16);

\path[draw=drawColor,line width= 0.4pt,line join=round,line cap=round,fill=fillColor] (182.55,209.81) circle (  1.16);

\path[draw=drawColor,line width= 0.4pt,line join=round,line cap=round,fill=fillColor] (182.78,209.81) circle (  1.16);

\path[draw=drawColor,line width= 0.4pt,line join=round,line cap=round,fill=fillColor] (183.00,209.81) circle (  1.16);

\path[draw=drawColor,line width= 0.4pt,line join=round,line cap=round,fill=fillColor] (183.23,209.81) circle (  1.16);

\path[draw=drawColor,line width= 0.4pt,line join=round,line cap=round,fill=fillColor] (183.45,209.81) circle (  1.16);

\path[draw=drawColor,line width= 0.4pt,line join=round,line cap=round,fill=fillColor] (183.67,209.81) circle (  1.16);

\path[draw=drawColor,line width= 0.4pt,line join=round,line cap=round,fill=fillColor] (183.89,209.81) circle (  1.16);

\path[draw=drawColor,line width= 0.4pt,line join=round,line cap=round,fill=fillColor] (184.11,209.81) circle (  1.16);

\path[draw=drawColor,line width= 0.4pt,line join=round,line cap=round,fill=fillColor] (184.33,209.81) circle (  1.16);

\path[draw=drawColor,line width= 0.4pt,line join=round,line cap=round,fill=fillColor] (184.55,209.81) circle (  1.16);

\path[draw=drawColor,line width= 0.4pt,line join=round,line cap=round,fill=fillColor] (184.77,209.81) circle (  1.16);

\path[draw=drawColor,line width= 0.4pt,line join=round,line cap=round,fill=fillColor] (184.98,209.81) circle (  1.16);

\path[draw=drawColor,line width= 0.4pt,line join=round,line cap=round,fill=fillColor] (185.20,209.81) circle (  1.16);

\path[draw=drawColor,line width= 0.4pt,line join=round,line cap=round,fill=fillColor] (185.41,209.81) circle (  1.16);

\path[draw=drawColor,line width= 0.4pt,line join=round,line cap=round,fill=fillColor] (185.63,209.81) circle (  1.16);

\path[draw=drawColor,line width= 0.4pt,line join=round,line cap=round,fill=fillColor] (185.84,209.81) circle (  1.16);

\path[draw=drawColor,line width= 0.4pt,line join=round,line cap=round,fill=fillColor] (186.06,209.81) circle (  1.16);

\path[draw=drawColor,line width= 0.4pt,line join=round,line cap=round,fill=fillColor] (186.27,209.81) circle (  1.16);

\path[draw=drawColor,line width= 0.4pt,line join=round,line cap=round,fill=fillColor] (186.48,209.81) circle (  1.16);

\path[draw=drawColor,line width= 0.4pt,line join=round,line cap=round,fill=fillColor] (186.69,209.81) circle (  1.16);

\path[draw=drawColor,line width= 0.4pt,line join=round,line cap=round,fill=fillColor] (186.91,209.81) circle (  1.16);

\path[draw=drawColor,line width= 0.4pt,line join=round,line cap=round,fill=fillColor] (187.12,209.81) circle (  1.16);

\path[draw=drawColor,line width= 0.4pt,line join=round,line cap=round,fill=fillColor] (187.33,209.81) circle (  1.16);

\path[draw=drawColor,line width= 0.4pt,line join=round,line cap=round,fill=fillColor] (187.53,209.81) circle (  1.16);

\path[draw=drawColor,line width= 0.4pt,line join=round,line cap=round,fill=fillColor] (187.74,209.81) circle (  1.16);

\path[draw=drawColor,line width= 0.4pt,line join=round,line cap=round,fill=fillColor] (187.95,209.81) circle (  1.16);

\path[draw=drawColor,line width= 0.4pt,line join=round,line cap=round,fill=fillColor] (188.16,209.81) circle (  1.16);

\path[draw=drawColor,line width= 0.4pt,line join=round,line cap=round,fill=fillColor] (188.36,209.81) circle (  1.16);

\path[draw=drawColor,line width= 0.4pt,line join=round,line cap=round,fill=fillColor] (188.57,209.81) circle (  1.16);

\path[draw=drawColor,line width= 0.4pt,line join=round,line cap=round,fill=fillColor] (188.77,209.81) circle (  1.16);

\path[draw=drawColor,line width= 0.4pt,line join=round,line cap=round,fill=fillColor] (188.98,209.81) circle (  1.16);

\path[draw=drawColor,line width= 0.4pt,line join=round,line cap=round,fill=fillColor] (189.18,209.81) circle (  1.16);

\path[draw=drawColor,line width= 0.4pt,line join=round,line cap=round,fill=fillColor] (189.39,209.81) circle (  1.16);

\path[draw=drawColor,line width= 0.4pt,line join=round,line cap=round,fill=fillColor] (189.59,209.81) circle (  1.16);

\path[draw=drawColor,line width= 0.4pt,line join=round,line cap=round,fill=fillColor] (189.79,209.81) circle (  1.16);

\path[draw=drawColor,line width= 0.4pt,line join=round,line cap=round,fill=fillColor] (189.99,209.81) circle (  1.16);

\path[draw=drawColor,line width= 0.4pt,line join=round,line cap=round,fill=fillColor] (190.19,209.81) circle (  1.16);

\path[draw=drawColor,line width= 0.4pt,line join=round,line cap=round,fill=fillColor] (190.39,209.81) circle (  1.16);

\path[draw=drawColor,line width= 0.4pt,line join=round,line cap=round,fill=fillColor] (190.59,209.81) circle (  1.16);

\path[draw=drawColor,line width= 0.4pt,line join=round,line cap=round,fill=fillColor] (190.79,209.81) circle (  1.16);

\path[draw=drawColor,line width= 0.4pt,line join=round,line cap=round,fill=fillColor] (190.99,209.81) circle (  1.16);

\path[draw=drawColor,line width= 0.4pt,line join=round,line cap=round,fill=fillColor] (191.19,209.81) circle (  1.16);

\path[draw=drawColor,line width= 0.4pt,line join=round,line cap=round,fill=fillColor] (191.39,209.81) circle (  1.16);

\path[draw=drawColor,line width= 0.4pt,line join=round,line cap=round,fill=fillColor] (191.58,209.81) circle (  1.16);

\path[draw=drawColor,line width= 0.4pt,line join=round,line cap=round,fill=fillColor] (191.78,209.81) circle (  1.16);

\path[draw=drawColor,line width= 0.4pt,line join=round,line cap=round,fill=fillColor] (191.98,209.81) circle (  1.16);

\path[draw=drawColor,line width= 0.4pt,line join=round,line cap=round,fill=fillColor] (192.17,209.81) circle (  1.16);

\path[draw=drawColor,line width= 0.4pt,line join=round,line cap=round,fill=fillColor] (192.37,209.81) circle (  1.16);

\path[draw=drawColor,line width= 0.4pt,line join=round,line cap=round,fill=fillColor] (192.56,209.81) circle (  1.16);

\path[draw=drawColor,line width= 0.4pt,line join=round,line cap=round,fill=fillColor] (192.75,209.81) circle (  1.16);

\path[draw=drawColor,line width= 0.4pt,line join=round,line cap=round,fill=fillColor] (192.95,209.81) circle (  1.16);

\path[draw=drawColor,line width= 0.4pt,line join=round,line cap=round,fill=fillColor] (193.14,209.81) circle (  1.16);

\path[draw=drawColor,line width= 0.4pt,line join=round,line cap=round,fill=fillColor] (193.33,209.81) circle (  1.16);

\path[draw=drawColor,line width= 0.4pt,line join=round,line cap=round,fill=fillColor] (193.52,209.81) circle (  1.16);

\path[draw=drawColor,line width= 0.4pt,line join=round,line cap=round,fill=fillColor] (193.71,209.81) circle (  1.16);

\path[draw=drawColor,line width= 0.4pt,line join=round,line cap=round,fill=fillColor] (193.90,209.81) circle (  1.16);

\path[draw=drawColor,line width= 0.4pt,line join=round,line cap=round,fill=fillColor] (194.09,209.81) circle (  1.16);

\path[draw=drawColor,line width= 0.4pt,line join=round,line cap=round,fill=fillColor] (194.28,209.81) circle (  1.16);

\path[draw=drawColor,line width= 0.4pt,line join=round,line cap=round,fill=fillColor] (194.47,209.81) circle (  1.16);

\path[draw=drawColor,line width= 0.4pt,line join=round,line cap=round,fill=fillColor] (194.66,209.81) circle (  1.16);

\path[draw=drawColor,line width= 0.4pt,line join=round,line cap=round,fill=fillColor] (194.85,209.81) circle (  1.16);

\path[draw=drawColor,line width= 0.4pt,line join=round,line cap=round,fill=fillColor] (195.04,209.81) circle (  1.16);

\path[draw=drawColor,line width= 0.4pt,line join=round,line cap=round,fill=fillColor] (195.22,209.81) circle (  1.16);

\path[draw=drawColor,line width= 0.4pt,line join=round,line cap=round,fill=fillColor] (195.41,209.81) circle (  1.16);

\path[draw=drawColor,line width= 0.4pt,line join=round,line cap=round,fill=fillColor] (195.60,209.81) circle (  1.16);

\path[draw=drawColor,line width= 0.4pt,line join=round,line cap=round,fill=fillColor] (195.78,209.81) circle (  1.16);

\path[draw=drawColor,line width= 0.4pt,line join=round,line cap=round,fill=fillColor] (195.97,209.81) circle (  1.16);

\path[draw=drawColor,line width= 0.4pt,line join=round,line cap=round,fill=fillColor] (196.15,209.81) circle (  1.16);

\path[draw=drawColor,line width= 0.4pt,line join=round,line cap=round,fill=fillColor] (196.34,209.81) circle (  1.16);

\path[draw=drawColor,line width= 0.4pt,line join=round,line cap=round,fill=fillColor] (196.52,209.81) circle (  1.16);

\path[draw=drawColor,line width= 0.4pt,line join=round,line cap=round,fill=fillColor] (196.70,209.81) circle (  1.16);

\path[draw=drawColor,line width= 0.4pt,line join=round,line cap=round,fill=fillColor] (196.89,209.81) circle (  1.16);

\path[draw=drawColor,line width= 0.4pt,line join=round,line cap=round,fill=fillColor] (197.07,209.81) circle (  1.16);

\path[draw=drawColor,line width= 0.4pt,line join=round,line cap=round,fill=fillColor] (197.25,209.81) circle (  1.16);

\path[draw=drawColor,line width= 0.4pt,line join=round,line cap=round,fill=fillColor] (197.43,209.81) circle (  1.16);

\path[draw=drawColor,line width= 0.4pt,line join=round,line cap=round,fill=fillColor] (197.61,209.81) circle (  1.16);

\path[draw=drawColor,line width= 0.4pt,line join=round,line cap=round,fill=fillColor] (197.79,209.81) circle (  1.16);

\path[draw=drawColor,line width= 0.4pt,line join=round,line cap=round,fill=fillColor] (197.97,209.81) circle (  1.16);

\path[draw=drawColor,line width= 0.4pt,line join=round,line cap=round,fill=fillColor] (198.15,209.81) circle (  1.16);

\path[draw=drawColor,line width= 0.4pt,line join=round,line cap=round,fill=fillColor] (198.33,209.81) circle (  1.16);

\path[draw=drawColor,line width= 0.4pt,line join=round,line cap=round,fill=fillColor] (198.51,209.81) circle (  1.16);

\path[draw=drawColor,line width= 0.4pt,line join=round,line cap=round,fill=fillColor] (198.69,209.81) circle (  1.16);

\path[draw=drawColor,line width= 0.4pt,line join=round,line cap=round,fill=fillColor] (198.87,209.81) circle (  1.16);

\path[draw=drawColor,line width= 0.4pt,line join=round,line cap=round,fill=fillColor] (199.04,209.81) circle (  1.16);

\path[draw=drawColor,line width= 0.4pt,line join=round,line cap=round,fill=fillColor] (199.22,209.81) circle (  1.16);

\path[draw=drawColor,line width= 0.4pt,line join=round,line cap=round,fill=fillColor] (199.40,209.81) circle (  1.16);

\path[draw=drawColor,line width= 0.4pt,line join=round,line cap=round,fill=fillColor] (199.57,209.81) circle (  1.16);

\path[draw=drawColor,line width= 0.4pt,line join=round,line cap=round,fill=fillColor] (199.75,209.81) circle (  1.16);

\path[draw=drawColor,line width= 0.4pt,line join=round,line cap=round,fill=fillColor] (199.92,209.81) circle (  1.16);

\path[draw=drawColor,line width= 0.4pt,line join=round,line cap=round,fill=fillColor] (200.10,209.81) circle (  1.16);

\path[draw=drawColor,line width= 0.4pt,line join=round,line cap=round,fill=fillColor] (200.27,209.81) circle (  1.16);

\path[draw=drawColor,line width= 0.4pt,line join=round,line cap=round,fill=fillColor] (200.45,209.81) circle (  1.16);

\path[draw=drawColor,line width= 0.4pt,line join=round,line cap=round,fill=fillColor] (200.62,209.81) circle (  1.16);

\path[draw=drawColor,line width= 0.4pt,line join=round,line cap=round,fill=fillColor] (200.79,209.81) circle (  1.16);

\path[draw=drawColor,line width= 0.4pt,line join=round,line cap=round,fill=fillColor] (200.97,209.81) circle (  1.16);

\path[draw=drawColor,line width= 0.4pt,line join=round,line cap=round,fill=fillColor] (201.14,209.81) circle (  1.16);

\path[draw=drawColor,line width= 0.4pt,line join=round,line cap=round,fill=fillColor] (201.31,209.81) circle (  1.16);

\path[draw=drawColor,line width= 0.4pt,line join=round,line cap=round,fill=fillColor] (201.48,209.81) circle (  1.16);

\path[draw=drawColor,line width= 0.4pt,line join=round,line cap=round,fill=fillColor] (201.65,209.81) circle (  1.16);

\path[draw=drawColor,line width= 0.4pt,line join=round,line cap=round,fill=fillColor] (201.83,209.81) circle (  1.16);

\path[draw=drawColor,line width= 0.4pt,line join=round,line cap=round,fill=fillColor] (202.00,209.81) circle (  1.16);

\path[draw=drawColor,line width= 0.4pt,line join=round,line cap=round,fill=fillColor] (202.17,209.81) circle (  1.16);

\path[draw=drawColor,line width= 0.4pt,line join=round,line cap=round,fill=fillColor] (202.34,209.81) circle (  1.16);

\path[draw=drawColor,line width= 0.4pt,line join=round,line cap=round,fill=fillColor] (202.50,209.81) circle (  1.16);

\path[draw=drawColor,line width= 0.4pt,line join=round,line cap=round,fill=fillColor] (202.67,209.81) circle (  1.16);

\path[draw=drawColor,line width= 0.4pt,line join=round,line cap=round,fill=fillColor] (202.84,209.81) circle (  1.16);

\path[draw=drawColor,line width= 0.4pt,line join=round,line cap=round,fill=fillColor] (203.01,209.81) circle (  1.16);

\path[draw=drawColor,line width= 0.4pt,line join=round,line cap=round,fill=fillColor] (203.18,209.81) circle (  1.16);

\path[draw=drawColor,line width= 0.4pt,line join=round,line cap=round,fill=fillColor] (203.35,209.81) circle (  1.16);

\path[draw=drawColor,line width= 0.4pt,line join=round,line cap=round,fill=fillColor] (203.51,209.81) circle (  1.16);

\path[draw=drawColor,line width= 0.4pt,line join=round,line cap=round,fill=fillColor] (203.68,209.81) circle (  1.16);

\path[draw=drawColor,line width= 0.4pt,line join=round,line cap=round,fill=fillColor] (203.85,209.81) circle (  1.16);

\path[draw=drawColor,line width= 0.4pt,line join=round,line cap=round,fill=fillColor] (204.01,209.81) circle (  1.16);

\path[draw=drawColor,line width= 0.4pt,line join=round,line cap=round,fill=fillColor] (204.18,209.81) circle (  1.16);

\path[draw=drawColor,line width= 0.4pt,line join=round,line cap=round,fill=fillColor] (204.34,209.81) circle (  1.16);

\path[draw=drawColor,line width= 0.4pt,line join=round,line cap=round,fill=fillColor] (204.51,209.81) circle (  1.16);

\path[draw=drawColor,line width= 0.4pt,line join=round,line cap=round,fill=fillColor] (204.67,209.81) circle (  1.16);

\path[draw=drawColor,line width= 0.4pt,line join=round,line cap=round,fill=fillColor] (204.84,209.81) circle (  1.16);

\path[draw=drawColor,line width= 0.4pt,line join=round,line cap=round,fill=fillColor] (205.00,209.81) circle (  1.16);

\path[draw=drawColor,line width= 0.4pt,line join=round,line cap=round,fill=fillColor] (205.16,209.81) circle (  1.16);

\path[draw=drawColor,line width= 0.4pt,line join=round,line cap=round,fill=fillColor] (205.33,209.81) circle (  1.16);

\path[draw=drawColor,line width= 0.4pt,line join=round,line cap=round,fill=fillColor] (205.49,209.81) circle (  1.16);

\path[draw=drawColor,line width= 0.4pt,line join=round,line cap=round,fill=fillColor] (205.65,209.81) circle (  1.16);

\path[draw=drawColor,line width= 0.4pt,line join=round,line cap=round,fill=fillColor] (205.81,209.81) circle (  1.16);

\path[draw=drawColor,line width= 0.4pt,line join=round,line cap=round,fill=fillColor] (205.97,209.81) circle (  1.16);

\path[draw=drawColor,line width= 0.4pt,line join=round,line cap=round,fill=fillColor] (206.14,209.81) circle (  1.16);

\path[draw=drawColor,line width= 0.4pt,line join=round,line cap=round,fill=fillColor] (206.30,209.81) circle (  1.16);

\path[draw=drawColor,line width= 0.4pt,line join=round,line cap=round,fill=fillColor] (206.46,209.81) circle (  1.16);

\path[draw=drawColor,line width= 0.4pt,line join=round,line cap=round,fill=fillColor] (206.62,209.81) circle (  1.16);

\path[draw=drawColor,line width= 0.4pt,line join=round,line cap=round,fill=fillColor] (206.78,209.81) circle (  1.16);

\path[draw=drawColor,line width= 0.4pt,line join=round,line cap=round,fill=fillColor] (206.94,209.81) circle (  1.16);

\path[draw=drawColor,line width= 0.4pt,line join=round,line cap=round,fill=fillColor] (207.10,209.81) circle (  1.16);

\path[draw=drawColor,line width= 0.4pt,line join=round,line cap=round,fill=fillColor] (207.26,209.81) circle (  1.16);

\path[draw=drawColor,line width= 0.4pt,line join=round,line cap=round,fill=fillColor] (207.42,209.81) circle (  1.16);

\path[draw=drawColor,line width= 0.4pt,line join=round,line cap=round,fill=fillColor] (207.57,209.81) circle (  1.16);

\path[draw=drawColor,line width= 0.4pt,line join=round,line cap=round,fill=fillColor] (207.73,209.81) circle (  1.16);

\path[draw=drawColor,line width= 0.4pt,line join=round,line cap=round,fill=fillColor] (207.89,209.81) circle (  1.16);

\path[draw=drawColor,line width= 0.4pt,line join=round,line cap=round,fill=fillColor] (208.05,209.81) circle (  1.16);

\path[draw=drawColor,line width= 0.4pt,line join=round,line cap=round,fill=fillColor] (208.20,209.81) circle (  1.16);

\path[draw=drawColor,line width= 0.4pt,line join=round,line cap=round,fill=fillColor] (208.36,209.81) circle (  1.16);

\path[draw=drawColor,line width= 0.4pt,line join=round,line cap=round,fill=fillColor] (208.52,209.81) circle (  1.16);

\path[draw=drawColor,line width= 0.4pt,line join=round,line cap=round,fill=fillColor] (208.67,209.81) circle (  1.16);

\path[draw=drawColor,line width= 0.4pt,line join=round,line cap=round,fill=fillColor] (208.83,209.81) circle (  1.16);

\path[draw=drawColor,line width= 0.4pt,line join=round,line cap=round,fill=fillColor] (208.98,209.81) circle (  1.16);

\path[draw=drawColor,line width= 0.4pt,line join=round,line cap=round,fill=fillColor] (209.14,209.81) circle (  1.16);

\path[draw=drawColor,line width= 0.4pt,line join=round,line cap=round,fill=fillColor] (209.29,209.81) circle (  1.16);

\path[draw=drawColor,line width= 0.4pt,line join=round,line cap=round,fill=fillColor] (209.45,209.81) circle (  1.16);

\path[draw=drawColor,line width= 0.4pt,line join=round,line cap=round,fill=fillColor] (209.60,209.81) circle (  1.16);

\path[draw=drawColor,line width= 0.4pt,line join=round,line cap=round,fill=fillColor] (209.76,209.81) circle (  1.16);

\path[draw=drawColor,line width= 0.4pt,line join=round,line cap=round,fill=fillColor] (209.91,209.81) circle (  1.16);

\path[draw=drawColor,line width= 0.4pt,line join=round,line cap=round,fill=fillColor] (210.07,209.81) circle (  1.16);

\path[draw=drawColor,line width= 0.4pt,line join=round,line cap=round,fill=fillColor] (210.22,209.81) circle (  1.16);

\path[draw=drawColor,line width= 0.4pt,line join=round,line cap=round,fill=fillColor] (210.37,209.81) circle (  1.16);

\path[draw=drawColor,line width= 0.4pt,line join=round,line cap=round,fill=fillColor] (210.52,209.81) circle (  1.16);

\path[draw=drawColor,line width= 0.4pt,line join=round,line cap=round,fill=fillColor] (210.68,209.81) circle (  1.16);

\path[draw=drawColor,line width= 0.4pt,line join=round,line cap=round,fill=fillColor] (210.83,209.81) circle (  1.16);

\path[draw=drawColor,line width= 0.4pt,line join=round,line cap=round,fill=fillColor] (210.98,209.81) circle (  1.16);

\path[draw=drawColor,line width= 0.4pt,line join=round,line cap=round,fill=fillColor] (211.13,209.81) circle (  1.16);

\path[draw=drawColor,line width= 0.4pt,line join=round,line cap=round,fill=fillColor] (211.28,209.81) circle (  1.16);

\path[draw=drawColor,line width= 0.4pt,line join=round,line cap=round,fill=fillColor] (211.43,209.81) circle (  1.16);

\path[draw=drawColor,line width= 0.4pt,line join=round,line cap=round,fill=fillColor] (211.58,209.81) circle (  1.16);

\path[draw=drawColor,line width= 0.4pt,line join=round,line cap=round,fill=fillColor] (211.73,209.81) circle (  1.16);

\path[draw=drawColor,line width= 0.4pt,line join=round,line cap=round,fill=fillColor] (211.88,209.81) circle (  1.16);

\path[draw=drawColor,line width= 0.4pt,line join=round,line cap=round,fill=fillColor] (212.03,209.81) circle (  1.16);

\path[draw=drawColor,line width= 0.4pt,line join=round,line cap=round,fill=fillColor] (212.18,209.81) circle (  1.16);

\path[draw=drawColor,line width= 0.4pt,line join=round,line cap=round,fill=fillColor] (212.33,209.81) circle (  1.16);

\path[draw=drawColor,line width= 0.4pt,line join=round,line cap=round,fill=fillColor] (212.48,209.81) circle (  1.16);

\path[draw=drawColor,line width= 0.4pt,line join=round,line cap=round,fill=fillColor] (212.63,209.81) circle (  1.16);

\path[draw=drawColor,line width= 0.4pt,line join=round,line cap=round,fill=fillColor] (212.78,209.81) circle (  1.16);

\path[draw=drawColor,line width= 0.4pt,line join=round,line cap=round,fill=fillColor] (212.93,209.81) circle (  1.16);

\path[draw=drawColor,line width= 0.4pt,line join=round,line cap=round,fill=fillColor] (213.07,209.81) circle (  1.16);

\path[draw=drawColor,line width= 0.4pt,line join=round,line cap=round,fill=fillColor] (213.22,209.81) circle (  1.16);

\path[draw=drawColor,line width= 0.4pt,line join=round,line cap=round,fill=fillColor] (213.37,209.81) circle (  1.16);

\path[draw=drawColor,line width= 0.4pt,line join=round,line cap=round,fill=fillColor] (213.51,209.81) circle (  1.16);

\path[draw=drawColor,line width= 0.4pt,line join=round,line cap=round,fill=fillColor] (213.66,209.81) circle (  1.16);

\path[draw=drawColor,line width= 0.4pt,line join=round,line cap=round,fill=fillColor] (213.81,209.81) circle (  1.16);

\path[draw=drawColor,line width= 0.4pt,line join=round,line cap=round,fill=fillColor] (213.95,209.81) circle (  1.16);

\path[draw=drawColor,line width= 0.4pt,line join=round,line cap=round,fill=fillColor] (214.10,209.81) circle (  1.16);

\path[draw=drawColor,line width= 0.4pt,line join=round,line cap=round,fill=fillColor] (214.24,209.81) circle (  1.16);

\path[draw=drawColor,line width= 0.4pt,line join=round,line cap=round,fill=fillColor] (214.39,209.81) circle (  1.16);

\path[draw=drawColor,line width= 0.4pt,line join=round,line cap=round,fill=fillColor] (214.54,209.81) circle (  1.16);

\path[draw=drawColor,line width= 0.4pt,line join=round,line cap=round,fill=fillColor] (214.68,209.81) circle (  1.16);

\path[draw=drawColor,line width= 0.4pt,line join=round,line cap=round,fill=fillColor] (214.82,209.81) circle (  1.16);

\path[draw=drawColor,line width= 0.4pt,line join=round,line cap=round,fill=fillColor] (214.97,209.81) circle (  1.16);

\path[draw=drawColor,line width= 0.4pt,line join=round,line cap=round,fill=fillColor] (215.11,209.81) circle (  1.16);

\path[draw=drawColor,line width= 0.4pt,line join=round,line cap=round,fill=fillColor] (215.26,209.81) circle (  1.16);

\path[draw=drawColor,line width= 0.4pt,line join=round,line cap=round,fill=fillColor] (215.40,209.81) circle (  1.16);

\path[draw=drawColor,line width= 0.4pt,line join=round,line cap=round,fill=fillColor] (215.54,209.81) circle (  1.16);

\path[draw=drawColor,line width= 0.4pt,line join=round,line cap=round,fill=fillColor] (215.69,209.81) circle (  1.16);

\path[draw=drawColor,line width= 0.4pt,line join=round,line cap=round,fill=fillColor] (215.83,209.81) circle (  1.16);

\path[draw=drawColor,line width= 0.4pt,line join=round,line cap=round,fill=fillColor] (215.97,209.81) circle (  1.16);

\path[draw=drawColor,line width= 0.4pt,line join=round,line cap=round,fill=fillColor] (216.11,209.81) circle (  1.16);

\path[draw=drawColor,line width= 0.4pt,line join=round,line cap=round,fill=fillColor] (216.26,209.81) circle (  1.16);

\path[draw=drawColor,line width= 0.4pt,line join=round,line cap=round,fill=fillColor] (216.40,209.81) circle (  1.16);

\path[draw=drawColor,line width= 0.4pt,line join=round,line cap=round,fill=fillColor] (216.54,209.81) circle (  1.16);

\path[draw=drawColor,line width= 0.4pt,line join=round,line cap=round,fill=fillColor] (216.68,209.81) circle (  1.16);

\path[draw=drawColor,line width= 0.4pt,line join=round,line cap=round,fill=fillColor] (216.82,209.81) circle (  1.16);

\path[draw=drawColor,line width= 0.4pt,line join=round,line cap=round,fill=fillColor] (216.96,209.81) circle (  1.16);

\path[draw=drawColor,line width= 0.4pt,line join=round,line cap=round,fill=fillColor] (217.10,209.81) circle (  1.16);

\path[draw=drawColor,line width= 0.4pt,line join=round,line cap=round,fill=fillColor] (217.24,209.81) circle (  1.16);

\path[draw=drawColor,line width= 0.4pt,line join=round,line cap=round,fill=fillColor] (217.38,209.81) circle (  1.16);

\path[draw=drawColor,line width= 0.4pt,line join=round,line cap=round,fill=fillColor] (217.52,209.81) circle (  1.16);

\path[draw=drawColor,line width= 0.4pt,line join=round,line cap=round,fill=fillColor] (217.66,209.81) circle (  1.16);

\path[draw=drawColor,line width= 0.4pt,line join=round,line cap=round,fill=fillColor] (217.80,209.81) circle (  1.16);

\path[draw=drawColor,line width= 0.4pt,line join=round,line cap=round,fill=fillColor] (217.94,209.81) circle (  1.16);

\path[draw=drawColor,line width= 0.4pt,line join=round,line cap=round,fill=fillColor] (218.08,209.81) circle (  1.16);

\path[draw=drawColor,line width= 0.4pt,line join=round,line cap=round,fill=fillColor] (218.22,209.81) circle (  1.16);

\path[draw=drawColor,line width= 0.4pt,line join=round,line cap=round,fill=fillColor] (218.36,209.81) circle (  1.16);

\path[draw=drawColor,line width= 0.4pt,line join=round,line cap=round,fill=fillColor] (218.50,209.81) circle (  1.16);

\path[draw=drawColor,line width= 0.4pt,line join=round,line cap=round,fill=fillColor] (218.63,209.81) circle (  1.16);

\path[draw=drawColor,line width= 0.4pt,line join=round,line cap=round,fill=fillColor] (218.77,209.81) circle (  1.16);

\path[draw=drawColor,line width= 0.4pt,line join=round,line cap=round,fill=fillColor] (218.91,209.81) circle (  1.16);

\path[draw=drawColor,line width= 0.4pt,line join=round,line cap=round,fill=fillColor] (219.05,209.81) circle (  1.16);

\path[draw=drawColor,line width= 0.4pt,line join=round,line cap=round,fill=fillColor] (219.18,209.81) circle (  1.16);

\path[draw=drawColor,line width= 0.4pt,line join=round,line cap=round,fill=fillColor] (219.32,209.81) circle (  1.16);

\path[draw=drawColor,line width= 0.4pt,line join=round,line cap=round,fill=fillColor] (219.46,209.81) circle (  1.16);

\path[draw=drawColor,line width= 0.4pt,line join=round,line cap=round,fill=fillColor] (219.59,209.81) circle (  1.16);

\path[draw=drawColor,line width= 0.4pt,line join=round,line cap=round,fill=fillColor] (219.73,209.81) circle (  1.16);

\path[draw=drawColor,line width= 0.4pt,line join=round,line cap=round,fill=fillColor] (219.87,209.81) circle (  1.16);

\path[draw=drawColor,line width= 0.4pt,line join=round,line cap=round,fill=fillColor] (220.00,209.81) circle (  1.16);

\path[draw=drawColor,line width= 0.4pt,line join=round,line cap=round,fill=fillColor] (220.14,209.81) circle (  1.16);

\path[draw=drawColor,line width= 0.4pt,line join=round,line cap=round,fill=fillColor] (220.27,209.81) circle (  1.16);

\path[draw=drawColor,line width= 0.4pt,line join=round,line cap=round,fill=fillColor] (220.41,209.81) circle (  1.16);

\path[draw=drawColor,line width= 0.4pt,line join=round,line cap=round,fill=fillColor] (220.54,209.81) circle (  1.16);

\path[draw=drawColor,line width= 0.4pt,line join=round,line cap=round,fill=fillColor] (220.68,209.81) circle (  1.16);

\path[draw=drawColor,line width= 0.4pt,line join=round,line cap=round,fill=fillColor] (220.81,209.81) circle (  1.16);

\path[draw=drawColor,line width= 0.4pt,line join=round,line cap=round,fill=fillColor] (220.95,209.81) circle (  1.16);

\path[draw=drawColor,line width= 0.4pt,line join=round,line cap=round,fill=fillColor] (221.08,209.81) circle (  1.16);

\path[draw=drawColor,line width= 0.4pt,line join=round,line cap=round,fill=fillColor] (221.21,209.81) circle (  1.16);

\path[draw=drawColor,line width= 0.4pt,line join=round,line cap=round,fill=fillColor] (221.35,209.81) circle (  1.16);

\path[draw=drawColor,line width= 0.4pt,line join=round,line cap=round,fill=fillColor] (221.48,209.81) circle (  1.16);

\path[draw=drawColor,line width= 0.4pt,line join=round,line cap=round,fill=fillColor] (221.61,209.81) circle (  1.16);

\path[draw=drawColor,line width= 0.4pt,line join=round,line cap=round,fill=fillColor] (221.75,209.81) circle (  1.16);

\path[draw=drawColor,line width= 0.4pt,line join=round,line cap=round,fill=fillColor] (221.88,209.81) circle (  1.16);

\path[draw=drawColor,line width= 0.4pt,line join=round,line cap=round,fill=fillColor] (222.01,209.81) circle (  1.16);

\path[draw=drawColor,line width= 0.4pt,line join=round,line cap=round,fill=fillColor] (222.14,209.81) circle (  1.16);

\path[draw=drawColor,line width= 0.4pt,line join=round,line cap=round,fill=fillColor] (222.28,209.81) circle (  1.16);

\path[draw=drawColor,line width= 0.4pt,line join=round,line cap=round,fill=fillColor] (222.41,209.81) circle (  1.16);

\path[draw=drawColor,line width= 0.4pt,line join=round,line cap=round,fill=fillColor] (222.54,209.81) circle (  1.16);

\path[draw=drawColor,line width= 0.4pt,line join=round,line cap=round,fill=fillColor] (222.67,209.81) circle (  1.16);

\path[draw=drawColor,line width= 0.4pt,line join=round,line cap=round,fill=fillColor] (222.80,209.81) circle (  1.16);

\path[draw=drawColor,line width= 0.4pt,line join=round,line cap=round,fill=fillColor] (222.93,209.81) circle (  1.16);

\path[draw=drawColor,line width= 0.4pt,line join=round,line cap=round,fill=fillColor] (223.07,209.81) circle (  1.16);

\path[draw=drawColor,line width= 0.4pt,line join=round,line cap=round,fill=fillColor] (223.20,209.81) circle (  1.16);

\path[draw=drawColor,line width= 0.4pt,line join=round,line cap=round,fill=fillColor] (223.33,209.81) circle (  1.16);

\path[draw=drawColor,line width= 0.4pt,line join=round,line cap=round,fill=fillColor] (223.46,209.81) circle (  1.16);

\path[draw=drawColor,line width= 0.4pt,line join=round,line cap=round,fill=fillColor] (223.59,209.81) circle (  1.16);

\path[draw=drawColor,line width= 0.4pt,line join=round,line cap=round,fill=fillColor] (223.72,209.81) circle (  1.16);

\path[draw=drawColor,line width= 0.4pt,line join=round,line cap=round,fill=fillColor] (223.85,209.81) circle (  1.16);

\path[draw=drawColor,line width= 0.4pt,line join=round,line cap=round,fill=fillColor] (223.98,209.81) circle (  1.16);

\path[draw=drawColor,line width= 0.4pt,line join=round,line cap=round,fill=fillColor] (224.11,209.81) circle (  1.16);

\path[draw=drawColor,line width= 0.4pt,line join=round,line cap=round,fill=fillColor] (224.23,209.81) circle (  1.16);

\path[draw=drawColor,line width= 0.4pt,line join=round,line cap=round,fill=fillColor] (224.36,209.81) circle (  1.16);

\path[draw=drawColor,line width= 0.4pt,line join=round,line cap=round,fill=fillColor] (224.49,209.81) circle (  1.16);

\path[draw=drawColor,line width= 0.4pt,line join=round,line cap=round,fill=fillColor] (224.62,209.81) circle (  1.16);

\path[draw=drawColor,line width= 0.4pt,line join=round,line cap=round,fill=fillColor] (224.75,209.81) circle (  1.16);

\path[draw=drawColor,line width= 0.4pt,line join=round,line cap=round,fill=fillColor] (224.88,209.81) circle (  1.16);

\path[draw=drawColor,line width= 0.4pt,line join=round,line cap=round,fill=fillColor] (225.01,209.81) circle (  1.16);

\path[draw=drawColor,line width= 0.4pt,line join=round,line cap=round,fill=fillColor] (225.13,209.81) circle (  1.16);

\path[draw=drawColor,line width= 0.4pt,line join=round,line cap=round,fill=fillColor] (225.26,209.81) circle (  1.16);
\definecolor[named]{drawColor}{rgb}{0.22,0.49,0.72}
\definecolor[named]{fillColor}{rgb}{0.22,0.49,0.72}

\path[draw=drawColor,line width= 0.4pt,line join=round,line cap=round,fill=fillColor] ( 74.88,292.18) circle (  1.16);

\path[draw=drawColor,line width= 0.4pt,line join=round,line cap=round,fill=fillColor] ( 80.74,290.04) circle (  1.16);

\path[draw=drawColor,line width= 0.4pt,line join=round,line cap=round,fill=fillColor] ( 84.84,289.76) circle (  1.16);

\path[draw=drawColor,line width= 0.4pt,line join=round,line cap=round,fill=fillColor] ( 88.11,289.65) circle (  1.16);

\path[draw=drawColor,line width= 0.4pt,line join=round,line cap=round,fill=fillColor] ( 90.88,289.54) circle (  1.16);

\path[draw=drawColor,line width= 0.4pt,line join=round,line cap=round,fill=fillColor] ( 93.29,288.71) circle (  1.16);

\path[draw=drawColor,line width= 0.4pt,line join=round,line cap=round,fill=fillColor] ( 95.45,288.66) circle (  1.16);

\path[draw=drawColor,line width= 0.4pt,line join=round,line cap=round,fill=fillColor] ( 97.41,288.12) circle (  1.16);

\path[draw=drawColor,line width= 0.4pt,line join=round,line cap=round,fill=fillColor] ( 99.21,287.62) circle (  1.16);

\path[draw=drawColor,line width= 0.4pt,line join=round,line cap=round,fill=fillColor] (100.89,287.38) circle (  1.16);

\path[draw=drawColor,line width= 0.4pt,line join=round,line cap=round,fill=fillColor] (102.46,286.32) circle (  1.16);

\path[draw=drawColor,line width= 0.4pt,line join=round,line cap=round,fill=fillColor] (103.93,283.89) circle (  1.16);

\path[draw=drawColor,line width= 0.4pt,line join=round,line cap=round,fill=fillColor] (105.33,283.00) circle (  1.16);

\path[draw=drawColor,line width= 0.4pt,line join=round,line cap=round,fill=fillColor] (106.65,282.79) circle (  1.16);

\path[draw=drawColor,line width= 0.4pt,line join=round,line cap=round,fill=fillColor] (107.91,282.07) circle (  1.16);

\path[draw=drawColor,line width= 0.4pt,line join=round,line cap=round,fill=fillColor] (109.12,281.20) circle (  1.16);

\path[draw=drawColor,line width= 0.4pt,line join=round,line cap=round,fill=fillColor] (110.28,279.20) circle (  1.16);

\path[draw=drawColor,line width= 0.4pt,line join=round,line cap=round,fill=fillColor] (111.40,278.78) circle (  1.16);

\path[draw=drawColor,line width= 0.4pt,line join=round,line cap=round,fill=fillColor] (112.47,278.16) circle (  1.16);

\path[draw=drawColor,line width= 0.4pt,line join=round,line cap=round,fill=fillColor] (113.51,277.38) circle (  1.16);

\path[draw=drawColor,line width= 0.4pt,line join=round,line cap=round,fill=fillColor] (114.51,277.19) circle (  1.16);

\path[draw=drawColor,line width= 0.4pt,line join=round,line cap=round,fill=fillColor] (115.48,273.55) circle (  1.16);

\path[draw=drawColor,line width= 0.4pt,line join=round,line cap=round,fill=fillColor] (116.42,273.32) circle (  1.16);

\path[draw=drawColor,line width= 0.4pt,line join=round,line cap=round,fill=fillColor] (117.34,271.47) circle (  1.16);

\path[draw=drawColor,line width= 0.4pt,line join=round,line cap=round,fill=fillColor] (118.23,269.67) circle (  1.16);

\path[draw=drawColor,line width= 0.4pt,line join=round,line cap=round,fill=fillColor] (119.10,269.38) circle (  1.16);

\path[draw=drawColor,line width= 0.4pt,line join=round,line cap=round,fill=fillColor] (119.94,268.84) circle (  1.16);

\path[draw=drawColor,line width= 0.4pt,line join=round,line cap=round,fill=fillColor] (120.77,268.80) circle (  1.16);

\path[draw=drawColor,line width= 0.4pt,line join=round,line cap=round,fill=fillColor] (121.57,266.56) circle (  1.16);

\path[draw=drawColor,line width= 0.4pt,line join=round,line cap=round,fill=fillColor] (122.36,264.55) circle (  1.16);

\path[draw=drawColor,line width= 0.4pt,line join=round,line cap=round,fill=fillColor] (123.13,262.50) circle (  1.16);

\path[draw=drawColor,line width= 0.4pt,line join=round,line cap=round,fill=fillColor] (123.88,262.04) circle (  1.16);

\path[draw=drawColor,line width= 0.4pt,line join=round,line cap=round,fill=fillColor] (124.62,261.92) circle (  1.16);

\path[draw=drawColor,line width= 0.4pt,line join=round,line cap=round,fill=fillColor] (125.34,260.63) circle (  1.16);

\path[draw=drawColor,line width= 0.4pt,line join=round,line cap=round,fill=fillColor] (126.05,259.77) circle (  1.16);

\path[draw=drawColor,line width= 0.4pt,line join=round,line cap=round,fill=fillColor] (126.74,258.76) circle (  1.16);

\path[draw=drawColor,line width= 0.4pt,line join=round,line cap=round,fill=fillColor] (127.43,257.89) circle (  1.16);

\path[draw=drawColor,line width= 0.4pt,line join=round,line cap=round,fill=fillColor] (128.10,257.61) circle (  1.16);

\path[draw=drawColor,line width= 0.4pt,line join=round,line cap=round,fill=fillColor] (128.76,257.17) circle (  1.16);

\path[draw=drawColor,line width= 0.4pt,line join=round,line cap=round,fill=fillColor] (129.40,256.62) circle (  1.16);

\path[draw=drawColor,line width= 0.4pt,line join=round,line cap=round,fill=fillColor] (130.04,256.60) circle (  1.16);

\path[draw=drawColor,line width= 0.4pt,line join=round,line cap=round,fill=fillColor] (130.67,256.26) circle (  1.16);

\path[draw=drawColor,line width= 0.4pt,line join=round,line cap=round,fill=fillColor] (131.28,256.08) circle (  1.16);

\path[draw=drawColor,line width= 0.4pt,line join=round,line cap=round,fill=fillColor] (131.89,255.62) circle (  1.16);

\path[draw=drawColor,line width= 0.4pt,line join=round,line cap=round,fill=fillColor] (132.49,255.48) circle (  1.16);

\path[draw=drawColor,line width= 0.4pt,line join=round,line cap=round,fill=fillColor] (133.08,255.42) circle (  1.16);

\path[draw=drawColor,line width= 0.4pt,line join=round,line cap=round,fill=fillColor] (133.66,255.31) circle (  1.16);

\path[draw=drawColor,line width= 0.4pt,line join=round,line cap=round,fill=fillColor] (134.23,254.46) circle (  1.16);

\path[draw=drawColor,line width= 0.4pt,line join=round,line cap=round,fill=fillColor] (134.80,254.22) circle (  1.16);

\path[draw=drawColor,line width= 0.4pt,line join=round,line cap=round,fill=fillColor] (135.35,254.13) circle (  1.16);

\path[draw=drawColor,line width= 0.4pt,line join=round,line cap=round,fill=fillColor] (135.90,253.50) circle (  1.16);

\path[draw=drawColor,line width= 0.4pt,line join=round,line cap=round,fill=fillColor] (136.45,253.27) circle (  1.16);

\path[draw=drawColor,line width= 0.4pt,line join=round,line cap=round,fill=fillColor] (136.98,253.27) circle (  1.16);

\path[draw=drawColor,line width= 0.4pt,line join=round,line cap=round,fill=fillColor] (137.51,253.14) circle (  1.16);

\path[draw=drawColor,line width= 0.4pt,line join=round,line cap=round,fill=fillColor] (138.03,253.10) circle (  1.16);

\path[draw=drawColor,line width= 0.4pt,line join=round,line cap=round,fill=fillColor] (138.55,252.90) circle (  1.16);

\path[draw=drawColor,line width= 0.4pt,line join=round,line cap=round,fill=fillColor] (139.06,252.68) circle (  1.16);

\path[draw=drawColor,line width= 0.4pt,line join=round,line cap=round,fill=fillColor] (139.56,252.67) circle (  1.16);

\path[draw=drawColor,line width= 0.4pt,line join=round,line cap=round,fill=fillColor] (140.06,252.35) circle (  1.16);

\path[draw=drawColor,line width= 0.4pt,line join=round,line cap=round,fill=fillColor] (140.55,252.33) circle (  1.16);

\path[draw=drawColor,line width= 0.4pt,line join=round,line cap=round,fill=fillColor] (141.04,252.27) circle (  1.16);

\path[draw=drawColor,line width= 0.4pt,line join=round,line cap=round,fill=fillColor] (141.52,252.21) circle (  1.16);

\path[draw=drawColor,line width= 0.4pt,line join=round,line cap=round,fill=fillColor] (142.00,252.18) circle (  1.16);

\path[draw=drawColor,line width= 0.4pt,line join=round,line cap=round,fill=fillColor] (142.47,252.01) circle (  1.16);

\path[draw=drawColor,line width= 0.4pt,line join=round,line cap=round,fill=fillColor] (142.94,251.88) circle (  1.16);

\path[draw=drawColor,line width= 0.4pt,line join=round,line cap=round,fill=fillColor] (143.40,251.82) circle (  1.16);

\path[draw=drawColor,line width= 0.4pt,line join=round,line cap=round,fill=fillColor] (143.86,251.75) circle (  1.16);

\path[draw=drawColor,line width= 0.4pt,line join=round,line cap=round,fill=fillColor] (144.31,251.73) circle (  1.16);

\path[draw=drawColor,line width= 0.4pt,line join=round,line cap=round,fill=fillColor] (144.76,251.65) circle (  1.16);

\path[draw=drawColor,line width= 0.4pt,line join=round,line cap=round,fill=fillColor] (145.20,251.55) circle (  1.16);

\path[draw=drawColor,line width= 0.4pt,line join=round,line cap=round,fill=fillColor] (145.64,251.32) circle (  1.16);

\path[draw=drawColor,line width= 0.4pt,line join=round,line cap=round,fill=fillColor] (146.08,251.28) circle (  1.16);

\path[draw=drawColor,line width= 0.4pt,line join=round,line cap=round,fill=fillColor] (146.51,251.07) circle (  1.16);

\path[draw=drawColor,line width= 0.4pt,line join=round,line cap=round,fill=fillColor] (146.94,250.84) circle (  1.16);

\path[draw=drawColor,line width= 0.4pt,line join=round,line cap=round,fill=fillColor] (147.36,250.81) circle (  1.16);

\path[draw=drawColor,line width= 0.4pt,line join=round,line cap=round,fill=fillColor] (147.79,250.75) circle (  1.16);

\path[draw=drawColor,line width= 0.4pt,line join=round,line cap=round,fill=fillColor] (148.20,250.72) circle (  1.16);

\path[draw=drawColor,line width= 0.4pt,line join=round,line cap=round,fill=fillColor] (148.62,250.27) circle (  1.16);

\path[draw=drawColor,line width= 0.4pt,line join=round,line cap=round,fill=fillColor] (149.03,250.24) circle (  1.16);

\path[draw=drawColor,line width= 0.4pt,line join=round,line cap=round,fill=fillColor] (149.43,250.03) circle (  1.16);

\path[draw=drawColor,line width= 0.4pt,line join=round,line cap=round,fill=fillColor] (149.83,249.91) circle (  1.16);

\path[draw=drawColor,line width= 0.4pt,line join=round,line cap=round,fill=fillColor] (150.23,249.73) circle (  1.16);

\path[draw=drawColor,line width= 0.4pt,line join=round,line cap=round,fill=fillColor] (150.63,249.47) circle (  1.16);

\path[draw=drawColor,line width= 0.4pt,line join=round,line cap=round,fill=fillColor] (151.02,249.45) circle (  1.16);

\path[draw=drawColor,line width= 0.4pt,line join=round,line cap=round,fill=fillColor] (151.41,249.35) circle (  1.16);

\path[draw=drawColor,line width= 0.4pt,line join=round,line cap=round,fill=fillColor] (151.80,249.34) circle (  1.16);

\path[draw=drawColor,line width= 0.4pt,line join=round,line cap=round,fill=fillColor] (152.18,249.32) circle (  1.16);

\path[draw=drawColor,line width= 0.4pt,line join=round,line cap=round,fill=fillColor] (152.57,249.20) circle (  1.16);

\path[draw=drawColor,line width= 0.4pt,line join=round,line cap=round,fill=fillColor] (152.94,249.19) circle (  1.16);

\path[draw=drawColor,line width= 0.4pt,line join=round,line cap=round,fill=fillColor] (153.32,248.91) circle (  1.16);

\path[draw=drawColor,line width= 0.4pt,line join=round,line cap=round,fill=fillColor] (153.69,248.80) circle (  1.16);

\path[draw=drawColor,line width= 0.4pt,line join=round,line cap=round,fill=fillColor] (154.06,248.71) circle (  1.16);

\path[draw=drawColor,line width= 0.4pt,line join=round,line cap=round,fill=fillColor] (154.43,248.62) circle (  1.16);

\path[draw=drawColor,line width= 0.4pt,line join=round,line cap=round,fill=fillColor] (154.79,248.57) circle (  1.16);

\path[draw=drawColor,line width= 0.4pt,line join=round,line cap=round,fill=fillColor] (155.15,248.24) circle (  1.16);

\path[draw=drawColor,line width= 0.4pt,line join=round,line cap=round,fill=fillColor] (155.51,248.16) circle (  1.16);

\path[draw=drawColor,line width= 0.4pt,line join=round,line cap=round,fill=fillColor] (155.87,247.92) circle (  1.16);

\path[draw=drawColor,line width= 0.4pt,line join=round,line cap=round,fill=fillColor] (156.23,247.84) circle (  1.16);

\path[draw=drawColor,line width= 0.4pt,line join=round,line cap=round,fill=fillColor] (156.58,247.47) circle (  1.16);

\path[draw=drawColor,line width= 0.4pt,line join=round,line cap=round,fill=fillColor] (156.93,247.11) circle (  1.16);

\path[draw=drawColor,line width= 0.4pt,line join=round,line cap=round,fill=fillColor] (157.28,246.92) circle (  1.16);

\path[draw=drawColor,line width= 0.4pt,line join=round,line cap=round,fill=fillColor] (157.62,246.83) circle (  1.16);

\path[draw=drawColor,line width= 0.4pt,line join=round,line cap=round,fill=fillColor] (157.96,246.69) circle (  1.16);

\path[draw=drawColor,line width= 0.4pt,line join=round,line cap=round,fill=fillColor] (158.30,246.34) circle (  1.16);

\path[draw=drawColor,line width= 0.4pt,line join=round,line cap=round,fill=fillColor] (158.64,246.32) circle (  1.16);

\path[draw=drawColor,line width= 0.4pt,line join=round,line cap=round,fill=fillColor] (158.98,245.93) circle (  1.16);

\path[draw=drawColor,line width= 0.4pt,line join=round,line cap=round,fill=fillColor] (159.31,245.89) circle (  1.16);

\path[draw=drawColor,line width= 0.4pt,line join=round,line cap=round,fill=fillColor] (159.65,245.71) circle (  1.16);

\path[draw=drawColor,line width= 0.4pt,line join=round,line cap=round,fill=fillColor] (159.98,245.50) circle (  1.16);

\path[draw=drawColor,line width= 0.4pt,line join=round,line cap=round,fill=fillColor] (160.30,245.47) circle (  1.16);

\path[draw=drawColor,line width= 0.4pt,line join=round,line cap=round,fill=fillColor] (160.63,245.42) circle (  1.16);

\path[draw=drawColor,line width= 0.4pt,line join=round,line cap=round,fill=fillColor] (160.95,245.03) circle (  1.16);

\path[draw=drawColor,line width= 0.4pt,line join=round,line cap=round,fill=fillColor] (161.28,245.03) circle (  1.16);

\path[draw=drawColor,line width= 0.4pt,line join=round,line cap=round,fill=fillColor] (161.60,245.00) circle (  1.16);

\path[draw=drawColor,line width= 0.4pt,line join=round,line cap=round,fill=fillColor] (161.92,244.78) circle (  1.16);

\path[draw=drawColor,line width= 0.4pt,line join=round,line cap=round,fill=fillColor] (162.23,244.78) circle (  1.16);

\path[draw=drawColor,line width= 0.4pt,line join=round,line cap=round,fill=fillColor] (162.55,244.73) circle (  1.16);

\path[draw=drawColor,line width= 0.4pt,line join=round,line cap=round,fill=fillColor] (162.86,244.70) circle (  1.16);

\path[draw=drawColor,line width= 0.4pt,line join=round,line cap=round,fill=fillColor] (163.17,244.67) circle (  1.16);

\path[draw=drawColor,line width= 0.4pt,line join=round,line cap=round,fill=fillColor] (163.48,244.58) circle (  1.16);

\path[draw=drawColor,line width= 0.4pt,line join=round,line cap=round,fill=fillColor] (163.79,244.52) circle (  1.16);

\path[draw=drawColor,line width= 0.4pt,line join=round,line cap=round,fill=fillColor] (164.09,244.50) circle (  1.16);

\path[draw=drawColor,line width= 0.4pt,line join=round,line cap=round,fill=fillColor] (164.40,244.50) circle (  1.16);

\path[draw=drawColor,line width= 0.4pt,line join=round,line cap=round,fill=fillColor] (164.70,244.50) circle (  1.16);

\path[draw=drawColor,line width= 0.4pt,line join=round,line cap=round,fill=fillColor] (165.00,244.43) circle (  1.16);

\path[draw=drawColor,line width= 0.4pt,line join=round,line cap=round,fill=fillColor] (165.30,244.09) circle (  1.16);

\path[draw=drawColor,line width= 0.4pt,line join=round,line cap=round,fill=fillColor] (165.60,244.08) circle (  1.16);

\path[draw=drawColor,line width= 0.4pt,line join=round,line cap=round,fill=fillColor] (165.90,244.07) circle (  1.16);

\path[draw=drawColor,line width= 0.4pt,line join=round,line cap=round,fill=fillColor] (166.19,244.03) circle (  1.16);

\path[draw=drawColor,line width= 0.4pt,line join=round,line cap=round,fill=fillColor] (166.49,243.81) circle (  1.16);

\path[draw=drawColor,line width= 0.4pt,line join=round,line cap=round,fill=fillColor] (166.78,243.70) circle (  1.16);

\path[draw=drawColor,line width= 0.4pt,line join=round,line cap=round,fill=fillColor] (167.07,243.38) circle (  1.16);

\path[draw=drawColor,line width= 0.4pt,line join=round,line cap=round,fill=fillColor] (167.36,243.34) circle (  1.16);

\path[draw=drawColor,line width= 0.4pt,line join=round,line cap=round,fill=fillColor] (167.64,243.33) circle (  1.16);

\path[draw=drawColor,line width= 0.4pt,line join=round,line cap=round,fill=fillColor] (167.93,243.30) circle (  1.16);

\path[draw=drawColor,line width= 0.4pt,line join=round,line cap=round,fill=fillColor] (168.22,243.04) circle (  1.16);

\path[draw=drawColor,line width= 0.4pt,line join=round,line cap=round,fill=fillColor] (168.50,242.74) circle (  1.16);

\path[draw=drawColor,line width= 0.4pt,line join=round,line cap=round,fill=fillColor] (168.78,242.53) circle (  1.16);

\path[draw=drawColor,line width= 0.4pt,line join=round,line cap=round,fill=fillColor] (169.06,242.45) circle (  1.16);

\path[draw=drawColor,line width= 0.4pt,line join=round,line cap=round,fill=fillColor] (169.34,242.45) circle (  1.16);

\path[draw=drawColor,line width= 0.4pt,line join=round,line cap=round,fill=fillColor] (169.62,242.41) circle (  1.16);

\path[draw=drawColor,line width= 0.4pt,line join=round,line cap=round,fill=fillColor] (169.89,242.34) circle (  1.16);

\path[draw=drawColor,line width= 0.4pt,line join=round,line cap=round,fill=fillColor] (170.17,242.30) circle (  1.16);

\path[draw=drawColor,line width= 0.4pt,line join=round,line cap=round,fill=fillColor] (170.44,242.28) circle (  1.16);

\path[draw=drawColor,line width= 0.4pt,line join=round,line cap=round,fill=fillColor] (170.72,242.16) circle (  1.16);

\path[draw=drawColor,line width= 0.4pt,line join=round,line cap=round,fill=fillColor] (170.99,242.03) circle (  1.16);

\path[draw=drawColor,line width= 0.4pt,line join=round,line cap=round,fill=fillColor] (171.26,241.99) circle (  1.16);

\path[draw=drawColor,line width= 0.4pt,line join=round,line cap=round,fill=fillColor] (171.53,241.98) circle (  1.16);

\path[draw=drawColor,line width= 0.4pt,line join=round,line cap=round,fill=fillColor] (171.80,241.97) circle (  1.16);

\path[draw=drawColor,line width= 0.4pt,line join=round,line cap=round,fill=fillColor] (172.06,241.95) circle (  1.16);

\path[draw=drawColor,line width= 0.4pt,line join=round,line cap=round,fill=fillColor] (172.33,241.93) circle (  1.16);

\path[draw=drawColor,line width= 0.4pt,line join=round,line cap=round,fill=fillColor] (172.59,241.78) circle (  1.16);

\path[draw=drawColor,line width= 0.4pt,line join=round,line cap=round,fill=fillColor] (172.85,241.73) circle (  1.16);

\path[draw=drawColor,line width= 0.4pt,line join=round,line cap=round,fill=fillColor] (173.12,241.73) circle (  1.16);

\path[draw=drawColor,line width= 0.4pt,line join=round,line cap=round,fill=fillColor] (173.38,241.71) circle (  1.16);

\path[draw=drawColor,line width= 0.4pt,line join=round,line cap=round,fill=fillColor] (173.64,241.70) circle (  1.16);

\path[draw=drawColor,line width= 0.4pt,line join=round,line cap=round,fill=fillColor] (173.90,241.56) circle (  1.16);

\path[draw=drawColor,line width= 0.4pt,line join=round,line cap=round,fill=fillColor] (174.15,241.54) circle (  1.16);

\path[draw=drawColor,line width= 0.4pt,line join=round,line cap=round,fill=fillColor] (174.41,241.51) circle (  1.16);

\path[draw=drawColor,line width= 0.4pt,line join=round,line cap=round,fill=fillColor] (174.67,241.33) circle (  1.16);

\path[draw=drawColor,line width= 0.4pt,line join=round,line cap=round,fill=fillColor] (174.92,241.29) circle (  1.16);

\path[draw=drawColor,line width= 0.4pt,line join=round,line cap=round,fill=fillColor] (175.17,241.26) circle (  1.16);

\path[draw=drawColor,line width= 0.4pt,line join=round,line cap=round,fill=fillColor] (175.42,241.26) circle (  1.16);

\path[draw=drawColor,line width= 0.4pt,line join=round,line cap=round,fill=fillColor] (175.68,241.23) circle (  1.16);

\path[draw=drawColor,line width= 0.4pt,line join=round,line cap=round,fill=fillColor] (175.93,241.17) circle (  1.16);

\path[draw=drawColor,line width= 0.4pt,line join=round,line cap=round,fill=fillColor] (176.18,241.14) circle (  1.16);

\path[draw=drawColor,line width= 0.4pt,line join=round,line cap=round,fill=fillColor] (176.42,241.00) circle (  1.16);

\path[draw=drawColor,line width= 0.4pt,line join=round,line cap=round,fill=fillColor] (176.67,240.87) circle (  1.16);

\path[draw=drawColor,line width= 0.4pt,line join=round,line cap=round,fill=fillColor] (176.92,240.76) circle (  1.16);

\path[draw=drawColor,line width= 0.4pt,line join=round,line cap=round,fill=fillColor] (177.16,240.75) circle (  1.16);

\path[draw=drawColor,line width= 0.4pt,line join=round,line cap=round,fill=fillColor] (177.41,240.75) circle (  1.16);

\path[draw=drawColor,line width= 0.4pt,line join=round,line cap=round,fill=fillColor] (177.65,240.51) circle (  1.16);

\path[draw=drawColor,line width= 0.4pt,line join=round,line cap=round,fill=fillColor] (177.89,240.50) circle (  1.16);

\path[draw=drawColor,line width= 0.4pt,line join=round,line cap=round,fill=fillColor] (178.13,240.37) circle (  1.16);

\path[draw=drawColor,line width= 0.4pt,line join=round,line cap=round,fill=fillColor] (178.37,240.37) circle (  1.16);

\path[draw=drawColor,line width= 0.4pt,line join=round,line cap=round,fill=fillColor] (178.61,240.35) circle (  1.16);

\path[draw=drawColor,line width= 0.4pt,line join=round,line cap=round,fill=fillColor] (178.85,240.25) circle (  1.16);

\path[draw=drawColor,line width= 0.4pt,line join=round,line cap=round,fill=fillColor] (179.09,240.21) circle (  1.16);

\path[draw=drawColor,line width= 0.4pt,line join=round,line cap=round,fill=fillColor] (179.33,240.19) circle (  1.16);

\path[draw=drawColor,line width= 0.4pt,line join=round,line cap=round,fill=fillColor] (179.56,240.17) circle (  1.16);

\path[draw=drawColor,line width= 0.4pt,line join=round,line cap=round,fill=fillColor] (179.80,240.09) circle (  1.16);

\path[draw=drawColor,line width= 0.4pt,line join=round,line cap=round,fill=fillColor] (180.03,240.06) circle (  1.16);

\path[draw=drawColor,line width= 0.4pt,line join=round,line cap=round,fill=fillColor] (180.27,240.05) circle (  1.16);

\path[draw=drawColor,line width= 0.4pt,line join=round,line cap=round,fill=fillColor] (180.50,240.00) circle (  1.16);

\path[draw=drawColor,line width= 0.4pt,line join=round,line cap=round,fill=fillColor] (180.73,239.99) circle (  1.16);

\path[draw=drawColor,line width= 0.4pt,line join=round,line cap=round,fill=fillColor] (180.96,239.92) circle (  1.16);

\path[draw=drawColor,line width= 0.4pt,line join=round,line cap=round,fill=fillColor] (181.19,239.74) circle (  1.16);

\path[draw=drawColor,line width= 0.4pt,line join=round,line cap=round,fill=fillColor] (181.42,239.72) circle (  1.16);

\path[draw=drawColor,line width= 0.4pt,line join=round,line cap=round,fill=fillColor] (181.65,239.68) circle (  1.16);

\path[draw=drawColor,line width= 0.4pt,line join=round,line cap=round,fill=fillColor] (181.88,239.50) circle (  1.16);

\path[draw=drawColor,line width= 0.4pt,line join=round,line cap=round,fill=fillColor] (182.10,239.47) circle (  1.16);

\path[draw=drawColor,line width= 0.4pt,line join=round,line cap=round,fill=fillColor] (182.33,239.27) circle (  1.16);

\path[draw=drawColor,line width= 0.4pt,line join=round,line cap=round,fill=fillColor] (182.55,239.14) circle (  1.16);

\path[draw=drawColor,line width= 0.4pt,line join=round,line cap=round,fill=fillColor] (182.78,239.08) circle (  1.16);

\path[draw=drawColor,line width= 0.4pt,line join=round,line cap=round,fill=fillColor] (183.00,239.00) circle (  1.16);

\path[draw=drawColor,line width= 0.4pt,line join=round,line cap=round,fill=fillColor] (183.23,238.99) circle (  1.16);

\path[draw=drawColor,line width= 0.4pt,line join=round,line cap=round,fill=fillColor] (183.45,238.91) circle (  1.16);

\path[draw=drawColor,line width= 0.4pt,line join=round,line cap=round,fill=fillColor] (183.67,238.87) circle (  1.16);

\path[draw=drawColor,line width= 0.4pt,line join=round,line cap=round,fill=fillColor] (183.89,238.84) circle (  1.16);

\path[draw=drawColor,line width= 0.4pt,line join=round,line cap=round,fill=fillColor] (184.11,238.81) circle (  1.16);

\path[draw=drawColor,line width= 0.4pt,line join=round,line cap=round,fill=fillColor] (184.33,238.73) circle (  1.16);

\path[draw=drawColor,line width= 0.4pt,line join=round,line cap=round,fill=fillColor] (184.55,238.64) circle (  1.16);

\path[draw=drawColor,line width= 0.4pt,line join=round,line cap=round,fill=fillColor] (184.77,238.62) circle (  1.16);

\path[draw=drawColor,line width= 0.4pt,line join=round,line cap=round,fill=fillColor] (184.98,238.56) circle (  1.16);

\path[draw=drawColor,line width= 0.4pt,line join=round,line cap=round,fill=fillColor] (185.20,238.50) circle (  1.16);

\path[draw=drawColor,line width= 0.4pt,line join=round,line cap=round,fill=fillColor] (185.41,238.48) circle (  1.16);

\path[draw=drawColor,line width= 0.4pt,line join=round,line cap=round,fill=fillColor] (185.63,238.46) circle (  1.16);

\path[draw=drawColor,line width= 0.4pt,line join=round,line cap=round,fill=fillColor] (185.84,238.44) circle (  1.16);

\path[draw=drawColor,line width= 0.4pt,line join=round,line cap=round,fill=fillColor] (186.06,238.34) circle (  1.16);

\path[draw=drawColor,line width= 0.4pt,line join=round,line cap=round,fill=fillColor] (186.27,238.26) circle (  1.16);

\path[draw=drawColor,line width= 0.4pt,line join=round,line cap=round,fill=fillColor] (186.48,238.23) circle (  1.16);

\path[draw=drawColor,line width= 0.4pt,line join=round,line cap=round,fill=fillColor] (186.69,238.21) circle (  1.16);

\path[draw=drawColor,line width= 0.4pt,line join=round,line cap=round,fill=fillColor] (186.91,238.17) circle (  1.16);

\path[draw=drawColor,line width= 0.4pt,line join=round,line cap=round,fill=fillColor] (187.12,238.12) circle (  1.16);

\path[draw=drawColor,line width= 0.4pt,line join=round,line cap=round,fill=fillColor] (187.33,238.06) circle (  1.16);

\path[draw=drawColor,line width= 0.4pt,line join=round,line cap=round,fill=fillColor] (187.53,237.98) circle (  1.16);

\path[draw=drawColor,line width= 0.4pt,line join=round,line cap=round,fill=fillColor] (187.74,237.87) circle (  1.16);

\path[draw=drawColor,line width= 0.4pt,line join=round,line cap=round,fill=fillColor] (187.95,237.87) circle (  1.16);

\path[draw=drawColor,line width= 0.4pt,line join=round,line cap=round,fill=fillColor] (188.16,237.82) circle (  1.16);

\path[draw=drawColor,line width= 0.4pt,line join=round,line cap=round,fill=fillColor] (188.36,237.78) circle (  1.16);

\path[draw=drawColor,line width= 0.4pt,line join=round,line cap=round,fill=fillColor] (188.57,237.72) circle (  1.16);

\path[draw=drawColor,line width= 0.4pt,line join=round,line cap=round,fill=fillColor] (188.77,237.71) circle (  1.16);

\path[draw=drawColor,line width= 0.4pt,line join=round,line cap=round,fill=fillColor] (188.98,237.68) circle (  1.16);

\path[draw=drawColor,line width= 0.4pt,line join=round,line cap=round,fill=fillColor] (189.18,237.65) circle (  1.16);

\path[draw=drawColor,line width= 0.4pt,line join=round,line cap=round,fill=fillColor] (189.39,237.61) circle (  1.16);

\path[draw=drawColor,line width= 0.4pt,line join=round,line cap=round,fill=fillColor] (189.59,237.37) circle (  1.16);

\path[draw=drawColor,line width= 0.4pt,line join=round,line cap=round,fill=fillColor] (189.79,237.35) circle (  1.16);

\path[draw=drawColor,line width= 0.4pt,line join=round,line cap=round,fill=fillColor] (189.99,237.24) circle (  1.16);

\path[draw=drawColor,line width= 0.4pt,line join=round,line cap=round,fill=fillColor] (190.19,237.22) circle (  1.16);

\path[draw=drawColor,line width= 0.4pt,line join=round,line cap=round,fill=fillColor] (190.39,237.20) circle (  1.16);

\path[draw=drawColor,line width= 0.4pt,line join=round,line cap=round,fill=fillColor] (190.59,237.17) circle (  1.16);

\path[draw=drawColor,line width= 0.4pt,line join=round,line cap=round,fill=fillColor] (190.79,237.05) circle (  1.16);

\path[draw=drawColor,line width= 0.4pt,line join=round,line cap=round,fill=fillColor] (190.99,237.00) circle (  1.16);

\path[draw=drawColor,line width= 0.4pt,line join=round,line cap=round,fill=fillColor] (191.19,236.89) circle (  1.16);

\path[draw=drawColor,line width= 0.4pt,line join=round,line cap=round,fill=fillColor] (191.39,236.80) circle (  1.16);

\path[draw=drawColor,line width= 0.4pt,line join=round,line cap=round,fill=fillColor] (191.58,236.78) circle (  1.16);

\path[draw=drawColor,line width= 0.4pt,line join=round,line cap=round,fill=fillColor] (191.78,236.75) circle (  1.16);

\path[draw=drawColor,line width= 0.4pt,line join=round,line cap=round,fill=fillColor] (191.98,236.75) circle (  1.16);

\path[draw=drawColor,line width= 0.4pt,line join=round,line cap=round,fill=fillColor] (192.17,236.70) circle (  1.16);

\path[draw=drawColor,line width= 0.4pt,line join=round,line cap=round,fill=fillColor] (192.37,236.66) circle (  1.16);

\path[draw=drawColor,line width= 0.4pt,line join=round,line cap=round,fill=fillColor] (192.56,236.63) circle (  1.16);

\path[draw=drawColor,line width= 0.4pt,line join=round,line cap=round,fill=fillColor] (192.75,236.60) circle (  1.16);

\path[draw=drawColor,line width= 0.4pt,line join=round,line cap=round,fill=fillColor] (192.95,236.38) circle (  1.16);

\path[draw=drawColor,line width= 0.4pt,line join=round,line cap=round,fill=fillColor] (193.14,236.31) circle (  1.16);

\path[draw=drawColor,line width= 0.4pt,line join=round,line cap=round,fill=fillColor] (193.33,236.27) circle (  1.16);

\path[draw=drawColor,line width= 0.4pt,line join=round,line cap=round,fill=fillColor] (193.52,236.26) circle (  1.16);

\path[draw=drawColor,line width= 0.4pt,line join=round,line cap=round,fill=fillColor] (193.71,236.19) circle (  1.16);

\path[draw=drawColor,line width= 0.4pt,line join=round,line cap=round,fill=fillColor] (193.90,236.15) circle (  1.16);

\path[draw=drawColor,line width= 0.4pt,line join=round,line cap=round,fill=fillColor] (194.09,236.08) circle (  1.16);

\path[draw=drawColor,line width= 0.4pt,line join=round,line cap=round,fill=fillColor] (194.28,235.95) circle (  1.16);

\path[draw=drawColor,line width= 0.4pt,line join=round,line cap=round,fill=fillColor] (194.47,235.95) circle (  1.16);

\path[draw=drawColor,line width= 0.4pt,line join=round,line cap=round,fill=fillColor] (194.66,235.91) circle (  1.16);

\path[draw=drawColor,line width= 0.4pt,line join=round,line cap=round,fill=fillColor] (194.85,235.88) circle (  1.16);

\path[draw=drawColor,line width= 0.4pt,line join=round,line cap=round,fill=fillColor] (195.04,235.88) circle (  1.16);

\path[draw=drawColor,line width= 0.4pt,line join=round,line cap=round,fill=fillColor] (195.22,235.84) circle (  1.16);

\path[draw=drawColor,line width= 0.4pt,line join=round,line cap=round,fill=fillColor] (195.41,235.82) circle (  1.16);

\path[draw=drawColor,line width= 0.4pt,line join=round,line cap=round,fill=fillColor] (195.60,235.81) circle (  1.16);

\path[draw=drawColor,line width= 0.4pt,line join=round,line cap=round,fill=fillColor] (195.78,235.80) circle (  1.16);

\path[draw=drawColor,line width= 0.4pt,line join=round,line cap=round,fill=fillColor] (195.97,235.70) circle (  1.16);

\path[draw=drawColor,line width= 0.4pt,line join=round,line cap=round,fill=fillColor] (196.15,235.66) circle (  1.16);

\path[draw=drawColor,line width= 0.4pt,line join=round,line cap=round,fill=fillColor] (196.34,235.62) circle (  1.16);

\path[draw=drawColor,line width= 0.4pt,line join=round,line cap=round,fill=fillColor] (196.52,235.60) circle (  1.16);

\path[draw=drawColor,line width= 0.4pt,line join=round,line cap=round,fill=fillColor] (196.70,235.53) circle (  1.16);

\path[draw=drawColor,line width= 0.4pt,line join=round,line cap=round,fill=fillColor] (196.89,235.36) circle (  1.16);

\path[draw=drawColor,line width= 0.4pt,line join=round,line cap=round,fill=fillColor] (197.07,235.18) circle (  1.16);

\path[draw=drawColor,line width= 0.4pt,line join=round,line cap=round,fill=fillColor] (197.25,235.14) circle (  1.16);

\path[draw=drawColor,line width= 0.4pt,line join=round,line cap=round,fill=fillColor] (197.43,235.06) circle (  1.16);

\path[draw=drawColor,line width= 0.4pt,line join=round,line cap=round,fill=fillColor] (197.61,234.97) circle (  1.16);

\path[draw=drawColor,line width= 0.4pt,line join=round,line cap=round,fill=fillColor] (197.79,234.81) circle (  1.16);

\path[draw=drawColor,line width= 0.4pt,line join=round,line cap=round,fill=fillColor] (197.97,234.79) circle (  1.16);

\path[draw=drawColor,line width= 0.4pt,line join=round,line cap=round,fill=fillColor] (198.15,234.62) circle (  1.16);

\path[draw=drawColor,line width= 0.4pt,line join=round,line cap=round,fill=fillColor] (198.33,234.50) circle (  1.16);

\path[draw=drawColor,line width= 0.4pt,line join=round,line cap=round,fill=fillColor] (198.51,234.39) circle (  1.16);

\path[draw=drawColor,line width= 0.4pt,line join=round,line cap=round,fill=fillColor] (198.69,234.38) circle (  1.16);

\path[draw=drawColor,line width= 0.4pt,line join=round,line cap=round,fill=fillColor] (198.87,234.36) circle (  1.16);

\path[draw=drawColor,line width= 0.4pt,line join=round,line cap=round,fill=fillColor] (199.04,234.18) circle (  1.16);

\path[draw=drawColor,line width= 0.4pt,line join=round,line cap=round,fill=fillColor] (199.22,234.18) circle (  1.16);

\path[draw=drawColor,line width= 0.4pt,line join=round,line cap=round,fill=fillColor] (199.40,234.18) circle (  1.16);

\path[draw=drawColor,line width= 0.4pt,line join=round,line cap=round,fill=fillColor] (199.57,234.18) circle (  1.16);

\path[draw=drawColor,line width= 0.4pt,line join=round,line cap=round,fill=fillColor] (199.75,234.05) circle (  1.16);

\path[draw=drawColor,line width= 0.4pt,line join=round,line cap=round,fill=fillColor] (199.92,234.04) circle (  1.16);

\path[draw=drawColor,line width= 0.4pt,line join=round,line cap=round,fill=fillColor] (200.10,234.03) circle (  1.16);

\path[draw=drawColor,line width= 0.4pt,line join=round,line cap=round,fill=fillColor] (200.27,234.00) circle (  1.16);

\path[draw=drawColor,line width= 0.4pt,line join=round,line cap=round,fill=fillColor] (200.45,233.98) circle (  1.16);

\path[draw=drawColor,line width= 0.4pt,line join=round,line cap=round,fill=fillColor] (200.62,233.91) circle (  1.16);

\path[draw=drawColor,line width= 0.4pt,line join=round,line cap=round,fill=fillColor] (200.79,233.84) circle (  1.16);

\path[draw=drawColor,line width= 0.4pt,line join=round,line cap=round,fill=fillColor] (200.97,233.82) circle (  1.16);

\path[draw=drawColor,line width= 0.4pt,line join=round,line cap=round,fill=fillColor] (201.14,233.81) circle (  1.16);

\path[draw=drawColor,line width= 0.4pt,line join=round,line cap=round,fill=fillColor] (201.31,233.80) circle (  1.16);

\path[draw=drawColor,line width= 0.4pt,line join=round,line cap=round,fill=fillColor] (201.48,233.73) circle (  1.16);

\path[draw=drawColor,line width= 0.4pt,line join=round,line cap=round,fill=fillColor] (201.65,233.70) circle (  1.16);

\path[draw=drawColor,line width= 0.4pt,line join=round,line cap=round,fill=fillColor] (201.83,233.55) circle (  1.16);

\path[draw=drawColor,line width= 0.4pt,line join=round,line cap=round,fill=fillColor] (202.00,233.36) circle (  1.16);

\path[draw=drawColor,line width= 0.4pt,line join=round,line cap=round,fill=fillColor] (202.17,233.36) circle (  1.16);

\path[draw=drawColor,line width= 0.4pt,line join=round,line cap=round,fill=fillColor] (202.34,233.27) circle (  1.16);

\path[draw=drawColor,line width= 0.4pt,line join=round,line cap=round,fill=fillColor] (202.50,233.14) circle (  1.16);

\path[draw=drawColor,line width= 0.4pt,line join=round,line cap=round,fill=fillColor] (202.67,233.06) circle (  1.16);

\path[draw=drawColor,line width= 0.4pt,line join=round,line cap=round,fill=fillColor] (202.84,233.00) circle (  1.16);

\path[draw=drawColor,line width= 0.4pt,line join=round,line cap=round,fill=fillColor] (203.01,232.88) circle (  1.16);

\path[draw=drawColor,line width= 0.4pt,line join=round,line cap=round,fill=fillColor] (203.18,232.59) circle (  1.16);

\path[draw=drawColor,line width= 0.4pt,line join=round,line cap=round,fill=fillColor] (203.35,232.59) circle (  1.16);

\path[draw=drawColor,line width= 0.4pt,line join=round,line cap=round,fill=fillColor] (203.51,232.50) circle (  1.16);

\path[draw=drawColor,line width= 0.4pt,line join=round,line cap=round,fill=fillColor] (203.68,232.24) circle (  1.16);

\path[draw=drawColor,line width= 0.4pt,line join=round,line cap=round,fill=fillColor] (203.85,232.00) circle (  1.16);

\path[draw=drawColor,line width= 0.4pt,line join=round,line cap=round,fill=fillColor] (204.01,231.99) circle (  1.16);

\path[draw=drawColor,line width= 0.4pt,line join=round,line cap=round,fill=fillColor] (204.18,231.96) circle (  1.16);

\path[draw=drawColor,line width= 0.4pt,line join=round,line cap=round,fill=fillColor] (204.34,231.67) circle (  1.16);

\path[draw=drawColor,line width= 0.4pt,line join=round,line cap=round,fill=fillColor] (204.51,231.58) circle (  1.16);

\path[draw=drawColor,line width= 0.4pt,line join=round,line cap=round,fill=fillColor] (204.67,231.51) circle (  1.16);

\path[draw=drawColor,line width= 0.4pt,line join=round,line cap=round,fill=fillColor] (204.84,231.50) circle (  1.16);

\path[draw=drawColor,line width= 0.4pt,line join=round,line cap=round,fill=fillColor] (205.00,231.24) circle (  1.16);

\path[draw=drawColor,line width= 0.4pt,line join=round,line cap=round,fill=fillColor] (205.16,231.23) circle (  1.16);

\path[draw=drawColor,line width= 0.4pt,line join=round,line cap=round,fill=fillColor] (205.33,231.20) circle (  1.16);

\path[draw=drawColor,line width= 0.4pt,line join=round,line cap=round,fill=fillColor] (205.49,231.17) circle (  1.16);

\path[draw=drawColor,line width= 0.4pt,line join=round,line cap=round,fill=fillColor] (205.65,230.92) circle (  1.16);

\path[draw=drawColor,line width= 0.4pt,line join=round,line cap=round,fill=fillColor] (205.81,230.91) circle (  1.16);

\path[draw=drawColor,line width= 0.4pt,line join=round,line cap=round,fill=fillColor] (205.97,230.69) circle (  1.16);

\path[draw=drawColor,line width= 0.4pt,line join=round,line cap=round,fill=fillColor] (206.14,230.67) circle (  1.16);

\path[draw=drawColor,line width= 0.4pt,line join=round,line cap=round,fill=fillColor] (206.30,230.65) circle (  1.16);

\path[draw=drawColor,line width= 0.4pt,line join=round,line cap=round,fill=fillColor] (206.46,230.62) circle (  1.16);

\path[draw=drawColor,line width= 0.4pt,line join=round,line cap=round,fill=fillColor] (206.62,230.57) circle (  1.16);

\path[draw=drawColor,line width= 0.4pt,line join=round,line cap=round,fill=fillColor] (206.78,230.56) circle (  1.16);

\path[draw=drawColor,line width= 0.4pt,line join=round,line cap=round,fill=fillColor] (206.94,230.52) circle (  1.16);

\path[draw=drawColor,line width= 0.4pt,line join=round,line cap=round,fill=fillColor] (207.10,230.48) circle (  1.16);

\path[draw=drawColor,line width= 0.4pt,line join=round,line cap=round,fill=fillColor] (207.26,230.23) circle (  1.16);

\path[draw=drawColor,line width= 0.4pt,line join=round,line cap=round,fill=fillColor] (207.42,230.18) circle (  1.16);

\path[draw=drawColor,line width= 0.4pt,line join=round,line cap=round,fill=fillColor] (207.57,230.14) circle (  1.16);

\path[draw=drawColor,line width= 0.4pt,line join=round,line cap=round,fill=fillColor] (207.73,229.84) circle (  1.16);

\path[draw=drawColor,line width= 0.4pt,line join=round,line cap=round,fill=fillColor] (207.89,229.76) circle (  1.16);

\path[draw=drawColor,line width= 0.4pt,line join=round,line cap=round,fill=fillColor] (208.05,229.73) circle (  1.16);

\path[draw=drawColor,line width= 0.4pt,line join=round,line cap=round,fill=fillColor] (208.20,229.57) circle (  1.16);

\path[draw=drawColor,line width= 0.4pt,line join=round,line cap=round,fill=fillColor] (208.36,229.53) circle (  1.16);

\path[draw=drawColor,line width= 0.4pt,line join=round,line cap=round,fill=fillColor] (208.52,229.51) circle (  1.16);

\path[draw=drawColor,line width= 0.4pt,line join=round,line cap=round,fill=fillColor] (208.67,229.41) circle (  1.16);

\path[draw=drawColor,line width= 0.4pt,line join=round,line cap=round,fill=fillColor] (208.83,229.33) circle (  1.16);

\path[draw=drawColor,line width= 0.4pt,line join=round,line cap=round,fill=fillColor] (208.98,229.30) circle (  1.16);

\path[draw=drawColor,line width= 0.4pt,line join=round,line cap=round,fill=fillColor] (209.14,229.07) circle (  1.16);

\path[draw=drawColor,line width= 0.4pt,line join=round,line cap=round,fill=fillColor] (209.29,228.93) circle (  1.16);

\path[draw=drawColor,line width= 0.4pt,line join=round,line cap=round,fill=fillColor] (209.45,228.77) circle (  1.16);

\path[draw=drawColor,line width= 0.4pt,line join=round,line cap=round,fill=fillColor] (209.60,228.68) circle (  1.16);

\path[draw=drawColor,line width= 0.4pt,line join=round,line cap=round,fill=fillColor] (209.76,228.67) circle (  1.16);

\path[draw=drawColor,line width= 0.4pt,line join=round,line cap=round,fill=fillColor] (209.91,228.65) circle (  1.16);

\path[draw=drawColor,line width= 0.4pt,line join=round,line cap=round,fill=fillColor] (210.07,228.44) circle (  1.16);

\path[draw=drawColor,line width= 0.4pt,line join=round,line cap=round,fill=fillColor] (210.22,228.37) circle (  1.16);

\path[draw=drawColor,line width= 0.4pt,line join=round,line cap=round,fill=fillColor] (210.37,228.29) circle (  1.16);

\path[draw=drawColor,line width= 0.4pt,line join=round,line cap=round,fill=fillColor] (210.52,227.94) circle (  1.16);

\path[draw=drawColor,line width= 0.4pt,line join=round,line cap=round,fill=fillColor] (210.68,227.91) circle (  1.16);

\path[draw=drawColor,line width= 0.4pt,line join=round,line cap=round,fill=fillColor] (210.83,227.90) circle (  1.16);

\path[draw=drawColor,line width= 0.4pt,line join=round,line cap=round,fill=fillColor] (210.98,227.77) circle (  1.16);

\path[draw=drawColor,line width= 0.4pt,line join=round,line cap=round,fill=fillColor] (211.13,227.76) circle (  1.16);

\path[draw=drawColor,line width= 0.4pt,line join=round,line cap=round,fill=fillColor] (211.28,227.73) circle (  1.16);

\path[draw=drawColor,line width= 0.4pt,line join=round,line cap=round,fill=fillColor] (211.43,227.51) circle (  1.16);

\path[draw=drawColor,line width= 0.4pt,line join=round,line cap=round,fill=fillColor] (211.58,227.50) circle (  1.16);

\path[draw=drawColor,line width= 0.4pt,line join=round,line cap=round,fill=fillColor] (211.73,227.43) circle (  1.16);

\path[draw=drawColor,line width= 0.4pt,line join=round,line cap=round,fill=fillColor] (211.88,227.36) circle (  1.16);

\path[draw=drawColor,line width= 0.4pt,line join=round,line cap=round,fill=fillColor] (212.03,227.35) circle (  1.16);

\path[draw=drawColor,line width= 0.4pt,line join=round,line cap=round,fill=fillColor] (212.18,227.32) circle (  1.16);

\path[draw=drawColor,line width= 0.4pt,line join=round,line cap=round,fill=fillColor] (212.33,227.31) circle (  1.16);

\path[draw=drawColor,line width= 0.4pt,line join=round,line cap=round,fill=fillColor] (212.48,227.25) circle (  1.16);

\path[draw=drawColor,line width= 0.4pt,line join=round,line cap=round,fill=fillColor] (212.63,227.24) circle (  1.16);

\path[draw=drawColor,line width= 0.4pt,line join=round,line cap=round,fill=fillColor] (212.78,227.06) circle (  1.16);

\path[draw=drawColor,line width= 0.4pt,line join=round,line cap=round,fill=fillColor] (212.93,226.93) circle (  1.16);

\path[draw=drawColor,line width= 0.4pt,line join=round,line cap=round,fill=fillColor] (213.07,226.81) circle (  1.16);

\path[draw=drawColor,line width= 0.4pt,line join=round,line cap=round,fill=fillColor] (213.22,226.56) circle (  1.16);

\path[draw=drawColor,line width= 0.4pt,line join=round,line cap=round,fill=fillColor] (213.37,226.30) circle (  1.16);

\path[draw=drawColor,line width= 0.4pt,line join=round,line cap=round,fill=fillColor] (213.51,226.23) circle (  1.16);

\path[draw=drawColor,line width= 0.4pt,line join=round,line cap=round,fill=fillColor] (213.66,225.86) circle (  1.16);

\path[draw=drawColor,line width= 0.4pt,line join=round,line cap=round,fill=fillColor] (213.81,225.57) circle (  1.16);

\path[draw=drawColor,line width= 0.4pt,line join=round,line cap=round,fill=fillColor] (213.95,225.55) circle (  1.16);

\path[draw=drawColor,line width= 0.4pt,line join=round,line cap=round,fill=fillColor] (214.10,225.47) circle (  1.16);

\path[draw=drawColor,line width= 0.4pt,line join=round,line cap=round,fill=fillColor] (214.24,225.39) circle (  1.16);

\path[draw=drawColor,line width= 0.4pt,line join=round,line cap=round,fill=fillColor] (214.39,225.35) circle (  1.16);

\path[draw=drawColor,line width= 0.4pt,line join=round,line cap=round,fill=fillColor] (214.54,225.30) circle (  1.16);

\path[draw=drawColor,line width= 0.4pt,line join=round,line cap=round,fill=fillColor] (214.68,225.29) circle (  1.16);

\path[draw=drawColor,line width= 0.4pt,line join=round,line cap=round,fill=fillColor] (214.82,225.04) circle (  1.16);

\path[draw=drawColor,line width= 0.4pt,line join=round,line cap=round,fill=fillColor] (214.97,224.99) circle (  1.16);

\path[draw=drawColor,line width= 0.4pt,line join=round,line cap=round,fill=fillColor] (215.11,224.86) circle (  1.16);

\path[draw=drawColor,line width= 0.4pt,line join=round,line cap=round,fill=fillColor] (215.26,224.53) circle (  1.16);

\path[draw=drawColor,line width= 0.4pt,line join=round,line cap=round,fill=fillColor] (215.40,224.43) circle (  1.16);

\path[draw=drawColor,line width= 0.4pt,line join=round,line cap=round,fill=fillColor] (215.54,224.29) circle (  1.16);

\path[draw=drawColor,line width= 0.4pt,line join=round,line cap=round,fill=fillColor] (215.69,223.81) circle (  1.16);

\path[draw=drawColor,line width= 0.4pt,line join=round,line cap=round,fill=fillColor] (215.83,223.68) circle (  1.16);

\path[draw=drawColor,line width= 0.4pt,line join=round,line cap=round,fill=fillColor] (215.97,222.73) circle (  1.16);

\path[draw=drawColor,line width= 0.4pt,line join=round,line cap=round,fill=fillColor] (216.11,222.71) circle (  1.16);

\path[draw=drawColor,line width= 0.4pt,line join=round,line cap=round,fill=fillColor] (216.26,222.64) circle (  1.16);

\path[draw=drawColor,line width= 0.4pt,line join=round,line cap=round,fill=fillColor] (216.40,222.26) circle (  1.16);

\path[draw=drawColor,line width= 0.4pt,line join=round,line cap=round,fill=fillColor] (216.54,221.34) circle (  1.16);

\path[draw=drawColor,line width= 0.4pt,line join=round,line cap=round,fill=fillColor] (216.68,221.23) circle (  1.16);

\path[draw=drawColor,line width= 0.4pt,line join=round,line cap=round,fill=fillColor] (216.82,221.18) circle (  1.16);

\path[draw=drawColor,line width= 0.4pt,line join=round,line cap=round,fill=fillColor] (216.96,220.49) circle (  1.16);

\path[draw=drawColor,line width= 0.4pt,line join=round,line cap=round,fill=fillColor] (217.10,220.32) circle (  1.16);

\path[draw=drawColor,line width= 0.4pt,line join=round,line cap=round,fill=fillColor] (217.24,220.28) circle (  1.16);

\path[draw=drawColor,line width= 0.4pt,line join=round,line cap=round,fill=fillColor] (217.38,220.26) circle (  1.16);

\path[draw=drawColor,line width= 0.4pt,line join=round,line cap=round,fill=fillColor] (217.52,220.02) circle (  1.16);

\path[draw=drawColor,line width= 0.4pt,line join=round,line cap=round,fill=fillColor] (217.66,219.74) circle (  1.16);

\path[draw=drawColor,line width= 0.4pt,line join=round,line cap=round,fill=fillColor] (217.80,219.74) circle (  1.16);

\path[draw=drawColor,line width= 0.4pt,line join=round,line cap=round,fill=fillColor] (217.94,219.62) circle (  1.16);

\path[draw=drawColor,line width= 0.4pt,line join=round,line cap=round,fill=fillColor] (218.08,219.03) circle (  1.16);

\path[draw=drawColor,line width= 0.4pt,line join=round,line cap=round,fill=fillColor] (218.22,216.80) circle (  1.16);

\path[draw=drawColor,line width= 0.4pt,line join=round,line cap=round,fill=fillColor] (218.36,214.17) circle (  1.16);

\path[draw=drawColor,line width= 0.4pt,line join=round,line cap=round,fill=fillColor] (218.50,213.59) circle (  1.16);

\path[draw=drawColor,line width= 0.4pt,line join=round,line cap=round,fill=fillColor] (218.63,209.81) circle (  1.16);

\path[draw=drawColor,line width= 0.4pt,line join=round,line cap=round,fill=fillColor] (218.77,209.81) circle (  1.16);

\path[draw=drawColor,line width= 0.4pt,line join=round,line cap=round,fill=fillColor] (218.91,209.81) circle (  1.16);

\path[draw=drawColor,line width= 0.4pt,line join=round,line cap=round,fill=fillColor] (219.05,209.81) circle (  1.16);

\path[draw=drawColor,line width= 0.4pt,line join=round,line cap=round,fill=fillColor] (219.18,209.81) circle (  1.16);

\path[draw=drawColor,line width= 0.4pt,line join=round,line cap=round,fill=fillColor] (219.32,209.81) circle (  1.16);

\path[draw=drawColor,line width= 0.4pt,line join=round,line cap=round,fill=fillColor] (219.46,209.81) circle (  1.16);

\path[draw=drawColor,line width= 0.4pt,line join=round,line cap=round,fill=fillColor] (219.59,209.81) circle (  1.16);

\path[draw=drawColor,line width= 0.4pt,line join=round,line cap=round,fill=fillColor] (219.73,209.81) circle (  1.16);

\path[draw=drawColor,line width= 0.4pt,line join=round,line cap=round,fill=fillColor] (219.87,209.81) circle (  1.16);

\path[draw=drawColor,line width= 0.4pt,line join=round,line cap=round,fill=fillColor] (220.00,209.81) circle (  1.16);

\path[draw=drawColor,line width= 0.4pt,line join=round,line cap=round,fill=fillColor] (220.14,209.81) circle (  1.16);

\path[draw=drawColor,line width= 0.4pt,line join=round,line cap=round,fill=fillColor] (220.27,209.81) circle (  1.16);

\path[draw=drawColor,line width= 0.4pt,line join=round,line cap=round,fill=fillColor] (220.41,209.81) circle (  1.16);

\path[draw=drawColor,line width= 0.4pt,line join=round,line cap=round,fill=fillColor] (220.54,209.81) circle (  1.16);

\path[draw=drawColor,line width= 0.4pt,line join=round,line cap=round,fill=fillColor] (220.68,209.81) circle (  1.16);

\path[draw=drawColor,line width= 0.4pt,line join=round,line cap=round,fill=fillColor] (220.81,209.81) circle (  1.16);

\path[draw=drawColor,line width= 0.4pt,line join=round,line cap=round,fill=fillColor] (220.95,209.81) circle (  1.16);

\path[draw=drawColor,line width= 0.4pt,line join=round,line cap=round,fill=fillColor] (221.08,209.81) circle (  1.16);

\path[draw=drawColor,line width= 0.4pt,line join=round,line cap=round,fill=fillColor] (221.21,209.81) circle (  1.16);

\path[draw=drawColor,line width= 0.4pt,line join=round,line cap=round,fill=fillColor] (221.35,209.81) circle (  1.16);

\path[draw=drawColor,line width= 0.4pt,line join=round,line cap=round,fill=fillColor] (221.48,209.81) circle (  1.16);

\path[draw=drawColor,line width= 0.4pt,line join=round,line cap=round,fill=fillColor] (221.61,209.81) circle (  1.16);

\path[draw=drawColor,line width= 0.4pt,line join=round,line cap=round,fill=fillColor] (221.75,209.81) circle (  1.16);

\path[draw=drawColor,line width= 0.4pt,line join=round,line cap=round,fill=fillColor] (221.88,209.81) circle (  1.16);

\path[draw=drawColor,line width= 0.4pt,line join=round,line cap=round,fill=fillColor] (222.01,209.81) circle (  1.16);

\path[draw=drawColor,line width= 0.4pt,line join=round,line cap=round,fill=fillColor] (222.14,209.81) circle (  1.16);

\path[draw=drawColor,line width= 0.4pt,line join=round,line cap=round,fill=fillColor] (222.28,209.81) circle (  1.16);

\path[draw=drawColor,line width= 0.4pt,line join=round,line cap=round,fill=fillColor] (222.41,209.81) circle (  1.16);

\path[draw=drawColor,line width= 0.4pt,line join=round,line cap=round,fill=fillColor] (222.54,209.81) circle (  1.16);

\path[draw=drawColor,line width= 0.4pt,line join=round,line cap=round,fill=fillColor] (222.67,209.81) circle (  1.16);

\path[draw=drawColor,line width= 0.4pt,line join=round,line cap=round,fill=fillColor] (222.80,209.81) circle (  1.16);

\path[draw=drawColor,line width= 0.4pt,line join=round,line cap=round,fill=fillColor] (222.93,209.81) circle (  1.16);

\path[draw=drawColor,line width= 0.4pt,line join=round,line cap=round,fill=fillColor] (223.07,209.81) circle (  1.16);

\path[draw=drawColor,line width= 0.4pt,line join=round,line cap=round,fill=fillColor] (223.20,209.81) circle (  1.16);

\path[draw=drawColor,line width= 0.4pt,line join=round,line cap=round,fill=fillColor] (223.33,209.81) circle (  1.16);

\path[draw=drawColor,line width= 0.4pt,line join=round,line cap=round,fill=fillColor] (223.46,209.81) circle (  1.16);

\path[draw=drawColor,line width= 0.4pt,line join=round,line cap=round,fill=fillColor] (223.59,209.81) circle (  1.16);

\path[draw=drawColor,line width= 0.4pt,line join=round,line cap=round,fill=fillColor] (223.72,209.81) circle (  1.16);

\path[draw=drawColor,line width= 0.4pt,line join=round,line cap=round,fill=fillColor] (223.85,209.81) circle (  1.16);

\path[draw=drawColor,line width= 0.4pt,line join=round,line cap=round,fill=fillColor] (223.98,209.81) circle (  1.16);

\path[draw=drawColor,line width= 0.4pt,line join=round,line cap=round,fill=fillColor] (224.11,209.81) circle (  1.16);

\path[draw=drawColor,line width= 0.4pt,line join=round,line cap=round,fill=fillColor] (224.23,209.81) circle (  1.16);

\path[draw=drawColor,line width= 0.4pt,line join=round,line cap=round,fill=fillColor] (224.36,209.81) circle (  1.16);

\path[draw=drawColor,line width= 0.4pt,line join=round,line cap=round,fill=fillColor] (224.49,209.81) circle (  1.16);

\path[draw=drawColor,line width= 0.4pt,line join=round,line cap=round,fill=fillColor] (224.62,209.81) circle (  1.16);

\path[draw=drawColor,line width= 0.4pt,line join=round,line cap=round,fill=fillColor] (224.75,209.81) circle (  1.16);

\path[draw=drawColor,line width= 0.4pt,line join=round,line cap=round,fill=fillColor] (224.88,209.81) circle (  1.16);

\path[draw=drawColor,line width= 0.4pt,line join=round,line cap=round,fill=fillColor] (225.01,209.81) circle (  1.16);

\path[draw=drawColor,line width= 0.4pt,line join=round,line cap=round,fill=fillColor] (225.13,209.81) circle (  1.16);

\path[draw=drawColor,line width= 0.4pt,line join=round,line cap=round,fill=fillColor] (225.26,209.81) circle (  1.16);
\definecolor[named]{drawColor}{rgb}{0.30,0.69,0.29}
\definecolor[named]{fillColor}{rgb}{0.30,0.69,0.29}

\path[draw=drawColor,line width= 0.4pt,line join=round,line cap=round,fill=fillColor] ( 74.88,295.45) circle (  1.16);

\path[draw=drawColor,line width= 0.4pt,line join=round,line cap=round,fill=fillColor] ( 80.74,295.45) circle (  1.16);

\path[draw=drawColor,line width= 0.4pt,line join=round,line cap=round,fill=fillColor] ( 84.84,295.45) circle (  1.16);

\path[draw=drawColor,line width= 0.4pt,line join=round,line cap=round,fill=fillColor] ( 88.11,295.45) circle (  1.16);

\path[draw=drawColor,line width= 0.4pt,line join=round,line cap=round,fill=fillColor] ( 90.88,295.45) circle (  1.16);

\path[draw=drawColor,line width= 0.4pt,line join=round,line cap=round,fill=fillColor] ( 93.29,295.45) circle (  1.16);

\path[draw=drawColor,line width= 0.4pt,line join=round,line cap=round,fill=fillColor] ( 95.45,295.45) circle (  1.16);

\path[draw=drawColor,line width= 0.4pt,line join=round,line cap=round,fill=fillColor] ( 97.41,295.45) circle (  1.16);

\path[draw=drawColor,line width= 0.4pt,line join=round,line cap=round,fill=fillColor] ( 99.21,295.45) circle (  1.16);

\path[draw=drawColor,line width= 0.4pt,line join=round,line cap=round,fill=fillColor] (100.89,295.45) circle (  1.16);

\path[draw=drawColor,line width= 0.4pt,line join=round,line cap=round,fill=fillColor] (102.46,295.45) circle (  1.16);

\path[draw=drawColor,line width= 0.4pt,line join=round,line cap=round,fill=fillColor] (103.93,295.45) circle (  1.16);

\path[draw=drawColor,line width= 0.4pt,line join=round,line cap=round,fill=fillColor] (105.33,295.45) circle (  1.16);

\path[draw=drawColor,line width= 0.4pt,line join=round,line cap=round,fill=fillColor] (106.65,295.45) circle (  1.16);

\path[draw=drawColor,line width= 0.4pt,line join=round,line cap=round,fill=fillColor] (107.91,295.45) circle (  1.16);

\path[draw=drawColor,line width= 0.4pt,line join=round,line cap=round,fill=fillColor] (109.12,295.45) circle (  1.16);

\path[draw=drawColor,line width= 0.4pt,line join=round,line cap=round,fill=fillColor] (110.28,295.45) circle (  1.16);

\path[draw=drawColor,line width= 0.4pt,line join=round,line cap=round,fill=fillColor] (111.40,295.45) circle (  1.16);

\path[draw=drawColor,line width= 0.4pt,line join=round,line cap=round,fill=fillColor] (112.47,295.45) circle (  1.16);

\path[draw=drawColor,line width= 0.4pt,line join=round,line cap=round,fill=fillColor] (113.51,295.45) circle (  1.16);

\path[draw=drawColor,line width= 0.4pt,line join=round,line cap=round,fill=fillColor] (114.51,295.45) circle (  1.16);

\path[draw=drawColor,line width= 0.4pt,line join=round,line cap=round,fill=fillColor] (115.48,295.45) circle (  1.16);

\path[draw=drawColor,line width= 0.4pt,line join=round,line cap=round,fill=fillColor] (116.42,295.45) circle (  1.16);

\path[draw=drawColor,line width= 0.4pt,line join=round,line cap=round,fill=fillColor] (117.34,295.45) circle (  1.16);

\path[draw=drawColor,line width= 0.4pt,line join=round,line cap=round,fill=fillColor] (118.23,295.45) circle (  1.16);

\path[draw=drawColor,line width= 0.4pt,line join=round,line cap=round,fill=fillColor] (119.10,295.45) circle (  1.16);

\path[draw=drawColor,line width= 0.4pt,line join=round,line cap=round,fill=fillColor] (119.94,295.45) circle (  1.16);

\path[draw=drawColor,line width= 0.4pt,line join=round,line cap=round,fill=fillColor] (120.77,295.45) circle (  1.16);

\path[draw=drawColor,line width= 0.4pt,line join=round,line cap=round,fill=fillColor] (121.57,295.45) circle (  1.16);

\path[draw=drawColor,line width= 0.4pt,line join=round,line cap=round,fill=fillColor] (122.36,295.45) circle (  1.16);

\path[draw=drawColor,line width= 0.4pt,line join=round,line cap=round,fill=fillColor] (123.13,295.45) circle (  1.16);

\path[draw=drawColor,line width= 0.4pt,line join=round,line cap=round,fill=fillColor] (123.88,295.45) circle (  1.16);

\path[draw=drawColor,line width= 0.4pt,line join=round,line cap=round,fill=fillColor] (124.62,295.45) circle (  1.16);

\path[draw=drawColor,line width= 0.4pt,line join=round,line cap=round,fill=fillColor] (125.34,295.45) circle (  1.16);

\path[draw=drawColor,line width= 0.4pt,line join=round,line cap=round,fill=fillColor] (126.05,295.45) circle (  1.16);

\path[draw=drawColor,line width= 0.4pt,line join=round,line cap=round,fill=fillColor] (126.74,295.45) circle (  1.16);

\path[draw=drawColor,line width= 0.4pt,line join=round,line cap=round,fill=fillColor] (127.43,295.45) circle (  1.16);

\path[draw=drawColor,line width= 0.4pt,line join=round,line cap=round,fill=fillColor] (128.10,295.45) circle (  1.16);

\path[draw=drawColor,line width= 0.4pt,line join=round,line cap=round,fill=fillColor] (128.76,295.45) circle (  1.16);

\path[draw=drawColor,line width= 0.4pt,line join=round,line cap=round,fill=fillColor] (129.40,295.45) circle (  1.16);

\path[draw=drawColor,line width= 0.4pt,line join=round,line cap=round,fill=fillColor] (130.04,295.45) circle (  1.16);

\path[draw=drawColor,line width= 0.4pt,line join=round,line cap=round,fill=fillColor] (130.67,295.45) circle (  1.16);

\path[draw=drawColor,line width= 0.4pt,line join=round,line cap=round,fill=fillColor] (131.28,295.45) circle (  1.16);

\path[draw=drawColor,line width= 0.4pt,line join=round,line cap=round,fill=fillColor] (131.89,295.45) circle (  1.16);

\path[draw=drawColor,line width= 0.4pt,line join=round,line cap=round,fill=fillColor] (132.49,295.45) circle (  1.16);

\path[draw=drawColor,line width= 0.4pt,line join=round,line cap=round,fill=fillColor] (133.08,295.45) circle (  1.16);

\path[draw=drawColor,line width= 0.4pt,line join=round,line cap=round,fill=fillColor] (133.66,295.45) circle (  1.16);

\path[draw=drawColor,line width= 0.4pt,line join=round,line cap=round,fill=fillColor] (134.23,295.45) circle (  1.16);

\path[draw=drawColor,line width= 0.4pt,line join=round,line cap=round,fill=fillColor] (134.80,295.45) circle (  1.16);

\path[draw=drawColor,line width= 0.4pt,line join=round,line cap=round,fill=fillColor] (135.35,295.45) circle (  1.16);

\path[draw=drawColor,line width= 0.4pt,line join=round,line cap=round,fill=fillColor] (135.90,295.45) circle (  1.16);

\path[draw=drawColor,line width= 0.4pt,line join=round,line cap=round,fill=fillColor] (136.45,295.45) circle (  1.16);

\path[draw=drawColor,line width= 0.4pt,line join=round,line cap=round,fill=fillColor] (136.98,295.45) circle (  1.16);

\path[draw=drawColor,line width= 0.4pt,line join=round,line cap=round,fill=fillColor] (137.51,295.45) circle (  1.16);

\path[draw=drawColor,line width= 0.4pt,line join=round,line cap=round,fill=fillColor] (138.03,295.45) circle (  1.16);

\path[draw=drawColor,line width= 0.4pt,line join=round,line cap=round,fill=fillColor] (138.55,295.45) circle (  1.16);

\path[draw=drawColor,line width= 0.4pt,line join=round,line cap=round,fill=fillColor] (139.06,295.45) circle (  1.16);

\path[draw=drawColor,line width= 0.4pt,line join=round,line cap=round,fill=fillColor] (139.56,295.45) circle (  1.16);

\path[draw=drawColor,line width= 0.4pt,line join=round,line cap=round,fill=fillColor] (140.06,295.45) circle (  1.16);

\path[draw=drawColor,line width= 0.4pt,line join=round,line cap=round,fill=fillColor] (140.55,295.45) circle (  1.16);

\path[draw=drawColor,line width= 0.4pt,line join=round,line cap=round,fill=fillColor] (141.04,295.45) circle (  1.16);

\path[draw=drawColor,line width= 0.4pt,line join=round,line cap=round,fill=fillColor] (141.52,295.45) circle (  1.16);

\path[draw=drawColor,line width= 0.4pt,line join=round,line cap=round,fill=fillColor] (142.00,295.45) circle (  1.16);

\path[draw=drawColor,line width= 0.4pt,line join=round,line cap=round,fill=fillColor] (142.47,295.45) circle (  1.16);

\path[draw=drawColor,line width= 0.4pt,line join=round,line cap=round,fill=fillColor] (142.94,295.45) circle (  1.16);

\path[draw=drawColor,line width= 0.4pt,line join=round,line cap=round,fill=fillColor] (143.40,295.45) circle (  1.16);

\path[draw=drawColor,line width= 0.4pt,line join=round,line cap=round,fill=fillColor] (143.86,295.45) circle (  1.16);

\path[draw=drawColor,line width= 0.4pt,line join=round,line cap=round,fill=fillColor] (144.31,295.45) circle (  1.16);

\path[draw=drawColor,line width= 0.4pt,line join=round,line cap=round,fill=fillColor] (144.76,295.45) circle (  1.16);

\path[draw=drawColor,line width= 0.4pt,line join=round,line cap=round,fill=fillColor] (145.20,295.45) circle (  1.16);

\path[draw=drawColor,line width= 0.4pt,line join=round,line cap=round,fill=fillColor] (145.64,295.45) circle (  1.16);

\path[draw=drawColor,line width= 0.4pt,line join=round,line cap=round,fill=fillColor] (146.08,295.45) circle (  1.16);

\path[draw=drawColor,line width= 0.4pt,line join=round,line cap=round,fill=fillColor] (146.51,295.45) circle (  1.16);

\path[draw=drawColor,line width= 0.4pt,line join=round,line cap=round,fill=fillColor] (146.94,295.45) circle (  1.16);

\path[draw=drawColor,line width= 0.4pt,line join=round,line cap=round,fill=fillColor] (147.36,295.45) circle (  1.16);

\path[draw=drawColor,line width= 0.4pt,line join=round,line cap=round,fill=fillColor] (147.79,295.45) circle (  1.16);

\path[draw=drawColor,line width= 0.4pt,line join=round,line cap=round,fill=fillColor] (148.20,295.45) circle (  1.16);

\path[draw=drawColor,line width= 0.4pt,line join=round,line cap=round,fill=fillColor] (148.62,295.45) circle (  1.16);

\path[draw=drawColor,line width= 0.4pt,line join=round,line cap=round,fill=fillColor] (149.03,295.45) circle (  1.16);

\path[draw=drawColor,line width= 0.4pt,line join=round,line cap=round,fill=fillColor] (149.43,295.45) circle (  1.16);

\path[draw=drawColor,line width= 0.4pt,line join=round,line cap=round,fill=fillColor] (149.83,295.45) circle (  1.16);

\path[draw=drawColor,line width= 0.4pt,line join=round,line cap=round,fill=fillColor] (150.23,295.45) circle (  1.16);

\path[draw=drawColor,line width= 0.4pt,line join=round,line cap=round,fill=fillColor] (150.63,295.45) circle (  1.16);

\path[draw=drawColor,line width= 0.4pt,line join=round,line cap=round,fill=fillColor] (151.02,295.45) circle (  1.16);

\path[draw=drawColor,line width= 0.4pt,line join=round,line cap=round,fill=fillColor] (151.41,295.45) circle (  1.16);

\path[draw=drawColor,line width= 0.4pt,line join=round,line cap=round,fill=fillColor] (151.80,295.45) circle (  1.16);

\path[draw=drawColor,line width= 0.4pt,line join=round,line cap=round,fill=fillColor] (152.18,295.45) circle (  1.16);

\path[draw=drawColor,line width= 0.4pt,line join=round,line cap=round,fill=fillColor] (152.57,295.45) circle (  1.16);

\path[draw=drawColor,line width= 0.4pt,line join=round,line cap=round,fill=fillColor] (152.94,295.45) circle (  1.16);

\path[draw=drawColor,line width= 0.4pt,line join=round,line cap=round,fill=fillColor] (153.32,295.45) circle (  1.16);

\path[draw=drawColor,line width= 0.4pt,line join=round,line cap=round,fill=fillColor] (153.69,295.45) circle (  1.16);

\path[draw=drawColor,line width= 0.4pt,line join=round,line cap=round,fill=fillColor] (154.06,295.45) circle (  1.16);

\path[draw=drawColor,line width= 0.4pt,line join=round,line cap=round,fill=fillColor] (154.43,295.45) circle (  1.16);

\path[draw=drawColor,line width= 0.4pt,line join=round,line cap=round,fill=fillColor] (154.79,295.45) circle (  1.16);

\path[draw=drawColor,line width= 0.4pt,line join=round,line cap=round,fill=fillColor] (155.15,295.45) circle (  1.16);

\path[draw=drawColor,line width= 0.4pt,line join=round,line cap=round,fill=fillColor] (155.51,295.45) circle (  1.16);

\path[draw=drawColor,line width= 0.4pt,line join=round,line cap=round,fill=fillColor] (155.87,295.45) circle (  1.16);

\path[draw=drawColor,line width= 0.4pt,line join=round,line cap=round,fill=fillColor] (156.23,295.45) circle (  1.16);

\path[draw=drawColor,line width= 0.4pt,line join=round,line cap=round,fill=fillColor] (156.58,295.45) circle (  1.16);

\path[draw=drawColor,line width= 0.4pt,line join=round,line cap=round,fill=fillColor] (156.93,295.45) circle (  1.16);

\path[draw=drawColor,line width= 0.4pt,line join=round,line cap=round,fill=fillColor] (157.28,295.45) circle (  1.16);

\path[draw=drawColor,line width= 0.4pt,line join=round,line cap=round,fill=fillColor] (157.62,295.45) circle (  1.16);

\path[draw=drawColor,line width= 0.4pt,line join=round,line cap=round,fill=fillColor] (157.96,295.45) circle (  1.16);

\path[draw=drawColor,line width= 0.4pt,line join=round,line cap=round,fill=fillColor] (158.30,295.45) circle (  1.16);

\path[draw=drawColor,line width= 0.4pt,line join=round,line cap=round,fill=fillColor] (158.64,295.45) circle (  1.16);

\path[draw=drawColor,line width= 0.4pt,line join=round,line cap=round,fill=fillColor] (158.98,295.45) circle (  1.16);

\path[draw=drawColor,line width= 0.4pt,line join=round,line cap=round,fill=fillColor] (159.31,295.45) circle (  1.16);

\path[draw=drawColor,line width= 0.4pt,line join=round,line cap=round,fill=fillColor] (159.65,295.45) circle (  1.16);

\path[draw=drawColor,line width= 0.4pt,line join=round,line cap=round,fill=fillColor] (159.98,295.45) circle (  1.16);

\path[draw=drawColor,line width= 0.4pt,line join=round,line cap=round,fill=fillColor] (160.30,295.45) circle (  1.16);

\path[draw=drawColor,line width= 0.4pt,line join=round,line cap=round,fill=fillColor] (160.63,295.45) circle (  1.16);

\path[draw=drawColor,line width= 0.4pt,line join=round,line cap=round,fill=fillColor] (160.95,295.45) circle (  1.16);

\path[draw=drawColor,line width= 0.4pt,line join=round,line cap=round,fill=fillColor] (161.28,295.45) circle (  1.16);

\path[draw=drawColor,line width= 0.4pt,line join=round,line cap=round,fill=fillColor] (161.60,295.45) circle (  1.16);

\path[draw=drawColor,line width= 0.4pt,line join=round,line cap=round,fill=fillColor] (161.92,295.45) circle (  1.16);

\path[draw=drawColor,line width= 0.4pt,line join=round,line cap=round,fill=fillColor] (162.23,295.45) circle (  1.16);

\path[draw=drawColor,line width= 0.4pt,line join=round,line cap=round,fill=fillColor] (162.55,295.45) circle (  1.16);

\path[draw=drawColor,line width= 0.4pt,line join=round,line cap=round,fill=fillColor] (162.86,295.45) circle (  1.16);

\path[draw=drawColor,line width= 0.4pt,line join=round,line cap=round,fill=fillColor] (163.17,295.45) circle (  1.16);

\path[draw=drawColor,line width= 0.4pt,line join=round,line cap=round,fill=fillColor] (163.48,295.45) circle (  1.16);

\path[draw=drawColor,line width= 0.4pt,line join=round,line cap=round,fill=fillColor] (163.79,295.45) circle (  1.16);

\path[draw=drawColor,line width= 0.4pt,line join=round,line cap=round,fill=fillColor] (164.09,295.45) circle (  1.16);

\path[draw=drawColor,line width= 0.4pt,line join=round,line cap=round,fill=fillColor] (164.40,295.45) circle (  1.16);

\path[draw=drawColor,line width= 0.4pt,line join=round,line cap=round,fill=fillColor] (164.70,295.45) circle (  1.16);

\path[draw=drawColor,line width= 0.4pt,line join=round,line cap=round,fill=fillColor] (165.00,295.45) circle (  1.16);

\path[draw=drawColor,line width= 0.4pt,line join=round,line cap=round,fill=fillColor] (165.30,295.45) circle (  1.16);

\path[draw=drawColor,line width= 0.4pt,line join=round,line cap=round,fill=fillColor] (165.60,295.45) circle (  1.16);

\path[draw=drawColor,line width= 0.4pt,line join=round,line cap=round,fill=fillColor] (165.90,295.45) circle (  1.16);

\path[draw=drawColor,line width= 0.4pt,line join=round,line cap=round,fill=fillColor] (166.19,295.45) circle (  1.16);

\path[draw=drawColor,line width= 0.4pt,line join=round,line cap=round,fill=fillColor] (166.49,295.45) circle (  1.16);

\path[draw=drawColor,line width= 0.4pt,line join=round,line cap=round,fill=fillColor] (166.78,295.45) circle (  1.16);

\path[draw=drawColor,line width= 0.4pt,line join=round,line cap=round,fill=fillColor] (167.07,295.45) circle (  1.16);

\path[draw=drawColor,line width= 0.4pt,line join=round,line cap=round,fill=fillColor] (167.36,295.45) circle (  1.16);

\path[draw=drawColor,line width= 0.4pt,line join=round,line cap=round,fill=fillColor] (167.64,295.45) circle (  1.16);

\path[draw=drawColor,line width= 0.4pt,line join=round,line cap=round,fill=fillColor] (167.93,295.45) circle (  1.16);

\path[draw=drawColor,line width= 0.4pt,line join=round,line cap=round,fill=fillColor] (168.22,295.45) circle (  1.16);

\path[draw=drawColor,line width= 0.4pt,line join=round,line cap=round,fill=fillColor] (168.50,295.45) circle (  1.16);

\path[draw=drawColor,line width= 0.4pt,line join=round,line cap=round,fill=fillColor] (168.78,295.45) circle (  1.16);

\path[draw=drawColor,line width= 0.4pt,line join=round,line cap=round,fill=fillColor] (169.06,295.45) circle (  1.16);

\path[draw=drawColor,line width= 0.4pt,line join=round,line cap=round,fill=fillColor] (169.34,295.45) circle (  1.16);

\path[draw=drawColor,line width= 0.4pt,line join=round,line cap=round,fill=fillColor] (169.62,295.45) circle (  1.16);

\path[draw=drawColor,line width= 0.4pt,line join=round,line cap=round,fill=fillColor] (169.89,295.45) circle (  1.16);

\path[draw=drawColor,line width= 0.4pt,line join=round,line cap=round,fill=fillColor] (170.17,295.45) circle (  1.16);

\path[draw=drawColor,line width= 0.4pt,line join=round,line cap=round,fill=fillColor] (170.44,295.45) circle (  1.16);

\path[draw=drawColor,line width= 0.4pt,line join=round,line cap=round,fill=fillColor] (170.72,295.45) circle (  1.16);

\path[draw=drawColor,line width= 0.4pt,line join=round,line cap=round,fill=fillColor] (170.99,295.45) circle (  1.16);

\path[draw=drawColor,line width= 0.4pt,line join=round,line cap=round,fill=fillColor] (171.26,295.45) circle (  1.16);

\path[draw=drawColor,line width= 0.4pt,line join=round,line cap=round,fill=fillColor] (171.53,295.45) circle (  1.16);

\path[draw=drawColor,line width= 0.4pt,line join=round,line cap=round,fill=fillColor] (171.80,295.45) circle (  1.16);

\path[draw=drawColor,line width= 0.4pt,line join=round,line cap=round,fill=fillColor] (172.06,295.45) circle (  1.16);

\path[draw=drawColor,line width= 0.4pt,line join=round,line cap=round,fill=fillColor] (172.33,295.45) circle (  1.16);

\path[draw=drawColor,line width= 0.4pt,line join=round,line cap=round,fill=fillColor] (172.59,295.45) circle (  1.16);

\path[draw=drawColor,line width= 0.4pt,line join=round,line cap=round,fill=fillColor] (172.85,295.45) circle (  1.16);

\path[draw=drawColor,line width= 0.4pt,line join=round,line cap=round,fill=fillColor] (173.12,295.45) circle (  1.16);

\path[draw=drawColor,line width= 0.4pt,line join=round,line cap=round,fill=fillColor] (173.38,295.45) circle (  1.16);

\path[draw=drawColor,line width= 0.4pt,line join=round,line cap=round,fill=fillColor] (173.64,295.45) circle (  1.16);

\path[draw=drawColor,line width= 0.4pt,line join=round,line cap=round,fill=fillColor] (173.90,295.45) circle (  1.16);

\path[draw=drawColor,line width= 0.4pt,line join=round,line cap=round,fill=fillColor] (174.15,295.45) circle (  1.16);

\path[draw=drawColor,line width= 0.4pt,line join=round,line cap=round,fill=fillColor] (174.41,295.45) circle (  1.16);

\path[draw=drawColor,line width= 0.4pt,line join=round,line cap=round,fill=fillColor] (174.67,295.45) circle (  1.16);

\path[draw=drawColor,line width= 0.4pt,line join=round,line cap=round,fill=fillColor] (174.92,295.45) circle (  1.16);

\path[draw=drawColor,line width= 0.4pt,line join=round,line cap=round,fill=fillColor] (175.17,295.45) circle (  1.16);

\path[draw=drawColor,line width= 0.4pt,line join=round,line cap=round,fill=fillColor] (175.42,295.45) circle (  1.16);

\path[draw=drawColor,line width= 0.4pt,line join=round,line cap=round,fill=fillColor] (175.68,295.45) circle (  1.16);

\path[draw=drawColor,line width= 0.4pt,line join=round,line cap=round,fill=fillColor] (175.93,295.45) circle (  1.16);

\path[draw=drawColor,line width= 0.4pt,line join=round,line cap=round,fill=fillColor] (176.18,295.45) circle (  1.16);

\path[draw=drawColor,line width= 0.4pt,line join=round,line cap=round,fill=fillColor] (176.42,295.45) circle (  1.16);

\path[draw=drawColor,line width= 0.4pt,line join=round,line cap=round,fill=fillColor] (176.67,295.45) circle (  1.16);

\path[draw=drawColor,line width= 0.4pt,line join=round,line cap=round,fill=fillColor] (176.92,295.45) circle (  1.16);

\path[draw=drawColor,line width= 0.4pt,line join=round,line cap=round,fill=fillColor] (177.16,295.45) circle (  1.16);

\path[draw=drawColor,line width= 0.4pt,line join=round,line cap=round,fill=fillColor] (177.41,295.45) circle (  1.16);

\path[draw=drawColor,line width= 0.4pt,line join=round,line cap=round,fill=fillColor] (177.65,295.45) circle (  1.16);

\path[draw=drawColor,line width= 0.4pt,line join=round,line cap=round,fill=fillColor] (177.89,295.45) circle (  1.16);

\path[draw=drawColor,line width= 0.4pt,line join=round,line cap=round,fill=fillColor] (178.13,295.45) circle (  1.16);

\path[draw=drawColor,line width= 0.4pt,line join=round,line cap=round,fill=fillColor] (178.37,295.45) circle (  1.16);

\path[draw=drawColor,line width= 0.4pt,line join=round,line cap=round,fill=fillColor] (178.61,295.45) circle (  1.16);

\path[draw=drawColor,line width= 0.4pt,line join=round,line cap=round,fill=fillColor] (178.85,295.45) circle (  1.16);

\path[draw=drawColor,line width= 0.4pt,line join=round,line cap=round,fill=fillColor] (179.09,295.45) circle (  1.16);

\path[draw=drawColor,line width= 0.4pt,line join=round,line cap=round,fill=fillColor] (179.33,295.45) circle (  1.16);

\path[draw=drawColor,line width= 0.4pt,line join=round,line cap=round,fill=fillColor] (179.56,295.45) circle (  1.16);

\path[draw=drawColor,line width= 0.4pt,line join=round,line cap=round,fill=fillColor] (179.80,295.45) circle (  1.16);

\path[draw=drawColor,line width= 0.4pt,line join=round,line cap=round,fill=fillColor] (180.03,295.45) circle (  1.16);

\path[draw=drawColor,line width= 0.4pt,line join=round,line cap=round,fill=fillColor] (180.27,295.45) circle (  1.16);

\path[draw=drawColor,line width= 0.4pt,line join=round,line cap=round,fill=fillColor] (180.50,295.45) circle (  1.16);

\path[draw=drawColor,line width= 0.4pt,line join=round,line cap=round,fill=fillColor] (180.73,295.45) circle (  1.16);

\path[draw=drawColor,line width= 0.4pt,line join=round,line cap=round,fill=fillColor] (180.96,295.45) circle (  1.16);

\path[draw=drawColor,line width= 0.4pt,line join=round,line cap=round,fill=fillColor] (181.19,295.45) circle (  1.16);

\path[draw=drawColor,line width= 0.4pt,line join=round,line cap=round,fill=fillColor] (181.42,295.45) circle (  1.16);

\path[draw=drawColor,line width= 0.4pt,line join=round,line cap=round,fill=fillColor] (181.65,295.45) circle (  1.16);

\path[draw=drawColor,line width= 0.4pt,line join=round,line cap=round,fill=fillColor] (181.88,295.45) circle (  1.16);

\path[draw=drawColor,line width= 0.4pt,line join=round,line cap=round,fill=fillColor] (182.10,295.45) circle (  1.16);

\path[draw=drawColor,line width= 0.4pt,line join=round,line cap=round,fill=fillColor] (182.33,295.45) circle (  1.16);

\path[draw=drawColor,line width= 0.4pt,line join=round,line cap=round,fill=fillColor] (182.55,295.45) circle (  1.16);

\path[draw=drawColor,line width= 0.4pt,line join=round,line cap=round,fill=fillColor] (182.78,295.45) circle (  1.16);

\path[draw=drawColor,line width= 0.4pt,line join=round,line cap=round,fill=fillColor] (183.00,295.45) circle (  1.16);

\path[draw=drawColor,line width= 0.4pt,line join=round,line cap=round,fill=fillColor] (183.23,295.45) circle (  1.16);

\path[draw=drawColor,line width= 0.4pt,line join=round,line cap=round,fill=fillColor] (183.45,295.45) circle (  1.16);

\path[draw=drawColor,line width= 0.4pt,line join=round,line cap=round,fill=fillColor] (183.67,295.45) circle (  1.16);

\path[draw=drawColor,line width= 0.4pt,line join=round,line cap=round,fill=fillColor] (183.89,295.45) circle (  1.16);

\path[draw=drawColor,line width= 0.4pt,line join=round,line cap=round,fill=fillColor] (184.11,295.45) circle (  1.16);

\path[draw=drawColor,line width= 0.4pt,line join=round,line cap=round,fill=fillColor] (184.33,295.45) circle (  1.16);

\path[draw=drawColor,line width= 0.4pt,line join=round,line cap=round,fill=fillColor] (184.55,295.45) circle (  1.16);

\path[draw=drawColor,line width= 0.4pt,line join=round,line cap=round,fill=fillColor] (184.77,295.45) circle (  1.16);

\path[draw=drawColor,line width= 0.4pt,line join=round,line cap=round,fill=fillColor] (184.98,295.45) circle (  1.16);

\path[draw=drawColor,line width= 0.4pt,line join=round,line cap=round,fill=fillColor] (185.20,295.45) circle (  1.16);

\path[draw=drawColor,line width= 0.4pt,line join=round,line cap=round,fill=fillColor] (185.41,295.45) circle (  1.16);

\path[draw=drawColor,line width= 0.4pt,line join=round,line cap=round,fill=fillColor] (185.63,295.45) circle (  1.16);

\path[draw=drawColor,line width= 0.4pt,line join=round,line cap=round,fill=fillColor] (185.84,295.45) circle (  1.16);

\path[draw=drawColor,line width= 0.4pt,line join=round,line cap=round,fill=fillColor] (186.06,295.45) circle (  1.16);

\path[draw=drawColor,line width= 0.4pt,line join=round,line cap=round,fill=fillColor] (186.27,295.45) circle (  1.16);

\path[draw=drawColor,line width= 0.4pt,line join=round,line cap=round,fill=fillColor] (186.48,295.45) circle (  1.16);

\path[draw=drawColor,line width= 0.4pt,line join=round,line cap=round,fill=fillColor] (186.69,295.45) circle (  1.16);

\path[draw=drawColor,line width= 0.4pt,line join=round,line cap=round,fill=fillColor] (186.91,295.45) circle (  1.16);

\path[draw=drawColor,line width= 0.4pt,line join=round,line cap=round,fill=fillColor] (187.12,295.45) circle (  1.16);

\path[draw=drawColor,line width= 0.4pt,line join=round,line cap=round,fill=fillColor] (187.33,295.45) circle (  1.16);

\path[draw=drawColor,line width= 0.4pt,line join=round,line cap=round,fill=fillColor] (187.53,295.45) circle (  1.16);

\path[draw=drawColor,line width= 0.4pt,line join=round,line cap=round,fill=fillColor] (187.74,295.45) circle (  1.16);

\path[draw=drawColor,line width= 0.4pt,line join=round,line cap=round,fill=fillColor] (187.95,295.45) circle (  1.16);

\path[draw=drawColor,line width= 0.4pt,line join=round,line cap=round,fill=fillColor] (188.16,295.45) circle (  1.16);

\path[draw=drawColor,line width= 0.4pt,line join=round,line cap=round,fill=fillColor] (188.36,295.45) circle (  1.16);

\path[draw=drawColor,line width= 0.4pt,line join=round,line cap=round,fill=fillColor] (188.57,295.45) circle (  1.16);

\path[draw=drawColor,line width= 0.4pt,line join=round,line cap=round,fill=fillColor] (188.77,295.45) circle (  1.16);

\path[draw=drawColor,line width= 0.4pt,line join=round,line cap=round,fill=fillColor] (188.98,295.45) circle (  1.16);

\path[draw=drawColor,line width= 0.4pt,line join=round,line cap=round,fill=fillColor] (189.18,295.45) circle (  1.16);

\path[draw=drawColor,line width= 0.4pt,line join=round,line cap=round,fill=fillColor] (189.39,295.45) circle (  1.16);

\path[draw=drawColor,line width= 0.4pt,line join=round,line cap=round,fill=fillColor] (189.59,295.45) circle (  1.16);

\path[draw=drawColor,line width= 0.4pt,line join=round,line cap=round,fill=fillColor] (189.79,295.45) circle (  1.16);

\path[draw=drawColor,line width= 0.4pt,line join=round,line cap=round,fill=fillColor] (189.99,295.45) circle (  1.16);

\path[draw=drawColor,line width= 0.4pt,line join=round,line cap=round,fill=fillColor] (190.19,295.45) circle (  1.16);

\path[draw=drawColor,line width= 0.4pt,line join=round,line cap=round,fill=fillColor] (190.39,295.45) circle (  1.16);

\path[draw=drawColor,line width= 0.4pt,line join=round,line cap=round,fill=fillColor] (190.59,295.45) circle (  1.16);

\path[draw=drawColor,line width= 0.4pt,line join=round,line cap=round,fill=fillColor] (190.79,291.51) circle (  1.16);

\path[draw=drawColor,line width= 0.4pt,line join=round,line cap=round,fill=fillColor] (190.99,291.44) circle (  1.16);

\path[draw=drawColor,line width= 0.4pt,line join=round,line cap=round,fill=fillColor] (191.19,287.54) circle (  1.16);

\path[draw=drawColor,line width= 0.4pt,line join=round,line cap=round,fill=fillColor] (191.39,286.09) circle (  1.16);

\path[draw=drawColor,line width= 0.4pt,line join=round,line cap=round,fill=fillColor] (191.58,284.88) circle (  1.16);

\path[draw=drawColor,line width= 0.4pt,line join=round,line cap=round,fill=fillColor] (191.78,284.51) circle (  1.16);

\path[draw=drawColor,line width= 0.4pt,line join=round,line cap=round,fill=fillColor] (191.98,281.92) circle (  1.16);

\path[draw=drawColor,line width= 0.4pt,line join=round,line cap=round,fill=fillColor] (192.17,281.27) circle (  1.16);

\path[draw=drawColor,line width= 0.4pt,line join=round,line cap=round,fill=fillColor] (192.37,279.37) circle (  1.16);

\path[draw=drawColor,line width= 0.4pt,line join=round,line cap=round,fill=fillColor] (192.56,278.23) circle (  1.16);

\path[draw=drawColor,line width= 0.4pt,line join=round,line cap=round,fill=fillColor] (192.75,277.14) circle (  1.16);

\path[draw=drawColor,line width= 0.4pt,line join=round,line cap=round,fill=fillColor] (192.95,277.02) circle (  1.16);

\path[draw=drawColor,line width= 0.4pt,line join=round,line cap=round,fill=fillColor] (193.14,275.84) circle (  1.16);

\path[draw=drawColor,line width= 0.4pt,line join=round,line cap=round,fill=fillColor] (193.33,275.59) circle (  1.16);

\path[draw=drawColor,line width= 0.4pt,line join=round,line cap=round,fill=fillColor] (193.52,273.81) circle (  1.16);

\path[draw=drawColor,line width= 0.4pt,line join=round,line cap=round,fill=fillColor] (193.71,273.68) circle (  1.16);

\path[draw=drawColor,line width= 0.4pt,line join=round,line cap=round,fill=fillColor] (193.90,272.02) circle (  1.16);

\path[draw=drawColor,line width= 0.4pt,line join=round,line cap=round,fill=fillColor] (194.09,271.49) circle (  1.16);

\path[draw=drawColor,line width= 0.4pt,line join=round,line cap=round,fill=fillColor] (194.28,270.95) circle (  1.16);

\path[draw=drawColor,line width= 0.4pt,line join=round,line cap=round,fill=fillColor] (194.47,270.92) circle (  1.16);

\path[draw=drawColor,line width= 0.4pt,line join=round,line cap=round,fill=fillColor] (194.66,270.17) circle (  1.16);

\path[draw=drawColor,line width= 0.4pt,line join=round,line cap=round,fill=fillColor] (194.85,270.09) circle (  1.16);

\path[draw=drawColor,line width= 0.4pt,line join=round,line cap=round,fill=fillColor] (195.04,267.21) circle (  1.16);

\path[draw=drawColor,line width= 0.4pt,line join=round,line cap=round,fill=fillColor] (195.22,266.38) circle (  1.16);

\path[draw=drawColor,line width= 0.4pt,line join=round,line cap=round,fill=fillColor] (195.41,265.33) circle (  1.16);

\path[draw=drawColor,line width= 0.4pt,line join=round,line cap=round,fill=fillColor] (195.60,265.18) circle (  1.16);

\path[draw=drawColor,line width= 0.4pt,line join=round,line cap=round,fill=fillColor] (195.78,264.94) circle (  1.16);

\path[draw=drawColor,line width= 0.4pt,line join=round,line cap=round,fill=fillColor] (195.97,264.53) circle (  1.16);

\path[draw=drawColor,line width= 0.4pt,line join=round,line cap=round,fill=fillColor] (196.15,264.32) circle (  1.16);

\path[draw=drawColor,line width= 0.4pt,line join=round,line cap=round,fill=fillColor] (196.34,263.81) circle (  1.16);

\path[draw=drawColor,line width= 0.4pt,line join=round,line cap=round,fill=fillColor] (196.52,262.05) circle (  1.16);

\path[draw=drawColor,line width= 0.4pt,line join=round,line cap=round,fill=fillColor] (196.70,261.47) circle (  1.16);

\path[draw=drawColor,line width= 0.4pt,line join=round,line cap=round,fill=fillColor] (196.89,261.10) circle (  1.16);

\path[draw=drawColor,line width= 0.4pt,line join=round,line cap=round,fill=fillColor] (197.07,261.04) circle (  1.16);

\path[draw=drawColor,line width= 0.4pt,line join=round,line cap=round,fill=fillColor] (197.25,260.94) circle (  1.16);

\path[draw=drawColor,line width= 0.4pt,line join=round,line cap=round,fill=fillColor] (197.43,260.54) circle (  1.16);

\path[draw=drawColor,line width= 0.4pt,line join=round,line cap=round,fill=fillColor] (197.61,259.75) circle (  1.16);

\path[draw=drawColor,line width= 0.4pt,line join=round,line cap=round,fill=fillColor] (197.79,259.21) circle (  1.16);

\path[draw=drawColor,line width= 0.4pt,line join=round,line cap=round,fill=fillColor] (197.97,258.58) circle (  1.16);

\path[draw=drawColor,line width= 0.4pt,line join=round,line cap=round,fill=fillColor] (198.15,258.56) circle (  1.16);

\path[draw=drawColor,line width= 0.4pt,line join=round,line cap=round,fill=fillColor] (198.33,256.69) circle (  1.16);

\path[draw=drawColor,line width= 0.4pt,line join=round,line cap=round,fill=fillColor] (198.51,256.55) circle (  1.16);

\path[draw=drawColor,line width= 0.4pt,line join=round,line cap=round,fill=fillColor] (198.69,256.32) circle (  1.16);

\path[draw=drawColor,line width= 0.4pt,line join=round,line cap=round,fill=fillColor] (198.87,256.02) circle (  1.16);

\path[draw=drawColor,line width= 0.4pt,line join=round,line cap=round,fill=fillColor] (199.04,255.65) circle (  1.16);

\path[draw=drawColor,line width= 0.4pt,line join=round,line cap=round,fill=fillColor] (199.22,254.77) circle (  1.16);

\path[draw=drawColor,line width= 0.4pt,line join=round,line cap=round,fill=fillColor] (199.40,254.75) circle (  1.16);

\path[draw=drawColor,line width= 0.4pt,line join=round,line cap=round,fill=fillColor] (199.57,254.63) circle (  1.16);

\path[draw=drawColor,line width= 0.4pt,line join=round,line cap=round,fill=fillColor] (199.75,254.54) circle (  1.16);

\path[draw=drawColor,line width= 0.4pt,line join=round,line cap=round,fill=fillColor] (199.92,254.42) circle (  1.16);

\path[draw=drawColor,line width= 0.4pt,line join=round,line cap=round,fill=fillColor] (200.10,254.29) circle (  1.16);

\path[draw=drawColor,line width= 0.4pt,line join=round,line cap=round,fill=fillColor] (200.27,254.25) circle (  1.16);

\path[draw=drawColor,line width= 0.4pt,line join=round,line cap=round,fill=fillColor] (200.45,254.20) circle (  1.16);

\path[draw=drawColor,line width= 0.4pt,line join=round,line cap=round,fill=fillColor] (200.62,254.09) circle (  1.16);

\path[draw=drawColor,line width= 0.4pt,line join=round,line cap=round,fill=fillColor] (200.79,253.83) circle (  1.16);

\path[draw=drawColor,line width= 0.4pt,line join=round,line cap=round,fill=fillColor] (200.97,253.67) circle (  1.16);

\path[draw=drawColor,line width= 0.4pt,line join=round,line cap=round,fill=fillColor] (201.14,252.99) circle (  1.16);

\path[draw=drawColor,line width= 0.4pt,line join=round,line cap=round,fill=fillColor] (201.31,252.69) circle (  1.16);

\path[draw=drawColor,line width= 0.4pt,line join=round,line cap=round,fill=fillColor] (201.48,252.20) circle (  1.16);

\path[draw=drawColor,line width= 0.4pt,line join=round,line cap=round,fill=fillColor] (201.65,252.02) circle (  1.16);

\path[draw=drawColor,line width= 0.4pt,line join=round,line cap=round,fill=fillColor] (201.83,251.67) circle (  1.16);

\path[draw=drawColor,line width= 0.4pt,line join=round,line cap=round,fill=fillColor] (202.00,251.49) circle (  1.16);

\path[draw=drawColor,line width= 0.4pt,line join=round,line cap=round,fill=fillColor] (202.17,251.01) circle (  1.16);

\path[draw=drawColor,line width= 0.4pt,line join=round,line cap=round,fill=fillColor] (202.34,250.89) circle (  1.16);

\path[draw=drawColor,line width= 0.4pt,line join=round,line cap=round,fill=fillColor] (202.50,250.46) circle (  1.16);

\path[draw=drawColor,line width= 0.4pt,line join=round,line cap=round,fill=fillColor] (202.67,249.84) circle (  1.16);

\path[draw=drawColor,line width= 0.4pt,line join=round,line cap=round,fill=fillColor] (202.84,249.83) circle (  1.16);

\path[draw=drawColor,line width= 0.4pt,line join=round,line cap=round,fill=fillColor] (203.01,249.76) circle (  1.16);

\path[draw=drawColor,line width= 0.4pt,line join=round,line cap=round,fill=fillColor] (203.18,249.25) circle (  1.16);

\path[draw=drawColor,line width= 0.4pt,line join=round,line cap=round,fill=fillColor] (203.35,249.05) circle (  1.16);

\path[draw=drawColor,line width= 0.4pt,line join=round,line cap=round,fill=fillColor] (203.51,248.68) circle (  1.16);

\path[draw=drawColor,line width= 0.4pt,line join=round,line cap=round,fill=fillColor] (203.68,248.03) circle (  1.16);

\path[draw=drawColor,line width= 0.4pt,line join=round,line cap=round,fill=fillColor] (203.85,248.00) circle (  1.16);

\path[draw=drawColor,line width= 0.4pt,line join=round,line cap=round,fill=fillColor] (204.01,247.80) circle (  1.16);

\path[draw=drawColor,line width= 0.4pt,line join=round,line cap=round,fill=fillColor] (204.18,247.52) circle (  1.16);

\path[draw=drawColor,line width= 0.4pt,line join=round,line cap=round,fill=fillColor] (204.34,247.22) circle (  1.16);

\path[draw=drawColor,line width= 0.4pt,line join=round,line cap=round,fill=fillColor] (204.51,246.65) circle (  1.16);

\path[draw=drawColor,line width= 0.4pt,line join=round,line cap=round,fill=fillColor] (204.67,246.58) circle (  1.16);

\path[draw=drawColor,line width= 0.4pt,line join=round,line cap=round,fill=fillColor] (204.84,245.95) circle (  1.16);

\path[draw=drawColor,line width= 0.4pt,line join=round,line cap=round,fill=fillColor] (205.00,245.76) circle (  1.16);

\path[draw=drawColor,line width= 0.4pt,line join=round,line cap=round,fill=fillColor] (205.16,245.43) circle (  1.16);

\path[draw=drawColor,line width= 0.4pt,line join=round,line cap=round,fill=fillColor] (205.33,245.41) circle (  1.16);

\path[draw=drawColor,line width= 0.4pt,line join=round,line cap=round,fill=fillColor] (205.49,245.20) circle (  1.16);

\path[draw=drawColor,line width= 0.4pt,line join=round,line cap=round,fill=fillColor] (205.65,245.15) circle (  1.16);

\path[draw=drawColor,line width= 0.4pt,line join=round,line cap=round,fill=fillColor] (205.81,244.89) circle (  1.16);

\path[draw=drawColor,line width= 0.4pt,line join=round,line cap=round,fill=fillColor] (205.97,244.74) circle (  1.16);

\path[draw=drawColor,line width= 0.4pt,line join=round,line cap=round,fill=fillColor] (206.14,244.71) circle (  1.16);

\path[draw=drawColor,line width= 0.4pt,line join=round,line cap=round,fill=fillColor] (206.30,244.69) circle (  1.16);

\path[draw=drawColor,line width= 0.4pt,line join=round,line cap=round,fill=fillColor] (206.46,244.58) circle (  1.16);

\path[draw=drawColor,line width= 0.4pt,line join=round,line cap=round,fill=fillColor] (206.62,244.43) circle (  1.16);

\path[draw=drawColor,line width= 0.4pt,line join=round,line cap=round,fill=fillColor] (206.78,244.04) circle (  1.16);

\path[draw=drawColor,line width= 0.4pt,line join=round,line cap=round,fill=fillColor] (206.94,244.04) circle (  1.16);

\path[draw=drawColor,line width= 0.4pt,line join=round,line cap=round,fill=fillColor] (207.10,243.90) circle (  1.16);

\path[draw=drawColor,line width= 0.4pt,line join=round,line cap=round,fill=fillColor] (207.26,243.66) circle (  1.16);

\path[draw=drawColor,line width= 0.4pt,line join=round,line cap=round,fill=fillColor] (207.42,243.63) circle (  1.16);

\path[draw=drawColor,line width= 0.4pt,line join=round,line cap=round,fill=fillColor] (207.57,243.62) circle (  1.16);

\path[draw=drawColor,line width= 0.4pt,line join=round,line cap=round,fill=fillColor] (207.73,243.32) circle (  1.16);

\path[draw=drawColor,line width= 0.4pt,line join=round,line cap=round,fill=fillColor] (207.89,242.86) circle (  1.16);

\path[draw=drawColor,line width= 0.4pt,line join=round,line cap=round,fill=fillColor] (208.05,242.71) circle (  1.16);

\path[draw=drawColor,line width= 0.4pt,line join=round,line cap=round,fill=fillColor] (208.20,242.54) circle (  1.16);

\path[draw=drawColor,line width= 0.4pt,line join=round,line cap=round,fill=fillColor] (208.36,242.52) circle (  1.16);

\path[draw=drawColor,line width= 0.4pt,line join=round,line cap=round,fill=fillColor] (208.52,242.38) circle (  1.16);

\path[draw=drawColor,line width= 0.4pt,line join=round,line cap=round,fill=fillColor] (208.67,242.28) circle (  1.16);

\path[draw=drawColor,line width= 0.4pt,line join=round,line cap=round,fill=fillColor] (208.83,242.19) circle (  1.16);

\path[draw=drawColor,line width= 0.4pt,line join=round,line cap=round,fill=fillColor] (208.98,242.01) circle (  1.16);

\path[draw=drawColor,line width= 0.4pt,line join=round,line cap=round,fill=fillColor] (209.14,241.78) circle (  1.16);

\path[draw=drawColor,line width= 0.4pt,line join=round,line cap=round,fill=fillColor] (209.29,240.99) circle (  1.16);

\path[draw=drawColor,line width= 0.4pt,line join=round,line cap=round,fill=fillColor] (209.45,240.88) circle (  1.16);

\path[draw=drawColor,line width= 0.4pt,line join=round,line cap=round,fill=fillColor] (209.60,240.76) circle (  1.16);

\path[draw=drawColor,line width= 0.4pt,line join=round,line cap=round,fill=fillColor] (209.76,240.61) circle (  1.16);

\path[draw=drawColor,line width= 0.4pt,line join=round,line cap=round,fill=fillColor] (209.91,240.57) circle (  1.16);

\path[draw=drawColor,line width= 0.4pt,line join=round,line cap=round,fill=fillColor] (210.07,240.43) circle (  1.16);

\path[draw=drawColor,line width= 0.4pt,line join=round,line cap=round,fill=fillColor] (210.22,240.35) circle (  1.16);

\path[draw=drawColor,line width= 0.4pt,line join=round,line cap=round,fill=fillColor] (210.37,239.99) circle (  1.16);

\path[draw=drawColor,line width= 0.4pt,line join=round,line cap=round,fill=fillColor] (210.52,239.76) circle (  1.16);

\path[draw=drawColor,line width= 0.4pt,line join=round,line cap=round,fill=fillColor] (210.68,239.66) circle (  1.16);

\path[draw=drawColor,line width= 0.4pt,line join=round,line cap=round,fill=fillColor] (210.83,239.41) circle (  1.16);

\path[draw=drawColor,line width= 0.4pt,line join=round,line cap=round,fill=fillColor] (210.98,239.40) circle (  1.16);

\path[draw=drawColor,line width= 0.4pt,line join=round,line cap=round,fill=fillColor] (211.13,239.29) circle (  1.16);

\path[draw=drawColor,line width= 0.4pt,line join=round,line cap=round,fill=fillColor] (211.28,239.28) circle (  1.16);

\path[draw=drawColor,line width= 0.4pt,line join=round,line cap=round,fill=fillColor] (211.43,239.22) circle (  1.16);

\path[draw=drawColor,line width= 0.4pt,line join=round,line cap=round,fill=fillColor] (211.58,239.19) circle (  1.16);

\path[draw=drawColor,line width= 0.4pt,line join=round,line cap=round,fill=fillColor] (211.73,239.03) circle (  1.16);

\path[draw=drawColor,line width= 0.4pt,line join=round,line cap=round,fill=fillColor] (211.88,238.92) circle (  1.16);

\path[draw=drawColor,line width= 0.4pt,line join=round,line cap=round,fill=fillColor] (212.03,238.82) circle (  1.16);

\path[draw=drawColor,line width= 0.4pt,line join=round,line cap=round,fill=fillColor] (212.18,238.82) circle (  1.16);

\path[draw=drawColor,line width= 0.4pt,line join=round,line cap=round,fill=fillColor] (212.33,238.71) circle (  1.16);

\path[draw=drawColor,line width= 0.4pt,line join=round,line cap=round,fill=fillColor] (212.48,238.67) circle (  1.16);

\path[draw=drawColor,line width= 0.4pt,line join=round,line cap=round,fill=fillColor] (212.63,238.63) circle (  1.16);

\path[draw=drawColor,line width= 0.4pt,line join=round,line cap=round,fill=fillColor] (212.78,238.59) circle (  1.16);

\path[draw=drawColor,line width= 0.4pt,line join=round,line cap=round,fill=fillColor] (212.93,238.55) circle (  1.16);

\path[draw=drawColor,line width= 0.4pt,line join=round,line cap=round,fill=fillColor] (213.07,238.50) circle (  1.16);

\path[draw=drawColor,line width= 0.4pt,line join=round,line cap=round,fill=fillColor] (213.22,238.47) circle (  1.16);

\path[draw=drawColor,line width= 0.4pt,line join=round,line cap=round,fill=fillColor] (213.37,238.44) circle (  1.16);

\path[draw=drawColor,line width= 0.4pt,line join=round,line cap=round,fill=fillColor] (213.51,238.43) circle (  1.16);

\path[draw=drawColor,line width= 0.4pt,line join=round,line cap=round,fill=fillColor] (213.66,238.43) circle (  1.16);

\path[draw=drawColor,line width= 0.4pt,line join=round,line cap=round,fill=fillColor] (213.81,238.30) circle (  1.16);

\path[draw=drawColor,line width= 0.4pt,line join=round,line cap=round,fill=fillColor] (213.95,237.93) circle (  1.16);

\path[draw=drawColor,line width= 0.4pt,line join=round,line cap=round,fill=fillColor] (214.10,237.91) circle (  1.16);

\path[draw=drawColor,line width= 0.4pt,line join=round,line cap=round,fill=fillColor] (214.24,237.86) circle (  1.16);

\path[draw=drawColor,line width= 0.4pt,line join=round,line cap=round,fill=fillColor] (214.39,237.73) circle (  1.16);

\path[draw=drawColor,line width= 0.4pt,line join=round,line cap=round,fill=fillColor] (214.54,237.60) circle (  1.16);

\path[draw=drawColor,line width= 0.4pt,line join=round,line cap=round,fill=fillColor] (214.68,237.45) circle (  1.16);

\path[draw=drawColor,line width= 0.4pt,line join=round,line cap=round,fill=fillColor] (214.82,237.38) circle (  1.16);

\path[draw=drawColor,line width= 0.4pt,line join=round,line cap=round,fill=fillColor] (214.97,237.37) circle (  1.16);

\path[draw=drawColor,line width= 0.4pt,line join=round,line cap=round,fill=fillColor] (215.11,237.25) circle (  1.16);

\path[draw=drawColor,line width= 0.4pt,line join=round,line cap=round,fill=fillColor] (215.26,237.03) circle (  1.16);

\path[draw=drawColor,line width= 0.4pt,line join=round,line cap=round,fill=fillColor] (215.40,237.01) circle (  1.16);

\path[draw=drawColor,line width= 0.4pt,line join=round,line cap=round,fill=fillColor] (215.54,236.95) circle (  1.16);

\path[draw=drawColor,line width= 0.4pt,line join=round,line cap=round,fill=fillColor] (215.69,236.66) circle (  1.16);

\path[draw=drawColor,line width= 0.4pt,line join=round,line cap=round,fill=fillColor] (215.83,236.53) circle (  1.16);

\path[draw=drawColor,line width= 0.4pt,line join=round,line cap=round,fill=fillColor] (215.97,236.46) circle (  1.16);

\path[draw=drawColor,line width= 0.4pt,line join=round,line cap=round,fill=fillColor] (216.11,236.35) circle (  1.16);

\path[draw=drawColor,line width= 0.4pt,line join=round,line cap=round,fill=fillColor] (216.26,236.10) circle (  1.16);

\path[draw=drawColor,line width= 0.4pt,line join=round,line cap=round,fill=fillColor] (216.40,236.05) circle (  1.16);

\path[draw=drawColor,line width= 0.4pt,line join=round,line cap=round,fill=fillColor] (216.54,236.02) circle (  1.16);

\path[draw=drawColor,line width= 0.4pt,line join=round,line cap=round,fill=fillColor] (216.68,236.00) circle (  1.16);

\path[draw=drawColor,line width= 0.4pt,line join=round,line cap=round,fill=fillColor] (216.82,235.93) circle (  1.16);

\path[draw=drawColor,line width= 0.4pt,line join=round,line cap=round,fill=fillColor] (216.96,235.80) circle (  1.16);

\path[draw=drawColor,line width= 0.4pt,line join=round,line cap=round,fill=fillColor] (217.10,235.77) circle (  1.16);

\path[draw=drawColor,line width= 0.4pt,line join=round,line cap=round,fill=fillColor] (217.24,235.72) circle (  1.16);

\path[draw=drawColor,line width= 0.4pt,line join=round,line cap=round,fill=fillColor] (217.38,235.64) circle (  1.16);

\path[draw=drawColor,line width= 0.4pt,line join=round,line cap=round,fill=fillColor] (217.52,235.61) circle (  1.16);

\path[draw=drawColor,line width= 0.4pt,line join=round,line cap=round,fill=fillColor] (217.66,235.40) circle (  1.16);

\path[draw=drawColor,line width= 0.4pt,line join=round,line cap=round,fill=fillColor] (217.80,235.37) circle (  1.16);

\path[draw=drawColor,line width= 0.4pt,line join=round,line cap=round,fill=fillColor] (217.94,235.08) circle (  1.16);

\path[draw=drawColor,line width= 0.4pt,line join=round,line cap=round,fill=fillColor] (218.08,234.87) circle (  1.16);

\path[draw=drawColor,line width= 0.4pt,line join=round,line cap=round,fill=fillColor] (218.22,234.85) circle (  1.16);

\path[draw=drawColor,line width= 0.4pt,line join=round,line cap=round,fill=fillColor] (218.36,234.67) circle (  1.16);

\path[draw=drawColor,line width= 0.4pt,line join=round,line cap=round,fill=fillColor] (218.50,234.62) circle (  1.16);

\path[draw=drawColor,line width= 0.4pt,line join=round,line cap=round,fill=fillColor] (218.63,234.40) circle (  1.16);

\path[draw=drawColor,line width= 0.4pt,line join=round,line cap=round,fill=fillColor] (218.77,234.33) circle (  1.16);

\path[draw=drawColor,line width= 0.4pt,line join=round,line cap=round,fill=fillColor] (218.91,234.33) circle (  1.16);

\path[draw=drawColor,line width= 0.4pt,line join=round,line cap=round,fill=fillColor] (219.05,233.84) circle (  1.16);

\path[draw=drawColor,line width= 0.4pt,line join=round,line cap=round,fill=fillColor] (219.18,233.68) circle (  1.16);

\path[draw=drawColor,line width= 0.4pt,line join=round,line cap=round,fill=fillColor] (219.32,233.54) circle (  1.16);

\path[draw=drawColor,line width= 0.4pt,line join=round,line cap=round,fill=fillColor] (219.46,233.53) circle (  1.16);

\path[draw=drawColor,line width= 0.4pt,line join=round,line cap=round,fill=fillColor] (219.59,233.38) circle (  1.16);

\path[draw=drawColor,line width= 0.4pt,line join=round,line cap=round,fill=fillColor] (219.73,233.37) circle (  1.16);

\path[draw=drawColor,line width= 0.4pt,line join=round,line cap=round,fill=fillColor] (219.87,233.37) circle (  1.16);

\path[draw=drawColor,line width= 0.4pt,line join=round,line cap=round,fill=fillColor] (220.00,233.25) circle (  1.16);

\path[draw=drawColor,line width= 0.4pt,line join=round,line cap=round,fill=fillColor] (220.14,233.13) circle (  1.16);

\path[draw=drawColor,line width= 0.4pt,line join=round,line cap=round,fill=fillColor] (220.27,233.02) circle (  1.16);

\path[draw=drawColor,line width= 0.4pt,line join=round,line cap=round,fill=fillColor] (220.41,232.85) circle (  1.16);

\path[draw=drawColor,line width= 0.4pt,line join=round,line cap=round,fill=fillColor] (220.54,232.61) circle (  1.16);

\path[draw=drawColor,line width= 0.4pt,line join=round,line cap=round,fill=fillColor] (220.68,232.27) circle (  1.16);

\path[draw=drawColor,line width= 0.4pt,line join=round,line cap=round,fill=fillColor] (220.81,232.11) circle (  1.16);

\path[draw=drawColor,line width= 0.4pt,line join=round,line cap=round,fill=fillColor] (220.95,231.93) circle (  1.16);

\path[draw=drawColor,line width= 0.4pt,line join=round,line cap=round,fill=fillColor] (221.08,231.85) circle (  1.16);

\path[draw=drawColor,line width= 0.4pt,line join=round,line cap=round,fill=fillColor] (221.21,231.85) circle (  1.16);

\path[draw=drawColor,line width= 0.4pt,line join=round,line cap=round,fill=fillColor] (221.35,231.67) circle (  1.16);

\path[draw=drawColor,line width= 0.4pt,line join=round,line cap=round,fill=fillColor] (221.48,230.99) circle (  1.16);

\path[draw=drawColor,line width= 0.4pt,line join=round,line cap=round,fill=fillColor] (221.61,230.89) circle (  1.16);

\path[draw=drawColor,line width= 0.4pt,line join=round,line cap=round,fill=fillColor] (221.75,229.01) circle (  1.16);

\path[draw=drawColor,line width= 0.4pt,line join=round,line cap=round,fill=fillColor] (221.88,228.58) circle (  1.16);

\path[draw=drawColor,line width= 0.4pt,line join=round,line cap=round,fill=fillColor] (222.01,228.43) circle (  1.16);

\path[draw=drawColor,line width= 0.4pt,line join=round,line cap=round,fill=fillColor] (222.14,227.28) circle (  1.16);

\path[draw=drawColor,line width= 0.4pt,line join=round,line cap=round,fill=fillColor] (222.28,227.12) circle (  1.16);

\path[draw=drawColor,line width= 0.4pt,line join=round,line cap=round,fill=fillColor] (222.41,226.87) circle (  1.16);

\path[draw=drawColor,line width= 0.4pt,line join=round,line cap=round,fill=fillColor] (222.54,226.84) circle (  1.16);

\path[draw=drawColor,line width= 0.4pt,line join=round,line cap=round,fill=fillColor] (222.67,226.73) circle (  1.16);

\path[draw=drawColor,line width= 0.4pt,line join=round,line cap=round,fill=fillColor] (222.80,225.81) circle (  1.16);

\path[draw=drawColor,line width= 0.4pt,line join=round,line cap=round,fill=fillColor] (222.93,225.66) circle (  1.16);

\path[draw=drawColor,line width= 0.4pt,line join=round,line cap=round,fill=fillColor] (223.07,225.66) circle (  1.16);

\path[draw=drawColor,line width= 0.4pt,line join=round,line cap=round,fill=fillColor] (223.20,225.50) circle (  1.16);

\path[draw=drawColor,line width= 0.4pt,line join=round,line cap=round,fill=fillColor] (223.33,224.68) circle (  1.16);

\path[draw=drawColor,line width= 0.4pt,line join=round,line cap=round,fill=fillColor] (223.46,223.82) circle (  1.16);

\path[draw=drawColor,line width= 0.4pt,line join=round,line cap=round,fill=fillColor] (223.59,223.77) circle (  1.16);

\path[draw=drawColor,line width= 0.4pt,line join=round,line cap=round,fill=fillColor] (223.72,223.48) circle (  1.16);

\path[draw=drawColor,line width= 0.4pt,line join=round,line cap=round,fill=fillColor] (223.85,223.04) circle (  1.16);

\path[draw=drawColor,line width= 0.4pt,line join=round,line cap=round,fill=fillColor] (223.98,209.81) circle (  1.16);

\path[draw=drawColor,line width= 0.4pt,line join=round,line cap=round,fill=fillColor] (224.11,209.81) circle (  1.16);

\path[draw=drawColor,line width= 0.4pt,line join=round,line cap=round,fill=fillColor] (224.23,209.81) circle (  1.16);

\path[draw=drawColor,line width= 0.4pt,line join=round,line cap=round,fill=fillColor] (224.36,209.81) circle (  1.16);

\path[draw=drawColor,line width= 0.4pt,line join=round,line cap=round,fill=fillColor] (224.49,209.81) circle (  1.16);

\path[draw=drawColor,line width= 0.4pt,line join=round,line cap=round,fill=fillColor] (224.62,209.81) circle (  1.16);

\path[draw=drawColor,line width= 0.4pt,line join=round,line cap=round,fill=fillColor] (224.75,209.81) circle (  1.16);

\path[draw=drawColor,line width= 0.4pt,line join=round,line cap=round,fill=fillColor] (224.88,209.81) circle (  1.16);

\path[draw=drawColor,line width= 0.4pt,line join=round,line cap=round,fill=fillColor] (225.01,209.81) circle (  1.16);

\path[draw=drawColor,line width= 0.4pt,line join=round,line cap=round,fill=fillColor] (225.13,209.81) circle (  1.16);

\path[draw=drawColor,line width= 0.4pt,line join=round,line cap=round,fill=fillColor] (225.26,209.81) circle (  1.16);
\definecolor[named]{drawColor}{rgb}{0.60,0.31,0.64}
\definecolor[named]{fillColor}{rgb}{0.60,0.31,0.64}

\path[draw=drawColor,line width= 0.4pt,line join=round,line cap=round,fill=fillColor] ( 74.88,281.17) circle (  1.16);

\path[draw=drawColor,line width= 0.4pt,line join=round,line cap=round,fill=fillColor] ( 80.74,273.33) circle (  1.16);

\path[draw=drawColor,line width= 0.4pt,line join=round,line cap=round,fill=fillColor] ( 84.84,272.43) circle (  1.16);

\path[draw=drawColor,line width= 0.4pt,line join=round,line cap=round,fill=fillColor] ( 88.11,271.01) circle (  1.16);

\path[draw=drawColor,line width= 0.4pt,line join=round,line cap=round,fill=fillColor] ( 90.88,269.23) circle (  1.16);

\path[draw=drawColor,line width= 0.4pt,line join=round,line cap=round,fill=fillColor] ( 93.29,268.96) circle (  1.16);

\path[draw=drawColor,line width= 0.4pt,line join=round,line cap=round,fill=fillColor] ( 95.45,268.42) circle (  1.16);

\path[draw=drawColor,line width= 0.4pt,line join=round,line cap=round,fill=fillColor] ( 97.41,268.29) circle (  1.16);

\path[draw=drawColor,line width= 0.4pt,line join=round,line cap=round,fill=fillColor] ( 99.21,268.08) circle (  1.16);

\path[draw=drawColor,line width= 0.4pt,line join=round,line cap=round,fill=fillColor] (100.89,267.94) circle (  1.16);

\path[draw=drawColor,line width= 0.4pt,line join=round,line cap=round,fill=fillColor] (102.46,267.57) circle (  1.16);

\path[draw=drawColor,line width= 0.4pt,line join=round,line cap=round,fill=fillColor] (103.93,267.14) circle (  1.16);

\path[draw=drawColor,line width= 0.4pt,line join=round,line cap=round,fill=fillColor] (105.33,266.28) circle (  1.16);

\path[draw=drawColor,line width= 0.4pt,line join=round,line cap=round,fill=fillColor] (106.65,266.14) circle (  1.16);

\path[draw=drawColor,line width= 0.4pt,line join=round,line cap=round,fill=fillColor] (107.91,265.48) circle (  1.16);

\path[draw=drawColor,line width= 0.4pt,line join=round,line cap=round,fill=fillColor] (109.12,265.42) circle (  1.16);

\path[draw=drawColor,line width= 0.4pt,line join=round,line cap=round,fill=fillColor] (110.28,264.63) circle (  1.16);

\path[draw=drawColor,line width= 0.4pt,line join=round,line cap=round,fill=fillColor] (111.40,264.62) circle (  1.16);

\path[draw=drawColor,line width= 0.4pt,line join=round,line cap=round,fill=fillColor] (112.47,264.39) circle (  1.16);

\path[draw=drawColor,line width= 0.4pt,line join=round,line cap=round,fill=fillColor] (113.51,263.86) circle (  1.16);

\path[draw=drawColor,line width= 0.4pt,line join=round,line cap=round,fill=fillColor] (114.51,263.58) circle (  1.16);

\path[draw=drawColor,line width= 0.4pt,line join=round,line cap=round,fill=fillColor] (115.48,263.50) circle (  1.16);

\path[draw=drawColor,line width= 0.4pt,line join=round,line cap=round,fill=fillColor] (116.42,263.12) circle (  1.16);

\path[draw=drawColor,line width= 0.4pt,line join=round,line cap=round,fill=fillColor] (117.34,263.12) circle (  1.16);

\path[draw=drawColor,line width= 0.4pt,line join=round,line cap=round,fill=fillColor] (118.23,262.51) circle (  1.16);

\path[draw=drawColor,line width= 0.4pt,line join=round,line cap=round,fill=fillColor] (119.10,262.32) circle (  1.16);

\path[draw=drawColor,line width= 0.4pt,line join=round,line cap=round,fill=fillColor] (119.94,261.97) circle (  1.16);

\path[draw=drawColor,line width= 0.4pt,line join=round,line cap=round,fill=fillColor] (120.77,261.93) circle (  1.16);

\path[draw=drawColor,line width= 0.4pt,line join=round,line cap=round,fill=fillColor] (121.57,261.76) circle (  1.16);

\path[draw=drawColor,line width= 0.4pt,line join=round,line cap=round,fill=fillColor] (122.36,261.71) circle (  1.16);

\path[draw=drawColor,line width= 0.4pt,line join=round,line cap=round,fill=fillColor] (123.13,261.66) circle (  1.16);

\path[draw=drawColor,line width= 0.4pt,line join=round,line cap=round,fill=fillColor] (123.88,261.24) circle (  1.16);

\path[draw=drawColor,line width= 0.4pt,line join=round,line cap=round,fill=fillColor] (124.62,260.97) circle (  1.16);

\path[draw=drawColor,line width= 0.4pt,line join=round,line cap=round,fill=fillColor] (125.34,260.69) circle (  1.16);

\path[draw=drawColor,line width= 0.4pt,line join=round,line cap=round,fill=fillColor] (126.05,260.63) circle (  1.16);

\path[draw=drawColor,line width= 0.4pt,line join=round,line cap=round,fill=fillColor] (126.74,260.55) circle (  1.16);

\path[draw=drawColor,line width= 0.4pt,line join=round,line cap=round,fill=fillColor] (127.43,260.17) circle (  1.16);

\path[draw=drawColor,line width= 0.4pt,line join=round,line cap=round,fill=fillColor] (128.10,260.08) circle (  1.16);

\path[draw=drawColor,line width= 0.4pt,line join=round,line cap=round,fill=fillColor] (128.76,259.98) circle (  1.16);

\path[draw=drawColor,line width= 0.4pt,line join=round,line cap=round,fill=fillColor] (129.40,259.90) circle (  1.16);

\path[draw=drawColor,line width= 0.4pt,line join=round,line cap=round,fill=fillColor] (130.04,259.66) circle (  1.16);

\path[draw=drawColor,line width= 0.4pt,line join=round,line cap=round,fill=fillColor] (130.67,259.63) circle (  1.16);

\path[draw=drawColor,line width= 0.4pt,line join=round,line cap=round,fill=fillColor] (131.28,259.38) circle (  1.16);

\path[draw=drawColor,line width= 0.4pt,line join=round,line cap=round,fill=fillColor] (131.89,259.21) circle (  1.16);

\path[draw=drawColor,line width= 0.4pt,line join=round,line cap=round,fill=fillColor] (132.49,259.20) circle (  1.16);

\path[draw=drawColor,line width= 0.4pt,line join=round,line cap=round,fill=fillColor] (133.08,259.06) circle (  1.16);

\path[draw=drawColor,line width= 0.4pt,line join=round,line cap=round,fill=fillColor] (133.66,258.93) circle (  1.16);

\path[draw=drawColor,line width= 0.4pt,line join=round,line cap=round,fill=fillColor] (134.23,258.80) circle (  1.16);

\path[draw=drawColor,line width= 0.4pt,line join=round,line cap=round,fill=fillColor] (134.80,258.70) circle (  1.16);

\path[draw=drawColor,line width= 0.4pt,line join=round,line cap=round,fill=fillColor] (135.35,258.65) circle (  1.16);

\path[draw=drawColor,line width= 0.4pt,line join=round,line cap=round,fill=fillColor] (135.90,258.58) circle (  1.16);

\path[draw=drawColor,line width= 0.4pt,line join=round,line cap=round,fill=fillColor] (136.45,258.36) circle (  1.16);

\path[draw=drawColor,line width= 0.4pt,line join=round,line cap=round,fill=fillColor] (136.98,257.92) circle (  1.16);

\path[draw=drawColor,line width= 0.4pt,line join=round,line cap=round,fill=fillColor] (137.51,257.90) circle (  1.16);

\path[draw=drawColor,line width= 0.4pt,line join=round,line cap=round,fill=fillColor] (138.03,257.76) circle (  1.16);

\path[draw=drawColor,line width= 0.4pt,line join=round,line cap=round,fill=fillColor] (138.55,257.55) circle (  1.16);

\path[draw=drawColor,line width= 0.4pt,line join=round,line cap=round,fill=fillColor] (139.06,257.40) circle (  1.16);

\path[draw=drawColor,line width= 0.4pt,line join=round,line cap=round,fill=fillColor] (139.56,257.35) circle (  1.16);

\path[draw=drawColor,line width= 0.4pt,line join=round,line cap=round,fill=fillColor] (140.06,257.18) circle (  1.16);

\path[draw=drawColor,line width= 0.4pt,line join=round,line cap=round,fill=fillColor] (140.55,256.93) circle (  1.16);

\path[draw=drawColor,line width= 0.4pt,line join=round,line cap=round,fill=fillColor] (141.04,256.91) circle (  1.16);

\path[draw=drawColor,line width= 0.4pt,line join=round,line cap=round,fill=fillColor] (141.52,256.91) circle (  1.16);

\path[draw=drawColor,line width= 0.4pt,line join=round,line cap=round,fill=fillColor] (142.00,256.80) circle (  1.16);

\path[draw=drawColor,line width= 0.4pt,line join=round,line cap=round,fill=fillColor] (142.47,256.65) circle (  1.16);

\path[draw=drawColor,line width= 0.4pt,line join=round,line cap=round,fill=fillColor] (142.94,256.62) circle (  1.16);

\path[draw=drawColor,line width= 0.4pt,line join=round,line cap=round,fill=fillColor] (143.40,256.57) circle (  1.16);

\path[draw=drawColor,line width= 0.4pt,line join=round,line cap=round,fill=fillColor] (143.86,256.41) circle (  1.16);

\path[draw=drawColor,line width= 0.4pt,line join=round,line cap=round,fill=fillColor] (144.31,256.30) circle (  1.16);

\path[draw=drawColor,line width= 0.4pt,line join=round,line cap=round,fill=fillColor] (144.76,255.99) circle (  1.16);

\path[draw=drawColor,line width= 0.4pt,line join=round,line cap=round,fill=fillColor] (145.20,255.94) circle (  1.16);

\path[draw=drawColor,line width= 0.4pt,line join=round,line cap=round,fill=fillColor] (145.64,255.77) circle (  1.16);

\path[draw=drawColor,line width= 0.4pt,line join=round,line cap=round,fill=fillColor] (146.08,255.75) circle (  1.16);

\path[draw=drawColor,line width= 0.4pt,line join=round,line cap=round,fill=fillColor] (146.51,255.62) circle (  1.16);

\path[draw=drawColor,line width= 0.4pt,line join=round,line cap=round,fill=fillColor] (146.94,255.61) circle (  1.16);

\path[draw=drawColor,line width= 0.4pt,line join=round,line cap=round,fill=fillColor] (147.36,255.45) circle (  1.16);

\path[draw=drawColor,line width= 0.4pt,line join=round,line cap=round,fill=fillColor] (147.79,255.45) circle (  1.16);

\path[draw=drawColor,line width= 0.4pt,line join=round,line cap=round,fill=fillColor] (148.20,255.44) circle (  1.16);

\path[draw=drawColor,line width= 0.4pt,line join=round,line cap=round,fill=fillColor] (148.62,255.31) circle (  1.16);

\path[draw=drawColor,line width= 0.4pt,line join=round,line cap=round,fill=fillColor] (149.03,255.27) circle (  1.16);

\path[draw=drawColor,line width= 0.4pt,line join=round,line cap=round,fill=fillColor] (149.43,255.18) circle (  1.16);

\path[draw=drawColor,line width= 0.4pt,line join=round,line cap=round,fill=fillColor] (149.83,255.08) circle (  1.16);

\path[draw=drawColor,line width= 0.4pt,line join=round,line cap=round,fill=fillColor] (150.23,254.99) circle (  1.16);

\path[draw=drawColor,line width= 0.4pt,line join=round,line cap=round,fill=fillColor] (150.63,254.93) circle (  1.16);

\path[draw=drawColor,line width= 0.4pt,line join=round,line cap=round,fill=fillColor] (151.02,254.90) circle (  1.16);

\path[draw=drawColor,line width= 0.4pt,line join=round,line cap=round,fill=fillColor] (151.41,254.88) circle (  1.16);

\path[draw=drawColor,line width= 0.4pt,line join=round,line cap=round,fill=fillColor] (151.80,254.76) circle (  1.16);

\path[draw=drawColor,line width= 0.4pt,line join=round,line cap=round,fill=fillColor] (152.18,254.69) circle (  1.16);

\path[draw=drawColor,line width= 0.4pt,line join=round,line cap=round,fill=fillColor] (152.57,254.64) circle (  1.16);

\path[draw=drawColor,line width= 0.4pt,line join=round,line cap=round,fill=fillColor] (152.94,254.61) circle (  1.16);

\path[draw=drawColor,line width= 0.4pt,line join=round,line cap=round,fill=fillColor] (153.32,254.57) circle (  1.16);

\path[draw=drawColor,line width= 0.4pt,line join=round,line cap=round,fill=fillColor] (153.69,254.55) circle (  1.16);

\path[draw=drawColor,line width= 0.4pt,line join=round,line cap=round,fill=fillColor] (154.06,254.54) circle (  1.16);

\path[draw=drawColor,line width= 0.4pt,line join=round,line cap=round,fill=fillColor] (154.43,254.52) circle (  1.16);

\path[draw=drawColor,line width= 0.4pt,line join=round,line cap=round,fill=fillColor] (154.79,254.36) circle (  1.16);

\path[draw=drawColor,line width= 0.4pt,line join=round,line cap=round,fill=fillColor] (155.15,254.34) circle (  1.16);

\path[draw=drawColor,line width= 0.4pt,line join=round,line cap=round,fill=fillColor] (155.51,254.26) circle (  1.16);

\path[draw=drawColor,line width= 0.4pt,line join=round,line cap=round,fill=fillColor] (155.87,254.26) circle (  1.16);

\path[draw=drawColor,line width= 0.4pt,line join=round,line cap=round,fill=fillColor] (156.23,254.22) circle (  1.16);

\path[draw=drawColor,line width= 0.4pt,line join=round,line cap=round,fill=fillColor] (156.58,253.61) circle (  1.16);

\path[draw=drawColor,line width= 0.4pt,line join=round,line cap=round,fill=fillColor] (156.93,253.59) circle (  1.16);

\path[draw=drawColor,line width= 0.4pt,line join=round,line cap=round,fill=fillColor] (157.28,253.38) circle (  1.16);

\path[draw=drawColor,line width= 0.4pt,line join=round,line cap=round,fill=fillColor] (157.62,253.27) circle (  1.16);

\path[draw=drawColor,line width= 0.4pt,line join=round,line cap=round,fill=fillColor] (157.96,253.26) circle (  1.16);

\path[draw=drawColor,line width= 0.4pt,line join=round,line cap=round,fill=fillColor] (158.30,253.15) circle (  1.16);

\path[draw=drawColor,line width= 0.4pt,line join=round,line cap=round,fill=fillColor] (158.64,253.01) circle (  1.16);

\path[draw=drawColor,line width= 0.4pt,line join=round,line cap=round,fill=fillColor] (158.98,252.87) circle (  1.16);

\path[draw=drawColor,line width= 0.4pt,line join=round,line cap=round,fill=fillColor] (159.31,252.67) circle (  1.16);

\path[draw=drawColor,line width= 0.4pt,line join=round,line cap=round,fill=fillColor] (159.65,252.66) circle (  1.16);

\path[draw=drawColor,line width= 0.4pt,line join=round,line cap=round,fill=fillColor] (159.98,252.60) circle (  1.16);

\path[draw=drawColor,line width= 0.4pt,line join=round,line cap=round,fill=fillColor] (160.30,252.52) circle (  1.16);

\path[draw=drawColor,line width= 0.4pt,line join=round,line cap=round,fill=fillColor] (160.63,252.48) circle (  1.16);

\path[draw=drawColor,line width= 0.4pt,line join=round,line cap=round,fill=fillColor] (160.95,252.46) circle (  1.16);

\path[draw=drawColor,line width= 0.4pt,line join=round,line cap=round,fill=fillColor] (161.28,252.44) circle (  1.16);

\path[draw=drawColor,line width= 0.4pt,line join=round,line cap=round,fill=fillColor] (161.60,252.38) circle (  1.16);

\path[draw=drawColor,line width= 0.4pt,line join=round,line cap=round,fill=fillColor] (161.92,252.29) circle (  1.16);

\path[draw=drawColor,line width= 0.4pt,line join=round,line cap=round,fill=fillColor] (162.23,252.01) circle (  1.16);

\path[draw=drawColor,line width= 0.4pt,line join=round,line cap=round,fill=fillColor] (162.55,251.99) circle (  1.16);

\path[draw=drawColor,line width= 0.4pt,line join=round,line cap=round,fill=fillColor] (162.86,251.83) circle (  1.16);

\path[draw=drawColor,line width= 0.4pt,line join=round,line cap=round,fill=fillColor] (163.17,251.82) circle (  1.16);

\path[draw=drawColor,line width= 0.4pt,line join=round,line cap=round,fill=fillColor] (163.48,251.77) circle (  1.16);

\path[draw=drawColor,line width= 0.4pt,line join=round,line cap=round,fill=fillColor] (163.79,251.73) circle (  1.16);

\path[draw=drawColor,line width= 0.4pt,line join=round,line cap=round,fill=fillColor] (164.09,251.63) circle (  1.16);

\path[draw=drawColor,line width= 0.4pt,line join=round,line cap=round,fill=fillColor] (164.40,251.63) circle (  1.16);

\path[draw=drawColor,line width= 0.4pt,line join=round,line cap=round,fill=fillColor] (164.70,251.51) circle (  1.16);

\path[draw=drawColor,line width= 0.4pt,line join=round,line cap=round,fill=fillColor] (165.00,251.50) circle (  1.16);

\path[draw=drawColor,line width= 0.4pt,line join=round,line cap=round,fill=fillColor] (165.30,251.42) circle (  1.16);

\path[draw=drawColor,line width= 0.4pt,line join=round,line cap=round,fill=fillColor] (165.60,251.39) circle (  1.16);

\path[draw=drawColor,line width= 0.4pt,line join=round,line cap=round,fill=fillColor] (165.90,251.38) circle (  1.16);

\path[draw=drawColor,line width= 0.4pt,line join=round,line cap=round,fill=fillColor] (166.19,251.30) circle (  1.16);

\path[draw=drawColor,line width= 0.4pt,line join=round,line cap=round,fill=fillColor] (166.49,251.06) circle (  1.16);

\path[draw=drawColor,line width= 0.4pt,line join=round,line cap=round,fill=fillColor] (166.78,251.02) circle (  1.16);

\path[draw=drawColor,line width= 0.4pt,line join=round,line cap=round,fill=fillColor] (167.07,250.75) circle (  1.16);

\path[draw=drawColor,line width= 0.4pt,line join=round,line cap=round,fill=fillColor] (167.36,250.71) circle (  1.16);

\path[draw=drawColor,line width= 0.4pt,line join=round,line cap=round,fill=fillColor] (167.64,250.67) circle (  1.16);

\path[draw=drawColor,line width= 0.4pt,line join=round,line cap=round,fill=fillColor] (167.93,250.66) circle (  1.16);

\path[draw=drawColor,line width= 0.4pt,line join=round,line cap=round,fill=fillColor] (168.22,250.63) circle (  1.16);

\path[draw=drawColor,line width= 0.4pt,line join=round,line cap=round,fill=fillColor] (168.50,250.62) circle (  1.16);

\path[draw=drawColor,line width= 0.4pt,line join=round,line cap=round,fill=fillColor] (168.78,250.60) circle (  1.16);

\path[draw=drawColor,line width= 0.4pt,line join=round,line cap=round,fill=fillColor] (169.06,250.47) circle (  1.16);

\path[draw=drawColor,line width= 0.4pt,line join=round,line cap=round,fill=fillColor] (169.34,250.44) circle (  1.16);

\path[draw=drawColor,line width= 0.4pt,line join=round,line cap=round,fill=fillColor] (169.62,250.38) circle (  1.16);

\path[draw=drawColor,line width= 0.4pt,line join=round,line cap=round,fill=fillColor] (169.89,250.14) circle (  1.16);

\path[draw=drawColor,line width= 0.4pt,line join=round,line cap=round,fill=fillColor] (170.17,250.13) circle (  1.16);

\path[draw=drawColor,line width= 0.4pt,line join=round,line cap=round,fill=fillColor] (170.44,250.10) circle (  1.16);

\path[draw=drawColor,line width= 0.4pt,line join=round,line cap=round,fill=fillColor] (170.72,250.06) circle (  1.16);

\path[draw=drawColor,line width= 0.4pt,line join=round,line cap=round,fill=fillColor] (170.99,249.99) circle (  1.16);

\path[draw=drawColor,line width= 0.4pt,line join=round,line cap=round,fill=fillColor] (171.26,249.95) circle (  1.16);

\path[draw=drawColor,line width= 0.4pt,line join=round,line cap=round,fill=fillColor] (171.53,249.93) circle (  1.16);

\path[draw=drawColor,line width= 0.4pt,line join=round,line cap=round,fill=fillColor] (171.80,249.85) circle (  1.16);

\path[draw=drawColor,line width= 0.4pt,line join=round,line cap=round,fill=fillColor] (172.06,249.79) circle (  1.16);

\path[draw=drawColor,line width= 0.4pt,line join=round,line cap=round,fill=fillColor] (172.33,249.75) circle (  1.16);

\path[draw=drawColor,line width= 0.4pt,line join=round,line cap=round,fill=fillColor] (172.59,249.74) circle (  1.16);

\path[draw=drawColor,line width= 0.4pt,line join=round,line cap=round,fill=fillColor] (172.85,249.71) circle (  1.16);

\path[draw=drawColor,line width= 0.4pt,line join=round,line cap=round,fill=fillColor] (173.12,249.57) circle (  1.16);

\path[draw=drawColor,line width= 0.4pt,line join=round,line cap=round,fill=fillColor] (173.38,249.41) circle (  1.16);

\path[draw=drawColor,line width= 0.4pt,line join=round,line cap=round,fill=fillColor] (173.64,249.26) circle (  1.16);

\path[draw=drawColor,line width= 0.4pt,line join=round,line cap=round,fill=fillColor] (173.90,249.26) circle (  1.16);

\path[draw=drawColor,line width= 0.4pt,line join=round,line cap=round,fill=fillColor] (174.15,249.18) circle (  1.16);

\path[draw=drawColor,line width= 0.4pt,line join=round,line cap=round,fill=fillColor] (174.41,249.13) circle (  1.16);

\path[draw=drawColor,line width= 0.4pt,line join=round,line cap=round,fill=fillColor] (174.67,249.06) circle (  1.16);

\path[draw=drawColor,line width= 0.4pt,line join=round,line cap=round,fill=fillColor] (174.92,249.02) circle (  1.16);

\path[draw=drawColor,line width= 0.4pt,line join=round,line cap=round,fill=fillColor] (175.17,248.98) circle (  1.16);

\path[draw=drawColor,line width= 0.4pt,line join=round,line cap=round,fill=fillColor] (175.42,248.86) circle (  1.16);

\path[draw=drawColor,line width= 0.4pt,line join=round,line cap=round,fill=fillColor] (175.68,248.84) circle (  1.16);

\path[draw=drawColor,line width= 0.4pt,line join=round,line cap=round,fill=fillColor] (175.93,248.83) circle (  1.16);

\path[draw=drawColor,line width= 0.4pt,line join=round,line cap=round,fill=fillColor] (176.18,248.76) circle (  1.16);

\path[draw=drawColor,line width= 0.4pt,line join=round,line cap=round,fill=fillColor] (176.42,248.76) circle (  1.16);

\path[draw=drawColor,line width= 0.4pt,line join=round,line cap=round,fill=fillColor] (176.67,248.73) circle (  1.16);

\path[draw=drawColor,line width= 0.4pt,line join=round,line cap=round,fill=fillColor] (176.92,248.72) circle (  1.16);

\path[draw=drawColor,line width= 0.4pt,line join=round,line cap=round,fill=fillColor] (177.16,248.62) circle (  1.16);

\path[draw=drawColor,line width= 0.4pt,line join=round,line cap=round,fill=fillColor] (177.41,248.58) circle (  1.16);

\path[draw=drawColor,line width= 0.4pt,line join=round,line cap=round,fill=fillColor] (177.65,248.51) circle (  1.16);

\path[draw=drawColor,line width= 0.4pt,line join=round,line cap=round,fill=fillColor] (177.89,248.48) circle (  1.16);

\path[draw=drawColor,line width= 0.4pt,line join=round,line cap=round,fill=fillColor] (178.13,248.48) circle (  1.16);

\path[draw=drawColor,line width= 0.4pt,line join=round,line cap=round,fill=fillColor] (178.37,248.45) circle (  1.16);

\path[draw=drawColor,line width= 0.4pt,line join=round,line cap=round,fill=fillColor] (178.61,248.26) circle (  1.16);

\path[draw=drawColor,line width= 0.4pt,line join=round,line cap=round,fill=fillColor] (178.85,248.05) circle (  1.16);

\path[draw=drawColor,line width= 0.4pt,line join=round,line cap=round,fill=fillColor] (179.09,248.02) circle (  1.16);

\path[draw=drawColor,line width= 0.4pt,line join=round,line cap=round,fill=fillColor] (179.33,248.01) circle (  1.16);

\path[draw=drawColor,line width= 0.4pt,line join=round,line cap=round,fill=fillColor] (179.56,248.01) circle (  1.16);

\path[draw=drawColor,line width= 0.4pt,line join=round,line cap=round,fill=fillColor] (179.80,247.98) circle (  1.16);

\path[draw=drawColor,line width= 0.4pt,line join=round,line cap=round,fill=fillColor] (180.03,247.93) circle (  1.16);

\path[draw=drawColor,line width= 0.4pt,line join=round,line cap=round,fill=fillColor] (180.27,247.80) circle (  1.16);

\path[draw=drawColor,line width= 0.4pt,line join=round,line cap=round,fill=fillColor] (180.50,247.76) circle (  1.16);

\path[draw=drawColor,line width= 0.4pt,line join=round,line cap=round,fill=fillColor] (180.73,247.73) circle (  1.16);

\path[draw=drawColor,line width= 0.4pt,line join=round,line cap=round,fill=fillColor] (180.96,247.71) circle (  1.16);

\path[draw=drawColor,line width= 0.4pt,line join=round,line cap=round,fill=fillColor] (181.19,247.71) circle (  1.16);

\path[draw=drawColor,line width= 0.4pt,line join=round,line cap=round,fill=fillColor] (181.42,247.68) circle (  1.16);

\path[draw=drawColor,line width= 0.4pt,line join=round,line cap=round,fill=fillColor] (181.65,247.62) circle (  1.16);

\path[draw=drawColor,line width= 0.4pt,line join=round,line cap=round,fill=fillColor] (181.88,247.61) circle (  1.16);

\path[draw=drawColor,line width= 0.4pt,line join=round,line cap=round,fill=fillColor] (182.10,247.58) circle (  1.16);

\path[draw=drawColor,line width= 0.4pt,line join=round,line cap=round,fill=fillColor] (182.33,247.43) circle (  1.16);

\path[draw=drawColor,line width= 0.4pt,line join=round,line cap=round,fill=fillColor] (182.55,247.42) circle (  1.16);

\path[draw=drawColor,line width= 0.4pt,line join=round,line cap=round,fill=fillColor] (182.78,247.41) circle (  1.16);

\path[draw=drawColor,line width= 0.4pt,line join=round,line cap=round,fill=fillColor] (183.00,247.29) circle (  1.16);

\path[draw=drawColor,line width= 0.4pt,line join=round,line cap=round,fill=fillColor] (183.23,247.21) circle (  1.16);

\path[draw=drawColor,line width= 0.4pt,line join=round,line cap=round,fill=fillColor] (183.45,247.10) circle (  1.16);

\path[draw=drawColor,line width= 0.4pt,line join=round,line cap=round,fill=fillColor] (183.67,247.09) circle (  1.16);

\path[draw=drawColor,line width= 0.4pt,line join=round,line cap=round,fill=fillColor] (183.89,246.93) circle (  1.16);

\path[draw=drawColor,line width= 0.4pt,line join=round,line cap=round,fill=fillColor] (184.11,246.92) circle (  1.16);

\path[draw=drawColor,line width= 0.4pt,line join=round,line cap=round,fill=fillColor] (184.33,246.88) circle (  1.16);

\path[draw=drawColor,line width= 0.4pt,line join=round,line cap=round,fill=fillColor] (184.55,246.87) circle (  1.16);

\path[draw=drawColor,line width= 0.4pt,line join=round,line cap=round,fill=fillColor] (184.77,246.86) circle (  1.16);

\path[draw=drawColor,line width= 0.4pt,line join=round,line cap=round,fill=fillColor] (184.98,246.86) circle (  1.16);

\path[draw=drawColor,line width= 0.4pt,line join=round,line cap=round,fill=fillColor] (185.20,246.85) circle (  1.16);

\path[draw=drawColor,line width= 0.4pt,line join=round,line cap=round,fill=fillColor] (185.41,246.84) circle (  1.16);

\path[draw=drawColor,line width= 0.4pt,line join=round,line cap=round,fill=fillColor] (185.63,246.80) circle (  1.16);

\path[draw=drawColor,line width= 0.4pt,line join=round,line cap=round,fill=fillColor] (185.84,246.73) circle (  1.16);

\path[draw=drawColor,line width= 0.4pt,line join=round,line cap=round,fill=fillColor] (186.06,246.71) circle (  1.16);

\path[draw=drawColor,line width= 0.4pt,line join=round,line cap=round,fill=fillColor] (186.27,246.68) circle (  1.16);

\path[draw=drawColor,line width= 0.4pt,line join=round,line cap=round,fill=fillColor] (186.48,246.61) circle (  1.16);

\path[draw=drawColor,line width= 0.4pt,line join=round,line cap=round,fill=fillColor] (186.69,246.61) circle (  1.16);

\path[draw=drawColor,line width= 0.4pt,line join=round,line cap=round,fill=fillColor] (186.91,246.61) circle (  1.16);

\path[draw=drawColor,line width= 0.4pt,line join=round,line cap=round,fill=fillColor] (187.12,246.57) circle (  1.16);

\path[draw=drawColor,line width= 0.4pt,line join=round,line cap=round,fill=fillColor] (187.33,246.57) circle (  1.16);

\path[draw=drawColor,line width= 0.4pt,line join=round,line cap=round,fill=fillColor] (187.53,246.57) circle (  1.16);

\path[draw=drawColor,line width= 0.4pt,line join=round,line cap=round,fill=fillColor] (187.74,246.54) circle (  1.16);

\path[draw=drawColor,line width= 0.4pt,line join=round,line cap=round,fill=fillColor] (187.95,246.54) circle (  1.16);

\path[draw=drawColor,line width= 0.4pt,line join=round,line cap=round,fill=fillColor] (188.16,246.52) circle (  1.16);

\path[draw=drawColor,line width= 0.4pt,line join=round,line cap=round,fill=fillColor] (188.36,246.50) circle (  1.16);

\path[draw=drawColor,line width= 0.4pt,line join=round,line cap=round,fill=fillColor] (188.57,246.44) circle (  1.16);

\path[draw=drawColor,line width= 0.4pt,line join=round,line cap=round,fill=fillColor] (188.77,246.44) circle (  1.16);

\path[draw=drawColor,line width= 0.4pt,line join=round,line cap=round,fill=fillColor] (188.98,246.35) circle (  1.16);

\path[draw=drawColor,line width= 0.4pt,line join=round,line cap=round,fill=fillColor] (189.18,246.29) circle (  1.16);

\path[draw=drawColor,line width= 0.4pt,line join=round,line cap=round,fill=fillColor] (189.39,246.26) circle (  1.16);

\path[draw=drawColor,line width= 0.4pt,line join=round,line cap=round,fill=fillColor] (189.59,246.25) circle (  1.16);

\path[draw=drawColor,line width= 0.4pt,line join=round,line cap=round,fill=fillColor] (189.79,246.17) circle (  1.16);

\path[draw=drawColor,line width= 0.4pt,line join=round,line cap=round,fill=fillColor] (189.99,246.15) circle (  1.16);

\path[draw=drawColor,line width= 0.4pt,line join=round,line cap=round,fill=fillColor] (190.19,246.10) circle (  1.16);

\path[draw=drawColor,line width= 0.4pt,line join=round,line cap=round,fill=fillColor] (190.39,246.09) circle (  1.16);

\path[draw=drawColor,line width= 0.4pt,line join=round,line cap=round,fill=fillColor] (190.59,246.03) circle (  1.16);

\path[draw=drawColor,line width= 0.4pt,line join=round,line cap=round,fill=fillColor] (190.79,246.01) circle (  1.16);

\path[draw=drawColor,line width= 0.4pt,line join=round,line cap=round,fill=fillColor] (190.99,245.98) circle (  1.16);

\path[draw=drawColor,line width= 0.4pt,line join=round,line cap=round,fill=fillColor] (191.19,245.95) circle (  1.16);

\path[draw=drawColor,line width= 0.4pt,line join=round,line cap=round,fill=fillColor] (191.39,245.93) circle (  1.16);

\path[draw=drawColor,line width= 0.4pt,line join=round,line cap=round,fill=fillColor] (191.58,245.92) circle (  1.16);

\path[draw=drawColor,line width= 0.4pt,line join=round,line cap=round,fill=fillColor] (191.78,245.92) circle (  1.16);

\path[draw=drawColor,line width= 0.4pt,line join=round,line cap=round,fill=fillColor] (191.98,245.91) circle (  1.16);

\path[draw=drawColor,line width= 0.4pt,line join=round,line cap=round,fill=fillColor] (192.17,245.84) circle (  1.16);

\path[draw=drawColor,line width= 0.4pt,line join=round,line cap=round,fill=fillColor] (192.37,245.82) circle (  1.16);

\path[draw=drawColor,line width= 0.4pt,line join=round,line cap=round,fill=fillColor] (192.56,245.80) circle (  1.16);

\path[draw=drawColor,line width= 0.4pt,line join=round,line cap=round,fill=fillColor] (192.75,245.73) circle (  1.16);

\path[draw=drawColor,line width= 0.4pt,line join=round,line cap=round,fill=fillColor] (192.95,245.72) circle (  1.16);

\path[draw=drawColor,line width= 0.4pt,line join=round,line cap=round,fill=fillColor] (193.14,245.68) circle (  1.16);

\path[draw=drawColor,line width= 0.4pt,line join=round,line cap=round,fill=fillColor] (193.33,245.67) circle (  1.16);

\path[draw=drawColor,line width= 0.4pt,line join=round,line cap=round,fill=fillColor] (193.52,245.66) circle (  1.16);

\path[draw=drawColor,line width= 0.4pt,line join=round,line cap=round,fill=fillColor] (193.71,245.57) circle (  1.16);

\path[draw=drawColor,line width= 0.4pt,line join=round,line cap=round,fill=fillColor] (193.90,245.57) circle (  1.16);

\path[draw=drawColor,line width= 0.4pt,line join=round,line cap=round,fill=fillColor] (194.09,245.51) circle (  1.16);

\path[draw=drawColor,line width= 0.4pt,line join=round,line cap=round,fill=fillColor] (194.28,245.49) circle (  1.16);

\path[draw=drawColor,line width= 0.4pt,line join=round,line cap=round,fill=fillColor] (194.47,245.47) circle (  1.16);

\path[draw=drawColor,line width= 0.4pt,line join=round,line cap=round,fill=fillColor] (194.66,245.33) circle (  1.16);

\path[draw=drawColor,line width= 0.4pt,line join=round,line cap=round,fill=fillColor] (194.85,245.30) circle (  1.16);

\path[draw=drawColor,line width= 0.4pt,line join=round,line cap=round,fill=fillColor] (195.04,245.22) circle (  1.16);

\path[draw=drawColor,line width= 0.4pt,line join=round,line cap=round,fill=fillColor] (195.22,245.19) circle (  1.16);

\path[draw=drawColor,line width= 0.4pt,line join=round,line cap=round,fill=fillColor] (195.41,245.12) circle (  1.16);

\path[draw=drawColor,line width= 0.4pt,line join=round,line cap=round,fill=fillColor] (195.60,245.09) circle (  1.16);

\path[draw=drawColor,line width= 0.4pt,line join=round,line cap=round,fill=fillColor] (195.78,245.05) circle (  1.16);

\path[draw=drawColor,line width= 0.4pt,line join=round,line cap=round,fill=fillColor] (195.97,245.04) circle (  1.16);

\path[draw=drawColor,line width= 0.4pt,line join=round,line cap=round,fill=fillColor] (196.15,244.92) circle (  1.16);

\path[draw=drawColor,line width= 0.4pt,line join=round,line cap=round,fill=fillColor] (196.34,244.86) circle (  1.16);

\path[draw=drawColor,line width= 0.4pt,line join=round,line cap=round,fill=fillColor] (196.52,244.82) circle (  1.16);

\path[draw=drawColor,line width= 0.4pt,line join=round,line cap=round,fill=fillColor] (196.70,244.81) circle (  1.16);

\path[draw=drawColor,line width= 0.4pt,line join=round,line cap=round,fill=fillColor] (196.89,244.76) circle (  1.16);

\path[draw=drawColor,line width= 0.4pt,line join=round,line cap=round,fill=fillColor] (197.07,244.75) circle (  1.16);

\path[draw=drawColor,line width= 0.4pt,line join=round,line cap=round,fill=fillColor] (197.25,244.75) circle (  1.16);

\path[draw=drawColor,line width= 0.4pt,line join=round,line cap=round,fill=fillColor] (197.43,244.68) circle (  1.16);

\path[draw=drawColor,line width= 0.4pt,line join=round,line cap=round,fill=fillColor] (197.61,244.65) circle (  1.16);

\path[draw=drawColor,line width= 0.4pt,line join=round,line cap=round,fill=fillColor] (197.79,244.61) circle (  1.16);

\path[draw=drawColor,line width= 0.4pt,line join=round,line cap=round,fill=fillColor] (197.97,244.58) circle (  1.16);

\path[draw=drawColor,line width= 0.4pt,line join=round,line cap=round,fill=fillColor] (198.15,244.44) circle (  1.16);

\path[draw=drawColor,line width= 0.4pt,line join=round,line cap=round,fill=fillColor] (198.33,244.26) circle (  1.16);

\path[draw=drawColor,line width= 0.4pt,line join=round,line cap=round,fill=fillColor] (198.51,244.17) circle (  1.16);

\path[draw=drawColor,line width= 0.4pt,line join=round,line cap=round,fill=fillColor] (198.69,244.14) circle (  1.16);

\path[draw=drawColor,line width= 0.4pt,line join=round,line cap=round,fill=fillColor] (198.87,244.11) circle (  1.16);

\path[draw=drawColor,line width= 0.4pt,line join=round,line cap=round,fill=fillColor] (199.04,244.08) circle (  1.16);

\path[draw=drawColor,line width= 0.4pt,line join=round,line cap=round,fill=fillColor] (199.22,244.08) circle (  1.16);

\path[draw=drawColor,line width= 0.4pt,line join=round,line cap=round,fill=fillColor] (199.40,244.06) circle (  1.16);

\path[draw=drawColor,line width= 0.4pt,line join=round,line cap=round,fill=fillColor] (199.57,244.06) circle (  1.16);

\path[draw=drawColor,line width= 0.4pt,line join=round,line cap=round,fill=fillColor] (199.75,244.02) circle (  1.16);

\path[draw=drawColor,line width= 0.4pt,line join=round,line cap=round,fill=fillColor] (199.92,243.96) circle (  1.16);

\path[draw=drawColor,line width= 0.4pt,line join=round,line cap=round,fill=fillColor] (200.10,243.95) circle (  1.16);

\path[draw=drawColor,line width= 0.4pt,line join=round,line cap=round,fill=fillColor] (200.27,243.91) circle (  1.16);

\path[draw=drawColor,line width= 0.4pt,line join=round,line cap=round,fill=fillColor] (200.45,243.91) circle (  1.16);

\path[draw=drawColor,line width= 0.4pt,line join=round,line cap=round,fill=fillColor] (200.62,243.90) circle (  1.16);

\path[draw=drawColor,line width= 0.4pt,line join=round,line cap=round,fill=fillColor] (200.79,243.87) circle (  1.16);

\path[draw=drawColor,line width= 0.4pt,line join=round,line cap=round,fill=fillColor] (200.97,243.76) circle (  1.16);

\path[draw=drawColor,line width= 0.4pt,line join=round,line cap=round,fill=fillColor] (201.14,243.75) circle (  1.16);

\path[draw=drawColor,line width= 0.4pt,line join=round,line cap=round,fill=fillColor] (201.31,243.74) circle (  1.16);

\path[draw=drawColor,line width= 0.4pt,line join=round,line cap=round,fill=fillColor] (201.48,243.73) circle (  1.16);

\path[draw=drawColor,line width= 0.4pt,line join=round,line cap=round,fill=fillColor] (201.65,243.70) circle (  1.16);

\path[draw=drawColor,line width= 0.4pt,line join=round,line cap=round,fill=fillColor] (201.83,243.66) circle (  1.16);

\path[draw=drawColor,line width= 0.4pt,line join=round,line cap=round,fill=fillColor] (202.00,243.51) circle (  1.16);

\path[draw=drawColor,line width= 0.4pt,line join=round,line cap=round,fill=fillColor] (202.17,243.48) circle (  1.16);

\path[draw=drawColor,line width= 0.4pt,line join=round,line cap=round,fill=fillColor] (202.34,243.42) circle (  1.16);

\path[draw=drawColor,line width= 0.4pt,line join=round,line cap=round,fill=fillColor] (202.50,243.33) circle (  1.16);

\path[draw=drawColor,line width= 0.4pt,line join=round,line cap=round,fill=fillColor] (202.67,243.32) circle (  1.16);

\path[draw=drawColor,line width= 0.4pt,line join=round,line cap=round,fill=fillColor] (202.84,243.19) circle (  1.16);

\path[draw=drawColor,line width= 0.4pt,line join=round,line cap=round,fill=fillColor] (203.01,243.10) circle (  1.16);

\path[draw=drawColor,line width= 0.4pt,line join=round,line cap=round,fill=fillColor] (203.18,243.07) circle (  1.16);

\path[draw=drawColor,line width= 0.4pt,line join=round,line cap=round,fill=fillColor] (203.35,243.03) circle (  1.16);

\path[draw=drawColor,line width= 0.4pt,line join=round,line cap=round,fill=fillColor] (203.51,243.00) circle (  1.16);

\path[draw=drawColor,line width= 0.4pt,line join=round,line cap=round,fill=fillColor] (203.68,242.95) circle (  1.16);

\path[draw=drawColor,line width= 0.4pt,line join=round,line cap=round,fill=fillColor] (203.85,242.90) circle (  1.16);

\path[draw=drawColor,line width= 0.4pt,line join=round,line cap=round,fill=fillColor] (204.01,242.90) circle (  1.16);

\path[draw=drawColor,line width= 0.4pt,line join=round,line cap=round,fill=fillColor] (204.18,242.90) circle (  1.16);

\path[draw=drawColor,line width= 0.4pt,line join=round,line cap=round,fill=fillColor] (204.34,242.77) circle (  1.16);

\path[draw=drawColor,line width= 0.4pt,line join=round,line cap=round,fill=fillColor] (204.51,242.77) circle (  1.16);

\path[draw=drawColor,line width= 0.4pt,line join=round,line cap=round,fill=fillColor] (204.67,242.77) circle (  1.16);

\path[draw=drawColor,line width= 0.4pt,line join=round,line cap=round,fill=fillColor] (204.84,242.76) circle (  1.16);

\path[draw=drawColor,line width= 0.4pt,line join=round,line cap=round,fill=fillColor] (205.00,242.70) circle (  1.16);

\path[draw=drawColor,line width= 0.4pt,line join=round,line cap=round,fill=fillColor] (205.16,242.67) circle (  1.16);

\path[draw=drawColor,line width= 0.4pt,line join=round,line cap=round,fill=fillColor] (205.33,242.60) circle (  1.16);

\path[draw=drawColor,line width= 0.4pt,line join=round,line cap=round,fill=fillColor] (205.49,242.55) circle (  1.16);

\path[draw=drawColor,line width= 0.4pt,line join=round,line cap=round,fill=fillColor] (205.65,242.53) circle (  1.16);

\path[draw=drawColor,line width= 0.4pt,line join=round,line cap=round,fill=fillColor] (205.81,242.53) circle (  1.16);

\path[draw=drawColor,line width= 0.4pt,line join=round,line cap=round,fill=fillColor] (205.97,242.52) circle (  1.16);

\path[draw=drawColor,line width= 0.4pt,line join=round,line cap=round,fill=fillColor] (206.14,242.51) circle (  1.16);

\path[draw=drawColor,line width= 0.4pt,line join=round,line cap=round,fill=fillColor] (206.30,242.46) circle (  1.16);

\path[draw=drawColor,line width= 0.4pt,line join=round,line cap=round,fill=fillColor] (206.46,242.44) circle (  1.16);

\path[draw=drawColor,line width= 0.4pt,line join=round,line cap=round,fill=fillColor] (206.62,242.38) circle (  1.16);

\path[draw=drawColor,line width= 0.4pt,line join=round,line cap=round,fill=fillColor] (206.78,242.34) circle (  1.16);

\path[draw=drawColor,line width= 0.4pt,line join=round,line cap=round,fill=fillColor] (206.94,242.28) circle (  1.16);

\path[draw=drawColor,line width= 0.4pt,line join=round,line cap=round,fill=fillColor] (207.10,242.28) circle (  1.16);

\path[draw=drawColor,line width= 0.4pt,line join=round,line cap=round,fill=fillColor] (207.26,242.24) circle (  1.16);

\path[draw=drawColor,line width= 0.4pt,line join=round,line cap=round,fill=fillColor] (207.42,242.19) circle (  1.16);

\path[draw=drawColor,line width= 0.4pt,line join=round,line cap=round,fill=fillColor] (207.57,242.13) circle (  1.16);

\path[draw=drawColor,line width= 0.4pt,line join=round,line cap=round,fill=fillColor] (207.73,242.04) circle (  1.16);

\path[draw=drawColor,line width= 0.4pt,line join=round,line cap=round,fill=fillColor] (207.89,241.93) circle (  1.16);

\path[draw=drawColor,line width= 0.4pt,line join=round,line cap=round,fill=fillColor] (208.05,241.89) circle (  1.16);

\path[draw=drawColor,line width= 0.4pt,line join=round,line cap=round,fill=fillColor] (208.20,241.89) circle (  1.16);

\path[draw=drawColor,line width= 0.4pt,line join=round,line cap=round,fill=fillColor] (208.36,241.83) circle (  1.16);

\path[draw=drawColor,line width= 0.4pt,line join=round,line cap=round,fill=fillColor] (208.52,241.79) circle (  1.16);

\path[draw=drawColor,line width= 0.4pt,line join=round,line cap=round,fill=fillColor] (208.67,241.77) circle (  1.16);

\path[draw=drawColor,line width= 0.4pt,line join=round,line cap=round,fill=fillColor] (208.83,241.77) circle (  1.16);

\path[draw=drawColor,line width= 0.4pt,line join=round,line cap=round,fill=fillColor] (208.98,241.75) circle (  1.16);

\path[draw=drawColor,line width= 0.4pt,line join=round,line cap=round,fill=fillColor] (209.14,241.56) circle (  1.16);

\path[draw=drawColor,line width= 0.4pt,line join=round,line cap=round,fill=fillColor] (209.29,241.42) circle (  1.16);

\path[draw=drawColor,line width= 0.4pt,line join=round,line cap=round,fill=fillColor] (209.45,241.33) circle (  1.16);

\path[draw=drawColor,line width= 0.4pt,line join=round,line cap=round,fill=fillColor] (209.60,241.29) circle (  1.16);

\path[draw=drawColor,line width= 0.4pt,line join=round,line cap=round,fill=fillColor] (209.76,241.23) circle (  1.16);

\path[draw=drawColor,line width= 0.4pt,line join=round,line cap=round,fill=fillColor] (209.91,241.17) circle (  1.16);

\path[draw=drawColor,line width= 0.4pt,line join=round,line cap=round,fill=fillColor] (210.07,241.12) circle (  1.16);

\path[draw=drawColor,line width= 0.4pt,line join=round,line cap=round,fill=fillColor] (210.22,241.03) circle (  1.16);

\path[draw=drawColor,line width= 0.4pt,line join=round,line cap=round,fill=fillColor] (210.37,241.01) circle (  1.16);

\path[draw=drawColor,line width= 0.4pt,line join=round,line cap=round,fill=fillColor] (210.52,240.96) circle (  1.16);

\path[draw=drawColor,line width= 0.4pt,line join=round,line cap=round,fill=fillColor] (210.68,240.94) circle (  1.16);

\path[draw=drawColor,line width= 0.4pt,line join=round,line cap=round,fill=fillColor] (210.83,240.91) circle (  1.16);

\path[draw=drawColor,line width= 0.4pt,line join=round,line cap=round,fill=fillColor] (210.98,240.86) circle (  1.16);

\path[draw=drawColor,line width= 0.4pt,line join=round,line cap=round,fill=fillColor] (211.13,240.86) circle (  1.16);

\path[draw=drawColor,line width= 0.4pt,line join=round,line cap=round,fill=fillColor] (211.28,240.74) circle (  1.16);

\path[draw=drawColor,line width= 0.4pt,line join=round,line cap=round,fill=fillColor] (211.43,240.73) circle (  1.16);

\path[draw=drawColor,line width= 0.4pt,line join=round,line cap=round,fill=fillColor] (211.58,240.73) circle (  1.16);

\path[draw=drawColor,line width= 0.4pt,line join=round,line cap=round,fill=fillColor] (211.73,240.68) circle (  1.16);

\path[draw=drawColor,line width= 0.4pt,line join=round,line cap=round,fill=fillColor] (211.88,240.62) circle (  1.16);

\path[draw=drawColor,line width= 0.4pt,line join=round,line cap=round,fill=fillColor] (212.03,240.59) circle (  1.16);

\path[draw=drawColor,line width= 0.4pt,line join=round,line cap=round,fill=fillColor] (212.18,240.53) circle (  1.16);

\path[draw=drawColor,line width= 0.4pt,line join=round,line cap=round,fill=fillColor] (212.33,240.51) circle (  1.16);

\path[draw=drawColor,line width= 0.4pt,line join=round,line cap=round,fill=fillColor] (212.48,240.44) circle (  1.16);

\path[draw=drawColor,line width= 0.4pt,line join=round,line cap=round,fill=fillColor] (212.63,240.39) circle (  1.16);

\path[draw=drawColor,line width= 0.4pt,line join=round,line cap=round,fill=fillColor] (212.78,240.35) circle (  1.16);

\path[draw=drawColor,line width= 0.4pt,line join=round,line cap=round,fill=fillColor] (212.93,240.29) circle (  1.16);

\path[draw=drawColor,line width= 0.4pt,line join=round,line cap=round,fill=fillColor] (213.07,240.26) circle (  1.16);

\path[draw=drawColor,line width= 0.4pt,line join=round,line cap=round,fill=fillColor] (213.22,240.12) circle (  1.16);

\path[draw=drawColor,line width= 0.4pt,line join=round,line cap=round,fill=fillColor] (213.37,240.11) circle (  1.16);

\path[draw=drawColor,line width= 0.4pt,line join=round,line cap=round,fill=fillColor] (213.51,240.09) circle (  1.16);

\path[draw=drawColor,line width= 0.4pt,line join=round,line cap=round,fill=fillColor] (213.66,239.98) circle (  1.16);

\path[draw=drawColor,line width= 0.4pt,line join=round,line cap=round,fill=fillColor] (213.81,239.80) circle (  1.16);

\path[draw=drawColor,line width= 0.4pt,line join=round,line cap=round,fill=fillColor] (213.95,239.72) circle (  1.16);

\path[draw=drawColor,line width= 0.4pt,line join=round,line cap=round,fill=fillColor] (214.10,239.65) circle (  1.16);

\path[draw=drawColor,line width= 0.4pt,line join=round,line cap=round,fill=fillColor] (214.24,239.56) circle (  1.16);

\path[draw=drawColor,line width= 0.4pt,line join=round,line cap=round,fill=fillColor] (214.39,239.51) circle (  1.16);

\path[draw=drawColor,line width= 0.4pt,line join=round,line cap=round,fill=fillColor] (214.54,239.47) circle (  1.16);

\path[draw=drawColor,line width= 0.4pt,line join=round,line cap=round,fill=fillColor] (214.68,239.36) circle (  1.16);

\path[draw=drawColor,line width= 0.4pt,line join=round,line cap=round,fill=fillColor] (214.82,239.30) circle (  1.16);

\path[draw=drawColor,line width= 0.4pt,line join=round,line cap=round,fill=fillColor] (214.97,239.15) circle (  1.16);

\path[draw=drawColor,line width= 0.4pt,line join=round,line cap=round,fill=fillColor] (215.11,239.12) circle (  1.16);

\path[draw=drawColor,line width= 0.4pt,line join=round,line cap=round,fill=fillColor] (215.26,238.85) circle (  1.16);

\path[draw=drawColor,line width= 0.4pt,line join=round,line cap=round,fill=fillColor] (215.40,238.80) circle (  1.16);

\path[draw=drawColor,line width= 0.4pt,line join=round,line cap=round,fill=fillColor] (215.54,238.57) circle (  1.16);

\path[draw=drawColor,line width= 0.4pt,line join=round,line cap=round,fill=fillColor] (215.69,238.45) circle (  1.16);

\path[draw=drawColor,line width= 0.4pt,line join=round,line cap=round,fill=fillColor] (215.83,238.43) circle (  1.16);

\path[draw=drawColor,line width= 0.4pt,line join=round,line cap=round,fill=fillColor] (215.97,238.25) circle (  1.16);

\path[draw=drawColor,line width= 0.4pt,line join=round,line cap=round,fill=fillColor] (216.11,238.20) circle (  1.16);

\path[draw=drawColor,line width= 0.4pt,line join=round,line cap=round,fill=fillColor] (216.26,238.20) circle (  1.16);

\path[draw=drawColor,line width= 0.4pt,line join=round,line cap=round,fill=fillColor] (216.40,238.09) circle (  1.16);

\path[draw=drawColor,line width= 0.4pt,line join=round,line cap=round,fill=fillColor] (216.54,238.09) circle (  1.16);

\path[draw=drawColor,line width= 0.4pt,line join=round,line cap=round,fill=fillColor] (216.68,238.08) circle (  1.16);

\path[draw=drawColor,line width= 0.4pt,line join=round,line cap=round,fill=fillColor] (216.82,238.03) circle (  1.16);

\path[draw=drawColor,line width= 0.4pt,line join=round,line cap=round,fill=fillColor] (216.96,238.03) circle (  1.16);

\path[draw=drawColor,line width= 0.4pt,line join=round,line cap=round,fill=fillColor] (217.10,237.97) circle (  1.16);

\path[draw=drawColor,line width= 0.4pt,line join=round,line cap=round,fill=fillColor] (217.24,237.96) circle (  1.16);

\path[draw=drawColor,line width= 0.4pt,line join=round,line cap=round,fill=fillColor] (217.38,237.94) circle (  1.16);

\path[draw=drawColor,line width= 0.4pt,line join=round,line cap=round,fill=fillColor] (217.52,237.87) circle (  1.16);

\path[draw=drawColor,line width= 0.4pt,line join=round,line cap=round,fill=fillColor] (217.66,237.65) circle (  1.16);

\path[draw=drawColor,line width= 0.4pt,line join=round,line cap=round,fill=fillColor] (217.80,237.56) circle (  1.16);

\path[draw=drawColor,line width= 0.4pt,line join=round,line cap=round,fill=fillColor] (217.94,237.52) circle (  1.16);

\path[draw=drawColor,line width= 0.4pt,line join=round,line cap=round,fill=fillColor] (218.08,237.38) circle (  1.16);

\path[draw=drawColor,line width= 0.4pt,line join=round,line cap=round,fill=fillColor] (218.22,237.30) circle (  1.16);

\path[draw=drawColor,line width= 0.4pt,line join=round,line cap=round,fill=fillColor] (218.36,237.13) circle (  1.16);

\path[draw=drawColor,line width= 0.4pt,line join=round,line cap=round,fill=fillColor] (218.50,236.90) circle (  1.16);

\path[draw=drawColor,line width= 0.4pt,line join=round,line cap=round,fill=fillColor] (218.63,236.81) circle (  1.16);

\path[draw=drawColor,line width= 0.4pt,line join=round,line cap=round,fill=fillColor] (218.77,236.70) circle (  1.16);

\path[draw=drawColor,line width= 0.4pt,line join=round,line cap=round,fill=fillColor] (218.91,236.69) circle (  1.16);

\path[draw=drawColor,line width= 0.4pt,line join=round,line cap=round,fill=fillColor] (219.05,236.51) circle (  1.16);

\path[draw=drawColor,line width= 0.4pt,line join=round,line cap=round,fill=fillColor] (219.18,236.48) circle (  1.16);

\path[draw=drawColor,line width= 0.4pt,line join=round,line cap=round,fill=fillColor] (219.32,236.25) circle (  1.16);

\path[draw=drawColor,line width= 0.4pt,line join=round,line cap=round,fill=fillColor] (219.46,236.08) circle (  1.16);

\path[draw=drawColor,line width= 0.4pt,line join=round,line cap=round,fill=fillColor] (219.59,236.03) circle (  1.16);

\path[draw=drawColor,line width= 0.4pt,line join=round,line cap=round,fill=fillColor] (219.73,235.35) circle (  1.16);

\path[draw=drawColor,line width= 0.4pt,line join=round,line cap=round,fill=fillColor] (219.87,235.21) circle (  1.16);

\path[draw=drawColor,line width= 0.4pt,line join=round,line cap=round,fill=fillColor] (220.00,234.97) circle (  1.16);

\path[draw=drawColor,line width= 0.4pt,line join=round,line cap=round,fill=fillColor] (220.14,234.86) circle (  1.16);

\path[draw=drawColor,line width= 0.4pt,line join=round,line cap=round,fill=fillColor] (220.27,234.74) circle (  1.16);

\path[draw=drawColor,line width= 0.4pt,line join=round,line cap=round,fill=fillColor] (220.41,234.59) circle (  1.16);

\path[draw=drawColor,line width= 0.4pt,line join=round,line cap=round,fill=fillColor] (220.54,234.20) circle (  1.16);

\path[draw=drawColor,line width= 0.4pt,line join=round,line cap=round,fill=fillColor] (220.68,234.08) circle (  1.16);

\path[draw=drawColor,line width= 0.4pt,line join=round,line cap=round,fill=fillColor] (220.81,233.95) circle (  1.16);

\path[draw=drawColor,line width= 0.4pt,line join=round,line cap=round,fill=fillColor] (220.95,233.95) circle (  1.16);

\path[draw=drawColor,line width= 0.4pt,line join=round,line cap=round,fill=fillColor] (221.08,233.84) circle (  1.16);

\path[draw=drawColor,line width= 0.4pt,line join=round,line cap=round,fill=fillColor] (221.21,233.84) circle (  1.16);

\path[draw=drawColor,line width= 0.4pt,line join=round,line cap=round,fill=fillColor] (221.35,233.82) circle (  1.16);

\path[draw=drawColor,line width= 0.4pt,line join=round,line cap=round,fill=fillColor] (221.48,233.62) circle (  1.16);

\path[draw=drawColor,line width= 0.4pt,line join=round,line cap=round,fill=fillColor] (221.61,233.25) circle (  1.16);

\path[draw=drawColor,line width= 0.4pt,line join=round,line cap=round,fill=fillColor] (221.75,233.08) circle (  1.16);

\path[draw=drawColor,line width= 0.4pt,line join=round,line cap=round,fill=fillColor] (221.88,231.27) circle (  1.16);

\path[draw=drawColor,line width= 0.4pt,line join=round,line cap=round,fill=fillColor] (222.01,231.21) circle (  1.16);

\path[draw=drawColor,line width= 0.4pt,line join=round,line cap=round,fill=fillColor] (222.14,230.25) circle (  1.16);

\path[draw=drawColor,line width= 0.4pt,line join=round,line cap=round,fill=fillColor] (222.28,230.22) circle (  1.16);

\path[draw=drawColor,line width= 0.4pt,line join=round,line cap=round,fill=fillColor] (222.41,230.21) circle (  1.16);

\path[draw=drawColor,line width= 0.4pt,line join=round,line cap=round,fill=fillColor] (222.54,229.85) circle (  1.16);

\path[draw=drawColor,line width= 0.4pt,line join=round,line cap=round,fill=fillColor] (222.67,229.44) circle (  1.16);

\path[draw=drawColor,line width= 0.4pt,line join=round,line cap=round,fill=fillColor] (222.80,229.18) circle (  1.16);

\path[draw=drawColor,line width= 0.4pt,line join=round,line cap=round,fill=fillColor] (222.93,229.10) circle (  1.16);

\path[draw=drawColor,line width= 0.4pt,line join=round,line cap=round,fill=fillColor] (223.07,229.06) circle (  1.16);

\path[draw=drawColor,line width= 0.4pt,line join=round,line cap=round,fill=fillColor] (223.20,228.72) circle (  1.16);

\path[draw=drawColor,line width= 0.4pt,line join=round,line cap=round,fill=fillColor] (223.33,228.50) circle (  1.16);

\path[draw=drawColor,line width= 0.4pt,line join=round,line cap=round,fill=fillColor] (223.46,227.55) circle (  1.16);

\path[draw=drawColor,line width= 0.4pt,line join=round,line cap=round,fill=fillColor] (223.59,226.63) circle (  1.16);

\path[draw=drawColor,line width= 0.4pt,line join=round,line cap=round,fill=fillColor] (223.72,223.93) circle (  1.16);

\path[draw=drawColor,line width= 0.4pt,line join=round,line cap=round,fill=fillColor] (223.85,223.47) circle (  1.16);

\path[draw=drawColor,line width= 0.4pt,line join=round,line cap=round,fill=fillColor] (223.98,222.82) circle (  1.16);

\path[draw=drawColor,line width= 0.4pt,line join=round,line cap=round,fill=fillColor] (224.11,220.28) circle (  1.16);

\path[draw=drawColor,line width= 0.4pt,line join=round,line cap=round,fill=fillColor] (224.23,219.57) circle (  1.16);

\path[draw=drawColor,line width= 0.4pt,line join=round,line cap=round,fill=fillColor] (224.36,209.81) circle (  1.16);

\path[draw=drawColor,line width= 0.4pt,line join=round,line cap=round,fill=fillColor] (224.49,209.81) circle (  1.16);

\path[draw=drawColor,line width= 0.4pt,line join=round,line cap=round,fill=fillColor] (224.62,209.81) circle (  1.16);

\path[draw=drawColor,line width= 0.4pt,line join=round,line cap=round,fill=fillColor] (224.75,209.81) circle (  1.16);

\path[draw=drawColor,line width= 0.4pt,line join=round,line cap=round,fill=fillColor] (224.88,209.81) circle (  1.16);

\path[draw=drawColor,line width= 0.4pt,line join=round,line cap=round,fill=fillColor] (225.01,209.81) circle (  1.16);

\path[draw=drawColor,line width= 0.4pt,line join=round,line cap=round,fill=fillColor] (225.13,209.81) circle (  1.16);

\path[draw=drawColor,line width= 0.4pt,line join=round,line cap=round,fill=fillColor] (225.26,209.81) circle (  1.16);
\definecolor[named]{drawColor}{rgb}{1.00,0.50,0.00}
\definecolor[named]{fillColor}{rgb}{1.00,0.50,0.00}

\path[draw=drawColor,line width= 0.4pt,line join=round,line cap=round,fill=fillColor] ( 74.88,280.51) circle (  1.16);

\path[draw=drawColor,line width= 0.4pt,line join=round,line cap=round,fill=fillColor] ( 80.74,278.01) circle (  1.16);

\path[draw=drawColor,line width= 0.4pt,line join=round,line cap=round,fill=fillColor] ( 84.84,273.84) circle (  1.16);

\path[draw=drawColor,line width= 0.4pt,line join=round,line cap=round,fill=fillColor] ( 88.11,273.71) circle (  1.16);

\path[draw=drawColor,line width= 0.4pt,line join=round,line cap=round,fill=fillColor] ( 90.88,273.00) circle (  1.16);

\path[draw=drawColor,line width= 0.4pt,line join=round,line cap=round,fill=fillColor] ( 93.29,272.51) circle (  1.16);

\path[draw=drawColor,line width= 0.4pt,line join=round,line cap=round,fill=fillColor] ( 95.45,272.50) circle (  1.16);

\path[draw=drawColor,line width= 0.4pt,line join=round,line cap=round,fill=fillColor] ( 97.41,270.07) circle (  1.16);

\path[draw=drawColor,line width= 0.4pt,line join=round,line cap=round,fill=fillColor] ( 99.21,269.21) circle (  1.16);

\path[draw=drawColor,line width= 0.4pt,line join=round,line cap=round,fill=fillColor] (100.89,268.86) circle (  1.16);

\path[draw=drawColor,line width= 0.4pt,line join=round,line cap=round,fill=fillColor] (102.46,268.44) circle (  1.16);

\path[draw=drawColor,line width= 0.4pt,line join=round,line cap=round,fill=fillColor] (103.93,268.13) circle (  1.16);

\path[draw=drawColor,line width= 0.4pt,line join=round,line cap=round,fill=fillColor] (105.33,267.99) circle (  1.16);

\path[draw=drawColor,line width= 0.4pt,line join=round,line cap=round,fill=fillColor] (106.65,267.40) circle (  1.16);

\path[draw=drawColor,line width= 0.4pt,line join=round,line cap=round,fill=fillColor] (107.91,267.38) circle (  1.16);

\path[draw=drawColor,line width= 0.4pt,line join=round,line cap=round,fill=fillColor] (109.12,266.97) circle (  1.16);

\path[draw=drawColor,line width= 0.4pt,line join=round,line cap=round,fill=fillColor] (110.28,266.22) circle (  1.16);

\path[draw=drawColor,line width= 0.4pt,line join=round,line cap=round,fill=fillColor] (111.40,266.18) circle (  1.16);

\path[draw=drawColor,line width= 0.4pt,line join=round,line cap=round,fill=fillColor] (112.47,266.01) circle (  1.16);

\path[draw=drawColor,line width= 0.4pt,line join=round,line cap=round,fill=fillColor] (113.51,265.84) circle (  1.16);

\path[draw=drawColor,line width= 0.4pt,line join=round,line cap=round,fill=fillColor] (114.51,265.66) circle (  1.16);

\path[draw=drawColor,line width= 0.4pt,line join=round,line cap=round,fill=fillColor] (115.48,265.35) circle (  1.16);

\path[draw=drawColor,line width= 0.4pt,line join=round,line cap=round,fill=fillColor] (116.42,265.27) circle (  1.16);

\path[draw=drawColor,line width= 0.4pt,line join=round,line cap=round,fill=fillColor] (117.34,265.12) circle (  1.16);

\path[draw=drawColor,line width= 0.4pt,line join=round,line cap=round,fill=fillColor] (118.23,264.29) circle (  1.16);

\path[draw=drawColor,line width= 0.4pt,line join=round,line cap=round,fill=fillColor] (119.10,264.27) circle (  1.16);

\path[draw=drawColor,line width= 0.4pt,line join=round,line cap=round,fill=fillColor] (119.94,263.62) circle (  1.16);

\path[draw=drawColor,line width= 0.4pt,line join=round,line cap=round,fill=fillColor] (120.77,263.55) circle (  1.16);

\path[draw=drawColor,line width= 0.4pt,line join=round,line cap=round,fill=fillColor] (121.57,263.26) circle (  1.16);

\path[draw=drawColor,line width= 0.4pt,line join=round,line cap=round,fill=fillColor] (122.36,262.96) circle (  1.16);

\path[draw=drawColor,line width= 0.4pt,line join=round,line cap=round,fill=fillColor] (123.13,262.82) circle (  1.16);

\path[draw=drawColor,line width= 0.4pt,line join=round,line cap=round,fill=fillColor] (123.88,262.77) circle (  1.16);

\path[draw=drawColor,line width= 0.4pt,line join=round,line cap=round,fill=fillColor] (124.62,262.53) circle (  1.16);

\path[draw=drawColor,line width= 0.4pt,line join=round,line cap=round,fill=fillColor] (125.34,262.52) circle (  1.16);

\path[draw=drawColor,line width= 0.4pt,line join=round,line cap=round,fill=fillColor] (126.05,262.49) circle (  1.16);

\path[draw=drawColor,line width= 0.4pt,line join=round,line cap=round,fill=fillColor] (126.74,262.21) circle (  1.16);

\path[draw=drawColor,line width= 0.4pt,line join=round,line cap=round,fill=fillColor] (127.43,262.00) circle (  1.16);

\path[draw=drawColor,line width= 0.4pt,line join=round,line cap=round,fill=fillColor] (128.10,261.99) circle (  1.16);

\path[draw=drawColor,line width= 0.4pt,line join=round,line cap=round,fill=fillColor] (128.76,261.81) circle (  1.16);

\path[draw=drawColor,line width= 0.4pt,line join=round,line cap=round,fill=fillColor] (129.40,261.76) circle (  1.16);

\path[draw=drawColor,line width= 0.4pt,line join=round,line cap=round,fill=fillColor] (130.04,261.53) circle (  1.16);

\path[draw=drawColor,line width= 0.4pt,line join=round,line cap=round,fill=fillColor] (130.67,261.46) circle (  1.16);

\path[draw=drawColor,line width= 0.4pt,line join=round,line cap=round,fill=fillColor] (131.28,261.35) circle (  1.16);

\path[draw=drawColor,line width= 0.4pt,line join=round,line cap=round,fill=fillColor] (131.89,261.29) circle (  1.16);

\path[draw=drawColor,line width= 0.4pt,line join=round,line cap=round,fill=fillColor] (132.49,261.20) circle (  1.16);

\path[draw=drawColor,line width= 0.4pt,line join=round,line cap=round,fill=fillColor] (133.08,261.12) circle (  1.16);

\path[draw=drawColor,line width= 0.4pt,line join=round,line cap=round,fill=fillColor] (133.66,261.03) circle (  1.16);

\path[draw=drawColor,line width= 0.4pt,line join=round,line cap=round,fill=fillColor] (134.23,260.73) circle (  1.16);

\path[draw=drawColor,line width= 0.4pt,line join=round,line cap=round,fill=fillColor] (134.80,260.46) circle (  1.16);

\path[draw=drawColor,line width= 0.4pt,line join=round,line cap=round,fill=fillColor] (135.35,260.30) circle (  1.16);

\path[draw=drawColor,line width= 0.4pt,line join=round,line cap=round,fill=fillColor] (135.90,260.25) circle (  1.16);

\path[draw=drawColor,line width= 0.4pt,line join=round,line cap=round,fill=fillColor] (136.45,260.22) circle (  1.16);

\path[draw=drawColor,line width= 0.4pt,line join=round,line cap=round,fill=fillColor] (136.98,260.16) circle (  1.16);

\path[draw=drawColor,line width= 0.4pt,line join=round,line cap=round,fill=fillColor] (137.51,259.98) circle (  1.16);

\path[draw=drawColor,line width= 0.4pt,line join=round,line cap=round,fill=fillColor] (138.03,259.85) circle (  1.16);

\path[draw=drawColor,line width= 0.4pt,line join=round,line cap=round,fill=fillColor] (138.55,259.81) circle (  1.16);

\path[draw=drawColor,line width= 0.4pt,line join=round,line cap=round,fill=fillColor] (139.06,259.66) circle (  1.16);

\path[draw=drawColor,line width= 0.4pt,line join=round,line cap=round,fill=fillColor] (139.56,259.61) circle (  1.16);

\path[draw=drawColor,line width= 0.4pt,line join=round,line cap=round,fill=fillColor] (140.06,259.46) circle (  1.16);

\path[draw=drawColor,line width= 0.4pt,line join=round,line cap=round,fill=fillColor] (140.55,259.37) circle (  1.16);

\path[draw=drawColor,line width= 0.4pt,line join=round,line cap=round,fill=fillColor] (141.04,259.21) circle (  1.16);

\path[draw=drawColor,line width= 0.4pt,line join=round,line cap=round,fill=fillColor] (141.52,259.21) circle (  1.16);

\path[draw=drawColor,line width= 0.4pt,line join=round,line cap=round,fill=fillColor] (142.00,258.74) circle (  1.16);

\path[draw=drawColor,line width= 0.4pt,line join=round,line cap=round,fill=fillColor] (142.47,258.72) circle (  1.16);

\path[draw=drawColor,line width= 0.4pt,line join=round,line cap=round,fill=fillColor] (142.94,258.67) circle (  1.16);

\path[draw=drawColor,line width= 0.4pt,line join=round,line cap=round,fill=fillColor] (143.40,258.63) circle (  1.16);

\path[draw=drawColor,line width= 0.4pt,line join=round,line cap=round,fill=fillColor] (143.86,258.61) circle (  1.16);

\path[draw=drawColor,line width= 0.4pt,line join=round,line cap=round,fill=fillColor] (144.31,258.59) circle (  1.16);

\path[draw=drawColor,line width= 0.4pt,line join=round,line cap=round,fill=fillColor] (144.76,258.59) circle (  1.16);

\path[draw=drawColor,line width= 0.4pt,line join=round,line cap=round,fill=fillColor] (145.20,258.39) circle (  1.16);

\path[draw=drawColor,line width= 0.4pt,line join=round,line cap=round,fill=fillColor] (145.64,258.31) circle (  1.16);

\path[draw=drawColor,line width= 0.4pt,line join=round,line cap=round,fill=fillColor] (146.08,258.15) circle (  1.16);

\path[draw=drawColor,line width= 0.4pt,line join=round,line cap=round,fill=fillColor] (146.51,258.13) circle (  1.16);

\path[draw=drawColor,line width= 0.4pt,line join=round,line cap=round,fill=fillColor] (146.94,258.00) circle (  1.16);

\path[draw=drawColor,line width= 0.4pt,line join=round,line cap=round,fill=fillColor] (147.36,257.94) circle (  1.16);

\path[draw=drawColor,line width= 0.4pt,line join=round,line cap=round,fill=fillColor] (147.79,257.83) circle (  1.16);

\path[draw=drawColor,line width= 0.4pt,line join=round,line cap=round,fill=fillColor] (148.20,257.80) circle (  1.16);

\path[draw=drawColor,line width= 0.4pt,line join=round,line cap=round,fill=fillColor] (148.62,257.70) circle (  1.16);

\path[draw=drawColor,line width= 0.4pt,line join=round,line cap=round,fill=fillColor] (149.03,257.65) circle (  1.16);

\path[draw=drawColor,line width= 0.4pt,line join=round,line cap=round,fill=fillColor] (149.43,257.64) circle (  1.16);

\path[draw=drawColor,line width= 0.4pt,line join=round,line cap=round,fill=fillColor] (149.83,257.61) circle (  1.16);

\path[draw=drawColor,line width= 0.4pt,line join=round,line cap=round,fill=fillColor] (150.23,257.58) circle (  1.16);

\path[draw=drawColor,line width= 0.4pt,line join=round,line cap=round,fill=fillColor] (150.63,257.58) circle (  1.16);

\path[draw=drawColor,line width= 0.4pt,line join=round,line cap=round,fill=fillColor] (151.02,257.58) circle (  1.16);

\path[draw=drawColor,line width= 0.4pt,line join=round,line cap=round,fill=fillColor] (151.41,257.21) circle (  1.16);

\path[draw=drawColor,line width= 0.4pt,line join=round,line cap=round,fill=fillColor] (151.80,257.17) circle (  1.16);

\path[draw=drawColor,line width= 0.4pt,line join=round,line cap=round,fill=fillColor] (152.18,257.16) circle (  1.16);

\path[draw=drawColor,line width= 0.4pt,line join=round,line cap=round,fill=fillColor] (152.57,257.05) circle (  1.16);

\path[draw=drawColor,line width= 0.4pt,line join=round,line cap=round,fill=fillColor] (152.94,257.01) circle (  1.16);

\path[draw=drawColor,line width= 0.4pt,line join=round,line cap=round,fill=fillColor] (153.32,256.97) circle (  1.16);

\path[draw=drawColor,line width= 0.4pt,line join=round,line cap=round,fill=fillColor] (153.69,256.89) circle (  1.16);

\path[draw=drawColor,line width= 0.4pt,line join=round,line cap=round,fill=fillColor] (154.06,256.73) circle (  1.16);

\path[draw=drawColor,line width= 0.4pt,line join=round,line cap=round,fill=fillColor] (154.43,256.66) circle (  1.16);

\path[draw=drawColor,line width= 0.4pt,line join=round,line cap=round,fill=fillColor] (154.79,256.59) circle (  1.16);

\path[draw=drawColor,line width= 0.4pt,line join=round,line cap=round,fill=fillColor] (155.15,256.57) circle (  1.16);

\path[draw=drawColor,line width= 0.4pt,line join=round,line cap=round,fill=fillColor] (155.51,256.51) circle (  1.16);

\path[draw=drawColor,line width= 0.4pt,line join=round,line cap=round,fill=fillColor] (155.87,256.49) circle (  1.16);

\path[draw=drawColor,line width= 0.4pt,line join=round,line cap=round,fill=fillColor] (156.23,256.48) circle (  1.16);

\path[draw=drawColor,line width= 0.4pt,line join=round,line cap=round,fill=fillColor] (156.58,256.31) circle (  1.16);

\path[draw=drawColor,line width= 0.4pt,line join=round,line cap=round,fill=fillColor] (156.93,256.15) circle (  1.16);

\path[draw=drawColor,line width= 0.4pt,line join=round,line cap=round,fill=fillColor] (157.28,256.12) circle (  1.16);

\path[draw=drawColor,line width= 0.4pt,line join=round,line cap=round,fill=fillColor] (157.62,256.06) circle (  1.16);

\path[draw=drawColor,line width= 0.4pt,line join=round,line cap=round,fill=fillColor] (157.96,255.77) circle (  1.16);

\path[draw=drawColor,line width= 0.4pt,line join=round,line cap=round,fill=fillColor] (158.30,255.73) circle (  1.16);

\path[draw=drawColor,line width= 0.4pt,line join=round,line cap=round,fill=fillColor] (158.64,255.69) circle (  1.16);

\path[draw=drawColor,line width= 0.4pt,line join=round,line cap=round,fill=fillColor] (158.98,255.62) circle (  1.16);

\path[draw=drawColor,line width= 0.4pt,line join=round,line cap=round,fill=fillColor] (159.31,255.57) circle (  1.16);

\path[draw=drawColor,line width= 0.4pt,line join=round,line cap=round,fill=fillColor] (159.65,255.55) circle (  1.16);

\path[draw=drawColor,line width= 0.4pt,line join=round,line cap=round,fill=fillColor] (159.98,255.45) circle (  1.16);

\path[draw=drawColor,line width= 0.4pt,line join=round,line cap=round,fill=fillColor] (160.30,255.40) circle (  1.16);

\path[draw=drawColor,line width= 0.4pt,line join=round,line cap=round,fill=fillColor] (160.63,255.31) circle (  1.16);

\path[draw=drawColor,line width= 0.4pt,line join=round,line cap=round,fill=fillColor] (160.95,255.21) circle (  1.16);

\path[draw=drawColor,line width= 0.4pt,line join=round,line cap=round,fill=fillColor] (161.28,255.13) circle (  1.16);

\path[draw=drawColor,line width= 0.4pt,line join=round,line cap=round,fill=fillColor] (161.60,255.11) circle (  1.16);

\path[draw=drawColor,line width= 0.4pt,line join=round,line cap=round,fill=fillColor] (161.92,254.96) circle (  1.16);

\path[draw=drawColor,line width= 0.4pt,line join=round,line cap=round,fill=fillColor] (162.23,254.89) circle (  1.16);

\path[draw=drawColor,line width= 0.4pt,line join=round,line cap=round,fill=fillColor] (162.55,254.84) circle (  1.16);

\path[draw=drawColor,line width= 0.4pt,line join=round,line cap=round,fill=fillColor] (162.86,254.81) circle (  1.16);

\path[draw=drawColor,line width= 0.4pt,line join=round,line cap=round,fill=fillColor] (163.17,254.64) circle (  1.16);

\path[draw=drawColor,line width= 0.4pt,line join=round,line cap=round,fill=fillColor] (163.48,254.63) circle (  1.16);

\path[draw=drawColor,line width= 0.4pt,line join=round,line cap=round,fill=fillColor] (163.79,254.57) circle (  1.16);

\path[draw=drawColor,line width= 0.4pt,line join=round,line cap=round,fill=fillColor] (164.09,254.52) circle (  1.16);

\path[draw=drawColor,line width= 0.4pt,line join=round,line cap=round,fill=fillColor] (164.40,254.49) circle (  1.16);

\path[draw=drawColor,line width= 0.4pt,line join=round,line cap=round,fill=fillColor] (164.70,254.49) circle (  1.16);

\path[draw=drawColor,line width= 0.4pt,line join=round,line cap=round,fill=fillColor] (165.00,254.45) circle (  1.16);

\path[draw=drawColor,line width= 0.4pt,line join=round,line cap=round,fill=fillColor] (165.30,254.36) circle (  1.16);

\path[draw=drawColor,line width= 0.4pt,line join=round,line cap=round,fill=fillColor] (165.60,254.16) circle (  1.16);

\path[draw=drawColor,line width= 0.4pt,line join=round,line cap=round,fill=fillColor] (165.90,254.16) circle (  1.16);

\path[draw=drawColor,line width= 0.4pt,line join=round,line cap=round,fill=fillColor] (166.19,254.08) circle (  1.16);

\path[draw=drawColor,line width= 0.4pt,line join=round,line cap=round,fill=fillColor] (166.49,253.96) circle (  1.16);

\path[draw=drawColor,line width= 0.4pt,line join=round,line cap=round,fill=fillColor] (166.78,253.92) circle (  1.16);

\path[draw=drawColor,line width= 0.4pt,line join=round,line cap=round,fill=fillColor] (167.07,253.90) circle (  1.16);

\path[draw=drawColor,line width= 0.4pt,line join=round,line cap=round,fill=fillColor] (167.36,253.87) circle (  1.16);

\path[draw=drawColor,line width= 0.4pt,line join=round,line cap=round,fill=fillColor] (167.64,253.86) circle (  1.16);

\path[draw=drawColor,line width= 0.4pt,line join=round,line cap=round,fill=fillColor] (167.93,253.82) circle (  1.16);

\path[draw=drawColor,line width= 0.4pt,line join=round,line cap=round,fill=fillColor] (168.22,253.81) circle (  1.16);

\path[draw=drawColor,line width= 0.4pt,line join=round,line cap=round,fill=fillColor] (168.50,253.79) circle (  1.16);

\path[draw=drawColor,line width= 0.4pt,line join=round,line cap=round,fill=fillColor] (168.78,253.53) circle (  1.16);

\path[draw=drawColor,line width= 0.4pt,line join=round,line cap=round,fill=fillColor] (169.06,253.39) circle (  1.16);

\path[draw=drawColor,line width= 0.4pt,line join=round,line cap=round,fill=fillColor] (169.34,253.29) circle (  1.16);

\path[draw=drawColor,line width= 0.4pt,line join=round,line cap=round,fill=fillColor] (169.62,253.04) circle (  1.16);

\path[draw=drawColor,line width= 0.4pt,line join=round,line cap=round,fill=fillColor] (169.89,252.92) circle (  1.16);

\path[draw=drawColor,line width= 0.4pt,line join=round,line cap=round,fill=fillColor] (170.17,252.78) circle (  1.16);

\path[draw=drawColor,line width= 0.4pt,line join=round,line cap=round,fill=fillColor] (170.44,252.77) circle (  1.16);

\path[draw=drawColor,line width= 0.4pt,line join=round,line cap=round,fill=fillColor] (170.72,252.70) circle (  1.16);

\path[draw=drawColor,line width= 0.4pt,line join=round,line cap=round,fill=fillColor] (170.99,252.63) circle (  1.16);

\path[draw=drawColor,line width= 0.4pt,line join=round,line cap=round,fill=fillColor] (171.26,252.62) circle (  1.16);

\path[draw=drawColor,line width= 0.4pt,line join=round,line cap=round,fill=fillColor] (171.53,252.48) circle (  1.16);

\path[draw=drawColor,line width= 0.4pt,line join=round,line cap=round,fill=fillColor] (171.80,252.33) circle (  1.16);

\path[draw=drawColor,line width= 0.4pt,line join=round,line cap=round,fill=fillColor] (172.06,252.13) circle (  1.16);

\path[draw=drawColor,line width= 0.4pt,line join=round,line cap=round,fill=fillColor] (172.33,252.12) circle (  1.16);

\path[draw=drawColor,line width= 0.4pt,line join=round,line cap=round,fill=fillColor] (172.59,252.04) circle (  1.16);

\path[draw=drawColor,line width= 0.4pt,line join=round,line cap=round,fill=fillColor] (172.85,251.94) circle (  1.16);

\path[draw=drawColor,line width= 0.4pt,line join=round,line cap=round,fill=fillColor] (173.12,251.89) circle (  1.16);

\path[draw=drawColor,line width= 0.4pt,line join=round,line cap=round,fill=fillColor] (173.38,251.87) circle (  1.16);

\path[draw=drawColor,line width= 0.4pt,line join=round,line cap=round,fill=fillColor] (173.64,251.79) circle (  1.16);

\path[draw=drawColor,line width= 0.4pt,line join=round,line cap=round,fill=fillColor] (173.90,251.77) circle (  1.16);

\path[draw=drawColor,line width= 0.4pt,line join=round,line cap=round,fill=fillColor] (174.15,251.66) circle (  1.16);

\path[draw=drawColor,line width= 0.4pt,line join=round,line cap=round,fill=fillColor] (174.41,251.66) circle (  1.16);

\path[draw=drawColor,line width= 0.4pt,line join=round,line cap=round,fill=fillColor] (174.67,251.61) circle (  1.16);

\path[draw=drawColor,line width= 0.4pt,line join=round,line cap=round,fill=fillColor] (174.92,251.57) circle (  1.16);

\path[draw=drawColor,line width= 0.4pt,line join=round,line cap=round,fill=fillColor] (175.17,251.55) circle (  1.16);

\path[draw=drawColor,line width= 0.4pt,line join=round,line cap=round,fill=fillColor] (175.42,251.54) circle (  1.16);

\path[draw=drawColor,line width= 0.4pt,line join=round,line cap=round,fill=fillColor] (175.68,251.49) circle (  1.16);

\path[draw=drawColor,line width= 0.4pt,line join=round,line cap=round,fill=fillColor] (175.93,251.44) circle (  1.16);

\path[draw=drawColor,line width= 0.4pt,line join=round,line cap=round,fill=fillColor] (176.18,251.37) circle (  1.16);

\path[draw=drawColor,line width= 0.4pt,line join=round,line cap=round,fill=fillColor] (176.42,251.10) circle (  1.16);

\path[draw=drawColor,line width= 0.4pt,line join=round,line cap=round,fill=fillColor] (176.67,251.04) circle (  1.16);

\path[draw=drawColor,line width= 0.4pt,line join=round,line cap=round,fill=fillColor] (176.92,251.00) circle (  1.16);

\path[draw=drawColor,line width= 0.4pt,line join=round,line cap=round,fill=fillColor] (177.16,250.96) circle (  1.16);

\path[draw=drawColor,line width= 0.4pt,line join=round,line cap=round,fill=fillColor] (177.41,250.96) circle (  1.16);

\path[draw=drawColor,line width= 0.4pt,line join=round,line cap=round,fill=fillColor] (177.65,250.91) circle (  1.16);

\path[draw=drawColor,line width= 0.4pt,line join=round,line cap=round,fill=fillColor] (177.89,250.89) circle (  1.16);

\path[draw=drawColor,line width= 0.4pt,line join=round,line cap=round,fill=fillColor] (178.13,250.83) circle (  1.16);

\path[draw=drawColor,line width= 0.4pt,line join=round,line cap=round,fill=fillColor] (178.37,250.80) circle (  1.16);

\path[draw=drawColor,line width= 0.4pt,line join=round,line cap=round,fill=fillColor] (178.61,250.65) circle (  1.16);

\path[draw=drawColor,line width= 0.4pt,line join=round,line cap=round,fill=fillColor] (178.85,250.65) circle (  1.16);

\path[draw=drawColor,line width= 0.4pt,line join=round,line cap=round,fill=fillColor] (179.09,250.59) circle (  1.16);

\path[draw=drawColor,line width= 0.4pt,line join=round,line cap=round,fill=fillColor] (179.33,250.42) circle (  1.16);

\path[draw=drawColor,line width= 0.4pt,line join=round,line cap=round,fill=fillColor] (179.56,250.40) circle (  1.16);

\path[draw=drawColor,line width= 0.4pt,line join=round,line cap=round,fill=fillColor] (179.80,250.40) circle (  1.16);

\path[draw=drawColor,line width= 0.4pt,line join=round,line cap=round,fill=fillColor] (180.03,250.39) circle (  1.16);

\path[draw=drawColor,line width= 0.4pt,line join=round,line cap=round,fill=fillColor] (180.27,250.35) circle (  1.16);

\path[draw=drawColor,line width= 0.4pt,line join=round,line cap=round,fill=fillColor] (180.50,250.31) circle (  1.16);

\path[draw=drawColor,line width= 0.4pt,line join=round,line cap=round,fill=fillColor] (180.73,250.27) circle (  1.16);

\path[draw=drawColor,line width= 0.4pt,line join=round,line cap=round,fill=fillColor] (180.96,250.24) circle (  1.16);

\path[draw=drawColor,line width= 0.4pt,line join=round,line cap=round,fill=fillColor] (181.19,250.22) circle (  1.16);

\path[draw=drawColor,line width= 0.4pt,line join=round,line cap=round,fill=fillColor] (181.42,250.17) circle (  1.16);

\path[draw=drawColor,line width= 0.4pt,line join=round,line cap=round,fill=fillColor] (181.65,250.07) circle (  1.16);

\path[draw=drawColor,line width= 0.4pt,line join=round,line cap=round,fill=fillColor] (181.88,250.05) circle (  1.16);

\path[draw=drawColor,line width= 0.4pt,line join=round,line cap=round,fill=fillColor] (182.10,250.00) circle (  1.16);

\path[draw=drawColor,line width= 0.4pt,line join=round,line cap=round,fill=fillColor] (182.33,249.99) circle (  1.16);

\path[draw=drawColor,line width= 0.4pt,line join=round,line cap=round,fill=fillColor] (182.55,249.96) circle (  1.16);

\path[draw=drawColor,line width= 0.4pt,line join=round,line cap=round,fill=fillColor] (182.78,249.86) circle (  1.16);

\path[draw=drawColor,line width= 0.4pt,line join=round,line cap=round,fill=fillColor] (183.00,249.85) circle (  1.16);

\path[draw=drawColor,line width= 0.4pt,line join=round,line cap=round,fill=fillColor] (183.23,249.84) circle (  1.16);

\path[draw=drawColor,line width= 0.4pt,line join=round,line cap=round,fill=fillColor] (183.45,249.80) circle (  1.16);

\path[draw=drawColor,line width= 0.4pt,line join=round,line cap=round,fill=fillColor] (183.67,249.78) circle (  1.16);

\path[draw=drawColor,line width= 0.4pt,line join=round,line cap=round,fill=fillColor] (183.89,249.75) circle (  1.16);

\path[draw=drawColor,line width= 0.4pt,line join=round,line cap=round,fill=fillColor] (184.11,249.73) circle (  1.16);

\path[draw=drawColor,line width= 0.4pt,line join=round,line cap=round,fill=fillColor] (184.33,249.73) circle (  1.16);

\path[draw=drawColor,line width= 0.4pt,line join=round,line cap=round,fill=fillColor] (184.55,249.59) circle (  1.16);

\path[draw=drawColor,line width= 0.4pt,line join=round,line cap=round,fill=fillColor] (184.77,249.53) circle (  1.16);

\path[draw=drawColor,line width= 0.4pt,line join=round,line cap=round,fill=fillColor] (184.98,249.50) circle (  1.16);

\path[draw=drawColor,line width= 0.4pt,line join=round,line cap=round,fill=fillColor] (185.20,249.44) circle (  1.16);

\path[draw=drawColor,line width= 0.4pt,line join=round,line cap=round,fill=fillColor] (185.41,249.43) circle (  1.16);

\path[draw=drawColor,line width= 0.4pt,line join=round,line cap=round,fill=fillColor] (185.63,249.28) circle (  1.16);

\path[draw=drawColor,line width= 0.4pt,line join=round,line cap=round,fill=fillColor] (185.84,249.26) circle (  1.16);

\path[draw=drawColor,line width= 0.4pt,line join=round,line cap=round,fill=fillColor] (186.06,249.23) circle (  1.16);

\path[draw=drawColor,line width= 0.4pt,line join=round,line cap=round,fill=fillColor] (186.27,249.23) circle (  1.16);

\path[draw=drawColor,line width= 0.4pt,line join=round,line cap=round,fill=fillColor] (186.48,249.17) circle (  1.16);

\path[draw=drawColor,line width= 0.4pt,line join=round,line cap=round,fill=fillColor] (186.69,249.14) circle (  1.16);

\path[draw=drawColor,line width= 0.4pt,line join=round,line cap=round,fill=fillColor] (186.91,249.13) circle (  1.16);

\path[draw=drawColor,line width= 0.4pt,line join=round,line cap=round,fill=fillColor] (187.12,249.11) circle (  1.16);

\path[draw=drawColor,line width= 0.4pt,line join=round,line cap=round,fill=fillColor] (187.33,249.10) circle (  1.16);

\path[draw=drawColor,line width= 0.4pt,line join=round,line cap=round,fill=fillColor] (187.53,249.05) circle (  1.16);

\path[draw=drawColor,line width= 0.4pt,line join=round,line cap=round,fill=fillColor] (187.74,249.04) circle (  1.16);

\path[draw=drawColor,line width= 0.4pt,line join=round,line cap=round,fill=fillColor] (187.95,248.98) circle (  1.16);

\path[draw=drawColor,line width= 0.4pt,line join=round,line cap=round,fill=fillColor] (188.16,248.97) circle (  1.16);

\path[draw=drawColor,line width= 0.4pt,line join=round,line cap=round,fill=fillColor] (188.36,248.94) circle (  1.16);

\path[draw=drawColor,line width= 0.4pt,line join=round,line cap=round,fill=fillColor] (188.57,248.92) circle (  1.16);

\path[draw=drawColor,line width= 0.4pt,line join=round,line cap=round,fill=fillColor] (188.77,248.88) circle (  1.16);

\path[draw=drawColor,line width= 0.4pt,line join=round,line cap=round,fill=fillColor] (188.98,248.83) circle (  1.16);

\path[draw=drawColor,line width= 0.4pt,line join=round,line cap=round,fill=fillColor] (189.18,248.82) circle (  1.16);

\path[draw=drawColor,line width= 0.4pt,line join=round,line cap=round,fill=fillColor] (189.39,248.82) circle (  1.16);

\path[draw=drawColor,line width= 0.4pt,line join=round,line cap=round,fill=fillColor] (189.59,248.70) circle (  1.16);

\path[draw=drawColor,line width= 0.4pt,line join=round,line cap=round,fill=fillColor] (189.79,248.65) circle (  1.16);

\path[draw=drawColor,line width= 0.4pt,line join=round,line cap=round,fill=fillColor] (189.99,248.58) circle (  1.16);

\path[draw=drawColor,line width= 0.4pt,line join=round,line cap=round,fill=fillColor] (190.19,248.54) circle (  1.16);

\path[draw=drawColor,line width= 0.4pt,line join=round,line cap=round,fill=fillColor] (190.39,248.53) circle (  1.16);

\path[draw=drawColor,line width= 0.4pt,line join=round,line cap=round,fill=fillColor] (190.59,248.48) circle (  1.16);

\path[draw=drawColor,line width= 0.4pt,line join=round,line cap=round,fill=fillColor] (190.79,248.46) circle (  1.16);

\path[draw=drawColor,line width= 0.4pt,line join=round,line cap=round,fill=fillColor] (190.99,248.40) circle (  1.16);

\path[draw=drawColor,line width= 0.4pt,line join=round,line cap=round,fill=fillColor] (191.19,248.33) circle (  1.16);

\path[draw=drawColor,line width= 0.4pt,line join=round,line cap=round,fill=fillColor] (191.39,248.33) circle (  1.16);

\path[draw=drawColor,line width= 0.4pt,line join=round,line cap=round,fill=fillColor] (191.58,248.29) circle (  1.16);

\path[draw=drawColor,line width= 0.4pt,line join=round,line cap=round,fill=fillColor] (191.78,248.29) circle (  1.16);

\path[draw=drawColor,line width= 0.4pt,line join=round,line cap=round,fill=fillColor] (191.98,248.26) circle (  1.16);

\path[draw=drawColor,line width= 0.4pt,line join=round,line cap=round,fill=fillColor] (192.17,248.22) circle (  1.16);

\path[draw=drawColor,line width= 0.4pt,line join=round,line cap=round,fill=fillColor] (192.37,248.20) circle (  1.16);

\path[draw=drawColor,line width= 0.4pt,line join=round,line cap=round,fill=fillColor] (192.56,248.20) circle (  1.16);

\path[draw=drawColor,line width= 0.4pt,line join=round,line cap=round,fill=fillColor] (192.75,248.18) circle (  1.16);

\path[draw=drawColor,line width= 0.4pt,line join=round,line cap=round,fill=fillColor] (192.95,248.16) circle (  1.16);

\path[draw=drawColor,line width= 0.4pt,line join=round,line cap=round,fill=fillColor] (193.14,248.16) circle (  1.16);

\path[draw=drawColor,line width= 0.4pt,line join=round,line cap=round,fill=fillColor] (193.33,248.10) circle (  1.16);

\path[draw=drawColor,line width= 0.4pt,line join=round,line cap=round,fill=fillColor] (193.52,248.02) circle (  1.16);

\path[draw=drawColor,line width= 0.4pt,line join=round,line cap=round,fill=fillColor] (193.71,248.01) circle (  1.16);

\path[draw=drawColor,line width= 0.4pt,line join=round,line cap=round,fill=fillColor] (193.90,247.98) circle (  1.16);

\path[draw=drawColor,line width= 0.4pt,line join=round,line cap=round,fill=fillColor] (194.09,247.85) circle (  1.16);

\path[draw=drawColor,line width= 0.4pt,line join=round,line cap=round,fill=fillColor] (194.28,247.82) circle (  1.16);

\path[draw=drawColor,line width= 0.4pt,line join=round,line cap=round,fill=fillColor] (194.47,247.79) circle (  1.16);

\path[draw=drawColor,line width= 0.4pt,line join=round,line cap=round,fill=fillColor] (194.66,247.73) circle (  1.16);

\path[draw=drawColor,line width= 0.4pt,line join=round,line cap=round,fill=fillColor] (194.85,247.72) circle (  1.16);

\path[draw=drawColor,line width= 0.4pt,line join=round,line cap=round,fill=fillColor] (195.04,247.61) circle (  1.16);

\path[draw=drawColor,line width= 0.4pt,line join=round,line cap=round,fill=fillColor] (195.22,247.61) circle (  1.16);

\path[draw=drawColor,line width= 0.4pt,line join=round,line cap=round,fill=fillColor] (195.41,247.61) circle (  1.16);

\path[draw=drawColor,line width= 0.4pt,line join=round,line cap=round,fill=fillColor] (195.60,247.58) circle (  1.16);

\path[draw=drawColor,line width= 0.4pt,line join=round,line cap=round,fill=fillColor] (195.78,247.54) circle (  1.16);

\path[draw=drawColor,line width= 0.4pt,line join=round,line cap=round,fill=fillColor] (195.97,247.53) circle (  1.16);

\path[draw=drawColor,line width= 0.4pt,line join=round,line cap=round,fill=fillColor] (196.15,247.48) circle (  1.16);

\path[draw=drawColor,line width= 0.4pt,line join=round,line cap=round,fill=fillColor] (196.34,247.40) circle (  1.16);

\path[draw=drawColor,line width= 0.4pt,line join=round,line cap=round,fill=fillColor] (196.52,247.37) circle (  1.16);

\path[draw=drawColor,line width= 0.4pt,line join=round,line cap=round,fill=fillColor] (196.70,247.32) circle (  1.16);

\path[draw=drawColor,line width= 0.4pt,line join=round,line cap=round,fill=fillColor] (196.89,247.28) circle (  1.16);

\path[draw=drawColor,line width= 0.4pt,line join=round,line cap=round,fill=fillColor] (197.07,247.24) circle (  1.16);

\path[draw=drawColor,line width= 0.4pt,line join=round,line cap=round,fill=fillColor] (197.25,247.17) circle (  1.16);

\path[draw=drawColor,line width= 0.4pt,line join=round,line cap=round,fill=fillColor] (197.43,247.13) circle (  1.16);

\path[draw=drawColor,line width= 0.4pt,line join=round,line cap=round,fill=fillColor] (197.61,247.03) circle (  1.16);

\path[draw=drawColor,line width= 0.4pt,line join=round,line cap=round,fill=fillColor] (197.79,246.99) circle (  1.16);

\path[draw=drawColor,line width= 0.4pt,line join=round,line cap=round,fill=fillColor] (197.97,246.94) circle (  1.16);

\path[draw=drawColor,line width= 0.4pt,line join=round,line cap=round,fill=fillColor] (198.15,246.91) circle (  1.16);

\path[draw=drawColor,line width= 0.4pt,line join=round,line cap=round,fill=fillColor] (198.33,246.88) circle (  1.16);

\path[draw=drawColor,line width= 0.4pt,line join=round,line cap=round,fill=fillColor] (198.51,246.83) circle (  1.16);

\path[draw=drawColor,line width= 0.4pt,line join=round,line cap=round,fill=fillColor] (198.69,246.76) circle (  1.16);

\path[draw=drawColor,line width= 0.4pt,line join=round,line cap=round,fill=fillColor] (198.87,246.75) circle (  1.16);

\path[draw=drawColor,line width= 0.4pt,line join=round,line cap=round,fill=fillColor] (199.04,246.70) circle (  1.16);

\path[draw=drawColor,line width= 0.4pt,line join=round,line cap=round,fill=fillColor] (199.22,246.66) circle (  1.16);

\path[draw=drawColor,line width= 0.4pt,line join=round,line cap=round,fill=fillColor] (199.40,246.63) circle (  1.16);

\path[draw=drawColor,line width= 0.4pt,line join=round,line cap=round,fill=fillColor] (199.57,246.62) circle (  1.16);

\path[draw=drawColor,line width= 0.4pt,line join=round,line cap=round,fill=fillColor] (199.75,246.58) circle (  1.16);

\path[draw=drawColor,line width= 0.4pt,line join=round,line cap=round,fill=fillColor] (199.92,246.58) circle (  1.16);

\path[draw=drawColor,line width= 0.4pt,line join=round,line cap=round,fill=fillColor] (200.10,246.57) circle (  1.16);

\path[draw=drawColor,line width= 0.4pt,line join=round,line cap=round,fill=fillColor] (200.27,246.47) circle (  1.16);

\path[draw=drawColor,line width= 0.4pt,line join=round,line cap=round,fill=fillColor] (200.45,246.45) circle (  1.16);

\path[draw=drawColor,line width= 0.4pt,line join=round,line cap=round,fill=fillColor] (200.62,246.38) circle (  1.16);

\path[draw=drawColor,line width= 0.4pt,line join=round,line cap=round,fill=fillColor] (200.79,246.36) circle (  1.16);

\path[draw=drawColor,line width= 0.4pt,line join=round,line cap=round,fill=fillColor] (200.97,246.34) circle (  1.16);

\path[draw=drawColor,line width= 0.4pt,line join=round,line cap=round,fill=fillColor] (201.14,246.33) circle (  1.16);

\path[draw=drawColor,line width= 0.4pt,line join=round,line cap=round,fill=fillColor] (201.31,246.20) circle (  1.16);

\path[draw=drawColor,line width= 0.4pt,line join=round,line cap=round,fill=fillColor] (201.48,246.18) circle (  1.16);

\path[draw=drawColor,line width= 0.4pt,line join=round,line cap=round,fill=fillColor] (201.65,246.16) circle (  1.16);

\path[draw=drawColor,line width= 0.4pt,line join=round,line cap=round,fill=fillColor] (201.83,246.16) circle (  1.16);

\path[draw=drawColor,line width= 0.4pt,line join=round,line cap=round,fill=fillColor] (202.00,246.09) circle (  1.16);

\path[draw=drawColor,line width= 0.4pt,line join=round,line cap=round,fill=fillColor] (202.17,246.09) circle (  1.16);

\path[draw=drawColor,line width= 0.4pt,line join=round,line cap=round,fill=fillColor] (202.34,245.98) circle (  1.16);

\path[draw=drawColor,line width= 0.4pt,line join=round,line cap=round,fill=fillColor] (202.50,245.95) circle (  1.16);

\path[draw=drawColor,line width= 0.4pt,line join=round,line cap=round,fill=fillColor] (202.67,245.86) circle (  1.16);

\path[draw=drawColor,line width= 0.4pt,line join=round,line cap=round,fill=fillColor] (202.84,245.86) circle (  1.16);

\path[draw=drawColor,line width= 0.4pt,line join=round,line cap=round,fill=fillColor] (203.01,245.85) circle (  1.16);

\path[draw=drawColor,line width= 0.4pt,line join=round,line cap=round,fill=fillColor] (203.18,245.81) circle (  1.16);

\path[draw=drawColor,line width= 0.4pt,line join=round,line cap=round,fill=fillColor] (203.35,245.80) circle (  1.16);

\path[draw=drawColor,line width= 0.4pt,line join=round,line cap=round,fill=fillColor] (203.51,245.78) circle (  1.16);

\path[draw=drawColor,line width= 0.4pt,line join=round,line cap=round,fill=fillColor] (203.68,245.53) circle (  1.16);

\path[draw=drawColor,line width= 0.4pt,line join=round,line cap=round,fill=fillColor] (203.85,245.47) circle (  1.16);

\path[draw=drawColor,line width= 0.4pt,line join=round,line cap=round,fill=fillColor] (204.01,245.46) circle (  1.16);

\path[draw=drawColor,line width= 0.4pt,line join=round,line cap=round,fill=fillColor] (204.18,245.44) circle (  1.16);

\path[draw=drawColor,line width= 0.4pt,line join=round,line cap=round,fill=fillColor] (204.34,245.41) circle (  1.16);

\path[draw=drawColor,line width= 0.4pt,line join=round,line cap=round,fill=fillColor] (204.51,245.40) circle (  1.16);

\path[draw=drawColor,line width= 0.4pt,line join=round,line cap=round,fill=fillColor] (204.67,245.37) circle (  1.16);

\path[draw=drawColor,line width= 0.4pt,line join=round,line cap=round,fill=fillColor] (204.84,245.35) circle (  1.16);

\path[draw=drawColor,line width= 0.4pt,line join=round,line cap=round,fill=fillColor] (205.00,245.28) circle (  1.16);

\path[draw=drawColor,line width= 0.4pt,line join=round,line cap=round,fill=fillColor] (205.16,245.27) circle (  1.16);

\path[draw=drawColor,line width= 0.4pt,line join=round,line cap=round,fill=fillColor] (205.33,245.20) circle (  1.16);

\path[draw=drawColor,line width= 0.4pt,line join=round,line cap=round,fill=fillColor] (205.49,245.17) circle (  1.16);

\path[draw=drawColor,line width= 0.4pt,line join=round,line cap=round,fill=fillColor] (205.65,245.13) circle (  1.16);

\path[draw=drawColor,line width= 0.4pt,line join=round,line cap=round,fill=fillColor] (205.81,245.08) circle (  1.16);

\path[draw=drawColor,line width= 0.4pt,line join=round,line cap=round,fill=fillColor] (205.97,245.01) circle (  1.16);

\path[draw=drawColor,line width= 0.4pt,line join=round,line cap=round,fill=fillColor] (206.14,245.00) circle (  1.16);

\path[draw=drawColor,line width= 0.4pt,line join=round,line cap=round,fill=fillColor] (206.30,244.99) circle (  1.16);

\path[draw=drawColor,line width= 0.4pt,line join=round,line cap=round,fill=fillColor] (206.46,244.98) circle (  1.16);

\path[draw=drawColor,line width= 0.4pt,line join=round,line cap=round,fill=fillColor] (206.62,244.95) circle (  1.16);

\path[draw=drawColor,line width= 0.4pt,line join=round,line cap=round,fill=fillColor] (206.78,244.89) circle (  1.16);

\path[draw=drawColor,line width= 0.4pt,line join=round,line cap=round,fill=fillColor] (206.94,244.79) circle (  1.16);

\path[draw=drawColor,line width= 0.4pt,line join=round,line cap=round,fill=fillColor] (207.10,244.77) circle (  1.16);

\path[draw=drawColor,line width= 0.4pt,line join=round,line cap=round,fill=fillColor] (207.26,244.76) circle (  1.16);

\path[draw=drawColor,line width= 0.4pt,line join=round,line cap=round,fill=fillColor] (207.42,244.70) circle (  1.16);

\path[draw=drawColor,line width= 0.4pt,line join=round,line cap=round,fill=fillColor] (207.57,244.60) circle (  1.16);

\path[draw=drawColor,line width= 0.4pt,line join=round,line cap=round,fill=fillColor] (207.73,244.57) circle (  1.16);

\path[draw=drawColor,line width= 0.4pt,line join=round,line cap=round,fill=fillColor] (207.89,244.55) circle (  1.16);

\path[draw=drawColor,line width= 0.4pt,line join=round,line cap=round,fill=fillColor] (208.05,244.54) circle (  1.16);

\path[draw=drawColor,line width= 0.4pt,line join=round,line cap=round,fill=fillColor] (208.20,244.50) circle (  1.16);

\path[draw=drawColor,line width= 0.4pt,line join=round,line cap=round,fill=fillColor] (208.36,244.22) circle (  1.16);

\path[draw=drawColor,line width= 0.4pt,line join=round,line cap=round,fill=fillColor] (208.52,244.20) circle (  1.16);

\path[draw=drawColor,line width= 0.4pt,line join=round,line cap=round,fill=fillColor] (208.67,244.02) circle (  1.16);

\path[draw=drawColor,line width= 0.4pt,line join=round,line cap=round,fill=fillColor] (208.83,244.01) circle (  1.16);

\path[draw=drawColor,line width= 0.4pt,line join=round,line cap=round,fill=fillColor] (208.98,243.91) circle (  1.16);

\path[draw=drawColor,line width= 0.4pt,line join=round,line cap=round,fill=fillColor] (209.14,243.79) circle (  1.16);

\path[draw=drawColor,line width= 0.4pt,line join=round,line cap=round,fill=fillColor] (209.29,243.77) circle (  1.16);

\path[draw=drawColor,line width= 0.4pt,line join=round,line cap=round,fill=fillColor] (209.45,243.77) circle (  1.16);

\path[draw=drawColor,line width= 0.4pt,line join=round,line cap=round,fill=fillColor] (209.60,243.68) circle (  1.16);

\path[draw=drawColor,line width= 0.4pt,line join=round,line cap=round,fill=fillColor] (209.76,243.66) circle (  1.16);

\path[draw=drawColor,line width= 0.4pt,line join=round,line cap=round,fill=fillColor] (209.91,243.63) circle (  1.16);

\path[draw=drawColor,line width= 0.4pt,line join=round,line cap=round,fill=fillColor] (210.07,243.52) circle (  1.16);

\path[draw=drawColor,line width= 0.4pt,line join=round,line cap=round,fill=fillColor] (210.22,243.47) circle (  1.16);

\path[draw=drawColor,line width= 0.4pt,line join=round,line cap=round,fill=fillColor] (210.37,243.36) circle (  1.16);

\path[draw=drawColor,line width= 0.4pt,line join=round,line cap=round,fill=fillColor] (210.52,243.31) circle (  1.16);

\path[draw=drawColor,line width= 0.4pt,line join=round,line cap=round,fill=fillColor] (210.68,243.30) circle (  1.16);

\path[draw=drawColor,line width= 0.4pt,line join=round,line cap=round,fill=fillColor] (210.83,243.28) circle (  1.16);

\path[draw=drawColor,line width= 0.4pt,line join=round,line cap=round,fill=fillColor] (210.98,243.24) circle (  1.16);

\path[draw=drawColor,line width= 0.4pt,line join=round,line cap=round,fill=fillColor] (211.13,243.20) circle (  1.16);

\path[draw=drawColor,line width= 0.4pt,line join=round,line cap=round,fill=fillColor] (211.28,243.13) circle (  1.16);

\path[draw=drawColor,line width= 0.4pt,line join=round,line cap=round,fill=fillColor] (211.43,243.12) circle (  1.16);

\path[draw=drawColor,line width= 0.4pt,line join=round,line cap=round,fill=fillColor] (211.58,243.10) circle (  1.16);

\path[draw=drawColor,line width= 0.4pt,line join=round,line cap=round,fill=fillColor] (211.73,243.05) circle (  1.16);

\path[draw=drawColor,line width= 0.4pt,line join=round,line cap=round,fill=fillColor] (211.88,243.04) circle (  1.16);

\path[draw=drawColor,line width= 0.4pt,line join=round,line cap=round,fill=fillColor] (212.03,243.02) circle (  1.16);

\path[draw=drawColor,line width= 0.4pt,line join=round,line cap=round,fill=fillColor] (212.18,242.94) circle (  1.16);

\path[draw=drawColor,line width= 0.4pt,line join=round,line cap=round,fill=fillColor] (212.33,242.82) circle (  1.16);

\path[draw=drawColor,line width= 0.4pt,line join=round,line cap=round,fill=fillColor] (212.48,242.80) circle (  1.16);

\path[draw=drawColor,line width= 0.4pt,line join=round,line cap=round,fill=fillColor] (212.63,242.76) circle (  1.16);

\path[draw=drawColor,line width= 0.4pt,line join=round,line cap=round,fill=fillColor] (212.78,242.73) circle (  1.16);

\path[draw=drawColor,line width= 0.4pt,line join=round,line cap=round,fill=fillColor] (212.93,242.68) circle (  1.16);

\path[draw=drawColor,line width= 0.4pt,line join=round,line cap=round,fill=fillColor] (213.07,242.60) circle (  1.16);

\path[draw=drawColor,line width= 0.4pt,line join=round,line cap=round,fill=fillColor] (213.22,242.60) circle (  1.16);

\path[draw=drawColor,line width= 0.4pt,line join=round,line cap=round,fill=fillColor] (213.37,242.60) circle (  1.16);

\path[draw=drawColor,line width= 0.4pt,line join=round,line cap=round,fill=fillColor] (213.51,242.54) circle (  1.16);

\path[draw=drawColor,line width= 0.4pt,line join=round,line cap=round,fill=fillColor] (213.66,242.45) circle (  1.16);

\path[draw=drawColor,line width= 0.4pt,line join=round,line cap=round,fill=fillColor] (213.81,242.44) circle (  1.16);

\path[draw=drawColor,line width= 0.4pt,line join=round,line cap=round,fill=fillColor] (213.95,242.40) circle (  1.16);

\path[draw=drawColor,line width= 0.4pt,line join=round,line cap=round,fill=fillColor] (214.10,242.40) circle (  1.16);

\path[draw=drawColor,line width= 0.4pt,line join=round,line cap=round,fill=fillColor] (214.24,242.37) circle (  1.16);

\path[draw=drawColor,line width= 0.4pt,line join=round,line cap=round,fill=fillColor] (214.39,242.37) circle (  1.16);

\path[draw=drawColor,line width= 0.4pt,line join=round,line cap=round,fill=fillColor] (214.54,242.07) circle (  1.16);

\path[draw=drawColor,line width= 0.4pt,line join=round,line cap=round,fill=fillColor] (214.68,241.98) circle (  1.16);

\path[draw=drawColor,line width= 0.4pt,line join=round,line cap=round,fill=fillColor] (214.82,241.81) circle (  1.16);

\path[draw=drawColor,line width= 0.4pt,line join=round,line cap=round,fill=fillColor] (214.97,241.64) circle (  1.16);

\path[draw=drawColor,line width= 0.4pt,line join=round,line cap=round,fill=fillColor] (215.11,241.64) circle (  1.16);

\path[draw=drawColor,line width= 0.4pt,line join=round,line cap=round,fill=fillColor] (215.26,241.58) circle (  1.16);

\path[draw=drawColor,line width= 0.4pt,line join=round,line cap=round,fill=fillColor] (215.40,241.42) circle (  1.16);

\path[draw=drawColor,line width= 0.4pt,line join=round,line cap=round,fill=fillColor] (215.54,241.30) circle (  1.16);

\path[draw=drawColor,line width= 0.4pt,line join=round,line cap=round,fill=fillColor] (215.69,241.23) circle (  1.16);

\path[draw=drawColor,line width= 0.4pt,line join=round,line cap=round,fill=fillColor] (215.83,241.20) circle (  1.16);

\path[draw=drawColor,line width= 0.4pt,line join=round,line cap=round,fill=fillColor] (215.97,241.19) circle (  1.16);

\path[draw=drawColor,line width= 0.4pt,line join=round,line cap=round,fill=fillColor] (216.11,241.07) circle (  1.16);

\path[draw=drawColor,line width= 0.4pt,line join=round,line cap=round,fill=fillColor] (216.26,240.82) circle (  1.16);

\path[draw=drawColor,line width= 0.4pt,line join=round,line cap=round,fill=fillColor] (216.40,240.76) circle (  1.16);

\path[draw=drawColor,line width= 0.4pt,line join=round,line cap=round,fill=fillColor] (216.54,240.72) circle (  1.16);

\path[draw=drawColor,line width= 0.4pt,line join=round,line cap=round,fill=fillColor] (216.68,240.67) circle (  1.16);

\path[draw=drawColor,line width= 0.4pt,line join=round,line cap=round,fill=fillColor] (216.82,240.63) circle (  1.16);

\path[draw=drawColor,line width= 0.4pt,line join=round,line cap=round,fill=fillColor] (216.96,240.59) circle (  1.16);

\path[draw=drawColor,line width= 0.4pt,line join=round,line cap=round,fill=fillColor] (217.10,240.59) circle (  1.16);

\path[draw=drawColor,line width= 0.4pt,line join=round,line cap=round,fill=fillColor] (217.24,240.42) circle (  1.16);

\path[draw=drawColor,line width= 0.4pt,line join=round,line cap=round,fill=fillColor] (217.38,239.98) circle (  1.16);

\path[draw=drawColor,line width= 0.4pt,line join=round,line cap=round,fill=fillColor] (217.52,239.83) circle (  1.16);

\path[draw=drawColor,line width= 0.4pt,line join=round,line cap=round,fill=fillColor] (217.66,239.81) circle (  1.16);

\path[draw=drawColor,line width= 0.4pt,line join=round,line cap=round,fill=fillColor] (217.80,239.61) circle (  1.16);

\path[draw=drawColor,line width= 0.4pt,line join=round,line cap=round,fill=fillColor] (217.94,239.52) circle (  1.16);

\path[draw=drawColor,line width= 0.4pt,line join=round,line cap=round,fill=fillColor] (218.08,239.21) circle (  1.16);

\path[draw=drawColor,line width= 0.4pt,line join=round,line cap=round,fill=fillColor] (218.22,239.13) circle (  1.16);

\path[draw=drawColor,line width= 0.4pt,line join=round,line cap=round,fill=fillColor] (218.36,238.93) circle (  1.16);

\path[draw=drawColor,line width= 0.4pt,line join=round,line cap=round,fill=fillColor] (218.50,238.86) circle (  1.16);

\path[draw=drawColor,line width= 0.4pt,line join=round,line cap=round,fill=fillColor] (218.63,238.73) circle (  1.16);

\path[draw=drawColor,line width= 0.4pt,line join=round,line cap=round,fill=fillColor] (218.77,238.63) circle (  1.16);

\path[draw=drawColor,line width= 0.4pt,line join=round,line cap=round,fill=fillColor] (218.91,238.63) circle (  1.16);

\path[draw=drawColor,line width= 0.4pt,line join=round,line cap=round,fill=fillColor] (219.05,238.58) circle (  1.16);

\path[draw=drawColor,line width= 0.4pt,line join=round,line cap=round,fill=fillColor] (219.18,238.54) circle (  1.16);

\path[draw=drawColor,line width= 0.4pt,line join=round,line cap=round,fill=fillColor] (219.32,238.47) circle (  1.16);

\path[draw=drawColor,line width= 0.4pt,line join=round,line cap=round,fill=fillColor] (219.46,238.39) circle (  1.16);

\path[draw=drawColor,line width= 0.4pt,line join=round,line cap=round,fill=fillColor] (219.59,237.73) circle (  1.16);

\path[draw=drawColor,line width= 0.4pt,line join=round,line cap=round,fill=fillColor] (219.73,237.72) circle (  1.16);

\path[draw=drawColor,line width= 0.4pt,line join=round,line cap=round,fill=fillColor] (219.87,237.69) circle (  1.16);

\path[draw=drawColor,line width= 0.4pt,line join=round,line cap=round,fill=fillColor] (220.00,237.66) circle (  1.16);

\path[draw=drawColor,line width= 0.4pt,line join=round,line cap=round,fill=fillColor] (220.14,237.53) circle (  1.16);

\path[draw=drawColor,line width= 0.4pt,line join=round,line cap=round,fill=fillColor] (220.27,237.43) circle (  1.16);

\path[draw=drawColor,line width= 0.4pt,line join=round,line cap=round,fill=fillColor] (220.41,237.38) circle (  1.16);

\path[draw=drawColor,line width= 0.4pt,line join=round,line cap=round,fill=fillColor] (220.54,237.26) circle (  1.16);

\path[draw=drawColor,line width= 0.4pt,line join=round,line cap=round,fill=fillColor] (220.68,237.22) circle (  1.16);

\path[draw=drawColor,line width= 0.4pt,line join=round,line cap=round,fill=fillColor] (220.81,236.91) circle (  1.16);

\path[draw=drawColor,line width= 0.4pt,line join=round,line cap=round,fill=fillColor] (220.95,236.76) circle (  1.16);

\path[draw=drawColor,line width= 0.4pt,line join=round,line cap=round,fill=fillColor] (221.08,236.70) circle (  1.16);

\path[draw=drawColor,line width= 0.4pt,line join=round,line cap=round,fill=fillColor] (221.21,236.61) circle (  1.16);

\path[draw=drawColor,line width= 0.4pt,line join=round,line cap=round,fill=fillColor] (221.35,236.37) circle (  1.16);

\path[draw=drawColor,line width= 0.4pt,line join=round,line cap=round,fill=fillColor] (221.48,236.31) circle (  1.16);

\path[draw=drawColor,line width= 0.4pt,line join=round,line cap=round,fill=fillColor] (221.61,235.99) circle (  1.16);

\path[draw=drawColor,line width= 0.4pt,line join=round,line cap=round,fill=fillColor] (221.75,235.61) circle (  1.16);

\path[draw=drawColor,line width= 0.4pt,line join=round,line cap=round,fill=fillColor] (221.88,235.29) circle (  1.16);

\path[draw=drawColor,line width= 0.4pt,line join=round,line cap=round,fill=fillColor] (222.01,235.06) circle (  1.16);

\path[draw=drawColor,line width= 0.4pt,line join=round,line cap=round,fill=fillColor] (222.14,234.90) circle (  1.16);

\path[draw=drawColor,line width= 0.4pt,line join=round,line cap=round,fill=fillColor] (222.28,234.73) circle (  1.16);

\path[draw=drawColor,line width= 0.4pt,line join=round,line cap=round,fill=fillColor] (222.41,234.69) circle (  1.16);

\path[draw=drawColor,line width= 0.4pt,line join=round,line cap=round,fill=fillColor] (222.54,234.69) circle (  1.16);

\path[draw=drawColor,line width= 0.4pt,line join=round,line cap=round,fill=fillColor] (222.67,234.53) circle (  1.16);

\path[draw=drawColor,line width= 0.4pt,line join=round,line cap=round,fill=fillColor] (222.80,234.48) circle (  1.16);

\path[draw=drawColor,line width= 0.4pt,line join=round,line cap=round,fill=fillColor] (222.93,234.34) circle (  1.16);

\path[draw=drawColor,line width= 0.4pt,line join=round,line cap=round,fill=fillColor] (223.07,233.85) circle (  1.16);

\path[draw=drawColor,line width= 0.4pt,line join=round,line cap=round,fill=fillColor] (223.20,233.72) circle (  1.16);

\path[draw=drawColor,line width= 0.4pt,line join=round,line cap=round,fill=fillColor] (223.33,233.50) circle (  1.16);

\path[draw=drawColor,line width= 0.4pt,line join=round,line cap=round,fill=fillColor] (223.46,233.32) circle (  1.16);

\path[draw=drawColor,line width= 0.4pt,line join=round,line cap=round,fill=fillColor] (223.59,232.83) circle (  1.16);

\path[draw=drawColor,line width= 0.4pt,line join=round,line cap=round,fill=fillColor] (223.72,231.58) circle (  1.16);

\path[draw=drawColor,line width= 0.4pt,line join=round,line cap=round,fill=fillColor] (223.85,231.43) circle (  1.16);

\path[draw=drawColor,line width= 0.4pt,line join=round,line cap=round,fill=fillColor] (223.98,230.76) circle (  1.16);

\path[draw=drawColor,line width= 0.4pt,line join=round,line cap=round,fill=fillColor] (224.11,230.52) circle (  1.16);

\path[draw=drawColor,line width= 0.4pt,line join=round,line cap=round,fill=fillColor] (224.23,229.37) circle (  1.16);

\path[draw=drawColor,line width= 0.4pt,line join=round,line cap=round,fill=fillColor] (224.36,228.21) circle (  1.16);

\path[draw=drawColor,line width= 0.4pt,line join=round,line cap=round,fill=fillColor] (224.49,227.82) circle (  1.16);

\path[draw=drawColor,line width= 0.4pt,line join=round,line cap=round,fill=fillColor] (224.62,226.11) circle (  1.16);

\path[draw=drawColor,line width= 0.4pt,line join=round,line cap=round,fill=fillColor] (224.75,221.66) circle (  1.16);

\path[draw=drawColor,line width= 0.4pt,line join=round,line cap=round,fill=fillColor] (224.88,219.75) circle (  1.16);

\path[draw=drawColor,line width= 0.4pt,line join=round,line cap=round,fill=fillColor] (225.01,209.81) circle (  1.16);

\path[draw=drawColor,line width= 0.4pt,line join=round,line cap=round,fill=fillColor] (225.13,209.81) circle (  1.16);

\path[draw=drawColor,line width= 0.4pt,line join=round,line cap=round,fill=fillColor] (225.26,209.81) circle (  1.16);
\definecolor[named]{drawColor}{rgb}{0.00,0.00,0.00}
\definecolor[named]{fillColor}{rgb}{0.00,0.00,0.00}

\path[draw=drawColor,line width= 0.6pt,line join=round,fill=fillColor] ( 67.36,292.78) -- (232.78,292.78);

\node[text=drawColor,anchor=base east,inner sep=0pt, outer sep=0pt, scale=  0.85] at (229.28,294.93) {infeasible solutions};

\path[draw=drawColor,line width= 0.6pt,line join=round,line cap=round] ( 67.36,201.52) rectangle (232.78,303.75);
\end{scope}
\begin{scope}
\path[clip] (  0.00,  0.00) rectangle (505.89,650.43);
\definecolor[named]{drawColor}{rgb}{0.00,0.00,0.00}

\node[text=drawColor,anchor=base east,inner sep=0pt, outer sep=0pt, scale=  0.80] at ( 61.96,207.06) {0.00};

\node[text=drawColor,anchor=base east,inner sep=0pt, outer sep=0pt, scale=  0.80] at ( 61.96,224.93) {0.01};

\node[text=drawColor,anchor=base east,inner sep=0pt, outer sep=0pt, scale=  0.80] at ( 61.96,237.62) {0.05};

\node[text=drawColor,anchor=base east,inner sep=0pt, outer sep=0pt, scale=  0.80] at ( 61.96,245.57) {0.10};

\node[text=drawColor,anchor=base east,inner sep=0pt, outer sep=0pt, scale=  0.80] at ( 61.96,255.57) {0.20};

\node[text=drawColor,anchor=base east,inner sep=0pt, outer sep=0pt, scale=  0.80] at ( 61.96,268.19) {0.40};

\node[text=drawColor,anchor=base east,inner sep=0pt, outer sep=0pt, scale=  0.80] at ( 61.96,277.03) {0.60};

\node[text=drawColor,anchor=base east,inner sep=0pt, outer sep=0pt, scale=  0.80] at ( 61.96,284.07) {0.80};

\node[text=drawColor,anchor=base east,inner sep=0pt, outer sep=0pt, scale=  0.80] at ( 61.96,290.02) {1.00};
\end{scope}
\begin{scope}
\path[clip] (  0.00,  0.00) rectangle (505.89,650.43);
\definecolor[named]{drawColor}{rgb}{0.00,0.00,0.00}

\path[draw=drawColor,line width= 0.6pt,line join=round] ( 64.36,209.81) --
	( 67.36,209.81);

\path[draw=drawColor,line width= 0.6pt,line join=round] ( 64.36,227.69) --
	( 67.36,227.69);

\path[draw=drawColor,line width= 0.6pt,line join=round] ( 64.36,240.38) --
	( 67.36,240.38);

\path[draw=drawColor,line width= 0.6pt,line join=round] ( 64.36,248.32) --
	( 67.36,248.32);

\path[draw=drawColor,line width= 0.6pt,line join=round] ( 64.36,258.33) --
	( 67.36,258.33);

\path[draw=drawColor,line width= 0.6pt,line join=round] ( 64.36,270.94) --
	( 67.36,270.94);

\path[draw=drawColor,line width= 0.6pt,line join=round] ( 64.36,279.79) --
	( 67.36,279.79);

\path[draw=drawColor,line width= 0.6pt,line join=round] ( 64.36,286.83) --
	( 67.36,286.83);

\path[draw=drawColor,line width= 0.6pt,line join=round] ( 64.36,292.78) --
	( 67.36,292.78);
\end{scope}
\begin{scope}
\path[clip] (  0.00,  0.00) rectangle (505.89,650.43);
\definecolor[named]{drawColor}{rgb}{0.00,0.00,0.00}

\path[draw=drawColor,line width= 0.6pt,line join=round] (156.93,198.52) --
	(156.93,201.52);

\path[draw=drawColor,line width= 0.6pt,line join=round] (184.11,198.52) --
	(184.11,201.52);

\path[draw=drawColor,line width= 0.6pt,line join=round] (203.18,198.52) --
	(203.18,201.52);

\path[draw=drawColor,line width= 0.6pt,line join=round] (218.36,198.52) --
	(218.36,201.52);

\path[draw=drawColor,line width= 0.6pt,line join=round] (231.18,198.52) --
	(231.18,201.52);
\end{scope}
\begin{scope}
\path[clip] (  0.00,  0.00) rectangle (505.89,650.43);
\definecolor[named]{drawColor}{rgb}{0.00,0.00,0.00}

\node[text=drawColor,rotate= 50.00,anchor=base east,inner sep=0pt, outer sep=0pt, scale=  0.80] at (161.15,192.57) {100};

\node[text=drawColor,rotate= 50.00,anchor=base east,inner sep=0pt, outer sep=0pt, scale=  0.80] at (188.33,192.57) {200};

\node[text=drawColor,rotate= 50.00,anchor=base east,inner sep=0pt, outer sep=0pt, scale=  0.80] at (207.40,192.57) {300};

\node[text=drawColor,rotate= 50.00,anchor=base east,inner sep=0pt, outer sep=0pt, scale=  0.80] at (222.58,192.57) {400};

\node[text=drawColor,rotate= 50.00,anchor=base east,inner sep=0pt, outer sep=0pt, scale=  0.80] at (235.40,192.57) {500};
\end{scope}
\begin{scope}
\path[clip] (  0.00,  0.00) rectangle (505.89,650.43);
\definecolor[named]{drawColor}{rgb}{0.00,0.00,0.00}

\node[text=drawColor,anchor=base,inner sep=0pt, outer sep=0pt, scale=  1.10] at (150.07,171.01) {\# Instances};
\end{scope}
\begin{scope}
\path[clip] (  0.00,  0.00) rectangle (505.89,650.43);
\definecolor[named]{drawColor}{rgb}{0.00,0.00,0.00}

\node[text=drawColor,rotate= 90.00,anchor=base,inner sep=0pt, outer sep=0pt, scale=  1.10] at ( 37.74,252.63) {1-(Best/Algorithm)};
\end{scope}
\begin{scope}
\path[clip] (  0.00,  0.00) rectangle (505.89,650.43);
\definecolor[named]{drawColor}{rgb}{0.00,0.00,0.00}

\node[text=drawColor,anchor=base,inner sep=0pt, outer sep=0pt, scale=  1.20] at (150.07,310.95) {$k=32$};
\end{scope}
\begin{scope}
\path[clip] (267.11,162.61) rectangle (491.72,325.21);
\definecolor[named]{drawColor}{rgb}{1.00,1.00,1.00}
\definecolor[named]{fillColor}{rgb}{1.00,1.00,1.00}

\path[draw=drawColor,line width= 0.6pt,line join=round,line cap=round,fill=fillColor] (267.11,162.61) rectangle (491.72,325.21);
\end{scope}
\begin{scope}
\path[clip] (320.31,201.52) rectangle (485.72,303.75);
\definecolor[named]{fillColor}{rgb}{1.00,1.00,1.00}

\path[fill=fillColor] (320.31,201.52) rectangle (485.72,303.75);
\definecolor[named]{drawColor}{rgb}{0.98,0.98,0.98}

\path[draw=drawColor,line width= 0.6pt,line join=round] (320.31,218.75) --
	(485.72,218.75);

\path[draw=drawColor,line width= 0.6pt,line join=round] (320.31,234.03) --
	(485.72,234.03);

\path[draw=drawColor,line width= 0.6pt,line join=round] (320.31,244.35) --
	(485.72,244.35);

\path[draw=drawColor,line width= 0.6pt,line join=round] (320.31,253.32) --
	(485.72,253.32);

\path[draw=drawColor,line width= 0.6pt,line join=round] (320.31,264.63) --
	(485.72,264.63);

\path[draw=drawColor,line width= 0.6pt,line join=round] (320.31,275.36) --
	(485.72,275.36);

\path[draw=drawColor,line width= 0.6pt,line join=round] (320.31,283.31) --
	(485.72,283.31);

\path[draw=drawColor,line width= 0.6pt,line join=round] (320.31,289.80) --
	(485.72,289.80);

\path[draw=drawColor,line width= 0.6pt,line join=round] (320.31,301.71) --
	(485.72,301.71);

\path[draw=drawColor,line width= 0.6pt,line join=round] (382.78,201.52) --
	(382.78,303.75);

\path[draw=drawColor,line width= 0.6pt,line join=round] (396.40,201.52) --
	(396.40,303.75);

\path[draw=drawColor,line width= 0.6pt,line join=round] (423.63,201.52) --
	(423.63,303.75);

\path[draw=drawColor,line width= 0.6pt,line join=round] (446.79,201.52) --
	(446.79,303.75);

\path[draw=drawColor,line width= 0.6pt,line join=round] (463.94,201.52) --
	(463.94,303.75);

\path[draw=drawColor,line width= 0.6pt,line join=round] (477.97,201.52) --
	(477.97,303.75);
\definecolor[named]{drawColor}{rgb}{0.75,0.75,0.75}

\path[draw=drawColor,line width= 0.6pt,dash pattern=on 1pt off 3pt ,line join=round] (320.31,209.81) --
	(485.72,209.81);

\path[draw=drawColor,line width= 0.6pt,dash pattern=on 1pt off 3pt ,line join=round] (320.31,227.69) --
	(485.72,227.69);

\path[draw=drawColor,line width= 0.6pt,dash pattern=on 1pt off 3pt ,line join=round] (320.31,240.38) --
	(485.72,240.38);

\path[draw=drawColor,line width= 0.6pt,dash pattern=on 1pt off 3pt ,line join=round] (320.31,248.32) --
	(485.72,248.32);

\path[draw=drawColor,line width= 0.6pt,dash pattern=on 1pt off 3pt ,line join=round] (320.31,258.33) --
	(485.72,258.33);

\path[draw=drawColor,line width= 0.6pt,dash pattern=on 1pt off 3pt ,line join=round] (320.31,270.94) --
	(485.72,270.94);

\path[draw=drawColor,line width= 0.6pt,dash pattern=on 1pt off 3pt ,line join=round] (320.31,279.79) --
	(485.72,279.79);

\path[draw=drawColor,line width= 0.6pt,dash pattern=on 1pt off 3pt ,line join=round] (320.31,286.83) --
	(485.72,286.83);

\path[draw=drawColor,line width= 0.6pt,dash pattern=on 1pt off 3pt ,line join=round] (320.31,292.78) --
	(485.72,292.78);

\path[draw=drawColor,line width= 0.6pt,dash pattern=on 1pt off 3pt ,line join=round] (410.01,201.52) --
	(410.01,303.75);

\path[draw=drawColor,line width= 0.6pt,dash pattern=on 1pt off 3pt ,line join=round] (437.24,201.52) --
	(437.24,303.75);

\path[draw=drawColor,line width= 0.6pt,dash pattern=on 1pt off 3pt ,line join=round] (456.34,201.52) --
	(456.34,303.75);

\path[draw=drawColor,line width= 0.6pt,dash pattern=on 1pt off 3pt ,line join=round] (471.55,201.52) --
	(471.55,303.75);

\path[draw=drawColor,line width= 0.6pt,dash pattern=on 1pt off 3pt ,line join=round] (484.39,201.52) --
	(484.39,303.75);
\definecolor[named]{drawColor}{rgb}{0.89,0.10,0.11}
\definecolor[named]{fillColor}{rgb}{0.89,0.10,0.11}

\path[draw=drawColor,line width= 0.4pt,line join=round,line cap=round,fill=fillColor] (327.82,265.80) circle (  1.16);

\path[draw=drawColor,line width= 0.4pt,line join=round,line cap=round,fill=fillColor] (333.69,264.43) circle (  1.16);

\path[draw=drawColor,line width= 0.4pt,line join=round,line cap=round,fill=fillColor] (337.81,262.96) circle (  1.16);

\path[draw=drawColor,line width= 0.4pt,line join=round,line cap=round,fill=fillColor] (341.08,260.51) circle (  1.16);

\path[draw=drawColor,line width= 0.4pt,line join=round,line cap=round,fill=fillColor] (343.85,259.88) circle (  1.16);

\path[draw=drawColor,line width= 0.4pt,line join=round,line cap=round,fill=fillColor] (346.27,257.30) circle (  1.16);

\path[draw=drawColor,line width= 0.4pt,line join=round,line cap=round,fill=fillColor] (348.43,254.55) circle (  1.16);

\path[draw=drawColor,line width= 0.4pt,line join=round,line cap=round,fill=fillColor] (350.39,252.09) circle (  1.16);

\path[draw=drawColor,line width= 0.4pt,line join=round,line cap=round,fill=fillColor] (352.20,250.84) circle (  1.16);

\path[draw=drawColor,line width= 0.4pt,line join=round,line cap=round,fill=fillColor] (353.88,249.42) circle (  1.16);

\path[draw=drawColor,line width= 0.4pt,line join=round,line cap=round,fill=fillColor] (355.45,249.30) circle (  1.16);

\path[draw=drawColor,line width= 0.4pt,line join=round,line cap=round,fill=fillColor] (356.93,247.78) circle (  1.16);

\path[draw=drawColor,line width= 0.4pt,line join=round,line cap=round,fill=fillColor] (358.32,247.29) circle (  1.16);

\path[draw=drawColor,line width= 0.4pt,line join=round,line cap=round,fill=fillColor] (359.65,246.74) circle (  1.16);

\path[draw=drawColor,line width= 0.4pt,line join=round,line cap=round,fill=fillColor] (360.92,246.19) circle (  1.16);

\path[draw=drawColor,line width= 0.4pt,line join=round,line cap=round,fill=fillColor] (362.13,246.12) circle (  1.16);

\path[draw=drawColor,line width= 0.4pt,line join=round,line cap=round,fill=fillColor] (363.29,245.90) circle (  1.16);

\path[draw=drawColor,line width= 0.4pt,line join=round,line cap=round,fill=fillColor] (364.40,245.53) circle (  1.16);

\path[draw=drawColor,line width= 0.4pt,line join=round,line cap=round,fill=fillColor] (365.48,245.43) circle (  1.16);

\path[draw=drawColor,line width= 0.4pt,line join=round,line cap=round,fill=fillColor] (366.52,245.33) circle (  1.16);

\path[draw=drawColor,line width= 0.4pt,line join=round,line cap=round,fill=fillColor] (367.52,245.30) circle (  1.16);

\path[draw=drawColor,line width= 0.4pt,line join=round,line cap=round,fill=fillColor] (368.49,245.27) circle (  1.16);

\path[draw=drawColor,line width= 0.4pt,line join=round,line cap=round,fill=fillColor] (369.44,244.99) circle (  1.16);

\path[draw=drawColor,line width= 0.4pt,line join=round,line cap=round,fill=fillColor] (370.36,244.96) circle (  1.16);

\path[draw=drawColor,line width= 0.4pt,line join=round,line cap=round,fill=fillColor] (371.25,244.81) circle (  1.16);

\path[draw=drawColor,line width= 0.4pt,line join=round,line cap=round,fill=fillColor] (372.12,244.57) circle (  1.16);

\path[draw=drawColor,line width= 0.4pt,line join=round,line cap=round,fill=fillColor] (372.96,244.41) circle (  1.16);

\path[draw=drawColor,line width= 0.4pt,line join=round,line cap=round,fill=fillColor] (373.79,244.39) circle (  1.16);

\path[draw=drawColor,line width= 0.4pt,line join=round,line cap=round,fill=fillColor] (374.59,243.92) circle (  1.16);

\path[draw=drawColor,line width= 0.4pt,line join=round,line cap=round,fill=fillColor] (375.38,243.32) circle (  1.16);

\path[draw=drawColor,line width= 0.4pt,line join=round,line cap=round,fill=fillColor] (376.15,242.76) circle (  1.16);

\path[draw=drawColor,line width= 0.4pt,line join=round,line cap=round,fill=fillColor] (376.91,242.59) circle (  1.16);

\path[draw=drawColor,line width= 0.4pt,line join=round,line cap=round,fill=fillColor] (377.65,242.15) circle (  1.16);

\path[draw=drawColor,line width= 0.4pt,line join=round,line cap=round,fill=fillColor] (378.37,241.91) circle (  1.16);

\path[draw=drawColor,line width= 0.4pt,line join=round,line cap=round,fill=fillColor] (379.08,241.78) circle (  1.16);

\path[draw=drawColor,line width= 0.4pt,line join=round,line cap=round,fill=fillColor] (379.78,241.70) circle (  1.16);

\path[draw=drawColor,line width= 0.4pt,line join=round,line cap=round,fill=fillColor] (380.46,241.59) circle (  1.16);

\path[draw=drawColor,line width= 0.4pt,line join=round,line cap=round,fill=fillColor] (381.13,241.45) circle (  1.16);

\path[draw=drawColor,line width= 0.4pt,line join=round,line cap=round,fill=fillColor] (381.79,241.33) circle (  1.16);

\path[draw=drawColor,line width= 0.4pt,line join=round,line cap=round,fill=fillColor] (382.44,241.32) circle (  1.16);

\path[draw=drawColor,line width= 0.4pt,line join=round,line cap=round,fill=fillColor] (383.08,241.21) circle (  1.16);

\path[draw=drawColor,line width= 0.4pt,line join=round,line cap=round,fill=fillColor] (383.71,241.16) circle (  1.16);

\path[draw=drawColor,line width= 0.4pt,line join=round,line cap=round,fill=fillColor] (384.32,241.15) circle (  1.16);

\path[draw=drawColor,line width= 0.4pt,line join=round,line cap=round,fill=fillColor] (384.93,241.13) circle (  1.16);

\path[draw=drawColor,line width= 0.4pt,line join=round,line cap=round,fill=fillColor] (385.53,240.87) circle (  1.16);

\path[draw=drawColor,line width= 0.4pt,line join=round,line cap=round,fill=fillColor] (386.12,240.70) circle (  1.16);

\path[draw=drawColor,line width= 0.4pt,line join=round,line cap=round,fill=fillColor] (386.70,240.68) circle (  1.16);

\path[draw=drawColor,line width= 0.4pt,line join=round,line cap=round,fill=fillColor] (387.28,240.52) circle (  1.16);

\path[draw=drawColor,line width= 0.4pt,line join=round,line cap=round,fill=fillColor] (387.84,240.43) circle (  1.16);

\path[draw=drawColor,line width= 0.4pt,line join=round,line cap=round,fill=fillColor] (388.40,240.40) circle (  1.16);

\path[draw=drawColor,line width= 0.4pt,line join=round,line cap=round,fill=fillColor] (388.95,240.36) circle (  1.16);

\path[draw=drawColor,line width= 0.4pt,line join=round,line cap=round,fill=fillColor] (389.50,240.24) circle (  1.16);

\path[draw=drawColor,line width= 0.4pt,line join=round,line cap=round,fill=fillColor] (390.03,240.19) circle (  1.16);

\path[draw=drawColor,line width= 0.4pt,line join=round,line cap=round,fill=fillColor] (390.56,240.19) circle (  1.16);

\path[draw=drawColor,line width= 0.4pt,line join=round,line cap=round,fill=fillColor] (391.08,240.16) circle (  1.16);

\path[draw=drawColor,line width= 0.4pt,line join=round,line cap=round,fill=fillColor] (391.60,239.98) circle (  1.16);

\path[draw=drawColor,line width= 0.4pt,line join=round,line cap=round,fill=fillColor] (392.11,239.96) circle (  1.16);

\path[draw=drawColor,line width= 0.4pt,line join=round,line cap=round,fill=fillColor] (392.62,239.85) circle (  1.16);

\path[draw=drawColor,line width= 0.4pt,line join=round,line cap=round,fill=fillColor] (393.12,239.72) circle (  1.16);

\path[draw=drawColor,line width= 0.4pt,line join=round,line cap=round,fill=fillColor] (393.61,239.63) circle (  1.16);

\path[draw=drawColor,line width= 0.4pt,line join=round,line cap=round,fill=fillColor] (394.10,239.50) circle (  1.16);

\path[draw=drawColor,line width= 0.4pt,line join=round,line cap=round,fill=fillColor] (394.58,239.29) circle (  1.16);

\path[draw=drawColor,line width= 0.4pt,line join=round,line cap=round,fill=fillColor] (395.06,239.26) circle (  1.16);

\path[draw=drawColor,line width= 0.4pt,line join=round,line cap=round,fill=fillColor] (395.53,239.26) circle (  1.16);

\path[draw=drawColor,line width= 0.4pt,line join=round,line cap=round,fill=fillColor] (396.00,239.24) circle (  1.16);

\path[draw=drawColor,line width= 0.4pt,line join=round,line cap=round,fill=fillColor] (396.46,239.19) circle (  1.16);

\path[draw=drawColor,line width= 0.4pt,line join=round,line cap=round,fill=fillColor] (396.92,239.15) circle (  1.16);

\path[draw=drawColor,line width= 0.4pt,line join=round,line cap=round,fill=fillColor] (397.37,239.10) circle (  1.16);

\path[draw=drawColor,line width= 0.4pt,line join=round,line cap=round,fill=fillColor] (397.82,239.10) circle (  1.16);

\path[draw=drawColor,line width= 0.4pt,line join=round,line cap=round,fill=fillColor] (398.27,239.09) circle (  1.16);

\path[draw=drawColor,line width= 0.4pt,line join=round,line cap=round,fill=fillColor] (398.71,239.02) circle (  1.16);

\path[draw=drawColor,line width= 0.4pt,line join=round,line cap=round,fill=fillColor] (399.15,238.85) circle (  1.16);

\path[draw=drawColor,line width= 0.4pt,line join=round,line cap=round,fill=fillColor] (399.58,238.83) circle (  1.16);

\path[draw=drawColor,line width= 0.4pt,line join=round,line cap=round,fill=fillColor] (400.01,238.81) circle (  1.16);

\path[draw=drawColor,line width= 0.4pt,line join=round,line cap=round,fill=fillColor] (400.43,238.79) circle (  1.16);

\path[draw=drawColor,line width= 0.4pt,line join=round,line cap=round,fill=fillColor] (400.85,238.72) circle (  1.16);

\path[draw=drawColor,line width= 0.4pt,line join=round,line cap=round,fill=fillColor] (401.27,238.63) circle (  1.16);

\path[draw=drawColor,line width= 0.4pt,line join=round,line cap=round,fill=fillColor] (401.69,238.58) circle (  1.16);

\path[draw=drawColor,line width= 0.4pt,line join=round,line cap=round,fill=fillColor] (402.10,238.49) circle (  1.16);

\path[draw=drawColor,line width= 0.4pt,line join=round,line cap=round,fill=fillColor] (402.50,238.25) circle (  1.16);

\path[draw=drawColor,line width= 0.4pt,line join=round,line cap=round,fill=fillColor] (402.91,238.20) circle (  1.16);

\path[draw=drawColor,line width= 0.4pt,line join=round,line cap=round,fill=fillColor] (403.31,238.17) circle (  1.16);

\path[draw=drawColor,line width= 0.4pt,line join=round,line cap=round,fill=fillColor] (403.70,238.16) circle (  1.16);

\path[draw=drawColor,line width= 0.4pt,line join=round,line cap=round,fill=fillColor] (404.10,237.88) circle (  1.16);

\path[draw=drawColor,line width= 0.4pt,line join=round,line cap=round,fill=fillColor] (404.49,237.80) circle (  1.16);

\path[draw=drawColor,line width= 0.4pt,line join=round,line cap=round,fill=fillColor] (404.88,237.80) circle (  1.16);

\path[draw=drawColor,line width= 0.4pt,line join=round,line cap=round,fill=fillColor] (405.26,237.78) circle (  1.16);

\path[draw=drawColor,line width= 0.4pt,line join=round,line cap=round,fill=fillColor] (405.64,237.74) circle (  1.16);

\path[draw=drawColor,line width= 0.4pt,line join=round,line cap=round,fill=fillColor] (406.02,237.73) circle (  1.16);

\path[draw=drawColor,line width= 0.4pt,line join=round,line cap=round,fill=fillColor] (406.40,237.53) circle (  1.16);

\path[draw=drawColor,line width= 0.4pt,line join=round,line cap=round,fill=fillColor] (406.77,237.48) circle (  1.16);

\path[draw=drawColor,line width= 0.4pt,line join=round,line cap=round,fill=fillColor] (407.14,237.48) circle (  1.16);

\path[draw=drawColor,line width= 0.4pt,line join=round,line cap=round,fill=fillColor] (407.51,237.44) circle (  1.16);

\path[draw=drawColor,line width= 0.4pt,line join=round,line cap=round,fill=fillColor] (407.87,237.33) circle (  1.16);

\path[draw=drawColor,line width= 0.4pt,line join=round,line cap=round,fill=fillColor] (408.24,237.32) circle (  1.16);

\path[draw=drawColor,line width= 0.4pt,line join=round,line cap=round,fill=fillColor] (408.60,237.15) circle (  1.16);

\path[draw=drawColor,line width= 0.4pt,line join=round,line cap=round,fill=fillColor] (408.95,236.99) circle (  1.16);

\path[draw=drawColor,line width= 0.4pt,line join=round,line cap=round,fill=fillColor] (409.31,236.92) circle (  1.16);

\path[draw=drawColor,line width= 0.4pt,line join=round,line cap=round,fill=fillColor] (409.66,236.82) circle (  1.16);

\path[draw=drawColor,line width= 0.4pt,line join=round,line cap=round,fill=fillColor] (410.01,236.76) circle (  1.16);

\path[draw=drawColor,line width= 0.4pt,line join=round,line cap=round,fill=fillColor] (410.36,236.67) circle (  1.16);

\path[draw=drawColor,line width= 0.4pt,line join=round,line cap=round,fill=fillColor] (410.71,236.58) circle (  1.16);

\path[draw=drawColor,line width= 0.4pt,line join=round,line cap=round,fill=fillColor] (411.05,236.54) circle (  1.16);

\path[draw=drawColor,line width= 0.4pt,line join=round,line cap=round,fill=fillColor] (411.39,236.54) circle (  1.16);

\path[draw=drawColor,line width= 0.4pt,line join=round,line cap=round,fill=fillColor] (411.73,236.52) circle (  1.16);

\path[draw=drawColor,line width= 0.4pt,line join=round,line cap=round,fill=fillColor] (412.07,236.47) circle (  1.16);

\path[draw=drawColor,line width= 0.4pt,line join=round,line cap=round,fill=fillColor] (412.40,236.43) circle (  1.16);

\path[draw=drawColor,line width= 0.4pt,line join=round,line cap=round,fill=fillColor] (412.73,236.30) circle (  1.16);

\path[draw=drawColor,line width= 0.4pt,line join=round,line cap=round,fill=fillColor] (413.07,236.30) circle (  1.16);

\path[draw=drawColor,line width= 0.4pt,line join=round,line cap=round,fill=fillColor] (413.39,236.30) circle (  1.16);

\path[draw=drawColor,line width= 0.4pt,line join=round,line cap=round,fill=fillColor] (413.72,236.12) circle (  1.16);

\path[draw=drawColor,line width= 0.4pt,line join=round,line cap=round,fill=fillColor] (414.05,236.04) circle (  1.16);

\path[draw=drawColor,line width= 0.4pt,line join=round,line cap=round,fill=fillColor] (414.37,236.00) circle (  1.16);

\path[draw=drawColor,line width= 0.4pt,line join=round,line cap=round,fill=fillColor] (414.69,235.85) circle (  1.16);

\path[draw=drawColor,line width= 0.4pt,line join=round,line cap=round,fill=fillColor] (415.01,235.84) circle (  1.16);

\path[draw=drawColor,line width= 0.4pt,line join=round,line cap=round,fill=fillColor] (415.33,235.61) circle (  1.16);

\path[draw=drawColor,line width= 0.4pt,line join=round,line cap=round,fill=fillColor] (415.64,235.52) circle (  1.16);

\path[draw=drawColor,line width= 0.4pt,line join=round,line cap=round,fill=fillColor] (415.95,235.42) circle (  1.16);

\path[draw=drawColor,line width= 0.4pt,line join=round,line cap=round,fill=fillColor] (416.27,235.42) circle (  1.16);

\path[draw=drawColor,line width= 0.4pt,line join=round,line cap=round,fill=fillColor] (416.58,235.42) circle (  1.16);

\path[draw=drawColor,line width= 0.4pt,line join=round,line cap=round,fill=fillColor] (416.88,235.40) circle (  1.16);

\path[draw=drawColor,line width= 0.4pt,line join=round,line cap=round,fill=fillColor] (417.19,235.27) circle (  1.16);

\path[draw=drawColor,line width= 0.4pt,line join=round,line cap=round,fill=fillColor] (417.50,235.27) circle (  1.16);

\path[draw=drawColor,line width= 0.4pt,line join=round,line cap=round,fill=fillColor] (417.80,235.16) circle (  1.16);

\path[draw=drawColor,line width= 0.4pt,line join=round,line cap=round,fill=fillColor] (418.10,235.15) circle (  1.16);

\path[draw=drawColor,line width= 0.4pt,line join=round,line cap=round,fill=fillColor] (418.40,235.12) circle (  1.16);

\path[draw=drawColor,line width= 0.4pt,line join=round,line cap=round,fill=fillColor] (418.70,235.05) circle (  1.16);

\path[draw=drawColor,line width= 0.4pt,line join=round,line cap=round,fill=fillColor] (419.00,235.05) circle (  1.16);

\path[draw=drawColor,line width= 0.4pt,line join=round,line cap=round,fill=fillColor] (419.29,234.89) circle (  1.16);

\path[draw=drawColor,line width= 0.4pt,line join=round,line cap=round,fill=fillColor] (419.59,234.75) circle (  1.16);

\path[draw=drawColor,line width= 0.4pt,line join=round,line cap=round,fill=fillColor] (419.88,234.73) circle (  1.16);

\path[draw=drawColor,line width= 0.4pt,line join=round,line cap=round,fill=fillColor] (420.17,234.63) circle (  1.16);

\path[draw=drawColor,line width= 0.4pt,line join=round,line cap=round,fill=fillColor] (420.46,234.59) circle (  1.16);

\path[draw=drawColor,line width= 0.4pt,line join=round,line cap=round,fill=fillColor] (420.75,234.57) circle (  1.16);

\path[draw=drawColor,line width= 0.4pt,line join=round,line cap=round,fill=fillColor] (421.03,234.47) circle (  1.16);

\path[draw=drawColor,line width= 0.4pt,line join=round,line cap=round,fill=fillColor] (421.32,234.47) circle (  1.16);

\path[draw=drawColor,line width= 0.4pt,line join=round,line cap=round,fill=fillColor] (421.60,234.46) circle (  1.16);

\path[draw=drawColor,line width= 0.4pt,line join=round,line cap=round,fill=fillColor] (421.89,234.39) circle (  1.16);

\path[draw=drawColor,line width= 0.4pt,line join=round,line cap=round,fill=fillColor] (422.17,234.38) circle (  1.16);

\path[draw=drawColor,line width= 0.4pt,line join=round,line cap=round,fill=fillColor] (422.45,234.36) circle (  1.16);

\path[draw=drawColor,line width= 0.4pt,line join=round,line cap=round,fill=fillColor] (422.72,234.33) circle (  1.16);

\path[draw=drawColor,line width= 0.4pt,line join=round,line cap=round,fill=fillColor] (423.00,234.24) circle (  1.16);

\path[draw=drawColor,line width= 0.4pt,line join=round,line cap=round,fill=fillColor] (423.28,234.14) circle (  1.16);

\path[draw=drawColor,line width= 0.4pt,line join=round,line cap=round,fill=fillColor] (423.55,234.14) circle (  1.16);

\path[draw=drawColor,line width= 0.4pt,line join=round,line cap=round,fill=fillColor] (423.82,234.12) circle (  1.16);

\path[draw=drawColor,line width= 0.4pt,line join=round,line cap=round,fill=fillColor] (424.10,234.08) circle (  1.16);

\path[draw=drawColor,line width= 0.4pt,line join=round,line cap=round,fill=fillColor] (424.37,234.08) circle (  1.16);

\path[draw=drawColor,line width= 0.4pt,line join=round,line cap=round,fill=fillColor] (424.64,234.08) circle (  1.16);

\path[draw=drawColor,line width= 0.4pt,line join=round,line cap=round,fill=fillColor] (424.91,234.01) circle (  1.16);

\path[draw=drawColor,line width= 0.4pt,line join=round,line cap=round,fill=fillColor] (425.17,233.97) circle (  1.16);

\path[draw=drawColor,line width= 0.4pt,line join=round,line cap=round,fill=fillColor] (425.44,233.88) circle (  1.16);

\path[draw=drawColor,line width= 0.4pt,line join=round,line cap=round,fill=fillColor] (425.70,233.85) circle (  1.16);

\path[draw=drawColor,line width= 0.4pt,line join=round,line cap=round,fill=fillColor] (425.97,233.85) circle (  1.16);

\path[draw=drawColor,line width= 0.4pt,line join=round,line cap=round,fill=fillColor] (426.23,233.81) circle (  1.16);

\path[draw=drawColor,line width= 0.4pt,line join=round,line cap=round,fill=fillColor] (426.49,233.79) circle (  1.16);

\path[draw=drawColor,line width= 0.4pt,line join=round,line cap=round,fill=fillColor] (426.75,233.73) circle (  1.16);

\path[draw=drawColor,line width= 0.4pt,line join=round,line cap=round,fill=fillColor] (427.01,233.68) circle (  1.16);

\path[draw=drawColor,line width= 0.4pt,line join=round,line cap=round,fill=fillColor] (427.27,233.68) circle (  1.16);

\path[draw=drawColor,line width= 0.4pt,line join=round,line cap=round,fill=fillColor] (427.52,233.62) circle (  1.16);

\path[draw=drawColor,line width= 0.4pt,line join=round,line cap=round,fill=fillColor] (427.78,233.61) circle (  1.16);

\path[draw=drawColor,line width= 0.4pt,line join=round,line cap=round,fill=fillColor] (428.03,233.53) circle (  1.16);

\path[draw=drawColor,line width= 0.4pt,line join=round,line cap=round,fill=fillColor] (428.29,233.52) circle (  1.16);

\path[draw=drawColor,line width= 0.4pt,line join=round,line cap=round,fill=fillColor] (428.54,233.47) circle (  1.16);

\path[draw=drawColor,line width= 0.4pt,line join=round,line cap=round,fill=fillColor] (428.79,233.37) circle (  1.16);

\path[draw=drawColor,line width= 0.4pt,line join=round,line cap=round,fill=fillColor] (429.04,233.26) circle (  1.16);

\path[draw=drawColor,line width= 0.4pt,line join=round,line cap=round,fill=fillColor] (429.29,233.22) circle (  1.16);

\path[draw=drawColor,line width= 0.4pt,line join=round,line cap=round,fill=fillColor] (429.54,232.92) circle (  1.16);

\path[draw=drawColor,line width= 0.4pt,line join=round,line cap=round,fill=fillColor] (429.79,232.88) circle (  1.16);

\path[draw=drawColor,line width= 0.4pt,line join=round,line cap=round,fill=fillColor] (430.04,232.87) circle (  1.16);

\path[draw=drawColor,line width= 0.4pt,line join=round,line cap=round,fill=fillColor] (430.28,232.86) circle (  1.16);

\path[draw=drawColor,line width= 0.4pt,line join=round,line cap=round,fill=fillColor] (430.53,232.82) circle (  1.16);

\path[draw=drawColor,line width= 0.4pt,line join=round,line cap=round,fill=fillColor] (430.77,232.73) circle (  1.16);

\path[draw=drawColor,line width= 0.4pt,line join=round,line cap=round,fill=fillColor] (431.01,232.70) circle (  1.16);

\path[draw=drawColor,line width= 0.4pt,line join=round,line cap=round,fill=fillColor] (431.25,232.70) circle (  1.16);

\path[draw=drawColor,line width= 0.4pt,line join=round,line cap=round,fill=fillColor] (431.50,232.68) circle (  1.16);

\path[draw=drawColor,line width= 0.4pt,line join=round,line cap=round,fill=fillColor] (431.74,232.68) circle (  1.16);

\path[draw=drawColor,line width= 0.4pt,line join=round,line cap=round,fill=fillColor] (431.97,232.66) circle (  1.16);

\path[draw=drawColor,line width= 0.4pt,line join=round,line cap=round,fill=fillColor] (432.21,232.66) circle (  1.16);

\path[draw=drawColor,line width= 0.4pt,line join=round,line cap=round,fill=fillColor] (432.45,232.65) circle (  1.16);

\path[draw=drawColor,line width= 0.4pt,line join=round,line cap=round,fill=fillColor] (432.69,232.63) circle (  1.16);

\path[draw=drawColor,line width= 0.4pt,line join=round,line cap=round,fill=fillColor] (432.92,232.62) circle (  1.16);

\path[draw=drawColor,line width= 0.4pt,line join=round,line cap=round,fill=fillColor] (433.16,232.61) circle (  1.16);

\path[draw=drawColor,line width= 0.4pt,line join=round,line cap=round,fill=fillColor] (433.39,232.54) circle (  1.16);

\path[draw=drawColor,line width= 0.4pt,line join=round,line cap=round,fill=fillColor] (433.62,232.51) circle (  1.16);

\path[draw=drawColor,line width= 0.4pt,line join=round,line cap=round,fill=fillColor] (433.86,232.47) circle (  1.16);

\path[draw=drawColor,line width= 0.4pt,line join=round,line cap=round,fill=fillColor] (434.09,232.46) circle (  1.16);

\path[draw=drawColor,line width= 0.4pt,line join=round,line cap=round,fill=fillColor] (434.32,232.46) circle (  1.16);

\path[draw=drawColor,line width= 0.4pt,line join=round,line cap=round,fill=fillColor] (434.55,232.39) circle (  1.16);

\path[draw=drawColor,line width= 0.4pt,line join=round,line cap=round,fill=fillColor] (434.78,232.38) circle (  1.16);

\path[draw=drawColor,line width= 0.4pt,line join=round,line cap=round,fill=fillColor] (435.00,232.37) circle (  1.16);

\path[draw=drawColor,line width= 0.4pt,line join=round,line cap=round,fill=fillColor] (435.23,232.36) circle (  1.16);

\path[draw=drawColor,line width= 0.4pt,line join=round,line cap=round,fill=fillColor] (435.46,232.36) circle (  1.16);

\path[draw=drawColor,line width= 0.4pt,line join=round,line cap=round,fill=fillColor] (435.68,232.35) circle (  1.16);

\path[draw=drawColor,line width= 0.4pt,line join=round,line cap=round,fill=fillColor] (435.91,232.33) circle (  1.16);

\path[draw=drawColor,line width= 0.4pt,line join=round,line cap=round,fill=fillColor] (436.13,232.27) circle (  1.16);

\path[draw=drawColor,line width= 0.4pt,line join=round,line cap=round,fill=fillColor] (436.36,232.25) circle (  1.16);

\path[draw=drawColor,line width= 0.4pt,line join=round,line cap=round,fill=fillColor] (436.58,232.25) circle (  1.16);

\path[draw=drawColor,line width= 0.4pt,line join=round,line cap=round,fill=fillColor] (436.80,232.22) circle (  1.16);

\path[draw=drawColor,line width= 0.4pt,line join=round,line cap=round,fill=fillColor] (437.02,232.16) circle (  1.16);

\path[draw=drawColor,line width= 0.4pt,line join=round,line cap=round,fill=fillColor] (437.24,232.14) circle (  1.16);

\path[draw=drawColor,line width= 0.4pt,line join=round,line cap=round,fill=fillColor] (437.46,232.14) circle (  1.16);

\path[draw=drawColor,line width= 0.4pt,line join=round,line cap=round,fill=fillColor] (437.68,232.12) circle (  1.16);

\path[draw=drawColor,line width= 0.4pt,line join=round,line cap=round,fill=fillColor] (437.90,232.11) circle (  1.16);

\path[draw=drawColor,line width= 0.4pt,line join=round,line cap=round,fill=fillColor] (438.12,232.10) circle (  1.16);

\path[draw=drawColor,line width= 0.4pt,line join=round,line cap=round,fill=fillColor] (438.33,232.07) circle (  1.16);

\path[draw=drawColor,line width= 0.4pt,line join=round,line cap=round,fill=fillColor] (438.55,232.07) circle (  1.16);

\path[draw=drawColor,line width= 0.4pt,line join=round,line cap=round,fill=fillColor] (438.76,231.98) circle (  1.16);

\path[draw=drawColor,line width= 0.4pt,line join=round,line cap=round,fill=fillColor] (438.98,231.93) circle (  1.16);

\path[draw=drawColor,line width= 0.4pt,line join=round,line cap=round,fill=fillColor] (439.19,231.88) circle (  1.16);

\path[draw=drawColor,line width= 0.4pt,line join=round,line cap=round,fill=fillColor] (439.41,231.81) circle (  1.16);

\path[draw=drawColor,line width= 0.4pt,line join=round,line cap=round,fill=fillColor] (439.62,231.80) circle (  1.16);

\path[draw=drawColor,line width= 0.4pt,line join=round,line cap=round,fill=fillColor] (439.83,231.75) circle (  1.16);

\path[draw=drawColor,line width= 0.4pt,line join=round,line cap=round,fill=fillColor] (440.04,231.67) circle (  1.16);

\path[draw=drawColor,line width= 0.4pt,line join=round,line cap=round,fill=fillColor] (440.25,231.65) circle (  1.16);

\path[draw=drawColor,line width= 0.4pt,line join=round,line cap=round,fill=fillColor] (440.46,231.62) circle (  1.16);

\path[draw=drawColor,line width= 0.4pt,line join=round,line cap=round,fill=fillColor] (440.67,231.61) circle (  1.16);

\path[draw=drawColor,line width= 0.4pt,line join=round,line cap=round,fill=fillColor] (440.88,231.47) circle (  1.16);

\path[draw=drawColor,line width= 0.4pt,line join=round,line cap=round,fill=fillColor] (441.09,231.43) circle (  1.16);

\path[draw=drawColor,line width= 0.4pt,line join=round,line cap=round,fill=fillColor] (441.29,231.35) circle (  1.16);

\path[draw=drawColor,line width= 0.4pt,line join=round,line cap=round,fill=fillColor] (441.50,231.32) circle (  1.16);

\path[draw=drawColor,line width= 0.4pt,line join=round,line cap=round,fill=fillColor] (441.71,231.30) circle (  1.16);

\path[draw=drawColor,line width= 0.4pt,line join=round,line cap=round,fill=fillColor] (441.91,231.29) circle (  1.16);

\path[draw=drawColor,line width= 0.4pt,line join=round,line cap=round,fill=fillColor] (442.12,231.29) circle (  1.16);

\path[draw=drawColor,line width= 0.4pt,line join=round,line cap=round,fill=fillColor] (442.32,231.26) circle (  1.16);

\path[draw=drawColor,line width= 0.4pt,line join=round,line cap=round,fill=fillColor] (442.53,231.25) circle (  1.16);

\path[draw=drawColor,line width= 0.4pt,line join=round,line cap=round,fill=fillColor] (442.73,231.24) circle (  1.16);

\path[draw=drawColor,line width= 0.4pt,line join=round,line cap=round,fill=fillColor] (442.93,231.24) circle (  1.16);

\path[draw=drawColor,line width= 0.4pt,line join=round,line cap=round,fill=fillColor] (443.13,231.15) circle (  1.16);

\path[draw=drawColor,line width= 0.4pt,line join=round,line cap=round,fill=fillColor] (443.33,231.06) circle (  1.16);

\path[draw=drawColor,line width= 0.4pt,line join=round,line cap=round,fill=fillColor] (443.54,231.04) circle (  1.16);

\path[draw=drawColor,line width= 0.4pt,line join=round,line cap=round,fill=fillColor] (443.74,230.98) circle (  1.16);

\path[draw=drawColor,line width= 0.4pt,line join=round,line cap=round,fill=fillColor] (443.94,230.96) circle (  1.16);

\path[draw=drawColor,line width= 0.4pt,line join=round,line cap=round,fill=fillColor] (444.13,230.89) circle (  1.16);

\path[draw=drawColor,line width= 0.4pt,line join=round,line cap=round,fill=fillColor] (444.33,230.89) circle (  1.16);

\path[draw=drawColor,line width= 0.4pt,line join=round,line cap=round,fill=fillColor] (444.53,230.86) circle (  1.16);

\path[draw=drawColor,line width= 0.4pt,line join=round,line cap=round,fill=fillColor] (444.73,230.85) circle (  1.16);

\path[draw=drawColor,line width= 0.4pt,line join=round,line cap=round,fill=fillColor] (444.92,230.85) circle (  1.16);

\path[draw=drawColor,line width= 0.4pt,line join=round,line cap=round,fill=fillColor] (445.12,230.84) circle (  1.16);

\path[draw=drawColor,line width= 0.4pt,line join=round,line cap=round,fill=fillColor] (445.32,230.84) circle (  1.16);

\path[draw=drawColor,line width= 0.4pt,line join=round,line cap=round,fill=fillColor] (445.51,230.83) circle (  1.16);

\path[draw=drawColor,line width= 0.4pt,line join=round,line cap=round,fill=fillColor] (445.71,230.83) circle (  1.16);

\path[draw=drawColor,line width= 0.4pt,line join=round,line cap=round,fill=fillColor] (445.90,230.79) circle (  1.16);

\path[draw=drawColor,line width= 0.4pt,line join=round,line cap=round,fill=fillColor] (446.09,230.78) circle (  1.16);

\path[draw=drawColor,line width= 0.4pt,line join=round,line cap=round,fill=fillColor] (446.29,230.72) circle (  1.16);

\path[draw=drawColor,line width= 0.4pt,line join=round,line cap=round,fill=fillColor] (446.48,230.72) circle (  1.16);

\path[draw=drawColor,line width= 0.4pt,line join=round,line cap=round,fill=fillColor] (446.67,230.70) circle (  1.16);

\path[draw=drawColor,line width= 0.4pt,line join=round,line cap=round,fill=fillColor] (446.86,230.64) circle (  1.16);

\path[draw=drawColor,line width= 0.4pt,line join=round,line cap=round,fill=fillColor] (447.05,230.62) circle (  1.16);

\path[draw=drawColor,line width= 0.4pt,line join=round,line cap=round,fill=fillColor] (447.24,230.60) circle (  1.16);

\path[draw=drawColor,line width= 0.4pt,line join=round,line cap=round,fill=fillColor] (447.43,230.59) circle (  1.16);

\path[draw=drawColor,line width= 0.4pt,line join=round,line cap=round,fill=fillColor] (447.62,230.55) circle (  1.16);

\path[draw=drawColor,line width= 0.4pt,line join=round,line cap=round,fill=fillColor] (447.81,230.54) circle (  1.16);

\path[draw=drawColor,line width= 0.4pt,line join=round,line cap=round,fill=fillColor] (448.00,230.51) circle (  1.16);

\path[draw=drawColor,line width= 0.4pt,line join=round,line cap=round,fill=fillColor] (448.19,230.47) circle (  1.16);

\path[draw=drawColor,line width= 0.4pt,line join=round,line cap=round,fill=fillColor] (448.37,230.46) circle (  1.16);

\path[draw=drawColor,line width= 0.4pt,line join=round,line cap=round,fill=fillColor] (448.56,230.45) circle (  1.16);

\path[draw=drawColor,line width= 0.4pt,line join=round,line cap=round,fill=fillColor] (448.75,230.37) circle (  1.16);

\path[draw=drawColor,line width= 0.4pt,line join=round,line cap=round,fill=fillColor] (448.93,230.37) circle (  1.16);

\path[draw=drawColor,line width= 0.4pt,line join=round,line cap=round,fill=fillColor] (449.12,230.34) circle (  1.16);

\path[draw=drawColor,line width= 0.4pt,line join=round,line cap=round,fill=fillColor] (449.30,230.32) circle (  1.16);

\path[draw=drawColor,line width= 0.4pt,line join=round,line cap=round,fill=fillColor] (449.49,230.28) circle (  1.16);

\path[draw=drawColor,line width= 0.4pt,line join=round,line cap=round,fill=fillColor] (449.67,230.25) circle (  1.16);

\path[draw=drawColor,line width= 0.4pt,line join=round,line cap=round,fill=fillColor] (449.86,230.24) circle (  1.16);

\path[draw=drawColor,line width= 0.4pt,line join=round,line cap=round,fill=fillColor] (450.04,230.20) circle (  1.16);

\path[draw=drawColor,line width= 0.4pt,line join=round,line cap=round,fill=fillColor] (450.22,230.14) circle (  1.16);

\path[draw=drawColor,line width= 0.4pt,line join=round,line cap=round,fill=fillColor] (450.40,230.08) circle (  1.16);

\path[draw=drawColor,line width= 0.4pt,line join=round,line cap=round,fill=fillColor] (450.58,230.04) circle (  1.16);

\path[draw=drawColor,line width= 0.4pt,line join=round,line cap=round,fill=fillColor] (450.77,230.01) circle (  1.16);

\path[draw=drawColor,line width= 0.4pt,line join=round,line cap=round,fill=fillColor] (450.95,229.86) circle (  1.16);

\path[draw=drawColor,line width= 0.4pt,line join=round,line cap=round,fill=fillColor] (451.13,229.83) circle (  1.16);

\path[draw=drawColor,line width= 0.4pt,line join=round,line cap=round,fill=fillColor] (451.31,229.82) circle (  1.16);

\path[draw=drawColor,line width= 0.4pt,line join=round,line cap=round,fill=fillColor] (451.49,229.79) circle (  1.16);

\path[draw=drawColor,line width= 0.4pt,line join=round,line cap=round,fill=fillColor] (451.67,229.79) circle (  1.16);

\path[draw=drawColor,line width= 0.4pt,line join=round,line cap=round,fill=fillColor] (451.84,229.72) circle (  1.16);

\path[draw=drawColor,line width= 0.4pt,line join=round,line cap=round,fill=fillColor] (452.02,229.60) circle (  1.16);

\path[draw=drawColor,line width= 0.4pt,line join=round,line cap=round,fill=fillColor] (452.20,229.59) circle (  1.16);

\path[draw=drawColor,line width= 0.4pt,line join=round,line cap=round,fill=fillColor] (452.38,229.57) circle (  1.16);

\path[draw=drawColor,line width= 0.4pt,line join=round,line cap=round,fill=fillColor] (452.55,229.55) circle (  1.16);

\path[draw=drawColor,line width= 0.4pt,line join=round,line cap=round,fill=fillColor] (452.73,229.52) circle (  1.16);

\path[draw=drawColor,line width= 0.4pt,line join=round,line cap=round,fill=fillColor] (452.91,229.51) circle (  1.16);

\path[draw=drawColor,line width= 0.4pt,line join=round,line cap=round,fill=fillColor] (453.08,229.48) circle (  1.16);

\path[draw=drawColor,line width= 0.4pt,line join=round,line cap=round,fill=fillColor] (453.26,229.36) circle (  1.16);

\path[draw=drawColor,line width= 0.4pt,line join=round,line cap=round,fill=fillColor] (453.43,229.36) circle (  1.16);

\path[draw=drawColor,line width= 0.4pt,line join=round,line cap=round,fill=fillColor] (453.61,229.36) circle (  1.16);

\path[draw=drawColor,line width= 0.4pt,line join=round,line cap=round,fill=fillColor] (453.78,229.23) circle (  1.16);

\path[draw=drawColor,line width= 0.4pt,line join=round,line cap=round,fill=fillColor] (453.95,229.20) circle (  1.16);

\path[draw=drawColor,line width= 0.4pt,line join=round,line cap=round,fill=fillColor] (454.13,229.01) circle (  1.16);

\path[draw=drawColor,line width= 0.4pt,line join=round,line cap=round,fill=fillColor] (454.30,228.99) circle (  1.16);

\path[draw=drawColor,line width= 0.4pt,line join=round,line cap=round,fill=fillColor] (454.47,228.97) circle (  1.16);

\path[draw=drawColor,line width= 0.4pt,line join=round,line cap=round,fill=fillColor] (454.64,228.96) circle (  1.16);

\path[draw=drawColor,line width= 0.4pt,line join=round,line cap=round,fill=fillColor] (454.82,228.92) circle (  1.16);

\path[draw=drawColor,line width= 0.4pt,line join=round,line cap=round,fill=fillColor] (454.99,228.88) circle (  1.16);

\path[draw=drawColor,line width= 0.4pt,line join=round,line cap=round,fill=fillColor] (455.16,228.87) circle (  1.16);

\path[draw=drawColor,line width= 0.4pt,line join=round,line cap=round,fill=fillColor] (455.33,228.86) circle (  1.16);

\path[draw=drawColor,line width= 0.4pt,line join=round,line cap=round,fill=fillColor] (455.50,228.78) circle (  1.16);

\path[draw=drawColor,line width= 0.4pt,line join=round,line cap=round,fill=fillColor] (455.67,228.77) circle (  1.16);

\path[draw=drawColor,line width= 0.4pt,line join=round,line cap=round,fill=fillColor] (455.84,228.77) circle (  1.16);

\path[draw=drawColor,line width= 0.4pt,line join=round,line cap=round,fill=fillColor] (456.00,228.75) circle (  1.16);

\path[draw=drawColor,line width= 0.4pt,line join=round,line cap=round,fill=fillColor] (456.17,228.67) circle (  1.16);

\path[draw=drawColor,line width= 0.4pt,line join=round,line cap=round,fill=fillColor] (456.34,228.66) circle (  1.16);

\path[draw=drawColor,line width= 0.4pt,line join=round,line cap=round,fill=fillColor] (456.51,228.64) circle (  1.16);

\path[draw=drawColor,line width= 0.4pt,line join=round,line cap=round,fill=fillColor] (456.68,228.62) circle (  1.16);

\path[draw=drawColor,line width= 0.4pt,line join=round,line cap=round,fill=fillColor] (456.84,228.53) circle (  1.16);

\path[draw=drawColor,line width= 0.4pt,line join=round,line cap=round,fill=fillColor] (457.01,228.50) circle (  1.16);

\path[draw=drawColor,line width= 0.4pt,line join=round,line cap=round,fill=fillColor] (457.18,228.45) circle (  1.16);

\path[draw=drawColor,line width= 0.4pt,line join=round,line cap=round,fill=fillColor] (457.34,228.38) circle (  1.16);

\path[draw=drawColor,line width= 0.4pt,line join=round,line cap=round,fill=fillColor] (457.51,228.37) circle (  1.16);

\path[draw=drawColor,line width= 0.4pt,line join=round,line cap=round,fill=fillColor] (457.67,228.27) circle (  1.16);

\path[draw=drawColor,line width= 0.4pt,line join=round,line cap=round,fill=fillColor] (457.84,228.26) circle (  1.16);

\path[draw=drawColor,line width= 0.4pt,line join=round,line cap=round,fill=fillColor] (458.00,228.23) circle (  1.16);

\path[draw=drawColor,line width= 0.4pt,line join=round,line cap=round,fill=fillColor] (458.17,228.21) circle (  1.16);

\path[draw=drawColor,line width= 0.4pt,line join=round,line cap=round,fill=fillColor] (458.33,228.16) circle (  1.16);

\path[draw=drawColor,line width= 0.4pt,line join=round,line cap=round,fill=fillColor] (458.49,228.15) circle (  1.16);

\path[draw=drawColor,line width= 0.4pt,line join=round,line cap=round,fill=fillColor] (458.66,228.09) circle (  1.16);

\path[draw=drawColor,line width= 0.4pt,line join=round,line cap=round,fill=fillColor] (458.82,228.03) circle (  1.16);

\path[draw=drawColor,line width= 0.4pt,line join=round,line cap=round,fill=fillColor] (458.98,227.92) circle (  1.16);

\path[draw=drawColor,line width= 0.4pt,line join=round,line cap=round,fill=fillColor] (459.14,227.89) circle (  1.16);

\path[draw=drawColor,line width= 0.4pt,line join=round,line cap=round,fill=fillColor] (459.30,227.88) circle (  1.16);

\path[draw=drawColor,line width= 0.4pt,line join=round,line cap=round,fill=fillColor] (459.47,227.77) circle (  1.16);

\path[draw=drawColor,line width= 0.4pt,line join=round,line cap=round,fill=fillColor] (459.63,227.67) circle (  1.16);

\path[draw=drawColor,line width= 0.4pt,line join=round,line cap=round,fill=fillColor] (459.79,227.59) circle (  1.16);

\path[draw=drawColor,line width= 0.4pt,line join=round,line cap=round,fill=fillColor] (459.95,227.56) circle (  1.16);

\path[draw=drawColor,line width= 0.4pt,line join=round,line cap=round,fill=fillColor] (460.11,227.52) circle (  1.16);

\path[draw=drawColor,line width= 0.4pt,line join=round,line cap=round,fill=fillColor] (460.27,227.48) circle (  1.16);

\path[draw=drawColor,line width= 0.4pt,line join=round,line cap=round,fill=fillColor] (460.43,227.29) circle (  1.16);

\path[draw=drawColor,line width= 0.4pt,line join=round,line cap=round,fill=fillColor] (460.59,227.27) circle (  1.16);

\path[draw=drawColor,line width= 0.4pt,line join=round,line cap=round,fill=fillColor] (460.74,227.26) circle (  1.16);

\path[draw=drawColor,line width= 0.4pt,line join=round,line cap=round,fill=fillColor] (460.90,227.25) circle (  1.16);

\path[draw=drawColor,line width= 0.4pt,line join=round,line cap=round,fill=fillColor] (461.06,227.25) circle (  1.16);

\path[draw=drawColor,line width= 0.4pt,line join=round,line cap=round,fill=fillColor] (461.22,227.20) circle (  1.16);

\path[draw=drawColor,line width= 0.4pt,line join=round,line cap=round,fill=fillColor] (461.38,227.17) circle (  1.16);

\path[draw=drawColor,line width= 0.4pt,line join=round,line cap=round,fill=fillColor] (461.53,227.16) circle (  1.16);

\path[draw=drawColor,line width= 0.4pt,line join=round,line cap=round,fill=fillColor] (461.69,227.07) circle (  1.16);

\path[draw=drawColor,line width= 0.4pt,line join=round,line cap=round,fill=fillColor] (461.85,227.06) circle (  1.16);

\path[draw=drawColor,line width= 0.4pt,line join=round,line cap=round,fill=fillColor] (462.00,227.04) circle (  1.16);

\path[draw=drawColor,line width= 0.4pt,line join=round,line cap=round,fill=fillColor] (462.16,226.82) circle (  1.16);

\path[draw=drawColor,line width= 0.4pt,line join=round,line cap=round,fill=fillColor] (462.31,226.74) circle (  1.16);

\path[draw=drawColor,line width= 0.4pt,line join=round,line cap=round,fill=fillColor] (462.47,226.67) circle (  1.16);

\path[draw=drawColor,line width= 0.4pt,line join=round,line cap=round,fill=fillColor] (462.62,226.60) circle (  1.16);

\path[draw=drawColor,line width= 0.4pt,line join=round,line cap=round,fill=fillColor] (462.78,226.58) circle (  1.16);

\path[draw=drawColor,line width= 0.4pt,line join=round,line cap=round,fill=fillColor] (462.93,226.52) circle (  1.16);

\path[draw=drawColor,line width= 0.4pt,line join=round,line cap=round,fill=fillColor] (463.09,226.46) circle (  1.16);

\path[draw=drawColor,line width= 0.4pt,line join=round,line cap=round,fill=fillColor] (463.24,226.40) circle (  1.16);

\path[draw=drawColor,line width= 0.4pt,line join=round,line cap=round,fill=fillColor] (463.39,226.38) circle (  1.16);

\path[draw=drawColor,line width= 0.4pt,line join=round,line cap=round,fill=fillColor] (463.55,226.38) circle (  1.16);

\path[draw=drawColor,line width= 0.4pt,line join=round,line cap=round,fill=fillColor] (463.70,226.36) circle (  1.16);

\path[draw=drawColor,line width= 0.4pt,line join=round,line cap=round,fill=fillColor] (463.85,226.35) circle (  1.16);

\path[draw=drawColor,line width= 0.4pt,line join=round,line cap=round,fill=fillColor] (464.00,226.26) circle (  1.16);

\path[draw=drawColor,line width= 0.4pt,line join=round,line cap=round,fill=fillColor] (464.16,226.20) circle (  1.16);

\path[draw=drawColor,line width= 0.4pt,line join=round,line cap=round,fill=fillColor] (464.31,226.15) circle (  1.16);

\path[draw=drawColor,line width= 0.4pt,line join=round,line cap=round,fill=fillColor] (464.46,226.04) circle (  1.16);

\path[draw=drawColor,line width= 0.4pt,line join=round,line cap=round,fill=fillColor] (464.61,226.02) circle (  1.16);

\path[draw=drawColor,line width= 0.4pt,line join=round,line cap=round,fill=fillColor] (464.76,225.99) circle (  1.16);

\path[draw=drawColor,line width= 0.4pt,line join=round,line cap=round,fill=fillColor] (464.91,225.91) circle (  1.16);

\path[draw=drawColor,line width= 0.4pt,line join=round,line cap=round,fill=fillColor] (465.06,225.68) circle (  1.16);

\path[draw=drawColor,line width= 0.4pt,line join=round,line cap=round,fill=fillColor] (465.21,225.66) circle (  1.16);

\path[draw=drawColor,line width= 0.4pt,line join=round,line cap=round,fill=fillColor] (465.36,225.63) circle (  1.16);

\path[draw=drawColor,line width= 0.4pt,line join=round,line cap=round,fill=fillColor] (465.51,225.48) circle (  1.16);

\path[draw=drawColor,line width= 0.4pt,line join=round,line cap=round,fill=fillColor] (465.66,225.34) circle (  1.16);

\path[draw=drawColor,line width= 0.4pt,line join=round,line cap=round,fill=fillColor] (465.81,225.29) circle (  1.16);

\path[draw=drawColor,line width= 0.4pt,line join=round,line cap=round,fill=fillColor] (465.96,225.17) circle (  1.16);

\path[draw=drawColor,line width= 0.4pt,line join=round,line cap=round,fill=fillColor] (466.10,225.15) circle (  1.16);

\path[draw=drawColor,line width= 0.4pt,line join=round,line cap=round,fill=fillColor] (466.25,225.14) circle (  1.16);

\path[draw=drawColor,line width= 0.4pt,line join=round,line cap=round,fill=fillColor] (466.40,225.13) circle (  1.16);

\path[draw=drawColor,line width= 0.4pt,line join=round,line cap=round,fill=fillColor] (466.55,225.10) circle (  1.16);

\path[draw=drawColor,line width= 0.4pt,line join=round,line cap=round,fill=fillColor] (466.70,225.07) circle (  1.16);

\path[draw=drawColor,line width= 0.4pt,line join=round,line cap=round,fill=fillColor] (466.84,225.05) circle (  1.16);

\path[draw=drawColor,line width= 0.4pt,line join=round,line cap=round,fill=fillColor] (466.99,225.04) circle (  1.16);

\path[draw=drawColor,line width= 0.4pt,line join=round,line cap=round,fill=fillColor] (467.14,225.01) circle (  1.16);

\path[draw=drawColor,line width= 0.4pt,line join=round,line cap=round,fill=fillColor] (467.28,225.00) circle (  1.16);

\path[draw=drawColor,line width= 0.4pt,line join=round,line cap=round,fill=fillColor] (467.43,224.95) circle (  1.16);

\path[draw=drawColor,line width= 0.4pt,line join=round,line cap=round,fill=fillColor] (467.57,224.93) circle (  1.16);

\path[draw=drawColor,line width= 0.4pt,line join=round,line cap=round,fill=fillColor] (467.72,224.91) circle (  1.16);

\path[draw=drawColor,line width= 0.4pt,line join=round,line cap=round,fill=fillColor] (467.86,224.90) circle (  1.16);

\path[draw=drawColor,line width= 0.4pt,line join=round,line cap=round,fill=fillColor] (468.01,224.62) circle (  1.16);

\path[draw=drawColor,line width= 0.4pt,line join=round,line cap=round,fill=fillColor] (468.15,224.61) circle (  1.16);

\path[draw=drawColor,line width= 0.4pt,line join=round,line cap=round,fill=fillColor] (468.30,224.58) circle (  1.16);

\path[draw=drawColor,line width= 0.4pt,line join=round,line cap=round,fill=fillColor] (468.44,224.53) circle (  1.16);

\path[draw=drawColor,line width= 0.4pt,line join=round,line cap=round,fill=fillColor] (468.58,224.48) circle (  1.16);

\path[draw=drawColor,line width= 0.4pt,line join=round,line cap=round,fill=fillColor] (468.73,224.38) circle (  1.16);

\path[draw=drawColor,line width= 0.4pt,line join=round,line cap=round,fill=fillColor] (468.87,224.17) circle (  1.16);

\path[draw=drawColor,line width= 0.4pt,line join=round,line cap=round,fill=fillColor] (469.01,223.98) circle (  1.16);

\path[draw=drawColor,line width= 0.4pt,line join=round,line cap=round,fill=fillColor] (469.16,223.83) circle (  1.16);

\path[draw=drawColor,line width= 0.4pt,line join=round,line cap=round,fill=fillColor] (469.30,223.73) circle (  1.16);

\path[draw=drawColor,line width= 0.4pt,line join=round,line cap=round,fill=fillColor] (469.44,223.58) circle (  1.16);

\path[draw=drawColor,line width= 0.4pt,line join=round,line cap=round,fill=fillColor] (469.58,223.51) circle (  1.16);

\path[draw=drawColor,line width= 0.4pt,line join=round,line cap=round,fill=fillColor] (469.73,223.47) circle (  1.16);

\path[draw=drawColor,line width= 0.4pt,line join=round,line cap=round,fill=fillColor] (469.87,223.38) circle (  1.16);

\path[draw=drawColor,line width= 0.4pt,line join=round,line cap=round,fill=fillColor] (470.01,223.33) circle (  1.16);

\path[draw=drawColor,line width= 0.4pt,line join=round,line cap=round,fill=fillColor] (470.15,223.21) circle (  1.16);

\path[draw=drawColor,line width= 0.4pt,line join=round,line cap=round,fill=fillColor] (470.29,223.08) circle (  1.16);

\path[draw=drawColor,line width= 0.4pt,line join=round,line cap=round,fill=fillColor] (470.43,223.00) circle (  1.16);

\path[draw=drawColor,line width= 0.4pt,line join=round,line cap=round,fill=fillColor] (470.57,222.99) circle (  1.16);

\path[draw=drawColor,line width= 0.4pt,line join=round,line cap=round,fill=fillColor] (470.71,222.93) circle (  1.16);

\path[draw=drawColor,line width= 0.4pt,line join=round,line cap=round,fill=fillColor] (470.85,222.72) circle (  1.16);

\path[draw=drawColor,line width= 0.4pt,line join=round,line cap=round,fill=fillColor] (470.99,222.55) circle (  1.16);

\path[draw=drawColor,line width= 0.4pt,line join=round,line cap=round,fill=fillColor] (471.13,222.32) circle (  1.16);

\path[draw=drawColor,line width= 0.4pt,line join=round,line cap=round,fill=fillColor] (471.27,222.27) circle (  1.16);

\path[draw=drawColor,line width= 0.4pt,line join=round,line cap=round,fill=fillColor] (471.41,221.99) circle (  1.16);

\path[draw=drawColor,line width= 0.4pt,line join=round,line cap=round,fill=fillColor] (471.55,221.50) circle (  1.16);

\path[draw=drawColor,line width= 0.4pt,line join=round,line cap=round,fill=fillColor] (471.69,221.32) circle (  1.16);

\path[draw=drawColor,line width= 0.4pt,line join=round,line cap=round,fill=fillColor] (471.82,220.57) circle (  1.16);

\path[draw=drawColor,line width= 0.4pt,line join=round,line cap=round,fill=fillColor] (471.96,220.54) circle (  1.16);

\path[draw=drawColor,line width= 0.4pt,line join=round,line cap=round,fill=fillColor] (472.10,220.52) circle (  1.16);

\path[draw=drawColor,line width= 0.4pt,line join=round,line cap=round,fill=fillColor] (472.24,220.49) circle (  1.16);

\path[draw=drawColor,line width= 0.4pt,line join=round,line cap=round,fill=fillColor] (472.37,220.20) circle (  1.16);

\path[draw=drawColor,line width= 0.4pt,line join=round,line cap=round,fill=fillColor] (472.51,220.06) circle (  1.16);

\path[draw=drawColor,line width= 0.4pt,line join=round,line cap=round,fill=fillColor] (472.65,219.47) circle (  1.16);

\path[draw=drawColor,line width= 0.4pt,line join=round,line cap=round,fill=fillColor] (472.78,219.18) circle (  1.16);

\path[draw=drawColor,line width= 0.4pt,line join=round,line cap=round,fill=fillColor] (472.92,219.06) circle (  1.16);

\path[draw=drawColor,line width= 0.4pt,line join=round,line cap=round,fill=fillColor] (473.06,218.28) circle (  1.16);

\path[draw=drawColor,line width= 0.4pt,line join=round,line cap=round,fill=fillColor] (473.19,218.17) circle (  1.16);

\path[draw=drawColor,line width= 0.4pt,line join=round,line cap=round,fill=fillColor] (473.33,216.48) circle (  1.16);

\path[draw=drawColor,line width= 0.4pt,line join=round,line cap=round,fill=fillColor] (473.46,216.32) circle (  1.16);

\path[draw=drawColor,line width= 0.4pt,line join=round,line cap=round,fill=fillColor] (473.60,216.02) circle (  1.16);

\path[draw=drawColor,line width= 0.4pt,line join=round,line cap=round,fill=fillColor] (473.74,214.41) circle (  1.16);

\path[draw=drawColor,line width= 0.4pt,line join=round,line cap=round,fill=fillColor] (473.87,209.81) circle (  1.16);

\path[draw=drawColor,line width= 0.4pt,line join=round,line cap=round,fill=fillColor] (474.00,209.81) circle (  1.16);

\path[draw=drawColor,line width= 0.4pt,line join=round,line cap=round,fill=fillColor] (474.14,209.81) circle (  1.16);

\path[draw=drawColor,line width= 0.4pt,line join=round,line cap=round,fill=fillColor] (474.27,209.81) circle (  1.16);

\path[draw=drawColor,line width= 0.4pt,line join=round,line cap=round,fill=fillColor] (474.41,209.81) circle (  1.16);

\path[draw=drawColor,line width= 0.4pt,line join=round,line cap=round,fill=fillColor] (474.54,209.81) circle (  1.16);

\path[draw=drawColor,line width= 0.4pt,line join=round,line cap=round,fill=fillColor] (474.68,209.81) circle (  1.16);

\path[draw=drawColor,line width= 0.4pt,line join=round,line cap=round,fill=fillColor] (474.81,209.81) circle (  1.16);

\path[draw=drawColor,line width= 0.4pt,line join=round,line cap=round,fill=fillColor] (474.94,209.81) circle (  1.16);

\path[draw=drawColor,line width= 0.4pt,line join=round,line cap=round,fill=fillColor] (475.07,209.81) circle (  1.16);

\path[draw=drawColor,line width= 0.4pt,line join=round,line cap=round,fill=fillColor] (475.21,209.81) circle (  1.16);

\path[draw=drawColor,line width= 0.4pt,line join=round,line cap=round,fill=fillColor] (475.34,209.81) circle (  1.16);

\path[draw=drawColor,line width= 0.4pt,line join=round,line cap=round,fill=fillColor] (475.47,209.81) circle (  1.16);

\path[draw=drawColor,line width= 0.4pt,line join=round,line cap=round,fill=fillColor] (475.60,209.81) circle (  1.16);

\path[draw=drawColor,line width= 0.4pt,line join=round,line cap=round,fill=fillColor] (475.74,209.81) circle (  1.16);

\path[draw=drawColor,line width= 0.4pt,line join=round,line cap=round,fill=fillColor] (475.87,209.81) circle (  1.16);

\path[draw=drawColor,line width= 0.4pt,line join=round,line cap=round,fill=fillColor] (476.00,209.81) circle (  1.16);

\path[draw=drawColor,line width= 0.4pt,line join=round,line cap=round,fill=fillColor] (476.13,209.81) circle (  1.16);

\path[draw=drawColor,line width= 0.4pt,line join=round,line cap=round,fill=fillColor] (476.26,209.81) circle (  1.16);

\path[draw=drawColor,line width= 0.4pt,line join=round,line cap=round,fill=fillColor] (476.39,209.81) circle (  1.16);

\path[draw=drawColor,line width= 0.4pt,line join=round,line cap=round,fill=fillColor] (476.52,209.81) circle (  1.16);

\path[draw=drawColor,line width= 0.4pt,line join=round,line cap=round,fill=fillColor] (476.65,209.81) circle (  1.16);

\path[draw=drawColor,line width= 0.4pt,line join=round,line cap=round,fill=fillColor] (476.78,209.81) circle (  1.16);

\path[draw=drawColor,line width= 0.4pt,line join=round,line cap=round,fill=fillColor] (476.91,209.81) circle (  1.16);

\path[draw=drawColor,line width= 0.4pt,line join=round,line cap=round,fill=fillColor] (477.04,209.81) circle (  1.16);

\path[draw=drawColor,line width= 0.4pt,line join=round,line cap=round,fill=fillColor] (477.17,209.81) circle (  1.16);

\path[draw=drawColor,line width= 0.4pt,line join=round,line cap=round,fill=fillColor] (477.30,209.81) circle (  1.16);

\path[draw=drawColor,line width= 0.4pt,line join=round,line cap=round,fill=fillColor] (477.43,209.81) circle (  1.16);

\path[draw=drawColor,line width= 0.4pt,line join=round,line cap=round,fill=fillColor] (477.56,209.81) circle (  1.16);

\path[draw=drawColor,line width= 0.4pt,line join=round,line cap=round,fill=fillColor] (477.69,209.81) circle (  1.16);

\path[draw=drawColor,line width= 0.4pt,line join=round,line cap=round,fill=fillColor] (477.82,209.81) circle (  1.16);

\path[draw=drawColor,line width= 0.4pt,line join=round,line cap=round,fill=fillColor] (477.95,209.81) circle (  1.16);

\path[draw=drawColor,line width= 0.4pt,line join=round,line cap=round,fill=fillColor] (478.08,209.81) circle (  1.16);

\path[draw=drawColor,line width= 0.4pt,line join=round,line cap=round,fill=fillColor] (478.21,209.81) circle (  1.16);
\definecolor[named]{drawColor}{rgb}{0.65,0.34,0.16}
\definecolor[named]{fillColor}{rgb}{0.65,0.34,0.16}

\path[draw=drawColor,line width= 0.4pt,line join=round,line cap=round,fill=fillColor] (327.82,263.63) circle (  1.16);

\path[draw=drawColor,line width= 0.4pt,line join=round,line cap=round,fill=fillColor] (333.69,262.28) circle (  1.16);

\path[draw=drawColor,line width= 0.4pt,line join=round,line cap=round,fill=fillColor] (337.81,262.11) circle (  1.16);

\path[draw=drawColor,line width= 0.4pt,line join=round,line cap=round,fill=fillColor] (341.08,258.13) circle (  1.16);

\path[draw=drawColor,line width= 0.4pt,line join=round,line cap=round,fill=fillColor] (343.85,258.10) circle (  1.16);

\path[draw=drawColor,line width= 0.4pt,line join=round,line cap=round,fill=fillColor] (346.27,256.29) circle (  1.16);

\path[draw=drawColor,line width= 0.4pt,line join=round,line cap=round,fill=fillColor] (348.43,254.15) circle (  1.16);

\path[draw=drawColor,line width= 0.4pt,line join=round,line cap=round,fill=fillColor] (350.39,248.42) circle (  1.16);

\path[draw=drawColor,line width= 0.4pt,line join=round,line cap=round,fill=fillColor] (352.20,247.89) circle (  1.16);

\path[draw=drawColor,line width= 0.4pt,line join=round,line cap=round,fill=fillColor] (353.88,244.95) circle (  1.16);

\path[draw=drawColor,line width= 0.4pt,line join=round,line cap=round,fill=fillColor] (355.45,244.93) circle (  1.16);

\path[draw=drawColor,line width= 0.4pt,line join=round,line cap=round,fill=fillColor] (356.93,244.83) circle (  1.16);

\path[draw=drawColor,line width= 0.4pt,line join=round,line cap=round,fill=fillColor] (358.32,244.39) circle (  1.16);

\path[draw=drawColor,line width= 0.4pt,line join=round,line cap=round,fill=fillColor] (359.65,243.76) circle (  1.16);

\path[draw=drawColor,line width= 0.4pt,line join=round,line cap=round,fill=fillColor] (360.92,243.46) circle (  1.16);

\path[draw=drawColor,line width= 0.4pt,line join=round,line cap=round,fill=fillColor] (362.13,242.55) circle (  1.16);

\path[draw=drawColor,line width= 0.4pt,line join=round,line cap=round,fill=fillColor] (363.29,241.96) circle (  1.16);

\path[draw=drawColor,line width= 0.4pt,line join=round,line cap=round,fill=fillColor] (364.40,240.70) circle (  1.16);

\path[draw=drawColor,line width= 0.4pt,line join=round,line cap=round,fill=fillColor] (365.48,239.48) circle (  1.16);

\path[draw=drawColor,line width= 0.4pt,line join=round,line cap=round,fill=fillColor] (366.52,239.09) circle (  1.16);

\path[draw=drawColor,line width= 0.4pt,line join=round,line cap=round,fill=fillColor] (367.52,239.08) circle (  1.16);

\path[draw=drawColor,line width= 0.4pt,line join=round,line cap=round,fill=fillColor] (368.49,237.76) circle (  1.16);

\path[draw=drawColor,line width= 0.4pt,line join=round,line cap=round,fill=fillColor] (369.44,236.09) circle (  1.16);

\path[draw=drawColor,line width= 0.4pt,line join=round,line cap=round,fill=fillColor] (370.36,235.60) circle (  1.16);

\path[draw=drawColor,line width= 0.4pt,line join=round,line cap=round,fill=fillColor] (371.25,235.58) circle (  1.16);

\path[draw=drawColor,line width= 0.4pt,line join=round,line cap=round,fill=fillColor] (372.12,235.49) circle (  1.16);

\path[draw=drawColor,line width= 0.4pt,line join=round,line cap=round,fill=fillColor] (372.96,235.45) circle (  1.16);

\path[draw=drawColor,line width= 0.4pt,line join=round,line cap=round,fill=fillColor] (373.79,235.23) circle (  1.16);

\path[draw=drawColor,line width= 0.4pt,line join=round,line cap=round,fill=fillColor] (374.59,235.03) circle (  1.16);

\path[draw=drawColor,line width= 0.4pt,line join=round,line cap=round,fill=fillColor] (375.38,235.02) circle (  1.16);

\path[draw=drawColor,line width= 0.4pt,line join=round,line cap=round,fill=fillColor] (376.15,235.01) circle (  1.16);

\path[draw=drawColor,line width= 0.4pt,line join=round,line cap=round,fill=fillColor] (376.91,234.83) circle (  1.16);

\path[draw=drawColor,line width= 0.4pt,line join=round,line cap=round,fill=fillColor] (377.65,234.68) circle (  1.16);

\path[draw=drawColor,line width= 0.4pt,line join=round,line cap=round,fill=fillColor] (378.37,234.49) circle (  1.16);

\path[draw=drawColor,line width= 0.4pt,line join=round,line cap=round,fill=fillColor] (379.08,234.44) circle (  1.16);

\path[draw=drawColor,line width= 0.4pt,line join=round,line cap=round,fill=fillColor] (379.78,233.81) circle (  1.16);

\path[draw=drawColor,line width= 0.4pt,line join=round,line cap=round,fill=fillColor] (380.46,233.70) circle (  1.16);

\path[draw=drawColor,line width= 0.4pt,line join=round,line cap=round,fill=fillColor] (381.13,233.17) circle (  1.16);

\path[draw=drawColor,line width= 0.4pt,line join=round,line cap=round,fill=fillColor] (381.79,233.11) circle (  1.16);

\path[draw=drawColor,line width= 0.4pt,line join=round,line cap=round,fill=fillColor] (382.44,232.99) circle (  1.16);

\path[draw=drawColor,line width= 0.4pt,line join=round,line cap=round,fill=fillColor] (383.08,232.92) circle (  1.16);

\path[draw=drawColor,line width= 0.4pt,line join=round,line cap=round,fill=fillColor] (383.71,232.77) circle (  1.16);

\path[draw=drawColor,line width= 0.4pt,line join=round,line cap=round,fill=fillColor] (384.32,232.70) circle (  1.16);

\path[draw=drawColor,line width= 0.4pt,line join=round,line cap=round,fill=fillColor] (384.93,232.62) circle (  1.16);

\path[draw=drawColor,line width= 0.4pt,line join=round,line cap=round,fill=fillColor] (385.53,232.62) circle (  1.16);

\path[draw=drawColor,line width= 0.4pt,line join=round,line cap=round,fill=fillColor] (386.12,232.38) circle (  1.16);

\path[draw=drawColor,line width= 0.4pt,line join=round,line cap=round,fill=fillColor] (386.70,232.38) circle (  1.16);

\path[draw=drawColor,line width= 0.4pt,line join=round,line cap=round,fill=fillColor] (387.28,232.30) circle (  1.16);

\path[draw=drawColor,line width= 0.4pt,line join=round,line cap=round,fill=fillColor] (387.84,232.19) circle (  1.16);

\path[draw=drawColor,line width= 0.4pt,line join=round,line cap=round,fill=fillColor] (388.40,232.15) circle (  1.16);

\path[draw=drawColor,line width= 0.4pt,line join=round,line cap=round,fill=fillColor] (388.95,232.12) circle (  1.16);

\path[draw=drawColor,line width= 0.4pt,line join=round,line cap=round,fill=fillColor] (389.50,232.11) circle (  1.16);

\path[draw=drawColor,line width= 0.4pt,line join=round,line cap=round,fill=fillColor] (390.03,232.04) circle (  1.16);

\path[draw=drawColor,line width= 0.4pt,line join=round,line cap=round,fill=fillColor] (390.56,232.01) circle (  1.16);

\path[draw=drawColor,line width= 0.4pt,line join=round,line cap=round,fill=fillColor] (391.08,231.78) circle (  1.16);

\path[draw=drawColor,line width= 0.4pt,line join=round,line cap=round,fill=fillColor] (391.60,231.47) circle (  1.16);

\path[draw=drawColor,line width= 0.4pt,line join=round,line cap=round,fill=fillColor] (392.11,231.22) circle (  1.16);

\path[draw=drawColor,line width= 0.4pt,line join=round,line cap=round,fill=fillColor] (392.62,230.95) circle (  1.16);

\path[draw=drawColor,line width= 0.4pt,line join=round,line cap=round,fill=fillColor] (393.12,230.92) circle (  1.16);

\path[draw=drawColor,line width= 0.4pt,line join=round,line cap=round,fill=fillColor] (393.61,230.66) circle (  1.16);

\path[draw=drawColor,line width= 0.4pt,line join=round,line cap=round,fill=fillColor] (394.10,230.59) circle (  1.16);

\path[draw=drawColor,line width= 0.4pt,line join=round,line cap=round,fill=fillColor] (394.58,230.14) circle (  1.16);

\path[draw=drawColor,line width= 0.4pt,line join=round,line cap=round,fill=fillColor] (395.06,230.09) circle (  1.16);

\path[draw=drawColor,line width= 0.4pt,line join=round,line cap=round,fill=fillColor] (395.53,229.54) circle (  1.16);

\path[draw=drawColor,line width= 0.4pt,line join=round,line cap=round,fill=fillColor] (396.00,229.32) circle (  1.16);

\path[draw=drawColor,line width= 0.4pt,line join=round,line cap=round,fill=fillColor] (396.46,228.90) circle (  1.16);

\path[draw=drawColor,line width= 0.4pt,line join=round,line cap=round,fill=fillColor] (396.92,228.64) circle (  1.16);

\path[draw=drawColor,line width= 0.4pt,line join=round,line cap=round,fill=fillColor] (397.37,228.54) circle (  1.16);

\path[draw=drawColor,line width= 0.4pt,line join=round,line cap=round,fill=fillColor] (397.82,228.28) circle (  1.16);

\path[draw=drawColor,line width= 0.4pt,line join=round,line cap=round,fill=fillColor] (398.27,228.09) circle (  1.16);

\path[draw=drawColor,line width= 0.4pt,line join=round,line cap=round,fill=fillColor] (398.71,227.85) circle (  1.16);

\path[draw=drawColor,line width= 0.4pt,line join=round,line cap=round,fill=fillColor] (399.15,227.84) circle (  1.16);

\path[draw=drawColor,line width= 0.4pt,line join=round,line cap=round,fill=fillColor] (399.58,227.82) circle (  1.16);

\path[draw=drawColor,line width= 0.4pt,line join=round,line cap=round,fill=fillColor] (400.01,227.51) circle (  1.16);

\path[draw=drawColor,line width= 0.4pt,line join=round,line cap=round,fill=fillColor] (400.43,227.39) circle (  1.16);

\path[draw=drawColor,line width= 0.4pt,line join=round,line cap=round,fill=fillColor] (400.85,227.35) circle (  1.16);

\path[draw=drawColor,line width= 0.4pt,line join=round,line cap=round,fill=fillColor] (401.27,227.27) circle (  1.16);

\path[draw=drawColor,line width= 0.4pt,line join=round,line cap=round,fill=fillColor] (401.69,226.86) circle (  1.16);

\path[draw=drawColor,line width= 0.4pt,line join=round,line cap=round,fill=fillColor] (402.10,226.85) circle (  1.16);

\path[draw=drawColor,line width= 0.4pt,line join=round,line cap=round,fill=fillColor] (402.50,226.38) circle (  1.16);

\path[draw=drawColor,line width= 0.4pt,line join=round,line cap=round,fill=fillColor] (402.91,226.33) circle (  1.16);

\path[draw=drawColor,line width= 0.4pt,line join=round,line cap=round,fill=fillColor] (403.31,226.30) circle (  1.16);

\path[draw=drawColor,line width= 0.4pt,line join=round,line cap=round,fill=fillColor] (403.70,225.63) circle (  1.16);

\path[draw=drawColor,line width= 0.4pt,line join=round,line cap=round,fill=fillColor] (404.10,225.17) circle (  1.16);

\path[draw=drawColor,line width= 0.4pt,line join=round,line cap=round,fill=fillColor] (404.49,225.13) circle (  1.16);

\path[draw=drawColor,line width= 0.4pt,line join=round,line cap=round,fill=fillColor] (404.88,225.07) circle (  1.16);

\path[draw=drawColor,line width= 0.4pt,line join=round,line cap=round,fill=fillColor] (405.26,224.87) circle (  1.16);

\path[draw=drawColor,line width= 0.4pt,line join=round,line cap=round,fill=fillColor] (405.64,224.57) circle (  1.16);

\path[draw=drawColor,line width= 0.4pt,line join=round,line cap=round,fill=fillColor] (406.02,224.22) circle (  1.16);

\path[draw=drawColor,line width= 0.4pt,line join=round,line cap=round,fill=fillColor] (406.40,223.62) circle (  1.16);

\path[draw=drawColor,line width= 0.4pt,line join=round,line cap=round,fill=fillColor] (406.77,223.53) circle (  1.16);

\path[draw=drawColor,line width= 0.4pt,line join=round,line cap=round,fill=fillColor] (407.14,223.44) circle (  1.16);

\path[draw=drawColor,line width= 0.4pt,line join=round,line cap=round,fill=fillColor] (407.51,223.33) circle (  1.16);

\path[draw=drawColor,line width= 0.4pt,line join=round,line cap=round,fill=fillColor] (407.87,223.14) circle (  1.16);

\path[draw=drawColor,line width= 0.4pt,line join=round,line cap=round,fill=fillColor] (408.24,222.99) circle (  1.16);

\path[draw=drawColor,line width= 0.4pt,line join=round,line cap=round,fill=fillColor] (408.60,222.64) circle (  1.16);

\path[draw=drawColor,line width= 0.4pt,line join=round,line cap=round,fill=fillColor] (408.95,222.31) circle (  1.16);

\path[draw=drawColor,line width= 0.4pt,line join=round,line cap=round,fill=fillColor] (409.31,222.23) circle (  1.16);

\path[draw=drawColor,line width= 0.4pt,line join=round,line cap=round,fill=fillColor] (409.66,221.36) circle (  1.16);

\path[draw=drawColor,line width= 0.4pt,line join=round,line cap=round,fill=fillColor] (410.01,219.96) circle (  1.16);

\path[draw=drawColor,line width= 0.4pt,line join=round,line cap=round,fill=fillColor] (410.36,219.81) circle (  1.16);

\path[draw=drawColor,line width= 0.4pt,line join=round,line cap=round,fill=fillColor] (410.71,219.14) circle (  1.16);

\path[draw=drawColor,line width= 0.4pt,line join=round,line cap=round,fill=fillColor] (411.05,219.13) circle (  1.16);

\path[draw=drawColor,line width= 0.4pt,line join=round,line cap=round,fill=fillColor] (411.39,219.13) circle (  1.16);

\path[draw=drawColor,line width= 0.4pt,line join=round,line cap=round,fill=fillColor] (411.73,218.82) circle (  1.16);

\path[draw=drawColor,line width= 0.4pt,line join=round,line cap=round,fill=fillColor] (412.07,218.65) circle (  1.16);

\path[draw=drawColor,line width= 0.4pt,line join=round,line cap=round,fill=fillColor] (412.40,218.38) circle (  1.16);

\path[draw=drawColor,line width= 0.4pt,line join=round,line cap=round,fill=fillColor] (412.73,218.35) circle (  1.16);

\path[draw=drawColor,line width= 0.4pt,line join=round,line cap=round,fill=fillColor] (413.07,217.20) circle (  1.16);

\path[draw=drawColor,line width= 0.4pt,line join=round,line cap=round,fill=fillColor] (413.39,216.80) circle (  1.16);

\path[draw=drawColor,line width= 0.4pt,line join=round,line cap=round,fill=fillColor] (413.72,216.41) circle (  1.16);

\path[draw=drawColor,line width= 0.4pt,line join=round,line cap=round,fill=fillColor] (414.05,216.32) circle (  1.16);

\path[draw=drawColor,line width= 0.4pt,line join=round,line cap=round,fill=fillColor] (414.37,216.17) circle (  1.16);

\path[draw=drawColor,line width= 0.4pt,line join=round,line cap=round,fill=fillColor] (414.69,215.26) circle (  1.16);

\path[draw=drawColor,line width= 0.4pt,line join=round,line cap=round,fill=fillColor] (415.01,214.41) circle (  1.16);

\path[draw=drawColor,line width= 0.4pt,line join=round,line cap=round,fill=fillColor] (415.33,213.34) circle (  1.16);

\path[draw=drawColor,line width= 0.4pt,line join=round,line cap=round,fill=fillColor] (415.64,212.75) circle (  1.16);

\path[draw=drawColor,line width= 0.4pt,line join=round,line cap=round,fill=fillColor] (415.95,209.81) circle (  1.16);

\path[draw=drawColor,line width= 0.4pt,line join=round,line cap=round,fill=fillColor] (416.27,209.81) circle (  1.16);

\path[draw=drawColor,line width= 0.4pt,line join=round,line cap=round,fill=fillColor] (416.58,209.81) circle (  1.16);

\path[draw=drawColor,line width= 0.4pt,line join=round,line cap=round,fill=fillColor] (416.88,209.81) circle (  1.16);

\path[draw=drawColor,line width= 0.4pt,line join=round,line cap=round,fill=fillColor] (417.19,209.81) circle (  1.16);

\path[draw=drawColor,line width= 0.4pt,line join=round,line cap=round,fill=fillColor] (417.50,209.81) circle (  1.16);

\path[draw=drawColor,line width= 0.4pt,line join=round,line cap=round,fill=fillColor] (417.80,209.81) circle (  1.16);

\path[draw=drawColor,line width= 0.4pt,line join=round,line cap=round,fill=fillColor] (418.10,209.81) circle (  1.16);

\path[draw=drawColor,line width= 0.4pt,line join=round,line cap=round,fill=fillColor] (418.40,209.81) circle (  1.16);

\path[draw=drawColor,line width= 0.4pt,line join=round,line cap=round,fill=fillColor] (418.70,209.81) circle (  1.16);

\path[draw=drawColor,line width= 0.4pt,line join=round,line cap=round,fill=fillColor] (419.00,209.81) circle (  1.16);

\path[draw=drawColor,line width= 0.4pt,line join=round,line cap=round,fill=fillColor] (419.29,209.81) circle (  1.16);

\path[draw=drawColor,line width= 0.4pt,line join=round,line cap=round,fill=fillColor] (419.59,209.81) circle (  1.16);

\path[draw=drawColor,line width= 0.4pt,line join=round,line cap=round,fill=fillColor] (419.88,209.81) circle (  1.16);

\path[draw=drawColor,line width= 0.4pt,line join=round,line cap=round,fill=fillColor] (420.17,209.81) circle (  1.16);

\path[draw=drawColor,line width= 0.4pt,line join=round,line cap=round,fill=fillColor] (420.46,209.81) circle (  1.16);

\path[draw=drawColor,line width= 0.4pt,line join=round,line cap=round,fill=fillColor] (420.75,209.81) circle (  1.16);

\path[draw=drawColor,line width= 0.4pt,line join=round,line cap=round,fill=fillColor] (421.03,209.81) circle (  1.16);

\path[draw=drawColor,line width= 0.4pt,line join=round,line cap=round,fill=fillColor] (421.32,209.81) circle (  1.16);

\path[draw=drawColor,line width= 0.4pt,line join=round,line cap=round,fill=fillColor] (421.60,209.81) circle (  1.16);

\path[draw=drawColor,line width= 0.4pt,line join=round,line cap=round,fill=fillColor] (421.89,209.81) circle (  1.16);

\path[draw=drawColor,line width= 0.4pt,line join=round,line cap=round,fill=fillColor] (422.17,209.81) circle (  1.16);

\path[draw=drawColor,line width= 0.4pt,line join=round,line cap=round,fill=fillColor] (422.45,209.81) circle (  1.16);

\path[draw=drawColor,line width= 0.4pt,line join=round,line cap=round,fill=fillColor] (422.72,209.81) circle (  1.16);

\path[draw=drawColor,line width= 0.4pt,line join=round,line cap=round,fill=fillColor] (423.00,209.81) circle (  1.16);

\path[draw=drawColor,line width= 0.4pt,line join=round,line cap=round,fill=fillColor] (423.28,209.81) circle (  1.16);

\path[draw=drawColor,line width= 0.4pt,line join=round,line cap=round,fill=fillColor] (423.55,209.81) circle (  1.16);

\path[draw=drawColor,line width= 0.4pt,line join=round,line cap=round,fill=fillColor] (423.82,209.81) circle (  1.16);

\path[draw=drawColor,line width= 0.4pt,line join=round,line cap=round,fill=fillColor] (424.10,209.81) circle (  1.16);

\path[draw=drawColor,line width= 0.4pt,line join=round,line cap=round,fill=fillColor] (424.37,209.81) circle (  1.16);

\path[draw=drawColor,line width= 0.4pt,line join=round,line cap=round,fill=fillColor] (424.64,209.81) circle (  1.16);

\path[draw=drawColor,line width= 0.4pt,line join=round,line cap=round,fill=fillColor] (424.91,209.81) circle (  1.16);

\path[draw=drawColor,line width= 0.4pt,line join=round,line cap=round,fill=fillColor] (425.17,209.81) circle (  1.16);

\path[draw=drawColor,line width= 0.4pt,line join=round,line cap=round,fill=fillColor] (425.44,209.81) circle (  1.16);

\path[draw=drawColor,line width= 0.4pt,line join=round,line cap=round,fill=fillColor] (425.70,209.81) circle (  1.16);

\path[draw=drawColor,line width= 0.4pt,line join=round,line cap=round,fill=fillColor] (425.97,209.81) circle (  1.16);

\path[draw=drawColor,line width= 0.4pt,line join=round,line cap=round,fill=fillColor] (426.23,209.81) circle (  1.16);

\path[draw=drawColor,line width= 0.4pt,line join=round,line cap=round,fill=fillColor] (426.49,209.81) circle (  1.16);

\path[draw=drawColor,line width= 0.4pt,line join=round,line cap=round,fill=fillColor] (426.75,209.81) circle (  1.16);

\path[draw=drawColor,line width= 0.4pt,line join=round,line cap=round,fill=fillColor] (427.01,209.81) circle (  1.16);

\path[draw=drawColor,line width= 0.4pt,line join=round,line cap=round,fill=fillColor] (427.27,209.81) circle (  1.16);

\path[draw=drawColor,line width= 0.4pt,line join=round,line cap=round,fill=fillColor] (427.52,209.81) circle (  1.16);

\path[draw=drawColor,line width= 0.4pt,line join=round,line cap=round,fill=fillColor] (427.78,209.81) circle (  1.16);

\path[draw=drawColor,line width= 0.4pt,line join=round,line cap=round,fill=fillColor] (428.03,209.81) circle (  1.16);

\path[draw=drawColor,line width= 0.4pt,line join=round,line cap=round,fill=fillColor] (428.29,209.81) circle (  1.16);

\path[draw=drawColor,line width= 0.4pt,line join=round,line cap=round,fill=fillColor] (428.54,209.81) circle (  1.16);

\path[draw=drawColor,line width= 0.4pt,line join=round,line cap=round,fill=fillColor] (428.79,209.81) circle (  1.16);

\path[draw=drawColor,line width= 0.4pt,line join=round,line cap=round,fill=fillColor] (429.04,209.81) circle (  1.16);

\path[draw=drawColor,line width= 0.4pt,line join=round,line cap=round,fill=fillColor] (429.29,209.81) circle (  1.16);

\path[draw=drawColor,line width= 0.4pt,line join=round,line cap=round,fill=fillColor] (429.54,209.81) circle (  1.16);

\path[draw=drawColor,line width= 0.4pt,line join=round,line cap=round,fill=fillColor] (429.79,209.81) circle (  1.16);

\path[draw=drawColor,line width= 0.4pt,line join=round,line cap=round,fill=fillColor] (430.04,209.81) circle (  1.16);

\path[draw=drawColor,line width= 0.4pt,line join=round,line cap=round,fill=fillColor] (430.28,209.81) circle (  1.16);

\path[draw=drawColor,line width= 0.4pt,line join=round,line cap=round,fill=fillColor] (430.53,209.81) circle (  1.16);

\path[draw=drawColor,line width= 0.4pt,line join=round,line cap=round,fill=fillColor] (430.77,209.81) circle (  1.16);

\path[draw=drawColor,line width= 0.4pt,line join=round,line cap=round,fill=fillColor] (431.01,209.81) circle (  1.16);

\path[draw=drawColor,line width= 0.4pt,line join=round,line cap=round,fill=fillColor] (431.25,209.81) circle (  1.16);

\path[draw=drawColor,line width= 0.4pt,line join=round,line cap=round,fill=fillColor] (431.50,209.81) circle (  1.16);

\path[draw=drawColor,line width= 0.4pt,line join=round,line cap=round,fill=fillColor] (431.74,209.81) circle (  1.16);

\path[draw=drawColor,line width= 0.4pt,line join=round,line cap=round,fill=fillColor] (431.97,209.81) circle (  1.16);

\path[draw=drawColor,line width= 0.4pt,line join=round,line cap=round,fill=fillColor] (432.21,209.81) circle (  1.16);

\path[draw=drawColor,line width= 0.4pt,line join=round,line cap=round,fill=fillColor] (432.45,209.81) circle (  1.16);

\path[draw=drawColor,line width= 0.4pt,line join=round,line cap=round,fill=fillColor] (432.69,209.81) circle (  1.16);

\path[draw=drawColor,line width= 0.4pt,line join=round,line cap=round,fill=fillColor] (432.92,209.81) circle (  1.16);

\path[draw=drawColor,line width= 0.4pt,line join=round,line cap=round,fill=fillColor] (433.16,209.81) circle (  1.16);

\path[draw=drawColor,line width= 0.4pt,line join=round,line cap=round,fill=fillColor] (433.39,209.81) circle (  1.16);

\path[draw=drawColor,line width= 0.4pt,line join=round,line cap=round,fill=fillColor] (433.62,209.81) circle (  1.16);

\path[draw=drawColor,line width= 0.4pt,line join=round,line cap=round,fill=fillColor] (433.86,209.81) circle (  1.16);

\path[draw=drawColor,line width= 0.4pt,line join=round,line cap=round,fill=fillColor] (434.09,209.81) circle (  1.16);

\path[draw=drawColor,line width= 0.4pt,line join=round,line cap=round,fill=fillColor] (434.32,209.81) circle (  1.16);

\path[draw=drawColor,line width= 0.4pt,line join=round,line cap=round,fill=fillColor] (434.55,209.81) circle (  1.16);

\path[draw=drawColor,line width= 0.4pt,line join=round,line cap=round,fill=fillColor] (434.78,209.81) circle (  1.16);

\path[draw=drawColor,line width= 0.4pt,line join=round,line cap=round,fill=fillColor] (435.00,209.81) circle (  1.16);

\path[draw=drawColor,line width= 0.4pt,line join=round,line cap=round,fill=fillColor] (435.23,209.81) circle (  1.16);

\path[draw=drawColor,line width= 0.4pt,line join=round,line cap=round,fill=fillColor] (435.46,209.81) circle (  1.16);

\path[draw=drawColor,line width= 0.4pt,line join=round,line cap=round,fill=fillColor] (435.68,209.81) circle (  1.16);

\path[draw=drawColor,line width= 0.4pt,line join=round,line cap=round,fill=fillColor] (435.91,209.81) circle (  1.16);

\path[draw=drawColor,line width= 0.4pt,line join=round,line cap=round,fill=fillColor] (436.13,209.81) circle (  1.16);

\path[draw=drawColor,line width= 0.4pt,line join=round,line cap=round,fill=fillColor] (436.36,209.81) circle (  1.16);

\path[draw=drawColor,line width= 0.4pt,line join=round,line cap=round,fill=fillColor] (436.58,209.81) circle (  1.16);

\path[draw=drawColor,line width= 0.4pt,line join=round,line cap=round,fill=fillColor] (436.80,209.81) circle (  1.16);

\path[draw=drawColor,line width= 0.4pt,line join=round,line cap=round,fill=fillColor] (437.02,209.81) circle (  1.16);

\path[draw=drawColor,line width= 0.4pt,line join=round,line cap=round,fill=fillColor] (437.24,209.81) circle (  1.16);

\path[draw=drawColor,line width= 0.4pt,line join=round,line cap=round,fill=fillColor] (437.46,209.81) circle (  1.16);

\path[draw=drawColor,line width= 0.4pt,line join=round,line cap=round,fill=fillColor] (437.68,209.81) circle (  1.16);

\path[draw=drawColor,line width= 0.4pt,line join=round,line cap=round,fill=fillColor] (437.90,209.81) circle (  1.16);

\path[draw=drawColor,line width= 0.4pt,line join=round,line cap=round,fill=fillColor] (438.12,209.81) circle (  1.16);

\path[draw=drawColor,line width= 0.4pt,line join=round,line cap=round,fill=fillColor] (438.33,209.81) circle (  1.16);

\path[draw=drawColor,line width= 0.4pt,line join=round,line cap=round,fill=fillColor] (438.55,209.81) circle (  1.16);

\path[draw=drawColor,line width= 0.4pt,line join=round,line cap=round,fill=fillColor] (438.76,209.81) circle (  1.16);

\path[draw=drawColor,line width= 0.4pt,line join=round,line cap=round,fill=fillColor] (438.98,209.81) circle (  1.16);

\path[draw=drawColor,line width= 0.4pt,line join=round,line cap=round,fill=fillColor] (439.19,209.81) circle (  1.16);

\path[draw=drawColor,line width= 0.4pt,line join=round,line cap=round,fill=fillColor] (439.41,209.81) circle (  1.16);

\path[draw=drawColor,line width= 0.4pt,line join=round,line cap=round,fill=fillColor] (439.62,209.81) circle (  1.16);

\path[draw=drawColor,line width= 0.4pt,line join=round,line cap=round,fill=fillColor] (439.83,209.81) circle (  1.16);

\path[draw=drawColor,line width= 0.4pt,line join=round,line cap=round,fill=fillColor] (440.04,209.81) circle (  1.16);

\path[draw=drawColor,line width= 0.4pt,line join=round,line cap=round,fill=fillColor] (440.25,209.81) circle (  1.16);

\path[draw=drawColor,line width= 0.4pt,line join=round,line cap=round,fill=fillColor] (440.46,209.81) circle (  1.16);

\path[draw=drawColor,line width= 0.4pt,line join=round,line cap=round,fill=fillColor] (440.67,209.81) circle (  1.16);

\path[draw=drawColor,line width= 0.4pt,line join=round,line cap=round,fill=fillColor] (440.88,209.81) circle (  1.16);

\path[draw=drawColor,line width= 0.4pt,line join=round,line cap=round,fill=fillColor] (441.09,209.81) circle (  1.16);

\path[draw=drawColor,line width= 0.4pt,line join=round,line cap=round,fill=fillColor] (441.29,209.81) circle (  1.16);

\path[draw=drawColor,line width= 0.4pt,line join=round,line cap=round,fill=fillColor] (441.50,209.81) circle (  1.16);

\path[draw=drawColor,line width= 0.4pt,line join=round,line cap=round,fill=fillColor] (441.71,209.81) circle (  1.16);

\path[draw=drawColor,line width= 0.4pt,line join=round,line cap=round,fill=fillColor] (441.91,209.81) circle (  1.16);

\path[draw=drawColor,line width= 0.4pt,line join=round,line cap=round,fill=fillColor] (442.12,209.81) circle (  1.16);

\path[draw=drawColor,line width= 0.4pt,line join=round,line cap=round,fill=fillColor] (442.32,209.81) circle (  1.16);

\path[draw=drawColor,line width= 0.4pt,line join=round,line cap=round,fill=fillColor] (442.53,209.81) circle (  1.16);

\path[draw=drawColor,line width= 0.4pt,line join=round,line cap=round,fill=fillColor] (442.73,209.81) circle (  1.16);

\path[draw=drawColor,line width= 0.4pt,line join=round,line cap=round,fill=fillColor] (442.93,209.81) circle (  1.16);

\path[draw=drawColor,line width= 0.4pt,line join=round,line cap=round,fill=fillColor] (443.13,209.81) circle (  1.16);

\path[draw=drawColor,line width= 0.4pt,line join=round,line cap=round,fill=fillColor] (443.33,209.81) circle (  1.16);

\path[draw=drawColor,line width= 0.4pt,line join=round,line cap=round,fill=fillColor] (443.54,209.81) circle (  1.16);

\path[draw=drawColor,line width= 0.4pt,line join=round,line cap=round,fill=fillColor] (443.74,209.81) circle (  1.16);

\path[draw=drawColor,line width= 0.4pt,line join=round,line cap=round,fill=fillColor] (443.94,209.81) circle (  1.16);

\path[draw=drawColor,line width= 0.4pt,line join=round,line cap=round,fill=fillColor] (444.13,209.81) circle (  1.16);

\path[draw=drawColor,line width= 0.4pt,line join=round,line cap=round,fill=fillColor] (444.33,209.81) circle (  1.16);

\path[draw=drawColor,line width= 0.4pt,line join=round,line cap=round,fill=fillColor] (444.53,209.81) circle (  1.16);

\path[draw=drawColor,line width= 0.4pt,line join=round,line cap=round,fill=fillColor] (444.73,209.81) circle (  1.16);

\path[draw=drawColor,line width= 0.4pt,line join=round,line cap=round,fill=fillColor] (444.92,209.81) circle (  1.16);

\path[draw=drawColor,line width= 0.4pt,line join=round,line cap=round,fill=fillColor] (445.12,209.81) circle (  1.16);

\path[draw=drawColor,line width= 0.4pt,line join=round,line cap=round,fill=fillColor] (445.32,209.81) circle (  1.16);

\path[draw=drawColor,line width= 0.4pt,line join=round,line cap=round,fill=fillColor] (445.51,209.81) circle (  1.16);

\path[draw=drawColor,line width= 0.4pt,line join=round,line cap=round,fill=fillColor] (445.71,209.81) circle (  1.16);

\path[draw=drawColor,line width= 0.4pt,line join=round,line cap=round,fill=fillColor] (445.90,209.81) circle (  1.16);

\path[draw=drawColor,line width= 0.4pt,line join=round,line cap=round,fill=fillColor] (446.09,209.81) circle (  1.16);

\path[draw=drawColor,line width= 0.4pt,line join=round,line cap=round,fill=fillColor] (446.29,209.81) circle (  1.16);

\path[draw=drawColor,line width= 0.4pt,line join=round,line cap=round,fill=fillColor] (446.48,209.81) circle (  1.16);

\path[draw=drawColor,line width= 0.4pt,line join=round,line cap=round,fill=fillColor] (446.67,209.81) circle (  1.16);

\path[draw=drawColor,line width= 0.4pt,line join=round,line cap=round,fill=fillColor] (446.86,209.81) circle (  1.16);

\path[draw=drawColor,line width= 0.4pt,line join=round,line cap=round,fill=fillColor] (447.05,209.81) circle (  1.16);

\path[draw=drawColor,line width= 0.4pt,line join=round,line cap=round,fill=fillColor] (447.24,209.81) circle (  1.16);

\path[draw=drawColor,line width= 0.4pt,line join=round,line cap=round,fill=fillColor] (447.43,209.81) circle (  1.16);

\path[draw=drawColor,line width= 0.4pt,line join=round,line cap=round,fill=fillColor] (447.62,209.81) circle (  1.16);

\path[draw=drawColor,line width= 0.4pt,line join=round,line cap=round,fill=fillColor] (447.81,209.81) circle (  1.16);

\path[draw=drawColor,line width= 0.4pt,line join=round,line cap=round,fill=fillColor] (448.00,209.81) circle (  1.16);

\path[draw=drawColor,line width= 0.4pt,line join=round,line cap=round,fill=fillColor] (448.19,209.81) circle (  1.16);

\path[draw=drawColor,line width= 0.4pt,line join=round,line cap=round,fill=fillColor] (448.37,209.81) circle (  1.16);

\path[draw=drawColor,line width= 0.4pt,line join=round,line cap=round,fill=fillColor] (448.56,209.81) circle (  1.16);

\path[draw=drawColor,line width= 0.4pt,line join=round,line cap=round,fill=fillColor] (448.75,209.81) circle (  1.16);

\path[draw=drawColor,line width= 0.4pt,line join=round,line cap=round,fill=fillColor] (448.93,209.81) circle (  1.16);

\path[draw=drawColor,line width= 0.4pt,line join=round,line cap=round,fill=fillColor] (449.12,209.81) circle (  1.16);

\path[draw=drawColor,line width= 0.4pt,line join=round,line cap=round,fill=fillColor] (449.30,209.81) circle (  1.16);

\path[draw=drawColor,line width= 0.4pt,line join=round,line cap=round,fill=fillColor] (449.49,209.81) circle (  1.16);

\path[draw=drawColor,line width= 0.4pt,line join=round,line cap=round,fill=fillColor] (449.67,209.81) circle (  1.16);

\path[draw=drawColor,line width= 0.4pt,line join=round,line cap=round,fill=fillColor] (449.86,209.81) circle (  1.16);

\path[draw=drawColor,line width= 0.4pt,line join=round,line cap=round,fill=fillColor] (450.04,209.81) circle (  1.16);

\path[draw=drawColor,line width= 0.4pt,line join=round,line cap=round,fill=fillColor] (450.22,209.81) circle (  1.16);

\path[draw=drawColor,line width= 0.4pt,line join=round,line cap=round,fill=fillColor] (450.40,209.81) circle (  1.16);

\path[draw=drawColor,line width= 0.4pt,line join=round,line cap=round,fill=fillColor] (450.58,209.81) circle (  1.16);

\path[draw=drawColor,line width= 0.4pt,line join=round,line cap=round,fill=fillColor] (450.77,209.81) circle (  1.16);

\path[draw=drawColor,line width= 0.4pt,line join=round,line cap=round,fill=fillColor] (450.95,209.81) circle (  1.16);

\path[draw=drawColor,line width= 0.4pt,line join=round,line cap=round,fill=fillColor] (451.13,209.81) circle (  1.16);

\path[draw=drawColor,line width= 0.4pt,line join=round,line cap=round,fill=fillColor] (451.31,209.81) circle (  1.16);

\path[draw=drawColor,line width= 0.4pt,line join=round,line cap=round,fill=fillColor] (451.49,209.81) circle (  1.16);

\path[draw=drawColor,line width= 0.4pt,line join=round,line cap=round,fill=fillColor] (451.67,209.81) circle (  1.16);

\path[draw=drawColor,line width= 0.4pt,line join=round,line cap=round,fill=fillColor] (451.84,209.81) circle (  1.16);

\path[draw=drawColor,line width= 0.4pt,line join=round,line cap=round,fill=fillColor] (452.02,209.81) circle (  1.16);

\path[draw=drawColor,line width= 0.4pt,line join=round,line cap=round,fill=fillColor] (452.20,209.81) circle (  1.16);

\path[draw=drawColor,line width= 0.4pt,line join=round,line cap=round,fill=fillColor] (452.38,209.81) circle (  1.16);

\path[draw=drawColor,line width= 0.4pt,line join=round,line cap=round,fill=fillColor] (452.55,209.81) circle (  1.16);

\path[draw=drawColor,line width= 0.4pt,line join=round,line cap=round,fill=fillColor] (452.73,209.81) circle (  1.16);

\path[draw=drawColor,line width= 0.4pt,line join=round,line cap=round,fill=fillColor] (452.91,209.81) circle (  1.16);

\path[draw=drawColor,line width= 0.4pt,line join=round,line cap=round,fill=fillColor] (453.08,209.81) circle (  1.16);

\path[draw=drawColor,line width= 0.4pt,line join=round,line cap=round,fill=fillColor] (453.26,209.81) circle (  1.16);

\path[draw=drawColor,line width= 0.4pt,line join=round,line cap=round,fill=fillColor] (453.43,209.81) circle (  1.16);

\path[draw=drawColor,line width= 0.4pt,line join=round,line cap=round,fill=fillColor] (453.61,209.81) circle (  1.16);

\path[draw=drawColor,line width= 0.4pt,line join=round,line cap=round,fill=fillColor] (453.78,209.81) circle (  1.16);

\path[draw=drawColor,line width= 0.4pt,line join=round,line cap=round,fill=fillColor] (453.95,209.81) circle (  1.16);

\path[draw=drawColor,line width= 0.4pt,line join=round,line cap=round,fill=fillColor] (454.13,209.81) circle (  1.16);

\path[draw=drawColor,line width= 0.4pt,line join=round,line cap=round,fill=fillColor] (454.30,209.81) circle (  1.16);

\path[draw=drawColor,line width= 0.4pt,line join=round,line cap=round,fill=fillColor] (454.47,209.81) circle (  1.16);

\path[draw=drawColor,line width= 0.4pt,line join=round,line cap=round,fill=fillColor] (454.64,209.81) circle (  1.16);

\path[draw=drawColor,line width= 0.4pt,line join=round,line cap=round,fill=fillColor] (454.82,209.81) circle (  1.16);

\path[draw=drawColor,line width= 0.4pt,line join=round,line cap=round,fill=fillColor] (454.99,209.81) circle (  1.16);

\path[draw=drawColor,line width= 0.4pt,line join=round,line cap=round,fill=fillColor] (455.16,209.81) circle (  1.16);

\path[draw=drawColor,line width= 0.4pt,line join=round,line cap=round,fill=fillColor] (455.33,209.81) circle (  1.16);

\path[draw=drawColor,line width= 0.4pt,line join=round,line cap=round,fill=fillColor] (455.50,209.81) circle (  1.16);

\path[draw=drawColor,line width= 0.4pt,line join=round,line cap=round,fill=fillColor] (455.67,209.81) circle (  1.16);

\path[draw=drawColor,line width= 0.4pt,line join=round,line cap=round,fill=fillColor] (455.84,209.81) circle (  1.16);

\path[draw=drawColor,line width= 0.4pt,line join=round,line cap=round,fill=fillColor] (456.00,209.81) circle (  1.16);

\path[draw=drawColor,line width= 0.4pt,line join=round,line cap=round,fill=fillColor] (456.17,209.81) circle (  1.16);

\path[draw=drawColor,line width= 0.4pt,line join=round,line cap=round,fill=fillColor] (456.34,209.81) circle (  1.16);

\path[draw=drawColor,line width= 0.4pt,line join=round,line cap=round,fill=fillColor] (456.51,209.81) circle (  1.16);

\path[draw=drawColor,line width= 0.4pt,line join=round,line cap=round,fill=fillColor] (456.68,209.81) circle (  1.16);

\path[draw=drawColor,line width= 0.4pt,line join=round,line cap=round,fill=fillColor] (456.84,209.81) circle (  1.16);

\path[draw=drawColor,line width= 0.4pt,line join=round,line cap=round,fill=fillColor] (457.01,209.81) circle (  1.16);

\path[draw=drawColor,line width= 0.4pt,line join=round,line cap=round,fill=fillColor] (457.18,209.81) circle (  1.16);

\path[draw=drawColor,line width= 0.4pt,line join=round,line cap=round,fill=fillColor] (457.34,209.81) circle (  1.16);

\path[draw=drawColor,line width= 0.4pt,line join=round,line cap=round,fill=fillColor] (457.51,209.81) circle (  1.16);

\path[draw=drawColor,line width= 0.4pt,line join=round,line cap=round,fill=fillColor] (457.67,209.81) circle (  1.16);

\path[draw=drawColor,line width= 0.4pt,line join=round,line cap=round,fill=fillColor] (457.84,209.81) circle (  1.16);

\path[draw=drawColor,line width= 0.4pt,line join=round,line cap=round,fill=fillColor] (458.00,209.81) circle (  1.16);

\path[draw=drawColor,line width= 0.4pt,line join=round,line cap=round,fill=fillColor] (458.17,209.81) circle (  1.16);

\path[draw=drawColor,line width= 0.4pt,line join=round,line cap=round,fill=fillColor] (458.33,209.81) circle (  1.16);

\path[draw=drawColor,line width= 0.4pt,line join=round,line cap=round,fill=fillColor] (458.49,209.81) circle (  1.16);

\path[draw=drawColor,line width= 0.4pt,line join=round,line cap=round,fill=fillColor] (458.66,209.81) circle (  1.16);

\path[draw=drawColor,line width= 0.4pt,line join=round,line cap=round,fill=fillColor] (458.82,209.81) circle (  1.16);

\path[draw=drawColor,line width= 0.4pt,line join=round,line cap=round,fill=fillColor] (458.98,209.81) circle (  1.16);

\path[draw=drawColor,line width= 0.4pt,line join=round,line cap=round,fill=fillColor] (459.14,209.81) circle (  1.16);

\path[draw=drawColor,line width= 0.4pt,line join=round,line cap=round,fill=fillColor] (459.30,209.81) circle (  1.16);

\path[draw=drawColor,line width= 0.4pt,line join=round,line cap=round,fill=fillColor] (459.47,209.81) circle (  1.16);

\path[draw=drawColor,line width= 0.4pt,line join=round,line cap=round,fill=fillColor] (459.63,209.81) circle (  1.16);

\path[draw=drawColor,line width= 0.4pt,line join=round,line cap=round,fill=fillColor] (459.79,209.81) circle (  1.16);

\path[draw=drawColor,line width= 0.4pt,line join=round,line cap=round,fill=fillColor] (459.95,209.81) circle (  1.16);

\path[draw=drawColor,line width= 0.4pt,line join=round,line cap=round,fill=fillColor] (460.11,209.81) circle (  1.16);

\path[draw=drawColor,line width= 0.4pt,line join=round,line cap=round,fill=fillColor] (460.27,209.81) circle (  1.16);

\path[draw=drawColor,line width= 0.4pt,line join=round,line cap=round,fill=fillColor] (460.43,209.81) circle (  1.16);

\path[draw=drawColor,line width= 0.4pt,line join=round,line cap=round,fill=fillColor] (460.59,209.81) circle (  1.16);

\path[draw=drawColor,line width= 0.4pt,line join=round,line cap=round,fill=fillColor] (460.74,209.81) circle (  1.16);

\path[draw=drawColor,line width= 0.4pt,line join=round,line cap=round,fill=fillColor] (460.90,209.81) circle (  1.16);

\path[draw=drawColor,line width= 0.4pt,line join=round,line cap=round,fill=fillColor] (461.06,209.81) circle (  1.16);

\path[draw=drawColor,line width= 0.4pt,line join=round,line cap=round,fill=fillColor] (461.22,209.81) circle (  1.16);

\path[draw=drawColor,line width= 0.4pt,line join=round,line cap=round,fill=fillColor] (461.38,209.81) circle (  1.16);

\path[draw=drawColor,line width= 0.4pt,line join=round,line cap=round,fill=fillColor] (461.53,209.81) circle (  1.16);

\path[draw=drawColor,line width= 0.4pt,line join=round,line cap=round,fill=fillColor] (461.69,209.81) circle (  1.16);

\path[draw=drawColor,line width= 0.4pt,line join=round,line cap=round,fill=fillColor] (461.85,209.81) circle (  1.16);

\path[draw=drawColor,line width= 0.4pt,line join=round,line cap=round,fill=fillColor] (462.00,209.81) circle (  1.16);

\path[draw=drawColor,line width= 0.4pt,line join=round,line cap=round,fill=fillColor] (462.16,209.81) circle (  1.16);

\path[draw=drawColor,line width= 0.4pt,line join=round,line cap=round,fill=fillColor] (462.31,209.81) circle (  1.16);

\path[draw=drawColor,line width= 0.4pt,line join=round,line cap=round,fill=fillColor] (462.47,209.81) circle (  1.16);

\path[draw=drawColor,line width= 0.4pt,line join=round,line cap=round,fill=fillColor] (462.62,209.81) circle (  1.16);

\path[draw=drawColor,line width= 0.4pt,line join=round,line cap=round,fill=fillColor] (462.78,209.81) circle (  1.16);

\path[draw=drawColor,line width= 0.4pt,line join=round,line cap=round,fill=fillColor] (462.93,209.81) circle (  1.16);

\path[draw=drawColor,line width= 0.4pt,line join=round,line cap=round,fill=fillColor] (463.09,209.81) circle (  1.16);

\path[draw=drawColor,line width= 0.4pt,line join=round,line cap=round,fill=fillColor] (463.24,209.81) circle (  1.16);

\path[draw=drawColor,line width= 0.4pt,line join=round,line cap=round,fill=fillColor] (463.39,209.81) circle (  1.16);

\path[draw=drawColor,line width= 0.4pt,line join=round,line cap=round,fill=fillColor] (463.55,209.81) circle (  1.16);

\path[draw=drawColor,line width= 0.4pt,line join=round,line cap=round,fill=fillColor] (463.70,209.81) circle (  1.16);

\path[draw=drawColor,line width= 0.4pt,line join=round,line cap=round,fill=fillColor] (463.85,209.81) circle (  1.16);

\path[draw=drawColor,line width= 0.4pt,line join=round,line cap=round,fill=fillColor] (464.00,209.81) circle (  1.16);

\path[draw=drawColor,line width= 0.4pt,line join=round,line cap=round,fill=fillColor] (464.16,209.81) circle (  1.16);

\path[draw=drawColor,line width= 0.4pt,line join=round,line cap=round,fill=fillColor] (464.31,209.81) circle (  1.16);

\path[draw=drawColor,line width= 0.4pt,line join=round,line cap=round,fill=fillColor] (464.46,209.81) circle (  1.16);

\path[draw=drawColor,line width= 0.4pt,line join=round,line cap=round,fill=fillColor] (464.61,209.81) circle (  1.16);

\path[draw=drawColor,line width= 0.4pt,line join=round,line cap=round,fill=fillColor] (464.76,209.81) circle (  1.16);

\path[draw=drawColor,line width= 0.4pt,line join=round,line cap=round,fill=fillColor] (464.91,209.81) circle (  1.16);

\path[draw=drawColor,line width= 0.4pt,line join=round,line cap=round,fill=fillColor] (465.06,209.81) circle (  1.16);

\path[draw=drawColor,line width= 0.4pt,line join=round,line cap=round,fill=fillColor] (465.21,209.81) circle (  1.16);

\path[draw=drawColor,line width= 0.4pt,line join=round,line cap=round,fill=fillColor] (465.36,209.81) circle (  1.16);

\path[draw=drawColor,line width= 0.4pt,line join=round,line cap=round,fill=fillColor] (465.51,209.81) circle (  1.16);

\path[draw=drawColor,line width= 0.4pt,line join=round,line cap=round,fill=fillColor] (465.66,209.81) circle (  1.16);

\path[draw=drawColor,line width= 0.4pt,line join=round,line cap=round,fill=fillColor] (465.81,209.81) circle (  1.16);

\path[draw=drawColor,line width= 0.4pt,line join=round,line cap=round,fill=fillColor] (465.96,209.81) circle (  1.16);

\path[draw=drawColor,line width= 0.4pt,line join=round,line cap=round,fill=fillColor] (466.10,209.81) circle (  1.16);

\path[draw=drawColor,line width= 0.4pt,line join=round,line cap=round,fill=fillColor] (466.25,209.81) circle (  1.16);

\path[draw=drawColor,line width= 0.4pt,line join=round,line cap=round,fill=fillColor] (466.40,209.81) circle (  1.16);

\path[draw=drawColor,line width= 0.4pt,line join=round,line cap=round,fill=fillColor] (466.55,209.81) circle (  1.16);

\path[draw=drawColor,line width= 0.4pt,line join=round,line cap=round,fill=fillColor] (466.70,209.81) circle (  1.16);

\path[draw=drawColor,line width= 0.4pt,line join=round,line cap=round,fill=fillColor] (466.84,209.81) circle (  1.16);

\path[draw=drawColor,line width= 0.4pt,line join=round,line cap=round,fill=fillColor] (466.99,209.81) circle (  1.16);

\path[draw=drawColor,line width= 0.4pt,line join=round,line cap=round,fill=fillColor] (467.14,209.81) circle (  1.16);

\path[draw=drawColor,line width= 0.4pt,line join=round,line cap=round,fill=fillColor] (467.28,209.81) circle (  1.16);

\path[draw=drawColor,line width= 0.4pt,line join=round,line cap=round,fill=fillColor] (467.43,209.81) circle (  1.16);

\path[draw=drawColor,line width= 0.4pt,line join=round,line cap=round,fill=fillColor] (467.57,209.81) circle (  1.16);

\path[draw=drawColor,line width= 0.4pt,line join=round,line cap=round,fill=fillColor] (467.72,209.81) circle (  1.16);

\path[draw=drawColor,line width= 0.4pt,line join=round,line cap=round,fill=fillColor] (467.86,209.81) circle (  1.16);

\path[draw=drawColor,line width= 0.4pt,line join=round,line cap=round,fill=fillColor] (468.01,209.81) circle (  1.16);

\path[draw=drawColor,line width= 0.4pt,line join=round,line cap=round,fill=fillColor] (468.15,209.81) circle (  1.16);

\path[draw=drawColor,line width= 0.4pt,line join=round,line cap=round,fill=fillColor] (468.30,209.81) circle (  1.16);

\path[draw=drawColor,line width= 0.4pt,line join=round,line cap=round,fill=fillColor] (468.44,209.81) circle (  1.16);

\path[draw=drawColor,line width= 0.4pt,line join=round,line cap=round,fill=fillColor] (468.58,209.81) circle (  1.16);

\path[draw=drawColor,line width= 0.4pt,line join=round,line cap=round,fill=fillColor] (468.73,209.81) circle (  1.16);

\path[draw=drawColor,line width= 0.4pt,line join=round,line cap=round,fill=fillColor] (468.87,209.81) circle (  1.16);

\path[draw=drawColor,line width= 0.4pt,line join=round,line cap=round,fill=fillColor] (469.01,209.81) circle (  1.16);

\path[draw=drawColor,line width= 0.4pt,line join=round,line cap=round,fill=fillColor] (469.16,209.81) circle (  1.16);

\path[draw=drawColor,line width= 0.4pt,line join=round,line cap=round,fill=fillColor] (469.30,209.81) circle (  1.16);

\path[draw=drawColor,line width= 0.4pt,line join=round,line cap=round,fill=fillColor] (469.44,209.81) circle (  1.16);

\path[draw=drawColor,line width= 0.4pt,line join=round,line cap=round,fill=fillColor] (469.58,209.81) circle (  1.16);

\path[draw=drawColor,line width= 0.4pt,line join=round,line cap=round,fill=fillColor] (469.73,209.81) circle (  1.16);

\path[draw=drawColor,line width= 0.4pt,line join=round,line cap=round,fill=fillColor] (469.87,209.81) circle (  1.16);

\path[draw=drawColor,line width= 0.4pt,line join=round,line cap=round,fill=fillColor] (470.01,209.81) circle (  1.16);

\path[draw=drawColor,line width= 0.4pt,line join=round,line cap=round,fill=fillColor] (470.15,209.81) circle (  1.16);

\path[draw=drawColor,line width= 0.4pt,line join=round,line cap=round,fill=fillColor] (470.29,209.81) circle (  1.16);

\path[draw=drawColor,line width= 0.4pt,line join=round,line cap=round,fill=fillColor] (470.43,209.81) circle (  1.16);

\path[draw=drawColor,line width= 0.4pt,line join=round,line cap=round,fill=fillColor] (470.57,209.81) circle (  1.16);

\path[draw=drawColor,line width= 0.4pt,line join=round,line cap=round,fill=fillColor] (470.71,209.81) circle (  1.16);

\path[draw=drawColor,line width= 0.4pt,line join=round,line cap=round,fill=fillColor] (470.85,209.81) circle (  1.16);

\path[draw=drawColor,line width= 0.4pt,line join=round,line cap=round,fill=fillColor] (470.99,209.81) circle (  1.16);

\path[draw=drawColor,line width= 0.4pt,line join=round,line cap=round,fill=fillColor] (471.13,209.81) circle (  1.16);

\path[draw=drawColor,line width= 0.4pt,line join=round,line cap=round,fill=fillColor] (471.27,209.81) circle (  1.16);

\path[draw=drawColor,line width= 0.4pt,line join=round,line cap=round,fill=fillColor] (471.41,209.81) circle (  1.16);

\path[draw=drawColor,line width= 0.4pt,line join=round,line cap=round,fill=fillColor] (471.55,209.81) circle (  1.16);

\path[draw=drawColor,line width= 0.4pt,line join=round,line cap=round,fill=fillColor] (471.69,209.81) circle (  1.16);

\path[draw=drawColor,line width= 0.4pt,line join=round,line cap=round,fill=fillColor] (471.82,209.81) circle (  1.16);

\path[draw=drawColor,line width= 0.4pt,line join=round,line cap=round,fill=fillColor] (471.96,209.81) circle (  1.16);

\path[draw=drawColor,line width= 0.4pt,line join=round,line cap=round,fill=fillColor] (472.10,209.81) circle (  1.16);

\path[draw=drawColor,line width= 0.4pt,line join=round,line cap=round,fill=fillColor] (472.24,209.81) circle (  1.16);

\path[draw=drawColor,line width= 0.4pt,line join=round,line cap=round,fill=fillColor] (472.37,209.81) circle (  1.16);

\path[draw=drawColor,line width= 0.4pt,line join=round,line cap=round,fill=fillColor] (472.51,209.81) circle (  1.16);

\path[draw=drawColor,line width= 0.4pt,line join=round,line cap=round,fill=fillColor] (472.65,209.81) circle (  1.16);

\path[draw=drawColor,line width= 0.4pt,line join=round,line cap=round,fill=fillColor] (472.78,209.81) circle (  1.16);

\path[draw=drawColor,line width= 0.4pt,line join=round,line cap=round,fill=fillColor] (472.92,209.81) circle (  1.16);

\path[draw=drawColor,line width= 0.4pt,line join=round,line cap=round,fill=fillColor] (473.06,209.81) circle (  1.16);

\path[draw=drawColor,line width= 0.4pt,line join=round,line cap=round,fill=fillColor] (473.19,209.81) circle (  1.16);

\path[draw=drawColor,line width= 0.4pt,line join=round,line cap=round,fill=fillColor] (473.33,209.81) circle (  1.16);

\path[draw=drawColor,line width= 0.4pt,line join=round,line cap=round,fill=fillColor] (473.46,209.81) circle (  1.16);

\path[draw=drawColor,line width= 0.4pt,line join=round,line cap=round,fill=fillColor] (473.60,209.81) circle (  1.16);

\path[draw=drawColor,line width= 0.4pt,line join=round,line cap=round,fill=fillColor] (473.74,209.81) circle (  1.16);

\path[draw=drawColor,line width= 0.4pt,line join=round,line cap=round,fill=fillColor] (473.87,209.81) circle (  1.16);

\path[draw=drawColor,line width= 0.4pt,line join=round,line cap=round,fill=fillColor] (474.00,209.81) circle (  1.16);

\path[draw=drawColor,line width= 0.4pt,line join=round,line cap=round,fill=fillColor] (474.14,209.81) circle (  1.16);

\path[draw=drawColor,line width= 0.4pt,line join=round,line cap=round,fill=fillColor] (474.27,209.81) circle (  1.16);

\path[draw=drawColor,line width= 0.4pt,line join=round,line cap=round,fill=fillColor] (474.41,209.81) circle (  1.16);

\path[draw=drawColor,line width= 0.4pt,line join=round,line cap=round,fill=fillColor] (474.54,209.81) circle (  1.16);

\path[draw=drawColor,line width= 0.4pt,line join=round,line cap=round,fill=fillColor] (474.68,209.81) circle (  1.16);

\path[draw=drawColor,line width= 0.4pt,line join=round,line cap=round,fill=fillColor] (474.81,209.81) circle (  1.16);

\path[draw=drawColor,line width= 0.4pt,line join=round,line cap=round,fill=fillColor] (474.94,209.81) circle (  1.16);

\path[draw=drawColor,line width= 0.4pt,line join=round,line cap=round,fill=fillColor] (475.07,209.81) circle (  1.16);

\path[draw=drawColor,line width= 0.4pt,line join=round,line cap=round,fill=fillColor] (475.21,209.81) circle (  1.16);

\path[draw=drawColor,line width= 0.4pt,line join=round,line cap=round,fill=fillColor] (475.34,209.81) circle (  1.16);

\path[draw=drawColor,line width= 0.4pt,line join=round,line cap=round,fill=fillColor] (475.47,209.81) circle (  1.16);

\path[draw=drawColor,line width= 0.4pt,line join=round,line cap=round,fill=fillColor] (475.60,209.81) circle (  1.16);

\path[draw=drawColor,line width= 0.4pt,line join=round,line cap=round,fill=fillColor] (475.74,209.81) circle (  1.16);

\path[draw=drawColor,line width= 0.4pt,line join=round,line cap=round,fill=fillColor] (475.87,209.81) circle (  1.16);

\path[draw=drawColor,line width= 0.4pt,line join=round,line cap=round,fill=fillColor] (476.00,209.81) circle (  1.16);

\path[draw=drawColor,line width= 0.4pt,line join=round,line cap=round,fill=fillColor] (476.13,209.81) circle (  1.16);

\path[draw=drawColor,line width= 0.4pt,line join=round,line cap=round,fill=fillColor] (476.26,209.81) circle (  1.16);

\path[draw=drawColor,line width= 0.4pt,line join=round,line cap=round,fill=fillColor] (476.39,209.81) circle (  1.16);

\path[draw=drawColor,line width= 0.4pt,line join=round,line cap=round,fill=fillColor] (476.52,209.81) circle (  1.16);

\path[draw=drawColor,line width= 0.4pt,line join=round,line cap=round,fill=fillColor] (476.65,209.81) circle (  1.16);

\path[draw=drawColor,line width= 0.4pt,line join=round,line cap=round,fill=fillColor] (476.78,209.81) circle (  1.16);

\path[draw=drawColor,line width= 0.4pt,line join=round,line cap=round,fill=fillColor] (476.91,209.81) circle (  1.16);

\path[draw=drawColor,line width= 0.4pt,line join=round,line cap=round,fill=fillColor] (477.04,209.81) circle (  1.16);

\path[draw=drawColor,line width= 0.4pt,line join=round,line cap=round,fill=fillColor] (477.17,209.81) circle (  1.16);

\path[draw=drawColor,line width= 0.4pt,line join=round,line cap=round,fill=fillColor] (477.30,209.81) circle (  1.16);

\path[draw=drawColor,line width= 0.4pt,line join=round,line cap=round,fill=fillColor] (477.43,209.81) circle (  1.16);

\path[draw=drawColor,line width= 0.4pt,line join=round,line cap=round,fill=fillColor] (477.56,209.81) circle (  1.16);

\path[draw=drawColor,line width= 0.4pt,line join=round,line cap=round,fill=fillColor] (477.69,209.81) circle (  1.16);

\path[draw=drawColor,line width= 0.4pt,line join=round,line cap=round,fill=fillColor] (477.82,209.81) circle (  1.16);

\path[draw=drawColor,line width= 0.4pt,line join=round,line cap=round,fill=fillColor] (477.95,209.81) circle (  1.16);

\path[draw=drawColor,line width= 0.4pt,line join=round,line cap=round,fill=fillColor] (478.08,209.81) circle (  1.16);

\path[draw=drawColor,line width= 0.4pt,line join=round,line cap=round,fill=fillColor] (478.21,209.81) circle (  1.16);
\definecolor[named]{drawColor}{rgb}{0.22,0.49,0.72}
\definecolor[named]{fillColor}{rgb}{0.22,0.49,0.72}

\path[draw=drawColor,line width= 0.4pt,line join=round,line cap=round,fill=fillColor] (327.82,290.39) circle (  1.16);

\path[draw=drawColor,line width= 0.4pt,line join=round,line cap=round,fill=fillColor] (333.69,288.94) circle (  1.16);

\path[draw=drawColor,line width= 0.4pt,line join=round,line cap=round,fill=fillColor] (337.81,288.48) circle (  1.16);

\path[draw=drawColor,line width= 0.4pt,line join=round,line cap=round,fill=fillColor] (341.08,288.21) circle (  1.16);

\path[draw=drawColor,line width= 0.4pt,line join=round,line cap=round,fill=fillColor] (343.85,287.88) circle (  1.16);

\path[draw=drawColor,line width= 0.4pt,line join=round,line cap=round,fill=fillColor] (346.27,286.25) circle (  1.16);

\path[draw=drawColor,line width= 0.4pt,line join=round,line cap=round,fill=fillColor] (348.43,284.48) circle (  1.16);

\path[draw=drawColor,line width= 0.4pt,line join=round,line cap=round,fill=fillColor] (350.39,282.19) circle (  1.16);

\path[draw=drawColor,line width= 0.4pt,line join=round,line cap=round,fill=fillColor] (352.20,281.69) circle (  1.16);

\path[draw=drawColor,line width= 0.4pt,line join=round,line cap=round,fill=fillColor] (353.88,281.64) circle (  1.16);

\path[draw=drawColor,line width= 0.4pt,line join=round,line cap=round,fill=fillColor] (355.45,281.40) circle (  1.16);

\path[draw=drawColor,line width= 0.4pt,line join=round,line cap=round,fill=fillColor] (356.93,280.76) circle (  1.16);

\path[draw=drawColor,line width= 0.4pt,line join=round,line cap=round,fill=fillColor] (358.32,280.70) circle (  1.16);

\path[draw=drawColor,line width= 0.4pt,line join=round,line cap=round,fill=fillColor] (359.65,277.62) circle (  1.16);

\path[draw=drawColor,line width= 0.4pt,line join=round,line cap=round,fill=fillColor] (360.92,275.80) circle (  1.16);

\path[draw=drawColor,line width= 0.4pt,line join=round,line cap=round,fill=fillColor] (362.13,275.69) circle (  1.16);

\path[draw=drawColor,line width= 0.4pt,line join=round,line cap=round,fill=fillColor] (363.29,275.41) circle (  1.16);

\path[draw=drawColor,line width= 0.4pt,line join=round,line cap=round,fill=fillColor] (364.40,275.25) circle (  1.16);

\path[draw=drawColor,line width= 0.4pt,line join=round,line cap=round,fill=fillColor] (365.48,274.60) circle (  1.16);

\path[draw=drawColor,line width= 0.4pt,line join=round,line cap=round,fill=fillColor] (366.52,274.52) circle (  1.16);

\path[draw=drawColor,line width= 0.4pt,line join=round,line cap=round,fill=fillColor] (367.52,273.09) circle (  1.16);

\path[draw=drawColor,line width= 0.4pt,line join=round,line cap=round,fill=fillColor] (368.49,272.41) circle (  1.16);

\path[draw=drawColor,line width= 0.4pt,line join=round,line cap=round,fill=fillColor] (369.44,271.42) circle (  1.16);

\path[draw=drawColor,line width= 0.4pt,line join=round,line cap=round,fill=fillColor] (370.36,270.42) circle (  1.16);

\path[draw=drawColor,line width= 0.4pt,line join=round,line cap=round,fill=fillColor] (371.25,267.10) circle (  1.16);

\path[draw=drawColor,line width= 0.4pt,line join=round,line cap=round,fill=fillColor] (372.12,264.72) circle (  1.16);

\path[draw=drawColor,line width= 0.4pt,line join=round,line cap=round,fill=fillColor] (372.96,263.72) circle (  1.16);

\path[draw=drawColor,line width= 0.4pt,line join=round,line cap=round,fill=fillColor] (373.79,263.01) circle (  1.16);

\path[draw=drawColor,line width= 0.4pt,line join=round,line cap=round,fill=fillColor] (374.59,262.83) circle (  1.16);

\path[draw=drawColor,line width= 0.4pt,line join=round,line cap=round,fill=fillColor] (375.38,262.78) circle (  1.16);

\path[draw=drawColor,line width= 0.4pt,line join=round,line cap=round,fill=fillColor] (376.15,262.50) circle (  1.16);

\path[draw=drawColor,line width= 0.4pt,line join=round,line cap=round,fill=fillColor] (376.91,262.11) circle (  1.16);

\path[draw=drawColor,line width= 0.4pt,line join=round,line cap=round,fill=fillColor] (377.65,261.30) circle (  1.16);

\path[draw=drawColor,line width= 0.4pt,line join=round,line cap=round,fill=fillColor] (378.37,260.77) circle (  1.16);

\path[draw=drawColor,line width= 0.4pt,line join=round,line cap=round,fill=fillColor] (379.08,260.63) circle (  1.16);

\path[draw=drawColor,line width= 0.4pt,line join=round,line cap=round,fill=fillColor] (379.78,259.46) circle (  1.16);

\path[draw=drawColor,line width= 0.4pt,line join=round,line cap=round,fill=fillColor] (380.46,258.71) circle (  1.16);

\path[draw=drawColor,line width= 0.4pt,line join=round,line cap=round,fill=fillColor] (381.13,258.63) circle (  1.16);

\path[draw=drawColor,line width= 0.4pt,line join=round,line cap=round,fill=fillColor] (381.79,258.39) circle (  1.16);

\path[draw=drawColor,line width= 0.4pt,line join=round,line cap=round,fill=fillColor] (382.44,258.16) circle (  1.16);

\path[draw=drawColor,line width= 0.4pt,line join=round,line cap=round,fill=fillColor] (383.08,257.87) circle (  1.16);

\path[draw=drawColor,line width= 0.4pt,line join=round,line cap=round,fill=fillColor] (383.71,257.31) circle (  1.16);

\path[draw=drawColor,line width= 0.4pt,line join=round,line cap=round,fill=fillColor] (384.32,257.27) circle (  1.16);

\path[draw=drawColor,line width= 0.4pt,line join=round,line cap=round,fill=fillColor] (384.93,257.05) circle (  1.16);

\path[draw=drawColor,line width= 0.4pt,line join=round,line cap=round,fill=fillColor] (385.53,256.94) circle (  1.16);

\path[draw=drawColor,line width= 0.4pt,line join=round,line cap=round,fill=fillColor] (386.12,256.75) circle (  1.16);

\path[draw=drawColor,line width= 0.4pt,line join=round,line cap=round,fill=fillColor] (386.70,256.74) circle (  1.16);

\path[draw=drawColor,line width= 0.4pt,line join=round,line cap=round,fill=fillColor] (387.28,256.65) circle (  1.16);

\path[draw=drawColor,line width= 0.4pt,line join=round,line cap=round,fill=fillColor] (387.84,256.03) circle (  1.16);

\path[draw=drawColor,line width= 0.4pt,line join=round,line cap=round,fill=fillColor] (388.40,255.85) circle (  1.16);

\path[draw=drawColor,line width= 0.4pt,line join=round,line cap=round,fill=fillColor] (388.95,255.71) circle (  1.16);

\path[draw=drawColor,line width= 0.4pt,line join=round,line cap=round,fill=fillColor] (389.50,255.69) circle (  1.16);

\path[draw=drawColor,line width= 0.4pt,line join=round,line cap=round,fill=fillColor] (390.03,255.56) circle (  1.16);

\path[draw=drawColor,line width= 0.4pt,line join=round,line cap=round,fill=fillColor] (390.56,255.25) circle (  1.16);

\path[draw=drawColor,line width= 0.4pt,line join=round,line cap=round,fill=fillColor] (391.08,255.07) circle (  1.16);

\path[draw=drawColor,line width= 0.4pt,line join=round,line cap=round,fill=fillColor] (391.60,254.58) circle (  1.16);

\path[draw=drawColor,line width= 0.4pt,line join=round,line cap=round,fill=fillColor] (392.11,254.49) circle (  1.16);

\path[draw=drawColor,line width= 0.4pt,line join=round,line cap=round,fill=fillColor] (392.62,254.33) circle (  1.16);

\path[draw=drawColor,line width= 0.4pt,line join=round,line cap=round,fill=fillColor] (393.12,254.31) circle (  1.16);

\path[draw=drawColor,line width= 0.4pt,line join=round,line cap=round,fill=fillColor] (393.61,254.20) circle (  1.16);

\path[draw=drawColor,line width= 0.4pt,line join=round,line cap=round,fill=fillColor] (394.10,254.02) circle (  1.16);

\path[draw=drawColor,line width= 0.4pt,line join=round,line cap=round,fill=fillColor] (394.58,253.05) circle (  1.16);

\path[draw=drawColor,line width= 0.4pt,line join=round,line cap=round,fill=fillColor] (395.06,253.04) circle (  1.16);

\path[draw=drawColor,line width= 0.4pt,line join=round,line cap=round,fill=fillColor] (395.53,253.01) circle (  1.16);

\path[draw=drawColor,line width= 0.4pt,line join=round,line cap=round,fill=fillColor] (396.00,252.16) circle (  1.16);

\path[draw=drawColor,line width= 0.4pt,line join=round,line cap=round,fill=fillColor] (396.46,252.14) circle (  1.16);

\path[draw=drawColor,line width= 0.4pt,line join=round,line cap=round,fill=fillColor] (396.92,251.62) circle (  1.16);

\path[draw=drawColor,line width= 0.4pt,line join=round,line cap=round,fill=fillColor] (397.37,251.26) circle (  1.16);

\path[draw=drawColor,line width= 0.4pt,line join=round,line cap=round,fill=fillColor] (397.82,251.19) circle (  1.16);

\path[draw=drawColor,line width= 0.4pt,line join=round,line cap=round,fill=fillColor] (398.27,251.11) circle (  1.16);

\path[draw=drawColor,line width= 0.4pt,line join=round,line cap=round,fill=fillColor] (398.71,251.10) circle (  1.16);

\path[draw=drawColor,line width= 0.4pt,line join=round,line cap=round,fill=fillColor] (399.15,251.06) circle (  1.16);

\path[draw=drawColor,line width= 0.4pt,line join=round,line cap=round,fill=fillColor] (399.58,250.96) circle (  1.16);

\path[draw=drawColor,line width= 0.4pt,line join=round,line cap=round,fill=fillColor] (400.01,250.53) circle (  1.16);

\path[draw=drawColor,line width= 0.4pt,line join=round,line cap=round,fill=fillColor] (400.43,250.17) circle (  1.16);

\path[draw=drawColor,line width= 0.4pt,line join=round,line cap=round,fill=fillColor] (400.85,249.83) circle (  1.16);

\path[draw=drawColor,line width= 0.4pt,line join=round,line cap=round,fill=fillColor] (401.27,249.72) circle (  1.16);

\path[draw=drawColor,line width= 0.4pt,line join=round,line cap=round,fill=fillColor] (401.69,249.47) circle (  1.16);

\path[draw=drawColor,line width= 0.4pt,line join=round,line cap=round,fill=fillColor] (402.10,248.97) circle (  1.16);

\path[draw=drawColor,line width= 0.4pt,line join=round,line cap=round,fill=fillColor] (402.50,248.95) circle (  1.16);

\path[draw=drawColor,line width= 0.4pt,line join=round,line cap=round,fill=fillColor] (402.91,248.59) circle (  1.16);

\path[draw=drawColor,line width= 0.4pt,line join=round,line cap=round,fill=fillColor] (403.31,248.43) circle (  1.16);

\path[draw=drawColor,line width= 0.4pt,line join=round,line cap=round,fill=fillColor] (403.70,248.30) circle (  1.16);

\path[draw=drawColor,line width= 0.4pt,line join=round,line cap=round,fill=fillColor] (404.10,248.21) circle (  1.16);

\path[draw=drawColor,line width= 0.4pt,line join=round,line cap=round,fill=fillColor] (404.49,248.18) circle (  1.16);

\path[draw=drawColor,line width= 0.4pt,line join=round,line cap=round,fill=fillColor] (404.88,248.10) circle (  1.16);

\path[draw=drawColor,line width= 0.4pt,line join=round,line cap=round,fill=fillColor] (405.26,247.69) circle (  1.16);

\path[draw=drawColor,line width= 0.4pt,line join=round,line cap=round,fill=fillColor] (405.64,247.48) circle (  1.16);

\path[draw=drawColor,line width= 0.4pt,line join=round,line cap=round,fill=fillColor] (406.02,247.22) circle (  1.16);

\path[draw=drawColor,line width= 0.4pt,line join=round,line cap=round,fill=fillColor] (406.40,246.89) circle (  1.16);

\path[draw=drawColor,line width= 0.4pt,line join=round,line cap=round,fill=fillColor] (406.77,246.57) circle (  1.16);

\path[draw=drawColor,line width= 0.4pt,line join=round,line cap=round,fill=fillColor] (407.14,246.42) circle (  1.16);

\path[draw=drawColor,line width= 0.4pt,line join=round,line cap=round,fill=fillColor] (407.51,246.39) circle (  1.16);

\path[draw=drawColor,line width= 0.4pt,line join=round,line cap=round,fill=fillColor] (407.87,246.28) circle (  1.16);

\path[draw=drawColor,line width= 0.4pt,line join=round,line cap=round,fill=fillColor] (408.24,246.28) circle (  1.16);

\path[draw=drawColor,line width= 0.4pt,line join=round,line cap=round,fill=fillColor] (408.60,246.28) circle (  1.16);

\path[draw=drawColor,line width= 0.4pt,line join=round,line cap=round,fill=fillColor] (408.95,246.20) circle (  1.16);

\path[draw=drawColor,line width= 0.4pt,line join=round,line cap=round,fill=fillColor] (409.31,246.17) circle (  1.16);

\path[draw=drawColor,line width= 0.4pt,line join=round,line cap=round,fill=fillColor] (409.66,246.10) circle (  1.16);

\path[draw=drawColor,line width= 0.4pt,line join=round,line cap=round,fill=fillColor] (410.01,246.06) circle (  1.16);

\path[draw=drawColor,line width= 0.4pt,line join=round,line cap=round,fill=fillColor] (410.36,245.86) circle (  1.16);

\path[draw=drawColor,line width= 0.4pt,line join=round,line cap=round,fill=fillColor] (410.71,245.73) circle (  1.16);

\path[draw=drawColor,line width= 0.4pt,line join=round,line cap=round,fill=fillColor] (411.05,245.55) circle (  1.16);

\path[draw=drawColor,line width= 0.4pt,line join=round,line cap=round,fill=fillColor] (411.39,245.54) circle (  1.16);

\path[draw=drawColor,line width= 0.4pt,line join=round,line cap=round,fill=fillColor] (411.73,245.47) circle (  1.16);

\path[draw=drawColor,line width= 0.4pt,line join=round,line cap=round,fill=fillColor] (412.07,245.39) circle (  1.16);

\path[draw=drawColor,line width= 0.4pt,line join=round,line cap=round,fill=fillColor] (412.40,245.36) circle (  1.16);

\path[draw=drawColor,line width= 0.4pt,line join=round,line cap=round,fill=fillColor] (412.73,245.23) circle (  1.16);

\path[draw=drawColor,line width= 0.4pt,line join=round,line cap=round,fill=fillColor] (413.07,245.22) circle (  1.16);

\path[draw=drawColor,line width= 0.4pt,line join=round,line cap=round,fill=fillColor] (413.39,245.22) circle (  1.16);

\path[draw=drawColor,line width= 0.4pt,line join=round,line cap=round,fill=fillColor] (413.72,245.10) circle (  1.16);

\path[draw=drawColor,line width= 0.4pt,line join=round,line cap=round,fill=fillColor] (414.05,244.87) circle (  1.16);

\path[draw=drawColor,line width= 0.4pt,line join=round,line cap=round,fill=fillColor] (414.37,244.56) circle (  1.16);

\path[draw=drawColor,line width= 0.4pt,line join=round,line cap=round,fill=fillColor] (414.69,243.91) circle (  1.16);

\path[draw=drawColor,line width= 0.4pt,line join=round,line cap=round,fill=fillColor] (415.01,243.89) circle (  1.16);

\path[draw=drawColor,line width= 0.4pt,line join=round,line cap=round,fill=fillColor] (415.33,243.72) circle (  1.16);

\path[draw=drawColor,line width= 0.4pt,line join=round,line cap=round,fill=fillColor] (415.64,243.65) circle (  1.16);

\path[draw=drawColor,line width= 0.4pt,line join=round,line cap=round,fill=fillColor] (415.95,243.58) circle (  1.16);

\path[draw=drawColor,line width= 0.4pt,line join=round,line cap=round,fill=fillColor] (416.27,243.47) circle (  1.16);

\path[draw=drawColor,line width= 0.4pt,line join=round,line cap=round,fill=fillColor] (416.58,243.39) circle (  1.16);

\path[draw=drawColor,line width= 0.4pt,line join=round,line cap=round,fill=fillColor] (416.88,243.30) circle (  1.16);

\path[draw=drawColor,line width= 0.4pt,line join=round,line cap=round,fill=fillColor] (417.19,243.16) circle (  1.16);

\path[draw=drawColor,line width= 0.4pt,line join=round,line cap=round,fill=fillColor] (417.50,243.15) circle (  1.16);

\path[draw=drawColor,line width= 0.4pt,line join=round,line cap=round,fill=fillColor] (417.80,243.08) circle (  1.16);

\path[draw=drawColor,line width= 0.4pt,line join=round,line cap=round,fill=fillColor] (418.10,243.03) circle (  1.16);

\path[draw=drawColor,line width= 0.4pt,line join=round,line cap=round,fill=fillColor] (418.40,242.97) circle (  1.16);

\path[draw=drawColor,line width= 0.4pt,line join=round,line cap=round,fill=fillColor] (418.70,242.93) circle (  1.16);

\path[draw=drawColor,line width= 0.4pt,line join=round,line cap=round,fill=fillColor] (419.00,242.91) circle (  1.16);

\path[draw=drawColor,line width= 0.4pt,line join=round,line cap=round,fill=fillColor] (419.29,242.90) circle (  1.16);

\path[draw=drawColor,line width= 0.4pt,line join=round,line cap=round,fill=fillColor] (419.59,242.89) circle (  1.16);

\path[draw=drawColor,line width= 0.4pt,line join=round,line cap=round,fill=fillColor] (419.88,242.88) circle (  1.16);

\path[draw=drawColor,line width= 0.4pt,line join=round,line cap=round,fill=fillColor] (420.17,242.76) circle (  1.16);

\path[draw=drawColor,line width= 0.4pt,line join=round,line cap=round,fill=fillColor] (420.46,242.68) circle (  1.16);

\path[draw=drawColor,line width= 0.4pt,line join=round,line cap=round,fill=fillColor] (420.75,242.65) circle (  1.16);

\path[draw=drawColor,line width= 0.4pt,line join=round,line cap=round,fill=fillColor] (421.03,242.60) circle (  1.16);

\path[draw=drawColor,line width= 0.4pt,line join=round,line cap=round,fill=fillColor] (421.32,242.57) circle (  1.16);

\path[draw=drawColor,line width= 0.4pt,line join=round,line cap=round,fill=fillColor] (421.60,242.41) circle (  1.16);

\path[draw=drawColor,line width= 0.4pt,line join=round,line cap=round,fill=fillColor] (421.89,242.37) circle (  1.16);

\path[draw=drawColor,line width= 0.4pt,line join=round,line cap=round,fill=fillColor] (422.17,242.03) circle (  1.16);

\path[draw=drawColor,line width= 0.4pt,line join=round,line cap=round,fill=fillColor] (422.45,241.95) circle (  1.16);

\path[draw=drawColor,line width= 0.4pt,line join=round,line cap=round,fill=fillColor] (422.72,241.89) circle (  1.16);

\path[draw=drawColor,line width= 0.4pt,line join=round,line cap=round,fill=fillColor] (423.00,241.72) circle (  1.16);

\path[draw=drawColor,line width= 0.4pt,line join=round,line cap=round,fill=fillColor] (423.28,241.67) circle (  1.16);

\path[draw=drawColor,line width= 0.4pt,line join=round,line cap=round,fill=fillColor] (423.55,241.61) circle (  1.16);

\path[draw=drawColor,line width= 0.4pt,line join=round,line cap=round,fill=fillColor] (423.82,241.36) circle (  1.16);

\path[draw=drawColor,line width= 0.4pt,line join=round,line cap=round,fill=fillColor] (424.10,241.32) circle (  1.16);

\path[draw=drawColor,line width= 0.4pt,line join=round,line cap=round,fill=fillColor] (424.37,241.31) circle (  1.16);

\path[draw=drawColor,line width= 0.4pt,line join=round,line cap=round,fill=fillColor] (424.64,241.27) circle (  1.16);

\path[draw=drawColor,line width= 0.4pt,line join=round,line cap=round,fill=fillColor] (424.91,241.13) circle (  1.16);

\path[draw=drawColor,line width= 0.4pt,line join=round,line cap=round,fill=fillColor] (425.17,241.00) circle (  1.16);

\path[draw=drawColor,line width= 0.4pt,line join=round,line cap=round,fill=fillColor] (425.44,240.97) circle (  1.16);

\path[draw=drawColor,line width= 0.4pt,line join=round,line cap=round,fill=fillColor] (425.70,240.92) circle (  1.16);

\path[draw=drawColor,line width= 0.4pt,line join=round,line cap=round,fill=fillColor] (425.97,240.87) circle (  1.16);

\path[draw=drawColor,line width= 0.4pt,line join=round,line cap=round,fill=fillColor] (426.23,240.84) circle (  1.16);

\path[draw=drawColor,line width= 0.4pt,line join=round,line cap=round,fill=fillColor] (426.49,240.77) circle (  1.16);

\path[draw=drawColor,line width= 0.4pt,line join=round,line cap=round,fill=fillColor] (426.75,240.66) circle (  1.16);

\path[draw=drawColor,line width= 0.4pt,line join=round,line cap=round,fill=fillColor] (427.01,240.63) circle (  1.16);

\path[draw=drawColor,line width= 0.4pt,line join=round,line cap=round,fill=fillColor] (427.27,240.60) circle (  1.16);

\path[draw=drawColor,line width= 0.4pt,line join=round,line cap=round,fill=fillColor] (427.52,240.59) circle (  1.16);

\path[draw=drawColor,line width= 0.4pt,line join=round,line cap=round,fill=fillColor] (427.78,240.56) circle (  1.16);

\path[draw=drawColor,line width= 0.4pt,line join=round,line cap=round,fill=fillColor] (428.03,240.54) circle (  1.16);

\path[draw=drawColor,line width= 0.4pt,line join=round,line cap=round,fill=fillColor] (428.29,240.41) circle (  1.16);

\path[draw=drawColor,line width= 0.4pt,line join=round,line cap=round,fill=fillColor] (428.54,240.27) circle (  1.16);

\path[draw=drawColor,line width= 0.4pt,line join=round,line cap=round,fill=fillColor] (428.79,240.26) circle (  1.16);

\path[draw=drawColor,line width= 0.4pt,line join=round,line cap=round,fill=fillColor] (429.04,240.21) circle (  1.16);

\path[draw=drawColor,line width= 0.4pt,line join=round,line cap=round,fill=fillColor] (429.29,240.14) circle (  1.16);

\path[draw=drawColor,line width= 0.4pt,line join=round,line cap=round,fill=fillColor] (429.54,240.11) circle (  1.16);

\path[draw=drawColor,line width= 0.4pt,line join=round,line cap=round,fill=fillColor] (429.79,240.07) circle (  1.16);

\path[draw=drawColor,line width= 0.4pt,line join=round,line cap=round,fill=fillColor] (430.04,239.95) circle (  1.16);

\path[draw=drawColor,line width= 0.4pt,line join=round,line cap=round,fill=fillColor] (430.28,239.93) circle (  1.16);

\path[draw=drawColor,line width= 0.4pt,line join=round,line cap=round,fill=fillColor] (430.53,239.73) circle (  1.16);

\path[draw=drawColor,line width= 0.4pt,line join=round,line cap=round,fill=fillColor] (430.77,239.54) circle (  1.16);

\path[draw=drawColor,line width= 0.4pt,line join=round,line cap=round,fill=fillColor] (431.01,239.51) circle (  1.16);

\path[draw=drawColor,line width= 0.4pt,line join=round,line cap=round,fill=fillColor] (431.25,239.50) circle (  1.16);

\path[draw=drawColor,line width= 0.4pt,line join=round,line cap=round,fill=fillColor] (431.50,239.50) circle (  1.16);

\path[draw=drawColor,line width= 0.4pt,line join=round,line cap=round,fill=fillColor] (431.74,239.44) circle (  1.16);

\path[draw=drawColor,line width= 0.4pt,line join=round,line cap=round,fill=fillColor] (431.97,239.37) circle (  1.16);

\path[draw=drawColor,line width= 0.4pt,line join=round,line cap=round,fill=fillColor] (432.21,239.32) circle (  1.16);

\path[draw=drawColor,line width= 0.4pt,line join=round,line cap=round,fill=fillColor] (432.45,239.28) circle (  1.16);

\path[draw=drawColor,line width= 0.4pt,line join=round,line cap=round,fill=fillColor] (432.69,239.27) circle (  1.16);

\path[draw=drawColor,line width= 0.4pt,line join=round,line cap=round,fill=fillColor] (432.92,239.11) circle (  1.16);

\path[draw=drawColor,line width= 0.4pt,line join=round,line cap=round,fill=fillColor] (433.16,239.09) circle (  1.16);

\path[draw=drawColor,line width= 0.4pt,line join=round,line cap=round,fill=fillColor] (433.39,238.87) circle (  1.16);

\path[draw=drawColor,line width= 0.4pt,line join=round,line cap=round,fill=fillColor] (433.62,238.87) circle (  1.16);

\path[draw=drawColor,line width= 0.4pt,line join=round,line cap=round,fill=fillColor] (433.86,238.83) circle (  1.16);

\path[draw=drawColor,line width= 0.4pt,line join=round,line cap=round,fill=fillColor] (434.09,238.72) circle (  1.16);

\path[draw=drawColor,line width= 0.4pt,line join=round,line cap=round,fill=fillColor] (434.32,238.71) circle (  1.16);

\path[draw=drawColor,line width= 0.4pt,line join=round,line cap=round,fill=fillColor] (434.55,238.68) circle (  1.16);

\path[draw=drawColor,line width= 0.4pt,line join=round,line cap=round,fill=fillColor] (434.78,238.68) circle (  1.16);

\path[draw=drawColor,line width= 0.4pt,line join=round,line cap=round,fill=fillColor] (435.00,238.60) circle (  1.16);

\path[draw=drawColor,line width= 0.4pt,line join=round,line cap=round,fill=fillColor] (435.23,238.60) circle (  1.16);

\path[draw=drawColor,line width= 0.4pt,line join=round,line cap=round,fill=fillColor] (435.46,238.57) circle (  1.16);

\path[draw=drawColor,line width= 0.4pt,line join=round,line cap=round,fill=fillColor] (435.68,238.51) circle (  1.16);

\path[draw=drawColor,line width= 0.4pt,line join=round,line cap=round,fill=fillColor] (435.91,238.50) circle (  1.16);

\path[draw=drawColor,line width= 0.4pt,line join=round,line cap=round,fill=fillColor] (436.13,238.48) circle (  1.16);

\path[draw=drawColor,line width= 0.4pt,line join=round,line cap=round,fill=fillColor] (436.36,238.15) circle (  1.16);

\path[draw=drawColor,line width= 0.4pt,line join=round,line cap=round,fill=fillColor] (436.58,238.01) circle (  1.16);

\path[draw=drawColor,line width= 0.4pt,line join=round,line cap=round,fill=fillColor] (436.80,238.01) circle (  1.16);

\path[draw=drawColor,line width= 0.4pt,line join=round,line cap=round,fill=fillColor] (437.02,237.90) circle (  1.16);

\path[draw=drawColor,line width= 0.4pt,line join=round,line cap=round,fill=fillColor] (437.24,237.84) circle (  1.16);

\path[draw=drawColor,line width= 0.4pt,line join=round,line cap=round,fill=fillColor] (437.46,237.83) circle (  1.16);

\path[draw=drawColor,line width= 0.4pt,line join=round,line cap=round,fill=fillColor] (437.68,237.72) circle (  1.16);

\path[draw=drawColor,line width= 0.4pt,line join=round,line cap=round,fill=fillColor] (437.90,237.67) circle (  1.16);

\path[draw=drawColor,line width= 0.4pt,line join=round,line cap=round,fill=fillColor] (438.12,237.61) circle (  1.16);

\path[draw=drawColor,line width= 0.4pt,line join=round,line cap=round,fill=fillColor] (438.33,237.55) circle (  1.16);

\path[draw=drawColor,line width= 0.4pt,line join=round,line cap=round,fill=fillColor] (438.55,237.38) circle (  1.16);

\path[draw=drawColor,line width= 0.4pt,line join=round,line cap=round,fill=fillColor] (438.76,237.35) circle (  1.16);

\path[draw=drawColor,line width= 0.4pt,line join=round,line cap=round,fill=fillColor] (438.98,237.34) circle (  1.16);

\path[draw=drawColor,line width= 0.4pt,line join=round,line cap=round,fill=fillColor] (439.19,237.33) circle (  1.16);

\path[draw=drawColor,line width= 0.4pt,line join=round,line cap=round,fill=fillColor] (439.41,237.14) circle (  1.16);

\path[draw=drawColor,line width= 0.4pt,line join=round,line cap=round,fill=fillColor] (439.62,237.14) circle (  1.16);

\path[draw=drawColor,line width= 0.4pt,line join=round,line cap=round,fill=fillColor] (439.83,237.07) circle (  1.16);

\path[draw=drawColor,line width= 0.4pt,line join=round,line cap=round,fill=fillColor] (440.04,237.01) circle (  1.16);

\path[draw=drawColor,line width= 0.4pt,line join=round,line cap=round,fill=fillColor] (440.25,237.00) circle (  1.16);

\path[draw=drawColor,line width= 0.4pt,line join=round,line cap=round,fill=fillColor] (440.46,236.97) circle (  1.16);

\path[draw=drawColor,line width= 0.4pt,line join=round,line cap=round,fill=fillColor] (440.67,236.95) circle (  1.16);

\path[draw=drawColor,line width= 0.4pt,line join=round,line cap=round,fill=fillColor] (440.88,236.95) circle (  1.16);

\path[draw=drawColor,line width= 0.4pt,line join=round,line cap=round,fill=fillColor] (441.09,236.92) circle (  1.16);

\path[draw=drawColor,line width= 0.4pt,line join=round,line cap=round,fill=fillColor] (441.29,236.89) circle (  1.16);

\path[draw=drawColor,line width= 0.4pt,line join=round,line cap=round,fill=fillColor] (441.50,236.89) circle (  1.16);

\path[draw=drawColor,line width= 0.4pt,line join=round,line cap=round,fill=fillColor] (441.71,236.84) circle (  1.16);

\path[draw=drawColor,line width= 0.4pt,line join=round,line cap=round,fill=fillColor] (441.91,236.80) circle (  1.16);

\path[draw=drawColor,line width= 0.4pt,line join=round,line cap=round,fill=fillColor] (442.12,236.79) circle (  1.16);

\path[draw=drawColor,line width= 0.4pt,line join=round,line cap=round,fill=fillColor] (442.32,236.76) circle (  1.16);

\path[draw=drawColor,line width= 0.4pt,line join=round,line cap=round,fill=fillColor] (442.53,236.71) circle (  1.16);

\path[draw=drawColor,line width= 0.4pt,line join=round,line cap=round,fill=fillColor] (442.73,236.63) circle (  1.16);

\path[draw=drawColor,line width= 0.4pt,line join=round,line cap=round,fill=fillColor] (442.93,236.61) circle (  1.16);

\path[draw=drawColor,line width= 0.4pt,line join=round,line cap=round,fill=fillColor] (443.13,236.52) circle (  1.16);

\path[draw=drawColor,line width= 0.4pt,line join=round,line cap=round,fill=fillColor] (443.33,236.48) circle (  1.16);

\path[draw=drawColor,line width= 0.4pt,line join=round,line cap=round,fill=fillColor] (443.54,236.31) circle (  1.16);

\path[draw=drawColor,line width= 0.4pt,line join=round,line cap=round,fill=fillColor] (443.74,236.29) circle (  1.16);

\path[draw=drawColor,line width= 0.4pt,line join=round,line cap=round,fill=fillColor] (443.94,236.20) circle (  1.16);

\path[draw=drawColor,line width= 0.4pt,line join=round,line cap=round,fill=fillColor] (444.13,236.16) circle (  1.16);

\path[draw=drawColor,line width= 0.4pt,line join=round,line cap=round,fill=fillColor] (444.33,236.16) circle (  1.16);

\path[draw=drawColor,line width= 0.4pt,line join=round,line cap=round,fill=fillColor] (444.53,235.99) circle (  1.16);

\path[draw=drawColor,line width= 0.4pt,line join=round,line cap=round,fill=fillColor] (444.73,235.96) circle (  1.16);

\path[draw=drawColor,line width= 0.4pt,line join=round,line cap=round,fill=fillColor] (444.92,235.95) circle (  1.16);

\path[draw=drawColor,line width= 0.4pt,line join=round,line cap=round,fill=fillColor] (445.12,235.93) circle (  1.16);

\path[draw=drawColor,line width= 0.4pt,line join=round,line cap=round,fill=fillColor] (445.32,235.90) circle (  1.16);

\path[draw=drawColor,line width= 0.4pt,line join=round,line cap=round,fill=fillColor] (445.51,235.82) circle (  1.16);

\path[draw=drawColor,line width= 0.4pt,line join=round,line cap=round,fill=fillColor] (445.71,235.76) circle (  1.16);

\path[draw=drawColor,line width= 0.4pt,line join=round,line cap=round,fill=fillColor] (445.90,235.64) circle (  1.16);

\path[draw=drawColor,line width= 0.4pt,line join=round,line cap=round,fill=fillColor] (446.09,235.61) circle (  1.16);

\path[draw=drawColor,line width= 0.4pt,line join=round,line cap=round,fill=fillColor] (446.29,235.56) circle (  1.16);

\path[draw=drawColor,line width= 0.4pt,line join=round,line cap=round,fill=fillColor] (446.48,235.48) circle (  1.16);

\path[draw=drawColor,line width= 0.4pt,line join=round,line cap=round,fill=fillColor] (446.67,235.37) circle (  1.16);

\path[draw=drawColor,line width= 0.4pt,line join=round,line cap=round,fill=fillColor] (446.86,235.35) circle (  1.16);

\path[draw=drawColor,line width= 0.4pt,line join=round,line cap=round,fill=fillColor] (447.05,235.28) circle (  1.16);

\path[draw=drawColor,line width= 0.4pt,line join=round,line cap=round,fill=fillColor] (447.24,235.27) circle (  1.16);

\path[draw=drawColor,line width= 0.4pt,line join=round,line cap=round,fill=fillColor] (447.43,235.27) circle (  1.16);

\path[draw=drawColor,line width= 0.4pt,line join=round,line cap=round,fill=fillColor] (447.62,235.20) circle (  1.16);

\path[draw=drawColor,line width= 0.4pt,line join=round,line cap=round,fill=fillColor] (447.81,235.17) circle (  1.16);

\path[draw=drawColor,line width= 0.4pt,line join=round,line cap=round,fill=fillColor] (448.00,235.16) circle (  1.16);

\path[draw=drawColor,line width= 0.4pt,line join=round,line cap=round,fill=fillColor] (448.19,235.13) circle (  1.16);

\path[draw=drawColor,line width= 0.4pt,line join=round,line cap=round,fill=fillColor] (448.37,235.03) circle (  1.16);

\path[draw=drawColor,line width= 0.4pt,line join=round,line cap=round,fill=fillColor] (448.56,235.01) circle (  1.16);

\path[draw=drawColor,line width= 0.4pt,line join=round,line cap=round,fill=fillColor] (448.75,234.95) circle (  1.16);

\path[draw=drawColor,line width= 0.4pt,line join=round,line cap=round,fill=fillColor] (448.93,234.81) circle (  1.16);

\path[draw=drawColor,line width= 0.4pt,line join=round,line cap=round,fill=fillColor] (449.12,234.73) circle (  1.16);

\path[draw=drawColor,line width= 0.4pt,line join=round,line cap=round,fill=fillColor] (449.30,234.58) circle (  1.16);

\path[draw=drawColor,line width= 0.4pt,line join=round,line cap=round,fill=fillColor] (449.49,234.54) circle (  1.16);

\path[draw=drawColor,line width= 0.4pt,line join=round,line cap=round,fill=fillColor] (449.67,234.49) circle (  1.16);

\path[draw=drawColor,line width= 0.4pt,line join=round,line cap=round,fill=fillColor] (449.86,234.45) circle (  1.16);

\path[draw=drawColor,line width= 0.4pt,line join=round,line cap=round,fill=fillColor] (450.04,234.44) circle (  1.16);

\path[draw=drawColor,line width= 0.4pt,line join=round,line cap=round,fill=fillColor] (450.22,234.34) circle (  1.16);

\path[draw=drawColor,line width= 0.4pt,line join=round,line cap=round,fill=fillColor] (450.40,234.34) circle (  1.16);

\path[draw=drawColor,line width= 0.4pt,line join=round,line cap=round,fill=fillColor] (450.58,234.24) circle (  1.16);

\path[draw=drawColor,line width= 0.4pt,line join=round,line cap=round,fill=fillColor] (450.77,234.17) circle (  1.16);

\path[draw=drawColor,line width= 0.4pt,line join=round,line cap=round,fill=fillColor] (450.95,234.00) circle (  1.16);

\path[draw=drawColor,line width= 0.4pt,line join=round,line cap=round,fill=fillColor] (451.13,233.90) circle (  1.16);

\path[draw=drawColor,line width= 0.4pt,line join=round,line cap=round,fill=fillColor] (451.31,233.89) circle (  1.16);

\path[draw=drawColor,line width= 0.4pt,line join=round,line cap=round,fill=fillColor] (451.49,233.84) circle (  1.16);

\path[draw=drawColor,line width= 0.4pt,line join=round,line cap=round,fill=fillColor] (451.67,233.84) circle (  1.16);

\path[draw=drawColor,line width= 0.4pt,line join=round,line cap=round,fill=fillColor] (451.84,233.75) circle (  1.16);

\path[draw=drawColor,line width= 0.4pt,line join=round,line cap=round,fill=fillColor] (452.02,233.74) circle (  1.16);

\path[draw=drawColor,line width= 0.4pt,line join=round,line cap=round,fill=fillColor] (452.20,233.73) circle (  1.16);

\path[draw=drawColor,line width= 0.4pt,line join=round,line cap=round,fill=fillColor] (452.38,233.69) circle (  1.16);

\path[draw=drawColor,line width= 0.4pt,line join=round,line cap=round,fill=fillColor] (452.55,233.55) circle (  1.16);

\path[draw=drawColor,line width= 0.4pt,line join=round,line cap=round,fill=fillColor] (452.73,233.50) circle (  1.16);

\path[draw=drawColor,line width= 0.4pt,line join=round,line cap=round,fill=fillColor] (452.91,233.46) circle (  1.16);

\path[draw=drawColor,line width= 0.4pt,line join=round,line cap=round,fill=fillColor] (453.08,233.45) circle (  1.16);

\path[draw=drawColor,line width= 0.4pt,line join=round,line cap=round,fill=fillColor] (453.26,233.45) circle (  1.16);

\path[draw=drawColor,line width= 0.4pt,line join=round,line cap=round,fill=fillColor] (453.43,233.34) circle (  1.16);

\path[draw=drawColor,line width= 0.4pt,line join=round,line cap=round,fill=fillColor] (453.61,233.31) circle (  1.16);

\path[draw=drawColor,line width= 0.4pt,line join=round,line cap=round,fill=fillColor] (453.78,233.31) circle (  1.16);

\path[draw=drawColor,line width= 0.4pt,line join=round,line cap=round,fill=fillColor] (453.95,233.29) circle (  1.16);

\path[draw=drawColor,line width= 0.4pt,line join=round,line cap=round,fill=fillColor] (454.13,233.28) circle (  1.16);

\path[draw=drawColor,line width= 0.4pt,line join=round,line cap=round,fill=fillColor] (454.30,233.16) circle (  1.16);

\path[draw=drawColor,line width= 0.4pt,line join=round,line cap=round,fill=fillColor] (454.47,233.07) circle (  1.16);

\path[draw=drawColor,line width= 0.4pt,line join=round,line cap=round,fill=fillColor] (454.64,233.05) circle (  1.16);

\path[draw=drawColor,line width= 0.4pt,line join=round,line cap=round,fill=fillColor] (454.82,232.95) circle (  1.16);

\path[draw=drawColor,line width= 0.4pt,line join=round,line cap=round,fill=fillColor] (454.99,232.92) circle (  1.16);

\path[draw=drawColor,line width= 0.4pt,line join=round,line cap=round,fill=fillColor] (455.16,232.87) circle (  1.16);

\path[draw=drawColor,line width= 0.4pt,line join=round,line cap=round,fill=fillColor] (455.33,232.79) circle (  1.16);

\path[draw=drawColor,line width= 0.4pt,line join=round,line cap=round,fill=fillColor] (455.50,232.55) circle (  1.16);

\path[draw=drawColor,line width= 0.4pt,line join=round,line cap=round,fill=fillColor] (455.67,232.51) circle (  1.16);

\path[draw=drawColor,line width= 0.4pt,line join=round,line cap=round,fill=fillColor] (455.84,232.49) circle (  1.16);

\path[draw=drawColor,line width= 0.4pt,line join=round,line cap=round,fill=fillColor] (456.00,232.48) circle (  1.16);

\path[draw=drawColor,line width= 0.4pt,line join=round,line cap=round,fill=fillColor] (456.17,232.46) circle (  1.16);

\path[draw=drawColor,line width= 0.4pt,line join=round,line cap=round,fill=fillColor] (456.34,232.33) circle (  1.16);

\path[draw=drawColor,line width= 0.4pt,line join=round,line cap=round,fill=fillColor] (456.51,232.33) circle (  1.16);

\path[draw=drawColor,line width= 0.4pt,line join=round,line cap=round,fill=fillColor] (456.68,232.32) circle (  1.16);

\path[draw=drawColor,line width= 0.4pt,line join=round,line cap=round,fill=fillColor] (456.84,232.25) circle (  1.16);

\path[draw=drawColor,line width= 0.4pt,line join=round,line cap=round,fill=fillColor] (457.01,231.99) circle (  1.16);

\path[draw=drawColor,line width= 0.4pt,line join=round,line cap=round,fill=fillColor] (457.18,231.89) circle (  1.16);

\path[draw=drawColor,line width= 0.4pt,line join=round,line cap=round,fill=fillColor] (457.34,231.84) circle (  1.16);

\path[draw=drawColor,line width= 0.4pt,line join=round,line cap=round,fill=fillColor] (457.51,231.69) circle (  1.16);

\path[draw=drawColor,line width= 0.4pt,line join=round,line cap=round,fill=fillColor] (457.67,231.66) circle (  1.16);

\path[draw=drawColor,line width= 0.4pt,line join=round,line cap=round,fill=fillColor] (457.84,231.55) circle (  1.16);

\path[draw=drawColor,line width= 0.4pt,line join=round,line cap=round,fill=fillColor] (458.00,231.25) circle (  1.16);

\path[draw=drawColor,line width= 0.4pt,line join=round,line cap=round,fill=fillColor] (458.17,231.25) circle (  1.16);

\path[draw=drawColor,line width= 0.4pt,line join=round,line cap=round,fill=fillColor] (458.33,230.90) circle (  1.16);

\path[draw=drawColor,line width= 0.4pt,line join=round,line cap=round,fill=fillColor] (458.49,230.81) circle (  1.16);

\path[draw=drawColor,line width= 0.4pt,line join=round,line cap=round,fill=fillColor] (458.66,230.72) circle (  1.16);

\path[draw=drawColor,line width= 0.4pt,line join=round,line cap=round,fill=fillColor] (458.82,230.68) circle (  1.16);

\path[draw=drawColor,line width= 0.4pt,line join=round,line cap=round,fill=fillColor] (458.98,230.56) circle (  1.16);

\path[draw=drawColor,line width= 0.4pt,line join=round,line cap=round,fill=fillColor] (459.14,230.50) circle (  1.16);

\path[draw=drawColor,line width= 0.4pt,line join=round,line cap=round,fill=fillColor] (459.30,230.22) circle (  1.16);

\path[draw=drawColor,line width= 0.4pt,line join=round,line cap=round,fill=fillColor] (459.47,230.13) circle (  1.16);

\path[draw=drawColor,line width= 0.4pt,line join=round,line cap=round,fill=fillColor] (459.63,230.07) circle (  1.16);

\path[draw=drawColor,line width= 0.4pt,line join=round,line cap=round,fill=fillColor] (459.79,229.94) circle (  1.16);

\path[draw=drawColor,line width= 0.4pt,line join=round,line cap=round,fill=fillColor] (459.95,229.94) circle (  1.16);

\path[draw=drawColor,line width= 0.4pt,line join=round,line cap=round,fill=fillColor] (460.11,229.89) circle (  1.16);

\path[draw=drawColor,line width= 0.4pt,line join=round,line cap=round,fill=fillColor] (460.27,229.86) circle (  1.16);

\path[draw=drawColor,line width= 0.4pt,line join=round,line cap=round,fill=fillColor] (460.43,229.82) circle (  1.16);

\path[draw=drawColor,line width= 0.4pt,line join=round,line cap=round,fill=fillColor] (460.59,229.73) circle (  1.16);

\path[draw=drawColor,line width= 0.4pt,line join=round,line cap=round,fill=fillColor] (460.74,229.19) circle (  1.16);

\path[draw=drawColor,line width= 0.4pt,line join=round,line cap=round,fill=fillColor] (460.90,229.08) circle (  1.16);

\path[draw=drawColor,line width= 0.4pt,line join=round,line cap=round,fill=fillColor] (461.06,228.98) circle (  1.16);

\path[draw=drawColor,line width= 0.4pt,line join=round,line cap=round,fill=fillColor] (461.22,228.89) circle (  1.16);

\path[draw=drawColor,line width= 0.4pt,line join=round,line cap=round,fill=fillColor] (461.38,228.70) circle (  1.16);

\path[draw=drawColor,line width= 0.4pt,line join=round,line cap=round,fill=fillColor] (461.53,228.63) circle (  1.16);

\path[draw=drawColor,line width= 0.4pt,line join=round,line cap=round,fill=fillColor] (461.69,228.37) circle (  1.16);

\path[draw=drawColor,line width= 0.4pt,line join=round,line cap=round,fill=fillColor] (461.85,228.27) circle (  1.16);

\path[draw=drawColor,line width= 0.4pt,line join=round,line cap=round,fill=fillColor] (462.00,228.21) circle (  1.16);

\path[draw=drawColor,line width= 0.4pt,line join=round,line cap=round,fill=fillColor] (462.16,228.17) circle (  1.16);

\path[draw=drawColor,line width= 0.4pt,line join=round,line cap=round,fill=fillColor] (462.31,228.15) circle (  1.16);

\path[draw=drawColor,line width= 0.4pt,line join=round,line cap=round,fill=fillColor] (462.47,228.05) circle (  1.16);

\path[draw=drawColor,line width= 0.4pt,line join=round,line cap=round,fill=fillColor] (462.62,228.00) circle (  1.16);

\path[draw=drawColor,line width= 0.4pt,line join=round,line cap=round,fill=fillColor] (462.78,227.99) circle (  1.16);

\path[draw=drawColor,line width= 0.4pt,line join=round,line cap=round,fill=fillColor] (462.93,227.86) circle (  1.16);

\path[draw=drawColor,line width= 0.4pt,line join=round,line cap=round,fill=fillColor] (463.09,227.76) circle (  1.16);

\path[draw=drawColor,line width= 0.4pt,line join=round,line cap=round,fill=fillColor] (463.24,227.73) circle (  1.16);

\path[draw=drawColor,line width= 0.4pt,line join=round,line cap=round,fill=fillColor] (463.39,227.72) circle (  1.16);

\path[draw=drawColor,line width= 0.4pt,line join=round,line cap=round,fill=fillColor] (463.55,227.47) circle (  1.16);

\path[draw=drawColor,line width= 0.4pt,line join=round,line cap=round,fill=fillColor] (463.70,227.34) circle (  1.16);

\path[draw=drawColor,line width= 0.4pt,line join=round,line cap=round,fill=fillColor] (463.85,227.12) circle (  1.16);

\path[draw=drawColor,line width= 0.4pt,line join=round,line cap=round,fill=fillColor] (464.00,226.94) circle (  1.16);

\path[draw=drawColor,line width= 0.4pt,line join=round,line cap=round,fill=fillColor] (464.16,226.83) circle (  1.16);

\path[draw=drawColor,line width= 0.4pt,line join=round,line cap=round,fill=fillColor] (464.31,226.78) circle (  1.16);

\path[draw=drawColor,line width= 0.4pt,line join=round,line cap=round,fill=fillColor] (464.46,226.68) circle (  1.16);

\path[draw=drawColor,line width= 0.4pt,line join=round,line cap=round,fill=fillColor] (464.61,226.46) circle (  1.16);

\path[draw=drawColor,line width= 0.4pt,line join=round,line cap=round,fill=fillColor] (464.76,226.18) circle (  1.16);

\path[draw=drawColor,line width= 0.4pt,line join=round,line cap=round,fill=fillColor] (464.91,226.03) circle (  1.16);

\path[draw=drawColor,line width= 0.4pt,line join=round,line cap=round,fill=fillColor] (465.06,225.62) circle (  1.16);

\path[draw=drawColor,line width= 0.4pt,line join=round,line cap=round,fill=fillColor] (465.21,225.61) circle (  1.16);

\path[draw=drawColor,line width= 0.4pt,line join=round,line cap=round,fill=fillColor] (465.36,225.57) circle (  1.16);

\path[draw=drawColor,line width= 0.4pt,line join=round,line cap=round,fill=fillColor] (465.51,225.55) circle (  1.16);

\path[draw=drawColor,line width= 0.4pt,line join=round,line cap=round,fill=fillColor] (465.66,224.91) circle (  1.16);

\path[draw=drawColor,line width= 0.4pt,line join=round,line cap=round,fill=fillColor] (465.81,224.80) circle (  1.16);

\path[draw=drawColor,line width= 0.4pt,line join=round,line cap=round,fill=fillColor] (465.96,224.20) circle (  1.16);

\path[draw=drawColor,line width= 0.4pt,line join=round,line cap=round,fill=fillColor] (466.10,224.06) circle (  1.16);

\path[draw=drawColor,line width= 0.4pt,line join=round,line cap=round,fill=fillColor] (466.25,223.95) circle (  1.16);

\path[draw=drawColor,line width= 0.4pt,line join=round,line cap=round,fill=fillColor] (466.40,223.23) circle (  1.16);

\path[draw=drawColor,line width= 0.4pt,line join=round,line cap=round,fill=fillColor] (466.55,223.18) circle (  1.16);

\path[draw=drawColor,line width= 0.4pt,line join=round,line cap=round,fill=fillColor] (466.70,223.11) circle (  1.16);

\path[draw=drawColor,line width= 0.4pt,line join=round,line cap=round,fill=fillColor] (466.84,222.59) circle (  1.16);

\path[draw=drawColor,line width= 0.4pt,line join=round,line cap=round,fill=fillColor] (466.99,222.58) circle (  1.16);

\path[draw=drawColor,line width= 0.4pt,line join=round,line cap=round,fill=fillColor] (467.14,221.85) circle (  1.16);

\path[draw=drawColor,line width= 0.4pt,line join=round,line cap=round,fill=fillColor] (467.28,221.42) circle (  1.16);

\path[draw=drawColor,line width= 0.4pt,line join=round,line cap=round,fill=fillColor] (467.43,220.92) circle (  1.16);

\path[draw=drawColor,line width= 0.4pt,line join=round,line cap=round,fill=fillColor] (467.57,220.24) circle (  1.16);

\path[draw=drawColor,line width= 0.4pt,line join=round,line cap=round,fill=fillColor] (467.72,220.14) circle (  1.16);

\path[draw=drawColor,line width= 0.4pt,line join=round,line cap=round,fill=fillColor] (467.86,219.75) circle (  1.16);

\path[draw=drawColor,line width= 0.4pt,line join=round,line cap=round,fill=fillColor] (468.01,219.65) circle (  1.16);

\path[draw=drawColor,line width= 0.4pt,line join=round,line cap=round,fill=fillColor] (468.15,219.42) circle (  1.16);

\path[draw=drawColor,line width= 0.4pt,line join=round,line cap=round,fill=fillColor] (468.30,219.33) circle (  1.16);

\path[draw=drawColor,line width= 0.4pt,line join=round,line cap=round,fill=fillColor] (468.44,219.31) circle (  1.16);

\path[draw=drawColor,line width= 0.4pt,line join=round,line cap=round,fill=fillColor] (468.58,219.26) circle (  1.16);

\path[draw=drawColor,line width= 0.4pt,line join=round,line cap=round,fill=fillColor] (468.73,219.06) circle (  1.16);

\path[draw=drawColor,line width= 0.4pt,line join=round,line cap=round,fill=fillColor] (468.87,218.63) circle (  1.16);

\path[draw=drawColor,line width= 0.4pt,line join=round,line cap=round,fill=fillColor] (469.01,218.54) circle (  1.16);

\path[draw=drawColor,line width= 0.4pt,line join=round,line cap=round,fill=fillColor] (469.16,218.26) circle (  1.16);

\path[draw=drawColor,line width= 0.4pt,line join=round,line cap=round,fill=fillColor] (469.30,217.95) circle (  1.16);

\path[draw=drawColor,line width= 0.4pt,line join=round,line cap=round,fill=fillColor] (469.44,216.59) circle (  1.16);

\path[draw=drawColor,line width= 0.4pt,line join=round,line cap=round,fill=fillColor] (469.58,215.08) circle (  1.16);

\path[draw=drawColor,line width= 0.4pt,line join=round,line cap=round,fill=fillColor] (469.73,209.81) circle (  1.16);

\path[draw=drawColor,line width= 0.4pt,line join=round,line cap=round,fill=fillColor] (469.87,209.81) circle (  1.16);

\path[draw=drawColor,line width= 0.4pt,line join=round,line cap=round,fill=fillColor] (470.01,209.81) circle (  1.16);

\path[draw=drawColor,line width= 0.4pt,line join=round,line cap=round,fill=fillColor] (470.15,209.81) circle (  1.16);

\path[draw=drawColor,line width= 0.4pt,line join=round,line cap=round,fill=fillColor] (470.29,209.81) circle (  1.16);

\path[draw=drawColor,line width= 0.4pt,line join=round,line cap=round,fill=fillColor] (470.43,209.81) circle (  1.16);

\path[draw=drawColor,line width= 0.4pt,line join=round,line cap=round,fill=fillColor] (470.57,209.81) circle (  1.16);

\path[draw=drawColor,line width= 0.4pt,line join=round,line cap=round,fill=fillColor] (470.71,209.81) circle (  1.16);

\path[draw=drawColor,line width= 0.4pt,line join=round,line cap=round,fill=fillColor] (470.85,209.81) circle (  1.16);

\path[draw=drawColor,line width= 0.4pt,line join=round,line cap=round,fill=fillColor] (470.99,209.81) circle (  1.16);

\path[draw=drawColor,line width= 0.4pt,line join=round,line cap=round,fill=fillColor] (471.13,209.81) circle (  1.16);

\path[draw=drawColor,line width= 0.4pt,line join=round,line cap=round,fill=fillColor] (471.27,209.81) circle (  1.16);

\path[draw=drawColor,line width= 0.4pt,line join=round,line cap=round,fill=fillColor] (471.41,209.81) circle (  1.16);

\path[draw=drawColor,line width= 0.4pt,line join=round,line cap=round,fill=fillColor] (471.55,209.81) circle (  1.16);

\path[draw=drawColor,line width= 0.4pt,line join=round,line cap=round,fill=fillColor] (471.69,209.81) circle (  1.16);

\path[draw=drawColor,line width= 0.4pt,line join=round,line cap=round,fill=fillColor] (471.82,209.81) circle (  1.16);

\path[draw=drawColor,line width= 0.4pt,line join=round,line cap=round,fill=fillColor] (471.96,209.81) circle (  1.16);

\path[draw=drawColor,line width= 0.4pt,line join=round,line cap=round,fill=fillColor] (472.10,209.81) circle (  1.16);

\path[draw=drawColor,line width= 0.4pt,line join=round,line cap=round,fill=fillColor] (472.24,209.81) circle (  1.16);

\path[draw=drawColor,line width= 0.4pt,line join=round,line cap=round,fill=fillColor] (472.37,209.81) circle (  1.16);

\path[draw=drawColor,line width= 0.4pt,line join=round,line cap=round,fill=fillColor] (472.51,209.81) circle (  1.16);

\path[draw=drawColor,line width= 0.4pt,line join=round,line cap=round,fill=fillColor] (472.65,209.81) circle (  1.16);

\path[draw=drawColor,line width= 0.4pt,line join=round,line cap=round,fill=fillColor] (472.78,209.81) circle (  1.16);

\path[draw=drawColor,line width= 0.4pt,line join=round,line cap=round,fill=fillColor] (472.92,209.81) circle (  1.16);

\path[draw=drawColor,line width= 0.4pt,line join=round,line cap=round,fill=fillColor] (473.06,209.81) circle (  1.16);

\path[draw=drawColor,line width= 0.4pt,line join=round,line cap=round,fill=fillColor] (473.19,209.81) circle (  1.16);

\path[draw=drawColor,line width= 0.4pt,line join=round,line cap=round,fill=fillColor] (473.33,209.81) circle (  1.16);

\path[draw=drawColor,line width= 0.4pt,line join=round,line cap=round,fill=fillColor] (473.46,209.81) circle (  1.16);

\path[draw=drawColor,line width= 0.4pt,line join=round,line cap=round,fill=fillColor] (473.60,209.81) circle (  1.16);

\path[draw=drawColor,line width= 0.4pt,line join=round,line cap=round,fill=fillColor] (473.74,209.81) circle (  1.16);

\path[draw=drawColor,line width= 0.4pt,line join=round,line cap=round,fill=fillColor] (473.87,209.81) circle (  1.16);

\path[draw=drawColor,line width= 0.4pt,line join=round,line cap=round,fill=fillColor] (474.00,209.81) circle (  1.16);

\path[draw=drawColor,line width= 0.4pt,line join=round,line cap=round,fill=fillColor] (474.14,209.81) circle (  1.16);

\path[draw=drawColor,line width= 0.4pt,line join=round,line cap=round,fill=fillColor] (474.27,209.81) circle (  1.16);

\path[draw=drawColor,line width= 0.4pt,line join=round,line cap=round,fill=fillColor] (474.41,209.81) circle (  1.16);

\path[draw=drawColor,line width= 0.4pt,line join=round,line cap=round,fill=fillColor] (474.54,209.81) circle (  1.16);

\path[draw=drawColor,line width= 0.4pt,line join=round,line cap=round,fill=fillColor] (474.68,209.81) circle (  1.16);

\path[draw=drawColor,line width= 0.4pt,line join=round,line cap=round,fill=fillColor] (474.81,209.81) circle (  1.16);

\path[draw=drawColor,line width= 0.4pt,line join=round,line cap=round,fill=fillColor] (474.94,209.81) circle (  1.16);

\path[draw=drawColor,line width= 0.4pt,line join=round,line cap=round,fill=fillColor] (475.07,209.81) circle (  1.16);

\path[draw=drawColor,line width= 0.4pt,line join=round,line cap=round,fill=fillColor] (475.21,209.81) circle (  1.16);

\path[draw=drawColor,line width= 0.4pt,line join=round,line cap=round,fill=fillColor] (475.34,209.81) circle (  1.16);

\path[draw=drawColor,line width= 0.4pt,line join=round,line cap=round,fill=fillColor] (475.47,209.81) circle (  1.16);

\path[draw=drawColor,line width= 0.4pt,line join=round,line cap=round,fill=fillColor] (475.60,209.81) circle (  1.16);

\path[draw=drawColor,line width= 0.4pt,line join=round,line cap=round,fill=fillColor] (475.74,209.81) circle (  1.16);

\path[draw=drawColor,line width= 0.4pt,line join=round,line cap=round,fill=fillColor] (475.87,209.81) circle (  1.16);

\path[draw=drawColor,line width= 0.4pt,line join=round,line cap=round,fill=fillColor] (476.00,209.81) circle (  1.16);

\path[draw=drawColor,line width= 0.4pt,line join=round,line cap=round,fill=fillColor] (476.13,209.81) circle (  1.16);

\path[draw=drawColor,line width= 0.4pt,line join=round,line cap=round,fill=fillColor] (476.26,209.81) circle (  1.16);

\path[draw=drawColor,line width= 0.4pt,line join=round,line cap=round,fill=fillColor] (476.39,209.81) circle (  1.16);

\path[draw=drawColor,line width= 0.4pt,line join=round,line cap=round,fill=fillColor] (476.52,209.81) circle (  1.16);

\path[draw=drawColor,line width= 0.4pt,line join=round,line cap=round,fill=fillColor] (476.65,209.81) circle (  1.16);

\path[draw=drawColor,line width= 0.4pt,line join=round,line cap=round,fill=fillColor] (476.78,209.81) circle (  1.16);

\path[draw=drawColor,line width= 0.4pt,line join=round,line cap=round,fill=fillColor] (476.91,209.81) circle (  1.16);

\path[draw=drawColor,line width= 0.4pt,line join=round,line cap=round,fill=fillColor] (477.04,209.81) circle (  1.16);

\path[draw=drawColor,line width= 0.4pt,line join=round,line cap=round,fill=fillColor] (477.17,209.81) circle (  1.16);

\path[draw=drawColor,line width= 0.4pt,line join=round,line cap=round,fill=fillColor] (477.30,209.81) circle (  1.16);

\path[draw=drawColor,line width= 0.4pt,line join=round,line cap=round,fill=fillColor] (477.43,209.81) circle (  1.16);

\path[draw=drawColor,line width= 0.4pt,line join=round,line cap=round,fill=fillColor] (477.56,209.81) circle (  1.16);

\path[draw=drawColor,line width= 0.4pt,line join=round,line cap=round,fill=fillColor] (477.69,209.81) circle (  1.16);

\path[draw=drawColor,line width= 0.4pt,line join=round,line cap=round,fill=fillColor] (477.82,209.81) circle (  1.16);

\path[draw=drawColor,line width= 0.4pt,line join=round,line cap=round,fill=fillColor] (477.95,209.81) circle (  1.16);

\path[draw=drawColor,line width= 0.4pt,line join=round,line cap=round,fill=fillColor] (478.08,209.81) circle (  1.16);

\path[draw=drawColor,line width= 0.4pt,line join=round,line cap=round,fill=fillColor] (478.21,209.81) circle (  1.16);
\definecolor[named]{drawColor}{rgb}{0.30,0.69,0.29}
\definecolor[named]{fillColor}{rgb}{0.30,0.69,0.29}

\path[draw=drawColor,line width= 0.4pt,line join=round,line cap=round,fill=fillColor] (327.82,295.45) circle (  1.16);

\path[draw=drawColor,line width= 0.4pt,line join=round,line cap=round,fill=fillColor] (333.69,295.45) circle (  1.16);

\path[draw=drawColor,line width= 0.4pt,line join=round,line cap=round,fill=fillColor] (337.81,295.45) circle (  1.16);

\path[draw=drawColor,line width= 0.4pt,line join=round,line cap=round,fill=fillColor] (341.08,295.45) circle (  1.16);

\path[draw=drawColor,line width= 0.4pt,line join=round,line cap=round,fill=fillColor] (343.85,295.45) circle (  1.16);

\path[draw=drawColor,line width= 0.4pt,line join=round,line cap=round,fill=fillColor] (346.27,295.45) circle (  1.16);

\path[draw=drawColor,line width= 0.4pt,line join=round,line cap=round,fill=fillColor] (348.43,295.45) circle (  1.16);

\path[draw=drawColor,line width= 0.4pt,line join=round,line cap=round,fill=fillColor] (350.39,295.45) circle (  1.16);

\path[draw=drawColor,line width= 0.4pt,line join=round,line cap=round,fill=fillColor] (352.20,295.45) circle (  1.16);

\path[draw=drawColor,line width= 0.4pt,line join=round,line cap=round,fill=fillColor] (353.88,295.45) circle (  1.16);

\path[draw=drawColor,line width= 0.4pt,line join=round,line cap=round,fill=fillColor] (355.45,295.45) circle (  1.16);

\path[draw=drawColor,line width= 0.4pt,line join=round,line cap=round,fill=fillColor] (356.93,295.45) circle (  1.16);

\path[draw=drawColor,line width= 0.4pt,line join=round,line cap=round,fill=fillColor] (358.32,295.45) circle (  1.16);

\path[draw=drawColor,line width= 0.4pt,line join=round,line cap=round,fill=fillColor] (359.65,295.45) circle (  1.16);

\path[draw=drawColor,line width= 0.4pt,line join=round,line cap=round,fill=fillColor] (360.92,295.45) circle (  1.16);

\path[draw=drawColor,line width= 0.4pt,line join=round,line cap=round,fill=fillColor] (362.13,295.45) circle (  1.16);

\path[draw=drawColor,line width= 0.4pt,line join=round,line cap=round,fill=fillColor] (363.29,295.45) circle (  1.16);

\path[draw=drawColor,line width= 0.4pt,line join=round,line cap=round,fill=fillColor] (364.40,295.45) circle (  1.16);

\path[draw=drawColor,line width= 0.4pt,line join=round,line cap=round,fill=fillColor] (365.48,295.45) circle (  1.16);

\path[draw=drawColor,line width= 0.4pt,line join=round,line cap=round,fill=fillColor] (366.52,295.45) circle (  1.16);

\path[draw=drawColor,line width= 0.4pt,line join=round,line cap=round,fill=fillColor] (367.52,295.45) circle (  1.16);

\path[draw=drawColor,line width= 0.4pt,line join=round,line cap=round,fill=fillColor] (368.49,295.45) circle (  1.16);

\path[draw=drawColor,line width= 0.4pt,line join=round,line cap=round,fill=fillColor] (369.44,295.45) circle (  1.16);

\path[draw=drawColor,line width= 0.4pt,line join=round,line cap=round,fill=fillColor] (370.36,295.45) circle (  1.16);

\path[draw=drawColor,line width= 0.4pt,line join=round,line cap=round,fill=fillColor] (371.25,295.45) circle (  1.16);

\path[draw=drawColor,line width= 0.4pt,line join=round,line cap=round,fill=fillColor] (372.12,295.45) circle (  1.16);

\path[draw=drawColor,line width= 0.4pt,line join=round,line cap=round,fill=fillColor] (372.96,295.45) circle (  1.16);

\path[draw=drawColor,line width= 0.4pt,line join=round,line cap=round,fill=fillColor] (373.79,295.45) circle (  1.16);

\path[draw=drawColor,line width= 0.4pt,line join=round,line cap=round,fill=fillColor] (374.59,295.45) circle (  1.16);

\path[draw=drawColor,line width= 0.4pt,line join=round,line cap=round,fill=fillColor] (375.38,295.45) circle (  1.16);

\path[draw=drawColor,line width= 0.4pt,line join=round,line cap=round,fill=fillColor] (376.15,295.45) circle (  1.16);

\path[draw=drawColor,line width= 0.4pt,line join=round,line cap=round,fill=fillColor] (376.91,295.45) circle (  1.16);

\path[draw=drawColor,line width= 0.4pt,line join=round,line cap=round,fill=fillColor] (377.65,295.45) circle (  1.16);

\path[draw=drawColor,line width= 0.4pt,line join=round,line cap=round,fill=fillColor] (378.37,295.45) circle (  1.16);

\path[draw=drawColor,line width= 0.4pt,line join=round,line cap=round,fill=fillColor] (379.08,295.45) circle (  1.16);

\path[draw=drawColor,line width= 0.4pt,line join=round,line cap=round,fill=fillColor] (379.78,295.45) circle (  1.16);

\path[draw=drawColor,line width= 0.4pt,line join=round,line cap=round,fill=fillColor] (380.46,295.45) circle (  1.16);

\path[draw=drawColor,line width= 0.4pt,line join=round,line cap=round,fill=fillColor] (381.13,295.45) circle (  1.16);

\path[draw=drawColor,line width= 0.4pt,line join=round,line cap=round,fill=fillColor] (381.79,295.45) circle (  1.16);

\path[draw=drawColor,line width= 0.4pt,line join=round,line cap=round,fill=fillColor] (382.44,295.45) circle (  1.16);

\path[draw=drawColor,line width= 0.4pt,line join=round,line cap=round,fill=fillColor] (383.08,295.45) circle (  1.16);

\path[draw=drawColor,line width= 0.4pt,line join=round,line cap=round,fill=fillColor] (383.71,295.45) circle (  1.16);

\path[draw=drawColor,line width= 0.4pt,line join=round,line cap=round,fill=fillColor] (384.32,295.45) circle (  1.16);

\path[draw=drawColor,line width= 0.4pt,line join=round,line cap=round,fill=fillColor] (384.93,295.45) circle (  1.16);

\path[draw=drawColor,line width= 0.4pt,line join=round,line cap=round,fill=fillColor] (385.53,295.45) circle (  1.16);

\path[draw=drawColor,line width= 0.4pt,line join=round,line cap=round,fill=fillColor] (386.12,295.45) circle (  1.16);

\path[draw=drawColor,line width= 0.4pt,line join=round,line cap=round,fill=fillColor] (386.70,295.45) circle (  1.16);

\path[draw=drawColor,line width= 0.4pt,line join=round,line cap=round,fill=fillColor] (387.28,295.45) circle (  1.16);

\path[draw=drawColor,line width= 0.4pt,line join=round,line cap=round,fill=fillColor] (387.84,295.45) circle (  1.16);

\path[draw=drawColor,line width= 0.4pt,line join=round,line cap=round,fill=fillColor] (388.40,295.45) circle (  1.16);

\path[draw=drawColor,line width= 0.4pt,line join=round,line cap=round,fill=fillColor] (388.95,295.45) circle (  1.16);

\path[draw=drawColor,line width= 0.4pt,line join=round,line cap=round,fill=fillColor] (389.50,295.45) circle (  1.16);

\path[draw=drawColor,line width= 0.4pt,line join=round,line cap=round,fill=fillColor] (390.03,295.45) circle (  1.16);

\path[draw=drawColor,line width= 0.4pt,line join=round,line cap=round,fill=fillColor] (390.56,295.45) circle (  1.16);

\path[draw=drawColor,line width= 0.4pt,line join=round,line cap=round,fill=fillColor] (391.08,295.45) circle (  1.16);

\path[draw=drawColor,line width= 0.4pt,line join=round,line cap=round,fill=fillColor] (391.60,295.45) circle (  1.16);

\path[draw=drawColor,line width= 0.4pt,line join=round,line cap=round,fill=fillColor] (392.11,295.45) circle (  1.16);

\path[draw=drawColor,line width= 0.4pt,line join=round,line cap=round,fill=fillColor] (392.62,295.45) circle (  1.16);

\path[draw=drawColor,line width= 0.4pt,line join=round,line cap=round,fill=fillColor] (393.12,295.45) circle (  1.16);

\path[draw=drawColor,line width= 0.4pt,line join=round,line cap=round,fill=fillColor] (393.61,295.45) circle (  1.16);

\path[draw=drawColor,line width= 0.4pt,line join=round,line cap=round,fill=fillColor] (394.10,295.45) circle (  1.16);

\path[draw=drawColor,line width= 0.4pt,line join=round,line cap=round,fill=fillColor] (394.58,295.45) circle (  1.16);

\path[draw=drawColor,line width= 0.4pt,line join=round,line cap=round,fill=fillColor] (395.06,295.45) circle (  1.16);

\path[draw=drawColor,line width= 0.4pt,line join=round,line cap=round,fill=fillColor] (395.53,295.45) circle (  1.16);

\path[draw=drawColor,line width= 0.4pt,line join=round,line cap=round,fill=fillColor] (396.00,295.45) circle (  1.16);

\path[draw=drawColor,line width= 0.4pt,line join=round,line cap=round,fill=fillColor] (396.46,295.45) circle (  1.16);

\path[draw=drawColor,line width= 0.4pt,line join=round,line cap=round,fill=fillColor] (396.92,295.45) circle (  1.16);

\path[draw=drawColor,line width= 0.4pt,line join=round,line cap=round,fill=fillColor] (397.37,295.45) circle (  1.16);

\path[draw=drawColor,line width= 0.4pt,line join=round,line cap=round,fill=fillColor] (397.82,295.45) circle (  1.16);

\path[draw=drawColor,line width= 0.4pt,line join=round,line cap=round,fill=fillColor] (398.27,295.45) circle (  1.16);

\path[draw=drawColor,line width= 0.4pt,line join=round,line cap=round,fill=fillColor] (398.71,295.45) circle (  1.16);

\path[draw=drawColor,line width= 0.4pt,line join=round,line cap=round,fill=fillColor] (399.15,295.45) circle (  1.16);

\path[draw=drawColor,line width= 0.4pt,line join=round,line cap=round,fill=fillColor] (399.58,295.45) circle (  1.16);

\path[draw=drawColor,line width= 0.4pt,line join=round,line cap=round,fill=fillColor] (400.01,295.45) circle (  1.16);

\path[draw=drawColor,line width= 0.4pt,line join=round,line cap=round,fill=fillColor] (400.43,295.45) circle (  1.16);

\path[draw=drawColor,line width= 0.4pt,line join=round,line cap=round,fill=fillColor] (400.85,295.45) circle (  1.16);

\path[draw=drawColor,line width= 0.4pt,line join=round,line cap=round,fill=fillColor] (401.27,295.45) circle (  1.16);

\path[draw=drawColor,line width= 0.4pt,line join=round,line cap=round,fill=fillColor] (401.69,295.45) circle (  1.16);

\path[draw=drawColor,line width= 0.4pt,line join=round,line cap=round,fill=fillColor] (402.10,295.45) circle (  1.16);

\path[draw=drawColor,line width= 0.4pt,line join=round,line cap=round,fill=fillColor] (402.50,295.45) circle (  1.16);

\path[draw=drawColor,line width= 0.4pt,line join=round,line cap=round,fill=fillColor] (402.91,295.45) circle (  1.16);

\path[draw=drawColor,line width= 0.4pt,line join=round,line cap=round,fill=fillColor] (403.31,295.45) circle (  1.16);

\path[draw=drawColor,line width= 0.4pt,line join=round,line cap=round,fill=fillColor] (403.70,295.45) circle (  1.16);

\path[draw=drawColor,line width= 0.4pt,line join=round,line cap=round,fill=fillColor] (404.10,295.45) circle (  1.16);

\path[draw=drawColor,line width= 0.4pt,line join=round,line cap=round,fill=fillColor] (404.49,295.45) circle (  1.16);

\path[draw=drawColor,line width= 0.4pt,line join=round,line cap=round,fill=fillColor] (404.88,295.45) circle (  1.16);

\path[draw=drawColor,line width= 0.4pt,line join=round,line cap=round,fill=fillColor] (405.26,295.45) circle (  1.16);

\path[draw=drawColor,line width= 0.4pt,line join=round,line cap=round,fill=fillColor] (405.64,295.45) circle (  1.16);

\path[draw=drawColor,line width= 0.4pt,line join=round,line cap=round,fill=fillColor] (406.02,295.45) circle (  1.16);

\path[draw=drawColor,line width= 0.4pt,line join=round,line cap=round,fill=fillColor] (406.40,295.45) circle (  1.16);

\path[draw=drawColor,line width= 0.4pt,line join=round,line cap=round,fill=fillColor] (406.77,295.45) circle (  1.16);

\path[draw=drawColor,line width= 0.4pt,line join=round,line cap=round,fill=fillColor] (407.14,295.45) circle (  1.16);

\path[draw=drawColor,line width= 0.4pt,line join=round,line cap=round,fill=fillColor] (407.51,295.45) circle (  1.16);

\path[draw=drawColor,line width= 0.4pt,line join=round,line cap=round,fill=fillColor] (407.87,295.45) circle (  1.16);

\path[draw=drawColor,line width= 0.4pt,line join=round,line cap=round,fill=fillColor] (408.24,295.45) circle (  1.16);

\path[draw=drawColor,line width= 0.4pt,line join=round,line cap=round,fill=fillColor] (408.60,295.45) circle (  1.16);

\path[draw=drawColor,line width= 0.4pt,line join=round,line cap=round,fill=fillColor] (408.95,295.45) circle (  1.16);

\path[draw=drawColor,line width= 0.4pt,line join=round,line cap=round,fill=fillColor] (409.31,295.45) circle (  1.16);

\path[draw=drawColor,line width= 0.4pt,line join=round,line cap=round,fill=fillColor] (409.66,295.45) circle (  1.16);

\path[draw=drawColor,line width= 0.4pt,line join=round,line cap=round,fill=fillColor] (410.01,295.45) circle (  1.16);

\path[draw=drawColor,line width= 0.4pt,line join=round,line cap=round,fill=fillColor] (410.36,295.45) circle (  1.16);

\path[draw=drawColor,line width= 0.4pt,line join=round,line cap=round,fill=fillColor] (410.71,295.45) circle (  1.16);

\path[draw=drawColor,line width= 0.4pt,line join=round,line cap=round,fill=fillColor] (411.05,295.45) circle (  1.16);

\path[draw=drawColor,line width= 0.4pt,line join=round,line cap=round,fill=fillColor] (411.39,295.45) circle (  1.16);

\path[draw=drawColor,line width= 0.4pt,line join=round,line cap=round,fill=fillColor] (411.73,295.45) circle (  1.16);

\path[draw=drawColor,line width= 0.4pt,line join=round,line cap=round,fill=fillColor] (412.07,295.45) circle (  1.16);

\path[draw=drawColor,line width= 0.4pt,line join=round,line cap=round,fill=fillColor] (412.40,295.45) circle (  1.16);

\path[draw=drawColor,line width= 0.4pt,line join=round,line cap=round,fill=fillColor] (412.73,295.45) circle (  1.16);

\path[draw=drawColor,line width= 0.4pt,line join=round,line cap=round,fill=fillColor] (413.07,295.45) circle (  1.16);

\path[draw=drawColor,line width= 0.4pt,line join=round,line cap=round,fill=fillColor] (413.39,295.45) circle (  1.16);

\path[draw=drawColor,line width= 0.4pt,line join=round,line cap=round,fill=fillColor] (413.72,295.45) circle (  1.16);

\path[draw=drawColor,line width= 0.4pt,line join=round,line cap=round,fill=fillColor] (414.05,295.45) circle (  1.16);

\path[draw=drawColor,line width= 0.4pt,line join=round,line cap=round,fill=fillColor] (414.37,295.45) circle (  1.16);

\path[draw=drawColor,line width= 0.4pt,line join=round,line cap=round,fill=fillColor] (414.69,295.45) circle (  1.16);

\path[draw=drawColor,line width= 0.4pt,line join=round,line cap=round,fill=fillColor] (415.01,295.45) circle (  1.16);

\path[draw=drawColor,line width= 0.4pt,line join=round,line cap=round,fill=fillColor] (415.33,295.45) circle (  1.16);

\path[draw=drawColor,line width= 0.4pt,line join=round,line cap=round,fill=fillColor] (415.64,295.45) circle (  1.16);

\path[draw=drawColor,line width= 0.4pt,line join=round,line cap=round,fill=fillColor] (415.95,295.45) circle (  1.16);

\path[draw=drawColor,line width= 0.4pt,line join=round,line cap=round,fill=fillColor] (416.27,295.45) circle (  1.16);

\path[draw=drawColor,line width= 0.4pt,line join=round,line cap=round,fill=fillColor] (416.58,295.45) circle (  1.16);

\path[draw=drawColor,line width= 0.4pt,line join=round,line cap=round,fill=fillColor] (416.88,295.45) circle (  1.16);

\path[draw=drawColor,line width= 0.4pt,line join=round,line cap=round,fill=fillColor] (417.19,295.45) circle (  1.16);

\path[draw=drawColor,line width= 0.4pt,line join=round,line cap=round,fill=fillColor] (417.50,295.45) circle (  1.16);

\path[draw=drawColor,line width= 0.4pt,line join=round,line cap=round,fill=fillColor] (417.80,295.45) circle (  1.16);

\path[draw=drawColor,line width= 0.4pt,line join=round,line cap=round,fill=fillColor] (418.10,295.45) circle (  1.16);

\path[draw=drawColor,line width= 0.4pt,line join=round,line cap=round,fill=fillColor] (418.40,295.45) circle (  1.16);

\path[draw=drawColor,line width= 0.4pt,line join=round,line cap=round,fill=fillColor] (418.70,295.45) circle (  1.16);

\path[draw=drawColor,line width= 0.4pt,line join=round,line cap=round,fill=fillColor] (419.00,295.45) circle (  1.16);

\path[draw=drawColor,line width= 0.4pt,line join=round,line cap=round,fill=fillColor] (419.29,295.45) circle (  1.16);

\path[draw=drawColor,line width= 0.4pt,line join=round,line cap=round,fill=fillColor] (419.59,295.45) circle (  1.16);

\path[draw=drawColor,line width= 0.4pt,line join=round,line cap=round,fill=fillColor] (419.88,295.45) circle (  1.16);

\path[draw=drawColor,line width= 0.4pt,line join=round,line cap=round,fill=fillColor] (420.17,295.45) circle (  1.16);

\path[draw=drawColor,line width= 0.4pt,line join=round,line cap=round,fill=fillColor] (420.46,295.45) circle (  1.16);

\path[draw=drawColor,line width= 0.4pt,line join=round,line cap=round,fill=fillColor] (420.75,295.45) circle (  1.16);

\path[draw=drawColor,line width= 0.4pt,line join=round,line cap=round,fill=fillColor] (421.03,295.45) circle (  1.16);

\path[draw=drawColor,line width= 0.4pt,line join=round,line cap=round,fill=fillColor] (421.32,295.45) circle (  1.16);

\path[draw=drawColor,line width= 0.4pt,line join=round,line cap=round,fill=fillColor] (421.60,295.45) circle (  1.16);

\path[draw=drawColor,line width= 0.4pt,line join=round,line cap=round,fill=fillColor] (421.89,295.45) circle (  1.16);

\path[draw=drawColor,line width= 0.4pt,line join=round,line cap=round,fill=fillColor] (422.17,295.45) circle (  1.16);

\path[draw=drawColor,line width= 0.4pt,line join=round,line cap=round,fill=fillColor] (422.45,295.45) circle (  1.16);

\path[draw=drawColor,line width= 0.4pt,line join=round,line cap=round,fill=fillColor] (422.72,295.45) circle (  1.16);

\path[draw=drawColor,line width= 0.4pt,line join=round,line cap=round,fill=fillColor] (423.00,295.45) circle (  1.16);

\path[draw=drawColor,line width= 0.4pt,line join=round,line cap=round,fill=fillColor] (423.28,295.45) circle (  1.16);

\path[draw=drawColor,line width= 0.4pt,line join=round,line cap=round,fill=fillColor] (423.55,295.45) circle (  1.16);

\path[draw=drawColor,line width= 0.4pt,line join=round,line cap=round,fill=fillColor] (423.82,295.45) circle (  1.16);

\path[draw=drawColor,line width= 0.4pt,line join=round,line cap=round,fill=fillColor] (424.10,295.45) circle (  1.16);

\path[draw=drawColor,line width= 0.4pt,line join=round,line cap=round,fill=fillColor] (424.37,295.45) circle (  1.16);

\path[draw=drawColor,line width= 0.4pt,line join=round,line cap=round,fill=fillColor] (424.64,295.45) circle (  1.16);

\path[draw=drawColor,line width= 0.4pt,line join=round,line cap=round,fill=fillColor] (424.91,295.45) circle (  1.16);

\path[draw=drawColor,line width= 0.4pt,line join=round,line cap=round,fill=fillColor] (425.17,295.45) circle (  1.16);

\path[draw=drawColor,line width= 0.4pt,line join=round,line cap=round,fill=fillColor] (425.44,295.45) circle (  1.16);

\path[draw=drawColor,line width= 0.4pt,line join=round,line cap=round,fill=fillColor] (425.70,295.45) circle (  1.16);

\path[draw=drawColor,line width= 0.4pt,line join=round,line cap=round,fill=fillColor] (425.97,295.45) circle (  1.16);

\path[draw=drawColor,line width= 0.4pt,line join=round,line cap=round,fill=fillColor] (426.23,295.45) circle (  1.16);

\path[draw=drawColor,line width= 0.4pt,line join=round,line cap=round,fill=fillColor] (426.49,295.45) circle (  1.16);

\path[draw=drawColor,line width= 0.4pt,line join=round,line cap=round,fill=fillColor] (426.75,295.45) circle (  1.16);

\path[draw=drawColor,line width= 0.4pt,line join=round,line cap=round,fill=fillColor] (427.01,295.45) circle (  1.16);

\path[draw=drawColor,line width= 0.4pt,line join=round,line cap=round,fill=fillColor] (427.27,295.45) circle (  1.16);

\path[draw=drawColor,line width= 0.4pt,line join=round,line cap=round,fill=fillColor] (427.52,295.45) circle (  1.16);

\path[draw=drawColor,line width= 0.4pt,line join=round,line cap=round,fill=fillColor] (427.78,295.45) circle (  1.16);

\path[draw=drawColor,line width= 0.4pt,line join=round,line cap=round,fill=fillColor] (428.03,295.45) circle (  1.16);

\path[draw=drawColor,line width= 0.4pt,line join=round,line cap=round,fill=fillColor] (428.29,295.45) circle (  1.16);

\path[draw=drawColor,line width= 0.4pt,line join=round,line cap=round,fill=fillColor] (428.54,295.45) circle (  1.16);

\path[draw=drawColor,line width= 0.4pt,line join=round,line cap=round,fill=fillColor] (428.79,295.45) circle (  1.16);

\path[draw=drawColor,line width= 0.4pt,line join=round,line cap=round,fill=fillColor] (429.04,295.45) circle (  1.16);

\path[draw=drawColor,line width= 0.4pt,line join=round,line cap=round,fill=fillColor] (429.29,295.45) circle (  1.16);

\path[draw=drawColor,line width= 0.4pt,line join=round,line cap=round,fill=fillColor] (429.54,295.45) circle (  1.16);

\path[draw=drawColor,line width= 0.4pt,line join=round,line cap=round,fill=fillColor] (429.79,295.45) circle (  1.16);

\path[draw=drawColor,line width= 0.4pt,line join=round,line cap=round,fill=fillColor] (430.04,295.45) circle (  1.16);

\path[draw=drawColor,line width= 0.4pt,line join=round,line cap=round,fill=fillColor] (430.28,295.45) circle (  1.16);

\path[draw=drawColor,line width= 0.4pt,line join=round,line cap=round,fill=fillColor] (430.53,295.45) circle (  1.16);

\path[draw=drawColor,line width= 0.4pt,line join=round,line cap=round,fill=fillColor] (430.77,295.45) circle (  1.16);

\path[draw=drawColor,line width= 0.4pt,line join=round,line cap=round,fill=fillColor] (431.01,295.45) circle (  1.16);

\path[draw=drawColor,line width= 0.4pt,line join=round,line cap=round,fill=fillColor] (431.25,295.45) circle (  1.16);

\path[draw=drawColor,line width= 0.4pt,line join=round,line cap=round,fill=fillColor] (431.50,295.45) circle (  1.16);

\path[draw=drawColor,line width= 0.4pt,line join=round,line cap=round,fill=fillColor] (431.74,295.45) circle (  1.16);

\path[draw=drawColor,line width= 0.4pt,line join=round,line cap=round,fill=fillColor] (431.97,295.45) circle (  1.16);

\path[draw=drawColor,line width= 0.4pt,line join=round,line cap=round,fill=fillColor] (432.21,295.45) circle (  1.16);

\path[draw=drawColor,line width= 0.4pt,line join=round,line cap=round,fill=fillColor] (432.45,295.45) circle (  1.16);

\path[draw=drawColor,line width= 0.4pt,line join=round,line cap=round,fill=fillColor] (432.69,295.45) circle (  1.16);

\path[draw=drawColor,line width= 0.4pt,line join=round,line cap=round,fill=fillColor] (432.92,295.45) circle (  1.16);

\path[draw=drawColor,line width= 0.4pt,line join=round,line cap=round,fill=fillColor] (433.16,295.45) circle (  1.16);

\path[draw=drawColor,line width= 0.4pt,line join=round,line cap=round,fill=fillColor] (433.39,295.45) circle (  1.16);

\path[draw=drawColor,line width= 0.4pt,line join=round,line cap=round,fill=fillColor] (433.62,295.45) circle (  1.16);

\path[draw=drawColor,line width= 0.4pt,line join=round,line cap=round,fill=fillColor] (433.86,295.45) circle (  1.16);

\path[draw=drawColor,line width= 0.4pt,line join=round,line cap=round,fill=fillColor] (434.09,295.45) circle (  1.16);

\path[draw=drawColor,line width= 0.4pt,line join=round,line cap=round,fill=fillColor] (434.32,295.45) circle (  1.16);

\path[draw=drawColor,line width= 0.4pt,line join=round,line cap=round,fill=fillColor] (434.55,295.45) circle (  1.16);

\path[draw=drawColor,line width= 0.4pt,line join=round,line cap=round,fill=fillColor] (434.78,295.45) circle (  1.16);

\path[draw=drawColor,line width= 0.4pt,line join=round,line cap=round,fill=fillColor] (435.00,295.45) circle (  1.16);

\path[draw=drawColor,line width= 0.4pt,line join=round,line cap=round,fill=fillColor] (435.23,295.45) circle (  1.16);

\path[draw=drawColor,line width= 0.4pt,line join=round,line cap=round,fill=fillColor] (435.46,295.45) circle (  1.16);

\path[draw=drawColor,line width= 0.4pt,line join=round,line cap=round,fill=fillColor] (435.68,295.45) circle (  1.16);

\path[draw=drawColor,line width= 0.4pt,line join=round,line cap=round,fill=fillColor] (435.91,295.45) circle (  1.16);

\path[draw=drawColor,line width= 0.4pt,line join=round,line cap=round,fill=fillColor] (436.13,295.45) circle (  1.16);

\path[draw=drawColor,line width= 0.4pt,line join=round,line cap=round,fill=fillColor] (436.36,295.45) circle (  1.16);

\path[draw=drawColor,line width= 0.4pt,line join=round,line cap=round,fill=fillColor] (436.58,295.45) circle (  1.16);

\path[draw=drawColor,line width= 0.4pt,line join=round,line cap=round,fill=fillColor] (436.80,295.45) circle (  1.16);

\path[draw=drawColor,line width= 0.4pt,line join=round,line cap=round,fill=fillColor] (437.02,295.45) circle (  1.16);

\path[draw=drawColor,line width= 0.4pt,line join=round,line cap=round,fill=fillColor] (437.24,295.45) circle (  1.16);

\path[draw=drawColor,line width= 0.4pt,line join=round,line cap=round,fill=fillColor] (437.46,295.45) circle (  1.16);

\path[draw=drawColor,line width= 0.4pt,line join=round,line cap=round,fill=fillColor] (437.68,295.45) circle (  1.16);

\path[draw=drawColor,line width= 0.4pt,line join=round,line cap=round,fill=fillColor] (437.90,295.45) circle (  1.16);

\path[draw=drawColor,line width= 0.4pt,line join=round,line cap=round,fill=fillColor] (438.12,295.45) circle (  1.16);

\path[draw=drawColor,line width= 0.4pt,line join=round,line cap=round,fill=fillColor] (438.33,295.45) circle (  1.16);

\path[draw=drawColor,line width= 0.4pt,line join=round,line cap=round,fill=fillColor] (438.55,295.45) circle (  1.16);

\path[draw=drawColor,line width= 0.4pt,line join=round,line cap=round,fill=fillColor] (438.76,295.45) circle (  1.16);

\path[draw=drawColor,line width= 0.4pt,line join=round,line cap=round,fill=fillColor] (438.98,295.45) circle (  1.16);

\path[draw=drawColor,line width= 0.4pt,line join=round,line cap=round,fill=fillColor] (439.19,295.45) circle (  1.16);

\path[draw=drawColor,line width= 0.4pt,line join=round,line cap=round,fill=fillColor] (439.41,295.45) circle (  1.16);

\path[draw=drawColor,line width= 0.4pt,line join=round,line cap=round,fill=fillColor] (439.62,295.45) circle (  1.16);

\path[draw=drawColor,line width= 0.4pt,line join=round,line cap=round,fill=fillColor] (439.83,295.45) circle (  1.16);

\path[draw=drawColor,line width= 0.4pt,line join=round,line cap=round,fill=fillColor] (440.04,295.45) circle (  1.16);

\path[draw=drawColor,line width= 0.4pt,line join=round,line cap=round,fill=fillColor] (440.25,295.45) circle (  1.16);

\path[draw=drawColor,line width= 0.4pt,line join=round,line cap=round,fill=fillColor] (440.46,295.45) circle (  1.16);

\path[draw=drawColor,line width= 0.4pt,line join=round,line cap=round,fill=fillColor] (440.67,295.45) circle (  1.16);

\path[draw=drawColor,line width= 0.4pt,line join=round,line cap=round,fill=fillColor] (440.88,295.45) circle (  1.16);

\path[draw=drawColor,line width= 0.4pt,line join=round,line cap=round,fill=fillColor] (441.09,295.45) circle (  1.16);

\path[draw=drawColor,line width= 0.4pt,line join=round,line cap=round,fill=fillColor] (441.29,295.45) circle (  1.16);

\path[draw=drawColor,line width= 0.4pt,line join=round,line cap=round,fill=fillColor] (441.50,295.45) circle (  1.16);

\path[draw=drawColor,line width= 0.4pt,line join=round,line cap=round,fill=fillColor] (441.71,295.45) circle (  1.16);

\path[draw=drawColor,line width= 0.4pt,line join=round,line cap=round,fill=fillColor] (441.91,295.45) circle (  1.16);

\path[draw=drawColor,line width= 0.4pt,line join=round,line cap=round,fill=fillColor] (442.12,295.45) circle (  1.16);

\path[draw=drawColor,line width= 0.4pt,line join=round,line cap=round,fill=fillColor] (442.32,295.45) circle (  1.16);

\path[draw=drawColor,line width= 0.4pt,line join=round,line cap=round,fill=fillColor] (442.53,295.45) circle (  1.16);

\path[draw=drawColor,line width= 0.4pt,line join=round,line cap=round,fill=fillColor] (442.73,295.45) circle (  1.16);

\path[draw=drawColor,line width= 0.4pt,line join=round,line cap=round,fill=fillColor] (442.93,295.45) circle (  1.16);

\path[draw=drawColor,line width= 0.4pt,line join=round,line cap=round,fill=fillColor] (443.13,295.45) circle (  1.16);

\path[draw=drawColor,line width= 0.4pt,line join=round,line cap=round,fill=fillColor] (443.33,295.45) circle (  1.16);

\path[draw=drawColor,line width= 0.4pt,line join=round,line cap=round,fill=fillColor] (443.54,295.45) circle (  1.16);

\path[draw=drawColor,line width= 0.4pt,line join=round,line cap=round,fill=fillColor] (443.74,295.45) circle (  1.16);

\path[draw=drawColor,line width= 0.4pt,line join=round,line cap=round,fill=fillColor] (443.94,295.45) circle (  1.16);

\path[draw=drawColor,line width= 0.4pt,line join=round,line cap=round,fill=fillColor] (444.13,295.45) circle (  1.16);

\path[draw=drawColor,line width= 0.4pt,line join=round,line cap=round,fill=fillColor] (444.33,295.45) circle (  1.16);

\path[draw=drawColor,line width= 0.4pt,line join=round,line cap=round,fill=fillColor] (444.53,295.45) circle (  1.16);

\path[draw=drawColor,line width= 0.4pt,line join=round,line cap=round,fill=fillColor] (444.73,295.45) circle (  1.16);

\path[draw=drawColor,line width= 0.4pt,line join=round,line cap=round,fill=fillColor] (444.92,295.45) circle (  1.16);

\path[draw=drawColor,line width= 0.4pt,line join=round,line cap=round,fill=fillColor] (445.12,295.45) circle (  1.16);

\path[draw=drawColor,line width= 0.4pt,line join=round,line cap=round,fill=fillColor] (445.32,295.45) circle (  1.16);

\path[draw=drawColor,line width= 0.4pt,line join=round,line cap=round,fill=fillColor] (445.51,295.45) circle (  1.16);

\path[draw=drawColor,line width= 0.4pt,line join=round,line cap=round,fill=fillColor] (445.71,295.45) circle (  1.16);

\path[draw=drawColor,line width= 0.4pt,line join=round,line cap=round,fill=fillColor] (445.90,295.45) circle (  1.16);

\path[draw=drawColor,line width= 0.4pt,line join=round,line cap=round,fill=fillColor] (446.09,295.45) circle (  1.16);

\path[draw=drawColor,line width= 0.4pt,line join=round,line cap=round,fill=fillColor] (446.29,295.45) circle (  1.16);

\path[draw=drawColor,line width= 0.4pt,line join=round,line cap=round,fill=fillColor] (446.48,295.45) circle (  1.16);

\path[draw=drawColor,line width= 0.4pt,line join=round,line cap=round,fill=fillColor] (446.67,295.45) circle (  1.16);

\path[draw=drawColor,line width= 0.4pt,line join=round,line cap=round,fill=fillColor] (446.86,295.45) circle (  1.16);

\path[draw=drawColor,line width= 0.4pt,line join=round,line cap=round,fill=fillColor] (447.05,295.45) circle (  1.16);

\path[draw=drawColor,line width= 0.4pt,line join=round,line cap=round,fill=fillColor] (447.24,295.45) circle (  1.16);

\path[draw=drawColor,line width= 0.4pt,line join=round,line cap=round,fill=fillColor] (447.43,295.45) circle (  1.16);

\path[draw=drawColor,line width= 0.4pt,line join=round,line cap=round,fill=fillColor] (447.62,295.45) circle (  1.16);

\path[draw=drawColor,line width= 0.4pt,line join=round,line cap=round,fill=fillColor] (447.81,295.45) circle (  1.16);

\path[draw=drawColor,line width= 0.4pt,line join=round,line cap=round,fill=fillColor] (448.00,295.45) circle (  1.16);

\path[draw=drawColor,line width= 0.4pt,line join=round,line cap=round,fill=fillColor] (448.19,295.45) circle (  1.16);

\path[draw=drawColor,line width= 0.4pt,line join=round,line cap=round,fill=fillColor] (448.37,295.45) circle (  1.16);

\path[draw=drawColor,line width= 0.4pt,line join=round,line cap=round,fill=fillColor] (448.56,295.45) circle (  1.16);

\path[draw=drawColor,line width= 0.4pt,line join=round,line cap=round,fill=fillColor] (448.75,295.45) circle (  1.16);

\path[draw=drawColor,line width= 0.4pt,line join=round,line cap=round,fill=fillColor] (448.93,295.45) circle (  1.16);

\path[draw=drawColor,line width= 0.4pt,line join=round,line cap=round,fill=fillColor] (449.12,295.45) circle (  1.16);

\path[draw=drawColor,line width= 0.4pt,line join=round,line cap=round,fill=fillColor] (449.30,295.45) circle (  1.16);

\path[draw=drawColor,line width= 0.4pt,line join=round,line cap=round,fill=fillColor] (449.49,295.45) circle (  1.16);

\path[draw=drawColor,line width= 0.4pt,line join=round,line cap=round,fill=fillColor] (449.67,295.45) circle (  1.16);

\path[draw=drawColor,line width= 0.4pt,line join=round,line cap=round,fill=fillColor] (449.86,295.45) circle (  1.16);

\path[draw=drawColor,line width= 0.4pt,line join=round,line cap=round,fill=fillColor] (450.04,295.45) circle (  1.16);

\path[draw=drawColor,line width= 0.4pt,line join=round,line cap=round,fill=fillColor] (450.22,295.45) circle (  1.16);

\path[draw=drawColor,line width= 0.4pt,line join=round,line cap=round,fill=fillColor] (450.40,295.45) circle (  1.16);

\path[draw=drawColor,line width= 0.4pt,line join=round,line cap=round,fill=fillColor] (450.58,295.45) circle (  1.16);

\path[draw=drawColor,line width= 0.4pt,line join=round,line cap=round,fill=fillColor] (450.77,295.45) circle (  1.16);

\path[draw=drawColor,line width= 0.4pt,line join=round,line cap=round,fill=fillColor] (450.95,295.45) circle (  1.16);

\path[draw=drawColor,line width= 0.4pt,line join=round,line cap=round,fill=fillColor] (451.13,295.45) circle (  1.16);

\path[draw=drawColor,line width= 0.4pt,line join=round,line cap=round,fill=fillColor] (451.31,295.45) circle (  1.16);

\path[draw=drawColor,line width= 0.4pt,line join=round,line cap=round,fill=fillColor] (451.49,295.45) circle (  1.16);

\path[draw=drawColor,line width= 0.4pt,line join=round,line cap=round,fill=fillColor] (451.67,295.45) circle (  1.16);

\path[draw=drawColor,line width= 0.4pt,line join=round,line cap=round,fill=fillColor] (451.84,295.45) circle (  1.16);

\path[draw=drawColor,line width= 0.4pt,line join=round,line cap=round,fill=fillColor] (452.02,295.45) circle (  1.16);

\path[draw=drawColor,line width= 0.4pt,line join=round,line cap=round,fill=fillColor] (452.20,295.45) circle (  1.16);

\path[draw=drawColor,line width= 0.4pt,line join=round,line cap=round,fill=fillColor] (452.38,295.45) circle (  1.16);

\path[draw=drawColor,line width= 0.4pt,line join=round,line cap=round,fill=fillColor] (452.55,295.45) circle (  1.16);

\path[draw=drawColor,line width= 0.4pt,line join=round,line cap=round,fill=fillColor] (452.73,295.45) circle (  1.16);

\path[draw=drawColor,line width= 0.4pt,line join=round,line cap=round,fill=fillColor] (452.91,295.45) circle (  1.16);

\path[draw=drawColor,line width= 0.4pt,line join=round,line cap=round,fill=fillColor] (453.08,295.45) circle (  1.16);

\path[draw=drawColor,line width= 0.4pt,line join=round,line cap=round,fill=fillColor] (453.26,295.45) circle (  1.16);

\path[draw=drawColor,line width= 0.4pt,line join=round,line cap=round,fill=fillColor] (453.43,295.45) circle (  1.16);

\path[draw=drawColor,line width= 0.4pt,line join=round,line cap=round,fill=fillColor] (453.61,295.45) circle (  1.16);

\path[draw=drawColor,line width= 0.4pt,line join=round,line cap=round,fill=fillColor] (453.78,295.45) circle (  1.16);

\path[draw=drawColor,line width= 0.4pt,line join=round,line cap=round,fill=fillColor] (453.95,295.45) circle (  1.16);

\path[draw=drawColor,line width= 0.4pt,line join=round,line cap=round,fill=fillColor] (454.13,295.45) circle (  1.16);

\path[draw=drawColor,line width= 0.4pt,line join=round,line cap=round,fill=fillColor] (454.30,295.45) circle (  1.16);

\path[draw=drawColor,line width= 0.4pt,line join=round,line cap=round,fill=fillColor] (454.47,295.45) circle (  1.16);

\path[draw=drawColor,line width= 0.4pt,line join=round,line cap=round,fill=fillColor] (454.64,295.45) circle (  1.16);

\path[draw=drawColor,line width= 0.4pt,line join=round,line cap=round,fill=fillColor] (454.82,295.45) circle (  1.16);

\path[draw=drawColor,line width= 0.4pt,line join=round,line cap=round,fill=fillColor] (454.99,295.45) circle (  1.16);

\path[draw=drawColor,line width= 0.4pt,line join=round,line cap=round,fill=fillColor] (455.16,295.45) circle (  1.16);

\path[draw=drawColor,line width= 0.4pt,line join=round,line cap=round,fill=fillColor] (455.33,295.45) circle (  1.16);

\path[draw=drawColor,line width= 0.4pt,line join=round,line cap=round,fill=fillColor] (455.50,295.45) circle (  1.16);

\path[draw=drawColor,line width= 0.4pt,line join=round,line cap=round,fill=fillColor] (455.67,295.45) circle (  1.16);

\path[draw=drawColor,line width= 0.4pt,line join=round,line cap=round,fill=fillColor] (455.84,292.33) circle (  1.16);

\path[draw=drawColor,line width= 0.4pt,line join=round,line cap=round,fill=fillColor] (456.00,290.09) circle (  1.16);

\path[draw=drawColor,line width= 0.4pt,line join=round,line cap=round,fill=fillColor] (456.17,284.66) circle (  1.16);

\path[draw=drawColor,line width= 0.4pt,line join=round,line cap=round,fill=fillColor] (456.34,277.41) circle (  1.16);

\path[draw=drawColor,line width= 0.4pt,line join=round,line cap=round,fill=fillColor] (456.51,276.87) circle (  1.16);

\path[draw=drawColor,line width= 0.4pt,line join=round,line cap=round,fill=fillColor] (456.68,275.80) circle (  1.16);

\path[draw=drawColor,line width= 0.4pt,line join=round,line cap=round,fill=fillColor] (456.84,275.35) circle (  1.16);

\path[draw=drawColor,line width= 0.4pt,line join=round,line cap=round,fill=fillColor] (457.01,274.83) circle (  1.16);

\path[draw=drawColor,line width= 0.4pt,line join=round,line cap=round,fill=fillColor] (457.18,274.59) circle (  1.16);

\path[draw=drawColor,line width= 0.4pt,line join=round,line cap=round,fill=fillColor] (457.34,273.64) circle (  1.16);

\path[draw=drawColor,line width= 0.4pt,line join=round,line cap=round,fill=fillColor] (457.51,272.65) circle (  1.16);

\path[draw=drawColor,line width= 0.4pt,line join=round,line cap=round,fill=fillColor] (457.67,269.15) circle (  1.16);

\path[draw=drawColor,line width= 0.4pt,line join=round,line cap=round,fill=fillColor] (457.84,268.88) circle (  1.16);

\path[draw=drawColor,line width= 0.4pt,line join=round,line cap=round,fill=fillColor] (458.00,268.39) circle (  1.16);

\path[draw=drawColor,line width= 0.4pt,line join=round,line cap=round,fill=fillColor] (458.17,267.00) circle (  1.16);

\path[draw=drawColor,line width= 0.4pt,line join=round,line cap=round,fill=fillColor] (458.33,266.57) circle (  1.16);

\path[draw=drawColor,line width= 0.4pt,line join=round,line cap=round,fill=fillColor] (458.49,265.31) circle (  1.16);

\path[draw=drawColor,line width= 0.4pt,line join=round,line cap=round,fill=fillColor] (458.66,264.87) circle (  1.16);

\path[draw=drawColor,line width= 0.4pt,line join=round,line cap=round,fill=fillColor] (458.82,263.91) circle (  1.16);

\path[draw=drawColor,line width= 0.4pt,line join=round,line cap=round,fill=fillColor] (458.98,263.65) circle (  1.16);

\path[draw=drawColor,line width= 0.4pt,line join=round,line cap=round,fill=fillColor] (459.14,263.10) circle (  1.16);

\path[draw=drawColor,line width= 0.4pt,line join=round,line cap=round,fill=fillColor] (459.30,262.54) circle (  1.16);

\path[draw=drawColor,line width= 0.4pt,line join=round,line cap=round,fill=fillColor] (459.47,262.34) circle (  1.16);

\path[draw=drawColor,line width= 0.4pt,line join=round,line cap=round,fill=fillColor] (459.63,260.62) circle (  1.16);

\path[draw=drawColor,line width= 0.4pt,line join=round,line cap=round,fill=fillColor] (459.79,260.39) circle (  1.16);

\path[draw=drawColor,line width= 0.4pt,line join=round,line cap=round,fill=fillColor] (459.95,260.25) circle (  1.16);

\path[draw=drawColor,line width= 0.4pt,line join=round,line cap=round,fill=fillColor] (460.11,257.09) circle (  1.16);

\path[draw=drawColor,line width= 0.4pt,line join=round,line cap=round,fill=fillColor] (460.27,255.79) circle (  1.16);

\path[draw=drawColor,line width= 0.4pt,line join=round,line cap=round,fill=fillColor] (460.43,255.74) circle (  1.16);

\path[draw=drawColor,line width= 0.4pt,line join=round,line cap=round,fill=fillColor] (460.59,255.64) circle (  1.16);

\path[draw=drawColor,line width= 0.4pt,line join=round,line cap=round,fill=fillColor] (460.74,255.14) circle (  1.16);

\path[draw=drawColor,line width= 0.4pt,line join=round,line cap=round,fill=fillColor] (460.90,255.01) circle (  1.16);

\path[draw=drawColor,line width= 0.4pt,line join=round,line cap=round,fill=fillColor] (461.06,254.79) circle (  1.16);

\path[draw=drawColor,line width= 0.4pt,line join=round,line cap=round,fill=fillColor] (461.22,254.06) circle (  1.16);

\path[draw=drawColor,line width= 0.4pt,line join=round,line cap=round,fill=fillColor] (461.38,252.96) circle (  1.16);

\path[draw=drawColor,line width= 0.4pt,line join=round,line cap=round,fill=fillColor] (461.53,252.79) circle (  1.16);

\path[draw=drawColor,line width= 0.4pt,line join=round,line cap=round,fill=fillColor] (461.69,252.54) circle (  1.16);

\path[draw=drawColor,line width= 0.4pt,line join=round,line cap=round,fill=fillColor] (461.85,252.36) circle (  1.16);

\path[draw=drawColor,line width= 0.4pt,line join=round,line cap=round,fill=fillColor] (462.00,252.14) circle (  1.16);

\path[draw=drawColor,line width= 0.4pt,line join=round,line cap=round,fill=fillColor] (462.16,251.79) circle (  1.16);

\path[draw=drawColor,line width= 0.4pt,line join=round,line cap=round,fill=fillColor] (462.31,251.27) circle (  1.16);

\path[draw=drawColor,line width= 0.4pt,line join=round,line cap=round,fill=fillColor] (462.47,251.19) circle (  1.16);

\path[draw=drawColor,line width= 0.4pt,line join=round,line cap=round,fill=fillColor] (462.62,251.11) circle (  1.16);

\path[draw=drawColor,line width= 0.4pt,line join=round,line cap=round,fill=fillColor] (462.78,251.00) circle (  1.16);

\path[draw=drawColor,line width= 0.4pt,line join=round,line cap=round,fill=fillColor] (462.93,250.25) circle (  1.16);

\path[draw=drawColor,line width= 0.4pt,line join=round,line cap=round,fill=fillColor] (463.09,250.20) circle (  1.16);

\path[draw=drawColor,line width= 0.4pt,line join=round,line cap=round,fill=fillColor] (463.24,249.15) circle (  1.16);

\path[draw=drawColor,line width= 0.4pt,line join=round,line cap=round,fill=fillColor] (463.39,248.64) circle (  1.16);

\path[draw=drawColor,line width= 0.4pt,line join=round,line cap=round,fill=fillColor] (463.55,248.54) circle (  1.16);

\path[draw=drawColor,line width= 0.4pt,line join=round,line cap=round,fill=fillColor] (463.70,248.51) circle (  1.16);

\path[draw=drawColor,line width= 0.4pt,line join=round,line cap=round,fill=fillColor] (463.85,248.22) circle (  1.16);

\path[draw=drawColor,line width= 0.4pt,line join=round,line cap=round,fill=fillColor] (464.00,248.16) circle (  1.16);

\path[draw=drawColor,line width= 0.4pt,line join=round,line cap=round,fill=fillColor] (464.16,247.64) circle (  1.16);

\path[draw=drawColor,line width= 0.4pt,line join=round,line cap=round,fill=fillColor] (464.31,247.62) circle (  1.16);

\path[draw=drawColor,line width= 0.4pt,line join=round,line cap=round,fill=fillColor] (464.46,247.44) circle (  1.16);

\path[draw=drawColor,line width= 0.4pt,line join=round,line cap=round,fill=fillColor] (464.61,247.28) circle (  1.16);

\path[draw=drawColor,line width= 0.4pt,line join=round,line cap=round,fill=fillColor] (464.76,247.16) circle (  1.16);

\path[draw=drawColor,line width= 0.4pt,line join=round,line cap=round,fill=fillColor] (464.91,247.09) circle (  1.16);

\path[draw=drawColor,line width= 0.4pt,line join=round,line cap=round,fill=fillColor] (465.06,247.09) circle (  1.16);

\path[draw=drawColor,line width= 0.4pt,line join=round,line cap=round,fill=fillColor] (465.21,246.21) circle (  1.16);

\path[draw=drawColor,line width= 0.4pt,line join=round,line cap=round,fill=fillColor] (465.36,246.08) circle (  1.16);

\path[draw=drawColor,line width= 0.4pt,line join=round,line cap=round,fill=fillColor] (465.51,245.99) circle (  1.16);

\path[draw=drawColor,line width= 0.4pt,line join=round,line cap=round,fill=fillColor] (465.66,245.68) circle (  1.16);

\path[draw=drawColor,line width= 0.4pt,line join=round,line cap=round,fill=fillColor] (465.81,245.14) circle (  1.16);

\path[draw=drawColor,line width= 0.4pt,line join=round,line cap=round,fill=fillColor] (465.96,245.11) circle (  1.16);

\path[draw=drawColor,line width= 0.4pt,line join=round,line cap=round,fill=fillColor] (466.10,245.06) circle (  1.16);

\path[draw=drawColor,line width= 0.4pt,line join=round,line cap=round,fill=fillColor] (466.25,244.97) circle (  1.16);

\path[draw=drawColor,line width= 0.4pt,line join=round,line cap=round,fill=fillColor] (466.40,244.95) circle (  1.16);

\path[draw=drawColor,line width= 0.4pt,line join=round,line cap=round,fill=fillColor] (466.55,243.51) circle (  1.16);

\path[draw=drawColor,line width= 0.4pt,line join=round,line cap=round,fill=fillColor] (466.70,243.14) circle (  1.16);

\path[draw=drawColor,line width= 0.4pt,line join=round,line cap=round,fill=fillColor] (466.84,243.07) circle (  1.16);

\path[draw=drawColor,line width= 0.4pt,line join=round,line cap=round,fill=fillColor] (466.99,243.05) circle (  1.16);

\path[draw=drawColor,line width= 0.4pt,line join=round,line cap=round,fill=fillColor] (467.14,242.92) circle (  1.16);

\path[draw=drawColor,line width= 0.4pt,line join=round,line cap=round,fill=fillColor] (467.28,242.90) circle (  1.16);

\path[draw=drawColor,line width= 0.4pt,line join=round,line cap=round,fill=fillColor] (467.43,242.43) circle (  1.16);

\path[draw=drawColor,line width= 0.4pt,line join=round,line cap=round,fill=fillColor] (467.57,242.19) circle (  1.16);

\path[draw=drawColor,line width= 0.4pt,line join=round,line cap=round,fill=fillColor] (467.72,242.19) circle (  1.16);

\path[draw=drawColor,line width= 0.4pt,line join=round,line cap=round,fill=fillColor] (467.86,242.18) circle (  1.16);

\path[draw=drawColor,line width= 0.4pt,line join=round,line cap=round,fill=fillColor] (468.01,241.82) circle (  1.16);

\path[draw=drawColor,line width= 0.4pt,line join=round,line cap=round,fill=fillColor] (468.15,241.78) circle (  1.16);

\path[draw=drawColor,line width= 0.4pt,line join=round,line cap=round,fill=fillColor] (468.30,241.74) circle (  1.16);

\path[draw=drawColor,line width= 0.4pt,line join=round,line cap=round,fill=fillColor] (468.44,241.70) circle (  1.16);

\path[draw=drawColor,line width= 0.4pt,line join=round,line cap=round,fill=fillColor] (468.58,241.37) circle (  1.16);

\path[draw=drawColor,line width= 0.4pt,line join=round,line cap=round,fill=fillColor] (468.73,241.23) circle (  1.16);

\path[draw=drawColor,line width= 0.4pt,line join=round,line cap=round,fill=fillColor] (468.87,240.87) circle (  1.16);

\path[draw=drawColor,line width= 0.4pt,line join=round,line cap=round,fill=fillColor] (469.01,240.59) circle (  1.16);

\path[draw=drawColor,line width= 0.4pt,line join=round,line cap=round,fill=fillColor] (469.16,240.25) circle (  1.16);

\path[draw=drawColor,line width= 0.4pt,line join=round,line cap=round,fill=fillColor] (469.30,239.55) circle (  1.16);

\path[draw=drawColor,line width= 0.4pt,line join=round,line cap=round,fill=fillColor] (469.44,239.55) circle (  1.16);

\path[draw=drawColor,line width= 0.4pt,line join=round,line cap=round,fill=fillColor] (469.58,239.47) circle (  1.16);

\path[draw=drawColor,line width= 0.4pt,line join=round,line cap=round,fill=fillColor] (469.73,239.32) circle (  1.16);

\path[draw=drawColor,line width= 0.4pt,line join=round,line cap=round,fill=fillColor] (469.87,239.11) circle (  1.16);

\path[draw=drawColor,line width= 0.4pt,line join=round,line cap=round,fill=fillColor] (470.01,239.05) circle (  1.16);

\path[draw=drawColor,line width= 0.4pt,line join=round,line cap=round,fill=fillColor] (470.15,238.73) circle (  1.16);

\path[draw=drawColor,line width= 0.4pt,line join=round,line cap=round,fill=fillColor] (470.29,238.52) circle (  1.16);

\path[draw=drawColor,line width= 0.4pt,line join=round,line cap=round,fill=fillColor] (470.43,238.28) circle (  1.16);

\path[draw=drawColor,line width= 0.4pt,line join=round,line cap=round,fill=fillColor] (470.57,238.03) circle (  1.16);

\path[draw=drawColor,line width= 0.4pt,line join=round,line cap=round,fill=fillColor] (470.71,237.83) circle (  1.16);

\path[draw=drawColor,line width= 0.4pt,line join=round,line cap=round,fill=fillColor] (470.85,237.74) circle (  1.16);

\path[draw=drawColor,line width= 0.4pt,line join=round,line cap=round,fill=fillColor] (470.99,237.63) circle (  1.16);

\path[draw=drawColor,line width= 0.4pt,line join=round,line cap=round,fill=fillColor] (471.13,237.59) circle (  1.16);

\path[draw=drawColor,line width= 0.4pt,line join=round,line cap=round,fill=fillColor] (471.27,237.36) circle (  1.16);

\path[draw=drawColor,line width= 0.4pt,line join=round,line cap=round,fill=fillColor] (471.41,237.15) circle (  1.16);

\path[draw=drawColor,line width= 0.4pt,line join=round,line cap=round,fill=fillColor] (471.55,236.92) circle (  1.16);

\path[draw=drawColor,line width= 0.4pt,line join=round,line cap=round,fill=fillColor] (471.69,236.69) circle (  1.16);

\path[draw=drawColor,line width= 0.4pt,line join=round,line cap=round,fill=fillColor] (471.82,236.67) circle (  1.16);

\path[draw=drawColor,line width= 0.4pt,line join=round,line cap=round,fill=fillColor] (471.96,236.39) circle (  1.16);

\path[draw=drawColor,line width= 0.4pt,line join=round,line cap=round,fill=fillColor] (472.10,236.28) circle (  1.16);

\path[draw=drawColor,line width= 0.4pt,line join=round,line cap=round,fill=fillColor] (472.24,236.27) circle (  1.16);

\path[draw=drawColor,line width= 0.4pt,line join=round,line cap=round,fill=fillColor] (472.37,236.05) circle (  1.16);

\path[draw=drawColor,line width= 0.4pt,line join=round,line cap=round,fill=fillColor] (472.51,235.70) circle (  1.16);

\path[draw=drawColor,line width= 0.4pt,line join=round,line cap=round,fill=fillColor] (472.65,235.36) circle (  1.16);

\path[draw=drawColor,line width= 0.4pt,line join=round,line cap=round,fill=fillColor] (472.78,235.30) circle (  1.16);

\path[draw=drawColor,line width= 0.4pt,line join=round,line cap=round,fill=fillColor] (472.92,235.18) circle (  1.16);

\path[draw=drawColor,line width= 0.4pt,line join=round,line cap=round,fill=fillColor] (473.06,234.63) circle (  1.16);

\path[draw=drawColor,line width= 0.4pt,line join=round,line cap=round,fill=fillColor] (473.19,234.40) circle (  1.16);

\path[draw=drawColor,line width= 0.4pt,line join=round,line cap=round,fill=fillColor] (473.33,234.34) circle (  1.16);

\path[draw=drawColor,line width= 0.4pt,line join=round,line cap=round,fill=fillColor] (473.46,234.28) circle (  1.16);

\path[draw=drawColor,line width= 0.4pt,line join=round,line cap=round,fill=fillColor] (473.60,234.19) circle (  1.16);

\path[draw=drawColor,line width= 0.4pt,line join=round,line cap=round,fill=fillColor] (473.74,234.19) circle (  1.16);

\path[draw=drawColor,line width= 0.4pt,line join=round,line cap=round,fill=fillColor] (473.87,233.74) circle (  1.16);

\path[draw=drawColor,line width= 0.4pt,line join=round,line cap=round,fill=fillColor] (474.00,233.62) circle (  1.16);

\path[draw=drawColor,line width= 0.4pt,line join=round,line cap=round,fill=fillColor] (474.14,233.35) circle (  1.16);

\path[draw=drawColor,line width= 0.4pt,line join=round,line cap=round,fill=fillColor] (474.27,233.31) circle (  1.16);

\path[draw=drawColor,line width= 0.4pt,line join=round,line cap=round,fill=fillColor] (474.41,233.20) circle (  1.16);

\path[draw=drawColor,line width= 0.4pt,line join=round,line cap=round,fill=fillColor] (474.54,233.11) circle (  1.16);

\path[draw=drawColor,line width= 0.4pt,line join=round,line cap=round,fill=fillColor] (474.68,232.93) circle (  1.16);

\path[draw=drawColor,line width= 0.4pt,line join=round,line cap=round,fill=fillColor] (474.81,232.80) circle (  1.16);

\path[draw=drawColor,line width= 0.4pt,line join=round,line cap=round,fill=fillColor] (474.94,232.75) circle (  1.16);

\path[draw=drawColor,line width= 0.4pt,line join=round,line cap=round,fill=fillColor] (475.07,232.19) circle (  1.16);

\path[draw=drawColor,line width= 0.4pt,line join=round,line cap=round,fill=fillColor] (475.21,232.12) circle (  1.16);

\path[draw=drawColor,line width= 0.4pt,line join=round,line cap=round,fill=fillColor] (475.34,231.90) circle (  1.16);

\path[draw=drawColor,line width= 0.4pt,line join=round,line cap=round,fill=fillColor] (475.47,231.63) circle (  1.16);

\path[draw=drawColor,line width= 0.4pt,line join=round,line cap=round,fill=fillColor] (475.60,231.11) circle (  1.16);

\path[draw=drawColor,line width= 0.4pt,line join=round,line cap=round,fill=fillColor] (475.74,230.96) circle (  1.16);

\path[draw=drawColor,line width= 0.4pt,line join=round,line cap=round,fill=fillColor] (475.87,230.76) circle (  1.16);

\path[draw=drawColor,line width= 0.4pt,line join=round,line cap=round,fill=fillColor] (476.00,230.57) circle (  1.16);

\path[draw=drawColor,line width= 0.4pt,line join=round,line cap=round,fill=fillColor] (476.13,230.37) circle (  1.16);

\path[draw=drawColor,line width= 0.4pt,line join=round,line cap=round,fill=fillColor] (476.26,230.05) circle (  1.16);

\path[draw=drawColor,line width= 0.4pt,line join=round,line cap=round,fill=fillColor] (476.39,229.73) circle (  1.16);

\path[draw=drawColor,line width= 0.4pt,line join=round,line cap=round,fill=fillColor] (476.52,229.70) circle (  1.16);

\path[draw=drawColor,line width= 0.4pt,line join=round,line cap=round,fill=fillColor] (476.65,229.36) circle (  1.16);

\path[draw=drawColor,line width= 0.4pt,line join=round,line cap=round,fill=fillColor] (476.78,227.91) circle (  1.16);

\path[draw=drawColor,line width= 0.4pt,line join=round,line cap=round,fill=fillColor] (476.91,227.54) circle (  1.16);

\path[draw=drawColor,line width= 0.4pt,line join=round,line cap=round,fill=fillColor] (477.04,226.41) circle (  1.16);

\path[draw=drawColor,line width= 0.4pt,line join=round,line cap=round,fill=fillColor] (477.17,225.98) circle (  1.16);

\path[draw=drawColor,line width= 0.4pt,line join=round,line cap=round,fill=fillColor] (477.30,225.83) circle (  1.16);

\path[draw=drawColor,line width= 0.4pt,line join=round,line cap=round,fill=fillColor] (477.43,225.48) circle (  1.16);

\path[draw=drawColor,line width= 0.4pt,line join=round,line cap=round,fill=fillColor] (477.56,223.89) circle (  1.16);

\path[draw=drawColor,line width= 0.4pt,line join=round,line cap=round,fill=fillColor] (477.69,209.81) circle (  1.16);

\path[draw=drawColor,line width= 0.4pt,line join=round,line cap=round,fill=fillColor] (477.82,209.81) circle (  1.16);

\path[draw=drawColor,line width= 0.4pt,line join=round,line cap=round,fill=fillColor] (477.95,209.81) circle (  1.16);

\path[draw=drawColor,line width= 0.4pt,line join=round,line cap=round,fill=fillColor] (478.08,209.81) circle (  1.16);

\path[draw=drawColor,line width= 0.4pt,line join=round,line cap=round,fill=fillColor] (478.21,209.81) circle (  1.16);
\definecolor[named]{drawColor}{rgb}{0.60,0.31,0.64}
\definecolor[named]{fillColor}{rgb}{0.60,0.31,0.64}

\path[draw=drawColor,line width= 0.4pt,line join=round,line cap=round,fill=fillColor] (327.82,279.64) circle (  1.16);

\path[draw=drawColor,line width= 0.4pt,line join=round,line cap=round,fill=fillColor] (333.69,273.84) circle (  1.16);

\path[draw=drawColor,line width= 0.4pt,line join=round,line cap=round,fill=fillColor] (337.81,273.50) circle (  1.16);

\path[draw=drawColor,line width= 0.4pt,line join=round,line cap=round,fill=fillColor] (341.08,272.64) circle (  1.16);

\path[draw=drawColor,line width= 0.4pt,line join=round,line cap=round,fill=fillColor] (343.85,269.73) circle (  1.16);

\path[draw=drawColor,line width= 0.4pt,line join=round,line cap=round,fill=fillColor] (346.27,269.59) circle (  1.16);

\path[draw=drawColor,line width= 0.4pt,line join=round,line cap=round,fill=fillColor] (348.43,269.06) circle (  1.16);

\path[draw=drawColor,line width= 0.4pt,line join=round,line cap=round,fill=fillColor] (350.39,267.01) circle (  1.16);

\path[draw=drawColor,line width= 0.4pt,line join=round,line cap=round,fill=fillColor] (352.20,266.31) circle (  1.16);

\path[draw=drawColor,line width= 0.4pt,line join=round,line cap=round,fill=fillColor] (353.88,265.06) circle (  1.16);

\path[draw=drawColor,line width= 0.4pt,line join=round,line cap=round,fill=fillColor] (355.45,264.58) circle (  1.16);

\path[draw=drawColor,line width= 0.4pt,line join=round,line cap=round,fill=fillColor] (356.93,264.48) circle (  1.16);

\path[draw=drawColor,line width= 0.4pt,line join=round,line cap=round,fill=fillColor] (358.32,264.43) circle (  1.16);

\path[draw=drawColor,line width= 0.4pt,line join=round,line cap=round,fill=fillColor] (359.65,263.09) circle (  1.16);

\path[draw=drawColor,line width= 0.4pt,line join=round,line cap=round,fill=fillColor] (360.92,262.84) circle (  1.16);

\path[draw=drawColor,line width= 0.4pt,line join=round,line cap=round,fill=fillColor] (362.13,262.84) circle (  1.16);

\path[draw=drawColor,line width= 0.4pt,line join=round,line cap=round,fill=fillColor] (363.29,262.68) circle (  1.16);

\path[draw=drawColor,line width= 0.4pt,line join=round,line cap=round,fill=fillColor] (364.40,262.23) circle (  1.16);

\path[draw=drawColor,line width= 0.4pt,line join=round,line cap=round,fill=fillColor] (365.48,262.08) circle (  1.16);

\path[draw=drawColor,line width= 0.4pt,line join=round,line cap=round,fill=fillColor] (366.52,262.02) circle (  1.16);

\path[draw=drawColor,line width= 0.4pt,line join=round,line cap=round,fill=fillColor] (367.52,262.01) circle (  1.16);

\path[draw=drawColor,line width= 0.4pt,line join=round,line cap=round,fill=fillColor] (368.49,261.94) circle (  1.16);

\path[draw=drawColor,line width= 0.4pt,line join=round,line cap=round,fill=fillColor] (369.44,261.76) circle (  1.16);

\path[draw=drawColor,line width= 0.4pt,line join=round,line cap=round,fill=fillColor] (370.36,261.71) circle (  1.16);

\path[draw=drawColor,line width= 0.4pt,line join=round,line cap=round,fill=fillColor] (371.25,261.39) circle (  1.16);

\path[draw=drawColor,line width= 0.4pt,line join=round,line cap=round,fill=fillColor] (372.12,261.12) circle (  1.16);

\path[draw=drawColor,line width= 0.4pt,line join=round,line cap=round,fill=fillColor] (372.96,260.92) circle (  1.16);

\path[draw=drawColor,line width= 0.4pt,line join=round,line cap=round,fill=fillColor] (373.79,260.64) circle (  1.16);

\path[draw=drawColor,line width= 0.4pt,line join=round,line cap=round,fill=fillColor] (374.59,260.61) circle (  1.16);

\path[draw=drawColor,line width= 0.4pt,line join=round,line cap=round,fill=fillColor] (375.38,260.36) circle (  1.16);

\path[draw=drawColor,line width= 0.4pt,line join=round,line cap=round,fill=fillColor] (376.15,260.28) circle (  1.16);

\path[draw=drawColor,line width= 0.4pt,line join=round,line cap=round,fill=fillColor] (376.91,260.00) circle (  1.16);

\path[draw=drawColor,line width= 0.4pt,line join=round,line cap=round,fill=fillColor] (377.65,259.83) circle (  1.16);

\path[draw=drawColor,line width= 0.4pt,line join=round,line cap=round,fill=fillColor] (378.37,259.74) circle (  1.16);

\path[draw=drawColor,line width= 0.4pt,line join=round,line cap=round,fill=fillColor] (379.08,259.53) circle (  1.16);

\path[draw=drawColor,line width= 0.4pt,line join=round,line cap=round,fill=fillColor] (379.78,259.40) circle (  1.16);

\path[draw=drawColor,line width= 0.4pt,line join=round,line cap=round,fill=fillColor] (380.46,259.26) circle (  1.16);

\path[draw=drawColor,line width= 0.4pt,line join=round,line cap=round,fill=fillColor] (381.13,259.00) circle (  1.16);

\path[draw=drawColor,line width= 0.4pt,line join=round,line cap=round,fill=fillColor] (381.79,258.66) circle (  1.16);

\path[draw=drawColor,line width= 0.4pt,line join=round,line cap=round,fill=fillColor] (382.44,258.44) circle (  1.16);

\path[draw=drawColor,line width= 0.4pt,line join=round,line cap=round,fill=fillColor] (383.08,258.23) circle (  1.16);

\path[draw=drawColor,line width= 0.4pt,line join=round,line cap=round,fill=fillColor] (383.71,258.13) circle (  1.16);

\path[draw=drawColor,line width= 0.4pt,line join=round,line cap=round,fill=fillColor] (384.32,258.08) circle (  1.16);

\path[draw=drawColor,line width= 0.4pt,line join=round,line cap=round,fill=fillColor] (384.93,257.89) circle (  1.16);

\path[draw=drawColor,line width= 0.4pt,line join=round,line cap=round,fill=fillColor] (385.53,257.86) circle (  1.16);

\path[draw=drawColor,line width= 0.4pt,line join=round,line cap=round,fill=fillColor] (386.12,257.83) circle (  1.16);

\path[draw=drawColor,line width= 0.4pt,line join=round,line cap=round,fill=fillColor] (386.70,257.66) circle (  1.16);

\path[draw=drawColor,line width= 0.4pt,line join=round,line cap=round,fill=fillColor] (387.28,257.56) circle (  1.16);

\path[draw=drawColor,line width= 0.4pt,line join=round,line cap=round,fill=fillColor] (387.84,257.51) circle (  1.16);

\path[draw=drawColor,line width= 0.4pt,line join=round,line cap=round,fill=fillColor] (388.40,257.36) circle (  1.16);

\path[draw=drawColor,line width= 0.4pt,line join=round,line cap=round,fill=fillColor] (388.95,257.34) circle (  1.16);

\path[draw=drawColor,line width= 0.4pt,line join=round,line cap=round,fill=fillColor] (389.50,257.30) circle (  1.16);

\path[draw=drawColor,line width= 0.4pt,line join=round,line cap=round,fill=fillColor] (390.03,257.20) circle (  1.16);

\path[draw=drawColor,line width= 0.4pt,line join=round,line cap=round,fill=fillColor] (390.56,256.91) circle (  1.16);

\path[draw=drawColor,line width= 0.4pt,line join=round,line cap=round,fill=fillColor] (391.08,256.76) circle (  1.16);

\path[draw=drawColor,line width= 0.4pt,line join=round,line cap=round,fill=fillColor] (391.60,256.76) circle (  1.16);

\path[draw=drawColor,line width= 0.4pt,line join=round,line cap=round,fill=fillColor] (392.11,256.66) circle (  1.16);

\path[draw=drawColor,line width= 0.4pt,line join=round,line cap=round,fill=fillColor] (392.62,256.63) circle (  1.16);

\path[draw=drawColor,line width= 0.4pt,line join=round,line cap=round,fill=fillColor] (393.12,256.30) circle (  1.16);

\path[draw=drawColor,line width= 0.4pt,line join=round,line cap=round,fill=fillColor] (393.61,256.17) circle (  1.16);

\path[draw=drawColor,line width= 0.4pt,line join=round,line cap=round,fill=fillColor] (394.10,255.75) circle (  1.16);

\path[draw=drawColor,line width= 0.4pt,line join=round,line cap=round,fill=fillColor] (394.58,255.71) circle (  1.16);

\path[draw=drawColor,line width= 0.4pt,line join=round,line cap=round,fill=fillColor] (395.06,255.65) circle (  1.16);

\path[draw=drawColor,line width= 0.4pt,line join=round,line cap=round,fill=fillColor] (395.53,255.54) circle (  1.16);

\path[draw=drawColor,line width= 0.4pt,line join=round,line cap=round,fill=fillColor] (396.00,255.48) circle (  1.16);

\path[draw=drawColor,line width= 0.4pt,line join=round,line cap=round,fill=fillColor] (396.46,255.47) circle (  1.16);

\path[draw=drawColor,line width= 0.4pt,line join=round,line cap=round,fill=fillColor] (396.92,255.43) circle (  1.16);

\path[draw=drawColor,line width= 0.4pt,line join=round,line cap=round,fill=fillColor] (397.37,255.25) circle (  1.16);

\path[draw=drawColor,line width= 0.4pt,line join=round,line cap=round,fill=fillColor] (397.82,255.19) circle (  1.16);

\path[draw=drawColor,line width= 0.4pt,line join=round,line cap=round,fill=fillColor] (398.27,254.99) circle (  1.16);

\path[draw=drawColor,line width= 0.4pt,line join=round,line cap=round,fill=fillColor] (398.71,254.85) circle (  1.16);

\path[draw=drawColor,line width= 0.4pt,line join=round,line cap=round,fill=fillColor] (399.15,254.83) circle (  1.16);

\path[draw=drawColor,line width= 0.4pt,line join=round,line cap=round,fill=fillColor] (399.58,254.79) circle (  1.16);

\path[draw=drawColor,line width= 0.4pt,line join=round,line cap=round,fill=fillColor] (400.01,254.74) circle (  1.16);

\path[draw=drawColor,line width= 0.4pt,line join=round,line cap=round,fill=fillColor] (400.43,254.40) circle (  1.16);

\path[draw=drawColor,line width= 0.4pt,line join=round,line cap=round,fill=fillColor] (400.85,254.39) circle (  1.16);

\path[draw=drawColor,line width= 0.4pt,line join=round,line cap=round,fill=fillColor] (401.27,254.29) circle (  1.16);

\path[draw=drawColor,line width= 0.4pt,line join=round,line cap=round,fill=fillColor] (401.69,254.28) circle (  1.16);

\path[draw=drawColor,line width= 0.4pt,line join=round,line cap=round,fill=fillColor] (402.10,254.02) circle (  1.16);

\path[draw=drawColor,line width= 0.4pt,line join=round,line cap=round,fill=fillColor] (402.50,253.88) circle (  1.16);

\path[draw=drawColor,line width= 0.4pt,line join=round,line cap=round,fill=fillColor] (402.91,253.81) circle (  1.16);

\path[draw=drawColor,line width= 0.4pt,line join=round,line cap=round,fill=fillColor] (403.31,253.76) circle (  1.16);

\path[draw=drawColor,line width= 0.4pt,line join=round,line cap=round,fill=fillColor] (403.70,253.73) circle (  1.16);

\path[draw=drawColor,line width= 0.4pt,line join=round,line cap=round,fill=fillColor] (404.10,253.71) circle (  1.16);

\path[draw=drawColor,line width= 0.4pt,line join=round,line cap=round,fill=fillColor] (404.49,253.57) circle (  1.16);

\path[draw=drawColor,line width= 0.4pt,line join=round,line cap=round,fill=fillColor] (404.88,253.50) circle (  1.16);

\path[draw=drawColor,line width= 0.4pt,line join=round,line cap=round,fill=fillColor] (405.26,253.44) circle (  1.16);

\path[draw=drawColor,line width= 0.4pt,line join=round,line cap=round,fill=fillColor] (405.64,253.24) circle (  1.16);

\path[draw=drawColor,line width= 0.4pt,line join=round,line cap=round,fill=fillColor] (406.02,253.16) circle (  1.16);

\path[draw=drawColor,line width= 0.4pt,line join=round,line cap=round,fill=fillColor] (406.40,253.10) circle (  1.16);

\path[draw=drawColor,line width= 0.4pt,line join=round,line cap=round,fill=fillColor] (406.77,252.87) circle (  1.16);

\path[draw=drawColor,line width= 0.4pt,line join=round,line cap=round,fill=fillColor] (407.14,252.80) circle (  1.16);

\path[draw=drawColor,line width= 0.4pt,line join=round,line cap=round,fill=fillColor] (407.51,252.72) circle (  1.16);

\path[draw=drawColor,line width= 0.4pt,line join=round,line cap=round,fill=fillColor] (407.87,252.62) circle (  1.16);

\path[draw=drawColor,line width= 0.4pt,line join=round,line cap=round,fill=fillColor] (408.24,252.43) circle (  1.16);

\path[draw=drawColor,line width= 0.4pt,line join=round,line cap=round,fill=fillColor] (408.60,252.35) circle (  1.16);

\path[draw=drawColor,line width= 0.4pt,line join=round,line cap=round,fill=fillColor] (408.95,252.34) circle (  1.16);

\path[draw=drawColor,line width= 0.4pt,line join=round,line cap=round,fill=fillColor] (409.31,252.32) circle (  1.16);

\path[draw=drawColor,line width= 0.4pt,line join=round,line cap=round,fill=fillColor] (409.66,252.20) circle (  1.16);

\path[draw=drawColor,line width= 0.4pt,line join=round,line cap=round,fill=fillColor] (410.01,252.16) circle (  1.16);

\path[draw=drawColor,line width= 0.4pt,line join=round,line cap=round,fill=fillColor] (410.36,252.12) circle (  1.16);

\path[draw=drawColor,line width= 0.4pt,line join=round,line cap=round,fill=fillColor] (410.71,252.09) circle (  1.16);

\path[draw=drawColor,line width= 0.4pt,line join=round,line cap=round,fill=fillColor] (411.05,251.84) circle (  1.16);

\path[draw=drawColor,line width= 0.4pt,line join=round,line cap=round,fill=fillColor] (411.39,251.71) circle (  1.16);

\path[draw=drawColor,line width= 0.4pt,line join=round,line cap=round,fill=fillColor] (411.73,251.64) circle (  1.16);

\path[draw=drawColor,line width= 0.4pt,line join=round,line cap=round,fill=fillColor] (412.07,251.61) circle (  1.16);

\path[draw=drawColor,line width= 0.4pt,line join=round,line cap=round,fill=fillColor] (412.40,251.61) circle (  1.16);

\path[draw=drawColor,line width= 0.4pt,line join=round,line cap=round,fill=fillColor] (412.73,251.57) circle (  1.16);

\path[draw=drawColor,line width= 0.4pt,line join=round,line cap=round,fill=fillColor] (413.07,251.54) circle (  1.16);

\path[draw=drawColor,line width= 0.4pt,line join=round,line cap=round,fill=fillColor] (413.39,251.50) circle (  1.16);

\path[draw=drawColor,line width= 0.4pt,line join=round,line cap=round,fill=fillColor] (413.72,251.40) circle (  1.16);

\path[draw=drawColor,line width= 0.4pt,line join=round,line cap=round,fill=fillColor] (414.05,251.24) circle (  1.16);

\path[draw=drawColor,line width= 0.4pt,line join=round,line cap=round,fill=fillColor] (414.37,251.19) circle (  1.16);

\path[draw=drawColor,line width= 0.4pt,line join=round,line cap=round,fill=fillColor] (414.69,251.09) circle (  1.16);

\path[draw=drawColor,line width= 0.4pt,line join=round,line cap=round,fill=fillColor] (415.01,251.04) circle (  1.16);

\path[draw=drawColor,line width= 0.4pt,line join=round,line cap=round,fill=fillColor] (415.33,251.04) circle (  1.16);

\path[draw=drawColor,line width= 0.4pt,line join=round,line cap=round,fill=fillColor] (415.64,250.93) circle (  1.16);

\path[draw=drawColor,line width= 0.4pt,line join=round,line cap=round,fill=fillColor] (415.95,250.67) circle (  1.16);

\path[draw=drawColor,line width= 0.4pt,line join=round,line cap=round,fill=fillColor] (416.27,250.65) circle (  1.16);

\path[draw=drawColor,line width= 0.4pt,line join=round,line cap=round,fill=fillColor] (416.58,250.64) circle (  1.16);

\path[draw=drawColor,line width= 0.4pt,line join=round,line cap=round,fill=fillColor] (416.88,250.52) circle (  1.16);

\path[draw=drawColor,line width= 0.4pt,line join=round,line cap=round,fill=fillColor] (417.19,250.50) circle (  1.16);

\path[draw=drawColor,line width= 0.4pt,line join=round,line cap=round,fill=fillColor] (417.50,250.48) circle (  1.16);

\path[draw=drawColor,line width= 0.4pt,line join=round,line cap=round,fill=fillColor] (417.80,250.44) circle (  1.16);

\path[draw=drawColor,line width= 0.4pt,line join=round,line cap=round,fill=fillColor] (418.10,250.41) circle (  1.16);

\path[draw=drawColor,line width= 0.4pt,line join=round,line cap=round,fill=fillColor] (418.40,250.29) circle (  1.16);

\path[draw=drawColor,line width= 0.4pt,line join=round,line cap=round,fill=fillColor] (418.70,250.24) circle (  1.16);

\path[draw=drawColor,line width= 0.4pt,line join=round,line cap=round,fill=fillColor] (419.00,250.17) circle (  1.16);

\path[draw=drawColor,line width= 0.4pt,line join=round,line cap=round,fill=fillColor] (419.29,250.15) circle (  1.16);

\path[draw=drawColor,line width= 0.4pt,line join=round,line cap=round,fill=fillColor] (419.59,250.10) circle (  1.16);

\path[draw=drawColor,line width= 0.4pt,line join=round,line cap=round,fill=fillColor] (419.88,250.09) circle (  1.16);

\path[draw=drawColor,line width= 0.4pt,line join=round,line cap=round,fill=fillColor] (420.17,249.99) circle (  1.16);

\path[draw=drawColor,line width= 0.4pt,line join=round,line cap=round,fill=fillColor] (420.46,249.96) circle (  1.16);

\path[draw=drawColor,line width= 0.4pt,line join=round,line cap=round,fill=fillColor] (420.75,249.96) circle (  1.16);

\path[draw=drawColor,line width= 0.4pt,line join=round,line cap=round,fill=fillColor] (421.03,249.94) circle (  1.16);

\path[draw=drawColor,line width= 0.4pt,line join=round,line cap=round,fill=fillColor] (421.32,249.89) circle (  1.16);

\path[draw=drawColor,line width= 0.4pt,line join=round,line cap=round,fill=fillColor] (421.60,249.87) circle (  1.16);

\path[draw=drawColor,line width= 0.4pt,line join=round,line cap=round,fill=fillColor] (421.89,249.84) circle (  1.16);

\path[draw=drawColor,line width= 0.4pt,line join=round,line cap=round,fill=fillColor] (422.17,249.67) circle (  1.16);

\path[draw=drawColor,line width= 0.4pt,line join=round,line cap=round,fill=fillColor] (422.45,249.64) circle (  1.16);

\path[draw=drawColor,line width= 0.4pt,line join=round,line cap=round,fill=fillColor] (422.72,249.63) circle (  1.16);

\path[draw=drawColor,line width= 0.4pt,line join=round,line cap=round,fill=fillColor] (423.00,249.63) circle (  1.16);

\path[draw=drawColor,line width= 0.4pt,line join=round,line cap=round,fill=fillColor] (423.28,249.59) circle (  1.16);

\path[draw=drawColor,line width= 0.4pt,line join=round,line cap=round,fill=fillColor] (423.55,249.51) circle (  1.16);

\path[draw=drawColor,line width= 0.4pt,line join=round,line cap=round,fill=fillColor] (423.82,249.50) circle (  1.16);

\path[draw=drawColor,line width= 0.4pt,line join=round,line cap=round,fill=fillColor] (424.10,249.28) circle (  1.16);

\path[draw=drawColor,line width= 0.4pt,line join=round,line cap=round,fill=fillColor] (424.37,249.27) circle (  1.16);

\path[draw=drawColor,line width= 0.4pt,line join=round,line cap=round,fill=fillColor] (424.64,249.24) circle (  1.16);

\path[draw=drawColor,line width= 0.4pt,line join=round,line cap=round,fill=fillColor] (424.91,249.14) circle (  1.16);

\path[draw=drawColor,line width= 0.4pt,line join=round,line cap=round,fill=fillColor] (425.17,248.97) circle (  1.16);

\path[draw=drawColor,line width= 0.4pt,line join=round,line cap=round,fill=fillColor] (425.44,248.96) circle (  1.16);

\path[draw=drawColor,line width= 0.4pt,line join=round,line cap=round,fill=fillColor] (425.70,248.82) circle (  1.16);

\path[draw=drawColor,line width= 0.4pt,line join=round,line cap=round,fill=fillColor] (425.97,248.74) circle (  1.16);

\path[draw=drawColor,line width= 0.4pt,line join=round,line cap=round,fill=fillColor] (426.23,248.47) circle (  1.16);

\path[draw=drawColor,line width= 0.4pt,line join=round,line cap=round,fill=fillColor] (426.49,248.44) circle (  1.16);

\path[draw=drawColor,line width= 0.4pt,line join=round,line cap=round,fill=fillColor] (426.75,248.43) circle (  1.16);

\path[draw=drawColor,line width= 0.4pt,line join=round,line cap=round,fill=fillColor] (427.01,248.27) circle (  1.16);

\path[draw=drawColor,line width= 0.4pt,line join=round,line cap=round,fill=fillColor] (427.27,248.26) circle (  1.16);

\path[draw=drawColor,line width= 0.4pt,line join=round,line cap=round,fill=fillColor] (427.52,248.23) circle (  1.16);

\path[draw=drawColor,line width= 0.4pt,line join=round,line cap=round,fill=fillColor] (427.78,248.20) circle (  1.16);

\path[draw=drawColor,line width= 0.4pt,line join=round,line cap=round,fill=fillColor] (428.03,248.17) circle (  1.16);

\path[draw=drawColor,line width= 0.4pt,line join=round,line cap=round,fill=fillColor] (428.29,248.16) circle (  1.16);

\path[draw=drawColor,line width= 0.4pt,line join=round,line cap=round,fill=fillColor] (428.54,248.14) circle (  1.16);

\path[draw=drawColor,line width= 0.4pt,line join=round,line cap=round,fill=fillColor] (428.79,248.09) circle (  1.16);

\path[draw=drawColor,line width= 0.4pt,line join=round,line cap=round,fill=fillColor] (429.04,248.08) circle (  1.16);

\path[draw=drawColor,line width= 0.4pt,line join=round,line cap=round,fill=fillColor] (429.29,248.06) circle (  1.16);

\path[draw=drawColor,line width= 0.4pt,line join=round,line cap=round,fill=fillColor] (429.54,248.04) circle (  1.16);

\path[draw=drawColor,line width= 0.4pt,line join=round,line cap=round,fill=fillColor] (429.79,248.00) circle (  1.16);

\path[draw=drawColor,line width= 0.4pt,line join=round,line cap=round,fill=fillColor] (430.04,248.00) circle (  1.16);

\path[draw=drawColor,line width= 0.4pt,line join=round,line cap=round,fill=fillColor] (430.28,247.99) circle (  1.16);

\path[draw=drawColor,line width= 0.4pt,line join=round,line cap=round,fill=fillColor] (430.53,247.96) circle (  1.16);

\path[draw=drawColor,line width= 0.4pt,line join=round,line cap=round,fill=fillColor] (430.77,247.88) circle (  1.16);

\path[draw=drawColor,line width= 0.4pt,line join=round,line cap=round,fill=fillColor] (431.01,247.88) circle (  1.16);

\path[draw=drawColor,line width= 0.4pt,line join=round,line cap=round,fill=fillColor] (431.25,247.85) circle (  1.16);

\path[draw=drawColor,line width= 0.4pt,line join=round,line cap=round,fill=fillColor] (431.50,247.81) circle (  1.16);

\path[draw=drawColor,line width= 0.4pt,line join=round,line cap=round,fill=fillColor] (431.74,247.81) circle (  1.16);

\path[draw=drawColor,line width= 0.4pt,line join=round,line cap=round,fill=fillColor] (431.97,247.80) circle (  1.16);

\path[draw=drawColor,line width= 0.4pt,line join=round,line cap=round,fill=fillColor] (432.21,247.78) circle (  1.16);

\path[draw=drawColor,line width= 0.4pt,line join=round,line cap=round,fill=fillColor] (432.45,247.73) circle (  1.16);

\path[draw=drawColor,line width= 0.4pt,line join=round,line cap=round,fill=fillColor] (432.69,247.71) circle (  1.16);

\path[draw=drawColor,line width= 0.4pt,line join=round,line cap=round,fill=fillColor] (432.92,247.69) circle (  1.16);

\path[draw=drawColor,line width= 0.4pt,line join=round,line cap=round,fill=fillColor] (433.16,247.56) circle (  1.16);

\path[draw=drawColor,line width= 0.4pt,line join=round,line cap=round,fill=fillColor] (433.39,247.34) circle (  1.16);

\path[draw=drawColor,line width= 0.4pt,line join=round,line cap=round,fill=fillColor] (433.62,247.28) circle (  1.16);

\path[draw=drawColor,line width= 0.4pt,line join=round,line cap=round,fill=fillColor] (433.86,247.23) circle (  1.16);

\path[draw=drawColor,line width= 0.4pt,line join=round,line cap=round,fill=fillColor] (434.09,247.10) circle (  1.16);

\path[draw=drawColor,line width= 0.4pt,line join=round,line cap=round,fill=fillColor] (434.32,247.10) circle (  1.16);

\path[draw=drawColor,line width= 0.4pt,line join=round,line cap=round,fill=fillColor] (434.55,247.07) circle (  1.16);

\path[draw=drawColor,line width= 0.4pt,line join=round,line cap=round,fill=fillColor] (434.78,247.04) circle (  1.16);

\path[draw=drawColor,line width= 0.4pt,line join=round,line cap=round,fill=fillColor] (435.00,247.04) circle (  1.16);

\path[draw=drawColor,line width= 0.4pt,line join=round,line cap=round,fill=fillColor] (435.23,247.03) circle (  1.16);

\path[draw=drawColor,line width= 0.4pt,line join=round,line cap=round,fill=fillColor] (435.46,246.94) circle (  1.16);

\path[draw=drawColor,line width= 0.4pt,line join=round,line cap=round,fill=fillColor] (435.68,246.90) circle (  1.16);

\path[draw=drawColor,line width= 0.4pt,line join=round,line cap=round,fill=fillColor] (435.91,246.70) circle (  1.16);

\path[draw=drawColor,line width= 0.4pt,line join=round,line cap=round,fill=fillColor] (436.13,246.70) circle (  1.16);

\path[draw=drawColor,line width= 0.4pt,line join=round,line cap=round,fill=fillColor] (436.36,246.66) circle (  1.16);

\path[draw=drawColor,line width= 0.4pt,line join=round,line cap=round,fill=fillColor] (436.58,246.57) circle (  1.16);

\path[draw=drawColor,line width= 0.4pt,line join=round,line cap=round,fill=fillColor] (436.80,246.55) circle (  1.16);

\path[draw=drawColor,line width= 0.4pt,line join=round,line cap=round,fill=fillColor] (437.02,246.48) circle (  1.16);

\path[draw=drawColor,line width= 0.4pt,line join=round,line cap=round,fill=fillColor] (437.24,246.45) circle (  1.16);

\path[draw=drawColor,line width= 0.4pt,line join=round,line cap=round,fill=fillColor] (437.46,246.45) circle (  1.16);

\path[draw=drawColor,line width= 0.4pt,line join=round,line cap=round,fill=fillColor] (437.68,246.42) circle (  1.16);

\path[draw=drawColor,line width= 0.4pt,line join=round,line cap=round,fill=fillColor] (437.90,246.39) circle (  1.16);

\path[draw=drawColor,line width= 0.4pt,line join=round,line cap=round,fill=fillColor] (438.12,246.25) circle (  1.16);

\path[draw=drawColor,line width= 0.4pt,line join=round,line cap=round,fill=fillColor] (438.33,246.17) circle (  1.16);

\path[draw=drawColor,line width= 0.4pt,line join=round,line cap=round,fill=fillColor] (438.55,246.12) circle (  1.16);

\path[draw=drawColor,line width= 0.4pt,line join=round,line cap=round,fill=fillColor] (438.76,246.07) circle (  1.16);

\path[draw=drawColor,line width= 0.4pt,line join=round,line cap=round,fill=fillColor] (438.98,246.04) circle (  1.16);

\path[draw=drawColor,line width= 0.4pt,line join=round,line cap=round,fill=fillColor] (439.19,245.97) circle (  1.16);

\path[draw=drawColor,line width= 0.4pt,line join=round,line cap=round,fill=fillColor] (439.41,245.78) circle (  1.16);

\path[draw=drawColor,line width= 0.4pt,line join=round,line cap=round,fill=fillColor] (439.62,245.75) circle (  1.16);

\path[draw=drawColor,line width= 0.4pt,line join=round,line cap=round,fill=fillColor] (439.83,245.71) circle (  1.16);

\path[draw=drawColor,line width= 0.4pt,line join=round,line cap=round,fill=fillColor] (440.04,245.70) circle (  1.16);

\path[draw=drawColor,line width= 0.4pt,line join=round,line cap=round,fill=fillColor] (440.25,245.63) circle (  1.16);

\path[draw=drawColor,line width= 0.4pt,line join=round,line cap=round,fill=fillColor] (440.46,245.60) circle (  1.16);

\path[draw=drawColor,line width= 0.4pt,line join=round,line cap=round,fill=fillColor] (440.67,245.60) circle (  1.16);

\path[draw=drawColor,line width= 0.4pt,line join=round,line cap=round,fill=fillColor] (440.88,245.57) circle (  1.16);

\path[draw=drawColor,line width= 0.4pt,line join=round,line cap=round,fill=fillColor] (441.09,245.49) circle (  1.16);

\path[draw=drawColor,line width= 0.4pt,line join=round,line cap=round,fill=fillColor] (441.29,245.40) circle (  1.16);

\path[draw=drawColor,line width= 0.4pt,line join=round,line cap=round,fill=fillColor] (441.50,245.38) circle (  1.16);

\path[draw=drawColor,line width= 0.4pt,line join=round,line cap=round,fill=fillColor] (441.71,245.30) circle (  1.16);

\path[draw=drawColor,line width= 0.4pt,line join=round,line cap=round,fill=fillColor] (441.91,245.24) circle (  1.16);

\path[draw=drawColor,line width= 0.4pt,line join=round,line cap=round,fill=fillColor] (442.12,245.20) circle (  1.16);

\path[draw=drawColor,line width= 0.4pt,line join=round,line cap=round,fill=fillColor] (442.32,245.20) circle (  1.16);

\path[draw=drawColor,line width= 0.4pt,line join=round,line cap=round,fill=fillColor] (442.53,245.17) circle (  1.16);

\path[draw=drawColor,line width= 0.4pt,line join=round,line cap=round,fill=fillColor] (442.73,245.10) circle (  1.16);

\path[draw=drawColor,line width= 0.4pt,line join=round,line cap=round,fill=fillColor] (442.93,245.08) circle (  1.16);

\path[draw=drawColor,line width= 0.4pt,line join=round,line cap=round,fill=fillColor] (443.13,244.99) circle (  1.16);

\path[draw=drawColor,line width= 0.4pt,line join=round,line cap=round,fill=fillColor] (443.33,244.95) circle (  1.16);

\path[draw=drawColor,line width= 0.4pt,line join=round,line cap=round,fill=fillColor] (443.54,244.94) circle (  1.16);

\path[draw=drawColor,line width= 0.4pt,line join=round,line cap=round,fill=fillColor] (443.74,244.89) circle (  1.16);

\path[draw=drawColor,line width= 0.4pt,line join=round,line cap=round,fill=fillColor] (443.94,244.87) circle (  1.16);

\path[draw=drawColor,line width= 0.4pt,line join=round,line cap=round,fill=fillColor] (444.13,244.85) circle (  1.16);

\path[draw=drawColor,line width= 0.4pt,line join=round,line cap=round,fill=fillColor] (444.33,244.83) circle (  1.16);

\path[draw=drawColor,line width= 0.4pt,line join=round,line cap=round,fill=fillColor] (444.53,244.75) circle (  1.16);

\path[draw=drawColor,line width= 0.4pt,line join=round,line cap=round,fill=fillColor] (444.73,244.69) circle (  1.16);

\path[draw=drawColor,line width= 0.4pt,line join=round,line cap=round,fill=fillColor] (444.92,244.66) circle (  1.16);

\path[draw=drawColor,line width= 0.4pt,line join=round,line cap=round,fill=fillColor] (445.12,244.64) circle (  1.16);

\path[draw=drawColor,line width= 0.4pt,line join=round,line cap=round,fill=fillColor] (445.32,244.64) circle (  1.16);

\path[draw=drawColor,line width= 0.4pt,line join=round,line cap=round,fill=fillColor] (445.51,244.63) circle (  1.16);

\path[draw=drawColor,line width= 0.4pt,line join=round,line cap=round,fill=fillColor] (445.71,244.63) circle (  1.16);

\path[draw=drawColor,line width= 0.4pt,line join=round,line cap=round,fill=fillColor] (445.90,244.60) circle (  1.16);

\path[draw=drawColor,line width= 0.4pt,line join=round,line cap=round,fill=fillColor] (446.09,244.51) circle (  1.16);

\path[draw=drawColor,line width= 0.4pt,line join=round,line cap=round,fill=fillColor] (446.29,244.40) circle (  1.16);

\path[draw=drawColor,line width= 0.4pt,line join=round,line cap=round,fill=fillColor] (446.48,244.29) circle (  1.16);

\path[draw=drawColor,line width= 0.4pt,line join=round,line cap=round,fill=fillColor] (446.67,244.29) circle (  1.16);

\path[draw=drawColor,line width= 0.4pt,line join=round,line cap=round,fill=fillColor] (446.86,244.28) circle (  1.16);

\path[draw=drawColor,line width= 0.4pt,line join=round,line cap=round,fill=fillColor] (447.05,244.28) circle (  1.16);

\path[draw=drawColor,line width= 0.4pt,line join=round,line cap=round,fill=fillColor] (447.24,244.27) circle (  1.16);

\path[draw=drawColor,line width= 0.4pt,line join=round,line cap=round,fill=fillColor] (447.43,244.23) circle (  1.16);

\path[draw=drawColor,line width= 0.4pt,line join=round,line cap=round,fill=fillColor] (447.62,244.07) circle (  1.16);

\path[draw=drawColor,line width= 0.4pt,line join=round,line cap=round,fill=fillColor] (447.81,244.06) circle (  1.16);

\path[draw=drawColor,line width= 0.4pt,line join=round,line cap=round,fill=fillColor] (448.00,244.04) circle (  1.16);

\path[draw=drawColor,line width= 0.4pt,line join=round,line cap=round,fill=fillColor] (448.19,244.02) circle (  1.16);

\path[draw=drawColor,line width= 0.4pt,line join=round,line cap=round,fill=fillColor] (448.37,243.94) circle (  1.16);

\path[draw=drawColor,line width= 0.4pt,line join=round,line cap=round,fill=fillColor] (448.56,243.90) circle (  1.16);

\path[draw=drawColor,line width= 0.4pt,line join=round,line cap=round,fill=fillColor] (448.75,243.89) circle (  1.16);

\path[draw=drawColor,line width= 0.4pt,line join=round,line cap=round,fill=fillColor] (448.93,243.89) circle (  1.16);

\path[draw=drawColor,line width= 0.4pt,line join=round,line cap=round,fill=fillColor] (449.12,243.85) circle (  1.16);

\path[draw=drawColor,line width= 0.4pt,line join=round,line cap=round,fill=fillColor] (449.30,243.80) circle (  1.16);

\path[draw=drawColor,line width= 0.4pt,line join=round,line cap=round,fill=fillColor] (449.49,243.75) circle (  1.16);

\path[draw=drawColor,line width= 0.4pt,line join=round,line cap=round,fill=fillColor] (449.67,243.68) circle (  1.16);

\path[draw=drawColor,line width= 0.4pt,line join=round,line cap=round,fill=fillColor] (449.86,243.61) circle (  1.16);

\path[draw=drawColor,line width= 0.4pt,line join=round,line cap=round,fill=fillColor] (450.04,243.56) circle (  1.16);

\path[draw=drawColor,line width= 0.4pt,line join=round,line cap=round,fill=fillColor] (450.22,243.48) circle (  1.16);

\path[draw=drawColor,line width= 0.4pt,line join=round,line cap=round,fill=fillColor] (450.40,243.45) circle (  1.16);

\path[draw=drawColor,line width= 0.4pt,line join=round,line cap=round,fill=fillColor] (450.58,243.42) circle (  1.16);

\path[draw=drawColor,line width= 0.4pt,line join=round,line cap=round,fill=fillColor] (450.77,243.39) circle (  1.16);

\path[draw=drawColor,line width= 0.4pt,line join=round,line cap=round,fill=fillColor] (450.95,243.39) circle (  1.16);

\path[draw=drawColor,line width= 0.4pt,line join=round,line cap=round,fill=fillColor] (451.13,243.39) circle (  1.16);

\path[draw=drawColor,line width= 0.4pt,line join=round,line cap=round,fill=fillColor] (451.31,243.34) circle (  1.16);

\path[draw=drawColor,line width= 0.4pt,line join=round,line cap=round,fill=fillColor] (451.49,243.28) circle (  1.16);

\path[draw=drawColor,line width= 0.4pt,line join=round,line cap=round,fill=fillColor] (451.67,243.19) circle (  1.16);

\path[draw=drawColor,line width= 0.4pt,line join=round,line cap=round,fill=fillColor] (451.84,243.17) circle (  1.16);

\path[draw=drawColor,line width= 0.4pt,line join=round,line cap=round,fill=fillColor] (452.02,243.06) circle (  1.16);

\path[draw=drawColor,line width= 0.4pt,line join=round,line cap=round,fill=fillColor] (452.20,243.00) circle (  1.16);

\path[draw=drawColor,line width= 0.4pt,line join=round,line cap=round,fill=fillColor] (452.38,242.96) circle (  1.16);

\path[draw=drawColor,line width= 0.4pt,line join=round,line cap=round,fill=fillColor] (452.55,242.93) circle (  1.16);

\path[draw=drawColor,line width= 0.4pt,line join=round,line cap=round,fill=fillColor] (452.73,242.90) circle (  1.16);

\path[draw=drawColor,line width= 0.4pt,line join=round,line cap=round,fill=fillColor] (452.91,242.84) circle (  1.16);

\path[draw=drawColor,line width= 0.4pt,line join=round,line cap=round,fill=fillColor] (453.08,242.84) circle (  1.16);

\path[draw=drawColor,line width= 0.4pt,line join=round,line cap=round,fill=fillColor] (453.26,242.77) circle (  1.16);

\path[draw=drawColor,line width= 0.4pt,line join=round,line cap=round,fill=fillColor] (453.43,242.72) circle (  1.16);

\path[draw=drawColor,line width= 0.4pt,line join=round,line cap=round,fill=fillColor] (453.61,242.72) circle (  1.16);

\path[draw=drawColor,line width= 0.4pt,line join=round,line cap=round,fill=fillColor] (453.78,242.67) circle (  1.16);

\path[draw=drawColor,line width= 0.4pt,line join=round,line cap=round,fill=fillColor] (453.95,242.57) circle (  1.16);

\path[draw=drawColor,line width= 0.4pt,line join=round,line cap=round,fill=fillColor] (454.13,242.54) circle (  1.16);

\path[draw=drawColor,line width= 0.4pt,line join=round,line cap=round,fill=fillColor] (454.30,242.49) circle (  1.16);

\path[draw=drawColor,line width= 0.4pt,line join=round,line cap=round,fill=fillColor] (454.47,242.48) circle (  1.16);

\path[draw=drawColor,line width= 0.4pt,line join=round,line cap=round,fill=fillColor] (454.64,242.47) circle (  1.16);

\path[draw=drawColor,line width= 0.4pt,line join=round,line cap=round,fill=fillColor] (454.82,242.42) circle (  1.16);

\path[draw=drawColor,line width= 0.4pt,line join=round,line cap=round,fill=fillColor] (454.99,242.39) circle (  1.16);

\path[draw=drawColor,line width= 0.4pt,line join=round,line cap=round,fill=fillColor] (455.16,242.39) circle (  1.16);

\path[draw=drawColor,line width= 0.4pt,line join=round,line cap=round,fill=fillColor] (455.33,242.28) circle (  1.16);

\path[draw=drawColor,line width= 0.4pt,line join=round,line cap=round,fill=fillColor] (455.50,242.28) circle (  1.16);

\path[draw=drawColor,line width= 0.4pt,line join=round,line cap=round,fill=fillColor] (455.67,242.27) circle (  1.16);

\path[draw=drawColor,line width= 0.4pt,line join=round,line cap=round,fill=fillColor] (455.84,242.27) circle (  1.16);

\path[draw=drawColor,line width= 0.4pt,line join=round,line cap=round,fill=fillColor] (456.00,242.24) circle (  1.16);

\path[draw=drawColor,line width= 0.4pt,line join=round,line cap=round,fill=fillColor] (456.17,242.14) circle (  1.16);

\path[draw=drawColor,line width= 0.4pt,line join=round,line cap=round,fill=fillColor] (456.34,242.14) circle (  1.16);

\path[draw=drawColor,line width= 0.4pt,line join=round,line cap=round,fill=fillColor] (456.51,242.09) circle (  1.16);

\path[draw=drawColor,line width= 0.4pt,line join=round,line cap=round,fill=fillColor] (456.68,242.08) circle (  1.16);

\path[draw=drawColor,line width= 0.4pt,line join=round,line cap=round,fill=fillColor] (456.84,242.07) circle (  1.16);

\path[draw=drawColor,line width= 0.4pt,line join=round,line cap=round,fill=fillColor] (457.01,242.01) circle (  1.16);

\path[draw=drawColor,line width= 0.4pt,line join=round,line cap=round,fill=fillColor] (457.18,241.98) circle (  1.16);

\path[draw=drawColor,line width= 0.4pt,line join=round,line cap=round,fill=fillColor] (457.34,241.97) circle (  1.16);

\path[draw=drawColor,line width= 0.4pt,line join=round,line cap=round,fill=fillColor] (457.51,241.94) circle (  1.16);

\path[draw=drawColor,line width= 0.4pt,line join=round,line cap=round,fill=fillColor] (457.67,241.82) circle (  1.16);

\path[draw=drawColor,line width= 0.4pt,line join=round,line cap=round,fill=fillColor] (457.84,241.81) circle (  1.16);

\path[draw=drawColor,line width= 0.4pt,line join=round,line cap=round,fill=fillColor] (458.00,241.77) circle (  1.16);

\path[draw=drawColor,line width= 0.4pt,line join=round,line cap=round,fill=fillColor] (458.17,241.73) circle (  1.16);

\path[draw=drawColor,line width= 0.4pt,line join=round,line cap=round,fill=fillColor] (458.33,241.68) circle (  1.16);

\path[draw=drawColor,line width= 0.4pt,line join=round,line cap=round,fill=fillColor] (458.49,241.68) circle (  1.16);

\path[draw=drawColor,line width= 0.4pt,line join=round,line cap=round,fill=fillColor] (458.66,241.65) circle (  1.16);

\path[draw=drawColor,line width= 0.4pt,line join=round,line cap=round,fill=fillColor] (458.82,241.65) circle (  1.16);

\path[draw=drawColor,line width= 0.4pt,line join=round,line cap=round,fill=fillColor] (458.98,241.56) circle (  1.16);

\path[draw=drawColor,line width= 0.4pt,line join=round,line cap=round,fill=fillColor] (459.14,241.48) circle (  1.16);

\path[draw=drawColor,line width= 0.4pt,line join=round,line cap=round,fill=fillColor] (459.30,241.46) circle (  1.16);

\path[draw=drawColor,line width= 0.4pt,line join=round,line cap=round,fill=fillColor] (459.47,241.44) circle (  1.16);

\path[draw=drawColor,line width= 0.4pt,line join=round,line cap=round,fill=fillColor] (459.63,241.28) circle (  1.16);

\path[draw=drawColor,line width= 0.4pt,line join=round,line cap=round,fill=fillColor] (459.79,241.24) circle (  1.16);

\path[draw=drawColor,line width= 0.4pt,line join=round,line cap=round,fill=fillColor] (459.95,241.11) circle (  1.16);

\path[draw=drawColor,line width= 0.4pt,line join=round,line cap=round,fill=fillColor] (460.11,241.07) circle (  1.16);

\path[draw=drawColor,line width= 0.4pt,line join=round,line cap=round,fill=fillColor] (460.27,241.05) circle (  1.16);

\path[draw=drawColor,line width= 0.4pt,line join=round,line cap=round,fill=fillColor] (460.43,241.04) circle (  1.16);

\path[draw=drawColor,line width= 0.4pt,line join=round,line cap=round,fill=fillColor] (460.59,241.03) circle (  1.16);

\path[draw=drawColor,line width= 0.4pt,line join=round,line cap=round,fill=fillColor] (460.74,241.00) circle (  1.16);

\path[draw=drawColor,line width= 0.4pt,line join=round,line cap=round,fill=fillColor] (460.90,240.93) circle (  1.16);

\path[draw=drawColor,line width= 0.4pt,line join=round,line cap=round,fill=fillColor] (461.06,240.79) circle (  1.16);

\path[draw=drawColor,line width= 0.4pt,line join=round,line cap=round,fill=fillColor] (461.22,240.70) circle (  1.16);

\path[draw=drawColor,line width= 0.4pt,line join=round,line cap=round,fill=fillColor] (461.38,240.60) circle (  1.16);

\path[draw=drawColor,line width= 0.4pt,line join=round,line cap=round,fill=fillColor] (461.53,240.52) circle (  1.16);

\path[draw=drawColor,line width= 0.4pt,line join=round,line cap=round,fill=fillColor] (461.69,240.48) circle (  1.16);

\path[draw=drawColor,line width= 0.4pt,line join=round,line cap=round,fill=fillColor] (461.85,240.48) circle (  1.16);

\path[draw=drawColor,line width= 0.4pt,line join=round,line cap=round,fill=fillColor] (462.00,240.46) circle (  1.16);

\path[draw=drawColor,line width= 0.4pt,line join=round,line cap=round,fill=fillColor] (462.16,240.42) circle (  1.16);

\path[draw=drawColor,line width= 0.4pt,line join=round,line cap=round,fill=fillColor] (462.31,240.39) circle (  1.16);

\path[draw=drawColor,line width= 0.4pt,line join=round,line cap=round,fill=fillColor] (462.47,240.38) circle (  1.16);

\path[draw=drawColor,line width= 0.4pt,line join=round,line cap=round,fill=fillColor] (462.62,240.33) circle (  1.16);

\path[draw=drawColor,line width= 0.4pt,line join=round,line cap=round,fill=fillColor] (462.78,240.33) circle (  1.16);

\path[draw=drawColor,line width= 0.4pt,line join=round,line cap=round,fill=fillColor] (462.93,240.30) circle (  1.16);

\path[draw=drawColor,line width= 0.4pt,line join=round,line cap=round,fill=fillColor] (463.09,240.25) circle (  1.16);

\path[draw=drawColor,line width= 0.4pt,line join=round,line cap=round,fill=fillColor] (463.24,240.24) circle (  1.16);

\path[draw=drawColor,line width= 0.4pt,line join=round,line cap=round,fill=fillColor] (463.39,240.06) circle (  1.16);

\path[draw=drawColor,line width= 0.4pt,line join=round,line cap=round,fill=fillColor] (463.55,240.03) circle (  1.16);

\path[draw=drawColor,line width= 0.4pt,line join=round,line cap=round,fill=fillColor] (463.70,240.00) circle (  1.16);

\path[draw=drawColor,line width= 0.4pt,line join=round,line cap=round,fill=fillColor] (463.85,240.00) circle (  1.16);

\path[draw=drawColor,line width= 0.4pt,line join=round,line cap=round,fill=fillColor] (464.00,239.92) circle (  1.16);

\path[draw=drawColor,line width= 0.4pt,line join=round,line cap=round,fill=fillColor] (464.16,239.92) circle (  1.16);

\path[draw=drawColor,line width= 0.4pt,line join=round,line cap=round,fill=fillColor] (464.31,239.84) circle (  1.16);

\path[draw=drawColor,line width= 0.4pt,line join=round,line cap=round,fill=fillColor] (464.46,239.76) circle (  1.16);

\path[draw=drawColor,line width= 0.4pt,line join=round,line cap=round,fill=fillColor] (464.61,239.73) circle (  1.16);

\path[draw=drawColor,line width= 0.4pt,line join=round,line cap=round,fill=fillColor] (464.76,239.67) circle (  1.16);

\path[draw=drawColor,line width= 0.4pt,line join=round,line cap=round,fill=fillColor] (464.91,239.60) circle (  1.16);

\path[draw=drawColor,line width= 0.4pt,line join=round,line cap=round,fill=fillColor] (465.06,239.53) circle (  1.16);

\path[draw=drawColor,line width= 0.4pt,line join=round,line cap=round,fill=fillColor] (465.21,239.51) circle (  1.16);

\path[draw=drawColor,line width= 0.4pt,line join=round,line cap=round,fill=fillColor] (465.36,239.44) circle (  1.16);

\path[draw=drawColor,line width= 0.4pt,line join=round,line cap=round,fill=fillColor] (465.51,239.40) circle (  1.16);

\path[draw=drawColor,line width= 0.4pt,line join=round,line cap=round,fill=fillColor] (465.66,239.25) circle (  1.16);

\path[draw=drawColor,line width= 0.4pt,line join=round,line cap=round,fill=fillColor] (465.81,239.19) circle (  1.16);

\path[draw=drawColor,line width= 0.4pt,line join=round,line cap=round,fill=fillColor] (465.96,239.18) circle (  1.16);

\path[draw=drawColor,line width= 0.4pt,line join=round,line cap=round,fill=fillColor] (466.10,239.14) circle (  1.16);

\path[draw=drawColor,line width= 0.4pt,line join=round,line cap=round,fill=fillColor] (466.25,238.88) circle (  1.16);

\path[draw=drawColor,line width= 0.4pt,line join=round,line cap=round,fill=fillColor] (466.40,238.77) circle (  1.16);

\path[draw=drawColor,line width= 0.4pt,line join=round,line cap=round,fill=fillColor] (466.55,238.61) circle (  1.16);

\path[draw=drawColor,line width= 0.4pt,line join=round,line cap=round,fill=fillColor] (466.70,238.54) circle (  1.16);

\path[draw=drawColor,line width= 0.4pt,line join=round,line cap=round,fill=fillColor] (466.84,238.46) circle (  1.16);

\path[draw=drawColor,line width= 0.4pt,line join=round,line cap=round,fill=fillColor] (466.99,238.46) circle (  1.16);

\path[draw=drawColor,line width= 0.4pt,line join=round,line cap=round,fill=fillColor] (467.14,238.38) circle (  1.16);

\path[draw=drawColor,line width= 0.4pt,line join=round,line cap=round,fill=fillColor] (467.28,238.34) circle (  1.16);

\path[draw=drawColor,line width= 0.4pt,line join=round,line cap=round,fill=fillColor] (467.43,238.33) circle (  1.16);

\path[draw=drawColor,line width= 0.4pt,line join=round,line cap=round,fill=fillColor] (467.57,238.30) circle (  1.16);

\path[draw=drawColor,line width= 0.4pt,line join=round,line cap=round,fill=fillColor] (467.72,238.28) circle (  1.16);

\path[draw=drawColor,line width= 0.4pt,line join=round,line cap=round,fill=fillColor] (467.86,238.11) circle (  1.16);

\path[draw=drawColor,line width= 0.4pt,line join=round,line cap=round,fill=fillColor] (468.01,238.09) circle (  1.16);

\path[draw=drawColor,line width= 0.4pt,line join=round,line cap=round,fill=fillColor] (468.15,238.09) circle (  1.16);

\path[draw=drawColor,line width= 0.4pt,line join=round,line cap=round,fill=fillColor] (468.30,238.03) circle (  1.16);

\path[draw=drawColor,line width= 0.4pt,line join=round,line cap=round,fill=fillColor] (468.44,237.96) circle (  1.16);

\path[draw=drawColor,line width= 0.4pt,line join=round,line cap=round,fill=fillColor] (468.58,237.89) circle (  1.16);

\path[draw=drawColor,line width= 0.4pt,line join=round,line cap=round,fill=fillColor] (468.73,237.81) circle (  1.16);

\path[draw=drawColor,line width= 0.4pt,line join=round,line cap=round,fill=fillColor] (468.87,237.79) circle (  1.16);

\path[draw=drawColor,line width= 0.4pt,line join=round,line cap=round,fill=fillColor] (469.01,237.71) circle (  1.16);

\path[draw=drawColor,line width= 0.4pt,line join=round,line cap=round,fill=fillColor] (469.16,237.37) circle (  1.16);

\path[draw=drawColor,line width= 0.4pt,line join=round,line cap=round,fill=fillColor] (469.30,237.29) circle (  1.16);

\path[draw=drawColor,line width= 0.4pt,line join=round,line cap=round,fill=fillColor] (469.44,237.16) circle (  1.16);

\path[draw=drawColor,line width= 0.4pt,line join=round,line cap=round,fill=fillColor] (469.58,237.15) circle (  1.16);

\path[draw=drawColor,line width= 0.4pt,line join=round,line cap=round,fill=fillColor] (469.73,237.14) circle (  1.16);

\path[draw=drawColor,line width= 0.4pt,line join=round,line cap=round,fill=fillColor] (469.87,236.97) circle (  1.16);

\path[draw=drawColor,line width= 0.4pt,line join=round,line cap=round,fill=fillColor] (470.01,236.89) circle (  1.16);

\path[draw=drawColor,line width= 0.4pt,line join=round,line cap=round,fill=fillColor] (470.15,236.81) circle (  1.16);

\path[draw=drawColor,line width= 0.4pt,line join=round,line cap=round,fill=fillColor] (470.29,236.78) circle (  1.16);

\path[draw=drawColor,line width= 0.4pt,line join=round,line cap=round,fill=fillColor] (470.43,236.74) circle (  1.16);

\path[draw=drawColor,line width= 0.4pt,line join=round,line cap=round,fill=fillColor] (470.57,236.73) circle (  1.16);

\path[draw=drawColor,line width= 0.4pt,line join=round,line cap=round,fill=fillColor] (470.71,236.71) circle (  1.16);

\path[draw=drawColor,line width= 0.4pt,line join=round,line cap=round,fill=fillColor] (470.85,236.66) circle (  1.16);

\path[draw=drawColor,line width= 0.4pt,line join=round,line cap=round,fill=fillColor] (470.99,236.41) circle (  1.16);

\path[draw=drawColor,line width= 0.4pt,line join=round,line cap=round,fill=fillColor] (471.13,236.32) circle (  1.16);

\path[draw=drawColor,line width= 0.4pt,line join=round,line cap=round,fill=fillColor] (471.27,236.27) circle (  1.16);

\path[draw=drawColor,line width= 0.4pt,line join=round,line cap=round,fill=fillColor] (471.41,236.24) circle (  1.16);

\path[draw=drawColor,line width= 0.4pt,line join=round,line cap=round,fill=fillColor] (471.55,236.22) circle (  1.16);

\path[draw=drawColor,line width= 0.4pt,line join=round,line cap=round,fill=fillColor] (471.69,236.01) circle (  1.16);

\path[draw=drawColor,line width= 0.4pt,line join=round,line cap=round,fill=fillColor] (471.82,235.93) circle (  1.16);

\path[draw=drawColor,line width= 0.4pt,line join=round,line cap=round,fill=fillColor] (471.96,235.92) circle (  1.16);

\path[draw=drawColor,line width= 0.4pt,line join=round,line cap=round,fill=fillColor] (472.10,235.82) circle (  1.16);

\path[draw=drawColor,line width= 0.4pt,line join=round,line cap=round,fill=fillColor] (472.24,235.82) circle (  1.16);

\path[draw=drawColor,line width= 0.4pt,line join=round,line cap=round,fill=fillColor] (472.37,235.76) circle (  1.16);

\path[draw=drawColor,line width= 0.4pt,line join=round,line cap=round,fill=fillColor] (472.51,235.64) circle (  1.16);

\path[draw=drawColor,line width= 0.4pt,line join=round,line cap=round,fill=fillColor] (472.65,235.51) circle (  1.16);

\path[draw=drawColor,line width= 0.4pt,line join=round,line cap=round,fill=fillColor] (472.78,235.47) circle (  1.16);

\path[draw=drawColor,line width= 0.4pt,line join=round,line cap=round,fill=fillColor] (472.92,235.39) circle (  1.16);

\path[draw=drawColor,line width= 0.4pt,line join=round,line cap=round,fill=fillColor] (473.06,235.17) circle (  1.16);

\path[draw=drawColor,line width= 0.4pt,line join=round,line cap=round,fill=fillColor] (473.19,234.95) circle (  1.16);

\path[draw=drawColor,line width= 0.4pt,line join=round,line cap=round,fill=fillColor] (473.33,234.73) circle (  1.16);

\path[draw=drawColor,line width= 0.4pt,line join=round,line cap=round,fill=fillColor] (473.46,234.55) circle (  1.16);

\path[draw=drawColor,line width= 0.4pt,line join=round,line cap=round,fill=fillColor] (473.60,234.52) circle (  1.16);

\path[draw=drawColor,line width= 0.4pt,line join=round,line cap=round,fill=fillColor] (473.74,234.49) circle (  1.16);

\path[draw=drawColor,line width= 0.4pt,line join=round,line cap=round,fill=fillColor] (473.87,234.33) circle (  1.16);

\path[draw=drawColor,line width= 0.4pt,line join=round,line cap=round,fill=fillColor] (474.00,233.90) circle (  1.16);

\path[draw=drawColor,line width= 0.4pt,line join=round,line cap=round,fill=fillColor] (474.14,233.85) circle (  1.16);

\path[draw=drawColor,line width= 0.4pt,line join=round,line cap=round,fill=fillColor] (474.27,233.80) circle (  1.16);

\path[draw=drawColor,line width= 0.4pt,line join=round,line cap=round,fill=fillColor] (474.41,233.73) circle (  1.16);

\path[draw=drawColor,line width= 0.4pt,line join=round,line cap=round,fill=fillColor] (474.54,233.52) circle (  1.16);

\path[draw=drawColor,line width= 0.4pt,line join=round,line cap=round,fill=fillColor] (474.68,233.37) circle (  1.16);

\path[draw=drawColor,line width= 0.4pt,line join=round,line cap=round,fill=fillColor] (474.81,232.92) circle (  1.16);

\path[draw=drawColor,line width= 0.4pt,line join=round,line cap=round,fill=fillColor] (474.94,232.86) circle (  1.16);

\path[draw=drawColor,line width= 0.4pt,line join=round,line cap=round,fill=fillColor] (475.07,232.77) circle (  1.16);

\path[draw=drawColor,line width= 0.4pt,line join=round,line cap=round,fill=fillColor] (475.21,232.54) circle (  1.16);

\path[draw=drawColor,line width= 0.4pt,line join=round,line cap=round,fill=fillColor] (475.34,232.39) circle (  1.16);

\path[draw=drawColor,line width= 0.4pt,line join=round,line cap=round,fill=fillColor] (475.47,231.71) circle (  1.16);

\path[draw=drawColor,line width= 0.4pt,line join=round,line cap=round,fill=fillColor] (475.60,231.69) circle (  1.16);

\path[draw=drawColor,line width= 0.4pt,line join=round,line cap=round,fill=fillColor] (475.74,231.47) circle (  1.16);

\path[draw=drawColor,line width= 0.4pt,line join=round,line cap=round,fill=fillColor] (475.87,231.40) circle (  1.16);

\path[draw=drawColor,line width= 0.4pt,line join=round,line cap=round,fill=fillColor] (476.00,231.13) circle (  1.16);

\path[draw=drawColor,line width= 0.4pt,line join=round,line cap=round,fill=fillColor] (476.13,230.55) circle (  1.16);

\path[draw=drawColor,line width= 0.4pt,line join=round,line cap=round,fill=fillColor] (476.26,230.33) circle (  1.16);

\path[draw=drawColor,line width= 0.4pt,line join=round,line cap=round,fill=fillColor] (476.39,229.67) circle (  1.16);

\path[draw=drawColor,line width= 0.4pt,line join=round,line cap=round,fill=fillColor] (476.52,228.98) circle (  1.16);

\path[draw=drawColor,line width= 0.4pt,line join=round,line cap=round,fill=fillColor] (476.65,226.96) circle (  1.16);

\path[draw=drawColor,line width= 0.4pt,line join=round,line cap=round,fill=fillColor] (476.78,226.20) circle (  1.16);

\path[draw=drawColor,line width= 0.4pt,line join=round,line cap=round,fill=fillColor] (476.91,225.02) circle (  1.16);

\path[draw=drawColor,line width= 0.4pt,line join=round,line cap=round,fill=fillColor] (477.04,224.85) circle (  1.16);

\path[draw=drawColor,line width= 0.4pt,line join=round,line cap=round,fill=fillColor] (477.17,223.67) circle (  1.16);

\path[draw=drawColor,line width= 0.4pt,line join=round,line cap=round,fill=fillColor] (477.30,222.79) circle (  1.16);

\path[draw=drawColor,line width= 0.4pt,line join=round,line cap=round,fill=fillColor] (477.43,209.81) circle (  1.16);

\path[draw=drawColor,line width= 0.4pt,line join=round,line cap=round,fill=fillColor] (477.56,209.81) circle (  1.16);

\path[draw=drawColor,line width= 0.4pt,line join=round,line cap=round,fill=fillColor] (477.69,209.81) circle (  1.16);

\path[draw=drawColor,line width= 0.4pt,line join=round,line cap=round,fill=fillColor] (477.82,209.81) circle (  1.16);

\path[draw=drawColor,line width= 0.4pt,line join=round,line cap=round,fill=fillColor] (477.95,209.81) circle (  1.16);

\path[draw=drawColor,line width= 0.4pt,line join=round,line cap=round,fill=fillColor] (478.08,209.81) circle (  1.16);

\path[draw=drawColor,line width= 0.4pt,line join=round,line cap=round,fill=fillColor] (478.21,209.81) circle (  1.16);
\definecolor[named]{drawColor}{rgb}{1.00,0.50,0.00}
\definecolor[named]{fillColor}{rgb}{1.00,0.50,0.00}

\path[draw=drawColor,line width= 0.4pt,line join=round,line cap=round,fill=fillColor] (327.82,284.56) circle (  1.16);

\path[draw=drawColor,line width= 0.4pt,line join=round,line cap=round,fill=fillColor] (333.69,277.83) circle (  1.16);

\path[draw=drawColor,line width= 0.4pt,line join=round,line cap=round,fill=fillColor] (337.81,276.72) circle (  1.16);

\path[draw=drawColor,line width= 0.4pt,line join=round,line cap=round,fill=fillColor] (341.08,276.02) circle (  1.16);

\path[draw=drawColor,line width= 0.4pt,line join=round,line cap=round,fill=fillColor] (343.85,273.16) circle (  1.16);

\path[draw=drawColor,line width= 0.4pt,line join=round,line cap=round,fill=fillColor] (346.27,272.19) circle (  1.16);

\path[draw=drawColor,line width= 0.4pt,line join=round,line cap=round,fill=fillColor] (348.43,270.25) circle (  1.16);

\path[draw=drawColor,line width= 0.4pt,line join=round,line cap=round,fill=fillColor] (350.39,269.35) circle (  1.16);

\path[draw=drawColor,line width= 0.4pt,line join=round,line cap=round,fill=fillColor] (352.20,268.51) circle (  1.16);

\path[draw=drawColor,line width= 0.4pt,line join=round,line cap=round,fill=fillColor] (353.88,268.05) circle (  1.16);

\path[draw=drawColor,line width= 0.4pt,line join=round,line cap=round,fill=fillColor] (355.45,267.77) circle (  1.16);

\path[draw=drawColor,line width= 0.4pt,line join=round,line cap=round,fill=fillColor] (356.93,267.03) circle (  1.16);

\path[draw=drawColor,line width= 0.4pt,line join=round,line cap=round,fill=fillColor] (358.32,266.89) circle (  1.16);

\path[draw=drawColor,line width= 0.4pt,line join=round,line cap=round,fill=fillColor] (359.65,266.68) circle (  1.16);

\path[draw=drawColor,line width= 0.4pt,line join=round,line cap=round,fill=fillColor] (360.92,266.66) circle (  1.16);

\path[draw=drawColor,line width= 0.4pt,line join=round,line cap=round,fill=fillColor] (362.13,266.51) circle (  1.16);

\path[draw=drawColor,line width= 0.4pt,line join=round,line cap=round,fill=fillColor] (363.29,266.28) circle (  1.16);

\path[draw=drawColor,line width= 0.4pt,line join=round,line cap=round,fill=fillColor] (364.40,265.69) circle (  1.16);

\path[draw=drawColor,line width= 0.4pt,line join=round,line cap=round,fill=fillColor] (365.48,265.01) circle (  1.16);

\path[draw=drawColor,line width= 0.4pt,line join=round,line cap=round,fill=fillColor] (366.52,264.98) circle (  1.16);

\path[draw=drawColor,line width= 0.4pt,line join=round,line cap=round,fill=fillColor] (367.52,264.73) circle (  1.16);

\path[draw=drawColor,line width= 0.4pt,line join=round,line cap=round,fill=fillColor] (368.49,264.40) circle (  1.16);

\path[draw=drawColor,line width= 0.4pt,line join=round,line cap=round,fill=fillColor] (369.44,264.17) circle (  1.16);

\path[draw=drawColor,line width= 0.4pt,line join=round,line cap=round,fill=fillColor] (370.36,263.99) circle (  1.16);

\path[draw=drawColor,line width= 0.4pt,line join=round,line cap=round,fill=fillColor] (371.25,263.66) circle (  1.16);

\path[draw=drawColor,line width= 0.4pt,line join=round,line cap=round,fill=fillColor] (372.12,263.63) circle (  1.16);

\path[draw=drawColor,line width= 0.4pt,line join=round,line cap=round,fill=fillColor] (372.96,263.55) circle (  1.16);

\path[draw=drawColor,line width= 0.4pt,line join=round,line cap=round,fill=fillColor] (373.79,263.39) circle (  1.16);

\path[draw=drawColor,line width= 0.4pt,line join=round,line cap=round,fill=fillColor] (374.59,263.24) circle (  1.16);

\path[draw=drawColor,line width= 0.4pt,line join=round,line cap=round,fill=fillColor] (375.38,263.19) circle (  1.16);

\path[draw=drawColor,line width= 0.4pt,line join=round,line cap=round,fill=fillColor] (376.15,263.09) circle (  1.16);

\path[draw=drawColor,line width= 0.4pt,line join=round,line cap=round,fill=fillColor] (376.91,262.85) circle (  1.16);

\path[draw=drawColor,line width= 0.4pt,line join=round,line cap=round,fill=fillColor] (377.65,262.72) circle (  1.16);

\path[draw=drawColor,line width= 0.4pt,line join=round,line cap=round,fill=fillColor] (378.37,262.68) circle (  1.16);

\path[draw=drawColor,line width= 0.4pt,line join=round,line cap=round,fill=fillColor] (379.08,262.46) circle (  1.16);

\path[draw=drawColor,line width= 0.4pt,line join=round,line cap=round,fill=fillColor] (379.78,262.34) circle (  1.16);

\path[draw=drawColor,line width= 0.4pt,line join=round,line cap=round,fill=fillColor] (380.46,262.26) circle (  1.16);

\path[draw=drawColor,line width= 0.4pt,line join=round,line cap=round,fill=fillColor] (381.13,262.22) circle (  1.16);

\path[draw=drawColor,line width= 0.4pt,line join=round,line cap=round,fill=fillColor] (381.79,262.01) circle (  1.16);

\path[draw=drawColor,line width= 0.4pt,line join=round,line cap=round,fill=fillColor] (382.44,261.74) circle (  1.16);

\path[draw=drawColor,line width= 0.4pt,line join=round,line cap=round,fill=fillColor] (383.08,261.53) circle (  1.16);

\path[draw=drawColor,line width= 0.4pt,line join=round,line cap=round,fill=fillColor] (383.71,261.49) circle (  1.16);

\path[draw=drawColor,line width= 0.4pt,line join=round,line cap=round,fill=fillColor] (384.32,261.13) circle (  1.16);

\path[draw=drawColor,line width= 0.4pt,line join=round,line cap=round,fill=fillColor] (384.93,261.07) circle (  1.16);

\path[draw=drawColor,line width= 0.4pt,line join=round,line cap=round,fill=fillColor] (385.53,261.01) circle (  1.16);

\path[draw=drawColor,line width= 0.4pt,line join=round,line cap=round,fill=fillColor] (386.12,260.94) circle (  1.16);

\path[draw=drawColor,line width= 0.4pt,line join=round,line cap=round,fill=fillColor] (386.70,260.90) circle (  1.16);

\path[draw=drawColor,line width= 0.4pt,line join=round,line cap=round,fill=fillColor] (387.28,260.86) circle (  1.16);

\path[draw=drawColor,line width= 0.4pt,line join=round,line cap=round,fill=fillColor] (387.84,260.79) circle (  1.16);

\path[draw=drawColor,line width= 0.4pt,line join=round,line cap=round,fill=fillColor] (388.40,260.72) circle (  1.16);

\path[draw=drawColor,line width= 0.4pt,line join=round,line cap=round,fill=fillColor] (388.95,260.68) circle (  1.16);

\path[draw=drawColor,line width= 0.4pt,line join=round,line cap=round,fill=fillColor] (389.50,260.49) circle (  1.16);

\path[draw=drawColor,line width= 0.4pt,line join=round,line cap=round,fill=fillColor] (390.03,260.30) circle (  1.16);

\path[draw=drawColor,line width= 0.4pt,line join=round,line cap=round,fill=fillColor] (390.56,260.01) circle (  1.16);

\path[draw=drawColor,line width= 0.4pt,line join=round,line cap=round,fill=fillColor] (391.08,259.88) circle (  1.16);

\path[draw=drawColor,line width= 0.4pt,line join=round,line cap=round,fill=fillColor] (391.60,259.85) circle (  1.16);

\path[draw=drawColor,line width= 0.4pt,line join=round,line cap=round,fill=fillColor] (392.11,259.84) circle (  1.16);

\path[draw=drawColor,line width= 0.4pt,line join=round,line cap=round,fill=fillColor] (392.62,259.80) circle (  1.16);

\path[draw=drawColor,line width= 0.4pt,line join=round,line cap=round,fill=fillColor] (393.12,259.67) circle (  1.16);

\path[draw=drawColor,line width= 0.4pt,line join=round,line cap=round,fill=fillColor] (393.61,259.62) circle (  1.16);

\path[draw=drawColor,line width= 0.4pt,line join=round,line cap=round,fill=fillColor] (394.10,259.57) circle (  1.16);

\path[draw=drawColor,line width= 0.4pt,line join=round,line cap=round,fill=fillColor] (394.58,259.57) circle (  1.16);

\path[draw=drawColor,line width= 0.4pt,line join=round,line cap=round,fill=fillColor] (395.06,259.40) circle (  1.16);

\path[draw=drawColor,line width= 0.4pt,line join=round,line cap=round,fill=fillColor] (395.53,259.38) circle (  1.16);

\path[draw=drawColor,line width= 0.4pt,line join=round,line cap=round,fill=fillColor] (396.00,259.02) circle (  1.16);

\path[draw=drawColor,line width= 0.4pt,line join=round,line cap=round,fill=fillColor] (396.46,259.01) circle (  1.16);

\path[draw=drawColor,line width= 0.4pt,line join=round,line cap=round,fill=fillColor] (396.92,259.01) circle (  1.16);

\path[draw=drawColor,line width= 0.4pt,line join=round,line cap=round,fill=fillColor] (397.37,258.75) circle (  1.16);

\path[draw=drawColor,line width= 0.4pt,line join=round,line cap=round,fill=fillColor] (397.82,258.71) circle (  1.16);

\path[draw=drawColor,line width= 0.4pt,line join=round,line cap=round,fill=fillColor] (398.27,258.29) circle (  1.16);

\path[draw=drawColor,line width= 0.4pt,line join=round,line cap=round,fill=fillColor] (398.71,258.27) circle (  1.16);

\path[draw=drawColor,line width= 0.4pt,line join=round,line cap=round,fill=fillColor] (399.15,258.24) circle (  1.16);

\path[draw=drawColor,line width= 0.4pt,line join=round,line cap=round,fill=fillColor] (399.58,258.22) circle (  1.16);

\path[draw=drawColor,line width= 0.4pt,line join=round,line cap=round,fill=fillColor] (400.01,258.22) circle (  1.16);

\path[draw=drawColor,line width= 0.4pt,line join=round,line cap=round,fill=fillColor] (400.43,258.08) circle (  1.16);

\path[draw=drawColor,line width= 0.4pt,line join=round,line cap=round,fill=fillColor] (400.85,258.03) circle (  1.16);

\path[draw=drawColor,line width= 0.4pt,line join=round,line cap=round,fill=fillColor] (401.27,257.91) circle (  1.16);

\path[draw=drawColor,line width= 0.4pt,line join=round,line cap=round,fill=fillColor] (401.69,257.90) circle (  1.16);

\path[draw=drawColor,line width= 0.4pt,line join=round,line cap=round,fill=fillColor] (402.10,257.86) circle (  1.16);

\path[draw=drawColor,line width= 0.4pt,line join=round,line cap=round,fill=fillColor] (402.50,257.86) circle (  1.16);

\path[draw=drawColor,line width= 0.4pt,line join=round,line cap=round,fill=fillColor] (402.91,257.81) circle (  1.16);

\path[draw=drawColor,line width= 0.4pt,line join=round,line cap=round,fill=fillColor] (403.31,257.78) circle (  1.16);

\path[draw=drawColor,line width= 0.4pt,line join=round,line cap=round,fill=fillColor] (403.70,257.54) circle (  1.16);

\path[draw=drawColor,line width= 0.4pt,line join=round,line cap=round,fill=fillColor] (404.10,257.36) circle (  1.16);

\path[draw=drawColor,line width= 0.4pt,line join=round,line cap=round,fill=fillColor] (404.49,257.32) circle (  1.16);

\path[draw=drawColor,line width= 0.4pt,line join=round,line cap=round,fill=fillColor] (404.88,257.23) circle (  1.16);

\path[draw=drawColor,line width= 0.4pt,line join=round,line cap=round,fill=fillColor] (405.26,257.18) circle (  1.16);

\path[draw=drawColor,line width= 0.4pt,line join=round,line cap=round,fill=fillColor] (405.64,257.12) circle (  1.16);

\path[draw=drawColor,line width= 0.4pt,line join=round,line cap=round,fill=fillColor] (406.02,257.07) circle (  1.16);

\path[draw=drawColor,line width= 0.4pt,line join=round,line cap=round,fill=fillColor] (406.40,257.04) circle (  1.16);

\path[draw=drawColor,line width= 0.4pt,line join=round,line cap=round,fill=fillColor] (406.77,256.97) circle (  1.16);

\path[draw=drawColor,line width= 0.4pt,line join=round,line cap=round,fill=fillColor] (407.14,256.92) circle (  1.16);

\path[draw=drawColor,line width= 0.4pt,line join=round,line cap=round,fill=fillColor] (407.51,256.81) circle (  1.16);

\path[draw=drawColor,line width= 0.4pt,line join=round,line cap=round,fill=fillColor] (407.87,256.62) circle (  1.16);

\path[draw=drawColor,line width= 0.4pt,line join=round,line cap=round,fill=fillColor] (408.24,256.59) circle (  1.16);

\path[draw=drawColor,line width= 0.4pt,line join=round,line cap=round,fill=fillColor] (408.60,256.53) circle (  1.16);

\path[draw=drawColor,line width= 0.4pt,line join=round,line cap=round,fill=fillColor] (408.95,256.22) circle (  1.16);

\path[draw=drawColor,line width= 0.4pt,line join=round,line cap=round,fill=fillColor] (409.31,256.11) circle (  1.16);

\path[draw=drawColor,line width= 0.4pt,line join=round,line cap=round,fill=fillColor] (409.66,256.06) circle (  1.16);

\path[draw=drawColor,line width= 0.4pt,line join=round,line cap=round,fill=fillColor] (410.01,256.02) circle (  1.16);

\path[draw=drawColor,line width= 0.4pt,line join=round,line cap=round,fill=fillColor] (410.36,255.86) circle (  1.16);

\path[draw=drawColor,line width= 0.4pt,line join=round,line cap=round,fill=fillColor] (410.71,255.79) circle (  1.16);

\path[draw=drawColor,line width= 0.4pt,line join=round,line cap=round,fill=fillColor] (411.05,255.56) circle (  1.16);

\path[draw=drawColor,line width= 0.4pt,line join=round,line cap=round,fill=fillColor] (411.39,255.51) circle (  1.16);

\path[draw=drawColor,line width= 0.4pt,line join=round,line cap=round,fill=fillColor] (411.73,255.42) circle (  1.16);

\path[draw=drawColor,line width= 0.4pt,line join=round,line cap=round,fill=fillColor] (412.07,255.34) circle (  1.16);

\path[draw=drawColor,line width= 0.4pt,line join=round,line cap=round,fill=fillColor] (412.40,255.20) circle (  1.16);

\path[draw=drawColor,line width= 0.4pt,line join=round,line cap=round,fill=fillColor] (412.73,255.14) circle (  1.16);

\path[draw=drawColor,line width= 0.4pt,line join=round,line cap=round,fill=fillColor] (413.07,255.12) circle (  1.16);

\path[draw=drawColor,line width= 0.4pt,line join=round,line cap=round,fill=fillColor] (413.39,255.09) circle (  1.16);

\path[draw=drawColor,line width= 0.4pt,line join=round,line cap=round,fill=fillColor] (413.72,255.04) circle (  1.16);

\path[draw=drawColor,line width= 0.4pt,line join=round,line cap=round,fill=fillColor] (414.05,254.96) circle (  1.16);

\path[draw=drawColor,line width= 0.4pt,line join=round,line cap=round,fill=fillColor] (414.37,254.88) circle (  1.16);

\path[draw=drawColor,line width= 0.4pt,line join=round,line cap=round,fill=fillColor] (414.69,254.78) circle (  1.16);

\path[draw=drawColor,line width= 0.4pt,line join=round,line cap=round,fill=fillColor] (415.01,254.67) circle (  1.16);

\path[draw=drawColor,line width= 0.4pt,line join=round,line cap=round,fill=fillColor] (415.33,254.35) circle (  1.16);

\path[draw=drawColor,line width= 0.4pt,line join=round,line cap=round,fill=fillColor] (415.64,254.35) circle (  1.16);

\path[draw=drawColor,line width= 0.4pt,line join=round,line cap=round,fill=fillColor] (415.95,254.19) circle (  1.16);

\path[draw=drawColor,line width= 0.4pt,line join=round,line cap=round,fill=fillColor] (416.27,254.17) circle (  1.16);

\path[draw=drawColor,line width= 0.4pt,line join=round,line cap=round,fill=fillColor] (416.58,254.17) circle (  1.16);

\path[draw=drawColor,line width= 0.4pt,line join=round,line cap=round,fill=fillColor] (416.88,254.04) circle (  1.16);

\path[draw=drawColor,line width= 0.4pt,line join=round,line cap=round,fill=fillColor] (417.19,253.98) circle (  1.16);

\path[draw=drawColor,line width= 0.4pt,line join=round,line cap=round,fill=fillColor] (417.50,253.89) circle (  1.16);

\path[draw=drawColor,line width= 0.4pt,line join=round,line cap=round,fill=fillColor] (417.80,253.83) circle (  1.16);

\path[draw=drawColor,line width= 0.4pt,line join=round,line cap=round,fill=fillColor] (418.10,253.80) circle (  1.16);

\path[draw=drawColor,line width= 0.4pt,line join=round,line cap=round,fill=fillColor] (418.40,253.79) circle (  1.16);

\path[draw=drawColor,line width= 0.4pt,line join=round,line cap=round,fill=fillColor] (418.70,253.79) circle (  1.16);

\path[draw=drawColor,line width= 0.4pt,line join=round,line cap=round,fill=fillColor] (419.00,253.68) circle (  1.16);

\path[draw=drawColor,line width= 0.4pt,line join=round,line cap=round,fill=fillColor] (419.29,253.63) circle (  1.16);

\path[draw=drawColor,line width= 0.4pt,line join=round,line cap=round,fill=fillColor] (419.59,253.63) circle (  1.16);

\path[draw=drawColor,line width= 0.4pt,line join=round,line cap=round,fill=fillColor] (419.88,253.62) circle (  1.16);

\path[draw=drawColor,line width= 0.4pt,line join=round,line cap=round,fill=fillColor] (420.17,253.60) circle (  1.16);

\path[draw=drawColor,line width= 0.4pt,line join=round,line cap=round,fill=fillColor] (420.46,253.44) circle (  1.16);

\path[draw=drawColor,line width= 0.4pt,line join=round,line cap=round,fill=fillColor] (420.75,253.32) circle (  1.16);

\path[draw=drawColor,line width= 0.4pt,line join=round,line cap=round,fill=fillColor] (421.03,253.31) circle (  1.16);

\path[draw=drawColor,line width= 0.4pt,line join=round,line cap=round,fill=fillColor] (421.32,253.28) circle (  1.16);

\path[draw=drawColor,line width= 0.4pt,line join=round,line cap=round,fill=fillColor] (421.60,253.11) circle (  1.16);

\path[draw=drawColor,line width= 0.4pt,line join=round,line cap=round,fill=fillColor] (421.89,253.10) circle (  1.16);

\path[draw=drawColor,line width= 0.4pt,line join=round,line cap=round,fill=fillColor] (422.17,253.07) circle (  1.16);

\path[draw=drawColor,line width= 0.4pt,line join=round,line cap=round,fill=fillColor] (422.45,253.02) circle (  1.16);

\path[draw=drawColor,line width= 0.4pt,line join=round,line cap=round,fill=fillColor] (422.72,252.93) circle (  1.16);

\path[draw=drawColor,line width= 0.4pt,line join=round,line cap=round,fill=fillColor] (423.00,252.83) circle (  1.16);

\path[draw=drawColor,line width= 0.4pt,line join=round,line cap=round,fill=fillColor] (423.28,252.82) circle (  1.16);

\path[draw=drawColor,line width= 0.4pt,line join=round,line cap=round,fill=fillColor] (423.55,252.82) circle (  1.16);

\path[draw=drawColor,line width= 0.4pt,line join=round,line cap=round,fill=fillColor] (423.82,252.70) circle (  1.16);

\path[draw=drawColor,line width= 0.4pt,line join=round,line cap=round,fill=fillColor] (424.10,252.63) circle (  1.16);

\path[draw=drawColor,line width= 0.4pt,line join=round,line cap=round,fill=fillColor] (424.37,252.45) circle (  1.16);

\path[draw=drawColor,line width= 0.4pt,line join=round,line cap=round,fill=fillColor] (424.64,252.43) circle (  1.16);

\path[draw=drawColor,line width= 0.4pt,line join=round,line cap=round,fill=fillColor] (424.91,252.38) circle (  1.16);

\path[draw=drawColor,line width= 0.4pt,line join=round,line cap=round,fill=fillColor] (425.17,252.29) circle (  1.16);

\path[draw=drawColor,line width= 0.4pt,line join=round,line cap=round,fill=fillColor] (425.44,252.19) circle (  1.16);

\path[draw=drawColor,line width= 0.4pt,line join=round,line cap=round,fill=fillColor] (425.70,252.03) circle (  1.16);

\path[draw=drawColor,line width= 0.4pt,line join=round,line cap=round,fill=fillColor] (425.97,251.98) circle (  1.16);

\path[draw=drawColor,line width= 0.4pt,line join=round,line cap=round,fill=fillColor] (426.23,251.93) circle (  1.16);

\path[draw=drawColor,line width= 0.4pt,line join=round,line cap=round,fill=fillColor] (426.49,251.92) circle (  1.16);

\path[draw=drawColor,line width= 0.4pt,line join=round,line cap=round,fill=fillColor] (426.75,251.88) circle (  1.16);

\path[draw=drawColor,line width= 0.4pt,line join=round,line cap=round,fill=fillColor] (427.01,251.68) circle (  1.16);

\path[draw=drawColor,line width= 0.4pt,line join=round,line cap=round,fill=fillColor] (427.27,251.65) circle (  1.16);

\path[draw=drawColor,line width= 0.4pt,line join=round,line cap=round,fill=fillColor] (427.52,251.59) circle (  1.16);

\path[draw=drawColor,line width= 0.4pt,line join=round,line cap=round,fill=fillColor] (427.78,251.36) circle (  1.16);

\path[draw=drawColor,line width= 0.4pt,line join=round,line cap=round,fill=fillColor] (428.03,251.36) circle (  1.16);

\path[draw=drawColor,line width= 0.4pt,line join=round,line cap=round,fill=fillColor] (428.29,251.22) circle (  1.16);

\path[draw=drawColor,line width= 0.4pt,line join=round,line cap=round,fill=fillColor] (428.54,251.09) circle (  1.16);

\path[draw=drawColor,line width= 0.4pt,line join=round,line cap=round,fill=fillColor] (428.79,251.08) circle (  1.16);

\path[draw=drawColor,line width= 0.4pt,line join=round,line cap=round,fill=fillColor] (429.04,251.05) circle (  1.16);

\path[draw=drawColor,line width= 0.4pt,line join=round,line cap=round,fill=fillColor] (429.29,251.01) circle (  1.16);

\path[draw=drawColor,line width= 0.4pt,line join=round,line cap=round,fill=fillColor] (429.54,250.95) circle (  1.16);

\path[draw=drawColor,line width= 0.4pt,line join=round,line cap=round,fill=fillColor] (429.79,250.90) circle (  1.16);

\path[draw=drawColor,line width= 0.4pt,line join=round,line cap=round,fill=fillColor] (430.04,250.79) circle (  1.16);

\path[draw=drawColor,line width= 0.4pt,line join=round,line cap=round,fill=fillColor] (430.28,250.79) circle (  1.16);

\path[draw=drawColor,line width= 0.4pt,line join=round,line cap=round,fill=fillColor] (430.53,250.69) circle (  1.16);

\path[draw=drawColor,line width= 0.4pt,line join=round,line cap=round,fill=fillColor] (430.77,250.64) circle (  1.16);

\path[draw=drawColor,line width= 0.4pt,line join=round,line cap=round,fill=fillColor] (431.01,250.54) circle (  1.16);

\path[draw=drawColor,line width= 0.4pt,line join=round,line cap=round,fill=fillColor] (431.25,250.50) circle (  1.16);

\path[draw=drawColor,line width= 0.4pt,line join=round,line cap=round,fill=fillColor] (431.50,250.45) circle (  1.16);

\path[draw=drawColor,line width= 0.4pt,line join=round,line cap=round,fill=fillColor] (431.74,250.43) circle (  1.16);

\path[draw=drawColor,line width= 0.4pt,line join=round,line cap=round,fill=fillColor] (431.97,250.41) circle (  1.16);

\path[draw=drawColor,line width= 0.4pt,line join=round,line cap=round,fill=fillColor] (432.21,250.37) circle (  1.16);

\path[draw=drawColor,line width= 0.4pt,line join=round,line cap=round,fill=fillColor] (432.45,250.36) circle (  1.16);

\path[draw=drawColor,line width= 0.4pt,line join=round,line cap=round,fill=fillColor] (432.69,250.26) circle (  1.16);

\path[draw=drawColor,line width= 0.4pt,line join=round,line cap=round,fill=fillColor] (432.92,250.23) circle (  1.16);

\path[draw=drawColor,line width= 0.4pt,line join=round,line cap=round,fill=fillColor] (433.16,250.21) circle (  1.16);

\path[draw=drawColor,line width= 0.4pt,line join=round,line cap=round,fill=fillColor] (433.39,250.17) circle (  1.16);

\path[draw=drawColor,line width= 0.4pt,line join=round,line cap=round,fill=fillColor] (433.62,250.07) circle (  1.16);

\path[draw=drawColor,line width= 0.4pt,line join=round,line cap=round,fill=fillColor] (433.86,250.05) circle (  1.16);

\path[draw=drawColor,line width= 0.4pt,line join=round,line cap=round,fill=fillColor] (434.09,250.04) circle (  1.16);

\path[draw=drawColor,line width= 0.4pt,line join=round,line cap=round,fill=fillColor] (434.32,249.89) circle (  1.16);

\path[draw=drawColor,line width= 0.4pt,line join=round,line cap=round,fill=fillColor] (434.55,249.88) circle (  1.16);

\path[draw=drawColor,line width= 0.4pt,line join=round,line cap=round,fill=fillColor] (434.78,249.88) circle (  1.16);

\path[draw=drawColor,line width= 0.4pt,line join=round,line cap=round,fill=fillColor] (435.00,249.80) circle (  1.16);

\path[draw=drawColor,line width= 0.4pt,line join=round,line cap=round,fill=fillColor] (435.23,249.80) circle (  1.16);

\path[draw=drawColor,line width= 0.4pt,line join=round,line cap=round,fill=fillColor] (435.46,249.78) circle (  1.16);

\path[draw=drawColor,line width= 0.4pt,line join=round,line cap=round,fill=fillColor] (435.68,249.76) circle (  1.16);

\path[draw=drawColor,line width= 0.4pt,line join=round,line cap=round,fill=fillColor] (435.91,249.75) circle (  1.16);

\path[draw=drawColor,line width= 0.4pt,line join=round,line cap=round,fill=fillColor] (436.13,249.74) circle (  1.16);

\path[draw=drawColor,line width= 0.4pt,line join=round,line cap=round,fill=fillColor] (436.36,249.73) circle (  1.16);

\path[draw=drawColor,line width= 0.4pt,line join=round,line cap=round,fill=fillColor] (436.58,249.69) circle (  1.16);

\path[draw=drawColor,line width= 0.4pt,line join=round,line cap=round,fill=fillColor] (436.80,249.68) circle (  1.16);

\path[draw=drawColor,line width= 0.4pt,line join=round,line cap=round,fill=fillColor] (437.02,249.62) circle (  1.16);

\path[draw=drawColor,line width= 0.4pt,line join=round,line cap=round,fill=fillColor] (437.24,249.60) circle (  1.16);

\path[draw=drawColor,line width= 0.4pt,line join=round,line cap=round,fill=fillColor] (437.46,249.59) circle (  1.16);

\path[draw=drawColor,line width= 0.4pt,line join=round,line cap=round,fill=fillColor] (437.68,249.59) circle (  1.16);

\path[draw=drawColor,line width= 0.4pt,line join=round,line cap=round,fill=fillColor] (437.90,249.56) circle (  1.16);

\path[draw=drawColor,line width= 0.4pt,line join=round,line cap=round,fill=fillColor] (438.12,249.54) circle (  1.16);

\path[draw=drawColor,line width= 0.4pt,line join=round,line cap=round,fill=fillColor] (438.33,249.47) circle (  1.16);

\path[draw=drawColor,line width= 0.4pt,line join=round,line cap=round,fill=fillColor] (438.55,249.45) circle (  1.16);

\path[draw=drawColor,line width= 0.4pt,line join=round,line cap=round,fill=fillColor] (438.76,249.29) circle (  1.16);

\path[draw=drawColor,line width= 0.4pt,line join=round,line cap=round,fill=fillColor] (438.98,249.26) circle (  1.16);

\path[draw=drawColor,line width= 0.4pt,line join=round,line cap=round,fill=fillColor] (439.19,249.25) circle (  1.16);

\path[draw=drawColor,line width= 0.4pt,line join=round,line cap=round,fill=fillColor] (439.41,249.18) circle (  1.16);

\path[draw=drawColor,line width= 0.4pt,line join=round,line cap=round,fill=fillColor] (439.62,249.17) circle (  1.16);

\path[draw=drawColor,line width= 0.4pt,line join=round,line cap=round,fill=fillColor] (439.83,249.13) circle (  1.16);

\path[draw=drawColor,line width= 0.4pt,line join=round,line cap=round,fill=fillColor] (440.04,249.12) circle (  1.16);

\path[draw=drawColor,line width= 0.4pt,line join=round,line cap=round,fill=fillColor] (440.25,248.89) circle (  1.16);

\path[draw=drawColor,line width= 0.4pt,line join=round,line cap=round,fill=fillColor] (440.46,248.87) circle (  1.16);

\path[draw=drawColor,line width= 0.4pt,line join=round,line cap=round,fill=fillColor] (440.67,248.86) circle (  1.16);

\path[draw=drawColor,line width= 0.4pt,line join=round,line cap=round,fill=fillColor] (440.88,248.76) circle (  1.16);

\path[draw=drawColor,line width= 0.4pt,line join=round,line cap=round,fill=fillColor] (441.09,248.68) circle (  1.16);

\path[draw=drawColor,line width= 0.4pt,line join=round,line cap=round,fill=fillColor] (441.29,248.64) circle (  1.16);

\path[draw=drawColor,line width= 0.4pt,line join=round,line cap=round,fill=fillColor] (441.50,248.63) circle (  1.16);

\path[draw=drawColor,line width= 0.4pt,line join=round,line cap=round,fill=fillColor] (441.71,248.58) circle (  1.16);

\path[draw=drawColor,line width= 0.4pt,line join=round,line cap=round,fill=fillColor] (441.91,248.51) circle (  1.16);

\path[draw=drawColor,line width= 0.4pt,line join=round,line cap=round,fill=fillColor] (442.12,248.48) circle (  1.16);

\path[draw=drawColor,line width= 0.4pt,line join=round,line cap=round,fill=fillColor] (442.32,248.47) circle (  1.16);

\path[draw=drawColor,line width= 0.4pt,line join=round,line cap=round,fill=fillColor] (442.53,248.44) circle (  1.16);

\path[draw=drawColor,line width= 0.4pt,line join=round,line cap=round,fill=fillColor] (442.73,248.44) circle (  1.16);

\path[draw=drawColor,line width= 0.4pt,line join=round,line cap=round,fill=fillColor] (442.93,248.41) circle (  1.16);

\path[draw=drawColor,line width= 0.4pt,line join=round,line cap=round,fill=fillColor] (443.13,248.34) circle (  1.16);

\path[draw=drawColor,line width= 0.4pt,line join=round,line cap=round,fill=fillColor] (443.33,248.23) circle (  1.16);

\path[draw=drawColor,line width= 0.4pt,line join=round,line cap=round,fill=fillColor] (443.54,248.19) circle (  1.16);

\path[draw=drawColor,line width= 0.4pt,line join=round,line cap=round,fill=fillColor] (443.74,248.15) circle (  1.16);

\path[draw=drawColor,line width= 0.4pt,line join=round,line cap=round,fill=fillColor] (443.94,248.10) circle (  1.16);

\path[draw=drawColor,line width= 0.4pt,line join=round,line cap=round,fill=fillColor] (444.13,248.09) circle (  1.16);

\path[draw=drawColor,line width= 0.4pt,line join=round,line cap=round,fill=fillColor] (444.33,248.06) circle (  1.16);

\path[draw=drawColor,line width= 0.4pt,line join=round,line cap=round,fill=fillColor] (444.53,247.86) circle (  1.16);

\path[draw=drawColor,line width= 0.4pt,line join=round,line cap=round,fill=fillColor] (444.73,247.81) circle (  1.16);

\path[draw=drawColor,line width= 0.4pt,line join=round,line cap=round,fill=fillColor] (444.92,247.80) circle (  1.16);

\path[draw=drawColor,line width= 0.4pt,line join=round,line cap=round,fill=fillColor] (445.12,247.79) circle (  1.16);

\path[draw=drawColor,line width= 0.4pt,line join=round,line cap=round,fill=fillColor] (445.32,247.78) circle (  1.16);

\path[draw=drawColor,line width= 0.4pt,line join=round,line cap=round,fill=fillColor] (445.51,247.77) circle (  1.16);

\path[draw=drawColor,line width= 0.4pt,line join=round,line cap=round,fill=fillColor] (445.71,247.77) circle (  1.16);

\path[draw=drawColor,line width= 0.4pt,line join=round,line cap=round,fill=fillColor] (445.90,247.73) circle (  1.16);

\path[draw=drawColor,line width= 0.4pt,line join=round,line cap=round,fill=fillColor] (446.09,247.70) circle (  1.16);

\path[draw=drawColor,line width= 0.4pt,line join=round,line cap=round,fill=fillColor] (446.29,247.49) circle (  1.16);

\path[draw=drawColor,line width= 0.4pt,line join=round,line cap=round,fill=fillColor] (446.48,247.40) circle (  1.16);

\path[draw=drawColor,line width= 0.4pt,line join=round,line cap=round,fill=fillColor] (446.67,247.37) circle (  1.16);

\path[draw=drawColor,line width= 0.4pt,line join=round,line cap=round,fill=fillColor] (446.86,247.28) circle (  1.16);

\path[draw=drawColor,line width= 0.4pt,line join=round,line cap=round,fill=fillColor] (447.05,247.27) circle (  1.16);

\path[draw=drawColor,line width= 0.4pt,line join=round,line cap=round,fill=fillColor] (447.24,247.26) circle (  1.16);

\path[draw=drawColor,line width= 0.4pt,line join=round,line cap=round,fill=fillColor] (447.43,247.25) circle (  1.16);

\path[draw=drawColor,line width= 0.4pt,line join=round,line cap=round,fill=fillColor] (447.62,247.24) circle (  1.16);

\path[draw=drawColor,line width= 0.4pt,line join=round,line cap=round,fill=fillColor] (447.81,247.21) circle (  1.16);

\path[draw=drawColor,line width= 0.4pt,line join=round,line cap=round,fill=fillColor] (448.00,247.18) circle (  1.16);

\path[draw=drawColor,line width= 0.4pt,line join=round,line cap=round,fill=fillColor] (448.19,247.18) circle (  1.16);

\path[draw=drawColor,line width= 0.4pt,line join=round,line cap=round,fill=fillColor] (448.37,247.13) circle (  1.16);

\path[draw=drawColor,line width= 0.4pt,line join=round,line cap=round,fill=fillColor] (448.56,247.04) circle (  1.16);

\path[draw=drawColor,line width= 0.4pt,line join=round,line cap=round,fill=fillColor] (448.75,247.02) circle (  1.16);

\path[draw=drawColor,line width= 0.4pt,line join=round,line cap=round,fill=fillColor] (448.93,246.98) circle (  1.16);

\path[draw=drawColor,line width= 0.4pt,line join=round,line cap=round,fill=fillColor] (449.12,246.87) circle (  1.16);

\path[draw=drawColor,line width= 0.4pt,line join=round,line cap=round,fill=fillColor] (449.30,246.79) circle (  1.16);

\path[draw=drawColor,line width= 0.4pt,line join=round,line cap=round,fill=fillColor] (449.49,246.77) circle (  1.16);

\path[draw=drawColor,line width= 0.4pt,line join=round,line cap=round,fill=fillColor] (449.67,246.77) circle (  1.16);

\path[draw=drawColor,line width= 0.4pt,line join=round,line cap=round,fill=fillColor] (449.86,246.75) circle (  1.16);

\path[draw=drawColor,line width= 0.4pt,line join=round,line cap=round,fill=fillColor] (450.04,246.74) circle (  1.16);

\path[draw=drawColor,line width= 0.4pt,line join=round,line cap=round,fill=fillColor] (450.22,246.71) circle (  1.16);

\path[draw=drawColor,line width= 0.4pt,line join=round,line cap=round,fill=fillColor] (450.40,246.65) circle (  1.16);

\path[draw=drawColor,line width= 0.4pt,line join=round,line cap=round,fill=fillColor] (450.58,246.64) circle (  1.16);

\path[draw=drawColor,line width= 0.4pt,line join=round,line cap=round,fill=fillColor] (450.77,246.63) circle (  1.16);

\path[draw=drawColor,line width= 0.4pt,line join=round,line cap=round,fill=fillColor] (450.95,246.57) circle (  1.16);

\path[draw=drawColor,line width= 0.4pt,line join=round,line cap=round,fill=fillColor] (451.13,246.49) circle (  1.16);

\path[draw=drawColor,line width= 0.4pt,line join=round,line cap=round,fill=fillColor] (451.31,246.45) circle (  1.16);

\path[draw=drawColor,line width= 0.4pt,line join=round,line cap=round,fill=fillColor] (451.49,246.39) circle (  1.16);

\path[draw=drawColor,line width= 0.4pt,line join=round,line cap=round,fill=fillColor] (451.67,246.38) circle (  1.16);

\path[draw=drawColor,line width= 0.4pt,line join=round,line cap=round,fill=fillColor] (451.84,246.36) circle (  1.16);

\path[draw=drawColor,line width= 0.4pt,line join=round,line cap=round,fill=fillColor] (452.02,246.34) circle (  1.16);

\path[draw=drawColor,line width= 0.4pt,line join=round,line cap=round,fill=fillColor] (452.20,246.33) circle (  1.16);

\path[draw=drawColor,line width= 0.4pt,line join=round,line cap=round,fill=fillColor] (452.38,246.29) circle (  1.16);

\path[draw=drawColor,line width= 0.4pt,line join=round,line cap=round,fill=fillColor] (452.55,246.26) circle (  1.16);

\path[draw=drawColor,line width= 0.4pt,line join=round,line cap=round,fill=fillColor] (452.73,246.26) circle (  1.16);

\path[draw=drawColor,line width= 0.4pt,line join=round,line cap=round,fill=fillColor] (452.91,246.24) circle (  1.16);

\path[draw=drawColor,line width= 0.4pt,line join=round,line cap=round,fill=fillColor] (453.08,246.24) circle (  1.16);

\path[draw=drawColor,line width= 0.4pt,line join=round,line cap=round,fill=fillColor] (453.26,246.23) circle (  1.16);

\path[draw=drawColor,line width= 0.4pt,line join=round,line cap=round,fill=fillColor] (453.43,246.21) circle (  1.16);

\path[draw=drawColor,line width= 0.4pt,line join=round,line cap=round,fill=fillColor] (453.61,246.13) circle (  1.16);

\path[draw=drawColor,line width= 0.4pt,line join=round,line cap=round,fill=fillColor] (453.78,246.12) circle (  1.16);

\path[draw=drawColor,line width= 0.4pt,line join=round,line cap=round,fill=fillColor] (453.95,246.11) circle (  1.16);

\path[draw=drawColor,line width= 0.4pt,line join=round,line cap=round,fill=fillColor] (454.13,246.04) circle (  1.16);

\path[draw=drawColor,line width= 0.4pt,line join=round,line cap=round,fill=fillColor] (454.30,246.02) circle (  1.16);

\path[draw=drawColor,line width= 0.4pt,line join=round,line cap=round,fill=fillColor] (454.47,246.01) circle (  1.16);

\path[draw=drawColor,line width= 0.4pt,line join=round,line cap=round,fill=fillColor] (454.64,245.99) circle (  1.16);

\path[draw=drawColor,line width= 0.4pt,line join=round,line cap=round,fill=fillColor] (454.82,245.97) circle (  1.16);

\path[draw=drawColor,line width= 0.4pt,line join=round,line cap=round,fill=fillColor] (454.99,245.93) circle (  1.16);

\path[draw=drawColor,line width= 0.4pt,line join=round,line cap=round,fill=fillColor] (455.16,245.90) circle (  1.16);

\path[draw=drawColor,line width= 0.4pt,line join=round,line cap=round,fill=fillColor] (455.33,245.84) circle (  1.16);

\path[draw=drawColor,line width= 0.4pt,line join=round,line cap=round,fill=fillColor] (455.50,245.82) circle (  1.16);

\path[draw=drawColor,line width= 0.4pt,line join=round,line cap=round,fill=fillColor] (455.67,245.78) circle (  1.16);

\path[draw=drawColor,line width= 0.4pt,line join=round,line cap=round,fill=fillColor] (455.84,245.68) circle (  1.16);

\path[draw=drawColor,line width= 0.4pt,line join=round,line cap=round,fill=fillColor] (456.00,245.68) circle (  1.16);

\path[draw=drawColor,line width= 0.4pt,line join=round,line cap=round,fill=fillColor] (456.17,245.67) circle (  1.16);

\path[draw=drawColor,line width= 0.4pt,line join=round,line cap=round,fill=fillColor] (456.34,245.65) circle (  1.16);

\path[draw=drawColor,line width= 0.4pt,line join=round,line cap=round,fill=fillColor] (456.51,245.64) circle (  1.16);

\path[draw=drawColor,line width= 0.4pt,line join=round,line cap=round,fill=fillColor] (456.68,245.63) circle (  1.16);

\path[draw=drawColor,line width= 0.4pt,line join=round,line cap=round,fill=fillColor] (456.84,245.63) circle (  1.16);

\path[draw=drawColor,line width= 0.4pt,line join=round,line cap=round,fill=fillColor] (457.01,245.62) circle (  1.16);

\path[draw=drawColor,line width= 0.4pt,line join=round,line cap=round,fill=fillColor] (457.18,245.59) circle (  1.16);

\path[draw=drawColor,line width= 0.4pt,line join=round,line cap=round,fill=fillColor] (457.34,245.58) circle (  1.16);

\path[draw=drawColor,line width= 0.4pt,line join=round,line cap=round,fill=fillColor] (457.51,245.58) circle (  1.16);

\path[draw=drawColor,line width= 0.4pt,line join=round,line cap=round,fill=fillColor] (457.67,245.56) circle (  1.16);

\path[draw=drawColor,line width= 0.4pt,line join=round,line cap=round,fill=fillColor] (457.84,245.53) circle (  1.16);

\path[draw=drawColor,line width= 0.4pt,line join=round,line cap=round,fill=fillColor] (458.00,245.46) circle (  1.16);

\path[draw=drawColor,line width= 0.4pt,line join=round,line cap=round,fill=fillColor] (458.17,245.45) circle (  1.16);

\path[draw=drawColor,line width= 0.4pt,line join=round,line cap=round,fill=fillColor] (458.33,245.45) circle (  1.16);

\path[draw=drawColor,line width= 0.4pt,line join=round,line cap=round,fill=fillColor] (458.49,245.44) circle (  1.16);

\path[draw=drawColor,line width= 0.4pt,line join=round,line cap=round,fill=fillColor] (458.66,245.42) circle (  1.16);

\path[draw=drawColor,line width= 0.4pt,line join=round,line cap=round,fill=fillColor] (458.82,245.40) circle (  1.16);

\path[draw=drawColor,line width= 0.4pt,line join=round,line cap=round,fill=fillColor] (458.98,245.38) circle (  1.16);

\path[draw=drawColor,line width= 0.4pt,line join=round,line cap=round,fill=fillColor] (459.14,245.33) circle (  1.16);

\path[draw=drawColor,line width= 0.4pt,line join=round,line cap=round,fill=fillColor] (459.30,245.27) circle (  1.16);

\path[draw=drawColor,line width= 0.4pt,line join=round,line cap=round,fill=fillColor] (459.47,245.22) circle (  1.16);

\path[draw=drawColor,line width= 0.4pt,line join=round,line cap=round,fill=fillColor] (459.63,245.16) circle (  1.16);

\path[draw=drawColor,line width= 0.4pt,line join=round,line cap=round,fill=fillColor] (459.79,245.13) circle (  1.16);

\path[draw=drawColor,line width= 0.4pt,line join=round,line cap=round,fill=fillColor] (459.95,245.07) circle (  1.16);

\path[draw=drawColor,line width= 0.4pt,line join=round,line cap=round,fill=fillColor] (460.11,245.05) circle (  1.16);

\path[draw=drawColor,line width= 0.4pt,line join=round,line cap=round,fill=fillColor] (460.27,244.97) circle (  1.16);

\path[draw=drawColor,line width= 0.4pt,line join=round,line cap=round,fill=fillColor] (460.43,244.92) circle (  1.16);

\path[draw=drawColor,line width= 0.4pt,line join=round,line cap=round,fill=fillColor] (460.59,244.86) circle (  1.16);

\path[draw=drawColor,line width= 0.4pt,line join=round,line cap=round,fill=fillColor] (460.74,244.76) circle (  1.16);

\path[draw=drawColor,line width= 0.4pt,line join=round,line cap=round,fill=fillColor] (460.90,244.74) circle (  1.16);

\path[draw=drawColor,line width= 0.4pt,line join=round,line cap=round,fill=fillColor] (461.06,244.68) circle (  1.16);

\path[draw=drawColor,line width= 0.4pt,line join=round,line cap=round,fill=fillColor] (461.22,244.61) circle (  1.16);

\path[draw=drawColor,line width= 0.4pt,line join=round,line cap=round,fill=fillColor] (461.38,244.57) circle (  1.16);

\path[draw=drawColor,line width= 0.4pt,line join=round,line cap=round,fill=fillColor] (461.53,244.48) circle (  1.16);

\path[draw=drawColor,line width= 0.4pt,line join=round,line cap=round,fill=fillColor] (461.69,244.41) circle (  1.16);

\path[draw=drawColor,line width= 0.4pt,line join=round,line cap=round,fill=fillColor] (461.85,244.41) circle (  1.16);

\path[draw=drawColor,line width= 0.4pt,line join=round,line cap=round,fill=fillColor] (462.00,244.39) circle (  1.16);

\path[draw=drawColor,line width= 0.4pt,line join=round,line cap=round,fill=fillColor] (462.16,244.34) circle (  1.16);

\path[draw=drawColor,line width= 0.4pt,line join=round,line cap=round,fill=fillColor] (462.31,244.33) circle (  1.16);

\path[draw=drawColor,line width= 0.4pt,line join=round,line cap=round,fill=fillColor] (462.47,244.28) circle (  1.16);

\path[draw=drawColor,line width= 0.4pt,line join=round,line cap=round,fill=fillColor] (462.62,244.19) circle (  1.16);

\path[draw=drawColor,line width= 0.4pt,line join=round,line cap=round,fill=fillColor] (462.78,244.11) circle (  1.16);

\path[draw=drawColor,line width= 0.4pt,line join=round,line cap=round,fill=fillColor] (462.93,244.11) circle (  1.16);

\path[draw=drawColor,line width= 0.4pt,line join=round,line cap=round,fill=fillColor] (463.09,244.08) circle (  1.16);

\path[draw=drawColor,line width= 0.4pt,line join=round,line cap=round,fill=fillColor] (463.24,244.07) circle (  1.16);

\path[draw=drawColor,line width= 0.4pt,line join=round,line cap=round,fill=fillColor] (463.39,244.05) circle (  1.16);

\path[draw=drawColor,line width= 0.4pt,line join=round,line cap=round,fill=fillColor] (463.55,243.99) circle (  1.16);

\path[draw=drawColor,line width= 0.4pt,line join=round,line cap=round,fill=fillColor] (463.70,243.96) circle (  1.16);

\path[draw=drawColor,line width= 0.4pt,line join=round,line cap=round,fill=fillColor] (463.85,243.94) circle (  1.16);

\path[draw=drawColor,line width= 0.4pt,line join=round,line cap=round,fill=fillColor] (464.00,243.86) circle (  1.16);

\path[draw=drawColor,line width= 0.4pt,line join=round,line cap=round,fill=fillColor] (464.16,243.86) circle (  1.16);

\path[draw=drawColor,line width= 0.4pt,line join=round,line cap=round,fill=fillColor] (464.31,243.81) circle (  1.16);

\path[draw=drawColor,line width= 0.4pt,line join=round,line cap=round,fill=fillColor] (464.46,243.78) circle (  1.16);

\path[draw=drawColor,line width= 0.4pt,line join=round,line cap=round,fill=fillColor] (464.61,243.78) circle (  1.16);

\path[draw=drawColor,line width= 0.4pt,line join=round,line cap=round,fill=fillColor] (464.76,243.73) circle (  1.16);

\path[draw=drawColor,line width= 0.4pt,line join=round,line cap=round,fill=fillColor] (464.91,243.73) circle (  1.16);

\path[draw=drawColor,line width= 0.4pt,line join=round,line cap=round,fill=fillColor] (465.06,243.72) circle (  1.16);

\path[draw=drawColor,line width= 0.4pt,line join=round,line cap=round,fill=fillColor] (465.21,243.67) circle (  1.16);

\path[draw=drawColor,line width= 0.4pt,line join=round,line cap=round,fill=fillColor] (465.36,243.59) circle (  1.16);

\path[draw=drawColor,line width= 0.4pt,line join=round,line cap=round,fill=fillColor] (465.51,243.47) circle (  1.16);

\path[draw=drawColor,line width= 0.4pt,line join=round,line cap=round,fill=fillColor] (465.66,243.46) circle (  1.16);

\path[draw=drawColor,line width= 0.4pt,line join=round,line cap=round,fill=fillColor] (465.81,243.42) circle (  1.16);

\path[draw=drawColor,line width= 0.4pt,line join=round,line cap=round,fill=fillColor] (465.96,243.42) circle (  1.16);

\path[draw=drawColor,line width= 0.4pt,line join=round,line cap=round,fill=fillColor] (466.10,243.42) circle (  1.16);

\path[draw=drawColor,line width= 0.4pt,line join=round,line cap=round,fill=fillColor] (466.25,243.37) circle (  1.16);

\path[draw=drawColor,line width= 0.4pt,line join=round,line cap=round,fill=fillColor] (466.40,243.33) circle (  1.16);

\path[draw=drawColor,line width= 0.4pt,line join=round,line cap=round,fill=fillColor] (466.55,243.31) circle (  1.16);

\path[draw=drawColor,line width= 0.4pt,line join=round,line cap=round,fill=fillColor] (466.70,243.31) circle (  1.16);

\path[draw=drawColor,line width= 0.4pt,line join=round,line cap=round,fill=fillColor] (466.84,243.20) circle (  1.16);

\path[draw=drawColor,line width= 0.4pt,line join=round,line cap=round,fill=fillColor] (466.99,243.16) circle (  1.16);

\path[draw=drawColor,line width= 0.4pt,line join=round,line cap=round,fill=fillColor] (467.14,243.15) circle (  1.16);

\path[draw=drawColor,line width= 0.4pt,line join=round,line cap=round,fill=fillColor] (467.28,243.10) circle (  1.16);

\path[draw=drawColor,line width= 0.4pt,line join=round,line cap=round,fill=fillColor] (467.43,243.08) circle (  1.16);

\path[draw=drawColor,line width= 0.4pt,line join=round,line cap=round,fill=fillColor] (467.57,243.05) circle (  1.16);

\path[draw=drawColor,line width= 0.4pt,line join=round,line cap=round,fill=fillColor] (467.72,242.96) circle (  1.16);

\path[draw=drawColor,line width= 0.4pt,line join=round,line cap=round,fill=fillColor] (467.86,242.91) circle (  1.16);

\path[draw=drawColor,line width= 0.4pt,line join=round,line cap=round,fill=fillColor] (468.01,242.42) circle (  1.16);

\path[draw=drawColor,line width= 0.4pt,line join=round,line cap=round,fill=fillColor] (468.15,242.29) circle (  1.16);

\path[draw=drawColor,line width= 0.4pt,line join=round,line cap=round,fill=fillColor] (468.30,242.24) circle (  1.16);

\path[draw=drawColor,line width= 0.4pt,line join=round,line cap=round,fill=fillColor] (468.44,242.23) circle (  1.16);

\path[draw=drawColor,line width= 0.4pt,line join=round,line cap=round,fill=fillColor] (468.58,241.90) circle (  1.16);

\path[draw=drawColor,line width= 0.4pt,line join=round,line cap=round,fill=fillColor] (468.73,241.85) circle (  1.16);

\path[draw=drawColor,line width= 0.4pt,line join=round,line cap=round,fill=fillColor] (468.87,241.79) circle (  1.16);

\path[draw=drawColor,line width= 0.4pt,line join=round,line cap=round,fill=fillColor] (469.01,241.77) circle (  1.16);

\path[draw=drawColor,line width= 0.4pt,line join=round,line cap=round,fill=fillColor] (469.16,241.66) circle (  1.16);

\path[draw=drawColor,line width= 0.4pt,line join=round,line cap=round,fill=fillColor] (469.30,241.65) circle (  1.16);

\path[draw=drawColor,line width= 0.4pt,line join=round,line cap=round,fill=fillColor] (469.44,241.62) circle (  1.16);

\path[draw=drawColor,line width= 0.4pt,line join=round,line cap=round,fill=fillColor] (469.58,241.61) circle (  1.16);

\path[draw=drawColor,line width= 0.4pt,line join=round,line cap=round,fill=fillColor] (469.73,241.50) circle (  1.16);

\path[draw=drawColor,line width= 0.4pt,line join=round,line cap=round,fill=fillColor] (469.87,241.49) circle (  1.16);

\path[draw=drawColor,line width= 0.4pt,line join=round,line cap=round,fill=fillColor] (470.01,241.43) circle (  1.16);

\path[draw=drawColor,line width= 0.4pt,line join=round,line cap=round,fill=fillColor] (470.15,241.23) circle (  1.16);

\path[draw=drawColor,line width= 0.4pt,line join=round,line cap=round,fill=fillColor] (470.29,241.20) circle (  1.16);

\path[draw=drawColor,line width= 0.4pt,line join=round,line cap=round,fill=fillColor] (470.43,241.20) circle (  1.16);

\path[draw=drawColor,line width= 0.4pt,line join=round,line cap=round,fill=fillColor] (470.57,241.15) circle (  1.16);

\path[draw=drawColor,line width= 0.4pt,line join=round,line cap=round,fill=fillColor] (470.71,241.13) circle (  1.16);

\path[draw=drawColor,line width= 0.4pt,line join=round,line cap=round,fill=fillColor] (470.85,240.99) circle (  1.16);

\path[draw=drawColor,line width= 0.4pt,line join=round,line cap=round,fill=fillColor] (470.99,240.93) circle (  1.16);

\path[draw=drawColor,line width= 0.4pt,line join=round,line cap=round,fill=fillColor] (471.13,240.88) circle (  1.16);

\path[draw=drawColor,line width= 0.4pt,line join=round,line cap=round,fill=fillColor] (471.27,240.74) circle (  1.16);

\path[draw=drawColor,line width= 0.4pt,line join=round,line cap=round,fill=fillColor] (471.41,240.57) circle (  1.16);

\path[draw=drawColor,line width= 0.4pt,line join=round,line cap=round,fill=fillColor] (471.55,240.44) circle (  1.16);

\path[draw=drawColor,line width= 0.4pt,line join=round,line cap=round,fill=fillColor] (471.69,240.42) circle (  1.16);

\path[draw=drawColor,line width= 0.4pt,line join=round,line cap=round,fill=fillColor] (471.82,240.39) circle (  1.16);

\path[draw=drawColor,line width= 0.4pt,line join=round,line cap=round,fill=fillColor] (471.96,240.34) circle (  1.16);

\path[draw=drawColor,line width= 0.4pt,line join=round,line cap=round,fill=fillColor] (472.10,240.26) circle (  1.16);

\path[draw=drawColor,line width= 0.4pt,line join=round,line cap=round,fill=fillColor] (472.24,240.17) circle (  1.16);

\path[draw=drawColor,line width= 0.4pt,line join=round,line cap=round,fill=fillColor] (472.37,239.90) circle (  1.16);

\path[draw=drawColor,line width= 0.4pt,line join=round,line cap=round,fill=fillColor] (472.51,239.79) circle (  1.16);

\path[draw=drawColor,line width= 0.4pt,line join=round,line cap=round,fill=fillColor] (472.65,239.38) circle (  1.16);

\path[draw=drawColor,line width= 0.4pt,line join=round,line cap=round,fill=fillColor] (472.78,238.97) circle (  1.16);

\path[draw=drawColor,line width= 0.4pt,line join=round,line cap=round,fill=fillColor] (472.92,238.70) circle (  1.16);

\path[draw=drawColor,line width= 0.4pt,line join=round,line cap=round,fill=fillColor] (473.06,238.68) circle (  1.16);

\path[draw=drawColor,line width= 0.4pt,line join=round,line cap=round,fill=fillColor] (473.19,238.68) circle (  1.16);

\path[draw=drawColor,line width= 0.4pt,line join=round,line cap=round,fill=fillColor] (473.33,238.62) circle (  1.16);

\path[draw=drawColor,line width= 0.4pt,line join=round,line cap=round,fill=fillColor] (473.46,238.52) circle (  1.16);

\path[draw=drawColor,line width= 0.4pt,line join=round,line cap=round,fill=fillColor] (473.60,238.50) circle (  1.16);

\path[draw=drawColor,line width= 0.4pt,line join=round,line cap=round,fill=fillColor] (473.74,238.46) circle (  1.16);

\path[draw=drawColor,line width= 0.4pt,line join=round,line cap=round,fill=fillColor] (473.87,238.43) circle (  1.16);

\path[draw=drawColor,line width= 0.4pt,line join=round,line cap=round,fill=fillColor] (474.00,238.34) circle (  1.16);

\path[draw=drawColor,line width= 0.4pt,line join=round,line cap=round,fill=fillColor] (474.14,238.14) circle (  1.16);

\path[draw=drawColor,line width= 0.4pt,line join=round,line cap=round,fill=fillColor] (474.27,237.96) circle (  1.16);

\path[draw=drawColor,line width= 0.4pt,line join=round,line cap=round,fill=fillColor] (474.41,237.85) circle (  1.16);

\path[draw=drawColor,line width= 0.4pt,line join=round,line cap=round,fill=fillColor] (474.54,237.49) circle (  1.16);

\path[draw=drawColor,line width= 0.4pt,line join=round,line cap=round,fill=fillColor] (474.68,237.46) circle (  1.16);

\path[draw=drawColor,line width= 0.4pt,line join=round,line cap=round,fill=fillColor] (474.81,237.31) circle (  1.16);

\path[draw=drawColor,line width= 0.4pt,line join=round,line cap=round,fill=fillColor] (474.94,237.14) circle (  1.16);

\path[draw=drawColor,line width= 0.4pt,line join=round,line cap=round,fill=fillColor] (475.07,236.82) circle (  1.16);

\path[draw=drawColor,line width= 0.4pt,line join=round,line cap=round,fill=fillColor] (475.21,236.42) circle (  1.16);

\path[draw=drawColor,line width= 0.4pt,line join=round,line cap=round,fill=fillColor] (475.34,236.28) circle (  1.16);

\path[draw=drawColor,line width= 0.4pt,line join=round,line cap=round,fill=fillColor] (475.47,235.57) circle (  1.16);

\path[draw=drawColor,line width= 0.4pt,line join=round,line cap=round,fill=fillColor] (475.60,235.56) circle (  1.16);

\path[draw=drawColor,line width= 0.4pt,line join=round,line cap=round,fill=fillColor] (475.74,235.36) circle (  1.16);

\path[draw=drawColor,line width= 0.4pt,line join=round,line cap=round,fill=fillColor] (475.87,235.35) circle (  1.16);

\path[draw=drawColor,line width= 0.4pt,line join=round,line cap=round,fill=fillColor] (476.00,235.30) circle (  1.16);

\path[draw=drawColor,line width= 0.4pt,line join=round,line cap=round,fill=fillColor] (476.13,235.10) circle (  1.16);

\path[draw=drawColor,line width= 0.4pt,line join=round,line cap=round,fill=fillColor] (476.26,234.86) circle (  1.16);

\path[draw=drawColor,line width= 0.4pt,line join=round,line cap=round,fill=fillColor] (476.39,234.74) circle (  1.16);

\path[draw=drawColor,line width= 0.4pt,line join=round,line cap=round,fill=fillColor] (476.52,234.74) circle (  1.16);

\path[draw=drawColor,line width= 0.4pt,line join=round,line cap=round,fill=fillColor] (476.65,234.74) circle (  1.16);

\path[draw=drawColor,line width= 0.4pt,line join=round,line cap=round,fill=fillColor] (476.78,234.73) circle (  1.16);

\path[draw=drawColor,line width= 0.4pt,line join=round,line cap=round,fill=fillColor] (476.91,233.47) circle (  1.16);

\path[draw=drawColor,line width= 0.4pt,line join=round,line cap=round,fill=fillColor] (477.04,233.16) circle (  1.16);

\path[draw=drawColor,line width= 0.4pt,line join=round,line cap=round,fill=fillColor] (477.17,232.65) circle (  1.16);

\path[draw=drawColor,line width= 0.4pt,line join=round,line cap=round,fill=fillColor] (477.30,231.38) circle (  1.16);

\path[draw=drawColor,line width= 0.4pt,line join=round,line cap=round,fill=fillColor] (477.43,229.89) circle (  1.16);

\path[draw=drawColor,line width= 0.4pt,line join=round,line cap=round,fill=fillColor] (477.56,229.09) circle (  1.16);

\path[draw=drawColor,line width= 0.4pt,line join=round,line cap=round,fill=fillColor] (477.69,227.37) circle (  1.16);

\path[draw=drawColor,line width= 0.4pt,line join=round,line cap=round,fill=fillColor] (477.82,227.35) circle (  1.16);

\path[draw=drawColor,line width= 0.4pt,line join=round,line cap=round,fill=fillColor] (477.95,227.18) circle (  1.16);

\path[draw=drawColor,line width= 0.4pt,line join=round,line cap=round,fill=fillColor] (478.08,209.81) circle (  1.16);

\path[draw=drawColor,line width= 0.4pt,line join=round,line cap=round,fill=fillColor] (478.21,209.81) circle (  1.16);
\definecolor[named]{drawColor}{rgb}{0.00,0.00,0.00}
\definecolor[named]{fillColor}{rgb}{0.00,0.00,0.00}

\path[draw=drawColor,line width= 0.6pt,line join=round,fill=fillColor] (320.31,292.78) -- (485.72,292.78);

\node[text=drawColor,anchor=base east,inner sep=0pt, outer sep=0pt, scale=  0.85] at (482.22,294.93) {infeasible solutions};

\path[draw=drawColor,line width= 0.6pt,line join=round,line cap=round] (320.31,201.52) rectangle (485.72,303.75);
\end{scope}
\begin{scope}
\path[clip] (  0.00,  0.00) rectangle (505.89,650.43);
\definecolor[named]{drawColor}{rgb}{0.00,0.00,0.00}

\node[text=drawColor,anchor=base east,inner sep=0pt, outer sep=0pt, scale=  0.80] at (314.91,207.06) {0.00};

\node[text=drawColor,anchor=base east,inner sep=0pt, outer sep=0pt, scale=  0.80] at (314.91,224.93) {0.01};

\node[text=drawColor,anchor=base east,inner sep=0pt, outer sep=0pt, scale=  0.80] at (314.91,237.62) {0.05};

\node[text=drawColor,anchor=base east,inner sep=0pt, outer sep=0pt, scale=  0.80] at (314.91,245.57) {0.10};

\node[text=drawColor,anchor=base east,inner sep=0pt, outer sep=0pt, scale=  0.80] at (314.91,255.57) {0.20};

\node[text=drawColor,anchor=base east,inner sep=0pt, outer sep=0pt, scale=  0.80] at (314.91,268.19) {0.40};

\node[text=drawColor,anchor=base east,inner sep=0pt, outer sep=0pt, scale=  0.80] at (314.91,277.03) {0.60};

\node[text=drawColor,anchor=base east,inner sep=0pt, outer sep=0pt, scale=  0.80] at (314.91,284.07) {0.80};

\node[text=drawColor,anchor=base east,inner sep=0pt, outer sep=0pt, scale=  0.80] at (314.91,290.02) {1.00};
\end{scope}
\begin{scope}
\path[clip] (  0.00,  0.00) rectangle (505.89,650.43);
\definecolor[named]{drawColor}{rgb}{0.00,0.00,0.00}

\path[draw=drawColor,line width= 0.6pt,line join=round] (317.31,209.81) --
	(320.31,209.81);

\path[draw=drawColor,line width= 0.6pt,line join=round] (317.31,227.69) --
	(320.31,227.69);

\path[draw=drawColor,line width= 0.6pt,line join=round] (317.31,240.38) --
	(320.31,240.38);

\path[draw=drawColor,line width= 0.6pt,line join=round] (317.31,248.32) --
	(320.31,248.32);

\path[draw=drawColor,line width= 0.6pt,line join=round] (317.31,258.33) --
	(320.31,258.33);

\path[draw=drawColor,line width= 0.6pt,line join=round] (317.31,270.94) --
	(320.31,270.94);

\path[draw=drawColor,line width= 0.6pt,line join=round] (317.31,279.79) --
	(320.31,279.79);

\path[draw=drawColor,line width= 0.6pt,line join=round] (317.31,286.83) --
	(320.31,286.83);

\path[draw=drawColor,line width= 0.6pt,line join=round] (317.31,292.78) --
	(320.31,292.78);
\end{scope}
\begin{scope}
\path[clip] (  0.00,  0.00) rectangle (505.89,650.43);
\definecolor[named]{drawColor}{rgb}{0.00,0.00,0.00}

\path[draw=drawColor,line width= 0.6pt,line join=round] (410.01,198.52) --
	(410.01,201.52);

\path[draw=drawColor,line width= 0.6pt,line join=round] (437.24,198.52) --
	(437.24,201.52);

\path[draw=drawColor,line width= 0.6pt,line join=round] (456.34,198.52) --
	(456.34,201.52);

\path[draw=drawColor,line width= 0.6pt,line join=round] (471.55,198.52) --
	(471.55,201.52);

\path[draw=drawColor,line width= 0.6pt,line join=round] (484.39,198.52) --
	(484.39,201.52);
\end{scope}
\begin{scope}
\path[clip] (  0.00,  0.00) rectangle (505.89,650.43);
\definecolor[named]{drawColor}{rgb}{0.00,0.00,0.00}

\node[text=drawColor,rotate= 50.00,anchor=base east,inner sep=0pt, outer sep=0pt, scale=  0.80] at (414.23,192.57) {100};

\node[text=drawColor,rotate= 50.00,anchor=base east,inner sep=0pt, outer sep=0pt, scale=  0.80] at (441.46,192.57) {200};

\node[text=drawColor,rotate= 50.00,anchor=base east,inner sep=0pt, outer sep=0pt, scale=  0.80] at (460.56,192.57) {300};

\node[text=drawColor,rotate= 50.00,anchor=base east,inner sep=0pt, outer sep=0pt, scale=  0.80] at (475.77,192.57) {400};

\node[text=drawColor,rotate= 50.00,anchor=base east,inner sep=0pt, outer sep=0pt, scale=  0.80] at (488.61,192.57) {500};
\end{scope}
\begin{scope}
\path[clip] (  0.00,  0.00) rectangle (505.89,650.43);
\definecolor[named]{drawColor}{rgb}{0.00,0.00,0.00}

\node[text=drawColor,anchor=base,inner sep=0pt, outer sep=0pt, scale=  1.10] at (403.01,171.01) {\# Instances};
\end{scope}
\begin{scope}
\path[clip] (  0.00,  0.00) rectangle (505.89,650.43);
\definecolor[named]{drawColor}{rgb}{0.00,0.00,0.00}

\node[text=drawColor,rotate= 90.00,anchor=base,inner sep=0pt, outer sep=0pt, scale=  1.10] at (290.69,252.63) {1-(Best/Algorithm)};
\end{scope}
\begin{scope}
\path[clip] (  0.00,  0.00) rectangle (505.89,650.43);
\definecolor[named]{drawColor}{rgb}{0.00,0.00,0.00}

\node[text=drawColor,anchor=base,inner sep=0pt, outer sep=0pt, scale=  1.20] at (403.01,310.95) {$k=64$};
\end{scope}
\begin{scope}
\path[clip] ( 14.17,  0.00) rectangle (238.78,162.61);
\definecolor[named]{drawColor}{rgb}{1.00,1.00,1.00}
\definecolor[named]{fillColor}{rgb}{1.00,1.00,1.00}

\path[draw=drawColor,line width= 0.6pt,line join=round,line cap=round,fill=fillColor] ( 14.17,  0.00) rectangle (238.78,162.61);
\end{scope}
\begin{scope}
\path[clip] ( 67.36, 38.91) rectangle (232.78,141.14);
\definecolor[named]{fillColor}{rgb}{1.00,1.00,1.00}

\path[fill=fillColor] ( 67.36, 38.91) rectangle (232.78,141.14);
\definecolor[named]{drawColor}{rgb}{0.98,0.98,0.98}

\path[draw=drawColor,line width= 0.6pt,line join=round] ( 67.36, 56.14) --
	(232.78, 56.14);

\path[draw=drawColor,line width= 0.6pt,line join=round] ( 67.36, 71.42) --
	(232.78, 71.42);

\path[draw=drawColor,line width= 0.6pt,line join=round] ( 67.36, 81.74) --
	(232.78, 81.74);

\path[draw=drawColor,line width= 0.6pt,line join=round] ( 67.36, 90.72) --
	(232.78, 90.72);

\path[draw=drawColor,line width= 0.6pt,line join=round] ( 67.36,102.03) --
	(232.78,102.03);

\path[draw=drawColor,line width= 0.6pt,line join=round] ( 67.36,112.76) --
	(232.78,112.76);

\path[draw=drawColor,line width= 0.6pt,line join=round] ( 67.36,120.70) --
	(232.78,120.70);

\path[draw=drawColor,line width= 0.6pt,line join=round] ( 67.36,127.19) --
	(232.78,127.19);

\path[draw=drawColor,line width= 0.6pt,line join=round] ( 67.36,139.11) --
	(232.78,139.11);

\path[draw=drawColor,line width= 0.6pt,line join=round] (130.17, 38.91) --
	(130.17,141.14);

\path[draw=drawColor,line width= 0.6pt,line join=round] (143.87, 38.91) --
	(143.87,141.14);

\path[draw=drawColor,line width= 0.6pt,line join=round] (171.26, 38.91) --
	(171.26,141.14);

\path[draw=drawColor,line width= 0.6pt,line join=round] (194.56, 38.91) --
	(194.56,141.14);

\path[draw=drawColor,line width= 0.6pt,line join=round] (211.82, 38.91) --
	(211.82,141.14);

\path[draw=drawColor,line width= 0.6pt,line join=round] (225.93, 38.91) --
	(225.93,141.14);
\definecolor[named]{drawColor}{rgb}{0.75,0.75,0.75}

\path[draw=drawColor,line width= 0.6pt,dash pattern=on 1pt off 3pt ,line join=round] ( 67.36, 47.20) --
	(232.78, 47.20);

\path[draw=drawColor,line width= 0.6pt,dash pattern=on 1pt off 3pt ,line join=round] ( 67.36, 65.08) --
	(232.78, 65.08);

\path[draw=drawColor,line width= 0.6pt,dash pattern=on 1pt off 3pt ,line join=round] ( 67.36, 77.77) --
	(232.78, 77.77);

\path[draw=drawColor,line width= 0.6pt,dash pattern=on 1pt off 3pt ,line join=round] ( 67.36, 85.71) --
	(232.78, 85.71);

\path[draw=drawColor,line width= 0.6pt,dash pattern=on 1pt off 3pt ,line join=round] ( 67.36, 95.72) --
	(232.78, 95.72);

\path[draw=drawColor,line width= 0.6pt,dash pattern=on 1pt off 3pt ,line join=round] ( 67.36,108.33) --
	(232.78,108.33);

\path[draw=drawColor,line width= 0.6pt,dash pattern=on 1pt off 3pt ,line join=round] ( 67.36,117.18) --
	(232.78,117.18);

\path[draw=drawColor,line width= 0.6pt,dash pattern=on 1pt off 3pt ,line join=round] ( 67.36,124.22) --
	(232.78,124.22);

\path[draw=drawColor,line width= 0.6pt,dash pattern=on 1pt off 3pt ,line join=round] ( 67.36,130.17) --
	(232.78,130.17);

\path[draw=drawColor,line width= 0.6pt,dash pattern=on 1pt off 3pt ,line join=round] (157.56, 38.91) --
	(157.56,141.14);

\path[draw=drawColor,line width= 0.6pt,dash pattern=on 1pt off 3pt ,line join=round] (184.96, 38.91) --
	(184.96,141.14);

\path[draw=drawColor,line width= 0.6pt,dash pattern=on 1pt off 3pt ,line join=round] (204.17, 38.91) --
	(204.17,141.14);

\path[draw=drawColor,line width= 0.6pt,dash pattern=on 1pt off 3pt ,line join=round] (219.47, 38.91) --
	(219.47,141.14);

\path[draw=drawColor,line width= 0.6pt,dash pattern=on 1pt off 3pt ,line join=round] (232.39, 38.91) --
	(232.39,141.14);
\definecolor[named]{drawColor}{rgb}{0.89,0.10,0.11}
\definecolor[named]{fillColor}{rgb}{0.89,0.10,0.11}

\path[draw=drawColor,line width= 0.4pt,line join=round,line cap=round,fill=fillColor] ( 74.88,132.85) circle (  1.16);

\path[draw=drawColor,line width= 0.4pt,line join=round,line cap=round,fill=fillColor] ( 80.78,132.85) circle (  1.16);

\path[draw=drawColor,line width= 0.4pt,line join=round,line cap=round,fill=fillColor] ( 84.92,102.02) circle (  1.16);

\path[draw=drawColor,line width= 0.4pt,line join=round,line cap=round,fill=fillColor] ( 88.22, 99.81) circle (  1.16);

\path[draw=drawColor,line width= 0.4pt,line join=round,line cap=round,fill=fillColor] ( 91.00, 99.74) circle (  1.16);

\path[draw=drawColor,line width= 0.4pt,line join=round,line cap=round,fill=fillColor] ( 93.43, 99.06) circle (  1.16);

\path[draw=drawColor,line width= 0.4pt,line join=round,line cap=round,fill=fillColor] ( 95.61, 96.18) circle (  1.16);

\path[draw=drawColor,line width= 0.4pt,line join=round,line cap=round,fill=fillColor] ( 97.58, 91.15) circle (  1.16);

\path[draw=drawColor,line width= 0.4pt,line join=round,line cap=round,fill=fillColor] ( 99.40, 90.59) circle (  1.16);

\path[draw=drawColor,line width= 0.4pt,line join=round,line cap=round,fill=fillColor] (101.09, 89.85) circle (  1.16);

\path[draw=drawColor,line width= 0.4pt,line join=round,line cap=round,fill=fillColor] (102.67, 88.04) circle (  1.16);

\path[draw=drawColor,line width= 0.4pt,line join=round,line cap=round,fill=fillColor] (104.16, 87.98) circle (  1.16);

\path[draw=drawColor,line width= 0.4pt,line join=round,line cap=round,fill=fillColor] (105.56, 87.79) circle (  1.16);

\path[draw=drawColor,line width= 0.4pt,line join=round,line cap=round,fill=fillColor] (106.90, 87.78) circle (  1.16);

\path[draw=drawColor,line width= 0.4pt,line join=round,line cap=round,fill=fillColor] (108.17, 87.10) circle (  1.16);

\path[draw=drawColor,line width= 0.4pt,line join=round,line cap=round,fill=fillColor] (109.39, 87.10) circle (  1.16);

\path[draw=drawColor,line width= 0.4pt,line join=round,line cap=round,fill=fillColor] (110.56, 87.10) circle (  1.16);

\path[draw=drawColor,line width= 0.4pt,line join=round,line cap=round,fill=fillColor] (111.68, 87.06) circle (  1.16);

\path[draw=drawColor,line width= 0.4pt,line join=round,line cap=round,fill=fillColor] (112.76, 86.73) circle (  1.16);

\path[draw=drawColor,line width= 0.4pt,line join=round,line cap=round,fill=fillColor] (113.81, 86.64) circle (  1.16);

\path[draw=drawColor,line width= 0.4pt,line join=round,line cap=round,fill=fillColor] (114.82, 86.39) circle (  1.16);

\path[draw=drawColor,line width= 0.4pt,line join=round,line cap=round,fill=fillColor] (115.80, 85.84) circle (  1.16);

\path[draw=drawColor,line width= 0.4pt,line join=round,line cap=round,fill=fillColor] (116.74, 85.69) circle (  1.16);

\path[draw=drawColor,line width= 0.4pt,line join=round,line cap=round,fill=fillColor] (117.67, 85.35) circle (  1.16);

\path[draw=drawColor,line width= 0.4pt,line join=round,line cap=round,fill=fillColor] (118.56, 84.13) circle (  1.16);

\path[draw=drawColor,line width= 0.4pt,line join=round,line cap=round,fill=fillColor] (119.44, 83.88) circle (  1.16);

\path[draw=drawColor,line width= 0.4pt,line join=round,line cap=round,fill=fillColor] (120.29, 83.08) circle (  1.16);

\path[draw=drawColor,line width= 0.4pt,line join=round,line cap=round,fill=fillColor] (121.12, 82.97) circle (  1.16);

\path[draw=drawColor,line width= 0.4pt,line join=round,line cap=round,fill=fillColor] (121.93, 82.95) circle (  1.16);

\path[draw=drawColor,line width= 0.4pt,line join=round,line cap=round,fill=fillColor] (122.72, 82.54) circle (  1.16);

\path[draw=drawColor,line width= 0.4pt,line join=round,line cap=round,fill=fillColor] (123.50, 81.59) circle (  1.16);

\path[draw=drawColor,line width= 0.4pt,line join=round,line cap=round,fill=fillColor] (124.26, 81.55) circle (  1.16);

\path[draw=drawColor,line width= 0.4pt,line join=round,line cap=round,fill=fillColor] (125.00, 81.45) circle (  1.16);

\path[draw=drawColor,line width= 0.4pt,line join=round,line cap=round,fill=fillColor] (125.73, 81.36) circle (  1.16);

\path[draw=drawColor,line width= 0.4pt,line join=round,line cap=round,fill=fillColor] (126.44, 80.87) circle (  1.16);

\path[draw=drawColor,line width= 0.4pt,line join=round,line cap=round,fill=fillColor] (127.15, 80.23) circle (  1.16);

\path[draw=drawColor,line width= 0.4pt,line join=round,line cap=round,fill=fillColor] (127.83, 80.22) circle (  1.16);

\path[draw=drawColor,line width= 0.4pt,line join=round,line cap=round,fill=fillColor] (128.51, 80.18) circle (  1.16);

\path[draw=drawColor,line width= 0.4pt,line join=round,line cap=round,fill=fillColor] (129.17, 79.96) circle (  1.16);

\path[draw=drawColor,line width= 0.4pt,line join=round,line cap=round,fill=fillColor] (129.83, 79.78) circle (  1.16);

\path[draw=drawColor,line width= 0.4pt,line join=round,line cap=round,fill=fillColor] (130.47, 79.62) circle (  1.16);

\path[draw=drawColor,line width= 0.4pt,line join=round,line cap=round,fill=fillColor] (131.10, 79.34) circle (  1.16);

\path[draw=drawColor,line width= 0.4pt,line join=round,line cap=round,fill=fillColor] (131.72, 78.75) circle (  1.16);

\path[draw=drawColor,line width= 0.4pt,line join=round,line cap=round,fill=fillColor] (132.33, 78.74) circle (  1.16);

\path[draw=drawColor,line width= 0.4pt,line join=round,line cap=round,fill=fillColor] (132.93, 78.72) circle (  1.16);

\path[draw=drawColor,line width= 0.4pt,line join=round,line cap=round,fill=fillColor] (133.53, 78.63) circle (  1.16);

\path[draw=drawColor,line width= 0.4pt,line join=round,line cap=round,fill=fillColor] (134.11, 78.34) circle (  1.16);

\path[draw=drawColor,line width= 0.4pt,line join=round,line cap=round,fill=fillColor] (134.69, 77.83) circle (  1.16);

\path[draw=drawColor,line width= 0.4pt,line join=round,line cap=round,fill=fillColor] (135.26, 77.80) circle (  1.16);

\path[draw=drawColor,line width= 0.4pt,line join=round,line cap=round,fill=fillColor] (135.82, 77.63) circle (  1.16);

\path[draw=drawColor,line width= 0.4pt,line join=round,line cap=round,fill=fillColor] (136.38, 77.47) circle (  1.16);

\path[draw=drawColor,line width= 0.4pt,line join=round,line cap=round,fill=fillColor] (136.92, 77.30) circle (  1.16);

\path[draw=drawColor,line width= 0.4pt,line join=round,line cap=round,fill=fillColor] (137.46, 77.20) circle (  1.16);

\path[draw=drawColor,line width= 0.4pt,line join=round,line cap=round,fill=fillColor] (137.99, 77.06) circle (  1.16);

\path[draw=drawColor,line width= 0.4pt,line join=round,line cap=round,fill=fillColor] (138.52, 77.04) circle (  1.16);

\path[draw=drawColor,line width= 0.4pt,line join=round,line cap=round,fill=fillColor] (139.04, 76.98) circle (  1.16);

\path[draw=drawColor,line width= 0.4pt,line join=round,line cap=round,fill=fillColor] (139.56, 76.90) circle (  1.16);

\path[draw=drawColor,line width= 0.4pt,line join=round,line cap=round,fill=fillColor] (140.06, 76.89) circle (  1.16);

\path[draw=drawColor,line width= 0.4pt,line join=round,line cap=round,fill=fillColor] (140.57, 76.87) circle (  1.16);

\path[draw=drawColor,line width= 0.4pt,line join=round,line cap=round,fill=fillColor] (141.06, 76.77) circle (  1.16);

\path[draw=drawColor,line width= 0.4pt,line join=round,line cap=round,fill=fillColor] (141.55, 76.74) circle (  1.16);

\path[draw=drawColor,line width= 0.4pt,line join=round,line cap=round,fill=fillColor] (142.04, 76.55) circle (  1.16);

\path[draw=drawColor,line width= 0.4pt,line join=round,line cap=round,fill=fillColor] (142.52, 76.52) circle (  1.16);

\path[draw=drawColor,line width= 0.4pt,line join=round,line cap=round,fill=fillColor] (143.00, 76.50) circle (  1.16);

\path[draw=drawColor,line width= 0.4pt,line join=round,line cap=round,fill=fillColor] (143.47, 76.46) circle (  1.16);

\path[draw=drawColor,line width= 0.4pt,line join=round,line cap=round,fill=fillColor] (143.93, 76.46) circle (  1.16);

\path[draw=drawColor,line width= 0.4pt,line join=round,line cap=round,fill=fillColor] (144.39, 76.41) circle (  1.16);

\path[draw=drawColor,line width= 0.4pt,line join=round,line cap=round,fill=fillColor] (144.85, 76.36) circle (  1.16);

\path[draw=drawColor,line width= 0.4pt,line join=round,line cap=round,fill=fillColor] (145.30, 76.32) circle (  1.16);

\path[draw=drawColor,line width= 0.4pt,line join=round,line cap=round,fill=fillColor] (145.75, 76.26) circle (  1.16);

\path[draw=drawColor,line width= 0.4pt,line join=round,line cap=round,fill=fillColor] (146.19, 76.16) circle (  1.16);

\path[draw=drawColor,line width= 0.4pt,line join=round,line cap=round,fill=fillColor] (146.63, 76.13) circle (  1.16);

\path[draw=drawColor,line width= 0.4pt,line join=round,line cap=round,fill=fillColor] (147.07, 76.11) circle (  1.16);

\path[draw=drawColor,line width= 0.4pt,line join=round,line cap=round,fill=fillColor] (147.50, 76.06) circle (  1.16);

\path[draw=drawColor,line width= 0.4pt,line join=round,line cap=round,fill=fillColor] (147.93, 75.99) circle (  1.16);

\path[draw=drawColor,line width= 0.4pt,line join=round,line cap=round,fill=fillColor] (148.35, 75.85) circle (  1.16);

\path[draw=drawColor,line width= 0.4pt,line join=round,line cap=round,fill=fillColor] (148.77, 75.72) circle (  1.16);

\path[draw=drawColor,line width= 0.4pt,line join=round,line cap=round,fill=fillColor] (149.19, 75.72) circle (  1.16);

\path[draw=drawColor,line width= 0.4pt,line join=round,line cap=round,fill=fillColor] (149.60, 75.67) circle (  1.16);

\path[draw=drawColor,line width= 0.4pt,line join=round,line cap=round,fill=fillColor] (150.01, 75.60) circle (  1.16);

\path[draw=drawColor,line width= 0.4pt,line join=round,line cap=round,fill=fillColor] (150.41, 75.54) circle (  1.16);

\path[draw=drawColor,line width= 0.4pt,line join=round,line cap=round,fill=fillColor] (150.82, 75.30) circle (  1.16);

\path[draw=drawColor,line width= 0.4pt,line join=round,line cap=round,fill=fillColor] (151.22, 75.29) circle (  1.16);

\path[draw=drawColor,line width= 0.4pt,line join=round,line cap=round,fill=fillColor] (151.61, 75.27) circle (  1.16);

\path[draw=drawColor,line width= 0.4pt,line join=round,line cap=round,fill=fillColor] (152.01, 75.20) circle (  1.16);

\path[draw=drawColor,line width= 0.4pt,line join=round,line cap=round,fill=fillColor] (152.40, 75.15) circle (  1.16);

\path[draw=drawColor,line width= 0.4pt,line join=round,line cap=round,fill=fillColor] (152.78, 75.13) circle (  1.16);

\path[draw=drawColor,line width= 0.4pt,line join=round,line cap=round,fill=fillColor] (153.17, 75.11) circle (  1.16);

\path[draw=drawColor,line width= 0.4pt,line join=round,line cap=round,fill=fillColor] (153.55, 75.10) circle (  1.16);

\path[draw=drawColor,line width= 0.4pt,line join=round,line cap=round,fill=fillColor] (153.93, 75.08) circle (  1.16);

\path[draw=drawColor,line width= 0.4pt,line join=round,line cap=round,fill=fillColor] (154.30, 75.05) circle (  1.16);

\path[draw=drawColor,line width= 0.4pt,line join=round,line cap=round,fill=fillColor] (154.67, 75.01) circle (  1.16);

\path[draw=drawColor,line width= 0.4pt,line join=round,line cap=round,fill=fillColor] (155.04, 74.98) circle (  1.16);

\path[draw=drawColor,line width= 0.4pt,line join=round,line cap=round,fill=fillColor] (155.41, 74.95) circle (  1.16);

\path[draw=drawColor,line width= 0.4pt,line join=round,line cap=round,fill=fillColor] (155.78, 74.83) circle (  1.16);

\path[draw=drawColor,line width= 0.4pt,line join=round,line cap=round,fill=fillColor] (156.14, 74.81) circle (  1.16);

\path[draw=drawColor,line width= 0.4pt,line join=round,line cap=round,fill=fillColor] (156.50, 74.75) circle (  1.16);

\path[draw=drawColor,line width= 0.4pt,line join=round,line cap=round,fill=fillColor] (156.86, 74.67) circle (  1.16);

\path[draw=drawColor,line width= 0.4pt,line join=round,line cap=round,fill=fillColor] (157.21, 74.51) circle (  1.16);

\path[draw=drawColor,line width= 0.4pt,line join=round,line cap=round,fill=fillColor] (157.56, 74.49) circle (  1.16);

\path[draw=drawColor,line width= 0.4pt,line join=round,line cap=round,fill=fillColor] (157.91, 74.27) circle (  1.16);

\path[draw=drawColor,line width= 0.4pt,line join=round,line cap=round,fill=fillColor] (158.26, 74.25) circle (  1.16);

\path[draw=drawColor,line width= 0.4pt,line join=round,line cap=round,fill=fillColor] (158.61, 74.22) circle (  1.16);

\path[draw=drawColor,line width= 0.4pt,line join=round,line cap=round,fill=fillColor] (158.95, 74.21) circle (  1.16);

\path[draw=drawColor,line width= 0.4pt,line join=round,line cap=round,fill=fillColor] (159.29, 74.13) circle (  1.16);

\path[draw=drawColor,line width= 0.4pt,line join=round,line cap=round,fill=fillColor] (159.63, 74.10) circle (  1.16);

\path[draw=drawColor,line width= 0.4pt,line join=round,line cap=round,fill=fillColor] (159.97, 74.06) circle (  1.16);

\path[draw=drawColor,line width= 0.4pt,line join=round,line cap=round,fill=fillColor] (160.30, 73.91) circle (  1.16);

\path[draw=drawColor,line width= 0.4pt,line join=round,line cap=round,fill=fillColor] (160.63, 73.86) circle (  1.16);

\path[draw=drawColor,line width= 0.4pt,line join=round,line cap=round,fill=fillColor] (160.96, 73.74) circle (  1.16);

\path[draw=drawColor,line width= 0.4pt,line join=round,line cap=round,fill=fillColor] (161.29, 73.71) circle (  1.16);

\path[draw=drawColor,line width= 0.4pt,line join=round,line cap=round,fill=fillColor] (161.62, 73.57) circle (  1.16);

\path[draw=drawColor,line width= 0.4pt,line join=round,line cap=round,fill=fillColor] (161.95, 73.53) circle (  1.16);

\path[draw=drawColor,line width= 0.4pt,line join=round,line cap=round,fill=fillColor] (162.27, 73.50) circle (  1.16);

\path[draw=drawColor,line width= 0.4pt,line join=round,line cap=round,fill=fillColor] (162.59, 73.44) circle (  1.16);

\path[draw=drawColor,line width= 0.4pt,line join=round,line cap=round,fill=fillColor] (162.91, 73.44) circle (  1.16);

\path[draw=drawColor,line width= 0.4pt,line join=round,line cap=round,fill=fillColor] (163.23, 73.43) circle (  1.16);

\path[draw=drawColor,line width= 0.4pt,line join=round,line cap=round,fill=fillColor] (163.54, 73.36) circle (  1.16);

\path[draw=drawColor,line width= 0.4pt,line join=round,line cap=round,fill=fillColor] (163.85, 73.36) circle (  1.16);

\path[draw=drawColor,line width= 0.4pt,line join=round,line cap=round,fill=fillColor] (164.17, 73.33) circle (  1.16);

\path[draw=drawColor,line width= 0.4pt,line join=round,line cap=round,fill=fillColor] (164.48, 73.30) circle (  1.16);

\path[draw=drawColor,line width= 0.4pt,line join=round,line cap=round,fill=fillColor] (164.79, 73.10) circle (  1.16);

\path[draw=drawColor,line width= 0.4pt,line join=round,line cap=round,fill=fillColor] (165.09, 73.08) circle (  1.16);

\path[draw=drawColor,line width= 0.4pt,line join=round,line cap=round,fill=fillColor] (165.40, 73.03) circle (  1.16);

\path[draw=drawColor,line width= 0.4pt,line join=round,line cap=round,fill=fillColor] (165.70, 73.00) circle (  1.16);

\path[draw=drawColor,line width= 0.4pt,line join=round,line cap=round,fill=fillColor] (166.00, 72.99) circle (  1.16);

\path[draw=drawColor,line width= 0.4pt,line join=round,line cap=round,fill=fillColor] (166.30, 72.94) circle (  1.16);

\path[draw=drawColor,line width= 0.4pt,line join=round,line cap=round,fill=fillColor] (166.60, 72.92) circle (  1.16);

\path[draw=drawColor,line width= 0.4pt,line join=round,line cap=round,fill=fillColor] (166.90, 72.92) circle (  1.16);

\path[draw=drawColor,line width= 0.4pt,line join=round,line cap=round,fill=fillColor] (167.19, 72.88) circle (  1.16);

\path[draw=drawColor,line width= 0.4pt,line join=round,line cap=round,fill=fillColor] (167.49, 72.86) circle (  1.16);

\path[draw=drawColor,line width= 0.4pt,line join=round,line cap=round,fill=fillColor] (167.78, 72.76) circle (  1.16);

\path[draw=drawColor,line width= 0.4pt,line join=round,line cap=round,fill=fillColor] (168.07, 72.69) circle (  1.16);

\path[draw=drawColor,line width= 0.4pt,line join=round,line cap=round,fill=fillColor] (168.36, 72.55) circle (  1.16);

\path[draw=drawColor,line width= 0.4pt,line join=round,line cap=round,fill=fillColor] (168.65, 72.51) circle (  1.16);

\path[draw=drawColor,line width= 0.4pt,line join=round,line cap=round,fill=fillColor] (168.94, 72.46) circle (  1.16);

\path[draw=drawColor,line width= 0.4pt,line join=round,line cap=round,fill=fillColor] (169.22, 72.36) circle (  1.16);

\path[draw=drawColor,line width= 0.4pt,line join=round,line cap=round,fill=fillColor] (169.51, 72.27) circle (  1.16);

\path[draw=drawColor,line width= 0.4pt,line join=round,line cap=round,fill=fillColor] (169.79, 72.21) circle (  1.16);

\path[draw=drawColor,line width= 0.4pt,line join=round,line cap=round,fill=fillColor] (170.07, 72.17) circle (  1.16);

\path[draw=drawColor,line width= 0.4pt,line join=round,line cap=round,fill=fillColor] (170.35, 72.11) circle (  1.16);

\path[draw=drawColor,line width= 0.4pt,line join=round,line cap=round,fill=fillColor] (170.63, 71.91) circle (  1.16);

\path[draw=drawColor,line width= 0.4pt,line join=round,line cap=round,fill=fillColor] (170.91, 71.87) circle (  1.16);

\path[draw=drawColor,line width= 0.4pt,line join=round,line cap=round,fill=fillColor] (171.18, 71.73) circle (  1.16);

\path[draw=drawColor,line width= 0.4pt,line join=round,line cap=round,fill=fillColor] (171.46, 71.69) circle (  1.16);

\path[draw=drawColor,line width= 0.4pt,line join=round,line cap=round,fill=fillColor] (171.73, 71.59) circle (  1.16);

\path[draw=drawColor,line width= 0.4pt,line join=round,line cap=round,fill=fillColor] (172.00, 71.46) circle (  1.16);

\path[draw=drawColor,line width= 0.4pt,line join=round,line cap=round,fill=fillColor] (172.28, 71.31) circle (  1.16);

\path[draw=drawColor,line width= 0.4pt,line join=round,line cap=round,fill=fillColor] (172.55, 71.31) circle (  1.16);

\path[draw=drawColor,line width= 0.4pt,line join=round,line cap=round,fill=fillColor] (172.81, 71.26) circle (  1.16);

\path[draw=drawColor,line width= 0.4pt,line join=round,line cap=round,fill=fillColor] (173.08, 71.25) circle (  1.16);

\path[draw=drawColor,line width= 0.4pt,line join=round,line cap=round,fill=fillColor] (173.35, 71.23) circle (  1.16);

\path[draw=drawColor,line width= 0.4pt,line join=round,line cap=round,fill=fillColor] (173.61, 71.11) circle (  1.16);

\path[draw=drawColor,line width= 0.4pt,line join=round,line cap=round,fill=fillColor] (173.88, 70.96) circle (  1.16);

\path[draw=drawColor,line width= 0.4pt,line join=round,line cap=round,fill=fillColor] (174.14, 70.89) circle (  1.16);

\path[draw=drawColor,line width= 0.4pt,line join=round,line cap=round,fill=fillColor] (174.40, 70.89) circle (  1.16);

\path[draw=drawColor,line width= 0.4pt,line join=round,line cap=round,fill=fillColor] (174.66, 70.87) circle (  1.16);

\path[draw=drawColor,line width= 0.4pt,line join=round,line cap=round,fill=fillColor] (174.92, 70.79) circle (  1.16);

\path[draw=drawColor,line width= 0.4pt,line join=round,line cap=round,fill=fillColor] (175.18, 70.79) circle (  1.16);

\path[draw=drawColor,line width= 0.4pt,line join=round,line cap=round,fill=fillColor] (175.44, 70.79) circle (  1.16);

\path[draw=drawColor,line width= 0.4pt,line join=round,line cap=round,fill=fillColor] (175.69, 70.69) circle (  1.16);

\path[draw=drawColor,line width= 0.4pt,line join=round,line cap=round,fill=fillColor] (175.95, 70.64) circle (  1.16);

\path[draw=drawColor,line width= 0.4pt,line join=round,line cap=round,fill=fillColor] (176.20, 70.57) circle (  1.16);

\path[draw=drawColor,line width= 0.4pt,line join=round,line cap=round,fill=fillColor] (176.46, 70.56) circle (  1.16);

\path[draw=drawColor,line width= 0.4pt,line join=round,line cap=round,fill=fillColor] (176.71, 70.39) circle (  1.16);

\path[draw=drawColor,line width= 0.4pt,line join=round,line cap=round,fill=fillColor] (176.96, 70.29) circle (  1.16);

\path[draw=drawColor,line width= 0.4pt,line join=round,line cap=round,fill=fillColor] (177.21, 70.15) circle (  1.16);

\path[draw=drawColor,line width= 0.4pt,line join=round,line cap=round,fill=fillColor] (177.46, 70.14) circle (  1.16);

\path[draw=drawColor,line width= 0.4pt,line join=round,line cap=round,fill=fillColor] (177.71, 70.03) circle (  1.16);

\path[draw=drawColor,line width= 0.4pt,line join=round,line cap=round,fill=fillColor] (177.95, 69.89) circle (  1.16);

\path[draw=drawColor,line width= 0.4pt,line join=round,line cap=round,fill=fillColor] (178.20, 69.85) circle (  1.16);

\path[draw=drawColor,line width= 0.4pt,line join=round,line cap=round,fill=fillColor] (178.45, 69.83) circle (  1.16);

\path[draw=drawColor,line width= 0.4pt,line join=round,line cap=round,fill=fillColor] (178.69, 69.82) circle (  1.16);

\path[draw=drawColor,line width= 0.4pt,line join=round,line cap=round,fill=fillColor] (178.93, 69.81) circle (  1.16);

\path[draw=drawColor,line width= 0.4pt,line join=round,line cap=round,fill=fillColor] (179.18, 69.77) circle (  1.16);

\path[draw=drawColor,line width= 0.4pt,line join=round,line cap=round,fill=fillColor] (179.42, 69.75) circle (  1.16);

\path[draw=drawColor,line width= 0.4pt,line join=round,line cap=round,fill=fillColor] (179.66, 69.74) circle (  1.16);

\path[draw=drawColor,line width= 0.4pt,line join=round,line cap=round,fill=fillColor] (179.90, 69.72) circle (  1.16);

\path[draw=drawColor,line width= 0.4pt,line join=round,line cap=round,fill=fillColor] (180.14, 69.69) circle (  1.16);

\path[draw=drawColor,line width= 0.4pt,line join=round,line cap=round,fill=fillColor] (180.37, 69.65) circle (  1.16);

\path[draw=drawColor,line width= 0.4pt,line join=round,line cap=round,fill=fillColor] (180.61, 69.63) circle (  1.16);

\path[draw=drawColor,line width= 0.4pt,line join=round,line cap=round,fill=fillColor] (180.85, 69.51) circle (  1.16);

\path[draw=drawColor,line width= 0.4pt,line join=round,line cap=round,fill=fillColor] (181.08, 69.36) circle (  1.16);

\path[draw=drawColor,line width= 0.4pt,line join=round,line cap=round,fill=fillColor] (181.32, 69.29) circle (  1.16);

\path[draw=drawColor,line width= 0.4pt,line join=round,line cap=round,fill=fillColor] (181.55, 69.26) circle (  1.16);

\path[draw=drawColor,line width= 0.4pt,line join=round,line cap=round,fill=fillColor] (181.78, 69.15) circle (  1.16);

\path[draw=drawColor,line width= 0.4pt,line join=round,line cap=round,fill=fillColor] (182.01, 69.13) circle (  1.16);

\path[draw=drawColor,line width= 0.4pt,line join=round,line cap=round,fill=fillColor] (182.25, 69.12) circle (  1.16);

\path[draw=drawColor,line width= 0.4pt,line join=round,line cap=round,fill=fillColor] (182.48, 69.11) circle (  1.16);

\path[draw=drawColor,line width= 0.4pt,line join=round,line cap=round,fill=fillColor] (182.70, 69.06) circle (  1.16);

\path[draw=drawColor,line width= 0.4pt,line join=round,line cap=round,fill=fillColor] (182.93, 69.04) circle (  1.16);

\path[draw=drawColor,line width= 0.4pt,line join=round,line cap=round,fill=fillColor] (183.16, 68.99) circle (  1.16);

\path[draw=drawColor,line width= 0.4pt,line join=round,line cap=round,fill=fillColor] (183.39, 68.96) circle (  1.16);

\path[draw=drawColor,line width= 0.4pt,line join=round,line cap=round,fill=fillColor] (183.61, 68.93) circle (  1.16);

\path[draw=drawColor,line width= 0.4pt,line join=round,line cap=round,fill=fillColor] (183.84, 68.91) circle (  1.16);

\path[draw=drawColor,line width= 0.4pt,line join=round,line cap=round,fill=fillColor] (184.06, 68.91) circle (  1.16);

\path[draw=drawColor,line width= 0.4pt,line join=round,line cap=round,fill=fillColor] (184.29, 68.91) circle (  1.16);

\path[draw=drawColor,line width= 0.4pt,line join=round,line cap=round,fill=fillColor] (184.51, 68.91) circle (  1.16);

\path[draw=drawColor,line width= 0.4pt,line join=round,line cap=round,fill=fillColor] (184.73, 68.85) circle (  1.16);

\path[draw=drawColor,line width= 0.4pt,line join=round,line cap=round,fill=fillColor] (184.96, 68.85) circle (  1.16);

\path[draw=drawColor,line width= 0.4pt,line join=round,line cap=round,fill=fillColor] (185.18, 68.82) circle (  1.16);

\path[draw=drawColor,line width= 0.4pt,line join=round,line cap=round,fill=fillColor] (185.40, 68.78) circle (  1.16);

\path[draw=drawColor,line width= 0.4pt,line join=round,line cap=round,fill=fillColor] (185.62, 68.75) circle (  1.16);

\path[draw=drawColor,line width= 0.4pt,line join=round,line cap=round,fill=fillColor] (185.84, 68.69) circle (  1.16);

\path[draw=drawColor,line width= 0.4pt,line join=round,line cap=round,fill=fillColor] (186.05, 68.64) circle (  1.16);

\path[draw=drawColor,line width= 0.4pt,line join=round,line cap=round,fill=fillColor] (186.27, 68.61) circle (  1.16);

\path[draw=drawColor,line width= 0.4pt,line join=round,line cap=round,fill=fillColor] (186.49, 68.61) circle (  1.16);

\path[draw=drawColor,line width= 0.4pt,line join=round,line cap=round,fill=fillColor] (186.70, 68.42) circle (  1.16);

\path[draw=drawColor,line width= 0.4pt,line join=round,line cap=round,fill=fillColor] (186.92, 68.42) circle (  1.16);

\path[draw=drawColor,line width= 0.4pt,line join=round,line cap=round,fill=fillColor] (187.13, 68.40) circle (  1.16);

\path[draw=drawColor,line width= 0.4pt,line join=round,line cap=round,fill=fillColor] (187.35, 68.32) circle (  1.16);

\path[draw=drawColor,line width= 0.4pt,line join=round,line cap=round,fill=fillColor] (187.56, 68.32) circle (  1.16);

\path[draw=drawColor,line width= 0.4pt,line join=round,line cap=round,fill=fillColor] (187.77, 68.25) circle (  1.16);

\path[draw=drawColor,line width= 0.4pt,line join=round,line cap=round,fill=fillColor] (187.98, 68.22) circle (  1.16);

\path[draw=drawColor,line width= 0.4pt,line join=round,line cap=round,fill=fillColor] (188.20, 68.22) circle (  1.16);

\path[draw=drawColor,line width= 0.4pt,line join=round,line cap=round,fill=fillColor] (188.41, 68.20) circle (  1.16);

\path[draw=drawColor,line width= 0.4pt,line join=round,line cap=round,fill=fillColor] (188.62, 68.07) circle (  1.16);

\path[draw=drawColor,line width= 0.4pt,line join=round,line cap=round,fill=fillColor] (188.83, 68.02) circle (  1.16);

\path[draw=drawColor,line width= 0.4pt,line join=round,line cap=round,fill=fillColor] (189.03, 68.02) circle (  1.16);

\path[draw=drawColor,line width= 0.4pt,line join=round,line cap=round,fill=fillColor] (189.24, 67.98) circle (  1.16);

\path[draw=drawColor,line width= 0.4pt,line join=round,line cap=round,fill=fillColor] (189.45, 67.95) circle (  1.16);

\path[draw=drawColor,line width= 0.4pt,line join=round,line cap=round,fill=fillColor] (189.66, 67.84) circle (  1.16);

\path[draw=drawColor,line width= 0.4pt,line join=round,line cap=round,fill=fillColor] (189.86, 67.83) circle (  1.16);

\path[draw=drawColor,line width= 0.4pt,line join=round,line cap=round,fill=fillColor] (190.07, 67.67) circle (  1.16);

\path[draw=drawColor,line width= 0.4pt,line join=round,line cap=round,fill=fillColor] (190.27, 67.67) circle (  1.16);

\path[draw=drawColor,line width= 0.4pt,line join=round,line cap=round,fill=fillColor] (190.48, 67.61) circle (  1.16);

\path[draw=drawColor,line width= 0.4pt,line join=round,line cap=round,fill=fillColor] (190.68, 67.60) circle (  1.16);

\path[draw=drawColor,line width= 0.4pt,line join=round,line cap=round,fill=fillColor] (190.88, 67.55) circle (  1.16);

\path[draw=drawColor,line width= 0.4pt,line join=round,line cap=round,fill=fillColor] (191.09, 67.52) circle (  1.16);

\path[draw=drawColor,line width= 0.4pt,line join=round,line cap=round,fill=fillColor] (191.29, 67.46) circle (  1.16);

\path[draw=drawColor,line width= 0.4pt,line join=round,line cap=round,fill=fillColor] (191.49, 67.45) circle (  1.16);

\path[draw=drawColor,line width= 0.4pt,line join=round,line cap=round,fill=fillColor] (191.69, 67.43) circle (  1.16);

\path[draw=drawColor,line width= 0.4pt,line join=round,line cap=round,fill=fillColor] (191.89, 67.42) circle (  1.16);

\path[draw=drawColor,line width= 0.4pt,line join=round,line cap=round,fill=fillColor] (192.09, 67.39) circle (  1.16);

\path[draw=drawColor,line width= 0.4pt,line join=round,line cap=round,fill=fillColor] (192.29, 67.32) circle (  1.16);

\path[draw=drawColor,line width= 0.4pt,line join=round,line cap=round,fill=fillColor] (192.49, 67.32) circle (  1.16);

\path[draw=drawColor,line width= 0.4pt,line join=round,line cap=round,fill=fillColor] (192.69, 67.27) circle (  1.16);

\path[draw=drawColor,line width= 0.4pt,line join=round,line cap=round,fill=fillColor] (192.88, 67.25) circle (  1.16);

\path[draw=drawColor,line width= 0.4pt,line join=round,line cap=round,fill=fillColor] (193.08, 67.21) circle (  1.16);

\path[draw=drawColor,line width= 0.4pt,line join=round,line cap=round,fill=fillColor] (193.28, 67.20) circle (  1.16);

\path[draw=drawColor,line width= 0.4pt,line join=round,line cap=round,fill=fillColor] (193.47, 67.18) circle (  1.16);

\path[draw=drawColor,line width= 0.4pt,line join=round,line cap=round,fill=fillColor] (193.67, 67.13) circle (  1.16);

\path[draw=drawColor,line width= 0.4pt,line join=round,line cap=round,fill=fillColor] (193.86, 67.05) circle (  1.16);

\path[draw=drawColor,line width= 0.4pt,line join=round,line cap=round,fill=fillColor] (194.06, 67.05) circle (  1.16);

\path[draw=drawColor,line width= 0.4pt,line join=round,line cap=round,fill=fillColor] (194.25, 66.99) circle (  1.16);

\path[draw=drawColor,line width= 0.4pt,line join=round,line cap=round,fill=fillColor] (194.44, 66.94) circle (  1.16);

\path[draw=drawColor,line width= 0.4pt,line join=round,line cap=round,fill=fillColor] (194.63, 66.93) circle (  1.16);

\path[draw=drawColor,line width= 0.4pt,line join=round,line cap=round,fill=fillColor] (194.83, 66.90) circle (  1.16);

\path[draw=drawColor,line width= 0.4pt,line join=round,line cap=round,fill=fillColor] (195.02, 66.87) circle (  1.16);

\path[draw=drawColor,line width= 0.4pt,line join=round,line cap=round,fill=fillColor] (195.21, 66.84) circle (  1.16);

\path[draw=drawColor,line width= 0.4pt,line join=round,line cap=round,fill=fillColor] (195.40, 66.82) circle (  1.16);

\path[draw=drawColor,line width= 0.4pt,line join=round,line cap=round,fill=fillColor] (195.59, 66.82) circle (  1.16);

\path[draw=drawColor,line width= 0.4pt,line join=round,line cap=round,fill=fillColor] (195.78, 66.62) circle (  1.16);

\path[draw=drawColor,line width= 0.4pt,line join=round,line cap=round,fill=fillColor] (195.97, 66.57) circle (  1.16);

\path[draw=drawColor,line width= 0.4pt,line join=round,line cap=round,fill=fillColor] (196.16, 66.53) circle (  1.16);

\path[draw=drawColor,line width= 0.4pt,line join=round,line cap=round,fill=fillColor] (196.34, 66.51) circle (  1.16);

\path[draw=drawColor,line width= 0.4pt,line join=round,line cap=round,fill=fillColor] (196.53, 66.42) circle (  1.16);

\path[draw=drawColor,line width= 0.4pt,line join=round,line cap=round,fill=fillColor] (196.72, 66.40) circle (  1.16);

\path[draw=drawColor,line width= 0.4pt,line join=round,line cap=round,fill=fillColor] (196.91, 66.32) circle (  1.16);

\path[draw=drawColor,line width= 0.4pt,line join=round,line cap=round,fill=fillColor] (197.09, 66.26) circle (  1.16);

\path[draw=drawColor,line width= 0.4pt,line join=round,line cap=round,fill=fillColor] (197.28, 66.26) circle (  1.16);

\path[draw=drawColor,line width= 0.4pt,line join=round,line cap=round,fill=fillColor] (197.46, 66.22) circle (  1.16);

\path[draw=drawColor,line width= 0.4pt,line join=round,line cap=round,fill=fillColor] (197.65, 66.22) circle (  1.16);

\path[draw=drawColor,line width= 0.4pt,line join=round,line cap=round,fill=fillColor] (197.83, 66.09) circle (  1.16);

\path[draw=drawColor,line width= 0.4pt,line join=round,line cap=round,fill=fillColor] (198.01, 66.08) circle (  1.16);

\path[draw=drawColor,line width= 0.4pt,line join=round,line cap=round,fill=fillColor] (198.20, 66.02) circle (  1.16);

\path[draw=drawColor,line width= 0.4pt,line join=round,line cap=round,fill=fillColor] (198.38, 65.95) circle (  1.16);

\path[draw=drawColor,line width= 0.4pt,line join=round,line cap=round,fill=fillColor] (198.56, 65.95) circle (  1.16);

\path[draw=drawColor,line width= 0.4pt,line join=round,line cap=round,fill=fillColor] (198.74, 65.86) circle (  1.16);

\path[draw=drawColor,line width= 0.4pt,line join=round,line cap=round,fill=fillColor] (198.93, 65.84) circle (  1.16);

\path[draw=drawColor,line width= 0.4pt,line join=round,line cap=round,fill=fillColor] (199.11, 65.77) circle (  1.16);

\path[draw=drawColor,line width= 0.4pt,line join=round,line cap=round,fill=fillColor] (199.29, 65.74) circle (  1.16);

\path[draw=drawColor,line width= 0.4pt,line join=round,line cap=round,fill=fillColor] (199.47, 65.71) circle (  1.16);

\path[draw=drawColor,line width= 0.4pt,line join=round,line cap=round,fill=fillColor] (199.65, 65.67) circle (  1.16);

\path[draw=drawColor,line width= 0.4pt,line join=round,line cap=round,fill=fillColor] (199.83, 65.53) circle (  1.16);

\path[draw=drawColor,line width= 0.4pt,line join=round,line cap=round,fill=fillColor] (200.00, 65.52) circle (  1.16);

\path[draw=drawColor,line width= 0.4pt,line join=round,line cap=round,fill=fillColor] (200.18, 65.39) circle (  1.16);

\path[draw=drawColor,line width= 0.4pt,line join=round,line cap=round,fill=fillColor] (200.36, 65.31) circle (  1.16);

\path[draw=drawColor,line width= 0.4pt,line join=round,line cap=round,fill=fillColor] (200.54, 65.30) circle (  1.16);

\path[draw=drawColor,line width= 0.4pt,line join=round,line cap=round,fill=fillColor] (200.72, 65.22) circle (  1.16);

\path[draw=drawColor,line width= 0.4pt,line join=round,line cap=round,fill=fillColor] (200.89, 65.21) circle (  1.16);

\path[draw=drawColor,line width= 0.4pt,line join=round,line cap=round,fill=fillColor] (201.07, 65.21) circle (  1.16);

\path[draw=drawColor,line width= 0.4pt,line join=round,line cap=round,fill=fillColor] (201.24, 65.20) circle (  1.16);

\path[draw=drawColor,line width= 0.4pt,line join=round,line cap=round,fill=fillColor] (201.42, 65.12) circle (  1.16);

\path[draw=drawColor,line width= 0.4pt,line join=round,line cap=round,fill=fillColor] (201.59, 65.10) circle (  1.16);

\path[draw=drawColor,line width= 0.4pt,line join=round,line cap=round,fill=fillColor] (201.77, 65.07) circle (  1.16);

\path[draw=drawColor,line width= 0.4pt,line join=round,line cap=round,fill=fillColor] (201.94, 64.89) circle (  1.16);

\path[draw=drawColor,line width= 0.4pt,line join=round,line cap=round,fill=fillColor] (202.12, 64.88) circle (  1.16);

\path[draw=drawColor,line width= 0.4pt,line join=round,line cap=round,fill=fillColor] (202.29, 64.82) circle (  1.16);

\path[draw=drawColor,line width= 0.4pt,line join=round,line cap=round,fill=fillColor] (202.46, 64.78) circle (  1.16);

\path[draw=drawColor,line width= 0.4pt,line join=round,line cap=round,fill=fillColor] (202.64, 64.72) circle (  1.16);

\path[draw=drawColor,line width= 0.4pt,line join=round,line cap=round,fill=fillColor] (202.81, 64.64) circle (  1.16);

\path[draw=drawColor,line width= 0.4pt,line join=round,line cap=round,fill=fillColor] (202.98, 64.58) circle (  1.16);

\path[draw=drawColor,line width= 0.4pt,line join=round,line cap=round,fill=fillColor] (203.15, 64.49) circle (  1.16);

\path[draw=drawColor,line width= 0.4pt,line join=round,line cap=round,fill=fillColor] (203.32, 64.27) circle (  1.16);

\path[draw=drawColor,line width= 0.4pt,line join=round,line cap=round,fill=fillColor] (203.49, 64.17) circle (  1.16);

\path[draw=drawColor,line width= 0.4pt,line join=round,line cap=round,fill=fillColor] (203.66, 64.16) circle (  1.16);

\path[draw=drawColor,line width= 0.4pt,line join=round,line cap=round,fill=fillColor] (203.83, 64.15) circle (  1.16);

\path[draw=drawColor,line width= 0.4pt,line join=round,line cap=round,fill=fillColor] (204.00, 64.13) circle (  1.16);

\path[draw=drawColor,line width= 0.4pt,line join=round,line cap=round,fill=fillColor] (204.17, 64.11) circle (  1.16);

\path[draw=drawColor,line width= 0.4pt,line join=round,line cap=round,fill=fillColor] (204.34, 64.06) circle (  1.16);

\path[draw=drawColor,line width= 0.4pt,line join=round,line cap=round,fill=fillColor] (204.51, 64.04) circle (  1.16);

\path[draw=drawColor,line width= 0.4pt,line join=round,line cap=round,fill=fillColor] (204.68, 64.04) circle (  1.16);

\path[draw=drawColor,line width= 0.4pt,line join=round,line cap=round,fill=fillColor] (204.84, 64.04) circle (  1.16);

\path[draw=drawColor,line width= 0.4pt,line join=round,line cap=round,fill=fillColor] (205.01, 64.02) circle (  1.16);

\path[draw=drawColor,line width= 0.4pt,line join=round,line cap=round,fill=fillColor] (205.18, 63.96) circle (  1.16);

\path[draw=drawColor,line width= 0.4pt,line join=round,line cap=round,fill=fillColor] (205.34, 63.95) circle (  1.16);

\path[draw=drawColor,line width= 0.4pt,line join=round,line cap=round,fill=fillColor] (205.51, 63.87) circle (  1.16);

\path[draw=drawColor,line width= 0.4pt,line join=round,line cap=round,fill=fillColor] (205.68, 63.67) circle (  1.16);

\path[draw=drawColor,line width= 0.4pt,line join=round,line cap=round,fill=fillColor] (205.84, 63.63) circle (  1.16);

\path[draw=drawColor,line width= 0.4pt,line join=round,line cap=round,fill=fillColor] (206.01, 63.63) circle (  1.16);

\path[draw=drawColor,line width= 0.4pt,line join=round,line cap=round,fill=fillColor] (206.17, 63.62) circle (  1.16);

\path[draw=drawColor,line width= 0.4pt,line join=round,line cap=round,fill=fillColor] (206.34, 63.57) circle (  1.16);

\path[draw=drawColor,line width= 0.4pt,line join=round,line cap=round,fill=fillColor] (206.50, 63.45) circle (  1.16);

\path[draw=drawColor,line width= 0.4pt,line join=round,line cap=round,fill=fillColor] (206.66, 63.40) circle (  1.16);

\path[draw=drawColor,line width= 0.4pt,line join=round,line cap=round,fill=fillColor] (206.83, 63.38) circle (  1.16);

\path[draw=drawColor,line width= 0.4pt,line join=round,line cap=round,fill=fillColor] (206.99, 63.33) circle (  1.16);

\path[draw=drawColor,line width= 0.4pt,line join=round,line cap=round,fill=fillColor] (207.15, 63.30) circle (  1.16);

\path[draw=drawColor,line width= 0.4pt,line join=round,line cap=round,fill=fillColor] (207.31, 63.24) circle (  1.16);

\path[draw=drawColor,line width= 0.4pt,line join=round,line cap=round,fill=fillColor] (207.48, 62.98) circle (  1.16);

\path[draw=drawColor,line width= 0.4pt,line join=round,line cap=round,fill=fillColor] (207.64, 62.79) circle (  1.16);

\path[draw=drawColor,line width= 0.4pt,line join=round,line cap=round,fill=fillColor] (207.80, 62.74) circle (  1.16);

\path[draw=drawColor,line width= 0.4pt,line join=round,line cap=round,fill=fillColor] (207.96, 62.67) circle (  1.16);

\path[draw=drawColor,line width= 0.4pt,line join=round,line cap=round,fill=fillColor] (208.12, 62.65) circle (  1.16);

\path[draw=drawColor,line width= 0.4pt,line join=round,line cap=round,fill=fillColor] (208.28, 62.55) circle (  1.16);

\path[draw=drawColor,line width= 0.4pt,line join=round,line cap=round,fill=fillColor] (208.44, 62.46) circle (  1.16);

\path[draw=drawColor,line width= 0.4pt,line join=round,line cap=round,fill=fillColor] (208.60, 62.43) circle (  1.16);

\path[draw=drawColor,line width= 0.4pt,line join=round,line cap=round,fill=fillColor] (208.76, 62.43) circle (  1.16);

\path[draw=drawColor,line width= 0.4pt,line join=round,line cap=round,fill=fillColor] (208.92, 62.25) circle (  1.16);

\path[draw=drawColor,line width= 0.4pt,line join=round,line cap=round,fill=fillColor] (209.08, 62.23) circle (  1.16);

\path[draw=drawColor,line width= 0.4pt,line join=round,line cap=round,fill=fillColor] (209.24, 62.22) circle (  1.16);

\path[draw=drawColor,line width= 0.4pt,line join=round,line cap=round,fill=fillColor] (209.39, 62.17) circle (  1.16);

\path[draw=drawColor,line width= 0.4pt,line join=round,line cap=round,fill=fillColor] (209.55, 62.11) circle (  1.16);

\path[draw=drawColor,line width= 0.4pt,line join=round,line cap=round,fill=fillColor] (209.71, 62.08) circle (  1.16);

\path[draw=drawColor,line width= 0.4pt,line join=round,line cap=round,fill=fillColor] (209.87, 62.06) circle (  1.16);

\path[draw=drawColor,line width= 0.4pt,line join=round,line cap=round,fill=fillColor] (210.02, 62.04) circle (  1.16);

\path[draw=drawColor,line width= 0.4pt,line join=round,line cap=round,fill=fillColor] (210.18, 62.02) circle (  1.16);

\path[draw=drawColor,line width= 0.4pt,line join=round,line cap=round,fill=fillColor] (210.34, 61.81) circle (  1.16);

\path[draw=drawColor,line width= 0.4pt,line join=round,line cap=round,fill=fillColor] (210.49, 61.81) circle (  1.16);

\path[draw=drawColor,line width= 0.4pt,line join=round,line cap=round,fill=fillColor] (210.65, 61.77) circle (  1.16);

\path[draw=drawColor,line width= 0.4pt,line join=round,line cap=round,fill=fillColor] (210.80, 61.71) circle (  1.16);

\path[draw=drawColor,line width= 0.4pt,line join=round,line cap=round,fill=fillColor] (210.96, 61.61) circle (  1.16);

\path[draw=drawColor,line width= 0.4pt,line join=round,line cap=round,fill=fillColor] (211.11, 61.61) circle (  1.16);

\path[draw=drawColor,line width= 0.4pt,line join=round,line cap=round,fill=fillColor] (211.27, 61.60) circle (  1.16);

\path[draw=drawColor,line width= 0.4pt,line join=round,line cap=round,fill=fillColor] (211.42, 61.57) circle (  1.16);

\path[draw=drawColor,line width= 0.4pt,line join=round,line cap=round,fill=fillColor] (211.57, 61.55) circle (  1.16);

\path[draw=drawColor,line width= 0.4pt,line join=round,line cap=round,fill=fillColor] (211.73, 61.47) circle (  1.16);

\path[draw=drawColor,line width= 0.4pt,line join=round,line cap=round,fill=fillColor] (211.88, 61.40) circle (  1.16);

\path[draw=drawColor,line width= 0.4pt,line join=round,line cap=round,fill=fillColor] (212.03, 61.35) circle (  1.16);

\path[draw=drawColor,line width= 0.4pt,line join=round,line cap=round,fill=fillColor] (212.19, 61.03) circle (  1.16);

\path[draw=drawColor,line width= 0.4pt,line join=round,line cap=round,fill=fillColor] (212.34, 61.01) circle (  1.16);

\path[draw=drawColor,line width= 0.4pt,line join=round,line cap=round,fill=fillColor] (212.49, 60.83) circle (  1.16);

\path[draw=drawColor,line width= 0.4pt,line join=round,line cap=round,fill=fillColor] (212.64, 60.65) circle (  1.16);

\path[draw=drawColor,line width= 0.4pt,line join=round,line cap=round,fill=fillColor] (212.79, 60.44) circle (  1.16);

\path[draw=drawColor,line width= 0.4pt,line join=round,line cap=round,fill=fillColor] (212.94, 60.28) circle (  1.16);

\path[draw=drawColor,line width= 0.4pt,line join=round,line cap=round,fill=fillColor] (213.09, 60.03) circle (  1.16);

\path[draw=drawColor,line width= 0.4pt,line join=round,line cap=round,fill=fillColor] (213.25, 60.02) circle (  1.16);

\path[draw=drawColor,line width= 0.4pt,line join=round,line cap=round,fill=fillColor] (213.40, 59.89) circle (  1.16);

\path[draw=drawColor,line width= 0.4pt,line join=round,line cap=round,fill=fillColor] (213.55, 59.85) circle (  1.16);

\path[draw=drawColor,line width= 0.4pt,line join=round,line cap=round,fill=fillColor] (213.70, 59.04) circle (  1.16);

\path[draw=drawColor,line width= 0.4pt,line join=round,line cap=round,fill=fillColor] (213.84, 58.76) circle (  1.16);

\path[draw=drawColor,line width= 0.4pt,line join=round,line cap=round,fill=fillColor] (213.99, 58.58) circle (  1.16);

\path[draw=drawColor,line width= 0.4pt,line join=round,line cap=round,fill=fillColor] (214.14, 58.35) circle (  1.16);

\path[draw=drawColor,line width= 0.4pt,line join=round,line cap=round,fill=fillColor] (214.29, 58.32) circle (  1.16);

\path[draw=drawColor,line width= 0.4pt,line join=round,line cap=round,fill=fillColor] (214.44, 58.21) circle (  1.16);

\path[draw=drawColor,line width= 0.4pt,line join=round,line cap=round,fill=fillColor] (214.59, 58.01) circle (  1.16);

\path[draw=drawColor,line width= 0.4pt,line join=round,line cap=round,fill=fillColor] (214.74, 57.97) circle (  1.16);

\path[draw=drawColor,line width= 0.4pt,line join=round,line cap=round,fill=fillColor] (214.88, 57.33) circle (  1.16);

\path[draw=drawColor,line width= 0.4pt,line join=round,line cap=round,fill=fillColor] (215.03, 57.07) circle (  1.16);

\path[draw=drawColor,line width= 0.4pt,line join=round,line cap=round,fill=fillColor] (215.18, 56.76) circle (  1.16);

\path[draw=drawColor,line width= 0.4pt,line join=round,line cap=round,fill=fillColor] (215.32, 56.51) circle (  1.16);

\path[draw=drawColor,line width= 0.4pt,line join=round,line cap=round,fill=fillColor] (215.47, 56.26) circle (  1.16);

\path[draw=drawColor,line width= 0.4pt,line join=round,line cap=round,fill=fillColor] (215.62, 55.94) circle (  1.16);

\path[draw=drawColor,line width= 0.4pt,line join=round,line cap=round,fill=fillColor] (215.76, 55.92) circle (  1.16);

\path[draw=drawColor,line width= 0.4pt,line join=round,line cap=round,fill=fillColor] (215.91, 55.77) circle (  1.16);

\path[draw=drawColor,line width= 0.4pt,line join=round,line cap=round,fill=fillColor] (216.05, 55.59) circle (  1.16);

\path[draw=drawColor,line width= 0.4pt,line join=round,line cap=round,fill=fillColor] (216.20, 55.32) circle (  1.16);

\path[draw=drawColor,line width= 0.4pt,line join=round,line cap=round,fill=fillColor] (216.34, 54.45) circle (  1.16);

\path[draw=drawColor,line width= 0.4pt,line join=round,line cap=round,fill=fillColor] (216.49, 54.39) circle (  1.16);

\path[draw=drawColor,line width= 0.4pt,line join=round,line cap=round,fill=fillColor] (216.63, 54.11) circle (  1.16);

\path[draw=drawColor,line width= 0.4pt,line join=round,line cap=round,fill=fillColor] (216.78, 53.72) circle (  1.16);

\path[draw=drawColor,line width= 0.4pt,line join=round,line cap=round,fill=fillColor] (216.92, 53.66) circle (  1.16);

\path[draw=drawColor,line width= 0.4pt,line join=round,line cap=round,fill=fillColor] (217.06, 53.43) circle (  1.16);

\path[draw=drawColor,line width= 0.4pt,line join=round,line cap=round,fill=fillColor] (217.21, 52.77) circle (  1.16);

\path[draw=drawColor,line width= 0.4pt,line join=round,line cap=round,fill=fillColor] (217.35, 51.02) circle (  1.16);

\path[draw=drawColor,line width= 0.4pt,line join=round,line cap=round,fill=fillColor] (217.49, 47.20) circle (  1.16);

\path[draw=drawColor,line width= 0.4pt,line join=round,line cap=round,fill=fillColor] (217.64, 47.20) circle (  1.16);

\path[draw=drawColor,line width= 0.4pt,line join=round,line cap=round,fill=fillColor] (217.78, 47.20) circle (  1.16);

\path[draw=drawColor,line width= 0.4pt,line join=round,line cap=round,fill=fillColor] (217.92, 47.20) circle (  1.16);

\path[draw=drawColor,line width= 0.4pt,line join=round,line cap=round,fill=fillColor] (218.06, 47.20) circle (  1.16);

\path[draw=drawColor,line width= 0.4pt,line join=round,line cap=round,fill=fillColor] (218.20, 47.20) circle (  1.16);

\path[draw=drawColor,line width= 0.4pt,line join=round,line cap=round,fill=fillColor] (218.35, 47.20) circle (  1.16);

\path[draw=drawColor,line width= 0.4pt,line join=round,line cap=round,fill=fillColor] (218.49, 47.20) circle (  1.16);

\path[draw=drawColor,line width= 0.4pt,line join=round,line cap=round,fill=fillColor] (218.63, 47.20) circle (  1.16);

\path[draw=drawColor,line width= 0.4pt,line join=round,line cap=round,fill=fillColor] (218.77, 47.20) circle (  1.16);

\path[draw=drawColor,line width= 0.4pt,line join=round,line cap=round,fill=fillColor] (218.91, 47.20) circle (  1.16);

\path[draw=drawColor,line width= 0.4pt,line join=round,line cap=round,fill=fillColor] (219.05, 47.20) circle (  1.16);

\path[draw=drawColor,line width= 0.4pt,line join=round,line cap=round,fill=fillColor] (219.19, 47.20) circle (  1.16);

\path[draw=drawColor,line width= 0.4pt,line join=round,line cap=round,fill=fillColor] (219.33, 47.20) circle (  1.16);

\path[draw=drawColor,line width= 0.4pt,line join=round,line cap=round,fill=fillColor] (219.47, 47.20) circle (  1.16);

\path[draw=drawColor,line width= 0.4pt,line join=round,line cap=round,fill=fillColor] (219.61, 47.20) circle (  1.16);

\path[draw=drawColor,line width= 0.4pt,line join=round,line cap=round,fill=fillColor] (219.75, 47.20) circle (  1.16);

\path[draw=drawColor,line width= 0.4pt,line join=round,line cap=round,fill=fillColor] (219.89, 47.20) circle (  1.16);

\path[draw=drawColor,line width= 0.4pt,line join=round,line cap=round,fill=fillColor] (220.02, 47.20) circle (  1.16);

\path[draw=drawColor,line width= 0.4pt,line join=round,line cap=round,fill=fillColor] (220.16, 47.20) circle (  1.16);

\path[draw=drawColor,line width= 0.4pt,line join=round,line cap=round,fill=fillColor] (220.30, 47.20) circle (  1.16);

\path[draw=drawColor,line width= 0.4pt,line join=round,line cap=round,fill=fillColor] (220.44, 47.20) circle (  1.16);

\path[draw=drawColor,line width= 0.4pt,line join=round,line cap=round,fill=fillColor] (220.58, 47.20) circle (  1.16);

\path[draw=drawColor,line width= 0.4pt,line join=round,line cap=round,fill=fillColor] (220.71, 47.20) circle (  1.16);

\path[draw=drawColor,line width= 0.4pt,line join=round,line cap=round,fill=fillColor] (220.85, 47.20) circle (  1.16);

\path[draw=drawColor,line width= 0.4pt,line join=round,line cap=round,fill=fillColor] (220.99, 47.20) circle (  1.16);

\path[draw=drawColor,line width= 0.4pt,line join=round,line cap=round,fill=fillColor] (221.13, 47.20) circle (  1.16);

\path[draw=drawColor,line width= 0.4pt,line join=round,line cap=round,fill=fillColor] (221.26, 47.20) circle (  1.16);

\path[draw=drawColor,line width= 0.4pt,line join=round,line cap=round,fill=fillColor] (221.40, 47.20) circle (  1.16);

\path[draw=drawColor,line width= 0.4pt,line join=round,line cap=round,fill=fillColor] (221.53, 47.20) circle (  1.16);

\path[draw=drawColor,line width= 0.4pt,line join=round,line cap=round,fill=fillColor] (221.67, 47.20) circle (  1.16);

\path[draw=drawColor,line width= 0.4pt,line join=round,line cap=round,fill=fillColor] (221.81, 47.20) circle (  1.16);

\path[draw=drawColor,line width= 0.4pt,line join=round,line cap=round,fill=fillColor] (221.94, 47.20) circle (  1.16);

\path[draw=drawColor,line width= 0.4pt,line join=round,line cap=round,fill=fillColor] (222.08, 47.20) circle (  1.16);

\path[draw=drawColor,line width= 0.4pt,line join=round,line cap=round,fill=fillColor] (222.21, 47.20) circle (  1.16);

\path[draw=drawColor,line width= 0.4pt,line join=round,line cap=round,fill=fillColor] (222.35, 47.20) circle (  1.16);

\path[draw=drawColor,line width= 0.4pt,line join=round,line cap=round,fill=fillColor] (222.48, 47.20) circle (  1.16);

\path[draw=drawColor,line width= 0.4pt,line join=round,line cap=round,fill=fillColor] (222.62, 47.20) circle (  1.16);

\path[draw=drawColor,line width= 0.4pt,line join=round,line cap=round,fill=fillColor] (222.75, 47.20) circle (  1.16);

\path[draw=drawColor,line width= 0.4pt,line join=round,line cap=round,fill=fillColor] (222.88, 47.20) circle (  1.16);

\path[draw=drawColor,line width= 0.4pt,line join=round,line cap=round,fill=fillColor] (223.02, 47.20) circle (  1.16);

\path[draw=drawColor,line width= 0.4pt,line join=round,line cap=round,fill=fillColor] (223.15, 47.20) circle (  1.16);

\path[draw=drawColor,line width= 0.4pt,line join=round,line cap=round,fill=fillColor] (223.28, 47.20) circle (  1.16);

\path[draw=drawColor,line width= 0.4pt,line join=round,line cap=round,fill=fillColor] (223.42, 47.20) circle (  1.16);

\path[draw=drawColor,line width= 0.4pt,line join=round,line cap=round,fill=fillColor] (223.55, 47.20) circle (  1.16);

\path[draw=drawColor,line width= 0.4pt,line join=round,line cap=round,fill=fillColor] (223.68, 47.20) circle (  1.16);

\path[draw=drawColor,line width= 0.4pt,line join=round,line cap=round,fill=fillColor] (223.82, 47.20) circle (  1.16);

\path[draw=drawColor,line width= 0.4pt,line join=round,line cap=round,fill=fillColor] (223.95, 47.20) circle (  1.16);

\path[draw=drawColor,line width= 0.4pt,line join=round,line cap=round,fill=fillColor] (224.08, 47.20) circle (  1.16);

\path[draw=drawColor,line width= 0.4pt,line join=round,line cap=round,fill=fillColor] (224.21, 47.20) circle (  1.16);

\path[draw=drawColor,line width= 0.4pt,line join=round,line cap=round,fill=fillColor] (224.34, 47.20) circle (  1.16);

\path[draw=drawColor,line width= 0.4pt,line join=round,line cap=round,fill=fillColor] (224.48, 47.20) circle (  1.16);

\path[draw=drawColor,line width= 0.4pt,line join=round,line cap=round,fill=fillColor] (224.61, 47.20) circle (  1.16);

\path[draw=drawColor,line width= 0.4pt,line join=round,line cap=round,fill=fillColor] (224.74, 47.20) circle (  1.16);

\path[draw=drawColor,line width= 0.4pt,line join=round,line cap=round,fill=fillColor] (224.87, 47.20) circle (  1.16);

\path[draw=drawColor,line width= 0.4pt,line join=round,line cap=round,fill=fillColor] (225.00, 47.20) circle (  1.16);

\path[draw=drawColor,line width= 0.4pt,line join=round,line cap=round,fill=fillColor] (225.13, 47.20) circle (  1.16);

\path[draw=drawColor,line width= 0.4pt,line join=round,line cap=round,fill=fillColor] (225.26, 47.20) circle (  1.16);
\definecolor[named]{drawColor}{rgb}{0.65,0.34,0.16}
\definecolor[named]{fillColor}{rgb}{0.65,0.34,0.16}

\path[draw=drawColor,line width= 0.4pt,line join=round,line cap=round,fill=fillColor] ( 74.88,132.85) circle (  1.16);

\path[draw=drawColor,line width= 0.4pt,line join=round,line cap=round,fill=fillColor] ( 80.78,132.85) circle (  1.16);

\path[draw=drawColor,line width= 0.4pt,line join=round,line cap=round,fill=fillColor] ( 84.92,102.02) circle (  1.16);

\path[draw=drawColor,line width= 0.4pt,line join=round,line cap=round,fill=fillColor] ( 88.22, 99.75) circle (  1.16);

\path[draw=drawColor,line width= 0.4pt,line join=round,line cap=round,fill=fillColor] ( 91.00, 97.48) circle (  1.16);

\path[draw=drawColor,line width= 0.4pt,line join=round,line cap=round,fill=fillColor] ( 93.43, 95.41) circle (  1.16);

\path[draw=drawColor,line width= 0.4pt,line join=round,line cap=round,fill=fillColor] ( 95.61, 92.68) circle (  1.16);

\path[draw=drawColor,line width= 0.4pt,line join=round,line cap=round,fill=fillColor] ( 97.58, 90.59) circle (  1.16);

\path[draw=drawColor,line width= 0.4pt,line join=round,line cap=round,fill=fillColor] ( 99.40, 89.85) circle (  1.16);

\path[draw=drawColor,line width= 0.4pt,line join=round,line cap=round,fill=fillColor] (101.09, 88.83) circle (  1.16);

\path[draw=drawColor,line width= 0.4pt,line join=round,line cap=round,fill=fillColor] (102.67, 85.99) circle (  1.16);

\path[draw=drawColor,line width= 0.4pt,line join=round,line cap=round,fill=fillColor] (104.16, 84.59) circle (  1.16);

\path[draw=drawColor,line width= 0.4pt,line join=round,line cap=round,fill=fillColor] (105.56, 84.59) circle (  1.16);

\path[draw=drawColor,line width= 0.4pt,line join=round,line cap=round,fill=fillColor] (106.90, 84.47) circle (  1.16);

\path[draw=drawColor,line width= 0.4pt,line join=round,line cap=round,fill=fillColor] (108.17, 81.90) circle (  1.16);

\path[draw=drawColor,line width= 0.4pt,line join=round,line cap=round,fill=fillColor] (109.39, 81.63) circle (  1.16);

\path[draw=drawColor,line width= 0.4pt,line join=round,line cap=round,fill=fillColor] (110.56, 81.25) circle (  1.16);

\path[draw=drawColor,line width= 0.4pt,line join=round,line cap=round,fill=fillColor] (111.68, 80.51) circle (  1.16);

\path[draw=drawColor,line width= 0.4pt,line join=round,line cap=round,fill=fillColor] (112.76, 79.34) circle (  1.16);

\path[draw=drawColor,line width= 0.4pt,line join=round,line cap=round,fill=fillColor] (113.81, 79.30) circle (  1.16);

\path[draw=drawColor,line width= 0.4pt,line join=round,line cap=round,fill=fillColor] (114.82, 78.74) circle (  1.16);

\path[draw=drawColor,line width= 0.4pt,line join=round,line cap=round,fill=fillColor] (115.80, 78.52) circle (  1.16);

\path[draw=drawColor,line width= 0.4pt,line join=round,line cap=round,fill=fillColor] (116.74, 78.39) circle (  1.16);

\path[draw=drawColor,line width= 0.4pt,line join=round,line cap=round,fill=fillColor] (117.67, 77.32) circle (  1.16);

\path[draw=drawColor,line width= 0.4pt,line join=round,line cap=round,fill=fillColor] (118.56, 77.29) circle (  1.16);

\path[draw=drawColor,line width= 0.4pt,line join=round,line cap=round,fill=fillColor] (119.44, 76.77) circle (  1.16);

\path[draw=drawColor,line width= 0.4pt,line join=round,line cap=round,fill=fillColor] (120.29, 76.39) circle (  1.16);

\path[draw=drawColor,line width= 0.4pt,line join=round,line cap=round,fill=fillColor] (121.12, 76.00) circle (  1.16);

\path[draw=drawColor,line width= 0.4pt,line join=round,line cap=round,fill=fillColor] (121.93, 75.89) circle (  1.16);

\path[draw=drawColor,line width= 0.4pt,line join=round,line cap=round,fill=fillColor] (122.72, 75.51) circle (  1.16);

\path[draw=drawColor,line width= 0.4pt,line join=round,line cap=round,fill=fillColor] (123.50, 75.43) circle (  1.16);

\path[draw=drawColor,line width= 0.4pt,line join=round,line cap=round,fill=fillColor] (124.26, 74.75) circle (  1.16);

\path[draw=drawColor,line width= 0.4pt,line join=round,line cap=round,fill=fillColor] (125.00, 74.22) circle (  1.16);

\path[draw=drawColor,line width= 0.4pt,line join=round,line cap=round,fill=fillColor] (125.73, 73.71) circle (  1.16);

\path[draw=drawColor,line width= 0.4pt,line join=round,line cap=round,fill=fillColor] (126.44, 73.71) circle (  1.16);

\path[draw=drawColor,line width= 0.4pt,line join=round,line cap=round,fill=fillColor] (127.15, 73.59) circle (  1.16);

\path[draw=drawColor,line width= 0.4pt,line join=round,line cap=round,fill=fillColor] (127.83, 73.30) circle (  1.16);

\path[draw=drawColor,line width= 0.4pt,line join=round,line cap=round,fill=fillColor] (128.51, 73.25) circle (  1.16);

\path[draw=drawColor,line width= 0.4pt,line join=round,line cap=round,fill=fillColor] (129.17, 72.99) circle (  1.16);

\path[draw=drawColor,line width= 0.4pt,line join=round,line cap=round,fill=fillColor] (129.83, 72.95) circle (  1.16);

\path[draw=drawColor,line width= 0.4pt,line join=round,line cap=round,fill=fillColor] (130.47, 72.83) circle (  1.16);

\path[draw=drawColor,line width= 0.4pt,line join=round,line cap=round,fill=fillColor] (131.10, 72.69) circle (  1.16);

\path[draw=drawColor,line width= 0.4pt,line join=round,line cap=round,fill=fillColor] (131.72, 72.55) circle (  1.16);

\path[draw=drawColor,line width= 0.4pt,line join=round,line cap=round,fill=fillColor] (132.33, 72.09) circle (  1.16);

\path[draw=drawColor,line width= 0.4pt,line join=round,line cap=round,fill=fillColor] (132.93, 71.43) circle (  1.16);

\path[draw=drawColor,line width= 0.4pt,line join=round,line cap=round,fill=fillColor] (133.53, 71.36) circle (  1.16);

\path[draw=drawColor,line width= 0.4pt,line join=round,line cap=round,fill=fillColor] (134.11, 71.22) circle (  1.16);

\path[draw=drawColor,line width= 0.4pt,line join=round,line cap=round,fill=fillColor] (134.69, 70.95) circle (  1.16);

\path[draw=drawColor,line width= 0.4pt,line join=round,line cap=round,fill=fillColor] (135.26, 70.79) circle (  1.16);

\path[draw=drawColor,line width= 0.4pt,line join=round,line cap=round,fill=fillColor] (135.82, 70.62) circle (  1.16);

\path[draw=drawColor,line width= 0.4pt,line join=round,line cap=round,fill=fillColor] (136.38, 70.29) circle (  1.16);

\path[draw=drawColor,line width= 0.4pt,line join=round,line cap=round,fill=fillColor] (136.92, 70.00) circle (  1.16);

\path[draw=drawColor,line width= 0.4pt,line join=round,line cap=round,fill=fillColor] (137.46, 69.83) circle (  1.16);

\path[draw=drawColor,line width= 0.4pt,line join=round,line cap=round,fill=fillColor] (137.99, 69.74) circle (  1.16);

\path[draw=drawColor,line width= 0.4pt,line join=round,line cap=round,fill=fillColor] (138.52, 69.63) circle (  1.16);

\path[draw=drawColor,line width= 0.4pt,line join=round,line cap=round,fill=fillColor] (139.04, 69.15) circle (  1.16);

\path[draw=drawColor,line width= 0.4pt,line join=round,line cap=round,fill=fillColor] (139.56, 68.99) circle (  1.16);

\path[draw=drawColor,line width= 0.4pt,line join=round,line cap=round,fill=fillColor] (140.06, 68.91) circle (  1.16);

\path[draw=drawColor,line width= 0.4pt,line join=round,line cap=round,fill=fillColor] (140.57, 68.82) circle (  1.16);

\path[draw=drawColor,line width= 0.4pt,line join=round,line cap=round,fill=fillColor] (141.06, 68.81) circle (  1.16);

\path[draw=drawColor,line width= 0.4pt,line join=round,line cap=round,fill=fillColor] (141.55, 68.70) circle (  1.16);

\path[draw=drawColor,line width= 0.4pt,line join=round,line cap=round,fill=fillColor] (142.04, 68.61) circle (  1.16);

\path[draw=drawColor,line width= 0.4pt,line join=round,line cap=round,fill=fillColor] (142.52, 68.60) circle (  1.16);

\path[draw=drawColor,line width= 0.4pt,line join=round,line cap=round,fill=fillColor] (143.00, 68.31) circle (  1.16);

\path[draw=drawColor,line width= 0.4pt,line join=round,line cap=round,fill=fillColor] (143.47, 68.20) circle (  1.16);

\path[draw=drawColor,line width= 0.4pt,line join=round,line cap=round,fill=fillColor] (143.93, 68.14) circle (  1.16);

\path[draw=drawColor,line width= 0.4pt,line join=round,line cap=round,fill=fillColor] (144.39, 68.01) circle (  1.16);

\path[draw=drawColor,line width= 0.4pt,line join=round,line cap=round,fill=fillColor] (144.85, 67.78) circle (  1.16);

\path[draw=drawColor,line width= 0.4pt,line join=round,line cap=round,fill=fillColor] (145.30, 67.42) circle (  1.16);

\path[draw=drawColor,line width= 0.4pt,line join=round,line cap=round,fill=fillColor] (145.75, 67.08) circle (  1.16);

\path[draw=drawColor,line width= 0.4pt,line join=round,line cap=round,fill=fillColor] (146.19, 66.87) circle (  1.16);

\path[draw=drawColor,line width= 0.4pt,line join=round,line cap=round,fill=fillColor] (146.63, 66.76) circle (  1.16);

\path[draw=drawColor,line width= 0.4pt,line join=round,line cap=round,fill=fillColor] (147.07, 66.28) circle (  1.16);

\path[draw=drawColor,line width= 0.4pt,line join=round,line cap=round,fill=fillColor] (147.50, 65.88) circle (  1.16);

\path[draw=drawColor,line width= 0.4pt,line join=round,line cap=round,fill=fillColor] (147.93, 65.86) circle (  1.16);

\path[draw=drawColor,line width= 0.4pt,line join=round,line cap=round,fill=fillColor] (148.35, 65.67) circle (  1.16);

\path[draw=drawColor,line width= 0.4pt,line join=round,line cap=round,fill=fillColor] (148.77, 65.53) circle (  1.16);

\path[draw=drawColor,line width= 0.4pt,line join=round,line cap=round,fill=fillColor] (149.19, 65.52) circle (  1.16);

\path[draw=drawColor,line width= 0.4pt,line join=round,line cap=round,fill=fillColor] (149.60, 65.09) circle (  1.16);

\path[draw=drawColor,line width= 0.4pt,line join=round,line cap=round,fill=fillColor] (150.01, 64.78) circle (  1.16);

\path[draw=drawColor,line width= 0.4pt,line join=round,line cap=round,fill=fillColor] (150.41, 64.64) circle (  1.16);

\path[draw=drawColor,line width= 0.4pt,line join=round,line cap=round,fill=fillColor] (150.82, 64.57) circle (  1.16);

\path[draw=drawColor,line width= 0.4pt,line join=round,line cap=round,fill=fillColor] (151.22, 64.49) circle (  1.16);

\path[draw=drawColor,line width= 0.4pt,line join=round,line cap=round,fill=fillColor] (151.61, 64.39) circle (  1.16);

\path[draw=drawColor,line width= 0.4pt,line join=round,line cap=round,fill=fillColor] (152.01, 64.17) circle (  1.16);

\path[draw=drawColor,line width= 0.4pt,line join=round,line cap=round,fill=fillColor] (152.40, 64.04) circle (  1.16);

\path[draw=drawColor,line width= 0.4pt,line join=round,line cap=round,fill=fillColor] (152.78, 63.89) circle (  1.16);

\path[draw=drawColor,line width= 0.4pt,line join=round,line cap=round,fill=fillColor] (153.17, 63.87) circle (  1.16);

\path[draw=drawColor,line width= 0.4pt,line join=round,line cap=round,fill=fillColor] (153.55, 63.82) circle (  1.16);

\path[draw=drawColor,line width= 0.4pt,line join=round,line cap=round,fill=fillColor] (153.93, 63.51) circle (  1.16);

\path[draw=drawColor,line width= 0.4pt,line join=round,line cap=round,fill=fillColor] (154.30, 62.43) circle (  1.16);

\path[draw=drawColor,line width= 0.4pt,line join=round,line cap=round,fill=fillColor] (154.67, 61.94) circle (  1.16);

\path[draw=drawColor,line width= 0.4pt,line join=round,line cap=round,fill=fillColor] (155.04, 61.77) circle (  1.16);

\path[draw=drawColor,line width= 0.4pt,line join=round,line cap=round,fill=fillColor] (155.41, 61.61) circle (  1.16);

\path[draw=drawColor,line width= 0.4pt,line join=round,line cap=round,fill=fillColor] (155.78, 61.58) circle (  1.16);

\path[draw=drawColor,line width= 0.4pt,line join=round,line cap=round,fill=fillColor] (156.14, 61.35) circle (  1.16);

\path[draw=drawColor,line width= 0.4pt,line join=round,line cap=round,fill=fillColor] (156.50, 60.60) circle (  1.16);

\path[draw=drawColor,line width= 0.4pt,line join=round,line cap=round,fill=fillColor] (156.86, 60.58) circle (  1.16);

\path[draw=drawColor,line width= 0.4pt,line join=round,line cap=round,fill=fillColor] (157.21, 60.44) circle (  1.16);

\path[draw=drawColor,line width= 0.4pt,line join=round,line cap=round,fill=fillColor] (157.56, 59.95) circle (  1.16);

\path[draw=drawColor,line width= 0.4pt,line join=round,line cap=round,fill=fillColor] (157.91, 59.67) circle (  1.16);

\path[draw=drawColor,line width= 0.4pt,line join=round,line cap=round,fill=fillColor] (158.26, 59.10) circle (  1.16);

\path[draw=drawColor,line width= 0.4pt,line join=round,line cap=round,fill=fillColor] (158.61, 59.04) circle (  1.16);

\path[draw=drawColor,line width= 0.4pt,line join=round,line cap=round,fill=fillColor] (158.95, 58.87) circle (  1.16);

\path[draw=drawColor,line width= 0.4pt,line join=round,line cap=round,fill=fillColor] (159.29, 58.81) circle (  1.16);

\path[draw=drawColor,line width= 0.4pt,line join=round,line cap=round,fill=fillColor] (159.63, 58.71) circle (  1.16);

\path[draw=drawColor,line width= 0.4pt,line join=round,line cap=round,fill=fillColor] (159.97, 58.35) circle (  1.16);

\path[draw=drawColor,line width= 0.4pt,line join=round,line cap=round,fill=fillColor] (160.30, 58.19) circle (  1.16);

\path[draw=drawColor,line width= 0.4pt,line join=round,line cap=round,fill=fillColor] (160.63, 58.05) circle (  1.16);

\path[draw=drawColor,line width= 0.4pt,line join=round,line cap=round,fill=fillColor] (160.96, 57.60) circle (  1.16);

\path[draw=drawColor,line width= 0.4pt,line join=round,line cap=round,fill=fillColor] (161.29, 57.07) circle (  1.16);

\path[draw=drawColor,line width= 0.4pt,line join=round,line cap=round,fill=fillColor] (161.62, 57.05) circle (  1.16);

\path[draw=drawColor,line width= 0.4pt,line join=round,line cap=round,fill=fillColor] (161.95, 56.43) circle (  1.16);

\path[draw=drawColor,line width= 0.4pt,line join=round,line cap=round,fill=fillColor] (162.27, 56.11) circle (  1.16);

\path[draw=drawColor,line width= 0.4pt,line join=round,line cap=round,fill=fillColor] (162.59, 55.66) circle (  1.16);

\path[draw=drawColor,line width= 0.4pt,line join=round,line cap=round,fill=fillColor] (162.91, 54.70) circle (  1.16);

\path[draw=drawColor,line width= 0.4pt,line join=round,line cap=round,fill=fillColor] (163.23, 54.64) circle (  1.16);

\path[draw=drawColor,line width= 0.4pt,line join=round,line cap=round,fill=fillColor] (163.54, 54.45) circle (  1.16);

\path[draw=drawColor,line width= 0.4pt,line join=round,line cap=round,fill=fillColor] (163.85, 54.39) circle (  1.16);

\path[draw=drawColor,line width= 0.4pt,line join=round,line cap=round,fill=fillColor] (164.17, 54.19) circle (  1.16);

\path[draw=drawColor,line width= 0.4pt,line join=round,line cap=round,fill=fillColor] (164.48, 54.08) circle (  1.16);

\path[draw=drawColor,line width= 0.4pt,line join=round,line cap=round,fill=fillColor] (164.79, 53.66) circle (  1.16);

\path[draw=drawColor,line width= 0.4pt,line join=round,line cap=round,fill=fillColor] (165.09, 53.38) circle (  1.16);

\path[draw=drawColor,line width= 0.4pt,line join=round,line cap=round,fill=fillColor] (165.40, 51.14) circle (  1.16);

\path[draw=drawColor,line width= 0.4pt,line join=round,line cap=round,fill=fillColor] (165.70, 51.02) circle (  1.16);

\path[draw=drawColor,line width= 0.4pt,line join=round,line cap=round,fill=fillColor] (166.00, 49.79) circle (  1.16);

\path[draw=drawColor,line width= 0.4pt,line join=round,line cap=round,fill=fillColor] (166.30, 47.20) circle (  1.16);

\path[draw=drawColor,line width= 0.4pt,line join=round,line cap=round,fill=fillColor] (166.60, 47.20) circle (  1.16);

\path[draw=drawColor,line width= 0.4pt,line join=round,line cap=round,fill=fillColor] (166.90, 47.20) circle (  1.16);

\path[draw=drawColor,line width= 0.4pt,line join=round,line cap=round,fill=fillColor] (167.19, 47.20) circle (  1.16);

\path[draw=drawColor,line width= 0.4pt,line join=round,line cap=round,fill=fillColor] (167.49, 47.20) circle (  1.16);

\path[draw=drawColor,line width= 0.4pt,line join=round,line cap=round,fill=fillColor] (167.78, 47.20) circle (  1.16);

\path[draw=drawColor,line width= 0.4pt,line join=round,line cap=round,fill=fillColor] (168.07, 47.20) circle (  1.16);

\path[draw=drawColor,line width= 0.4pt,line join=round,line cap=round,fill=fillColor] (168.36, 47.20) circle (  1.16);

\path[draw=drawColor,line width= 0.4pt,line join=round,line cap=round,fill=fillColor] (168.65, 47.20) circle (  1.16);

\path[draw=drawColor,line width= 0.4pt,line join=round,line cap=round,fill=fillColor] (168.94, 47.20) circle (  1.16);

\path[draw=drawColor,line width= 0.4pt,line join=round,line cap=round,fill=fillColor] (169.22, 47.20) circle (  1.16);

\path[draw=drawColor,line width= 0.4pt,line join=round,line cap=round,fill=fillColor] (169.51, 47.20) circle (  1.16);

\path[draw=drawColor,line width= 0.4pt,line join=round,line cap=round,fill=fillColor] (169.79, 47.20) circle (  1.16);

\path[draw=drawColor,line width= 0.4pt,line join=round,line cap=round,fill=fillColor] (170.07, 47.20) circle (  1.16);

\path[draw=drawColor,line width= 0.4pt,line join=round,line cap=round,fill=fillColor] (170.35, 47.20) circle (  1.16);

\path[draw=drawColor,line width= 0.4pt,line join=round,line cap=round,fill=fillColor] (170.63, 47.20) circle (  1.16);

\path[draw=drawColor,line width= 0.4pt,line join=round,line cap=round,fill=fillColor] (170.91, 47.20) circle (  1.16);

\path[draw=drawColor,line width= 0.4pt,line join=round,line cap=round,fill=fillColor] (171.18, 47.20) circle (  1.16);

\path[draw=drawColor,line width= 0.4pt,line join=round,line cap=round,fill=fillColor] (171.46, 47.20) circle (  1.16);

\path[draw=drawColor,line width= 0.4pt,line join=round,line cap=round,fill=fillColor] (171.73, 47.20) circle (  1.16);

\path[draw=drawColor,line width= 0.4pt,line join=round,line cap=round,fill=fillColor] (172.00, 47.20) circle (  1.16);

\path[draw=drawColor,line width= 0.4pt,line join=round,line cap=round,fill=fillColor] (172.28, 47.20) circle (  1.16);

\path[draw=drawColor,line width= 0.4pt,line join=round,line cap=round,fill=fillColor] (172.55, 47.20) circle (  1.16);

\path[draw=drawColor,line width= 0.4pt,line join=round,line cap=round,fill=fillColor] (172.81, 47.20) circle (  1.16);

\path[draw=drawColor,line width= 0.4pt,line join=round,line cap=round,fill=fillColor] (173.08, 47.20) circle (  1.16);

\path[draw=drawColor,line width= 0.4pt,line join=round,line cap=round,fill=fillColor] (173.35, 47.20) circle (  1.16);

\path[draw=drawColor,line width= 0.4pt,line join=round,line cap=round,fill=fillColor] (173.61, 47.20) circle (  1.16);

\path[draw=drawColor,line width= 0.4pt,line join=round,line cap=round,fill=fillColor] (173.88, 47.20) circle (  1.16);

\path[draw=drawColor,line width= 0.4pt,line join=round,line cap=round,fill=fillColor] (174.14, 47.20) circle (  1.16);

\path[draw=drawColor,line width= 0.4pt,line join=round,line cap=round,fill=fillColor] (174.40, 47.20) circle (  1.16);

\path[draw=drawColor,line width= 0.4pt,line join=round,line cap=round,fill=fillColor] (174.66, 47.20) circle (  1.16);

\path[draw=drawColor,line width= 0.4pt,line join=round,line cap=round,fill=fillColor] (174.92, 47.20) circle (  1.16);

\path[draw=drawColor,line width= 0.4pt,line join=round,line cap=round,fill=fillColor] (175.18, 47.20) circle (  1.16);

\path[draw=drawColor,line width= 0.4pt,line join=round,line cap=round,fill=fillColor] (175.44, 47.20) circle (  1.16);

\path[draw=drawColor,line width= 0.4pt,line join=round,line cap=round,fill=fillColor] (175.69, 47.20) circle (  1.16);

\path[draw=drawColor,line width= 0.4pt,line join=round,line cap=round,fill=fillColor] (175.95, 47.20) circle (  1.16);

\path[draw=drawColor,line width= 0.4pt,line join=round,line cap=round,fill=fillColor] (176.20, 47.20) circle (  1.16);

\path[draw=drawColor,line width= 0.4pt,line join=round,line cap=round,fill=fillColor] (176.46, 47.20) circle (  1.16);

\path[draw=drawColor,line width= 0.4pt,line join=round,line cap=round,fill=fillColor] (176.71, 47.20) circle (  1.16);

\path[draw=drawColor,line width= 0.4pt,line join=round,line cap=round,fill=fillColor] (176.96, 47.20) circle (  1.16);

\path[draw=drawColor,line width= 0.4pt,line join=round,line cap=round,fill=fillColor] (177.21, 47.20) circle (  1.16);

\path[draw=drawColor,line width= 0.4pt,line join=round,line cap=round,fill=fillColor] (177.46, 47.20) circle (  1.16);

\path[draw=drawColor,line width= 0.4pt,line join=round,line cap=round,fill=fillColor] (177.71, 47.20) circle (  1.16);

\path[draw=drawColor,line width= 0.4pt,line join=round,line cap=round,fill=fillColor] (177.95, 47.20) circle (  1.16);

\path[draw=drawColor,line width= 0.4pt,line join=round,line cap=round,fill=fillColor] (178.20, 47.20) circle (  1.16);

\path[draw=drawColor,line width= 0.4pt,line join=round,line cap=round,fill=fillColor] (178.45, 47.20) circle (  1.16);

\path[draw=drawColor,line width= 0.4pt,line join=round,line cap=round,fill=fillColor] (178.69, 47.20) circle (  1.16);

\path[draw=drawColor,line width= 0.4pt,line join=round,line cap=round,fill=fillColor] (178.93, 47.20) circle (  1.16);

\path[draw=drawColor,line width= 0.4pt,line join=round,line cap=round,fill=fillColor] (179.18, 47.20) circle (  1.16);

\path[draw=drawColor,line width= 0.4pt,line join=round,line cap=round,fill=fillColor] (179.42, 47.20) circle (  1.16);

\path[draw=drawColor,line width= 0.4pt,line join=round,line cap=round,fill=fillColor] (179.66, 47.20) circle (  1.16);

\path[draw=drawColor,line width= 0.4pt,line join=round,line cap=round,fill=fillColor] (179.90, 47.20) circle (  1.16);

\path[draw=drawColor,line width= 0.4pt,line join=round,line cap=round,fill=fillColor] (180.14, 47.20) circle (  1.16);

\path[draw=drawColor,line width= 0.4pt,line join=round,line cap=round,fill=fillColor] (180.37, 47.20) circle (  1.16);

\path[draw=drawColor,line width= 0.4pt,line join=round,line cap=round,fill=fillColor] (180.61, 47.20) circle (  1.16);

\path[draw=drawColor,line width= 0.4pt,line join=round,line cap=round,fill=fillColor] (180.85, 47.20) circle (  1.16);

\path[draw=drawColor,line width= 0.4pt,line join=round,line cap=round,fill=fillColor] (181.08, 47.20) circle (  1.16);

\path[draw=drawColor,line width= 0.4pt,line join=round,line cap=round,fill=fillColor] (181.32, 47.20) circle (  1.16);

\path[draw=drawColor,line width= 0.4pt,line join=round,line cap=round,fill=fillColor] (181.55, 47.20) circle (  1.16);

\path[draw=drawColor,line width= 0.4pt,line join=round,line cap=round,fill=fillColor] (181.78, 47.20) circle (  1.16);

\path[draw=drawColor,line width= 0.4pt,line join=round,line cap=round,fill=fillColor] (182.01, 47.20) circle (  1.16);

\path[draw=drawColor,line width= 0.4pt,line join=round,line cap=round,fill=fillColor] (182.25, 47.20) circle (  1.16);

\path[draw=drawColor,line width= 0.4pt,line join=round,line cap=round,fill=fillColor] (182.48, 47.20) circle (  1.16);

\path[draw=drawColor,line width= 0.4pt,line join=round,line cap=round,fill=fillColor] (182.70, 47.20) circle (  1.16);

\path[draw=drawColor,line width= 0.4pt,line join=round,line cap=round,fill=fillColor] (182.93, 47.20) circle (  1.16);

\path[draw=drawColor,line width= 0.4pt,line join=round,line cap=round,fill=fillColor] (183.16, 47.20) circle (  1.16);

\path[draw=drawColor,line width= 0.4pt,line join=round,line cap=round,fill=fillColor] (183.39, 47.20) circle (  1.16);

\path[draw=drawColor,line width= 0.4pt,line join=round,line cap=round,fill=fillColor] (183.61, 47.20) circle (  1.16);

\path[draw=drawColor,line width= 0.4pt,line join=round,line cap=round,fill=fillColor] (183.84, 47.20) circle (  1.16);

\path[draw=drawColor,line width= 0.4pt,line join=round,line cap=round,fill=fillColor] (184.06, 47.20) circle (  1.16);

\path[draw=drawColor,line width= 0.4pt,line join=round,line cap=round,fill=fillColor] (184.29, 47.20) circle (  1.16);

\path[draw=drawColor,line width= 0.4pt,line join=round,line cap=round,fill=fillColor] (184.51, 47.20) circle (  1.16);

\path[draw=drawColor,line width= 0.4pt,line join=round,line cap=round,fill=fillColor] (184.73, 47.20) circle (  1.16);

\path[draw=drawColor,line width= 0.4pt,line join=round,line cap=round,fill=fillColor] (184.96, 47.20) circle (  1.16);

\path[draw=drawColor,line width= 0.4pt,line join=round,line cap=round,fill=fillColor] (185.18, 47.20) circle (  1.16);

\path[draw=drawColor,line width= 0.4pt,line join=round,line cap=round,fill=fillColor] (185.40, 47.20) circle (  1.16);

\path[draw=drawColor,line width= 0.4pt,line join=round,line cap=round,fill=fillColor] (185.62, 47.20) circle (  1.16);

\path[draw=drawColor,line width= 0.4pt,line join=round,line cap=round,fill=fillColor] (185.84, 47.20) circle (  1.16);

\path[draw=drawColor,line width= 0.4pt,line join=round,line cap=round,fill=fillColor] (186.05, 47.20) circle (  1.16);

\path[draw=drawColor,line width= 0.4pt,line join=round,line cap=round,fill=fillColor] (186.27, 47.20) circle (  1.16);

\path[draw=drawColor,line width= 0.4pt,line join=round,line cap=round,fill=fillColor] (186.49, 47.20) circle (  1.16);

\path[draw=drawColor,line width= 0.4pt,line join=round,line cap=round,fill=fillColor] (186.70, 47.20) circle (  1.16);

\path[draw=drawColor,line width= 0.4pt,line join=round,line cap=round,fill=fillColor] (186.92, 47.20) circle (  1.16);

\path[draw=drawColor,line width= 0.4pt,line join=round,line cap=round,fill=fillColor] (187.13, 47.20) circle (  1.16);

\path[draw=drawColor,line width= 0.4pt,line join=round,line cap=round,fill=fillColor] (187.35, 47.20) circle (  1.16);

\path[draw=drawColor,line width= 0.4pt,line join=round,line cap=round,fill=fillColor] (187.56, 47.20) circle (  1.16);

\path[draw=drawColor,line width= 0.4pt,line join=round,line cap=round,fill=fillColor] (187.77, 47.20) circle (  1.16);

\path[draw=drawColor,line width= 0.4pt,line join=round,line cap=round,fill=fillColor] (187.98, 47.20) circle (  1.16);

\path[draw=drawColor,line width= 0.4pt,line join=round,line cap=round,fill=fillColor] (188.20, 47.20) circle (  1.16);

\path[draw=drawColor,line width= 0.4pt,line join=round,line cap=round,fill=fillColor] (188.41, 47.20) circle (  1.16);

\path[draw=drawColor,line width= 0.4pt,line join=round,line cap=round,fill=fillColor] (188.62, 47.20) circle (  1.16);

\path[draw=drawColor,line width= 0.4pt,line join=round,line cap=round,fill=fillColor] (188.83, 47.20) circle (  1.16);

\path[draw=drawColor,line width= 0.4pt,line join=round,line cap=round,fill=fillColor] (189.03, 47.20) circle (  1.16);

\path[draw=drawColor,line width= 0.4pt,line join=round,line cap=round,fill=fillColor] (189.24, 47.20) circle (  1.16);

\path[draw=drawColor,line width= 0.4pt,line join=round,line cap=round,fill=fillColor] (189.45, 47.20) circle (  1.16);

\path[draw=drawColor,line width= 0.4pt,line join=round,line cap=round,fill=fillColor] (189.66, 47.20) circle (  1.16);

\path[draw=drawColor,line width= 0.4pt,line join=round,line cap=round,fill=fillColor] (189.86, 47.20) circle (  1.16);

\path[draw=drawColor,line width= 0.4pt,line join=round,line cap=round,fill=fillColor] (190.07, 47.20) circle (  1.16);

\path[draw=drawColor,line width= 0.4pt,line join=round,line cap=round,fill=fillColor] (190.27, 47.20) circle (  1.16);

\path[draw=drawColor,line width= 0.4pt,line join=round,line cap=round,fill=fillColor] (190.48, 47.20) circle (  1.16);

\path[draw=drawColor,line width= 0.4pt,line join=round,line cap=round,fill=fillColor] (190.68, 47.20) circle (  1.16);

\path[draw=drawColor,line width= 0.4pt,line join=round,line cap=round,fill=fillColor] (190.88, 47.20) circle (  1.16);

\path[draw=drawColor,line width= 0.4pt,line join=round,line cap=round,fill=fillColor] (191.09, 47.20) circle (  1.16);

\path[draw=drawColor,line width= 0.4pt,line join=round,line cap=round,fill=fillColor] (191.29, 47.20) circle (  1.16);

\path[draw=drawColor,line width= 0.4pt,line join=round,line cap=round,fill=fillColor] (191.49, 47.20) circle (  1.16);

\path[draw=drawColor,line width= 0.4pt,line join=round,line cap=round,fill=fillColor] (191.69, 47.20) circle (  1.16);

\path[draw=drawColor,line width= 0.4pt,line join=round,line cap=round,fill=fillColor] (191.89, 47.20) circle (  1.16);

\path[draw=drawColor,line width= 0.4pt,line join=round,line cap=round,fill=fillColor] (192.09, 47.20) circle (  1.16);

\path[draw=drawColor,line width= 0.4pt,line join=round,line cap=round,fill=fillColor] (192.29, 47.20) circle (  1.16);

\path[draw=drawColor,line width= 0.4pt,line join=round,line cap=round,fill=fillColor] (192.49, 47.20) circle (  1.16);

\path[draw=drawColor,line width= 0.4pt,line join=round,line cap=round,fill=fillColor] (192.69, 47.20) circle (  1.16);

\path[draw=drawColor,line width= 0.4pt,line join=round,line cap=round,fill=fillColor] (192.88, 47.20) circle (  1.16);

\path[draw=drawColor,line width= 0.4pt,line join=round,line cap=round,fill=fillColor] (193.08, 47.20) circle (  1.16);

\path[draw=drawColor,line width= 0.4pt,line join=round,line cap=round,fill=fillColor] (193.28, 47.20) circle (  1.16);

\path[draw=drawColor,line width= 0.4pt,line join=round,line cap=round,fill=fillColor] (193.47, 47.20) circle (  1.16);

\path[draw=drawColor,line width= 0.4pt,line join=round,line cap=round,fill=fillColor] (193.67, 47.20) circle (  1.16);

\path[draw=drawColor,line width= 0.4pt,line join=round,line cap=round,fill=fillColor] (193.86, 47.20) circle (  1.16);

\path[draw=drawColor,line width= 0.4pt,line join=round,line cap=round,fill=fillColor] (194.06, 47.20) circle (  1.16);

\path[draw=drawColor,line width= 0.4pt,line join=round,line cap=round,fill=fillColor] (194.25, 47.20) circle (  1.16);

\path[draw=drawColor,line width= 0.4pt,line join=round,line cap=round,fill=fillColor] (194.44, 47.20) circle (  1.16);

\path[draw=drawColor,line width= 0.4pt,line join=round,line cap=round,fill=fillColor] (194.63, 47.20) circle (  1.16);

\path[draw=drawColor,line width= 0.4pt,line join=round,line cap=round,fill=fillColor] (194.83, 47.20) circle (  1.16);

\path[draw=drawColor,line width= 0.4pt,line join=round,line cap=round,fill=fillColor] (195.02, 47.20) circle (  1.16);

\path[draw=drawColor,line width= 0.4pt,line join=round,line cap=round,fill=fillColor] (195.21, 47.20) circle (  1.16);

\path[draw=drawColor,line width= 0.4pt,line join=round,line cap=round,fill=fillColor] (195.40, 47.20) circle (  1.16);

\path[draw=drawColor,line width= 0.4pt,line join=round,line cap=round,fill=fillColor] (195.59, 47.20) circle (  1.16);

\path[draw=drawColor,line width= 0.4pt,line join=round,line cap=round,fill=fillColor] (195.78, 47.20) circle (  1.16);

\path[draw=drawColor,line width= 0.4pt,line join=round,line cap=round,fill=fillColor] (195.97, 47.20) circle (  1.16);

\path[draw=drawColor,line width= 0.4pt,line join=round,line cap=round,fill=fillColor] (196.16, 47.20) circle (  1.16);

\path[draw=drawColor,line width= 0.4pt,line join=round,line cap=round,fill=fillColor] (196.34, 47.20) circle (  1.16);

\path[draw=drawColor,line width= 0.4pt,line join=round,line cap=round,fill=fillColor] (196.53, 47.20) circle (  1.16);

\path[draw=drawColor,line width= 0.4pt,line join=round,line cap=round,fill=fillColor] (196.72, 47.20) circle (  1.16);

\path[draw=drawColor,line width= 0.4pt,line join=round,line cap=round,fill=fillColor] (196.91, 47.20) circle (  1.16);

\path[draw=drawColor,line width= 0.4pt,line join=round,line cap=round,fill=fillColor] (197.09, 47.20) circle (  1.16);

\path[draw=drawColor,line width= 0.4pt,line join=round,line cap=round,fill=fillColor] (197.28, 47.20) circle (  1.16);

\path[draw=drawColor,line width= 0.4pt,line join=round,line cap=round,fill=fillColor] (197.46, 47.20) circle (  1.16);

\path[draw=drawColor,line width= 0.4pt,line join=round,line cap=round,fill=fillColor] (197.65, 47.20) circle (  1.16);

\path[draw=drawColor,line width= 0.4pt,line join=round,line cap=round,fill=fillColor] (197.83, 47.20) circle (  1.16);

\path[draw=drawColor,line width= 0.4pt,line join=round,line cap=round,fill=fillColor] (198.01, 47.20) circle (  1.16);

\path[draw=drawColor,line width= 0.4pt,line join=round,line cap=round,fill=fillColor] (198.20, 47.20) circle (  1.16);

\path[draw=drawColor,line width= 0.4pt,line join=round,line cap=round,fill=fillColor] (198.38, 47.20) circle (  1.16);

\path[draw=drawColor,line width= 0.4pt,line join=round,line cap=round,fill=fillColor] (198.56, 47.20) circle (  1.16);

\path[draw=drawColor,line width= 0.4pt,line join=round,line cap=round,fill=fillColor] (198.74, 47.20) circle (  1.16);

\path[draw=drawColor,line width= 0.4pt,line join=round,line cap=round,fill=fillColor] (198.93, 47.20) circle (  1.16);

\path[draw=drawColor,line width= 0.4pt,line join=round,line cap=round,fill=fillColor] (199.11, 47.20) circle (  1.16);

\path[draw=drawColor,line width= 0.4pt,line join=round,line cap=round,fill=fillColor] (199.29, 47.20) circle (  1.16);

\path[draw=drawColor,line width= 0.4pt,line join=round,line cap=round,fill=fillColor] (199.47, 47.20) circle (  1.16);

\path[draw=drawColor,line width= 0.4pt,line join=round,line cap=round,fill=fillColor] (199.65, 47.20) circle (  1.16);

\path[draw=drawColor,line width= 0.4pt,line join=round,line cap=round,fill=fillColor] (199.83, 47.20) circle (  1.16);

\path[draw=drawColor,line width= 0.4pt,line join=round,line cap=round,fill=fillColor] (200.00, 47.20) circle (  1.16);

\path[draw=drawColor,line width= 0.4pt,line join=round,line cap=round,fill=fillColor] (200.18, 47.20) circle (  1.16);

\path[draw=drawColor,line width= 0.4pt,line join=round,line cap=round,fill=fillColor] (200.36, 47.20) circle (  1.16);

\path[draw=drawColor,line width= 0.4pt,line join=round,line cap=round,fill=fillColor] (200.54, 47.20) circle (  1.16);

\path[draw=drawColor,line width= 0.4pt,line join=round,line cap=round,fill=fillColor] (200.72, 47.20) circle (  1.16);

\path[draw=drawColor,line width= 0.4pt,line join=round,line cap=round,fill=fillColor] (200.89, 47.20) circle (  1.16);

\path[draw=drawColor,line width= 0.4pt,line join=round,line cap=round,fill=fillColor] (201.07, 47.20) circle (  1.16);

\path[draw=drawColor,line width= 0.4pt,line join=round,line cap=round,fill=fillColor] (201.24, 47.20) circle (  1.16);

\path[draw=drawColor,line width= 0.4pt,line join=round,line cap=round,fill=fillColor] (201.42, 47.20) circle (  1.16);

\path[draw=drawColor,line width= 0.4pt,line join=round,line cap=round,fill=fillColor] (201.59, 47.20) circle (  1.16);

\path[draw=drawColor,line width= 0.4pt,line join=round,line cap=round,fill=fillColor] (201.77, 47.20) circle (  1.16);

\path[draw=drawColor,line width= 0.4pt,line join=round,line cap=round,fill=fillColor] (201.94, 47.20) circle (  1.16);

\path[draw=drawColor,line width= 0.4pt,line join=round,line cap=round,fill=fillColor] (202.12, 47.20) circle (  1.16);

\path[draw=drawColor,line width= 0.4pt,line join=round,line cap=round,fill=fillColor] (202.29, 47.20) circle (  1.16);

\path[draw=drawColor,line width= 0.4pt,line join=round,line cap=round,fill=fillColor] (202.46, 47.20) circle (  1.16);

\path[draw=drawColor,line width= 0.4pt,line join=round,line cap=round,fill=fillColor] (202.64, 47.20) circle (  1.16);

\path[draw=drawColor,line width= 0.4pt,line join=round,line cap=round,fill=fillColor] (202.81, 47.20) circle (  1.16);

\path[draw=drawColor,line width= 0.4pt,line join=round,line cap=round,fill=fillColor] (202.98, 47.20) circle (  1.16);

\path[draw=drawColor,line width= 0.4pt,line join=round,line cap=round,fill=fillColor] (203.15, 47.20) circle (  1.16);

\path[draw=drawColor,line width= 0.4pt,line join=round,line cap=round,fill=fillColor] (203.32, 47.20) circle (  1.16);

\path[draw=drawColor,line width= 0.4pt,line join=round,line cap=round,fill=fillColor] (203.49, 47.20) circle (  1.16);

\path[draw=drawColor,line width= 0.4pt,line join=round,line cap=round,fill=fillColor] (203.66, 47.20) circle (  1.16);

\path[draw=drawColor,line width= 0.4pt,line join=round,line cap=round,fill=fillColor] (203.83, 47.20) circle (  1.16);

\path[draw=drawColor,line width= 0.4pt,line join=round,line cap=round,fill=fillColor] (204.00, 47.20) circle (  1.16);

\path[draw=drawColor,line width= 0.4pt,line join=round,line cap=round,fill=fillColor] (204.17, 47.20) circle (  1.16);

\path[draw=drawColor,line width= 0.4pt,line join=round,line cap=round,fill=fillColor] (204.34, 47.20) circle (  1.16);

\path[draw=drawColor,line width= 0.4pt,line join=round,line cap=round,fill=fillColor] (204.51, 47.20) circle (  1.16);

\path[draw=drawColor,line width= 0.4pt,line join=round,line cap=round,fill=fillColor] (204.68, 47.20) circle (  1.16);

\path[draw=drawColor,line width= 0.4pt,line join=round,line cap=round,fill=fillColor] (204.84, 47.20) circle (  1.16);

\path[draw=drawColor,line width= 0.4pt,line join=round,line cap=round,fill=fillColor] (205.01, 47.20) circle (  1.16);

\path[draw=drawColor,line width= 0.4pt,line join=round,line cap=round,fill=fillColor] (205.18, 47.20) circle (  1.16);

\path[draw=drawColor,line width= 0.4pt,line join=round,line cap=round,fill=fillColor] (205.34, 47.20) circle (  1.16);

\path[draw=drawColor,line width= 0.4pt,line join=round,line cap=round,fill=fillColor] (205.51, 47.20) circle (  1.16);

\path[draw=drawColor,line width= 0.4pt,line join=round,line cap=round,fill=fillColor] (205.68, 47.20) circle (  1.16);

\path[draw=drawColor,line width= 0.4pt,line join=round,line cap=round,fill=fillColor] (205.84, 47.20) circle (  1.16);

\path[draw=drawColor,line width= 0.4pt,line join=round,line cap=round,fill=fillColor] (206.01, 47.20) circle (  1.16);

\path[draw=drawColor,line width= 0.4pt,line join=round,line cap=round,fill=fillColor] (206.17, 47.20) circle (  1.16);

\path[draw=drawColor,line width= 0.4pt,line join=round,line cap=round,fill=fillColor] (206.34, 47.20) circle (  1.16);

\path[draw=drawColor,line width= 0.4pt,line join=round,line cap=round,fill=fillColor] (206.50, 47.20) circle (  1.16);

\path[draw=drawColor,line width= 0.4pt,line join=round,line cap=round,fill=fillColor] (206.66, 47.20) circle (  1.16);

\path[draw=drawColor,line width= 0.4pt,line join=round,line cap=round,fill=fillColor] (206.83, 47.20) circle (  1.16);

\path[draw=drawColor,line width= 0.4pt,line join=round,line cap=round,fill=fillColor] (206.99, 47.20) circle (  1.16);

\path[draw=drawColor,line width= 0.4pt,line join=round,line cap=round,fill=fillColor] (207.15, 47.20) circle (  1.16);

\path[draw=drawColor,line width= 0.4pt,line join=round,line cap=round,fill=fillColor] (207.31, 47.20) circle (  1.16);

\path[draw=drawColor,line width= 0.4pt,line join=round,line cap=round,fill=fillColor] (207.48, 47.20) circle (  1.16);

\path[draw=drawColor,line width= 0.4pt,line join=round,line cap=round,fill=fillColor] (207.64, 47.20) circle (  1.16);

\path[draw=drawColor,line width= 0.4pt,line join=round,line cap=round,fill=fillColor] (207.80, 47.20) circle (  1.16);

\path[draw=drawColor,line width= 0.4pt,line join=round,line cap=round,fill=fillColor] (207.96, 47.20) circle (  1.16);

\path[draw=drawColor,line width= 0.4pt,line join=round,line cap=round,fill=fillColor] (208.12, 47.20) circle (  1.16);

\path[draw=drawColor,line width= 0.4pt,line join=round,line cap=round,fill=fillColor] (208.28, 47.20) circle (  1.16);

\path[draw=drawColor,line width= 0.4pt,line join=round,line cap=round,fill=fillColor] (208.44, 47.20) circle (  1.16);

\path[draw=drawColor,line width= 0.4pt,line join=round,line cap=round,fill=fillColor] (208.60, 47.20) circle (  1.16);

\path[draw=drawColor,line width= 0.4pt,line join=round,line cap=round,fill=fillColor] (208.76, 47.20) circle (  1.16);

\path[draw=drawColor,line width= 0.4pt,line join=round,line cap=round,fill=fillColor] (208.92, 47.20) circle (  1.16);

\path[draw=drawColor,line width= 0.4pt,line join=round,line cap=round,fill=fillColor] (209.08, 47.20) circle (  1.16);

\path[draw=drawColor,line width= 0.4pt,line join=round,line cap=round,fill=fillColor] (209.24, 47.20) circle (  1.16);

\path[draw=drawColor,line width= 0.4pt,line join=round,line cap=round,fill=fillColor] (209.39, 47.20) circle (  1.16);

\path[draw=drawColor,line width= 0.4pt,line join=round,line cap=round,fill=fillColor] (209.55, 47.20) circle (  1.16);

\path[draw=drawColor,line width= 0.4pt,line join=round,line cap=round,fill=fillColor] (209.71, 47.20) circle (  1.16);

\path[draw=drawColor,line width= 0.4pt,line join=round,line cap=round,fill=fillColor] (209.87, 47.20) circle (  1.16);

\path[draw=drawColor,line width= 0.4pt,line join=round,line cap=round,fill=fillColor] (210.02, 47.20) circle (  1.16);

\path[draw=drawColor,line width= 0.4pt,line join=round,line cap=round,fill=fillColor] (210.18, 47.20) circle (  1.16);

\path[draw=drawColor,line width= 0.4pt,line join=round,line cap=round,fill=fillColor] (210.34, 47.20) circle (  1.16);

\path[draw=drawColor,line width= 0.4pt,line join=round,line cap=round,fill=fillColor] (210.49, 47.20) circle (  1.16);

\path[draw=drawColor,line width= 0.4pt,line join=round,line cap=round,fill=fillColor] (210.65, 47.20) circle (  1.16);

\path[draw=drawColor,line width= 0.4pt,line join=round,line cap=round,fill=fillColor] (210.80, 47.20) circle (  1.16);

\path[draw=drawColor,line width= 0.4pt,line join=round,line cap=round,fill=fillColor] (210.96, 47.20) circle (  1.16);

\path[draw=drawColor,line width= 0.4pt,line join=round,line cap=round,fill=fillColor] (211.11, 47.20) circle (  1.16);

\path[draw=drawColor,line width= 0.4pt,line join=round,line cap=round,fill=fillColor] (211.27, 47.20) circle (  1.16);

\path[draw=drawColor,line width= 0.4pt,line join=round,line cap=round,fill=fillColor] (211.42, 47.20) circle (  1.16);

\path[draw=drawColor,line width= 0.4pt,line join=round,line cap=round,fill=fillColor] (211.57, 47.20) circle (  1.16);

\path[draw=drawColor,line width= 0.4pt,line join=round,line cap=round,fill=fillColor] (211.73, 47.20) circle (  1.16);

\path[draw=drawColor,line width= 0.4pt,line join=round,line cap=round,fill=fillColor] (211.88, 47.20) circle (  1.16);

\path[draw=drawColor,line width= 0.4pt,line join=round,line cap=round,fill=fillColor] (212.03, 47.20) circle (  1.16);

\path[draw=drawColor,line width= 0.4pt,line join=round,line cap=round,fill=fillColor] (212.19, 47.20) circle (  1.16);

\path[draw=drawColor,line width= 0.4pt,line join=round,line cap=round,fill=fillColor] (212.34, 47.20) circle (  1.16);

\path[draw=drawColor,line width= 0.4pt,line join=round,line cap=round,fill=fillColor] (212.49, 47.20) circle (  1.16);

\path[draw=drawColor,line width= 0.4pt,line join=round,line cap=round,fill=fillColor] (212.64, 47.20) circle (  1.16);

\path[draw=drawColor,line width= 0.4pt,line join=round,line cap=round,fill=fillColor] (212.79, 47.20) circle (  1.16);

\path[draw=drawColor,line width= 0.4pt,line join=round,line cap=round,fill=fillColor] (212.94, 47.20) circle (  1.16);

\path[draw=drawColor,line width= 0.4pt,line join=round,line cap=round,fill=fillColor] (213.09, 47.20) circle (  1.16);

\path[draw=drawColor,line width= 0.4pt,line join=round,line cap=round,fill=fillColor] (213.25, 47.20) circle (  1.16);

\path[draw=drawColor,line width= 0.4pt,line join=round,line cap=round,fill=fillColor] (213.40, 47.20) circle (  1.16);

\path[draw=drawColor,line width= 0.4pt,line join=round,line cap=round,fill=fillColor] (213.55, 47.20) circle (  1.16);

\path[draw=drawColor,line width= 0.4pt,line join=round,line cap=round,fill=fillColor] (213.70, 47.20) circle (  1.16);

\path[draw=drawColor,line width= 0.4pt,line join=round,line cap=round,fill=fillColor] (213.84, 47.20) circle (  1.16);

\path[draw=drawColor,line width= 0.4pt,line join=round,line cap=round,fill=fillColor] (213.99, 47.20) circle (  1.16);

\path[draw=drawColor,line width= 0.4pt,line join=round,line cap=round,fill=fillColor] (214.14, 47.20) circle (  1.16);

\path[draw=drawColor,line width= 0.4pt,line join=round,line cap=round,fill=fillColor] (214.29, 47.20) circle (  1.16);

\path[draw=drawColor,line width= 0.4pt,line join=round,line cap=round,fill=fillColor] (214.44, 47.20) circle (  1.16);

\path[draw=drawColor,line width= 0.4pt,line join=round,line cap=round,fill=fillColor] (214.59, 47.20) circle (  1.16);

\path[draw=drawColor,line width= 0.4pt,line join=round,line cap=round,fill=fillColor] (214.74, 47.20) circle (  1.16);

\path[draw=drawColor,line width= 0.4pt,line join=round,line cap=round,fill=fillColor] (214.88, 47.20) circle (  1.16);

\path[draw=drawColor,line width= 0.4pt,line join=round,line cap=round,fill=fillColor] (215.03, 47.20) circle (  1.16);

\path[draw=drawColor,line width= 0.4pt,line join=round,line cap=round,fill=fillColor] (215.18, 47.20) circle (  1.16);

\path[draw=drawColor,line width= 0.4pt,line join=round,line cap=round,fill=fillColor] (215.32, 47.20) circle (  1.16);

\path[draw=drawColor,line width= 0.4pt,line join=round,line cap=round,fill=fillColor] (215.47, 47.20) circle (  1.16);

\path[draw=drawColor,line width= 0.4pt,line join=round,line cap=round,fill=fillColor] (215.62, 47.20) circle (  1.16);

\path[draw=drawColor,line width= 0.4pt,line join=round,line cap=round,fill=fillColor] (215.76, 47.20) circle (  1.16);

\path[draw=drawColor,line width= 0.4pt,line join=round,line cap=round,fill=fillColor] (215.91, 47.20) circle (  1.16);

\path[draw=drawColor,line width= 0.4pt,line join=round,line cap=round,fill=fillColor] (216.05, 47.20) circle (  1.16);

\path[draw=drawColor,line width= 0.4pt,line join=round,line cap=round,fill=fillColor] (216.20, 47.20) circle (  1.16);

\path[draw=drawColor,line width= 0.4pt,line join=round,line cap=round,fill=fillColor] (216.34, 47.20) circle (  1.16);

\path[draw=drawColor,line width= 0.4pt,line join=round,line cap=round,fill=fillColor] (216.49, 47.20) circle (  1.16);

\path[draw=drawColor,line width= 0.4pt,line join=round,line cap=round,fill=fillColor] (216.63, 47.20) circle (  1.16);

\path[draw=drawColor,line width= 0.4pt,line join=round,line cap=round,fill=fillColor] (216.78, 47.20) circle (  1.16);

\path[draw=drawColor,line width= 0.4pt,line join=round,line cap=round,fill=fillColor] (216.92, 47.20) circle (  1.16);

\path[draw=drawColor,line width= 0.4pt,line join=round,line cap=round,fill=fillColor] (217.06, 47.20) circle (  1.16);

\path[draw=drawColor,line width= 0.4pt,line join=round,line cap=round,fill=fillColor] (217.21, 47.20) circle (  1.16);

\path[draw=drawColor,line width= 0.4pt,line join=round,line cap=round,fill=fillColor] (217.35, 47.20) circle (  1.16);

\path[draw=drawColor,line width= 0.4pt,line join=round,line cap=round,fill=fillColor] (217.49, 47.20) circle (  1.16);

\path[draw=drawColor,line width= 0.4pt,line join=round,line cap=round,fill=fillColor] (217.64, 47.20) circle (  1.16);

\path[draw=drawColor,line width= 0.4pt,line join=round,line cap=round,fill=fillColor] (217.78, 47.20) circle (  1.16);

\path[draw=drawColor,line width= 0.4pt,line join=round,line cap=round,fill=fillColor] (217.92, 47.20) circle (  1.16);

\path[draw=drawColor,line width= 0.4pt,line join=round,line cap=round,fill=fillColor] (218.06, 47.20) circle (  1.16);

\path[draw=drawColor,line width= 0.4pt,line join=round,line cap=round,fill=fillColor] (218.20, 47.20) circle (  1.16);

\path[draw=drawColor,line width= 0.4pt,line join=round,line cap=round,fill=fillColor] (218.35, 47.20) circle (  1.16);

\path[draw=drawColor,line width= 0.4pt,line join=round,line cap=round,fill=fillColor] (218.49, 47.20) circle (  1.16);

\path[draw=drawColor,line width= 0.4pt,line join=round,line cap=round,fill=fillColor] (218.63, 47.20) circle (  1.16);

\path[draw=drawColor,line width= 0.4pt,line join=round,line cap=round,fill=fillColor] (218.77, 47.20) circle (  1.16);

\path[draw=drawColor,line width= 0.4pt,line join=round,line cap=round,fill=fillColor] (218.91, 47.20) circle (  1.16);

\path[draw=drawColor,line width= 0.4pt,line join=round,line cap=round,fill=fillColor] (219.05, 47.20) circle (  1.16);

\path[draw=drawColor,line width= 0.4pt,line join=round,line cap=round,fill=fillColor] (219.19, 47.20) circle (  1.16);

\path[draw=drawColor,line width= 0.4pt,line join=round,line cap=round,fill=fillColor] (219.33, 47.20) circle (  1.16);

\path[draw=drawColor,line width= 0.4pt,line join=round,line cap=round,fill=fillColor] (219.47, 47.20) circle (  1.16);

\path[draw=drawColor,line width= 0.4pt,line join=round,line cap=round,fill=fillColor] (219.61, 47.20) circle (  1.16);

\path[draw=drawColor,line width= 0.4pt,line join=round,line cap=round,fill=fillColor] (219.75, 47.20) circle (  1.16);

\path[draw=drawColor,line width= 0.4pt,line join=round,line cap=round,fill=fillColor] (219.89, 47.20) circle (  1.16);

\path[draw=drawColor,line width= 0.4pt,line join=round,line cap=round,fill=fillColor] (220.02, 47.20) circle (  1.16);

\path[draw=drawColor,line width= 0.4pt,line join=round,line cap=round,fill=fillColor] (220.16, 47.20) circle (  1.16);

\path[draw=drawColor,line width= 0.4pt,line join=round,line cap=round,fill=fillColor] (220.30, 47.20) circle (  1.16);

\path[draw=drawColor,line width= 0.4pt,line join=round,line cap=round,fill=fillColor] (220.44, 47.20) circle (  1.16);

\path[draw=drawColor,line width= 0.4pt,line join=round,line cap=round,fill=fillColor] (220.58, 47.20) circle (  1.16);

\path[draw=drawColor,line width= 0.4pt,line join=round,line cap=round,fill=fillColor] (220.71, 47.20) circle (  1.16);

\path[draw=drawColor,line width= 0.4pt,line join=round,line cap=round,fill=fillColor] (220.85, 47.20) circle (  1.16);

\path[draw=drawColor,line width= 0.4pt,line join=round,line cap=round,fill=fillColor] (220.99, 47.20) circle (  1.16);

\path[draw=drawColor,line width= 0.4pt,line join=round,line cap=round,fill=fillColor] (221.13, 47.20) circle (  1.16);

\path[draw=drawColor,line width= 0.4pt,line join=round,line cap=round,fill=fillColor] (221.26, 47.20) circle (  1.16);

\path[draw=drawColor,line width= 0.4pt,line join=round,line cap=round,fill=fillColor] (221.40, 47.20) circle (  1.16);

\path[draw=drawColor,line width= 0.4pt,line join=round,line cap=round,fill=fillColor] (221.53, 47.20) circle (  1.16);

\path[draw=drawColor,line width= 0.4pt,line join=round,line cap=round,fill=fillColor] (221.67, 47.20) circle (  1.16);

\path[draw=drawColor,line width= 0.4pt,line join=round,line cap=round,fill=fillColor] (221.81, 47.20) circle (  1.16);

\path[draw=drawColor,line width= 0.4pt,line join=round,line cap=round,fill=fillColor] (221.94, 47.20) circle (  1.16);

\path[draw=drawColor,line width= 0.4pt,line join=round,line cap=round,fill=fillColor] (222.08, 47.20) circle (  1.16);

\path[draw=drawColor,line width= 0.4pt,line join=round,line cap=round,fill=fillColor] (222.21, 47.20) circle (  1.16);

\path[draw=drawColor,line width= 0.4pt,line join=round,line cap=round,fill=fillColor] (222.35, 47.20) circle (  1.16);

\path[draw=drawColor,line width= 0.4pt,line join=round,line cap=round,fill=fillColor] (222.48, 47.20) circle (  1.16);

\path[draw=drawColor,line width= 0.4pt,line join=round,line cap=round,fill=fillColor] (222.62, 47.20) circle (  1.16);

\path[draw=drawColor,line width= 0.4pt,line join=round,line cap=round,fill=fillColor] (222.75, 47.20) circle (  1.16);

\path[draw=drawColor,line width= 0.4pt,line join=round,line cap=round,fill=fillColor] (222.88, 47.20) circle (  1.16);

\path[draw=drawColor,line width= 0.4pt,line join=round,line cap=round,fill=fillColor] (223.02, 47.20) circle (  1.16);

\path[draw=drawColor,line width= 0.4pt,line join=round,line cap=round,fill=fillColor] (223.15, 47.20) circle (  1.16);

\path[draw=drawColor,line width= 0.4pt,line join=round,line cap=round,fill=fillColor] (223.28, 47.20) circle (  1.16);

\path[draw=drawColor,line width= 0.4pt,line join=round,line cap=round,fill=fillColor] (223.42, 47.20) circle (  1.16);

\path[draw=drawColor,line width= 0.4pt,line join=round,line cap=round,fill=fillColor] (223.55, 47.20) circle (  1.16);

\path[draw=drawColor,line width= 0.4pt,line join=round,line cap=round,fill=fillColor] (223.68, 47.20) circle (  1.16);

\path[draw=drawColor,line width= 0.4pt,line join=round,line cap=round,fill=fillColor] (223.82, 47.20) circle (  1.16);

\path[draw=drawColor,line width= 0.4pt,line join=round,line cap=round,fill=fillColor] (223.95, 47.20) circle (  1.16);

\path[draw=drawColor,line width= 0.4pt,line join=round,line cap=round,fill=fillColor] (224.08, 47.20) circle (  1.16);

\path[draw=drawColor,line width= 0.4pt,line join=round,line cap=round,fill=fillColor] (224.21, 47.20) circle (  1.16);

\path[draw=drawColor,line width= 0.4pt,line join=round,line cap=round,fill=fillColor] (224.34, 47.20) circle (  1.16);

\path[draw=drawColor,line width= 0.4pt,line join=round,line cap=round,fill=fillColor] (224.48, 47.20) circle (  1.16);

\path[draw=drawColor,line width= 0.4pt,line join=round,line cap=round,fill=fillColor] (224.61, 47.20) circle (  1.16);

\path[draw=drawColor,line width= 0.4pt,line join=round,line cap=round,fill=fillColor] (224.74, 47.20) circle (  1.16);

\path[draw=drawColor,line width= 0.4pt,line join=round,line cap=round,fill=fillColor] (224.87, 47.20) circle (  1.16);

\path[draw=drawColor,line width= 0.4pt,line join=round,line cap=round,fill=fillColor] (225.00, 47.20) circle (  1.16);

\path[draw=drawColor,line width= 0.4pt,line join=round,line cap=round,fill=fillColor] (225.13, 47.20) circle (  1.16);

\path[draw=drawColor,line width= 0.4pt,line join=round,line cap=round,fill=fillColor] (225.26, 47.20) circle (  1.16);
\definecolor[named]{drawColor}{rgb}{0.22,0.49,0.72}
\definecolor[named]{fillColor}{rgb}{0.22,0.49,0.72}

\path[draw=drawColor,line width= 0.4pt,line join=round,line cap=round,fill=fillColor] ( 74.88,125.28) circle (  1.16);

\path[draw=drawColor,line width= 0.4pt,line join=round,line cap=round,fill=fillColor] ( 80.78,123.50) circle (  1.16);

\path[draw=drawColor,line width= 0.4pt,line join=round,line cap=round,fill=fillColor] ( 84.92,123.08) circle (  1.16);

\path[draw=drawColor,line width= 0.4pt,line join=round,line cap=round,fill=fillColor] ( 88.22,121.68) circle (  1.16);

\path[draw=drawColor,line width= 0.4pt,line join=round,line cap=round,fill=fillColor] ( 91.00,121.42) circle (  1.16);

\path[draw=drawColor,line width= 0.4pt,line join=round,line cap=round,fill=fillColor] ( 93.43,118.07) circle (  1.16);

\path[draw=drawColor,line width= 0.4pt,line join=round,line cap=round,fill=fillColor] ( 95.61,116.48) circle (  1.16);

\path[draw=drawColor,line width= 0.4pt,line join=round,line cap=round,fill=fillColor] ( 97.58,116.32) circle (  1.16);

\path[draw=drawColor,line width= 0.4pt,line join=round,line cap=round,fill=fillColor] ( 99.40,116.17) circle (  1.16);

\path[draw=drawColor,line width= 0.4pt,line join=round,line cap=round,fill=fillColor] (101.09,115.91) circle (  1.16);

\path[draw=drawColor,line width= 0.4pt,line join=round,line cap=round,fill=fillColor] (102.67,115.79) circle (  1.16);

\path[draw=drawColor,line width= 0.4pt,line join=round,line cap=round,fill=fillColor] (104.16,115.03) circle (  1.16);

\path[draw=drawColor,line width= 0.4pt,line join=round,line cap=round,fill=fillColor] (105.56,109.65) circle (  1.16);

\path[draw=drawColor,line width= 0.4pt,line join=round,line cap=round,fill=fillColor] (106.90,109.27) circle (  1.16);

\path[draw=drawColor,line width= 0.4pt,line join=round,line cap=round,fill=fillColor] (108.17,109.06) circle (  1.16);

\path[draw=drawColor,line width= 0.4pt,line join=round,line cap=round,fill=fillColor] (109.39,108.92) circle (  1.16);

\path[draw=drawColor,line width= 0.4pt,line join=round,line cap=round,fill=fillColor] (110.56,108.57) circle (  1.16);

\path[draw=drawColor,line width= 0.4pt,line join=round,line cap=round,fill=fillColor] (111.68,108.10) circle (  1.16);

\path[draw=drawColor,line width= 0.4pt,line join=round,line cap=round,fill=fillColor] (112.76,107.62) circle (  1.16);

\path[draw=drawColor,line width= 0.4pt,line join=round,line cap=round,fill=fillColor] (113.81,107.30) circle (  1.16);

\path[draw=drawColor,line width= 0.4pt,line join=round,line cap=round,fill=fillColor] (114.82,106.69) circle (  1.16);

\path[draw=drawColor,line width= 0.4pt,line join=round,line cap=round,fill=fillColor] (115.80,104.63) circle (  1.16);

\path[draw=drawColor,line width= 0.4pt,line join=round,line cap=round,fill=fillColor] (116.74,104.41) circle (  1.16);

\path[draw=drawColor,line width= 0.4pt,line join=round,line cap=round,fill=fillColor] (117.67,104.20) circle (  1.16);

\path[draw=drawColor,line width= 0.4pt,line join=round,line cap=round,fill=fillColor] (118.56,102.60) circle (  1.16);

\path[draw=drawColor,line width= 0.4pt,line join=round,line cap=round,fill=fillColor] (119.44,102.36) circle (  1.16);

\path[draw=drawColor,line width= 0.4pt,line join=round,line cap=round,fill=fillColor] (120.29,101.99) circle (  1.16);

\path[draw=drawColor,line width= 0.4pt,line join=round,line cap=round,fill=fillColor] (121.12,101.94) circle (  1.16);

\path[draw=drawColor,line width= 0.4pt,line join=round,line cap=round,fill=fillColor] (121.93,101.01) circle (  1.16);

\path[draw=drawColor,line width= 0.4pt,line join=round,line cap=round,fill=fillColor] (122.72,100.00) circle (  1.16);

\path[draw=drawColor,line width= 0.4pt,line join=round,line cap=round,fill=fillColor] (123.50, 99.47) circle (  1.16);

\path[draw=drawColor,line width= 0.4pt,line join=round,line cap=round,fill=fillColor] (124.26, 99.03) circle (  1.16);

\path[draw=drawColor,line width= 0.4pt,line join=round,line cap=round,fill=fillColor] (125.00, 98.61) circle (  1.16);

\path[draw=drawColor,line width= 0.4pt,line join=round,line cap=round,fill=fillColor] (125.73, 98.56) circle (  1.16);

\path[draw=drawColor,line width= 0.4pt,line join=round,line cap=round,fill=fillColor] (126.44, 98.48) circle (  1.16);

\path[draw=drawColor,line width= 0.4pt,line join=round,line cap=round,fill=fillColor] (127.15, 98.25) circle (  1.16);

\path[draw=drawColor,line width= 0.4pt,line join=round,line cap=round,fill=fillColor] (127.83, 97.50) circle (  1.16);

\path[draw=drawColor,line width= 0.4pt,line join=round,line cap=round,fill=fillColor] (128.51, 96.93) circle (  1.16);

\path[draw=drawColor,line width= 0.4pt,line join=round,line cap=round,fill=fillColor] (129.17, 94.82) circle (  1.16);

\path[draw=drawColor,line width= 0.4pt,line join=round,line cap=round,fill=fillColor] (129.83, 94.77) circle (  1.16);

\path[draw=drawColor,line width= 0.4pt,line join=round,line cap=round,fill=fillColor] (130.47, 94.73) circle (  1.16);

\path[draw=drawColor,line width= 0.4pt,line join=round,line cap=round,fill=fillColor] (131.10, 94.46) circle (  1.16);

\path[draw=drawColor,line width= 0.4pt,line join=round,line cap=round,fill=fillColor] (131.72, 94.44) circle (  1.16);

\path[draw=drawColor,line width= 0.4pt,line join=round,line cap=round,fill=fillColor] (132.33, 93.78) circle (  1.16);

\path[draw=drawColor,line width= 0.4pt,line join=round,line cap=round,fill=fillColor] (132.93, 93.71) circle (  1.16);

\path[draw=drawColor,line width= 0.4pt,line join=round,line cap=round,fill=fillColor] (133.53, 92.26) circle (  1.16);

\path[draw=drawColor,line width= 0.4pt,line join=round,line cap=round,fill=fillColor] (134.11, 92.10) circle (  1.16);

\path[draw=drawColor,line width= 0.4pt,line join=round,line cap=round,fill=fillColor] (134.69, 91.89) circle (  1.16);

\path[draw=drawColor,line width= 0.4pt,line join=round,line cap=round,fill=fillColor] (135.26, 91.71) circle (  1.16);

\path[draw=drawColor,line width= 0.4pt,line join=round,line cap=round,fill=fillColor] (135.82, 91.55) circle (  1.16);

\path[draw=drawColor,line width= 0.4pt,line join=round,line cap=round,fill=fillColor] (136.38, 90.92) circle (  1.16);

\path[draw=drawColor,line width= 0.4pt,line join=round,line cap=round,fill=fillColor] (136.92, 90.84) circle (  1.16);

\path[draw=drawColor,line width= 0.4pt,line join=round,line cap=round,fill=fillColor] (137.46, 90.70) circle (  1.16);

\path[draw=drawColor,line width= 0.4pt,line join=round,line cap=round,fill=fillColor] (137.99, 90.53) circle (  1.16);

\path[draw=drawColor,line width= 0.4pt,line join=round,line cap=round,fill=fillColor] (138.52, 90.33) circle (  1.16);

\path[draw=drawColor,line width= 0.4pt,line join=round,line cap=round,fill=fillColor] (139.04, 90.13) circle (  1.16);

\path[draw=drawColor,line width= 0.4pt,line join=round,line cap=round,fill=fillColor] (139.56, 90.09) circle (  1.16);

\path[draw=drawColor,line width= 0.4pt,line join=round,line cap=round,fill=fillColor] (140.06, 90.02) circle (  1.16);

\path[draw=drawColor,line width= 0.4pt,line join=round,line cap=round,fill=fillColor] (140.57, 89.53) circle (  1.16);

\path[draw=drawColor,line width= 0.4pt,line join=round,line cap=round,fill=fillColor] (141.06, 89.48) circle (  1.16);

\path[draw=drawColor,line width= 0.4pt,line join=round,line cap=round,fill=fillColor] (141.55, 89.18) circle (  1.16);

\path[draw=drawColor,line width= 0.4pt,line join=round,line cap=round,fill=fillColor] (142.04, 88.52) circle (  1.16);

\path[draw=drawColor,line width= 0.4pt,line join=round,line cap=round,fill=fillColor] (142.52, 88.10) circle (  1.16);

\path[draw=drawColor,line width= 0.4pt,line join=round,line cap=round,fill=fillColor] (143.00, 88.03) circle (  1.16);

\path[draw=drawColor,line width= 0.4pt,line join=round,line cap=round,fill=fillColor] (143.47, 88.00) circle (  1.16);

\path[draw=drawColor,line width= 0.4pt,line join=round,line cap=round,fill=fillColor] (143.93, 87.89) circle (  1.16);

\path[draw=drawColor,line width= 0.4pt,line join=round,line cap=round,fill=fillColor] (144.39, 87.62) circle (  1.16);

\path[draw=drawColor,line width= 0.4pt,line join=round,line cap=round,fill=fillColor] (144.85, 87.47) circle (  1.16);

\path[draw=drawColor,line width= 0.4pt,line join=round,line cap=round,fill=fillColor] (145.30, 87.11) circle (  1.16);

\path[draw=drawColor,line width= 0.4pt,line join=round,line cap=round,fill=fillColor] (145.75, 86.99) circle (  1.16);

\path[draw=drawColor,line width= 0.4pt,line join=round,line cap=round,fill=fillColor] (146.19, 86.94) circle (  1.16);

\path[draw=drawColor,line width= 0.4pt,line join=round,line cap=round,fill=fillColor] (146.63, 86.84) circle (  1.16);

\path[draw=drawColor,line width= 0.4pt,line join=round,line cap=round,fill=fillColor] (147.07, 86.76) circle (  1.16);

\path[draw=drawColor,line width= 0.4pt,line join=round,line cap=round,fill=fillColor] (147.50, 86.71) circle (  1.16);

\path[draw=drawColor,line width= 0.4pt,line join=round,line cap=round,fill=fillColor] (147.93, 86.49) circle (  1.16);

\path[draw=drawColor,line width= 0.4pt,line join=round,line cap=round,fill=fillColor] (148.35, 85.78) circle (  1.16);

\path[draw=drawColor,line width= 0.4pt,line join=round,line cap=round,fill=fillColor] (148.77, 84.80) circle (  1.16);

\path[draw=drawColor,line width= 0.4pt,line join=round,line cap=round,fill=fillColor] (149.19, 84.68) circle (  1.16);

\path[draw=drawColor,line width= 0.4pt,line join=round,line cap=round,fill=fillColor] (149.60, 84.57) circle (  1.16);

\path[draw=drawColor,line width= 0.4pt,line join=round,line cap=round,fill=fillColor] (150.01, 84.51) circle (  1.16);

\path[draw=drawColor,line width= 0.4pt,line join=round,line cap=round,fill=fillColor] (150.41, 84.46) circle (  1.16);

\path[draw=drawColor,line width= 0.4pt,line join=round,line cap=round,fill=fillColor] (150.82, 84.19) circle (  1.16);

\path[draw=drawColor,line width= 0.4pt,line join=round,line cap=round,fill=fillColor] (151.22, 84.19) circle (  1.16);

\path[draw=drawColor,line width= 0.4pt,line join=round,line cap=round,fill=fillColor] (151.61, 84.15) circle (  1.16);

\path[draw=drawColor,line width= 0.4pt,line join=round,line cap=round,fill=fillColor] (152.01, 83.79) circle (  1.16);

\path[draw=drawColor,line width= 0.4pt,line join=round,line cap=round,fill=fillColor] (152.40, 83.63) circle (  1.16);

\path[draw=drawColor,line width= 0.4pt,line join=round,line cap=round,fill=fillColor] (152.78, 83.52) circle (  1.16);

\path[draw=drawColor,line width= 0.4pt,line join=round,line cap=round,fill=fillColor] (153.17, 83.50) circle (  1.16);

\path[draw=drawColor,line width= 0.4pt,line join=round,line cap=round,fill=fillColor] (153.55, 82.67) circle (  1.16);

\path[draw=drawColor,line width= 0.4pt,line join=round,line cap=round,fill=fillColor] (153.93, 82.58) circle (  1.16);

\path[draw=drawColor,line width= 0.4pt,line join=round,line cap=round,fill=fillColor] (154.30, 82.48) circle (  1.16);

\path[draw=drawColor,line width= 0.4pt,line join=round,line cap=round,fill=fillColor] (154.67, 82.44) circle (  1.16);

\path[draw=drawColor,line width= 0.4pt,line join=round,line cap=round,fill=fillColor] (155.04, 82.40) circle (  1.16);

\path[draw=drawColor,line width= 0.4pt,line join=round,line cap=round,fill=fillColor] (155.41, 82.15) circle (  1.16);

\path[draw=drawColor,line width= 0.4pt,line join=round,line cap=round,fill=fillColor] (155.78, 81.96) circle (  1.16);

\path[draw=drawColor,line width= 0.4pt,line join=round,line cap=round,fill=fillColor] (156.14, 81.74) circle (  1.16);

\path[draw=drawColor,line width= 0.4pt,line join=round,line cap=round,fill=fillColor] (156.50, 81.70) circle (  1.16);

\path[draw=drawColor,line width= 0.4pt,line join=round,line cap=round,fill=fillColor] (156.86, 81.68) circle (  1.16);

\path[draw=drawColor,line width= 0.4pt,line join=round,line cap=round,fill=fillColor] (157.21, 81.68) circle (  1.16);

\path[draw=drawColor,line width= 0.4pt,line join=round,line cap=round,fill=fillColor] (157.56, 81.41) circle (  1.16);

\path[draw=drawColor,line width= 0.4pt,line join=round,line cap=round,fill=fillColor] (157.91, 81.34) circle (  1.16);

\path[draw=drawColor,line width= 0.4pt,line join=round,line cap=round,fill=fillColor] (158.26, 81.32) circle (  1.16);

\path[draw=drawColor,line width= 0.4pt,line join=round,line cap=round,fill=fillColor] (158.61, 81.29) circle (  1.16);

\path[draw=drawColor,line width= 0.4pt,line join=round,line cap=round,fill=fillColor] (158.95, 81.27) circle (  1.16);

\path[draw=drawColor,line width= 0.4pt,line join=round,line cap=round,fill=fillColor] (159.29, 81.20) circle (  1.16);

\path[draw=drawColor,line width= 0.4pt,line join=round,line cap=round,fill=fillColor] (159.63, 81.15) circle (  1.16);

\path[draw=drawColor,line width= 0.4pt,line join=round,line cap=round,fill=fillColor] (159.97, 81.08) circle (  1.16);

\path[draw=drawColor,line width= 0.4pt,line join=round,line cap=round,fill=fillColor] (160.30, 80.99) circle (  1.16);

\path[draw=drawColor,line width= 0.4pt,line join=round,line cap=round,fill=fillColor] (160.63, 80.99) circle (  1.16);

\path[draw=drawColor,line width= 0.4pt,line join=round,line cap=round,fill=fillColor] (160.96, 80.90) circle (  1.16);

\path[draw=drawColor,line width= 0.4pt,line join=round,line cap=round,fill=fillColor] (161.29, 80.82) circle (  1.16);

\path[draw=drawColor,line width= 0.4pt,line join=round,line cap=round,fill=fillColor] (161.62, 80.70) circle (  1.16);

\path[draw=drawColor,line width= 0.4pt,line join=round,line cap=round,fill=fillColor] (161.95, 80.64) circle (  1.16);

\path[draw=drawColor,line width= 0.4pt,line join=round,line cap=round,fill=fillColor] (162.27, 80.54) circle (  1.16);

\path[draw=drawColor,line width= 0.4pt,line join=round,line cap=round,fill=fillColor] (162.59, 80.51) circle (  1.16);

\path[draw=drawColor,line width= 0.4pt,line join=round,line cap=round,fill=fillColor] (162.91, 80.46) circle (  1.16);

\path[draw=drawColor,line width= 0.4pt,line join=round,line cap=round,fill=fillColor] (163.23, 80.44) circle (  1.16);

\path[draw=drawColor,line width= 0.4pt,line join=round,line cap=round,fill=fillColor] (163.54, 80.32) circle (  1.16);

\path[draw=drawColor,line width= 0.4pt,line join=round,line cap=round,fill=fillColor] (163.85, 80.15) circle (  1.16);

\path[draw=drawColor,line width= 0.4pt,line join=round,line cap=round,fill=fillColor] (164.17, 80.09) circle (  1.16);

\path[draw=drawColor,line width= 0.4pt,line join=round,line cap=round,fill=fillColor] (164.48, 79.96) circle (  1.16);

\path[draw=drawColor,line width= 0.4pt,line join=round,line cap=round,fill=fillColor] (164.79, 79.87) circle (  1.16);

\path[draw=drawColor,line width= 0.4pt,line join=round,line cap=round,fill=fillColor] (165.09, 79.83) circle (  1.16);

\path[draw=drawColor,line width= 0.4pt,line join=round,line cap=round,fill=fillColor] (165.40, 79.79) circle (  1.16);

\path[draw=drawColor,line width= 0.4pt,line join=round,line cap=round,fill=fillColor] (165.70, 79.67) circle (  1.16);

\path[draw=drawColor,line width= 0.4pt,line join=round,line cap=round,fill=fillColor] (166.00, 79.57) circle (  1.16);

\path[draw=drawColor,line width= 0.4pt,line join=round,line cap=round,fill=fillColor] (166.30, 79.23) circle (  1.16);

\path[draw=drawColor,line width= 0.4pt,line join=round,line cap=round,fill=fillColor] (166.60, 79.04) circle (  1.16);

\path[draw=drawColor,line width= 0.4pt,line join=round,line cap=round,fill=fillColor] (166.90, 78.90) circle (  1.16);

\path[draw=drawColor,line width= 0.4pt,line join=round,line cap=round,fill=fillColor] (167.19, 78.87) circle (  1.16);

\path[draw=drawColor,line width= 0.4pt,line join=round,line cap=round,fill=fillColor] (167.49, 78.80) circle (  1.16);

\path[draw=drawColor,line width= 0.4pt,line join=round,line cap=round,fill=fillColor] (167.78, 78.73) circle (  1.16);

\path[draw=drawColor,line width= 0.4pt,line join=round,line cap=round,fill=fillColor] (168.07, 78.68) circle (  1.16);

\path[draw=drawColor,line width= 0.4pt,line join=round,line cap=round,fill=fillColor] (168.36, 78.13) circle (  1.16);

\path[draw=drawColor,line width= 0.4pt,line join=round,line cap=round,fill=fillColor] (168.65, 78.08) circle (  1.16);

\path[draw=drawColor,line width= 0.4pt,line join=round,line cap=round,fill=fillColor] (168.94, 78.04) circle (  1.16);

\path[draw=drawColor,line width= 0.4pt,line join=round,line cap=round,fill=fillColor] (169.22, 77.94) circle (  1.16);

\path[draw=drawColor,line width= 0.4pt,line join=round,line cap=round,fill=fillColor] (169.51, 77.81) circle (  1.16);

\path[draw=drawColor,line width= 0.4pt,line join=round,line cap=round,fill=fillColor] (169.79, 77.76) circle (  1.16);

\path[draw=drawColor,line width= 0.4pt,line join=round,line cap=round,fill=fillColor] (170.07, 77.70) circle (  1.16);

\path[draw=drawColor,line width= 0.4pt,line join=round,line cap=round,fill=fillColor] (170.35, 77.69) circle (  1.16);

\path[draw=drawColor,line width= 0.4pt,line join=round,line cap=round,fill=fillColor] (170.63, 77.55) circle (  1.16);

\path[draw=drawColor,line width= 0.4pt,line join=round,line cap=round,fill=fillColor] (170.91, 77.54) circle (  1.16);

\path[draw=drawColor,line width= 0.4pt,line join=round,line cap=round,fill=fillColor] (171.18, 77.50) circle (  1.16);

\path[draw=drawColor,line width= 0.4pt,line join=round,line cap=round,fill=fillColor] (171.46, 77.31) circle (  1.16);

\path[draw=drawColor,line width= 0.4pt,line join=round,line cap=round,fill=fillColor] (171.73, 77.20) circle (  1.16);

\path[draw=drawColor,line width= 0.4pt,line join=round,line cap=round,fill=fillColor] (172.00, 77.19) circle (  1.16);

\path[draw=drawColor,line width= 0.4pt,line join=round,line cap=round,fill=fillColor] (172.28, 77.17) circle (  1.16);

\path[draw=drawColor,line width= 0.4pt,line join=round,line cap=round,fill=fillColor] (172.55, 77.15) circle (  1.16);

\path[draw=drawColor,line width= 0.4pt,line join=round,line cap=round,fill=fillColor] (172.81, 76.99) circle (  1.16);

\path[draw=drawColor,line width= 0.4pt,line join=round,line cap=round,fill=fillColor] (173.08, 76.89) circle (  1.16);

\path[draw=drawColor,line width= 0.4pt,line join=round,line cap=round,fill=fillColor] (173.35, 76.88) circle (  1.16);

\path[draw=drawColor,line width= 0.4pt,line join=round,line cap=round,fill=fillColor] (173.61, 76.87) circle (  1.16);

\path[draw=drawColor,line width= 0.4pt,line join=round,line cap=round,fill=fillColor] (173.88, 76.86) circle (  1.16);

\path[draw=drawColor,line width= 0.4pt,line join=round,line cap=round,fill=fillColor] (174.14, 76.81) circle (  1.16);

\path[draw=drawColor,line width= 0.4pt,line join=round,line cap=round,fill=fillColor] (174.40, 76.81) circle (  1.16);

\path[draw=drawColor,line width= 0.4pt,line join=round,line cap=round,fill=fillColor] (174.66, 76.54) circle (  1.16);

\path[draw=drawColor,line width= 0.4pt,line join=round,line cap=round,fill=fillColor] (174.92, 76.51) circle (  1.16);

\path[draw=drawColor,line width= 0.4pt,line join=round,line cap=round,fill=fillColor] (175.18, 76.36) circle (  1.16);

\path[draw=drawColor,line width= 0.4pt,line join=round,line cap=round,fill=fillColor] (175.44, 76.25) circle (  1.16);

\path[draw=drawColor,line width= 0.4pt,line join=round,line cap=round,fill=fillColor] (175.69, 76.23) circle (  1.16);

\path[draw=drawColor,line width= 0.4pt,line join=round,line cap=round,fill=fillColor] (175.95, 75.96) circle (  1.16);

\path[draw=drawColor,line width= 0.4pt,line join=round,line cap=round,fill=fillColor] (176.20, 75.85) circle (  1.16);

\path[draw=drawColor,line width= 0.4pt,line join=round,line cap=round,fill=fillColor] (176.46, 75.71) circle (  1.16);

\path[draw=drawColor,line width= 0.4pt,line join=round,line cap=round,fill=fillColor] (176.71, 75.39) circle (  1.16);

\path[draw=drawColor,line width= 0.4pt,line join=round,line cap=round,fill=fillColor] (176.96, 75.34) circle (  1.16);

\path[draw=drawColor,line width= 0.4pt,line join=round,line cap=round,fill=fillColor] (177.21, 75.32) circle (  1.16);

\path[draw=drawColor,line width= 0.4pt,line join=round,line cap=round,fill=fillColor] (177.46, 75.32) circle (  1.16);

\path[draw=drawColor,line width= 0.4pt,line join=round,line cap=round,fill=fillColor] (177.71, 75.31) circle (  1.16);

\path[draw=drawColor,line width= 0.4pt,line join=round,line cap=round,fill=fillColor] (177.95, 75.19) circle (  1.16);

\path[draw=drawColor,line width= 0.4pt,line join=round,line cap=round,fill=fillColor] (178.20, 75.19) circle (  1.16);

\path[draw=drawColor,line width= 0.4pt,line join=round,line cap=round,fill=fillColor] (178.45, 75.11) circle (  1.16);

\path[draw=drawColor,line width= 0.4pt,line join=round,line cap=round,fill=fillColor] (178.69, 75.10) circle (  1.16);

\path[draw=drawColor,line width= 0.4pt,line join=round,line cap=round,fill=fillColor] (178.93, 75.04) circle (  1.16);

\path[draw=drawColor,line width= 0.4pt,line join=round,line cap=round,fill=fillColor] (179.18, 75.01) circle (  1.16);

\path[draw=drawColor,line width= 0.4pt,line join=round,line cap=round,fill=fillColor] (179.42, 75.00) circle (  1.16);

\path[draw=drawColor,line width= 0.4pt,line join=round,line cap=round,fill=fillColor] (179.66, 74.92) circle (  1.16);

\path[draw=drawColor,line width= 0.4pt,line join=round,line cap=round,fill=fillColor] (179.90, 74.77) circle (  1.16);

\path[draw=drawColor,line width= 0.4pt,line join=round,line cap=round,fill=fillColor] (180.14, 74.71) circle (  1.16);

\path[draw=drawColor,line width= 0.4pt,line join=round,line cap=round,fill=fillColor] (180.37, 74.66) circle (  1.16);

\path[draw=drawColor,line width= 0.4pt,line join=round,line cap=round,fill=fillColor] (180.61, 74.58) circle (  1.16);

\path[draw=drawColor,line width= 0.4pt,line join=round,line cap=round,fill=fillColor] (180.85, 74.51) circle (  1.16);

\path[draw=drawColor,line width= 0.4pt,line join=round,line cap=round,fill=fillColor] (181.08, 74.51) circle (  1.16);

\path[draw=drawColor,line width= 0.4pt,line join=round,line cap=round,fill=fillColor] (181.32, 74.51) circle (  1.16);

\path[draw=drawColor,line width= 0.4pt,line join=round,line cap=round,fill=fillColor] (181.55, 74.47) circle (  1.16);

\path[draw=drawColor,line width= 0.4pt,line join=round,line cap=round,fill=fillColor] (181.78, 74.36) circle (  1.16);

\path[draw=drawColor,line width= 0.4pt,line join=round,line cap=round,fill=fillColor] (182.01, 74.33) circle (  1.16);

\path[draw=drawColor,line width= 0.4pt,line join=round,line cap=round,fill=fillColor] (182.25, 74.33) circle (  1.16);

\path[draw=drawColor,line width= 0.4pt,line join=round,line cap=round,fill=fillColor] (182.48, 74.22) circle (  1.16);

\path[draw=drawColor,line width= 0.4pt,line join=round,line cap=round,fill=fillColor] (182.70, 74.16) circle (  1.16);

\path[draw=drawColor,line width= 0.4pt,line join=round,line cap=round,fill=fillColor] (182.93, 74.16) circle (  1.16);

\path[draw=drawColor,line width= 0.4pt,line join=round,line cap=round,fill=fillColor] (183.16, 74.09) circle (  1.16);

\path[draw=drawColor,line width= 0.4pt,line join=round,line cap=round,fill=fillColor] (183.39, 73.99) circle (  1.16);

\path[draw=drawColor,line width= 0.4pt,line join=round,line cap=round,fill=fillColor] (183.61, 73.97) circle (  1.16);

\path[draw=drawColor,line width= 0.4pt,line join=round,line cap=round,fill=fillColor] (183.84, 73.93) circle (  1.16);

\path[draw=drawColor,line width= 0.4pt,line join=round,line cap=round,fill=fillColor] (184.06, 73.89) circle (  1.16);

\path[draw=drawColor,line width= 0.4pt,line join=round,line cap=round,fill=fillColor] (184.29, 73.87) circle (  1.16);

\path[draw=drawColor,line width= 0.4pt,line join=round,line cap=round,fill=fillColor] (184.51, 73.86) circle (  1.16);

\path[draw=drawColor,line width= 0.4pt,line join=round,line cap=round,fill=fillColor] (184.73, 73.84) circle (  1.16);

\path[draw=drawColor,line width= 0.4pt,line join=round,line cap=round,fill=fillColor] (184.96, 73.78) circle (  1.16);

\path[draw=drawColor,line width= 0.4pt,line join=round,line cap=round,fill=fillColor] (185.18, 73.76) circle (  1.16);

\path[draw=drawColor,line width= 0.4pt,line join=round,line cap=round,fill=fillColor] (185.40, 73.76) circle (  1.16);

\path[draw=drawColor,line width= 0.4pt,line join=round,line cap=round,fill=fillColor] (185.62, 73.74) circle (  1.16);

\path[draw=drawColor,line width= 0.4pt,line join=round,line cap=round,fill=fillColor] (185.84, 73.72) circle (  1.16);

\path[draw=drawColor,line width= 0.4pt,line join=round,line cap=round,fill=fillColor] (186.05, 73.70) circle (  1.16);

\path[draw=drawColor,line width= 0.4pt,line join=round,line cap=round,fill=fillColor] (186.27, 73.69) circle (  1.16);

\path[draw=drawColor,line width= 0.4pt,line join=round,line cap=round,fill=fillColor] (186.49, 73.59) circle (  1.16);

\path[draw=drawColor,line width= 0.4pt,line join=round,line cap=round,fill=fillColor] (186.70, 73.48) circle (  1.16);

\path[draw=drawColor,line width= 0.4pt,line join=round,line cap=round,fill=fillColor] (186.92, 73.42) circle (  1.16);

\path[draw=drawColor,line width= 0.4pt,line join=round,line cap=round,fill=fillColor] (187.13, 73.42) circle (  1.16);

\path[draw=drawColor,line width= 0.4pt,line join=round,line cap=round,fill=fillColor] (187.35, 73.38) circle (  1.16);

\path[draw=drawColor,line width= 0.4pt,line join=round,line cap=round,fill=fillColor] (187.56, 73.38) circle (  1.16);

\path[draw=drawColor,line width= 0.4pt,line join=round,line cap=round,fill=fillColor] (187.77, 73.32) circle (  1.16);

\path[draw=drawColor,line width= 0.4pt,line join=round,line cap=round,fill=fillColor] (187.98, 73.14) circle (  1.16);

\path[draw=drawColor,line width= 0.4pt,line join=round,line cap=round,fill=fillColor] (188.20, 72.97) circle (  1.16);

\path[draw=drawColor,line width= 0.4pt,line join=round,line cap=round,fill=fillColor] (188.41, 72.96) circle (  1.16);

\path[draw=drawColor,line width= 0.4pt,line join=round,line cap=round,fill=fillColor] (188.62, 72.93) circle (  1.16);

\path[draw=drawColor,line width= 0.4pt,line join=round,line cap=round,fill=fillColor] (188.83, 72.92) circle (  1.16);

\path[draw=drawColor,line width= 0.4pt,line join=round,line cap=round,fill=fillColor] (189.03, 72.88) circle (  1.16);

\path[draw=drawColor,line width= 0.4pt,line join=round,line cap=round,fill=fillColor] (189.24, 72.85) circle (  1.16);

\path[draw=drawColor,line width= 0.4pt,line join=round,line cap=round,fill=fillColor] (189.45, 72.84) circle (  1.16);

\path[draw=drawColor,line width= 0.4pt,line join=round,line cap=round,fill=fillColor] (189.66, 72.80) circle (  1.16);

\path[draw=drawColor,line width= 0.4pt,line join=round,line cap=round,fill=fillColor] (189.86, 72.38) circle (  1.16);

\path[draw=drawColor,line width= 0.4pt,line join=round,line cap=round,fill=fillColor] (190.07, 72.37) circle (  1.16);

\path[draw=drawColor,line width= 0.4pt,line join=round,line cap=round,fill=fillColor] (190.27, 72.31) circle (  1.16);

\path[draw=drawColor,line width= 0.4pt,line join=round,line cap=round,fill=fillColor] (190.48, 72.00) circle (  1.16);

\path[draw=drawColor,line width= 0.4pt,line join=round,line cap=round,fill=fillColor] (190.68, 71.93) circle (  1.16);

\path[draw=drawColor,line width= 0.4pt,line join=round,line cap=round,fill=fillColor] (190.88, 71.91) circle (  1.16);

\path[draw=drawColor,line width= 0.4pt,line join=round,line cap=round,fill=fillColor] (191.09, 71.75) circle (  1.16);

\path[draw=drawColor,line width= 0.4pt,line join=round,line cap=round,fill=fillColor] (191.29, 71.66) circle (  1.16);

\path[draw=drawColor,line width= 0.4pt,line join=round,line cap=round,fill=fillColor] (191.49, 71.64) circle (  1.16);

\path[draw=drawColor,line width= 0.4pt,line join=round,line cap=round,fill=fillColor] (191.69, 71.59) circle (  1.16);

\path[draw=drawColor,line width= 0.4pt,line join=round,line cap=round,fill=fillColor] (191.89, 71.52) circle (  1.16);

\path[draw=drawColor,line width= 0.4pt,line join=round,line cap=round,fill=fillColor] (192.09, 71.48) circle (  1.16);

\path[draw=drawColor,line width= 0.4pt,line join=round,line cap=round,fill=fillColor] (192.29, 71.47) circle (  1.16);

\path[draw=drawColor,line width= 0.4pt,line join=round,line cap=round,fill=fillColor] (192.49, 71.43) circle (  1.16);

\path[draw=drawColor,line width= 0.4pt,line join=round,line cap=round,fill=fillColor] (192.69, 71.37) circle (  1.16);

\path[draw=drawColor,line width= 0.4pt,line join=round,line cap=round,fill=fillColor] (192.88, 71.35) circle (  1.16);

\path[draw=drawColor,line width= 0.4pt,line join=round,line cap=round,fill=fillColor] (193.08, 71.35) circle (  1.16);

\path[draw=drawColor,line width= 0.4pt,line join=round,line cap=round,fill=fillColor] (193.28, 71.32) circle (  1.16);

\path[draw=drawColor,line width= 0.4pt,line join=round,line cap=round,fill=fillColor] (193.47, 71.31) circle (  1.16);

\path[draw=drawColor,line width= 0.4pt,line join=round,line cap=round,fill=fillColor] (193.67, 71.28) circle (  1.16);

\path[draw=drawColor,line width= 0.4pt,line join=round,line cap=round,fill=fillColor] (193.86, 71.16) circle (  1.16);

\path[draw=drawColor,line width= 0.4pt,line join=round,line cap=round,fill=fillColor] (194.06, 71.11) circle (  1.16);

\path[draw=drawColor,line width= 0.4pt,line join=round,line cap=round,fill=fillColor] (194.25, 71.05) circle (  1.16);

\path[draw=drawColor,line width= 0.4pt,line join=round,line cap=round,fill=fillColor] (194.44, 70.98) circle (  1.16);

\path[draw=drawColor,line width= 0.4pt,line join=round,line cap=round,fill=fillColor] (194.63, 70.85) circle (  1.16);

\path[draw=drawColor,line width= 0.4pt,line join=round,line cap=round,fill=fillColor] (194.83, 70.84) circle (  1.16);

\path[draw=drawColor,line width= 0.4pt,line join=round,line cap=round,fill=fillColor] (195.02, 70.59) circle (  1.16);

\path[draw=drawColor,line width= 0.4pt,line join=round,line cap=round,fill=fillColor] (195.21, 70.52) circle (  1.16);

\path[draw=drawColor,line width= 0.4pt,line join=round,line cap=round,fill=fillColor] (195.40, 70.52) circle (  1.16);

\path[draw=drawColor,line width= 0.4pt,line join=round,line cap=round,fill=fillColor] (195.59, 70.51) circle (  1.16);

\path[draw=drawColor,line width= 0.4pt,line join=round,line cap=round,fill=fillColor] (195.78, 70.50) circle (  1.16);

\path[draw=drawColor,line width= 0.4pt,line join=round,line cap=round,fill=fillColor] (195.97, 70.50) circle (  1.16);

\path[draw=drawColor,line width= 0.4pt,line join=round,line cap=round,fill=fillColor] (196.16, 70.50) circle (  1.16);

\path[draw=drawColor,line width= 0.4pt,line join=round,line cap=round,fill=fillColor] (196.34, 70.48) circle (  1.16);

\path[draw=drawColor,line width= 0.4pt,line join=round,line cap=round,fill=fillColor] (196.53, 70.47) circle (  1.16);

\path[draw=drawColor,line width= 0.4pt,line join=round,line cap=round,fill=fillColor] (196.72, 70.34) circle (  1.16);

\path[draw=drawColor,line width= 0.4pt,line join=round,line cap=round,fill=fillColor] (196.91, 70.34) circle (  1.16);

\path[draw=drawColor,line width= 0.4pt,line join=round,line cap=round,fill=fillColor] (197.09, 70.21) circle (  1.16);

\path[draw=drawColor,line width= 0.4pt,line join=round,line cap=round,fill=fillColor] (197.28, 70.20) circle (  1.16);

\path[draw=drawColor,line width= 0.4pt,line join=round,line cap=round,fill=fillColor] (197.46, 70.20) circle (  1.16);

\path[draw=drawColor,line width= 0.4pt,line join=round,line cap=round,fill=fillColor] (197.65, 70.19) circle (  1.16);

\path[draw=drawColor,line width= 0.4pt,line join=round,line cap=round,fill=fillColor] (197.83, 70.15) circle (  1.16);

\path[draw=drawColor,line width= 0.4pt,line join=round,line cap=round,fill=fillColor] (198.01, 70.10) circle (  1.16);

\path[draw=drawColor,line width= 0.4pt,line join=round,line cap=round,fill=fillColor] (198.20, 70.05) circle (  1.16);

\path[draw=drawColor,line width= 0.4pt,line join=round,line cap=round,fill=fillColor] (198.38, 70.05) circle (  1.16);

\path[draw=drawColor,line width= 0.4pt,line join=round,line cap=round,fill=fillColor] (198.56, 70.03) circle (  1.16);

\path[draw=drawColor,line width= 0.4pt,line join=round,line cap=round,fill=fillColor] (198.74, 69.96) circle (  1.16);

\path[draw=drawColor,line width= 0.4pt,line join=round,line cap=round,fill=fillColor] (198.93, 69.93) circle (  1.16);

\path[draw=drawColor,line width= 0.4pt,line join=round,line cap=round,fill=fillColor] (199.11, 69.90) circle (  1.16);

\path[draw=drawColor,line width= 0.4pt,line join=round,line cap=round,fill=fillColor] (199.29, 69.82) circle (  1.16);

\path[draw=drawColor,line width= 0.4pt,line join=round,line cap=round,fill=fillColor] (199.47, 69.80) circle (  1.16);

\path[draw=drawColor,line width= 0.4pt,line join=round,line cap=round,fill=fillColor] (199.65, 69.75) circle (  1.16);

\path[draw=drawColor,line width= 0.4pt,line join=round,line cap=round,fill=fillColor] (199.83, 69.69) circle (  1.16);

\path[draw=drawColor,line width= 0.4pt,line join=round,line cap=round,fill=fillColor] (200.00, 69.68) circle (  1.16);

\path[draw=drawColor,line width= 0.4pt,line join=round,line cap=round,fill=fillColor] (200.18, 69.63) circle (  1.16);

\path[draw=drawColor,line width= 0.4pt,line join=round,line cap=round,fill=fillColor] (200.36, 69.58) circle (  1.16);

\path[draw=drawColor,line width= 0.4pt,line join=round,line cap=round,fill=fillColor] (200.54, 69.55) circle (  1.16);

\path[draw=drawColor,line width= 0.4pt,line join=round,line cap=round,fill=fillColor] (200.72, 69.52) circle (  1.16);

\path[draw=drawColor,line width= 0.4pt,line join=round,line cap=round,fill=fillColor] (200.89, 69.44) circle (  1.16);

\path[draw=drawColor,line width= 0.4pt,line join=round,line cap=round,fill=fillColor] (201.07, 69.44) circle (  1.16);

\path[draw=drawColor,line width= 0.4pt,line join=round,line cap=round,fill=fillColor] (201.24, 69.41) circle (  1.16);

\path[draw=drawColor,line width= 0.4pt,line join=round,line cap=round,fill=fillColor] (201.42, 69.30) circle (  1.16);

\path[draw=drawColor,line width= 0.4pt,line join=round,line cap=round,fill=fillColor] (201.59, 69.21) circle (  1.16);

\path[draw=drawColor,line width= 0.4pt,line join=round,line cap=round,fill=fillColor] (201.77, 69.10) circle (  1.16);

\path[draw=drawColor,line width= 0.4pt,line join=round,line cap=round,fill=fillColor] (201.94, 69.08) circle (  1.16);

\path[draw=drawColor,line width= 0.4pt,line join=round,line cap=round,fill=fillColor] (202.12, 69.03) circle (  1.16);

\path[draw=drawColor,line width= 0.4pt,line join=round,line cap=round,fill=fillColor] (202.29, 69.01) circle (  1.16);

\path[draw=drawColor,line width= 0.4pt,line join=round,line cap=round,fill=fillColor] (202.46, 68.97) circle (  1.16);

\path[draw=drawColor,line width= 0.4pt,line join=round,line cap=round,fill=fillColor] (202.64, 68.89) circle (  1.16);

\path[draw=drawColor,line width= 0.4pt,line join=round,line cap=round,fill=fillColor] (202.81, 68.88) circle (  1.16);

\path[draw=drawColor,line width= 0.4pt,line join=round,line cap=round,fill=fillColor] (202.98, 68.78) circle (  1.16);

\path[draw=drawColor,line width= 0.4pt,line join=round,line cap=round,fill=fillColor] (203.15, 68.72) circle (  1.16);

\path[draw=drawColor,line width= 0.4pt,line join=round,line cap=round,fill=fillColor] (203.32, 68.71) circle (  1.16);

\path[draw=drawColor,line width= 0.4pt,line join=round,line cap=round,fill=fillColor] (203.49, 68.49) circle (  1.16);

\path[draw=drawColor,line width= 0.4pt,line join=round,line cap=round,fill=fillColor] (203.66, 68.45) circle (  1.16);

\path[draw=drawColor,line width= 0.4pt,line join=round,line cap=round,fill=fillColor] (203.83, 68.44) circle (  1.16);

\path[draw=drawColor,line width= 0.4pt,line join=round,line cap=round,fill=fillColor] (204.00, 68.10) circle (  1.16);

\path[draw=drawColor,line width= 0.4pt,line join=round,line cap=round,fill=fillColor] (204.17, 68.02) circle (  1.16);

\path[draw=drawColor,line width= 0.4pt,line join=round,line cap=round,fill=fillColor] (204.34, 68.00) circle (  1.16);

\path[draw=drawColor,line width= 0.4pt,line join=round,line cap=round,fill=fillColor] (204.51, 67.99) circle (  1.16);

\path[draw=drawColor,line width= 0.4pt,line join=round,line cap=round,fill=fillColor] (204.68, 67.99) circle (  1.16);

\path[draw=drawColor,line width= 0.4pt,line join=round,line cap=round,fill=fillColor] (204.84, 67.92) circle (  1.16);

\path[draw=drawColor,line width= 0.4pt,line join=round,line cap=round,fill=fillColor] (205.01, 67.88) circle (  1.16);

\path[draw=drawColor,line width= 0.4pt,line join=round,line cap=round,fill=fillColor] (205.18, 67.82) circle (  1.16);

\path[draw=drawColor,line width= 0.4pt,line join=round,line cap=round,fill=fillColor] (205.34, 67.77) circle (  1.16);

\path[draw=drawColor,line width= 0.4pt,line join=round,line cap=round,fill=fillColor] (205.51, 67.54) circle (  1.16);

\path[draw=drawColor,line width= 0.4pt,line join=round,line cap=round,fill=fillColor] (205.68, 67.53) circle (  1.16);

\path[draw=drawColor,line width= 0.4pt,line join=round,line cap=round,fill=fillColor] (205.84, 67.23) circle (  1.16);

\path[draw=drawColor,line width= 0.4pt,line join=round,line cap=round,fill=fillColor] (206.01, 67.14) circle (  1.16);

\path[draw=drawColor,line width= 0.4pt,line join=round,line cap=round,fill=fillColor] (206.17, 66.97) circle (  1.16);

\path[draw=drawColor,line width= 0.4pt,line join=round,line cap=round,fill=fillColor] (206.34, 66.79) circle (  1.16);

\path[draw=drawColor,line width= 0.4pt,line join=round,line cap=round,fill=fillColor] (206.50, 66.73) circle (  1.16);

\path[draw=drawColor,line width= 0.4pt,line join=round,line cap=round,fill=fillColor] (206.66, 66.65) circle (  1.16);

\path[draw=drawColor,line width= 0.4pt,line join=round,line cap=round,fill=fillColor] (206.83, 66.62) circle (  1.16);

\path[draw=drawColor,line width= 0.4pt,line join=round,line cap=round,fill=fillColor] (206.99, 66.43) circle (  1.16);

\path[draw=drawColor,line width= 0.4pt,line join=round,line cap=round,fill=fillColor] (207.15, 66.36) circle (  1.16);

\path[draw=drawColor,line width= 0.4pt,line join=round,line cap=round,fill=fillColor] (207.31, 66.33) circle (  1.16);

\path[draw=drawColor,line width= 0.4pt,line join=round,line cap=round,fill=fillColor] (207.48, 66.04) circle (  1.16);

\path[draw=drawColor,line width= 0.4pt,line join=round,line cap=round,fill=fillColor] (207.64, 66.01) circle (  1.16);

\path[draw=drawColor,line width= 0.4pt,line join=round,line cap=round,fill=fillColor] (207.80, 66.01) circle (  1.16);

\path[draw=drawColor,line width= 0.4pt,line join=round,line cap=round,fill=fillColor] (207.96, 65.93) circle (  1.16);

\path[draw=drawColor,line width= 0.4pt,line join=round,line cap=round,fill=fillColor] (208.12, 65.70) circle (  1.16);

\path[draw=drawColor,line width= 0.4pt,line join=round,line cap=round,fill=fillColor] (208.28, 65.59) circle (  1.16);

\path[draw=drawColor,line width= 0.4pt,line join=round,line cap=round,fill=fillColor] (208.44, 65.57) circle (  1.16);

\path[draw=drawColor,line width= 0.4pt,line join=round,line cap=round,fill=fillColor] (208.60, 65.41) circle (  1.16);

\path[draw=drawColor,line width= 0.4pt,line join=round,line cap=round,fill=fillColor] (208.76, 65.24) circle (  1.16);

\path[draw=drawColor,line width= 0.4pt,line join=round,line cap=round,fill=fillColor] (208.92, 65.23) circle (  1.16);

\path[draw=drawColor,line width= 0.4pt,line join=round,line cap=round,fill=fillColor] (209.08, 65.08) circle (  1.16);

\path[draw=drawColor,line width= 0.4pt,line join=round,line cap=round,fill=fillColor] (209.24, 64.94) circle (  1.16);

\path[draw=drawColor,line width= 0.4pt,line join=round,line cap=round,fill=fillColor] (209.39, 64.92) circle (  1.16);

\path[draw=drawColor,line width= 0.4pt,line join=round,line cap=round,fill=fillColor] (209.55, 64.83) circle (  1.16);

\path[draw=drawColor,line width= 0.4pt,line join=round,line cap=round,fill=fillColor] (209.71, 64.58) circle (  1.16);

\path[draw=drawColor,line width= 0.4pt,line join=round,line cap=round,fill=fillColor] (209.87, 64.42) circle (  1.16);

\path[draw=drawColor,line width= 0.4pt,line join=round,line cap=round,fill=fillColor] (210.02, 64.36) circle (  1.16);

\path[draw=drawColor,line width= 0.4pt,line join=round,line cap=round,fill=fillColor] (210.18, 64.34) circle (  1.16);

\path[draw=drawColor,line width= 0.4pt,line join=round,line cap=round,fill=fillColor] (210.34, 64.30) circle (  1.16);

\path[draw=drawColor,line width= 0.4pt,line join=round,line cap=round,fill=fillColor] (210.49, 64.08) circle (  1.16);

\path[draw=drawColor,line width= 0.4pt,line join=round,line cap=round,fill=fillColor] (210.65, 64.04) circle (  1.16);

\path[draw=drawColor,line width= 0.4pt,line join=round,line cap=round,fill=fillColor] (210.80, 63.90) circle (  1.16);

\path[draw=drawColor,line width= 0.4pt,line join=round,line cap=round,fill=fillColor] (210.96, 63.82) circle (  1.16);

\path[draw=drawColor,line width= 0.4pt,line join=round,line cap=round,fill=fillColor] (211.11, 63.59) circle (  1.16);

\path[draw=drawColor,line width= 0.4pt,line join=round,line cap=round,fill=fillColor] (211.27, 63.50) circle (  1.16);

\path[draw=drawColor,line width= 0.4pt,line join=round,line cap=round,fill=fillColor] (211.42, 63.30) circle (  1.16);

\path[draw=drawColor,line width= 0.4pt,line join=round,line cap=round,fill=fillColor] (211.57, 63.25) circle (  1.16);

\path[draw=drawColor,line width= 0.4pt,line join=round,line cap=round,fill=fillColor] (211.73, 62.96) circle (  1.16);

\path[draw=drawColor,line width= 0.4pt,line join=round,line cap=round,fill=fillColor] (211.88, 62.87) circle (  1.16);

\path[draw=drawColor,line width= 0.4pt,line join=round,line cap=round,fill=fillColor] (212.03, 62.72) circle (  1.16);

\path[draw=drawColor,line width= 0.4pt,line join=round,line cap=round,fill=fillColor] (212.19, 62.64) circle (  1.16);

\path[draw=drawColor,line width= 0.4pt,line join=round,line cap=round,fill=fillColor] (212.34, 62.39) circle (  1.16);

\path[draw=drawColor,line width= 0.4pt,line join=round,line cap=round,fill=fillColor] (212.49, 62.31) circle (  1.16);

\path[draw=drawColor,line width= 0.4pt,line join=round,line cap=round,fill=fillColor] (212.64, 62.31) circle (  1.16);

\path[draw=drawColor,line width= 0.4pt,line join=round,line cap=round,fill=fillColor] (212.79, 62.15) circle (  1.16);

\path[draw=drawColor,line width= 0.4pt,line join=round,line cap=round,fill=fillColor] (212.94, 62.07) circle (  1.16);

\path[draw=drawColor,line width= 0.4pt,line join=round,line cap=round,fill=fillColor] (213.09, 61.80) circle (  1.16);

\path[draw=drawColor,line width= 0.4pt,line join=round,line cap=round,fill=fillColor] (213.25, 61.72) circle (  1.16);

\path[draw=drawColor,line width= 0.4pt,line join=round,line cap=round,fill=fillColor] (213.40, 61.29) circle (  1.16);

\path[draw=drawColor,line width= 0.4pt,line join=round,line cap=round,fill=fillColor] (213.55, 61.09) circle (  1.16);

\path[draw=drawColor,line width= 0.4pt,line join=round,line cap=round,fill=fillColor] (213.70, 60.60) circle (  1.16);

\path[draw=drawColor,line width= 0.4pt,line join=round,line cap=round,fill=fillColor] (213.84, 60.35) circle (  1.16);

\path[draw=drawColor,line width= 0.4pt,line join=round,line cap=round,fill=fillColor] (213.99, 60.25) circle (  1.16);

\path[draw=drawColor,line width= 0.4pt,line join=round,line cap=round,fill=fillColor] (214.14, 60.12) circle (  1.16);

\path[draw=drawColor,line width= 0.4pt,line join=round,line cap=round,fill=fillColor] (214.29, 59.70) circle (  1.16);

\path[draw=drawColor,line width= 0.4pt,line join=round,line cap=round,fill=fillColor] (214.44, 59.68) circle (  1.16);

\path[draw=drawColor,line width= 0.4pt,line join=round,line cap=round,fill=fillColor] (214.59, 59.67) circle (  1.16);

\path[draw=drawColor,line width= 0.4pt,line join=round,line cap=round,fill=fillColor] (214.74, 59.35) circle (  1.16);

\path[draw=drawColor,line width= 0.4pt,line join=round,line cap=round,fill=fillColor] (214.88, 59.30) circle (  1.16);

\path[draw=drawColor,line width= 0.4pt,line join=round,line cap=round,fill=fillColor] (215.03, 59.29) circle (  1.16);

\path[draw=drawColor,line width= 0.4pt,line join=round,line cap=round,fill=fillColor] (215.18, 58.73) circle (  1.16);

\path[draw=drawColor,line width= 0.4pt,line join=round,line cap=round,fill=fillColor] (215.32, 58.50) circle (  1.16);

\path[draw=drawColor,line width= 0.4pt,line join=round,line cap=round,fill=fillColor] (215.47, 58.29) circle (  1.16);

\path[draw=drawColor,line width= 0.4pt,line join=round,line cap=round,fill=fillColor] (215.62, 57.99) circle (  1.16);

\path[draw=drawColor,line width= 0.4pt,line join=round,line cap=round,fill=fillColor] (215.76, 56.96) circle (  1.16);

\path[draw=drawColor,line width= 0.4pt,line join=round,line cap=round,fill=fillColor] (215.91, 56.44) circle (  1.16);

\path[draw=drawColor,line width= 0.4pt,line join=round,line cap=round,fill=fillColor] (216.05, 55.63) circle (  1.16);

\path[draw=drawColor,line width= 0.4pt,line join=round,line cap=round,fill=fillColor] (216.20, 54.12) circle (  1.16);

\path[draw=drawColor,line width= 0.4pt,line join=round,line cap=round,fill=fillColor] (216.34, 54.05) circle (  1.16);

\path[draw=drawColor,line width= 0.4pt,line join=round,line cap=round,fill=fillColor] (216.49, 53.99) circle (  1.16);

\path[draw=drawColor,line width= 0.4pt,line join=round,line cap=round,fill=fillColor] (216.63, 53.74) circle (  1.16);

\path[draw=drawColor,line width= 0.4pt,line join=round,line cap=round,fill=fillColor] (216.78, 51.42) circle (  1.16);

\path[draw=drawColor,line width= 0.4pt,line join=round,line cap=round,fill=fillColor] (216.92, 50.99) circle (  1.16);

\path[draw=drawColor,line width= 0.4pt,line join=round,line cap=round,fill=fillColor] (217.06, 47.20) circle (  1.16);

\path[draw=drawColor,line width= 0.4pt,line join=round,line cap=round,fill=fillColor] (217.21, 47.20) circle (  1.16);

\path[draw=drawColor,line width= 0.4pt,line join=round,line cap=round,fill=fillColor] (217.35, 47.20) circle (  1.16);

\path[draw=drawColor,line width= 0.4pt,line join=round,line cap=round,fill=fillColor] (217.49, 47.20) circle (  1.16);

\path[draw=drawColor,line width= 0.4pt,line join=round,line cap=round,fill=fillColor] (217.64, 47.20) circle (  1.16);

\path[draw=drawColor,line width= 0.4pt,line join=round,line cap=round,fill=fillColor] (217.78, 47.20) circle (  1.16);

\path[draw=drawColor,line width= 0.4pt,line join=round,line cap=round,fill=fillColor] (217.92, 47.20) circle (  1.16);

\path[draw=drawColor,line width= 0.4pt,line join=round,line cap=round,fill=fillColor] (218.06, 47.20) circle (  1.16);

\path[draw=drawColor,line width= 0.4pt,line join=round,line cap=round,fill=fillColor] (218.20, 47.20) circle (  1.16);

\path[draw=drawColor,line width= 0.4pt,line join=round,line cap=round,fill=fillColor] (218.35, 47.20) circle (  1.16);

\path[draw=drawColor,line width= 0.4pt,line join=round,line cap=round,fill=fillColor] (218.49, 47.20) circle (  1.16);

\path[draw=drawColor,line width= 0.4pt,line join=round,line cap=round,fill=fillColor] (218.63, 47.20) circle (  1.16);

\path[draw=drawColor,line width= 0.4pt,line join=round,line cap=round,fill=fillColor] (218.77, 47.20) circle (  1.16);

\path[draw=drawColor,line width= 0.4pt,line join=round,line cap=round,fill=fillColor] (218.91, 47.20) circle (  1.16);

\path[draw=drawColor,line width= 0.4pt,line join=round,line cap=round,fill=fillColor] (219.05, 47.20) circle (  1.16);

\path[draw=drawColor,line width= 0.4pt,line join=round,line cap=round,fill=fillColor] (219.19, 47.20) circle (  1.16);

\path[draw=drawColor,line width= 0.4pt,line join=round,line cap=round,fill=fillColor] (219.33, 47.20) circle (  1.16);

\path[draw=drawColor,line width= 0.4pt,line join=round,line cap=round,fill=fillColor] (219.47, 47.20) circle (  1.16);

\path[draw=drawColor,line width= 0.4pt,line join=round,line cap=round,fill=fillColor] (219.61, 47.20) circle (  1.16);

\path[draw=drawColor,line width= 0.4pt,line join=round,line cap=round,fill=fillColor] (219.75, 47.20) circle (  1.16);

\path[draw=drawColor,line width= 0.4pt,line join=round,line cap=round,fill=fillColor] (219.89, 47.20) circle (  1.16);

\path[draw=drawColor,line width= 0.4pt,line join=round,line cap=round,fill=fillColor] (220.02, 47.20) circle (  1.16);

\path[draw=drawColor,line width= 0.4pt,line join=round,line cap=round,fill=fillColor] (220.16, 47.20) circle (  1.16);

\path[draw=drawColor,line width= 0.4pt,line join=round,line cap=round,fill=fillColor] (220.30, 47.20) circle (  1.16);

\path[draw=drawColor,line width= 0.4pt,line join=round,line cap=round,fill=fillColor] (220.44, 47.20) circle (  1.16);

\path[draw=drawColor,line width= 0.4pt,line join=round,line cap=round,fill=fillColor] (220.58, 47.20) circle (  1.16);

\path[draw=drawColor,line width= 0.4pt,line join=round,line cap=round,fill=fillColor] (220.71, 47.20) circle (  1.16);

\path[draw=drawColor,line width= 0.4pt,line join=round,line cap=round,fill=fillColor] (220.85, 47.20) circle (  1.16);

\path[draw=drawColor,line width= 0.4pt,line join=round,line cap=round,fill=fillColor] (220.99, 47.20) circle (  1.16);

\path[draw=drawColor,line width= 0.4pt,line join=round,line cap=round,fill=fillColor] (221.13, 47.20) circle (  1.16);

\path[draw=drawColor,line width= 0.4pt,line join=round,line cap=round,fill=fillColor] (221.26, 47.20) circle (  1.16);

\path[draw=drawColor,line width= 0.4pt,line join=round,line cap=round,fill=fillColor] (221.40, 47.20) circle (  1.16);

\path[draw=drawColor,line width= 0.4pt,line join=round,line cap=round,fill=fillColor] (221.53, 47.20) circle (  1.16);

\path[draw=drawColor,line width= 0.4pt,line join=round,line cap=round,fill=fillColor] (221.67, 47.20) circle (  1.16);

\path[draw=drawColor,line width= 0.4pt,line join=round,line cap=round,fill=fillColor] (221.81, 47.20) circle (  1.16);

\path[draw=drawColor,line width= 0.4pt,line join=round,line cap=round,fill=fillColor] (221.94, 47.20) circle (  1.16);

\path[draw=drawColor,line width= 0.4pt,line join=round,line cap=round,fill=fillColor] (222.08, 47.20) circle (  1.16);

\path[draw=drawColor,line width= 0.4pt,line join=round,line cap=round,fill=fillColor] (222.21, 47.20) circle (  1.16);

\path[draw=drawColor,line width= 0.4pt,line join=round,line cap=round,fill=fillColor] (222.35, 47.20) circle (  1.16);

\path[draw=drawColor,line width= 0.4pt,line join=round,line cap=round,fill=fillColor] (222.48, 47.20) circle (  1.16);

\path[draw=drawColor,line width= 0.4pt,line join=round,line cap=round,fill=fillColor] (222.62, 47.20) circle (  1.16);

\path[draw=drawColor,line width= 0.4pt,line join=round,line cap=round,fill=fillColor] (222.75, 47.20) circle (  1.16);

\path[draw=drawColor,line width= 0.4pt,line join=round,line cap=round,fill=fillColor] (222.88, 47.20) circle (  1.16);

\path[draw=drawColor,line width= 0.4pt,line join=round,line cap=round,fill=fillColor] (223.02, 47.20) circle (  1.16);

\path[draw=drawColor,line width= 0.4pt,line join=round,line cap=round,fill=fillColor] (223.15, 47.20) circle (  1.16);

\path[draw=drawColor,line width= 0.4pt,line join=round,line cap=round,fill=fillColor] (223.28, 47.20) circle (  1.16);

\path[draw=drawColor,line width= 0.4pt,line join=round,line cap=round,fill=fillColor] (223.42, 47.20) circle (  1.16);

\path[draw=drawColor,line width= 0.4pt,line join=round,line cap=round,fill=fillColor] (223.55, 47.20) circle (  1.16);

\path[draw=drawColor,line width= 0.4pt,line join=round,line cap=round,fill=fillColor] (223.68, 47.20) circle (  1.16);

\path[draw=drawColor,line width= 0.4pt,line join=round,line cap=round,fill=fillColor] (223.82, 47.20) circle (  1.16);

\path[draw=drawColor,line width= 0.4pt,line join=round,line cap=round,fill=fillColor] (223.95, 47.20) circle (  1.16);

\path[draw=drawColor,line width= 0.4pt,line join=round,line cap=round,fill=fillColor] (224.08, 47.20) circle (  1.16);

\path[draw=drawColor,line width= 0.4pt,line join=round,line cap=round,fill=fillColor] (224.21, 47.20) circle (  1.16);

\path[draw=drawColor,line width= 0.4pt,line join=round,line cap=round,fill=fillColor] (224.34, 47.20) circle (  1.16);

\path[draw=drawColor,line width= 0.4pt,line join=round,line cap=round,fill=fillColor] (224.48, 47.20) circle (  1.16);

\path[draw=drawColor,line width= 0.4pt,line join=round,line cap=round,fill=fillColor] (224.61, 47.20) circle (  1.16);

\path[draw=drawColor,line width= 0.4pt,line join=round,line cap=round,fill=fillColor] (224.74, 47.20) circle (  1.16);

\path[draw=drawColor,line width= 0.4pt,line join=round,line cap=round,fill=fillColor] (224.87, 47.20) circle (  1.16);

\path[draw=drawColor,line width= 0.4pt,line join=round,line cap=round,fill=fillColor] (225.00, 47.20) circle (  1.16);

\path[draw=drawColor,line width= 0.4pt,line join=round,line cap=round,fill=fillColor] (225.13, 47.20) circle (  1.16);

\path[draw=drawColor,line width= 0.4pt,line join=round,line cap=round,fill=fillColor] (225.26, 47.20) circle (  1.16);
\definecolor[named]{drawColor}{rgb}{0.30,0.69,0.29}
\definecolor[named]{fillColor}{rgb}{0.30,0.69,0.29}

\path[draw=drawColor,line width= 0.4pt,line join=round,line cap=round,fill=fillColor] ( 74.88,132.85) circle (  1.16);

\path[draw=drawColor,line width= 0.4pt,line join=round,line cap=round,fill=fillColor] ( 80.78,132.85) circle (  1.16);

\path[draw=drawColor,line width= 0.4pt,line join=round,line cap=round,fill=fillColor] ( 84.92,132.85) circle (  1.16);

\path[draw=drawColor,line width= 0.4pt,line join=round,line cap=round,fill=fillColor] ( 88.22,132.85) circle (  1.16);

\path[draw=drawColor,line width= 0.4pt,line join=round,line cap=round,fill=fillColor] ( 91.00,132.85) circle (  1.16);

\path[draw=drawColor,line width= 0.4pt,line join=round,line cap=round,fill=fillColor] ( 93.43,132.85) circle (  1.16);

\path[draw=drawColor,line width= 0.4pt,line join=round,line cap=round,fill=fillColor] ( 95.61,132.85) circle (  1.16);

\path[draw=drawColor,line width= 0.4pt,line join=round,line cap=round,fill=fillColor] ( 97.58,132.85) circle (  1.16);

\path[draw=drawColor,line width= 0.4pt,line join=round,line cap=round,fill=fillColor] ( 99.40,132.85) circle (  1.16);

\path[draw=drawColor,line width= 0.4pt,line join=round,line cap=round,fill=fillColor] (101.09,132.85) circle (  1.16);

\path[draw=drawColor,line width= 0.4pt,line join=round,line cap=round,fill=fillColor] (102.67,132.85) circle (  1.16);

\path[draw=drawColor,line width= 0.4pt,line join=round,line cap=round,fill=fillColor] (104.16,132.85) circle (  1.16);

\path[draw=drawColor,line width= 0.4pt,line join=round,line cap=round,fill=fillColor] (105.56,132.85) circle (  1.16);

\path[draw=drawColor,line width= 0.4pt,line join=round,line cap=round,fill=fillColor] (106.90,132.85) circle (  1.16);

\path[draw=drawColor,line width= 0.4pt,line join=round,line cap=round,fill=fillColor] (108.17,132.85) circle (  1.16);

\path[draw=drawColor,line width= 0.4pt,line join=round,line cap=round,fill=fillColor] (109.39,132.85) circle (  1.16);

\path[draw=drawColor,line width= 0.4pt,line join=round,line cap=round,fill=fillColor] (110.56,132.85) circle (  1.16);

\path[draw=drawColor,line width= 0.4pt,line join=round,line cap=round,fill=fillColor] (111.68,132.85) circle (  1.16);

\path[draw=drawColor,line width= 0.4pt,line join=round,line cap=round,fill=fillColor] (112.76,132.85) circle (  1.16);

\path[draw=drawColor,line width= 0.4pt,line join=round,line cap=round,fill=fillColor] (113.81,132.85) circle (  1.16);

\path[draw=drawColor,line width= 0.4pt,line join=round,line cap=round,fill=fillColor] (114.82,132.85) circle (  1.16);

\path[draw=drawColor,line width= 0.4pt,line join=round,line cap=round,fill=fillColor] (115.80,132.85) circle (  1.16);

\path[draw=drawColor,line width= 0.4pt,line join=round,line cap=round,fill=fillColor] (116.74,132.85) circle (  1.16);

\path[draw=drawColor,line width= 0.4pt,line join=round,line cap=round,fill=fillColor] (117.67,132.85) circle (  1.16);

\path[draw=drawColor,line width= 0.4pt,line join=round,line cap=round,fill=fillColor] (118.56,132.85) circle (  1.16);

\path[draw=drawColor,line width= 0.4pt,line join=round,line cap=round,fill=fillColor] (119.44,132.85) circle (  1.16);

\path[draw=drawColor,line width= 0.4pt,line join=round,line cap=round,fill=fillColor] (120.29,132.85) circle (  1.16);

\path[draw=drawColor,line width= 0.4pt,line join=round,line cap=round,fill=fillColor] (121.12,132.85) circle (  1.16);

\path[draw=drawColor,line width= 0.4pt,line join=round,line cap=round,fill=fillColor] (121.93,132.85) circle (  1.16);

\path[draw=drawColor,line width= 0.4pt,line join=round,line cap=round,fill=fillColor] (122.72,132.85) circle (  1.16);

\path[draw=drawColor,line width= 0.4pt,line join=round,line cap=round,fill=fillColor] (123.50,132.85) circle (  1.16);

\path[draw=drawColor,line width= 0.4pt,line join=round,line cap=round,fill=fillColor] (124.26,132.85) circle (  1.16);

\path[draw=drawColor,line width= 0.4pt,line join=round,line cap=round,fill=fillColor] (125.00,132.85) circle (  1.16);

\path[draw=drawColor,line width= 0.4pt,line join=round,line cap=round,fill=fillColor] (125.73,132.85) circle (  1.16);

\path[draw=drawColor,line width= 0.4pt,line join=round,line cap=round,fill=fillColor] (126.44,132.85) circle (  1.16);

\path[draw=drawColor,line width= 0.4pt,line join=round,line cap=round,fill=fillColor] (127.15,132.85) circle (  1.16);

\path[draw=drawColor,line width= 0.4pt,line join=round,line cap=round,fill=fillColor] (127.83,132.85) circle (  1.16);

\path[draw=drawColor,line width= 0.4pt,line join=round,line cap=round,fill=fillColor] (128.51,132.85) circle (  1.16);

\path[draw=drawColor,line width= 0.4pt,line join=round,line cap=round,fill=fillColor] (129.17,132.85) circle (  1.16);

\path[draw=drawColor,line width= 0.4pt,line join=round,line cap=round,fill=fillColor] (129.83,132.85) circle (  1.16);

\path[draw=drawColor,line width= 0.4pt,line join=round,line cap=round,fill=fillColor] (130.47,132.85) circle (  1.16);

\path[draw=drawColor,line width= 0.4pt,line join=round,line cap=round,fill=fillColor] (131.10,132.85) circle (  1.16);

\path[draw=drawColor,line width= 0.4pt,line join=round,line cap=round,fill=fillColor] (131.72,132.85) circle (  1.16);

\path[draw=drawColor,line width= 0.4pt,line join=round,line cap=round,fill=fillColor] (132.33,132.85) circle (  1.16);

\path[draw=drawColor,line width= 0.4pt,line join=round,line cap=round,fill=fillColor] (132.93,132.85) circle (  1.16);

\path[draw=drawColor,line width= 0.4pt,line join=round,line cap=round,fill=fillColor] (133.53,132.85) circle (  1.16);

\path[draw=drawColor,line width= 0.4pt,line join=round,line cap=round,fill=fillColor] (134.11,132.85) circle (  1.16);

\path[draw=drawColor,line width= 0.4pt,line join=round,line cap=round,fill=fillColor] (134.69,132.85) circle (  1.16);

\path[draw=drawColor,line width= 0.4pt,line join=round,line cap=round,fill=fillColor] (135.26,132.85) circle (  1.16);

\path[draw=drawColor,line width= 0.4pt,line join=round,line cap=round,fill=fillColor] (135.82,132.85) circle (  1.16);

\path[draw=drawColor,line width= 0.4pt,line join=round,line cap=round,fill=fillColor] (136.38,132.85) circle (  1.16);

\path[draw=drawColor,line width= 0.4pt,line join=round,line cap=round,fill=fillColor] (136.92,132.85) circle (  1.16);

\path[draw=drawColor,line width= 0.4pt,line join=round,line cap=round,fill=fillColor] (137.46,132.85) circle (  1.16);

\path[draw=drawColor,line width= 0.4pt,line join=round,line cap=round,fill=fillColor] (137.99,132.85) circle (  1.16);

\path[draw=drawColor,line width= 0.4pt,line join=round,line cap=round,fill=fillColor] (138.52,132.85) circle (  1.16);

\path[draw=drawColor,line width= 0.4pt,line join=round,line cap=round,fill=fillColor] (139.04,132.85) circle (  1.16);

\path[draw=drawColor,line width= 0.4pt,line join=round,line cap=round,fill=fillColor] (139.56,132.85) circle (  1.16);

\path[draw=drawColor,line width= 0.4pt,line join=round,line cap=round,fill=fillColor] (140.06,132.85) circle (  1.16);

\path[draw=drawColor,line width= 0.4pt,line join=round,line cap=round,fill=fillColor] (140.57,132.85) circle (  1.16);

\path[draw=drawColor,line width= 0.4pt,line join=round,line cap=round,fill=fillColor] (141.06,132.85) circle (  1.16);

\path[draw=drawColor,line width= 0.4pt,line join=round,line cap=round,fill=fillColor] (141.55,132.85) circle (  1.16);

\path[draw=drawColor,line width= 0.4pt,line join=round,line cap=round,fill=fillColor] (142.04,132.85) circle (  1.16);

\path[draw=drawColor,line width= 0.4pt,line join=round,line cap=round,fill=fillColor] (142.52,132.85) circle (  1.16);

\path[draw=drawColor,line width= 0.4pt,line join=round,line cap=round,fill=fillColor] (143.00,132.85) circle (  1.16);

\path[draw=drawColor,line width= 0.4pt,line join=round,line cap=round,fill=fillColor] (143.47,132.85) circle (  1.16);

\path[draw=drawColor,line width= 0.4pt,line join=round,line cap=round,fill=fillColor] (143.93,132.85) circle (  1.16);

\path[draw=drawColor,line width= 0.4pt,line join=round,line cap=round,fill=fillColor] (144.39,132.85) circle (  1.16);

\path[draw=drawColor,line width= 0.4pt,line join=round,line cap=round,fill=fillColor] (144.85,132.85) circle (  1.16);

\path[draw=drawColor,line width= 0.4pt,line join=round,line cap=round,fill=fillColor] (145.30,132.85) circle (  1.16);

\path[draw=drawColor,line width= 0.4pt,line join=round,line cap=round,fill=fillColor] (145.75,132.85) circle (  1.16);

\path[draw=drawColor,line width= 0.4pt,line join=round,line cap=round,fill=fillColor] (146.19,132.85) circle (  1.16);

\path[draw=drawColor,line width= 0.4pt,line join=round,line cap=round,fill=fillColor] (146.63,132.85) circle (  1.16);

\path[draw=drawColor,line width= 0.4pt,line join=round,line cap=round,fill=fillColor] (147.07,132.85) circle (  1.16);

\path[draw=drawColor,line width= 0.4pt,line join=round,line cap=round,fill=fillColor] (147.50,132.85) circle (  1.16);

\path[draw=drawColor,line width= 0.4pt,line join=round,line cap=round,fill=fillColor] (147.93,132.85) circle (  1.16);

\path[draw=drawColor,line width= 0.4pt,line join=round,line cap=round,fill=fillColor] (148.35,132.85) circle (  1.16);

\path[draw=drawColor,line width= 0.4pt,line join=round,line cap=round,fill=fillColor] (148.77,132.85) circle (  1.16);

\path[draw=drawColor,line width= 0.4pt,line join=round,line cap=round,fill=fillColor] (149.19,132.85) circle (  1.16);

\path[draw=drawColor,line width= 0.4pt,line join=round,line cap=round,fill=fillColor] (149.60,132.85) circle (  1.16);

\path[draw=drawColor,line width= 0.4pt,line join=round,line cap=round,fill=fillColor] (150.01,132.85) circle (  1.16);

\path[draw=drawColor,line width= 0.4pt,line join=round,line cap=round,fill=fillColor] (150.41,132.85) circle (  1.16);

\path[draw=drawColor,line width= 0.4pt,line join=round,line cap=round,fill=fillColor] (150.82,132.85) circle (  1.16);

\path[draw=drawColor,line width= 0.4pt,line join=round,line cap=round,fill=fillColor] (151.22,132.85) circle (  1.16);

\path[draw=drawColor,line width= 0.4pt,line join=round,line cap=round,fill=fillColor] (151.61,132.85) circle (  1.16);

\path[draw=drawColor,line width= 0.4pt,line join=round,line cap=round,fill=fillColor] (152.01,132.85) circle (  1.16);

\path[draw=drawColor,line width= 0.4pt,line join=round,line cap=round,fill=fillColor] (152.40,132.85) circle (  1.16);

\path[draw=drawColor,line width= 0.4pt,line join=round,line cap=round,fill=fillColor] (152.78,132.85) circle (  1.16);

\path[draw=drawColor,line width= 0.4pt,line join=round,line cap=round,fill=fillColor] (153.17,132.85) circle (  1.16);

\path[draw=drawColor,line width= 0.4pt,line join=round,line cap=round,fill=fillColor] (153.55,132.85) circle (  1.16);

\path[draw=drawColor,line width= 0.4pt,line join=round,line cap=round,fill=fillColor] (153.93,132.85) circle (  1.16);

\path[draw=drawColor,line width= 0.4pt,line join=round,line cap=round,fill=fillColor] (154.30,132.85) circle (  1.16);

\path[draw=drawColor,line width= 0.4pt,line join=round,line cap=round,fill=fillColor] (154.67,132.85) circle (  1.16);

\path[draw=drawColor,line width= 0.4pt,line join=round,line cap=round,fill=fillColor] (155.04,132.85) circle (  1.16);

\path[draw=drawColor,line width= 0.4pt,line join=round,line cap=round,fill=fillColor] (155.41,132.85) circle (  1.16);

\path[draw=drawColor,line width= 0.4pt,line join=round,line cap=round,fill=fillColor] (155.78,132.85) circle (  1.16);

\path[draw=drawColor,line width= 0.4pt,line join=round,line cap=round,fill=fillColor] (156.14,132.85) circle (  1.16);

\path[draw=drawColor,line width= 0.4pt,line join=round,line cap=round,fill=fillColor] (156.50,132.85) circle (  1.16);

\path[draw=drawColor,line width= 0.4pt,line join=round,line cap=round,fill=fillColor] (156.86,132.85) circle (  1.16);

\path[draw=drawColor,line width= 0.4pt,line join=round,line cap=round,fill=fillColor] (157.21,132.85) circle (  1.16);

\path[draw=drawColor,line width= 0.4pt,line join=round,line cap=round,fill=fillColor] (157.56,132.85) circle (  1.16);

\path[draw=drawColor,line width= 0.4pt,line join=round,line cap=round,fill=fillColor] (157.91,132.85) circle (  1.16);

\path[draw=drawColor,line width= 0.4pt,line join=round,line cap=round,fill=fillColor] (158.26,132.85) circle (  1.16);

\path[draw=drawColor,line width= 0.4pt,line join=round,line cap=round,fill=fillColor] (158.61,132.85) circle (  1.16);

\path[draw=drawColor,line width= 0.4pt,line join=round,line cap=round,fill=fillColor] (158.95,132.85) circle (  1.16);

\path[draw=drawColor,line width= 0.4pt,line join=round,line cap=round,fill=fillColor] (159.29,132.85) circle (  1.16);

\path[draw=drawColor,line width= 0.4pt,line join=round,line cap=round,fill=fillColor] (159.63,132.85) circle (  1.16);

\path[draw=drawColor,line width= 0.4pt,line join=round,line cap=round,fill=fillColor] (159.97,132.85) circle (  1.16);

\path[draw=drawColor,line width= 0.4pt,line join=round,line cap=round,fill=fillColor] (160.30,132.85) circle (  1.16);

\path[draw=drawColor,line width= 0.4pt,line join=round,line cap=round,fill=fillColor] (160.63,132.85) circle (  1.16);

\path[draw=drawColor,line width= 0.4pt,line join=round,line cap=round,fill=fillColor] (160.96,132.85) circle (  1.16);

\path[draw=drawColor,line width= 0.4pt,line join=round,line cap=round,fill=fillColor] (161.29,132.85) circle (  1.16);

\path[draw=drawColor,line width= 0.4pt,line join=round,line cap=round,fill=fillColor] (161.62,132.85) circle (  1.16);

\path[draw=drawColor,line width= 0.4pt,line join=round,line cap=round,fill=fillColor] (161.95,132.85) circle (  1.16);

\path[draw=drawColor,line width= 0.4pt,line join=round,line cap=round,fill=fillColor] (162.27,132.85) circle (  1.16);

\path[draw=drawColor,line width= 0.4pt,line join=round,line cap=round,fill=fillColor] (162.59,132.85) circle (  1.16);

\path[draw=drawColor,line width= 0.4pt,line join=round,line cap=round,fill=fillColor] (162.91,132.85) circle (  1.16);

\path[draw=drawColor,line width= 0.4pt,line join=round,line cap=round,fill=fillColor] (163.23,132.85) circle (  1.16);

\path[draw=drawColor,line width= 0.4pt,line join=round,line cap=round,fill=fillColor] (163.54,132.85) circle (  1.16);

\path[draw=drawColor,line width= 0.4pt,line join=round,line cap=round,fill=fillColor] (163.85,132.85) circle (  1.16);

\path[draw=drawColor,line width= 0.4pt,line join=round,line cap=round,fill=fillColor] (164.17,132.85) circle (  1.16);

\path[draw=drawColor,line width= 0.4pt,line join=round,line cap=round,fill=fillColor] (164.48,132.85) circle (  1.16);

\path[draw=drawColor,line width= 0.4pt,line join=round,line cap=round,fill=fillColor] (164.79,132.85) circle (  1.16);

\path[draw=drawColor,line width= 0.4pt,line join=round,line cap=round,fill=fillColor] (165.09,132.85) circle (  1.16);

\path[draw=drawColor,line width= 0.4pt,line join=round,line cap=round,fill=fillColor] (165.40,132.85) circle (  1.16);

\path[draw=drawColor,line width= 0.4pt,line join=round,line cap=round,fill=fillColor] (165.70,132.85) circle (  1.16);

\path[draw=drawColor,line width= 0.4pt,line join=round,line cap=round,fill=fillColor] (166.00,132.85) circle (  1.16);

\path[draw=drawColor,line width= 0.4pt,line join=round,line cap=round,fill=fillColor] (166.30,132.85) circle (  1.16);

\path[draw=drawColor,line width= 0.4pt,line join=round,line cap=round,fill=fillColor] (166.60,132.85) circle (  1.16);

\path[draw=drawColor,line width= 0.4pt,line join=round,line cap=round,fill=fillColor] (166.90,132.85) circle (  1.16);

\path[draw=drawColor,line width= 0.4pt,line join=round,line cap=round,fill=fillColor] (167.19,132.85) circle (  1.16);

\path[draw=drawColor,line width= 0.4pt,line join=round,line cap=round,fill=fillColor] (167.49,132.85) circle (  1.16);

\path[draw=drawColor,line width= 0.4pt,line join=round,line cap=round,fill=fillColor] (167.78,132.85) circle (  1.16);

\path[draw=drawColor,line width= 0.4pt,line join=round,line cap=round,fill=fillColor] (168.07,132.85) circle (  1.16);

\path[draw=drawColor,line width= 0.4pt,line join=round,line cap=round,fill=fillColor] (168.36,132.85) circle (  1.16);

\path[draw=drawColor,line width= 0.4pt,line join=round,line cap=round,fill=fillColor] (168.65,132.85) circle (  1.16);

\path[draw=drawColor,line width= 0.4pt,line join=round,line cap=round,fill=fillColor] (168.94,132.85) circle (  1.16);

\path[draw=drawColor,line width= 0.4pt,line join=round,line cap=round,fill=fillColor] (169.22,132.85) circle (  1.16);

\path[draw=drawColor,line width= 0.4pt,line join=round,line cap=round,fill=fillColor] (169.51,132.85) circle (  1.16);

\path[draw=drawColor,line width= 0.4pt,line join=round,line cap=round,fill=fillColor] (169.79,132.85) circle (  1.16);

\path[draw=drawColor,line width= 0.4pt,line join=round,line cap=round,fill=fillColor] (170.07,132.85) circle (  1.16);

\path[draw=drawColor,line width= 0.4pt,line join=round,line cap=round,fill=fillColor] (170.35,132.85) circle (  1.16);

\path[draw=drawColor,line width= 0.4pt,line join=round,line cap=round,fill=fillColor] (170.63,132.85) circle (  1.16);

\path[draw=drawColor,line width= 0.4pt,line join=round,line cap=round,fill=fillColor] (170.91,132.85) circle (  1.16);

\path[draw=drawColor,line width= 0.4pt,line join=round,line cap=round,fill=fillColor] (171.18,132.85) circle (  1.16);

\path[draw=drawColor,line width= 0.4pt,line join=round,line cap=round,fill=fillColor] (171.46,132.85) circle (  1.16);

\path[draw=drawColor,line width= 0.4pt,line join=round,line cap=round,fill=fillColor] (171.73,132.85) circle (  1.16);

\path[draw=drawColor,line width= 0.4pt,line join=round,line cap=round,fill=fillColor] (172.00,132.85) circle (  1.16);

\path[draw=drawColor,line width= 0.4pt,line join=round,line cap=round,fill=fillColor] (172.28,132.85) circle (  1.16);

\path[draw=drawColor,line width= 0.4pt,line join=round,line cap=round,fill=fillColor] (172.55,132.85) circle (  1.16);

\path[draw=drawColor,line width= 0.4pt,line join=round,line cap=round,fill=fillColor] (172.81,132.85) circle (  1.16);

\path[draw=drawColor,line width= 0.4pt,line join=round,line cap=round,fill=fillColor] (173.08,132.85) circle (  1.16);

\path[draw=drawColor,line width= 0.4pt,line join=round,line cap=round,fill=fillColor] (173.35,132.85) circle (  1.16);

\path[draw=drawColor,line width= 0.4pt,line join=round,line cap=round,fill=fillColor] (173.61,132.85) circle (  1.16);

\path[draw=drawColor,line width= 0.4pt,line join=round,line cap=round,fill=fillColor] (173.88,132.85) circle (  1.16);

\path[draw=drawColor,line width= 0.4pt,line join=round,line cap=round,fill=fillColor] (174.14,132.85) circle (  1.16);

\path[draw=drawColor,line width= 0.4pt,line join=round,line cap=round,fill=fillColor] (174.40,132.85) circle (  1.16);

\path[draw=drawColor,line width= 0.4pt,line join=round,line cap=round,fill=fillColor] (174.66,132.85) circle (  1.16);

\path[draw=drawColor,line width= 0.4pt,line join=round,line cap=round,fill=fillColor] (174.92,132.85) circle (  1.16);

\path[draw=drawColor,line width= 0.4pt,line join=round,line cap=round,fill=fillColor] (175.18,132.85) circle (  1.16);

\path[draw=drawColor,line width= 0.4pt,line join=round,line cap=round,fill=fillColor] (175.44,132.85) circle (  1.16);

\path[draw=drawColor,line width= 0.4pt,line join=round,line cap=round,fill=fillColor] (175.69,132.85) circle (  1.16);

\path[draw=drawColor,line width= 0.4pt,line join=round,line cap=round,fill=fillColor] (175.95,132.85) circle (  1.16);

\path[draw=drawColor,line width= 0.4pt,line join=round,line cap=round,fill=fillColor] (176.20,132.85) circle (  1.16);

\path[draw=drawColor,line width= 0.4pt,line join=round,line cap=round,fill=fillColor] (176.46,132.85) circle (  1.16);

\path[draw=drawColor,line width= 0.4pt,line join=round,line cap=round,fill=fillColor] (176.71,132.85) circle (  1.16);

\path[draw=drawColor,line width= 0.4pt,line join=round,line cap=round,fill=fillColor] (176.96,132.85) circle (  1.16);

\path[draw=drawColor,line width= 0.4pt,line join=round,line cap=round,fill=fillColor] (177.21,132.85) circle (  1.16);

\path[draw=drawColor,line width= 0.4pt,line join=round,line cap=round,fill=fillColor] (177.46,132.85) circle (  1.16);

\path[draw=drawColor,line width= 0.4pt,line join=round,line cap=round,fill=fillColor] (177.71,132.85) circle (  1.16);

\path[draw=drawColor,line width= 0.4pt,line join=round,line cap=round,fill=fillColor] (177.95,132.85) circle (  1.16);

\path[draw=drawColor,line width= 0.4pt,line join=round,line cap=round,fill=fillColor] (178.20,132.85) circle (  1.16);

\path[draw=drawColor,line width= 0.4pt,line join=round,line cap=round,fill=fillColor] (178.45,132.85) circle (  1.16);

\path[draw=drawColor,line width= 0.4pt,line join=round,line cap=round,fill=fillColor] (178.69,132.85) circle (  1.16);

\path[draw=drawColor,line width= 0.4pt,line join=round,line cap=round,fill=fillColor] (178.93,132.85) circle (  1.16);

\path[draw=drawColor,line width= 0.4pt,line join=round,line cap=round,fill=fillColor] (179.18,132.85) circle (  1.16);

\path[draw=drawColor,line width= 0.4pt,line join=round,line cap=round,fill=fillColor] (179.42,132.85) circle (  1.16);

\path[draw=drawColor,line width= 0.4pt,line join=round,line cap=round,fill=fillColor] (179.66,132.85) circle (  1.16);

\path[draw=drawColor,line width= 0.4pt,line join=round,line cap=round,fill=fillColor] (179.90,132.85) circle (  1.16);

\path[draw=drawColor,line width= 0.4pt,line join=round,line cap=round,fill=fillColor] (180.14,132.85) circle (  1.16);

\path[draw=drawColor,line width= 0.4pt,line join=round,line cap=round,fill=fillColor] (180.37,132.85) circle (  1.16);

\path[draw=drawColor,line width= 0.4pt,line join=round,line cap=round,fill=fillColor] (180.61,132.85) circle (  1.16);

\path[draw=drawColor,line width= 0.4pt,line join=round,line cap=round,fill=fillColor] (180.85,132.85) circle (  1.16);

\path[draw=drawColor,line width= 0.4pt,line join=round,line cap=round,fill=fillColor] (181.08,132.85) circle (  1.16);

\path[draw=drawColor,line width= 0.4pt,line join=round,line cap=round,fill=fillColor] (181.32,132.85) circle (  1.16);

\path[draw=drawColor,line width= 0.4pt,line join=round,line cap=round,fill=fillColor] (181.55,132.85) circle (  1.16);

\path[draw=drawColor,line width= 0.4pt,line join=round,line cap=round,fill=fillColor] (181.78,132.85) circle (  1.16);

\path[draw=drawColor,line width= 0.4pt,line join=round,line cap=round,fill=fillColor] (182.01,132.85) circle (  1.16);

\path[draw=drawColor,line width= 0.4pt,line join=round,line cap=round,fill=fillColor] (182.25,132.85) circle (  1.16);

\path[draw=drawColor,line width= 0.4pt,line join=round,line cap=round,fill=fillColor] (182.48,132.85) circle (  1.16);

\path[draw=drawColor,line width= 0.4pt,line join=round,line cap=round,fill=fillColor] (182.70,132.85) circle (  1.16);

\path[draw=drawColor,line width= 0.4pt,line join=round,line cap=round,fill=fillColor] (182.93,132.85) circle (  1.16);

\path[draw=drawColor,line width= 0.4pt,line join=round,line cap=round,fill=fillColor] (183.16,132.85) circle (  1.16);

\path[draw=drawColor,line width= 0.4pt,line join=round,line cap=round,fill=fillColor] (183.39,132.85) circle (  1.16);

\path[draw=drawColor,line width= 0.4pt,line join=round,line cap=round,fill=fillColor] (183.61,132.85) circle (  1.16);

\path[draw=drawColor,line width= 0.4pt,line join=round,line cap=round,fill=fillColor] (183.84,132.85) circle (  1.16);

\path[draw=drawColor,line width= 0.4pt,line join=round,line cap=round,fill=fillColor] (184.06,132.85) circle (  1.16);

\path[draw=drawColor,line width= 0.4pt,line join=round,line cap=round,fill=fillColor] (184.29,132.85) circle (  1.16);

\path[draw=drawColor,line width= 0.4pt,line join=round,line cap=round,fill=fillColor] (184.51,132.85) circle (  1.16);

\path[draw=drawColor,line width= 0.4pt,line join=round,line cap=round,fill=fillColor] (184.73,132.85) circle (  1.16);

\path[draw=drawColor,line width= 0.4pt,line join=round,line cap=round,fill=fillColor] (184.96,132.85) circle (  1.16);

\path[draw=drawColor,line width= 0.4pt,line join=round,line cap=round,fill=fillColor] (185.18,132.85) circle (  1.16);

\path[draw=drawColor,line width= 0.4pt,line join=round,line cap=round,fill=fillColor] (185.40,132.85) circle (  1.16);

\path[draw=drawColor,line width= 0.4pt,line join=round,line cap=round,fill=fillColor] (185.62,132.85) circle (  1.16);

\path[draw=drawColor,line width= 0.4pt,line join=round,line cap=round,fill=fillColor] (185.84,132.85) circle (  1.16);

\path[draw=drawColor,line width= 0.4pt,line join=round,line cap=round,fill=fillColor] (186.05,132.85) circle (  1.16);

\path[draw=drawColor,line width= 0.4pt,line join=round,line cap=round,fill=fillColor] (186.27,132.85) circle (  1.16);

\path[draw=drawColor,line width= 0.4pt,line join=round,line cap=round,fill=fillColor] (186.49,132.85) circle (  1.16);

\path[draw=drawColor,line width= 0.4pt,line join=round,line cap=round,fill=fillColor] (186.70,132.85) circle (  1.16);

\path[draw=drawColor,line width= 0.4pt,line join=round,line cap=round,fill=fillColor] (186.92,132.85) circle (  1.16);

\path[draw=drawColor,line width= 0.4pt,line join=round,line cap=round,fill=fillColor] (187.13,132.85) circle (  1.16);

\path[draw=drawColor,line width= 0.4pt,line join=round,line cap=round,fill=fillColor] (187.35,132.85) circle (  1.16);

\path[draw=drawColor,line width= 0.4pt,line join=round,line cap=round,fill=fillColor] (187.56,132.85) circle (  1.16);

\path[draw=drawColor,line width= 0.4pt,line join=round,line cap=round,fill=fillColor] (187.77,132.85) circle (  1.16);

\path[draw=drawColor,line width= 0.4pt,line join=round,line cap=round,fill=fillColor] (187.98,132.85) circle (  1.16);

\path[draw=drawColor,line width= 0.4pt,line join=round,line cap=round,fill=fillColor] (188.20,132.85) circle (  1.16);

\path[draw=drawColor,line width= 0.4pt,line join=round,line cap=round,fill=fillColor] (188.41,132.85) circle (  1.16);

\path[draw=drawColor,line width= 0.4pt,line join=round,line cap=round,fill=fillColor] (188.62,132.85) circle (  1.16);

\path[draw=drawColor,line width= 0.4pt,line join=round,line cap=round,fill=fillColor] (188.83,132.85) circle (  1.16);

\path[draw=drawColor,line width= 0.4pt,line join=round,line cap=round,fill=fillColor] (189.03,132.85) circle (  1.16);

\path[draw=drawColor,line width= 0.4pt,line join=round,line cap=round,fill=fillColor] (189.24,132.85) circle (  1.16);

\path[draw=drawColor,line width= 0.4pt,line join=round,line cap=round,fill=fillColor] (189.45,132.85) circle (  1.16);

\path[draw=drawColor,line width= 0.4pt,line join=round,line cap=round,fill=fillColor] (189.66,132.85) circle (  1.16);

\path[draw=drawColor,line width= 0.4pt,line join=round,line cap=round,fill=fillColor] (189.86,132.85) circle (  1.16);

\path[draw=drawColor,line width= 0.4pt,line join=round,line cap=round,fill=fillColor] (190.07,132.85) circle (  1.16);

\path[draw=drawColor,line width= 0.4pt,line join=round,line cap=round,fill=fillColor] (190.27,132.85) circle (  1.16);

\path[draw=drawColor,line width= 0.4pt,line join=round,line cap=round,fill=fillColor] (190.48,132.85) circle (  1.16);

\path[draw=drawColor,line width= 0.4pt,line join=round,line cap=round,fill=fillColor] (190.68,132.85) circle (  1.16);

\path[draw=drawColor,line width= 0.4pt,line join=round,line cap=round,fill=fillColor] (190.88,132.85) circle (  1.16);

\path[draw=drawColor,line width= 0.4pt,line join=round,line cap=round,fill=fillColor] (191.09,132.85) circle (  1.16);

\path[draw=drawColor,line width= 0.4pt,line join=round,line cap=round,fill=fillColor] (191.29,132.85) circle (  1.16);

\path[draw=drawColor,line width= 0.4pt,line join=round,line cap=round,fill=fillColor] (191.49,132.85) circle (  1.16);

\path[draw=drawColor,line width= 0.4pt,line join=round,line cap=round,fill=fillColor] (191.69,132.85) circle (  1.16);

\path[draw=drawColor,line width= 0.4pt,line join=round,line cap=round,fill=fillColor] (191.89,132.85) circle (  1.16);

\path[draw=drawColor,line width= 0.4pt,line join=round,line cap=round,fill=fillColor] (192.09,132.85) circle (  1.16);

\path[draw=drawColor,line width= 0.4pt,line join=round,line cap=round,fill=fillColor] (192.29,132.85) circle (  1.16);

\path[draw=drawColor,line width= 0.4pt,line join=round,line cap=round,fill=fillColor] (192.49,132.85) circle (  1.16);

\path[draw=drawColor,line width= 0.4pt,line join=round,line cap=round,fill=fillColor] (192.69,132.85) circle (  1.16);

\path[draw=drawColor,line width= 0.4pt,line join=round,line cap=round,fill=fillColor] (192.88,132.85) circle (  1.16);

\path[draw=drawColor,line width= 0.4pt,line join=round,line cap=round,fill=fillColor] (193.08,132.85) circle (  1.16);

\path[draw=drawColor,line width= 0.4pt,line join=round,line cap=round,fill=fillColor] (193.28,132.85) circle (  1.16);

\path[draw=drawColor,line width= 0.4pt,line join=round,line cap=round,fill=fillColor] (193.47,132.85) circle (  1.16);

\path[draw=drawColor,line width= 0.4pt,line join=round,line cap=round,fill=fillColor] (193.67,132.85) circle (  1.16);

\path[draw=drawColor,line width= 0.4pt,line join=round,line cap=round,fill=fillColor] (193.86,132.85) circle (  1.16);

\path[draw=drawColor,line width= 0.4pt,line join=round,line cap=round,fill=fillColor] (194.06,132.85) circle (  1.16);

\path[draw=drawColor,line width= 0.4pt,line join=round,line cap=round,fill=fillColor] (194.25,132.85) circle (  1.16);

\path[draw=drawColor,line width= 0.4pt,line join=round,line cap=round,fill=fillColor] (194.44,132.85) circle (  1.16);

\path[draw=drawColor,line width= 0.4pt,line join=round,line cap=round,fill=fillColor] (194.63,132.85) circle (  1.16);

\path[draw=drawColor,line width= 0.4pt,line join=round,line cap=round,fill=fillColor] (194.83,132.85) circle (  1.16);

\path[draw=drawColor,line width= 0.4pt,line join=round,line cap=round,fill=fillColor] (195.02,132.85) circle (  1.16);

\path[draw=drawColor,line width= 0.4pt,line join=round,line cap=round,fill=fillColor] (195.21,132.85) circle (  1.16);

\path[draw=drawColor,line width= 0.4pt,line join=round,line cap=round,fill=fillColor] (195.40,132.85) circle (  1.16);

\path[draw=drawColor,line width= 0.4pt,line join=round,line cap=round,fill=fillColor] (195.59,132.85) circle (  1.16);

\path[draw=drawColor,line width= 0.4pt,line join=round,line cap=round,fill=fillColor] (195.78,132.85) circle (  1.16);

\path[draw=drawColor,line width= 0.4pt,line join=round,line cap=round,fill=fillColor] (195.97,132.85) circle (  1.16);

\path[draw=drawColor,line width= 0.4pt,line join=round,line cap=round,fill=fillColor] (196.16,132.85) circle (  1.16);

\path[draw=drawColor,line width= 0.4pt,line join=round,line cap=round,fill=fillColor] (196.34,132.85) circle (  1.16);

\path[draw=drawColor,line width= 0.4pt,line join=round,line cap=round,fill=fillColor] (196.53,132.85) circle (  1.16);

\path[draw=drawColor,line width= 0.4pt,line join=round,line cap=round,fill=fillColor] (196.72,132.85) circle (  1.16);

\path[draw=drawColor,line width= 0.4pt,line join=round,line cap=round,fill=fillColor] (196.91,132.85) circle (  1.16);

\path[draw=drawColor,line width= 0.4pt,line join=round,line cap=round,fill=fillColor] (197.09,132.85) circle (  1.16);

\path[draw=drawColor,line width= 0.4pt,line join=round,line cap=round,fill=fillColor] (197.28,132.85) circle (  1.16);

\path[draw=drawColor,line width= 0.4pt,line join=round,line cap=round,fill=fillColor] (197.46,132.85) circle (  1.16);

\path[draw=drawColor,line width= 0.4pt,line join=round,line cap=round,fill=fillColor] (197.65,132.85) circle (  1.16);

\path[draw=drawColor,line width= 0.4pt,line join=round,line cap=round,fill=fillColor] (197.83,132.85) circle (  1.16);

\path[draw=drawColor,line width= 0.4pt,line join=round,line cap=round,fill=fillColor] (198.01,132.85) circle (  1.16);

\path[draw=drawColor,line width= 0.4pt,line join=round,line cap=round,fill=fillColor] (198.20,132.85) circle (  1.16);

\path[draw=drawColor,line width= 0.4pt,line join=round,line cap=round,fill=fillColor] (198.38,132.85) circle (  1.16);

\path[draw=drawColor,line width= 0.4pt,line join=round,line cap=round,fill=fillColor] (198.56,132.85) circle (  1.16);

\path[draw=drawColor,line width= 0.4pt,line join=round,line cap=round,fill=fillColor] (198.74,132.85) circle (  1.16);

\path[draw=drawColor,line width= 0.4pt,line join=round,line cap=round,fill=fillColor] (198.93,132.85) circle (  1.16);

\path[draw=drawColor,line width= 0.4pt,line join=round,line cap=round,fill=fillColor] (199.11,132.85) circle (  1.16);

\path[draw=drawColor,line width= 0.4pt,line join=round,line cap=round,fill=fillColor] (199.29,132.85) circle (  1.16);

\path[draw=drawColor,line width= 0.4pt,line join=round,line cap=round,fill=fillColor] (199.47,132.85) circle (  1.16);

\path[draw=drawColor,line width= 0.4pt,line join=round,line cap=round,fill=fillColor] (199.65,132.85) circle (  1.16);

\path[draw=drawColor,line width= 0.4pt,line join=round,line cap=round,fill=fillColor] (199.83,132.85) circle (  1.16);

\path[draw=drawColor,line width= 0.4pt,line join=round,line cap=round,fill=fillColor] (200.00,132.85) circle (  1.16);

\path[draw=drawColor,line width= 0.4pt,line join=round,line cap=round,fill=fillColor] (200.18,132.85) circle (  1.16);

\path[draw=drawColor,line width= 0.4pt,line join=round,line cap=round,fill=fillColor] (200.36,132.85) circle (  1.16);

\path[draw=drawColor,line width= 0.4pt,line join=round,line cap=round,fill=fillColor] (200.54,132.85) circle (  1.16);

\path[draw=drawColor,line width= 0.4pt,line join=round,line cap=round,fill=fillColor] (200.72,132.85) circle (  1.16);

\path[draw=drawColor,line width= 0.4pt,line join=round,line cap=round,fill=fillColor] (200.89,132.85) circle (  1.16);

\path[draw=drawColor,line width= 0.4pt,line join=round,line cap=round,fill=fillColor] (201.07,132.85) circle (  1.16);

\path[draw=drawColor,line width= 0.4pt,line join=round,line cap=round,fill=fillColor] (201.24,132.85) circle (  1.16);

\path[draw=drawColor,line width= 0.4pt,line join=round,line cap=round,fill=fillColor] (201.42,132.85) circle (  1.16);

\path[draw=drawColor,line width= 0.4pt,line join=round,line cap=round,fill=fillColor] (201.59,132.85) circle (  1.16);

\path[draw=drawColor,line width= 0.4pt,line join=round,line cap=round,fill=fillColor] (201.77,132.85) circle (  1.16);

\path[draw=drawColor,line width= 0.4pt,line join=round,line cap=round,fill=fillColor] (201.94,132.85) circle (  1.16);

\path[draw=drawColor,line width= 0.4pt,line join=round,line cap=round,fill=fillColor] (202.12,132.85) circle (  1.16);

\path[draw=drawColor,line width= 0.4pt,line join=round,line cap=round,fill=fillColor] (202.29,132.85) circle (  1.16);

\path[draw=drawColor,line width= 0.4pt,line join=round,line cap=round,fill=fillColor] (202.46,132.85) circle (  1.16);

\path[draw=drawColor,line width= 0.4pt,line join=round,line cap=round,fill=fillColor] (202.64,132.85) circle (  1.16);

\path[draw=drawColor,line width= 0.4pt,line join=round,line cap=round,fill=fillColor] (202.81,132.85) circle (  1.16);

\path[draw=drawColor,line width= 0.4pt,line join=round,line cap=round,fill=fillColor] (202.98,132.85) circle (  1.16);

\path[draw=drawColor,line width= 0.4pt,line join=round,line cap=round,fill=fillColor] (203.15,132.85) circle (  1.16);

\path[draw=drawColor,line width= 0.4pt,line join=round,line cap=round,fill=fillColor] (203.32,132.85) circle (  1.16);

\path[draw=drawColor,line width= 0.4pt,line join=round,line cap=round,fill=fillColor] (203.49,132.85) circle (  1.16);

\path[draw=drawColor,line width= 0.4pt,line join=round,line cap=round,fill=fillColor] (203.66,132.85) circle (  1.16);

\path[draw=drawColor,line width= 0.4pt,line join=round,line cap=round,fill=fillColor] (203.83,132.85) circle (  1.16);

\path[draw=drawColor,line width= 0.4pt,line join=round,line cap=round,fill=fillColor] (204.00,132.85) circle (  1.16);

\path[draw=drawColor,line width= 0.4pt,line join=round,line cap=round,fill=fillColor] (204.17,132.85) circle (  1.16);

\path[draw=drawColor,line width= 0.4pt,line join=round,line cap=round,fill=fillColor] (204.34,132.85) circle (  1.16);

\path[draw=drawColor,line width= 0.4pt,line join=round,line cap=round,fill=fillColor] (204.51,132.85) circle (  1.16);

\path[draw=drawColor,line width= 0.4pt,line join=round,line cap=round,fill=fillColor] (204.68,132.85) circle (  1.16);

\path[draw=drawColor,line width= 0.4pt,line join=round,line cap=round,fill=fillColor] (204.84,132.85) circle (  1.16);

\path[draw=drawColor,line width= 0.4pt,line join=round,line cap=round,fill=fillColor] (205.01,132.85) circle (  1.16);

\path[draw=drawColor,line width= 0.4pt,line join=round,line cap=round,fill=fillColor] (205.18,132.85) circle (  1.16);

\path[draw=drawColor,line width= 0.4pt,line join=round,line cap=round,fill=fillColor] (205.34,132.85) circle (  1.16);

\path[draw=drawColor,line width= 0.4pt,line join=round,line cap=round,fill=fillColor] (205.51,132.85) circle (  1.16);

\path[draw=drawColor,line width= 0.4pt,line join=round,line cap=round,fill=fillColor] (205.68,132.85) circle (  1.16);

\path[draw=drawColor,line width= 0.4pt,line join=round,line cap=round,fill=fillColor] (205.84,132.85) circle (  1.16);

\path[draw=drawColor,line width= 0.4pt,line join=round,line cap=round,fill=fillColor] (206.01,132.85) circle (  1.16);

\path[draw=drawColor,line width= 0.4pt,line join=round,line cap=round,fill=fillColor] (206.17,132.85) circle (  1.16);

\path[draw=drawColor,line width= 0.4pt,line join=round,line cap=round,fill=fillColor] (206.34,132.85) circle (  1.16);

\path[draw=drawColor,line width= 0.4pt,line join=round,line cap=round,fill=fillColor] (206.50,132.85) circle (  1.16);

\path[draw=drawColor,line width= 0.4pt,line join=round,line cap=round,fill=fillColor] (206.66,132.85) circle (  1.16);

\path[draw=drawColor,line width= 0.4pt,line join=round,line cap=round,fill=fillColor] (206.83,132.85) circle (  1.16);

\path[draw=drawColor,line width= 0.4pt,line join=round,line cap=round,fill=fillColor] (206.99,132.85) circle (  1.16);

\path[draw=drawColor,line width= 0.4pt,line join=round,line cap=round,fill=fillColor] (207.15,132.85) circle (  1.16);

\path[draw=drawColor,line width= 0.4pt,line join=round,line cap=round,fill=fillColor] (207.31,132.85) circle (  1.16);

\path[draw=drawColor,line width= 0.4pt,line join=round,line cap=round,fill=fillColor] (207.48,132.85) circle (  1.16);

\path[draw=drawColor,line width= 0.4pt,line join=round,line cap=round,fill=fillColor] (207.64,132.85) circle (  1.16);

\path[draw=drawColor,line width= 0.4pt,line join=round,line cap=round,fill=fillColor] (207.80,132.85) circle (  1.16);

\path[draw=drawColor,line width= 0.4pt,line join=round,line cap=round,fill=fillColor] (207.96,132.85) circle (  1.16);

\path[draw=drawColor,line width= 0.4pt,line join=round,line cap=round,fill=fillColor] (208.12,132.85) circle (  1.16);

\path[draw=drawColor,line width= 0.4pt,line join=round,line cap=round,fill=fillColor] (208.28,132.85) circle (  1.16);

\path[draw=drawColor,line width= 0.4pt,line join=round,line cap=round,fill=fillColor] (208.44,132.85) circle (  1.16);

\path[draw=drawColor,line width= 0.4pt,line join=round,line cap=round,fill=fillColor] (208.60,132.85) circle (  1.16);

\path[draw=drawColor,line width= 0.4pt,line join=round,line cap=round,fill=fillColor] (208.76,132.85) circle (  1.16);

\path[draw=drawColor,line width= 0.4pt,line join=round,line cap=round,fill=fillColor] (208.92,132.85) circle (  1.16);

\path[draw=drawColor,line width= 0.4pt,line join=round,line cap=round,fill=fillColor] (209.08,132.85) circle (  1.16);

\path[draw=drawColor,line width= 0.4pt,line join=round,line cap=round,fill=fillColor] (209.24,132.85) circle (  1.16);

\path[draw=drawColor,line width= 0.4pt,line join=round,line cap=round,fill=fillColor] (209.39,132.85) circle (  1.16);

\path[draw=drawColor,line width= 0.4pt,line join=round,line cap=round,fill=fillColor] (209.55,132.85) circle (  1.16);

\path[draw=drawColor,line width= 0.4pt,line join=round,line cap=round,fill=fillColor] (209.71,132.85) circle (  1.16);

\path[draw=drawColor,line width= 0.4pt,line join=round,line cap=round,fill=fillColor] (209.87,132.85) circle (  1.16);

\path[draw=drawColor,line width= 0.4pt,line join=round,line cap=round,fill=fillColor] (210.02,132.85) circle (  1.16);

\path[draw=drawColor,line width= 0.4pt,line join=round,line cap=round,fill=fillColor] (210.18,132.85) circle (  1.16);

\path[draw=drawColor,line width= 0.4pt,line join=round,line cap=round,fill=fillColor] (210.34,132.85) circle (  1.16);

\path[draw=drawColor,line width= 0.4pt,line join=round,line cap=round,fill=fillColor] (210.49,129.98) circle (  1.16);

\path[draw=drawColor,line width= 0.4pt,line join=round,line cap=round,fill=fillColor] (210.65,124.74) circle (  1.16);

\path[draw=drawColor,line width= 0.4pt,line join=round,line cap=round,fill=fillColor] (210.80,119.10) circle (  1.16);

\path[draw=drawColor,line width= 0.4pt,line join=round,line cap=round,fill=fillColor] (210.96,109.69) circle (  1.16);

\path[draw=drawColor,line width= 0.4pt,line join=round,line cap=round,fill=fillColor] (211.11,109.50) circle (  1.16);

\path[draw=drawColor,line width= 0.4pt,line join=round,line cap=round,fill=fillColor] (211.27,109.05) circle (  1.16);

\path[draw=drawColor,line width= 0.4pt,line join=round,line cap=round,fill=fillColor] (211.42,107.27) circle (  1.16);

\path[draw=drawColor,line width= 0.4pt,line join=round,line cap=round,fill=fillColor] (211.57,105.43) circle (  1.16);

\path[draw=drawColor,line width= 0.4pt,line join=round,line cap=round,fill=fillColor] (211.73,105.36) circle (  1.16);

\path[draw=drawColor,line width= 0.4pt,line join=round,line cap=round,fill=fillColor] (211.88,104.85) circle (  1.16);

\path[draw=drawColor,line width= 0.4pt,line join=round,line cap=round,fill=fillColor] (212.03,104.84) circle (  1.16);

\path[draw=drawColor,line width= 0.4pt,line join=round,line cap=round,fill=fillColor] (212.19,103.38) circle (  1.16);

\path[draw=drawColor,line width= 0.4pt,line join=round,line cap=round,fill=fillColor] (212.34,102.75) circle (  1.16);

\path[draw=drawColor,line width= 0.4pt,line join=round,line cap=round,fill=fillColor] (212.49, 97.05) circle (  1.16);

\path[draw=drawColor,line width= 0.4pt,line join=round,line cap=round,fill=fillColor] (212.64, 96.76) circle (  1.16);

\path[draw=drawColor,line width= 0.4pt,line join=round,line cap=round,fill=fillColor] (212.79, 95.85) circle (  1.16);

\path[draw=drawColor,line width= 0.4pt,line join=round,line cap=round,fill=fillColor] (212.94, 94.88) circle (  1.16);

\path[draw=drawColor,line width= 0.4pt,line join=round,line cap=round,fill=fillColor] (213.09, 94.42) circle (  1.16);

\path[draw=drawColor,line width= 0.4pt,line join=round,line cap=round,fill=fillColor] (213.25, 92.74) circle (  1.16);

\path[draw=drawColor,line width= 0.4pt,line join=round,line cap=round,fill=fillColor] (213.40, 92.26) circle (  1.16);

\path[draw=drawColor,line width= 0.4pt,line join=round,line cap=round,fill=fillColor] (213.55, 91.68) circle (  1.16);

\path[draw=drawColor,line width= 0.4pt,line join=round,line cap=round,fill=fillColor] (213.70, 91.03) circle (  1.16);

\path[draw=drawColor,line width= 0.4pt,line join=round,line cap=round,fill=fillColor] (213.84, 90.87) circle (  1.16);

\path[draw=drawColor,line width= 0.4pt,line join=round,line cap=round,fill=fillColor] (213.99, 90.73) circle (  1.16);

\path[draw=drawColor,line width= 0.4pt,line join=round,line cap=round,fill=fillColor] (214.14, 90.36) circle (  1.16);

\path[draw=drawColor,line width= 0.4pt,line join=round,line cap=round,fill=fillColor] (214.29, 89.23) circle (  1.16);

\path[draw=drawColor,line width= 0.4pt,line join=round,line cap=round,fill=fillColor] (214.44, 88.62) circle (  1.16);

\path[draw=drawColor,line width= 0.4pt,line join=round,line cap=round,fill=fillColor] (214.59, 88.12) circle (  1.16);

\path[draw=drawColor,line width= 0.4pt,line join=round,line cap=round,fill=fillColor] (214.74, 87.97) circle (  1.16);

\path[draw=drawColor,line width= 0.4pt,line join=round,line cap=round,fill=fillColor] (214.88, 87.83) circle (  1.16);

\path[draw=drawColor,line width= 0.4pt,line join=round,line cap=round,fill=fillColor] (215.03, 87.42) circle (  1.16);

\path[draw=drawColor,line width= 0.4pt,line join=round,line cap=round,fill=fillColor] (215.18, 87.35) circle (  1.16);

\path[draw=drawColor,line width= 0.4pt,line join=round,line cap=round,fill=fillColor] (215.32, 86.48) circle (  1.16);

\path[draw=drawColor,line width= 0.4pt,line join=round,line cap=round,fill=fillColor] (215.47, 84.23) circle (  1.16);

\path[draw=drawColor,line width= 0.4pt,line join=round,line cap=round,fill=fillColor] (215.62, 83.78) circle (  1.16);

\path[draw=drawColor,line width= 0.4pt,line join=round,line cap=round,fill=fillColor] (215.76, 83.58) circle (  1.16);

\path[draw=drawColor,line width= 0.4pt,line join=round,line cap=round,fill=fillColor] (215.91, 83.20) circle (  1.16);

\path[draw=drawColor,line width= 0.4pt,line join=round,line cap=round,fill=fillColor] (216.05, 82.48) circle (  1.16);

\path[draw=drawColor,line width= 0.4pt,line join=round,line cap=round,fill=fillColor] (216.20, 82.37) circle (  1.16);

\path[draw=drawColor,line width= 0.4pt,line join=round,line cap=round,fill=fillColor] (216.34, 81.83) circle (  1.16);

\path[draw=drawColor,line width= 0.4pt,line join=round,line cap=round,fill=fillColor] (216.49, 81.56) circle (  1.16);

\path[draw=drawColor,line width= 0.4pt,line join=round,line cap=round,fill=fillColor] (216.63, 81.52) circle (  1.16);

\path[draw=drawColor,line width= 0.4pt,line join=round,line cap=round,fill=fillColor] (216.78, 81.34) circle (  1.16);

\path[draw=drawColor,line width= 0.4pt,line join=round,line cap=round,fill=fillColor] (216.92, 80.93) circle (  1.16);

\path[draw=drawColor,line width= 0.4pt,line join=round,line cap=round,fill=fillColor] (217.06, 80.70) circle (  1.16);

\path[draw=drawColor,line width= 0.4pt,line join=round,line cap=round,fill=fillColor] (217.21, 80.35) circle (  1.16);

\path[draw=drawColor,line width= 0.4pt,line join=round,line cap=round,fill=fillColor] (217.35, 79.94) circle (  1.16);

\path[draw=drawColor,line width= 0.4pt,line join=round,line cap=round,fill=fillColor] (217.49, 79.75) circle (  1.16);

\path[draw=drawColor,line width= 0.4pt,line join=round,line cap=round,fill=fillColor] (217.64, 79.12) circle (  1.16);

\path[draw=drawColor,line width= 0.4pt,line join=round,line cap=round,fill=fillColor] (217.78, 78.75) circle (  1.16);

\path[draw=drawColor,line width= 0.4pt,line join=round,line cap=round,fill=fillColor] (217.92, 78.44) circle (  1.16);

\path[draw=drawColor,line width= 0.4pt,line join=round,line cap=round,fill=fillColor] (218.06, 78.26) circle (  1.16);

\path[draw=drawColor,line width= 0.4pt,line join=round,line cap=round,fill=fillColor] (218.20, 78.22) circle (  1.16);

\path[draw=drawColor,line width= 0.4pt,line join=round,line cap=round,fill=fillColor] (218.35, 78.07) circle (  1.16);

\path[draw=drawColor,line width= 0.4pt,line join=round,line cap=round,fill=fillColor] (218.49, 78.05) circle (  1.16);

\path[draw=drawColor,line width= 0.4pt,line join=round,line cap=round,fill=fillColor] (218.63, 78.02) circle (  1.16);

\path[draw=drawColor,line width= 0.4pt,line join=round,line cap=round,fill=fillColor] (218.77, 78.00) circle (  1.16);

\path[draw=drawColor,line width= 0.4pt,line join=round,line cap=round,fill=fillColor] (218.91, 77.83) circle (  1.16);

\path[draw=drawColor,line width= 0.4pt,line join=round,line cap=round,fill=fillColor] (219.05, 77.69) circle (  1.16);

\path[draw=drawColor,line width= 0.4pt,line join=round,line cap=round,fill=fillColor] (219.19, 77.67) circle (  1.16);

\path[draw=drawColor,line width= 0.4pt,line join=round,line cap=round,fill=fillColor] (219.33, 77.39) circle (  1.16);

\path[draw=drawColor,line width= 0.4pt,line join=round,line cap=round,fill=fillColor] (219.47, 77.19) circle (  1.16);

\path[draw=drawColor,line width= 0.4pt,line join=round,line cap=round,fill=fillColor] (219.61, 76.78) circle (  1.16);

\path[draw=drawColor,line width= 0.4pt,line join=round,line cap=round,fill=fillColor] (219.75, 76.33) circle (  1.16);

\path[draw=drawColor,line width= 0.4pt,line join=round,line cap=round,fill=fillColor] (219.89, 75.89) circle (  1.16);

\path[draw=drawColor,line width= 0.4pt,line join=round,line cap=round,fill=fillColor] (220.02, 75.45) circle (  1.16);

\path[draw=drawColor,line width= 0.4pt,line join=round,line cap=round,fill=fillColor] (220.16, 75.24) circle (  1.16);

\path[draw=drawColor,line width= 0.4pt,line join=round,line cap=round,fill=fillColor] (220.30, 74.80) circle (  1.16);

\path[draw=drawColor,line width= 0.4pt,line join=round,line cap=round,fill=fillColor] (220.44, 74.74) circle (  1.16);

\path[draw=drawColor,line width= 0.4pt,line join=round,line cap=round,fill=fillColor] (220.58, 74.61) circle (  1.16);

\path[draw=drawColor,line width= 0.4pt,line join=round,line cap=round,fill=fillColor] (220.71, 74.57) circle (  1.16);

\path[draw=drawColor,line width= 0.4pt,line join=round,line cap=round,fill=fillColor] (220.85, 74.53) circle (  1.16);

\path[draw=drawColor,line width= 0.4pt,line join=round,line cap=round,fill=fillColor] (220.99, 74.27) circle (  1.16);

\path[draw=drawColor,line width= 0.4pt,line join=round,line cap=round,fill=fillColor] (221.13, 74.07) circle (  1.16);

\path[draw=drawColor,line width= 0.4pt,line join=round,line cap=round,fill=fillColor] (221.26, 73.84) circle (  1.16);

\path[draw=drawColor,line width= 0.4pt,line join=round,line cap=round,fill=fillColor] (221.40, 73.55) circle (  1.16);

\path[draw=drawColor,line width= 0.4pt,line join=round,line cap=round,fill=fillColor] (221.53, 73.41) circle (  1.16);

\path[draw=drawColor,line width= 0.4pt,line join=round,line cap=round,fill=fillColor] (221.67, 73.21) circle (  1.16);

\path[draw=drawColor,line width= 0.4pt,line join=round,line cap=round,fill=fillColor] (221.81, 73.14) circle (  1.16);

\path[draw=drawColor,line width= 0.4pt,line join=round,line cap=round,fill=fillColor] (221.94, 73.01) circle (  1.16);

\path[draw=drawColor,line width= 0.4pt,line join=round,line cap=round,fill=fillColor] (222.08, 73.00) circle (  1.16);

\path[draw=drawColor,line width= 0.4pt,line join=round,line cap=round,fill=fillColor] (222.21, 72.93) circle (  1.16);

\path[draw=drawColor,line width= 0.4pt,line join=round,line cap=round,fill=fillColor] (222.35, 72.58) circle (  1.16);

\path[draw=drawColor,line width= 0.4pt,line join=round,line cap=round,fill=fillColor] (222.48, 72.47) circle (  1.16);

\path[draw=drawColor,line width= 0.4pt,line join=round,line cap=round,fill=fillColor] (222.62, 72.40) circle (  1.16);

\path[draw=drawColor,line width= 0.4pt,line join=round,line cap=round,fill=fillColor] (222.75, 72.31) circle (  1.16);

\path[draw=drawColor,line width= 0.4pt,line join=round,line cap=round,fill=fillColor] (222.88, 72.16) circle (  1.16);

\path[draw=drawColor,line width= 0.4pt,line join=round,line cap=round,fill=fillColor] (223.02, 72.05) circle (  1.16);

\path[draw=drawColor,line width= 0.4pt,line join=round,line cap=round,fill=fillColor] (223.15, 70.16) circle (  1.16);

\path[draw=drawColor,line width= 0.4pt,line join=round,line cap=round,fill=fillColor] (223.28, 69.90) circle (  1.16);

\path[draw=drawColor,line width= 0.4pt,line join=round,line cap=round,fill=fillColor] (223.42, 69.80) circle (  1.16);

\path[draw=drawColor,line width= 0.4pt,line join=round,line cap=round,fill=fillColor] (223.55, 69.16) circle (  1.16);

\path[draw=drawColor,line width= 0.4pt,line join=round,line cap=round,fill=fillColor] (223.68, 68.31) circle (  1.16);

\path[draw=drawColor,line width= 0.4pt,line join=round,line cap=round,fill=fillColor] (223.82, 68.29) circle (  1.16);

\path[draw=drawColor,line width= 0.4pt,line join=round,line cap=round,fill=fillColor] (223.95, 68.26) circle (  1.16);

\path[draw=drawColor,line width= 0.4pt,line join=round,line cap=round,fill=fillColor] (224.08, 68.09) circle (  1.16);

\path[draw=drawColor,line width= 0.4pt,line join=round,line cap=round,fill=fillColor] (224.21, 65.73) circle (  1.16);

\path[draw=drawColor,line width= 0.4pt,line join=round,line cap=round,fill=fillColor] (224.34, 64.11) circle (  1.16);

\path[draw=drawColor,line width= 0.4pt,line join=round,line cap=round,fill=fillColor] (224.48, 61.80) circle (  1.16);

\path[draw=drawColor,line width= 0.4pt,line join=round,line cap=round,fill=fillColor] (224.61, 47.20) circle (  1.16);

\path[draw=drawColor,line width= 0.4pt,line join=round,line cap=round,fill=fillColor] (224.74, 47.20) circle (  1.16);

\path[draw=drawColor,line width= 0.4pt,line join=round,line cap=round,fill=fillColor] (224.87, 47.20) circle (  1.16);

\path[draw=drawColor,line width= 0.4pt,line join=round,line cap=round,fill=fillColor] (225.00, 47.20) circle (  1.16);

\path[draw=drawColor,line width= 0.4pt,line join=round,line cap=round,fill=fillColor] (225.13, 47.20) circle (  1.16);

\path[draw=drawColor,line width= 0.4pt,line join=round,line cap=round,fill=fillColor] (225.26, 47.20) circle (  1.16);
\definecolor[named]{drawColor}{rgb}{0.60,0.31,0.64}
\definecolor[named]{fillColor}{rgb}{0.60,0.31,0.64}

\path[draw=drawColor,line width= 0.4pt,line join=round,line cap=round,fill=fillColor] ( 74.88,116.78) circle (  1.16);

\path[draw=drawColor,line width= 0.4pt,line join=round,line cap=round,fill=fillColor] ( 80.78,111.65) circle (  1.16);

\path[draw=drawColor,line width= 0.4pt,line join=round,line cap=round,fill=fillColor] ( 84.92,111.01) circle (  1.16);

\path[draw=drawColor,line width= 0.4pt,line join=round,line cap=round,fill=fillColor] ( 88.22,109.98) circle (  1.16);

\path[draw=drawColor,line width= 0.4pt,line join=round,line cap=round,fill=fillColor] ( 91.00,108.29) circle (  1.16);

\path[draw=drawColor,line width= 0.4pt,line join=round,line cap=round,fill=fillColor] ( 93.43,107.67) circle (  1.16);

\path[draw=drawColor,line width= 0.4pt,line join=round,line cap=round,fill=fillColor] ( 95.61,106.93) circle (  1.16);

\path[draw=drawColor,line width= 0.4pt,line join=round,line cap=round,fill=fillColor] ( 97.58,104.88) circle (  1.16);

\path[draw=drawColor,line width= 0.4pt,line join=round,line cap=round,fill=fillColor] ( 99.40,104.72) circle (  1.16);

\path[draw=drawColor,line width= 0.4pt,line join=round,line cap=round,fill=fillColor] (101.09,104.42) circle (  1.16);

\path[draw=drawColor,line width= 0.4pt,line join=round,line cap=round,fill=fillColor] (102.67,103.58) circle (  1.16);

\path[draw=drawColor,line width= 0.4pt,line join=round,line cap=round,fill=fillColor] (104.16,103.34) circle (  1.16);

\path[draw=drawColor,line width= 0.4pt,line join=round,line cap=round,fill=fillColor] (105.56,102.72) circle (  1.16);

\path[draw=drawColor,line width= 0.4pt,line join=round,line cap=round,fill=fillColor] (106.90,102.45) circle (  1.16);

\path[draw=drawColor,line width= 0.4pt,line join=round,line cap=round,fill=fillColor] (108.17,102.15) circle (  1.16);

\path[draw=drawColor,line width= 0.4pt,line join=round,line cap=round,fill=fillColor] (109.39,102.12) circle (  1.16);

\path[draw=drawColor,line width= 0.4pt,line join=round,line cap=round,fill=fillColor] (110.56,101.99) circle (  1.16);

\path[draw=drawColor,line width= 0.4pt,line join=round,line cap=round,fill=fillColor] (111.68,101.81) circle (  1.16);

\path[draw=drawColor,line width= 0.4pt,line join=round,line cap=round,fill=fillColor] (112.76,101.71) circle (  1.16);

\path[draw=drawColor,line width= 0.4pt,line join=round,line cap=round,fill=fillColor] (113.81,101.49) circle (  1.16);

\path[draw=drawColor,line width= 0.4pt,line join=round,line cap=round,fill=fillColor] (114.82,101.38) circle (  1.16);

\path[draw=drawColor,line width= 0.4pt,line join=round,line cap=round,fill=fillColor] (115.80,100.72) circle (  1.16);

\path[draw=drawColor,line width= 0.4pt,line join=round,line cap=round,fill=fillColor] (116.74,100.00) circle (  1.16);

\path[draw=drawColor,line width= 0.4pt,line join=round,line cap=round,fill=fillColor] (117.67, 99.74) circle (  1.16);

\path[draw=drawColor,line width= 0.4pt,line join=round,line cap=round,fill=fillColor] (118.56, 98.75) circle (  1.16);

\path[draw=drawColor,line width= 0.4pt,line join=round,line cap=round,fill=fillColor] (119.44, 98.40) circle (  1.16);

\path[draw=drawColor,line width= 0.4pt,line join=round,line cap=round,fill=fillColor] (120.29, 98.31) circle (  1.16);

\path[draw=drawColor,line width= 0.4pt,line join=round,line cap=round,fill=fillColor] (121.12, 97.88) circle (  1.16);

\path[draw=drawColor,line width= 0.4pt,line join=round,line cap=round,fill=fillColor] (121.93, 97.77) circle (  1.16);

\path[draw=drawColor,line width= 0.4pt,line join=round,line cap=round,fill=fillColor] (122.72, 97.49) circle (  1.16);

\path[draw=drawColor,line width= 0.4pt,line join=round,line cap=round,fill=fillColor] (123.50, 97.43) circle (  1.16);

\path[draw=drawColor,line width= 0.4pt,line join=round,line cap=round,fill=fillColor] (124.26, 97.31) circle (  1.16);

\path[draw=drawColor,line width= 0.4pt,line join=round,line cap=round,fill=fillColor] (125.00, 97.15) circle (  1.16);

\path[draw=drawColor,line width= 0.4pt,line join=round,line cap=round,fill=fillColor] (125.73, 96.95) circle (  1.16);

\path[draw=drawColor,line width= 0.4pt,line join=round,line cap=round,fill=fillColor] (126.44, 96.62) circle (  1.16);

\path[draw=drawColor,line width= 0.4pt,line join=round,line cap=round,fill=fillColor] (127.15, 96.23) circle (  1.16);

\path[draw=drawColor,line width= 0.4pt,line join=round,line cap=round,fill=fillColor] (127.83, 96.14) circle (  1.16);

\path[draw=drawColor,line width= 0.4pt,line join=round,line cap=round,fill=fillColor] (128.51, 96.01) circle (  1.16);

\path[draw=drawColor,line width= 0.4pt,line join=round,line cap=round,fill=fillColor] (129.17, 95.37) circle (  1.16);

\path[draw=drawColor,line width= 0.4pt,line join=round,line cap=round,fill=fillColor] (129.83, 95.22) circle (  1.16);

\path[draw=drawColor,line width= 0.4pt,line join=round,line cap=round,fill=fillColor] (130.47, 95.04) circle (  1.16);

\path[draw=drawColor,line width= 0.4pt,line join=round,line cap=round,fill=fillColor] (131.10, 94.58) circle (  1.16);

\path[draw=drawColor,line width= 0.4pt,line join=round,line cap=round,fill=fillColor] (131.72, 94.55) circle (  1.16);

\path[draw=drawColor,line width= 0.4pt,line join=round,line cap=round,fill=fillColor] (132.33, 94.38) circle (  1.16);

\path[draw=drawColor,line width= 0.4pt,line join=round,line cap=round,fill=fillColor] (132.93, 94.19) circle (  1.16);

\path[draw=drawColor,line width= 0.4pt,line join=round,line cap=round,fill=fillColor] (133.53, 94.08) circle (  1.16);

\path[draw=drawColor,line width= 0.4pt,line join=round,line cap=round,fill=fillColor] (134.11, 93.81) circle (  1.16);

\path[draw=drawColor,line width= 0.4pt,line join=round,line cap=round,fill=fillColor] (134.69, 93.45) circle (  1.16);

\path[draw=drawColor,line width= 0.4pt,line join=round,line cap=round,fill=fillColor] (135.26, 93.17) circle (  1.16);

\path[draw=drawColor,line width= 0.4pt,line join=round,line cap=round,fill=fillColor] (135.82, 93.02) circle (  1.16);

\path[draw=drawColor,line width= 0.4pt,line join=round,line cap=round,fill=fillColor] (136.38, 92.68) circle (  1.16);

\path[draw=drawColor,line width= 0.4pt,line join=round,line cap=round,fill=fillColor] (136.92, 92.31) circle (  1.16);

\path[draw=drawColor,line width= 0.4pt,line join=round,line cap=round,fill=fillColor] (137.46, 92.26) circle (  1.16);

\path[draw=drawColor,line width= 0.4pt,line join=round,line cap=round,fill=fillColor] (137.99, 92.24) circle (  1.16);

\path[draw=drawColor,line width= 0.4pt,line join=round,line cap=round,fill=fillColor] (138.52, 92.21) circle (  1.16);

\path[draw=drawColor,line width= 0.4pt,line join=round,line cap=round,fill=fillColor] (139.04, 91.93) circle (  1.16);

\path[draw=drawColor,line width= 0.4pt,line join=round,line cap=round,fill=fillColor] (139.56, 91.84) circle (  1.16);

\path[draw=drawColor,line width= 0.4pt,line join=round,line cap=round,fill=fillColor] (140.06, 91.80) circle (  1.16);

\path[draw=drawColor,line width= 0.4pt,line join=round,line cap=round,fill=fillColor] (140.57, 91.63) circle (  1.16);

\path[draw=drawColor,line width= 0.4pt,line join=round,line cap=round,fill=fillColor] (141.06, 91.57) circle (  1.16);

\path[draw=drawColor,line width= 0.4pt,line join=round,line cap=round,fill=fillColor] (141.55, 91.41) circle (  1.16);

\path[draw=drawColor,line width= 0.4pt,line join=round,line cap=round,fill=fillColor] (142.04, 91.40) circle (  1.16);

\path[draw=drawColor,line width= 0.4pt,line join=round,line cap=round,fill=fillColor] (142.52, 90.94) circle (  1.16);

\path[draw=drawColor,line width= 0.4pt,line join=round,line cap=round,fill=fillColor] (143.00, 90.92) circle (  1.16);

\path[draw=drawColor,line width= 0.4pt,line join=round,line cap=round,fill=fillColor] (143.47, 90.81) circle (  1.16);

\path[draw=drawColor,line width= 0.4pt,line join=round,line cap=round,fill=fillColor] (143.93, 90.80) circle (  1.16);

\path[draw=drawColor,line width= 0.4pt,line join=round,line cap=round,fill=fillColor] (144.39, 90.70) circle (  1.16);

\path[draw=drawColor,line width= 0.4pt,line join=round,line cap=round,fill=fillColor] (144.85, 90.60) circle (  1.16);

\path[draw=drawColor,line width= 0.4pt,line join=round,line cap=round,fill=fillColor] (145.30, 90.18) circle (  1.16);

\path[draw=drawColor,line width= 0.4pt,line join=round,line cap=round,fill=fillColor] (145.75, 90.09) circle (  1.16);

\path[draw=drawColor,line width= 0.4pt,line join=round,line cap=round,fill=fillColor] (146.19, 90.06) circle (  1.16);

\path[draw=drawColor,line width= 0.4pt,line join=round,line cap=round,fill=fillColor] (146.63, 89.76) circle (  1.16);

\path[draw=drawColor,line width= 0.4pt,line join=round,line cap=round,fill=fillColor] (147.07, 89.74) circle (  1.16);

\path[draw=drawColor,line width= 0.4pt,line join=round,line cap=round,fill=fillColor] (147.50, 89.52) circle (  1.16);

\path[draw=drawColor,line width= 0.4pt,line join=round,line cap=round,fill=fillColor] (147.93, 89.44) circle (  1.16);

\path[draw=drawColor,line width= 0.4pt,line join=round,line cap=round,fill=fillColor] (148.35, 89.44) circle (  1.16);

\path[draw=drawColor,line width= 0.4pt,line join=round,line cap=round,fill=fillColor] (148.77, 89.07) circle (  1.16);

\path[draw=drawColor,line width= 0.4pt,line join=round,line cap=round,fill=fillColor] (149.19, 88.85) circle (  1.16);

\path[draw=drawColor,line width= 0.4pt,line join=round,line cap=round,fill=fillColor] (149.60, 88.66) circle (  1.16);

\path[draw=drawColor,line width= 0.4pt,line join=round,line cap=round,fill=fillColor] (150.01, 88.62) circle (  1.16);

\path[draw=drawColor,line width= 0.4pt,line join=round,line cap=round,fill=fillColor] (150.41, 88.57) circle (  1.16);

\path[draw=drawColor,line width= 0.4pt,line join=round,line cap=round,fill=fillColor] (150.82, 88.51) circle (  1.16);

\path[draw=drawColor,line width= 0.4pt,line join=round,line cap=round,fill=fillColor] (151.22, 88.42) circle (  1.16);

\path[draw=drawColor,line width= 0.4pt,line join=round,line cap=round,fill=fillColor] (151.61, 88.03) circle (  1.16);

\path[draw=drawColor,line width= 0.4pt,line join=round,line cap=round,fill=fillColor] (152.01, 87.99) circle (  1.16);

\path[draw=drawColor,line width= 0.4pt,line join=round,line cap=round,fill=fillColor] (152.40, 87.90) circle (  1.16);

\path[draw=drawColor,line width= 0.4pt,line join=round,line cap=round,fill=fillColor] (152.78, 87.82) circle (  1.16);

\path[draw=drawColor,line width= 0.4pt,line join=round,line cap=round,fill=fillColor] (153.17, 87.41) circle (  1.16);

\path[draw=drawColor,line width= 0.4pt,line join=round,line cap=round,fill=fillColor] (153.55, 87.35) circle (  1.16);

\path[draw=drawColor,line width= 0.4pt,line join=round,line cap=round,fill=fillColor] (153.93, 87.23) circle (  1.16);

\path[draw=drawColor,line width= 0.4pt,line join=round,line cap=round,fill=fillColor] (154.30, 87.08) circle (  1.16);

\path[draw=drawColor,line width= 0.4pt,line join=round,line cap=round,fill=fillColor] (154.67, 87.06) circle (  1.16);

\path[draw=drawColor,line width= 0.4pt,line join=round,line cap=round,fill=fillColor] (155.04, 86.97) circle (  1.16);

\path[draw=drawColor,line width= 0.4pt,line join=round,line cap=round,fill=fillColor] (155.41, 86.96) circle (  1.16);

\path[draw=drawColor,line width= 0.4pt,line join=round,line cap=round,fill=fillColor] (155.78, 86.74) circle (  1.16);

\path[draw=drawColor,line width= 0.4pt,line join=round,line cap=round,fill=fillColor] (156.14, 86.65) circle (  1.16);

\path[draw=drawColor,line width= 0.4pt,line join=round,line cap=round,fill=fillColor] (156.50, 86.62) circle (  1.16);

\path[draw=drawColor,line width= 0.4pt,line join=round,line cap=round,fill=fillColor] (156.86, 86.53) circle (  1.16);

\path[draw=drawColor,line width= 0.4pt,line join=round,line cap=round,fill=fillColor] (157.21, 86.49) circle (  1.16);

\path[draw=drawColor,line width= 0.4pt,line join=round,line cap=round,fill=fillColor] (157.56, 86.45) circle (  1.16);

\path[draw=drawColor,line width= 0.4pt,line join=round,line cap=round,fill=fillColor] (157.91, 86.39) circle (  1.16);

\path[draw=drawColor,line width= 0.4pt,line join=round,line cap=round,fill=fillColor] (158.26, 86.26) circle (  1.16);

\path[draw=drawColor,line width= 0.4pt,line join=round,line cap=round,fill=fillColor] (158.61, 85.97) circle (  1.16);

\path[draw=drawColor,line width= 0.4pt,line join=round,line cap=round,fill=fillColor] (158.95, 85.96) circle (  1.16);

\path[draw=drawColor,line width= 0.4pt,line join=round,line cap=round,fill=fillColor] (159.29, 85.93) circle (  1.16);

\path[draw=drawColor,line width= 0.4pt,line join=round,line cap=round,fill=fillColor] (159.63, 85.91) circle (  1.16);

\path[draw=drawColor,line width= 0.4pt,line join=round,line cap=round,fill=fillColor] (159.97, 85.87) circle (  1.16);

\path[draw=drawColor,line width= 0.4pt,line join=round,line cap=round,fill=fillColor] (160.30, 85.78) circle (  1.16);

\path[draw=drawColor,line width= 0.4pt,line join=round,line cap=round,fill=fillColor] (160.63, 85.69) circle (  1.16);

\path[draw=drawColor,line width= 0.4pt,line join=round,line cap=round,fill=fillColor] (160.96, 85.66) circle (  1.16);

\path[draw=drawColor,line width= 0.4pt,line join=round,line cap=round,fill=fillColor] (161.29, 85.65) circle (  1.16);

\path[draw=drawColor,line width= 0.4pt,line join=round,line cap=round,fill=fillColor] (161.62, 85.64) circle (  1.16);

\path[draw=drawColor,line width= 0.4pt,line join=round,line cap=round,fill=fillColor] (161.95, 85.61) circle (  1.16);

\path[draw=drawColor,line width= 0.4pt,line join=round,line cap=round,fill=fillColor] (162.27, 85.56) circle (  1.16);

\path[draw=drawColor,line width= 0.4pt,line join=round,line cap=round,fill=fillColor] (162.59, 85.55) circle (  1.16);

\path[draw=drawColor,line width= 0.4pt,line join=round,line cap=round,fill=fillColor] (162.91, 85.55) circle (  1.16);

\path[draw=drawColor,line width= 0.4pt,line join=round,line cap=round,fill=fillColor] (163.23, 85.43) circle (  1.16);

\path[draw=drawColor,line width= 0.4pt,line join=round,line cap=round,fill=fillColor] (163.54, 85.18) circle (  1.16);

\path[draw=drawColor,line width= 0.4pt,line join=round,line cap=round,fill=fillColor] (163.85, 85.18) circle (  1.16);

\path[draw=drawColor,line width= 0.4pt,line join=round,line cap=round,fill=fillColor] (164.17, 85.11) circle (  1.16);

\path[draw=drawColor,line width= 0.4pt,line join=round,line cap=round,fill=fillColor] (164.48, 85.11) circle (  1.16);

\path[draw=drawColor,line width= 0.4pt,line join=round,line cap=round,fill=fillColor] (164.79, 85.05) circle (  1.16);

\path[draw=drawColor,line width= 0.4pt,line join=round,line cap=round,fill=fillColor] (165.09, 84.95) circle (  1.16);

\path[draw=drawColor,line width= 0.4pt,line join=round,line cap=round,fill=fillColor] (165.40, 84.93) circle (  1.16);

\path[draw=drawColor,line width= 0.4pt,line join=round,line cap=round,fill=fillColor] (165.70, 84.92) circle (  1.16);

\path[draw=drawColor,line width= 0.4pt,line join=round,line cap=round,fill=fillColor] (166.00, 84.89) circle (  1.16);

\path[draw=drawColor,line width= 0.4pt,line join=round,line cap=round,fill=fillColor] (166.30, 84.86) circle (  1.16);

\path[draw=drawColor,line width= 0.4pt,line join=round,line cap=round,fill=fillColor] (166.60, 84.86) circle (  1.16);

\path[draw=drawColor,line width= 0.4pt,line join=round,line cap=round,fill=fillColor] (166.90, 84.77) circle (  1.16);

\path[draw=drawColor,line width= 0.4pt,line join=round,line cap=round,fill=fillColor] (167.19, 84.73) circle (  1.16);

\path[draw=drawColor,line width= 0.4pt,line join=round,line cap=round,fill=fillColor] (167.49, 84.64) circle (  1.16);

\path[draw=drawColor,line width= 0.4pt,line join=round,line cap=round,fill=fillColor] (167.78, 84.59) circle (  1.16);

\path[draw=drawColor,line width= 0.4pt,line join=round,line cap=round,fill=fillColor] (168.07, 84.58) circle (  1.16);

\path[draw=drawColor,line width= 0.4pt,line join=round,line cap=round,fill=fillColor] (168.36, 84.58) circle (  1.16);

\path[draw=drawColor,line width= 0.4pt,line join=round,line cap=round,fill=fillColor] (168.65, 84.58) circle (  1.16);

\path[draw=drawColor,line width= 0.4pt,line join=round,line cap=round,fill=fillColor] (168.94, 84.50) circle (  1.16);

\path[draw=drawColor,line width= 0.4pt,line join=round,line cap=round,fill=fillColor] (169.22, 84.44) circle (  1.16);

\path[draw=drawColor,line width= 0.4pt,line join=round,line cap=round,fill=fillColor] (169.51, 84.38) circle (  1.16);

\path[draw=drawColor,line width= 0.4pt,line join=round,line cap=round,fill=fillColor] (169.79, 84.37) circle (  1.16);

\path[draw=drawColor,line width= 0.4pt,line join=round,line cap=round,fill=fillColor] (170.07, 84.29) circle (  1.16);

\path[draw=drawColor,line width= 0.4pt,line join=round,line cap=round,fill=fillColor] (170.35, 84.29) circle (  1.16);

\path[draw=drawColor,line width= 0.4pt,line join=round,line cap=round,fill=fillColor] (170.63, 84.28) circle (  1.16);

\path[draw=drawColor,line width= 0.4pt,line join=round,line cap=round,fill=fillColor] (170.91, 84.23) circle (  1.16);

\path[draw=drawColor,line width= 0.4pt,line join=round,line cap=round,fill=fillColor] (171.18, 84.16) circle (  1.16);

\path[draw=drawColor,line width= 0.4pt,line join=round,line cap=round,fill=fillColor] (171.46, 84.14) circle (  1.16);

\path[draw=drawColor,line width= 0.4pt,line join=round,line cap=round,fill=fillColor] (171.73, 84.02) circle (  1.16);

\path[draw=drawColor,line width= 0.4pt,line join=round,line cap=round,fill=fillColor] (172.00, 83.99) circle (  1.16);

\path[draw=drawColor,line width= 0.4pt,line join=round,line cap=round,fill=fillColor] (172.28, 83.88) circle (  1.16);

\path[draw=drawColor,line width= 0.4pt,line join=round,line cap=round,fill=fillColor] (172.55, 83.82) circle (  1.16);

\path[draw=drawColor,line width= 0.4pt,line join=round,line cap=round,fill=fillColor] (172.81, 83.75) circle (  1.16);

\path[draw=drawColor,line width= 0.4pt,line join=round,line cap=round,fill=fillColor] (173.08, 83.74) circle (  1.16);

\path[draw=drawColor,line width= 0.4pt,line join=round,line cap=round,fill=fillColor] (173.35, 83.68) circle (  1.16);

\path[draw=drawColor,line width= 0.4pt,line join=round,line cap=round,fill=fillColor] (173.61, 83.64) circle (  1.16);

\path[draw=drawColor,line width= 0.4pt,line join=round,line cap=round,fill=fillColor] (173.88, 83.63) circle (  1.16);

\path[draw=drawColor,line width= 0.4pt,line join=round,line cap=round,fill=fillColor] (174.14, 83.56) circle (  1.16);

\path[draw=drawColor,line width= 0.4pt,line join=round,line cap=round,fill=fillColor] (174.40, 83.54) circle (  1.16);

\path[draw=drawColor,line width= 0.4pt,line join=round,line cap=round,fill=fillColor] (174.66, 83.52) circle (  1.16);

\path[draw=drawColor,line width= 0.4pt,line join=round,line cap=round,fill=fillColor] (174.92, 83.49) circle (  1.16);

\path[draw=drawColor,line width= 0.4pt,line join=round,line cap=round,fill=fillColor] (175.18, 83.47) circle (  1.16);

\path[draw=drawColor,line width= 0.4pt,line join=round,line cap=round,fill=fillColor] (175.44, 83.42) circle (  1.16);

\path[draw=drawColor,line width= 0.4pt,line join=round,line cap=round,fill=fillColor] (175.69, 83.42) circle (  1.16);

\path[draw=drawColor,line width= 0.4pt,line join=round,line cap=round,fill=fillColor] (175.95, 83.40) circle (  1.16);

\path[draw=drawColor,line width= 0.4pt,line join=round,line cap=round,fill=fillColor] (176.20, 83.37) circle (  1.16);

\path[draw=drawColor,line width= 0.4pt,line join=round,line cap=round,fill=fillColor] (176.46, 83.37) circle (  1.16);

\path[draw=drawColor,line width= 0.4pt,line join=round,line cap=round,fill=fillColor] (176.71, 83.33) circle (  1.16);

\path[draw=drawColor,line width= 0.4pt,line join=round,line cap=round,fill=fillColor] (176.96, 83.31) circle (  1.16);

\path[draw=drawColor,line width= 0.4pt,line join=round,line cap=round,fill=fillColor] (177.21, 83.28) circle (  1.16);

\path[draw=drawColor,line width= 0.4pt,line join=round,line cap=round,fill=fillColor] (177.46, 83.26) circle (  1.16);

\path[draw=drawColor,line width= 0.4pt,line join=round,line cap=round,fill=fillColor] (177.71, 83.18) circle (  1.16);

\path[draw=drawColor,line width= 0.4pt,line join=round,line cap=round,fill=fillColor] (177.95, 83.17) circle (  1.16);

\path[draw=drawColor,line width= 0.4pt,line join=round,line cap=round,fill=fillColor] (178.20, 83.10) circle (  1.16);

\path[draw=drawColor,line width= 0.4pt,line join=round,line cap=round,fill=fillColor] (178.45, 83.07) circle (  1.16);

\path[draw=drawColor,line width= 0.4pt,line join=round,line cap=round,fill=fillColor] (178.69, 83.06) circle (  1.16);

\path[draw=drawColor,line width= 0.4pt,line join=round,line cap=round,fill=fillColor] (178.93, 82.95) circle (  1.16);

\path[draw=drawColor,line width= 0.4pt,line join=round,line cap=round,fill=fillColor] (179.18, 82.93) circle (  1.16);

\path[draw=drawColor,line width= 0.4pt,line join=round,line cap=round,fill=fillColor] (179.42, 82.92) circle (  1.16);

\path[draw=drawColor,line width= 0.4pt,line join=round,line cap=round,fill=fillColor] (179.66, 82.83) circle (  1.16);

\path[draw=drawColor,line width= 0.4pt,line join=round,line cap=round,fill=fillColor] (179.90, 82.66) circle (  1.16);

\path[draw=drawColor,line width= 0.4pt,line join=round,line cap=round,fill=fillColor] (180.14, 82.62) circle (  1.16);

\path[draw=drawColor,line width= 0.4pt,line join=round,line cap=round,fill=fillColor] (180.37, 82.61) circle (  1.16);

\path[draw=drawColor,line width= 0.4pt,line join=round,line cap=round,fill=fillColor] (180.61, 82.60) circle (  1.16);

\path[draw=drawColor,line width= 0.4pt,line join=round,line cap=round,fill=fillColor] (180.85, 82.46) circle (  1.16);

\path[draw=drawColor,line width= 0.4pt,line join=round,line cap=round,fill=fillColor] (181.08, 82.35) circle (  1.16);

\path[draw=drawColor,line width= 0.4pt,line join=round,line cap=round,fill=fillColor] (181.32, 82.34) circle (  1.16);

\path[draw=drawColor,line width= 0.4pt,line join=round,line cap=round,fill=fillColor] (181.55, 82.32) circle (  1.16);

\path[draw=drawColor,line width= 0.4pt,line join=round,line cap=round,fill=fillColor] (181.78, 82.27) circle (  1.16);

\path[draw=drawColor,line width= 0.4pt,line join=round,line cap=round,fill=fillColor] (182.01, 82.22) circle (  1.16);

\path[draw=drawColor,line width= 0.4pt,line join=round,line cap=round,fill=fillColor] (182.25, 82.19) circle (  1.16);

\path[draw=drawColor,line width= 0.4pt,line join=round,line cap=round,fill=fillColor] (182.48, 82.12) circle (  1.16);

\path[draw=drawColor,line width= 0.4pt,line join=round,line cap=round,fill=fillColor] (182.70, 82.11) circle (  1.16);

\path[draw=drawColor,line width= 0.4pt,line join=round,line cap=round,fill=fillColor] (182.93, 82.10) circle (  1.16);

\path[draw=drawColor,line width= 0.4pt,line join=round,line cap=round,fill=fillColor] (183.16, 82.09) circle (  1.16);

\path[draw=drawColor,line width= 0.4pt,line join=round,line cap=round,fill=fillColor] (183.39, 82.05) circle (  1.16);

\path[draw=drawColor,line width= 0.4pt,line join=round,line cap=round,fill=fillColor] (183.61, 82.04) circle (  1.16);

\path[draw=drawColor,line width= 0.4pt,line join=round,line cap=round,fill=fillColor] (183.84, 81.88) circle (  1.16);

\path[draw=drawColor,line width= 0.4pt,line join=round,line cap=round,fill=fillColor] (184.06, 81.74) circle (  1.16);

\path[draw=drawColor,line width= 0.4pt,line join=round,line cap=round,fill=fillColor] (184.29, 81.74) circle (  1.16);

\path[draw=drawColor,line width= 0.4pt,line join=round,line cap=round,fill=fillColor] (184.51, 81.70) circle (  1.16);

\path[draw=drawColor,line width= 0.4pt,line join=round,line cap=round,fill=fillColor] (184.73, 81.66) circle (  1.16);

\path[draw=drawColor,line width= 0.4pt,line join=round,line cap=round,fill=fillColor] (184.96, 81.64) circle (  1.16);

\path[draw=drawColor,line width= 0.4pt,line join=round,line cap=round,fill=fillColor] (185.18, 81.57) circle (  1.16);

\path[draw=drawColor,line width= 0.4pt,line join=round,line cap=round,fill=fillColor] (185.40, 81.55) circle (  1.16);

\path[draw=drawColor,line width= 0.4pt,line join=round,line cap=round,fill=fillColor] (185.62, 81.48) circle (  1.16);

\path[draw=drawColor,line width= 0.4pt,line join=round,line cap=round,fill=fillColor] (185.84, 81.46) circle (  1.16);

\path[draw=drawColor,line width= 0.4pt,line join=round,line cap=round,fill=fillColor] (186.05, 81.46) circle (  1.16);

\path[draw=drawColor,line width= 0.4pt,line join=round,line cap=round,fill=fillColor] (186.27, 81.44) circle (  1.16);

\path[draw=drawColor,line width= 0.4pt,line join=round,line cap=round,fill=fillColor] (186.49, 81.43) circle (  1.16);

\path[draw=drawColor,line width= 0.4pt,line join=round,line cap=round,fill=fillColor] (186.70, 81.31) circle (  1.16);

\path[draw=drawColor,line width= 0.4pt,line join=round,line cap=round,fill=fillColor] (186.92, 81.29) circle (  1.16);

\path[draw=drawColor,line width= 0.4pt,line join=round,line cap=round,fill=fillColor] (187.13, 81.25) circle (  1.16);

\path[draw=drawColor,line width= 0.4pt,line join=round,line cap=round,fill=fillColor] (187.35, 81.21) circle (  1.16);

\path[draw=drawColor,line width= 0.4pt,line join=round,line cap=round,fill=fillColor] (187.56, 81.20) circle (  1.16);

\path[draw=drawColor,line width= 0.4pt,line join=round,line cap=round,fill=fillColor] (187.77, 81.17) circle (  1.16);

\path[draw=drawColor,line width= 0.4pt,line join=round,line cap=round,fill=fillColor] (187.98, 81.10) circle (  1.16);

\path[draw=drawColor,line width= 0.4pt,line join=round,line cap=round,fill=fillColor] (188.20, 81.09) circle (  1.16);

\path[draw=drawColor,line width= 0.4pt,line join=round,line cap=round,fill=fillColor] (188.41, 81.08) circle (  1.16);

\path[draw=drawColor,line width= 0.4pt,line join=round,line cap=round,fill=fillColor] (188.62, 81.08) circle (  1.16);

\path[draw=drawColor,line width= 0.4pt,line join=round,line cap=round,fill=fillColor] (188.83, 80.98) circle (  1.16);

\path[draw=drawColor,line width= 0.4pt,line join=round,line cap=round,fill=fillColor] (189.03, 80.79) circle (  1.16);

\path[draw=drawColor,line width= 0.4pt,line join=round,line cap=round,fill=fillColor] (189.24, 80.78) circle (  1.16);

\path[draw=drawColor,line width= 0.4pt,line join=round,line cap=round,fill=fillColor] (189.45, 80.71) circle (  1.16);

\path[draw=drawColor,line width= 0.4pt,line join=round,line cap=round,fill=fillColor] (189.66, 80.64) circle (  1.16);

\path[draw=drawColor,line width= 0.4pt,line join=round,line cap=round,fill=fillColor] (189.86, 80.58) circle (  1.16);

\path[draw=drawColor,line width= 0.4pt,line join=round,line cap=round,fill=fillColor] (190.07, 80.57) circle (  1.16);

\path[draw=drawColor,line width= 0.4pt,line join=round,line cap=round,fill=fillColor] (190.27, 80.53) circle (  1.16);

\path[draw=drawColor,line width= 0.4pt,line join=round,line cap=round,fill=fillColor] (190.48, 80.52) circle (  1.16);

\path[draw=drawColor,line width= 0.4pt,line join=round,line cap=round,fill=fillColor] (190.68, 80.48) circle (  1.16);

\path[draw=drawColor,line width= 0.4pt,line join=round,line cap=round,fill=fillColor] (190.88, 80.48) circle (  1.16);

\path[draw=drawColor,line width= 0.4pt,line join=round,line cap=round,fill=fillColor] (191.09, 80.43) circle (  1.16);

\path[draw=drawColor,line width= 0.4pt,line join=round,line cap=round,fill=fillColor] (191.29, 80.40) circle (  1.16);

\path[draw=drawColor,line width= 0.4pt,line join=round,line cap=round,fill=fillColor] (191.49, 80.38) circle (  1.16);

\path[draw=drawColor,line width= 0.4pt,line join=round,line cap=round,fill=fillColor] (191.69, 80.37) circle (  1.16);

\path[draw=drawColor,line width= 0.4pt,line join=round,line cap=round,fill=fillColor] (191.89, 80.31) circle (  1.16);

\path[draw=drawColor,line width= 0.4pt,line join=round,line cap=round,fill=fillColor] (192.09, 80.17) circle (  1.16);

\path[draw=drawColor,line width= 0.4pt,line join=round,line cap=round,fill=fillColor] (192.29, 80.07) circle (  1.16);

\path[draw=drawColor,line width= 0.4pt,line join=round,line cap=round,fill=fillColor] (192.49, 80.06) circle (  1.16);

\path[draw=drawColor,line width= 0.4pt,line join=round,line cap=round,fill=fillColor] (192.69, 80.06) circle (  1.16);

\path[draw=drawColor,line width= 0.4pt,line join=round,line cap=round,fill=fillColor] (192.88, 80.03) circle (  1.16);

\path[draw=drawColor,line width= 0.4pt,line join=round,line cap=round,fill=fillColor] (193.08, 80.03) circle (  1.16);

\path[draw=drawColor,line width= 0.4pt,line join=round,line cap=round,fill=fillColor] (193.28, 79.97) circle (  1.16);

\path[draw=drawColor,line width= 0.4pt,line join=round,line cap=round,fill=fillColor] (193.47, 79.94) circle (  1.16);

\path[draw=drawColor,line width= 0.4pt,line join=round,line cap=round,fill=fillColor] (193.67, 79.83) circle (  1.16);

\path[draw=drawColor,line width= 0.4pt,line join=round,line cap=round,fill=fillColor] (193.86, 79.69) circle (  1.16);

\path[draw=drawColor,line width= 0.4pt,line join=round,line cap=round,fill=fillColor] (194.06, 79.67) circle (  1.16);

\path[draw=drawColor,line width= 0.4pt,line join=round,line cap=round,fill=fillColor] (194.25, 79.61) circle (  1.16);

\path[draw=drawColor,line width= 0.4pt,line join=round,line cap=round,fill=fillColor] (194.44, 79.55) circle (  1.16);

\path[draw=drawColor,line width= 0.4pt,line join=round,line cap=round,fill=fillColor] (194.63, 79.55) circle (  1.16);

\path[draw=drawColor,line width= 0.4pt,line join=round,line cap=round,fill=fillColor] (194.83, 79.51) circle (  1.16);

\path[draw=drawColor,line width= 0.4pt,line join=round,line cap=round,fill=fillColor] (195.02, 79.48) circle (  1.16);

\path[draw=drawColor,line width= 0.4pt,line join=round,line cap=round,fill=fillColor] (195.21, 79.47) circle (  1.16);

\path[draw=drawColor,line width= 0.4pt,line join=round,line cap=round,fill=fillColor] (195.40, 79.46) circle (  1.16);

\path[draw=drawColor,line width= 0.4pt,line join=round,line cap=round,fill=fillColor] (195.59, 79.34) circle (  1.16);

\path[draw=drawColor,line width= 0.4pt,line join=round,line cap=round,fill=fillColor] (195.78, 79.30) circle (  1.16);

\path[draw=drawColor,line width= 0.4pt,line join=round,line cap=round,fill=fillColor] (195.97, 79.29) circle (  1.16);

\path[draw=drawColor,line width= 0.4pt,line join=round,line cap=round,fill=fillColor] (196.16, 79.29) circle (  1.16);

\path[draw=drawColor,line width= 0.4pt,line join=round,line cap=round,fill=fillColor] (196.34, 79.27) circle (  1.16);

\path[draw=drawColor,line width= 0.4pt,line join=round,line cap=round,fill=fillColor] (196.53, 79.26) circle (  1.16);

\path[draw=drawColor,line width= 0.4pt,line join=round,line cap=round,fill=fillColor] (196.72, 79.25) circle (  1.16);

\path[draw=drawColor,line width= 0.4pt,line join=round,line cap=round,fill=fillColor] (196.91, 79.25) circle (  1.16);

\path[draw=drawColor,line width= 0.4pt,line join=round,line cap=round,fill=fillColor] (197.09, 79.24) circle (  1.16);

\path[draw=drawColor,line width= 0.4pt,line join=round,line cap=round,fill=fillColor] (197.28, 79.13) circle (  1.16);

\path[draw=drawColor,line width= 0.4pt,line join=round,line cap=round,fill=fillColor] (197.46, 79.12) circle (  1.16);

\path[draw=drawColor,line width= 0.4pt,line join=round,line cap=round,fill=fillColor] (197.65, 79.11) circle (  1.16);

\path[draw=drawColor,line width= 0.4pt,line join=round,line cap=round,fill=fillColor] (197.83, 79.07) circle (  1.16);

\path[draw=drawColor,line width= 0.4pt,line join=round,line cap=round,fill=fillColor] (198.01, 79.03) circle (  1.16);

\path[draw=drawColor,line width= 0.4pt,line join=round,line cap=round,fill=fillColor] (198.20, 79.03) circle (  1.16);

\path[draw=drawColor,line width= 0.4pt,line join=round,line cap=round,fill=fillColor] (198.38, 79.02) circle (  1.16);

\path[draw=drawColor,line width= 0.4pt,line join=round,line cap=round,fill=fillColor] (198.56, 79.00) circle (  1.16);

\path[draw=drawColor,line width= 0.4pt,line join=round,line cap=round,fill=fillColor] (198.74, 79.00) circle (  1.16);

\path[draw=drawColor,line width= 0.4pt,line join=round,line cap=round,fill=fillColor] (198.93, 78.97) circle (  1.16);

\path[draw=drawColor,line width= 0.4pt,line join=round,line cap=round,fill=fillColor] (199.11, 78.93) circle (  1.16);

\path[draw=drawColor,line width= 0.4pt,line join=round,line cap=round,fill=fillColor] (199.29, 78.92) circle (  1.16);

\path[draw=drawColor,line width= 0.4pt,line join=round,line cap=round,fill=fillColor] (199.47, 78.86) circle (  1.16);

\path[draw=drawColor,line width= 0.4pt,line join=round,line cap=round,fill=fillColor] (199.65, 78.84) circle (  1.16);

\path[draw=drawColor,line width= 0.4pt,line join=round,line cap=round,fill=fillColor] (199.83, 78.83) circle (  1.16);

\path[draw=drawColor,line width= 0.4pt,line join=round,line cap=round,fill=fillColor] (200.00, 78.83) circle (  1.16);

\path[draw=drawColor,line width= 0.4pt,line join=round,line cap=round,fill=fillColor] (200.18, 78.77) circle (  1.16);

\path[draw=drawColor,line width= 0.4pt,line join=round,line cap=round,fill=fillColor] (200.36, 78.77) circle (  1.16);

\path[draw=drawColor,line width= 0.4pt,line join=round,line cap=round,fill=fillColor] (200.54, 78.76) circle (  1.16);

\path[draw=drawColor,line width= 0.4pt,line join=round,line cap=round,fill=fillColor] (200.72, 78.71) circle (  1.16);

\path[draw=drawColor,line width= 0.4pt,line join=round,line cap=round,fill=fillColor] (200.89, 78.70) circle (  1.16);

\path[draw=drawColor,line width= 0.4pt,line join=round,line cap=round,fill=fillColor] (201.07, 78.60) circle (  1.16);

\path[draw=drawColor,line width= 0.4pt,line join=round,line cap=round,fill=fillColor] (201.24, 78.53) circle (  1.16);

\path[draw=drawColor,line width= 0.4pt,line join=round,line cap=round,fill=fillColor] (201.42, 78.44) circle (  1.16);

\path[draw=drawColor,line width= 0.4pt,line join=round,line cap=round,fill=fillColor] (201.59, 78.42) circle (  1.16);

\path[draw=drawColor,line width= 0.4pt,line join=round,line cap=round,fill=fillColor] (201.77, 78.37) circle (  1.16);

\path[draw=drawColor,line width= 0.4pt,line join=round,line cap=round,fill=fillColor] (201.94, 78.30) circle (  1.16);

\path[draw=drawColor,line width= 0.4pt,line join=round,line cap=round,fill=fillColor] (202.12, 78.29) circle (  1.16);

\path[draw=drawColor,line width= 0.4pt,line join=round,line cap=round,fill=fillColor] (202.29, 78.20) circle (  1.16);

\path[draw=drawColor,line width= 0.4pt,line join=round,line cap=round,fill=fillColor] (202.46, 78.13) circle (  1.16);

\path[draw=drawColor,line width= 0.4pt,line join=round,line cap=round,fill=fillColor] (202.64, 78.02) circle (  1.16);

\path[draw=drawColor,line width= 0.4pt,line join=round,line cap=round,fill=fillColor] (202.81, 77.92) circle (  1.16);

\path[draw=drawColor,line width= 0.4pt,line join=round,line cap=round,fill=fillColor] (202.98, 77.84) circle (  1.16);

\path[draw=drawColor,line width= 0.4pt,line join=round,line cap=round,fill=fillColor] (203.15, 77.80) circle (  1.16);

\path[draw=drawColor,line width= 0.4pt,line join=round,line cap=round,fill=fillColor] (203.32, 77.77) circle (  1.16);

\path[draw=drawColor,line width= 0.4pt,line join=round,line cap=round,fill=fillColor] (203.49, 77.73) circle (  1.16);

\path[draw=drawColor,line width= 0.4pt,line join=round,line cap=round,fill=fillColor] (203.66, 77.72) circle (  1.16);

\path[draw=drawColor,line width= 0.4pt,line join=round,line cap=round,fill=fillColor] (203.83, 77.70) circle (  1.16);

\path[draw=drawColor,line width= 0.4pt,line join=round,line cap=round,fill=fillColor] (204.00, 77.69) circle (  1.16);

\path[draw=drawColor,line width= 0.4pt,line join=round,line cap=round,fill=fillColor] (204.17, 77.68) circle (  1.16);

\path[draw=drawColor,line width= 0.4pt,line join=round,line cap=round,fill=fillColor] (204.34, 77.66) circle (  1.16);

\path[draw=drawColor,line width= 0.4pt,line join=round,line cap=round,fill=fillColor] (204.51, 77.60) circle (  1.16);

\path[draw=drawColor,line width= 0.4pt,line join=round,line cap=round,fill=fillColor] (204.68, 77.57) circle (  1.16);

\path[draw=drawColor,line width= 0.4pt,line join=round,line cap=round,fill=fillColor] (204.84, 77.35) circle (  1.16);

\path[draw=drawColor,line width= 0.4pt,line join=round,line cap=round,fill=fillColor] (205.01, 77.34) circle (  1.16);

\path[draw=drawColor,line width= 0.4pt,line join=round,line cap=round,fill=fillColor] (205.18, 77.32) circle (  1.16);

\path[draw=drawColor,line width= 0.4pt,line join=round,line cap=round,fill=fillColor] (205.34, 77.24) circle (  1.16);

\path[draw=drawColor,line width= 0.4pt,line join=round,line cap=round,fill=fillColor] (205.51, 77.24) circle (  1.16);

\path[draw=drawColor,line width= 0.4pt,line join=round,line cap=round,fill=fillColor] (205.68, 77.12) circle (  1.16);

\path[draw=drawColor,line width= 0.4pt,line join=round,line cap=round,fill=fillColor] (205.84, 77.10) circle (  1.16);

\path[draw=drawColor,line width= 0.4pt,line join=round,line cap=round,fill=fillColor] (206.01, 77.08) circle (  1.16);

\path[draw=drawColor,line width= 0.4pt,line join=round,line cap=round,fill=fillColor] (206.17, 77.06) circle (  1.16);

\path[draw=drawColor,line width= 0.4pt,line join=round,line cap=round,fill=fillColor] (206.34, 77.05) circle (  1.16);

\path[draw=drawColor,line width= 0.4pt,line join=round,line cap=round,fill=fillColor] (206.50, 77.04) circle (  1.16);

\path[draw=drawColor,line width= 0.4pt,line join=round,line cap=round,fill=fillColor] (206.66, 76.83) circle (  1.16);

\path[draw=drawColor,line width= 0.4pt,line join=round,line cap=round,fill=fillColor] (206.83, 76.82) circle (  1.16);

\path[draw=drawColor,line width= 0.4pt,line join=round,line cap=round,fill=fillColor] (206.99, 76.78) circle (  1.16);

\path[draw=drawColor,line width= 0.4pt,line join=round,line cap=round,fill=fillColor] (207.15, 76.69) circle (  1.16);

\path[draw=drawColor,line width= 0.4pt,line join=round,line cap=round,fill=fillColor] (207.31, 76.68) circle (  1.16);

\path[draw=drawColor,line width= 0.4pt,line join=round,line cap=round,fill=fillColor] (207.48, 76.66) circle (  1.16);

\path[draw=drawColor,line width= 0.4pt,line join=round,line cap=round,fill=fillColor] (207.64, 76.64) circle (  1.16);

\path[draw=drawColor,line width= 0.4pt,line join=round,line cap=round,fill=fillColor] (207.80, 76.53) circle (  1.16);

\path[draw=drawColor,line width= 0.4pt,line join=round,line cap=round,fill=fillColor] (207.96, 76.51) circle (  1.16);

\path[draw=drawColor,line width= 0.4pt,line join=round,line cap=round,fill=fillColor] (208.12, 76.49) circle (  1.16);

\path[draw=drawColor,line width= 0.4pt,line join=round,line cap=round,fill=fillColor] (208.28, 76.47) circle (  1.16);

\path[draw=drawColor,line width= 0.4pt,line join=round,line cap=round,fill=fillColor] (208.44, 76.47) circle (  1.16);

\path[draw=drawColor,line width= 0.4pt,line join=round,line cap=round,fill=fillColor] (208.60, 76.45) circle (  1.16);

\path[draw=drawColor,line width= 0.4pt,line join=round,line cap=round,fill=fillColor] (208.76, 76.41) circle (  1.16);

\path[draw=drawColor,line width= 0.4pt,line join=round,line cap=round,fill=fillColor] (208.92, 76.41) circle (  1.16);

\path[draw=drawColor,line width= 0.4pt,line join=round,line cap=round,fill=fillColor] (209.08, 76.35) circle (  1.16);

\path[draw=drawColor,line width= 0.4pt,line join=round,line cap=round,fill=fillColor] (209.24, 76.33) circle (  1.16);

\path[draw=drawColor,line width= 0.4pt,line join=round,line cap=round,fill=fillColor] (209.39, 76.30) circle (  1.16);

\path[draw=drawColor,line width= 0.4pt,line join=round,line cap=round,fill=fillColor] (209.55, 76.30) circle (  1.16);

\path[draw=drawColor,line width= 0.4pt,line join=round,line cap=round,fill=fillColor] (209.71, 76.19) circle (  1.16);

\path[draw=drawColor,line width= 0.4pt,line join=round,line cap=round,fill=fillColor] (209.87, 76.19) circle (  1.16);

\path[draw=drawColor,line width= 0.4pt,line join=round,line cap=round,fill=fillColor] (210.02, 76.19) circle (  1.16);

\path[draw=drawColor,line width= 0.4pt,line join=round,line cap=round,fill=fillColor] (210.18, 76.17) circle (  1.16);

\path[draw=drawColor,line width= 0.4pt,line join=round,line cap=round,fill=fillColor] (210.34, 76.12) circle (  1.16);

\path[draw=drawColor,line width= 0.4pt,line join=round,line cap=round,fill=fillColor] (210.49, 76.05) circle (  1.16);

\path[draw=drawColor,line width= 0.4pt,line join=round,line cap=round,fill=fillColor] (210.65, 75.93) circle (  1.16);

\path[draw=drawColor,line width= 0.4pt,line join=round,line cap=round,fill=fillColor] (210.80, 75.79) circle (  1.16);

\path[draw=drawColor,line width= 0.4pt,line join=round,line cap=round,fill=fillColor] (210.96, 75.72) circle (  1.16);

\path[draw=drawColor,line width= 0.4pt,line join=round,line cap=round,fill=fillColor] (211.11, 75.45) circle (  1.16);

\path[draw=drawColor,line width= 0.4pt,line join=round,line cap=round,fill=fillColor] (211.27, 75.30) circle (  1.16);

\path[draw=drawColor,line width= 0.4pt,line join=round,line cap=round,fill=fillColor] (211.42, 74.97) circle (  1.16);

\path[draw=drawColor,line width= 0.4pt,line join=round,line cap=round,fill=fillColor] (211.57, 74.95) circle (  1.16);

\path[draw=drawColor,line width= 0.4pt,line join=round,line cap=round,fill=fillColor] (211.73, 74.88) circle (  1.16);

\path[draw=drawColor,line width= 0.4pt,line join=round,line cap=round,fill=fillColor] (211.88, 74.85) circle (  1.16);

\path[draw=drawColor,line width= 0.4pt,line join=round,line cap=round,fill=fillColor] (212.03, 74.77) circle (  1.16);

\path[draw=drawColor,line width= 0.4pt,line join=round,line cap=round,fill=fillColor] (212.19, 74.77) circle (  1.16);

\path[draw=drawColor,line width= 0.4pt,line join=round,line cap=round,fill=fillColor] (212.34, 74.71) circle (  1.16);

\path[draw=drawColor,line width= 0.4pt,line join=round,line cap=round,fill=fillColor] (212.49, 74.70) circle (  1.16);

\path[draw=drawColor,line width= 0.4pt,line join=round,line cap=round,fill=fillColor] (212.64, 74.65) circle (  1.16);

\path[draw=drawColor,line width= 0.4pt,line join=round,line cap=round,fill=fillColor] (212.79, 74.58) circle (  1.16);

\path[draw=drawColor,line width= 0.4pt,line join=round,line cap=round,fill=fillColor] (212.94, 74.57) circle (  1.16);

\path[draw=drawColor,line width= 0.4pt,line join=round,line cap=round,fill=fillColor] (213.09, 74.54) circle (  1.16);

\path[draw=drawColor,line width= 0.4pt,line join=round,line cap=round,fill=fillColor] (213.25, 74.53) circle (  1.16);

\path[draw=drawColor,line width= 0.4pt,line join=round,line cap=round,fill=fillColor] (213.40, 74.53) circle (  1.16);

\path[draw=drawColor,line width= 0.4pt,line join=round,line cap=round,fill=fillColor] (213.55, 74.46) circle (  1.16);

\path[draw=drawColor,line width= 0.4pt,line join=round,line cap=round,fill=fillColor] (213.70, 74.45) circle (  1.16);

\path[draw=drawColor,line width= 0.4pt,line join=round,line cap=round,fill=fillColor] (213.84, 74.42) circle (  1.16);

\path[draw=drawColor,line width= 0.4pt,line join=round,line cap=round,fill=fillColor] (213.99, 74.42) circle (  1.16);

\path[draw=drawColor,line width= 0.4pt,line join=round,line cap=round,fill=fillColor] (214.14, 74.37) circle (  1.16);

\path[draw=drawColor,line width= 0.4pt,line join=round,line cap=round,fill=fillColor] (214.29, 74.34) circle (  1.16);

\path[draw=drawColor,line width= 0.4pt,line join=round,line cap=round,fill=fillColor] (214.44, 74.30) circle (  1.16);

\path[draw=drawColor,line width= 0.4pt,line join=round,line cap=round,fill=fillColor] (214.59, 74.24) circle (  1.16);

\path[draw=drawColor,line width= 0.4pt,line join=round,line cap=round,fill=fillColor] (214.74, 74.19) circle (  1.16);

\path[draw=drawColor,line width= 0.4pt,line join=round,line cap=round,fill=fillColor] (214.88, 74.13) circle (  1.16);

\path[draw=drawColor,line width= 0.4pt,line join=round,line cap=round,fill=fillColor] (215.03, 74.00) circle (  1.16);

\path[draw=drawColor,line width= 0.4pt,line join=round,line cap=round,fill=fillColor] (215.18, 73.83) circle (  1.16);

\path[draw=drawColor,line width= 0.4pt,line join=round,line cap=round,fill=fillColor] (215.32, 73.82) circle (  1.16);

\path[draw=drawColor,line width= 0.4pt,line join=round,line cap=round,fill=fillColor] (215.47, 73.80) circle (  1.16);

\path[draw=drawColor,line width= 0.4pt,line join=round,line cap=round,fill=fillColor] (215.62, 73.76) circle (  1.16);

\path[draw=drawColor,line width= 0.4pt,line join=round,line cap=round,fill=fillColor] (215.76, 73.56) circle (  1.16);

\path[draw=drawColor,line width= 0.4pt,line join=round,line cap=round,fill=fillColor] (215.91, 73.43) circle (  1.16);

\path[draw=drawColor,line width= 0.4pt,line join=round,line cap=round,fill=fillColor] (216.05, 73.40) circle (  1.16);

\path[draw=drawColor,line width= 0.4pt,line join=round,line cap=round,fill=fillColor] (216.20, 73.35) circle (  1.16);

\path[draw=drawColor,line width= 0.4pt,line join=round,line cap=round,fill=fillColor] (216.34, 73.19) circle (  1.16);

\path[draw=drawColor,line width= 0.4pt,line join=round,line cap=round,fill=fillColor] (216.49, 73.15) circle (  1.16);

\path[draw=drawColor,line width= 0.4pt,line join=round,line cap=round,fill=fillColor] (216.63, 73.04) circle (  1.16);

\path[draw=drawColor,line width= 0.4pt,line join=round,line cap=round,fill=fillColor] (216.78, 73.03) circle (  1.16);

\path[draw=drawColor,line width= 0.4pt,line join=round,line cap=round,fill=fillColor] (216.92, 72.89) circle (  1.16);

\path[draw=drawColor,line width= 0.4pt,line join=round,line cap=round,fill=fillColor] (217.06, 72.57) circle (  1.16);

\path[draw=drawColor,line width= 0.4pt,line join=round,line cap=round,fill=fillColor] (217.21, 72.51) circle (  1.16);

\path[draw=drawColor,line width= 0.4pt,line join=round,line cap=round,fill=fillColor] (217.35, 72.46) circle (  1.16);

\path[draw=drawColor,line width= 0.4pt,line join=round,line cap=round,fill=fillColor] (217.49, 72.38) circle (  1.16);

\path[draw=drawColor,line width= 0.4pt,line join=round,line cap=round,fill=fillColor] (217.64, 72.05) circle (  1.16);

\path[draw=drawColor,line width= 0.4pt,line join=round,line cap=round,fill=fillColor] (217.78, 71.96) circle (  1.16);

\path[draw=drawColor,line width= 0.4pt,line join=round,line cap=round,fill=fillColor] (217.92, 71.79) circle (  1.16);

\path[draw=drawColor,line width= 0.4pt,line join=round,line cap=round,fill=fillColor] (218.06, 71.77) circle (  1.16);

\path[draw=drawColor,line width= 0.4pt,line join=round,line cap=round,fill=fillColor] (218.20, 71.39) circle (  1.16);

\path[draw=drawColor,line width= 0.4pt,line join=round,line cap=round,fill=fillColor] (218.35, 71.24) circle (  1.16);

\path[draw=drawColor,line width= 0.4pt,line join=round,line cap=round,fill=fillColor] (218.49, 70.99) circle (  1.16);

\path[draw=drawColor,line width= 0.4pt,line join=round,line cap=round,fill=fillColor] (218.63, 70.95) circle (  1.16);

\path[draw=drawColor,line width= 0.4pt,line join=round,line cap=round,fill=fillColor] (218.77, 70.92) circle (  1.16);

\path[draw=drawColor,line width= 0.4pt,line join=round,line cap=round,fill=fillColor] (218.91, 70.86) circle (  1.16);

\path[draw=drawColor,line width= 0.4pt,line join=round,line cap=round,fill=fillColor] (219.05, 70.76) circle (  1.16);

\path[draw=drawColor,line width= 0.4pt,line join=round,line cap=round,fill=fillColor] (219.19, 70.48) circle (  1.16);

\path[draw=drawColor,line width= 0.4pt,line join=round,line cap=round,fill=fillColor] (219.33, 70.41) circle (  1.16);

\path[draw=drawColor,line width= 0.4pt,line join=round,line cap=round,fill=fillColor] (219.47, 70.29) circle (  1.16);

\path[draw=drawColor,line width= 0.4pt,line join=round,line cap=round,fill=fillColor] (219.61, 70.21) circle (  1.16);

\path[draw=drawColor,line width= 0.4pt,line join=round,line cap=round,fill=fillColor] (219.75, 70.06) circle (  1.16);

\path[draw=drawColor,line width= 0.4pt,line join=round,line cap=round,fill=fillColor] (219.89, 69.84) circle (  1.16);

\path[draw=drawColor,line width= 0.4pt,line join=round,line cap=round,fill=fillColor] (220.02, 69.81) circle (  1.16);

\path[draw=drawColor,line width= 0.4pt,line join=round,line cap=round,fill=fillColor] (220.16, 69.67) circle (  1.16);

\path[draw=drawColor,line width= 0.4pt,line join=round,line cap=round,fill=fillColor] (220.30, 69.66) circle (  1.16);

\path[draw=drawColor,line width= 0.4pt,line join=round,line cap=round,fill=fillColor] (220.44, 69.61) circle (  1.16);

\path[draw=drawColor,line width= 0.4pt,line join=round,line cap=round,fill=fillColor] (220.58, 69.24) circle (  1.16);

\path[draw=drawColor,line width= 0.4pt,line join=round,line cap=round,fill=fillColor] (220.71, 69.00) circle (  1.16);

\path[draw=drawColor,line width= 0.4pt,line join=round,line cap=round,fill=fillColor] (220.85, 68.89) circle (  1.16);

\path[draw=drawColor,line width= 0.4pt,line join=round,line cap=round,fill=fillColor] (220.99, 68.63) circle (  1.16);

\path[draw=drawColor,line width= 0.4pt,line join=round,line cap=round,fill=fillColor] (221.13, 68.41) circle (  1.16);

\path[draw=drawColor,line width= 0.4pt,line join=round,line cap=round,fill=fillColor] (221.26, 68.12) circle (  1.16);

\path[draw=drawColor,line width= 0.4pt,line join=round,line cap=round,fill=fillColor] (221.40, 68.10) circle (  1.16);

\path[draw=drawColor,line width= 0.4pt,line join=round,line cap=round,fill=fillColor] (221.53, 68.00) circle (  1.16);

\path[draw=drawColor,line width= 0.4pt,line join=round,line cap=round,fill=fillColor] (221.67, 67.95) circle (  1.16);

\path[draw=drawColor,line width= 0.4pt,line join=round,line cap=round,fill=fillColor] (221.81, 67.67) circle (  1.16);

\path[draw=drawColor,line width= 0.4pt,line join=round,line cap=round,fill=fillColor] (221.94, 67.28) circle (  1.16);

\path[draw=drawColor,line width= 0.4pt,line join=round,line cap=round,fill=fillColor] (222.08, 67.24) circle (  1.16);

\path[draw=drawColor,line width= 0.4pt,line join=round,line cap=round,fill=fillColor] (222.21, 66.99) circle (  1.16);

\path[draw=drawColor,line width= 0.4pt,line join=round,line cap=round,fill=fillColor] (222.35, 66.47) circle (  1.16);

\path[draw=drawColor,line width= 0.4pt,line join=round,line cap=round,fill=fillColor] (222.48, 66.46) circle (  1.16);

\path[draw=drawColor,line width= 0.4pt,line join=round,line cap=round,fill=fillColor] (222.62, 66.30) circle (  1.16);

\path[draw=drawColor,line width= 0.4pt,line join=round,line cap=round,fill=fillColor] (222.75, 66.11) circle (  1.16);

\path[draw=drawColor,line width= 0.4pt,line join=round,line cap=round,fill=fillColor] (222.88, 65.76) circle (  1.16);

\path[draw=drawColor,line width= 0.4pt,line join=round,line cap=round,fill=fillColor] (223.02, 64.91) circle (  1.16);

\path[draw=drawColor,line width= 0.4pt,line join=round,line cap=round,fill=fillColor] (223.15, 64.56) circle (  1.16);

\path[draw=drawColor,line width= 0.4pt,line join=round,line cap=round,fill=fillColor] (223.28, 64.14) circle (  1.16);

\path[draw=drawColor,line width= 0.4pt,line join=round,line cap=round,fill=fillColor] (223.42, 63.12) circle (  1.16);

\path[draw=drawColor,line width= 0.4pt,line join=round,line cap=round,fill=fillColor] (223.55, 62.78) circle (  1.16);

\path[draw=drawColor,line width= 0.4pt,line join=round,line cap=round,fill=fillColor] (223.68, 61.07) circle (  1.16);

\path[draw=drawColor,line width= 0.4pt,line join=round,line cap=round,fill=fillColor] (223.82, 58.98) circle (  1.16);

\path[draw=drawColor,line width= 0.4pt,line join=round,line cap=round,fill=fillColor] (223.95, 57.46) circle (  1.16);

\path[draw=drawColor,line width= 0.4pt,line join=round,line cap=round,fill=fillColor] (224.08, 47.20) circle (  1.16);

\path[draw=drawColor,line width= 0.4pt,line join=round,line cap=round,fill=fillColor] (224.21, 47.20) circle (  1.16);

\path[draw=drawColor,line width= 0.4pt,line join=round,line cap=round,fill=fillColor] (224.34, 47.20) circle (  1.16);

\path[draw=drawColor,line width= 0.4pt,line join=round,line cap=round,fill=fillColor] (224.48, 47.20) circle (  1.16);

\path[draw=drawColor,line width= 0.4pt,line join=round,line cap=round,fill=fillColor] (224.61, 47.20) circle (  1.16);

\path[draw=drawColor,line width= 0.4pt,line join=round,line cap=round,fill=fillColor] (224.74, 47.20) circle (  1.16);

\path[draw=drawColor,line width= 0.4pt,line join=round,line cap=round,fill=fillColor] (224.87, 47.20) circle (  1.16);

\path[draw=drawColor,line width= 0.4pt,line join=round,line cap=round,fill=fillColor] (225.00, 47.20) circle (  1.16);

\path[draw=drawColor,line width= 0.4pt,line join=round,line cap=round,fill=fillColor] (225.13, 47.20) circle (  1.16);

\path[draw=drawColor,line width= 0.4pt,line join=round,line cap=round,fill=fillColor] (225.26, 47.20) circle (  1.16);
\definecolor[named]{drawColor}{rgb}{1.00,0.50,0.00}
\definecolor[named]{fillColor}{rgb}{1.00,0.50,0.00}

\path[draw=drawColor,line width= 0.4pt,line join=round,line cap=round,fill=fillColor] ( 74.88,118.30) circle (  1.16);

\path[draw=drawColor,line width= 0.4pt,line join=round,line cap=round,fill=fillColor] ( 80.78,112.79) circle (  1.16);

\path[draw=drawColor,line width= 0.4pt,line join=round,line cap=round,fill=fillColor] ( 84.92,111.63) circle (  1.16);

\path[draw=drawColor,line width= 0.4pt,line join=round,line cap=round,fill=fillColor] ( 88.22,111.33) circle (  1.16);

\path[draw=drawColor,line width= 0.4pt,line join=round,line cap=round,fill=fillColor] ( 91.00,110.78) circle (  1.16);

\path[draw=drawColor,line width= 0.4pt,line join=round,line cap=round,fill=fillColor] ( 93.43,110.06) circle (  1.16);

\path[draw=drawColor,line width= 0.4pt,line join=round,line cap=round,fill=fillColor] ( 95.61,107.12) circle (  1.16);

\path[draw=drawColor,line width= 0.4pt,line join=round,line cap=round,fill=fillColor] ( 97.58,107.07) circle (  1.16);

\path[draw=drawColor,line width= 0.4pt,line join=round,line cap=round,fill=fillColor] ( 99.40,106.20) circle (  1.16);

\path[draw=drawColor,line width= 0.4pt,line join=round,line cap=round,fill=fillColor] (101.09,106.00) circle (  1.16);

\path[draw=drawColor,line width= 0.4pt,line join=round,line cap=round,fill=fillColor] (102.67,105.78) circle (  1.16);

\path[draw=drawColor,line width= 0.4pt,line join=round,line cap=round,fill=fillColor] (104.16,105.76) circle (  1.16);

\path[draw=drawColor,line width= 0.4pt,line join=round,line cap=round,fill=fillColor] (105.56,105.63) circle (  1.16);

\path[draw=drawColor,line width= 0.4pt,line join=round,line cap=round,fill=fillColor] (106.90,105.61) circle (  1.16);

\path[draw=drawColor,line width= 0.4pt,line join=round,line cap=round,fill=fillColor] (108.17,105.04) circle (  1.16);

\path[draw=drawColor,line width= 0.4pt,line join=round,line cap=round,fill=fillColor] (109.39,104.92) circle (  1.16);

\path[draw=drawColor,line width= 0.4pt,line join=round,line cap=round,fill=fillColor] (110.56,104.08) circle (  1.16);

\path[draw=drawColor,line width= 0.4pt,line join=round,line cap=round,fill=fillColor] (111.68,103.86) circle (  1.16);

\path[draw=drawColor,line width= 0.4pt,line join=round,line cap=round,fill=fillColor] (112.76,103.17) circle (  1.16);

\path[draw=drawColor,line width= 0.4pt,line join=round,line cap=round,fill=fillColor] (113.81,103.00) circle (  1.16);

\path[draw=drawColor,line width= 0.4pt,line join=round,line cap=round,fill=fillColor] (114.82,102.31) circle (  1.16);

\path[draw=drawColor,line width= 0.4pt,line join=round,line cap=round,fill=fillColor] (115.80,102.25) circle (  1.16);

\path[draw=drawColor,line width= 0.4pt,line join=round,line cap=round,fill=fillColor] (116.74,102.19) circle (  1.16);

\path[draw=drawColor,line width= 0.4pt,line join=round,line cap=round,fill=fillColor] (117.67,101.74) circle (  1.16);

\path[draw=drawColor,line width= 0.4pt,line join=round,line cap=round,fill=fillColor] (118.56,101.67) circle (  1.16);

\path[draw=drawColor,line width= 0.4pt,line join=round,line cap=round,fill=fillColor] (119.44,101.13) circle (  1.16);

\path[draw=drawColor,line width= 0.4pt,line join=round,line cap=round,fill=fillColor] (120.29,101.05) circle (  1.16);

\path[draw=drawColor,line width= 0.4pt,line join=round,line cap=round,fill=fillColor] (121.12,100.75) circle (  1.16);

\path[draw=drawColor,line width= 0.4pt,line join=round,line cap=round,fill=fillColor] (121.93,100.64) circle (  1.16);

\path[draw=drawColor,line width= 0.4pt,line join=round,line cap=round,fill=fillColor] (122.72,100.55) circle (  1.16);

\path[draw=drawColor,line width= 0.4pt,line join=round,line cap=round,fill=fillColor] (123.50,100.46) circle (  1.16);

\path[draw=drawColor,line width= 0.4pt,line join=round,line cap=round,fill=fillColor] (124.26,100.45) circle (  1.16);

\path[draw=drawColor,line width= 0.4pt,line join=round,line cap=round,fill=fillColor] (125.00,100.32) circle (  1.16);

\path[draw=drawColor,line width= 0.4pt,line join=round,line cap=round,fill=fillColor] (125.73,100.30) circle (  1.16);

\path[draw=drawColor,line width= 0.4pt,line join=round,line cap=round,fill=fillColor] (126.44,100.20) circle (  1.16);

\path[draw=drawColor,line width= 0.4pt,line join=round,line cap=round,fill=fillColor] (127.15,100.15) circle (  1.16);

\path[draw=drawColor,line width= 0.4pt,line join=round,line cap=round,fill=fillColor] (127.83,100.14) circle (  1.16);

\path[draw=drawColor,line width= 0.4pt,line join=round,line cap=round,fill=fillColor] (128.51, 99.92) circle (  1.16);

\path[draw=drawColor,line width= 0.4pt,line join=round,line cap=round,fill=fillColor] (129.17, 99.87) circle (  1.16);

\path[draw=drawColor,line width= 0.4pt,line join=round,line cap=round,fill=fillColor] (129.83, 99.45) circle (  1.16);

\path[draw=drawColor,line width= 0.4pt,line join=round,line cap=round,fill=fillColor] (130.47, 99.32) circle (  1.16);

\path[draw=drawColor,line width= 0.4pt,line join=round,line cap=round,fill=fillColor] (131.10, 99.28) circle (  1.16);

\path[draw=drawColor,line width= 0.4pt,line join=round,line cap=round,fill=fillColor] (131.72, 99.10) circle (  1.16);

\path[draw=drawColor,line width= 0.4pt,line join=round,line cap=round,fill=fillColor] (132.33, 98.20) circle (  1.16);

\path[draw=drawColor,line width= 0.4pt,line join=round,line cap=round,fill=fillColor] (132.93, 98.19) circle (  1.16);

\path[draw=drawColor,line width= 0.4pt,line join=round,line cap=round,fill=fillColor] (133.53, 98.14) circle (  1.16);

\path[draw=drawColor,line width= 0.4pt,line join=round,line cap=round,fill=fillColor] (134.11, 98.12) circle (  1.16);

\path[draw=drawColor,line width= 0.4pt,line join=round,line cap=round,fill=fillColor] (134.69, 97.95) circle (  1.16);

\path[draw=drawColor,line width= 0.4pt,line join=round,line cap=round,fill=fillColor] (135.26, 97.93) circle (  1.16);

\path[draw=drawColor,line width= 0.4pt,line join=round,line cap=round,fill=fillColor] (135.82, 97.70) circle (  1.16);

\path[draw=drawColor,line width= 0.4pt,line join=round,line cap=round,fill=fillColor] (136.38, 97.66) circle (  1.16);

\path[draw=drawColor,line width= 0.4pt,line join=round,line cap=round,fill=fillColor] (136.92, 97.56) circle (  1.16);

\path[draw=drawColor,line width= 0.4pt,line join=round,line cap=round,fill=fillColor] (137.46, 97.53) circle (  1.16);

\path[draw=drawColor,line width= 0.4pt,line join=round,line cap=round,fill=fillColor] (137.99, 97.53) circle (  1.16);

\path[draw=drawColor,line width= 0.4pt,line join=round,line cap=round,fill=fillColor] (138.52, 97.34) circle (  1.16);

\path[draw=drawColor,line width= 0.4pt,line join=round,line cap=round,fill=fillColor] (139.04, 97.17) circle (  1.16);

\path[draw=drawColor,line width= 0.4pt,line join=round,line cap=round,fill=fillColor] (139.56, 96.42) circle (  1.16);

\path[draw=drawColor,line width= 0.4pt,line join=round,line cap=round,fill=fillColor] (140.06, 96.36) circle (  1.16);

\path[draw=drawColor,line width= 0.4pt,line join=round,line cap=round,fill=fillColor] (140.57, 95.69) circle (  1.16);

\path[draw=drawColor,line width= 0.4pt,line join=round,line cap=round,fill=fillColor] (141.06, 95.51) circle (  1.16);

\path[draw=drawColor,line width= 0.4pt,line join=round,line cap=round,fill=fillColor] (141.55, 95.46) circle (  1.16);

\path[draw=drawColor,line width= 0.4pt,line join=round,line cap=round,fill=fillColor] (142.04, 95.45) circle (  1.16);

\path[draw=drawColor,line width= 0.4pt,line join=round,line cap=round,fill=fillColor] (142.52, 95.31) circle (  1.16);

\path[draw=drawColor,line width= 0.4pt,line join=round,line cap=round,fill=fillColor] (143.00, 95.24) circle (  1.16);

\path[draw=drawColor,line width= 0.4pt,line join=round,line cap=round,fill=fillColor] (143.47, 95.24) circle (  1.16);

\path[draw=drawColor,line width= 0.4pt,line join=round,line cap=round,fill=fillColor] (143.93, 95.02) circle (  1.16);

\path[draw=drawColor,line width= 0.4pt,line join=round,line cap=round,fill=fillColor] (144.39, 94.28) circle (  1.16);

\path[draw=drawColor,line width= 0.4pt,line join=round,line cap=round,fill=fillColor] (144.85, 94.13) circle (  1.16);

\path[draw=drawColor,line width= 0.4pt,line join=round,line cap=round,fill=fillColor] (145.30, 93.95) circle (  1.16);

\path[draw=drawColor,line width= 0.4pt,line join=round,line cap=round,fill=fillColor] (145.75, 93.94) circle (  1.16);

\path[draw=drawColor,line width= 0.4pt,line join=round,line cap=round,fill=fillColor] (146.19, 93.88) circle (  1.16);

\path[draw=drawColor,line width= 0.4pt,line join=round,line cap=round,fill=fillColor] (146.63, 93.87) circle (  1.16);

\path[draw=drawColor,line width= 0.4pt,line join=round,line cap=round,fill=fillColor] (147.07, 93.79) circle (  1.16);

\path[draw=drawColor,line width= 0.4pt,line join=round,line cap=round,fill=fillColor] (147.50, 93.78) circle (  1.16);

\path[draw=drawColor,line width= 0.4pt,line join=round,line cap=round,fill=fillColor] (147.93, 93.57) circle (  1.16);

\path[draw=drawColor,line width= 0.4pt,line join=round,line cap=round,fill=fillColor] (148.35, 93.55) circle (  1.16);

\path[draw=drawColor,line width= 0.4pt,line join=round,line cap=round,fill=fillColor] (148.77, 93.01) circle (  1.16);

\path[draw=drawColor,line width= 0.4pt,line join=round,line cap=round,fill=fillColor] (149.19, 92.98) circle (  1.16);

\path[draw=drawColor,line width= 0.4pt,line join=round,line cap=round,fill=fillColor] (149.60, 92.91) circle (  1.16);

\path[draw=drawColor,line width= 0.4pt,line join=round,line cap=round,fill=fillColor] (150.01, 92.87) circle (  1.16);

\path[draw=drawColor,line width= 0.4pt,line join=round,line cap=round,fill=fillColor] (150.41, 92.71) circle (  1.16);

\path[draw=drawColor,line width= 0.4pt,line join=round,line cap=round,fill=fillColor] (150.82, 92.58) circle (  1.16);

\path[draw=drawColor,line width= 0.4pt,line join=round,line cap=round,fill=fillColor] (151.22, 92.48) circle (  1.16);

\path[draw=drawColor,line width= 0.4pt,line join=round,line cap=round,fill=fillColor] (151.61, 92.33) circle (  1.16);

\path[draw=drawColor,line width= 0.4pt,line join=round,line cap=round,fill=fillColor] (152.01, 92.31) circle (  1.16);

\path[draw=drawColor,line width= 0.4pt,line join=round,line cap=round,fill=fillColor] (152.40, 92.11) circle (  1.16);

\path[draw=drawColor,line width= 0.4pt,line join=round,line cap=round,fill=fillColor] (152.78, 92.05) circle (  1.16);

\path[draw=drawColor,line width= 0.4pt,line join=round,line cap=round,fill=fillColor] (153.17, 91.90) circle (  1.16);

\path[draw=drawColor,line width= 0.4pt,line join=round,line cap=round,fill=fillColor] (153.55, 91.72) circle (  1.16);

\path[draw=drawColor,line width= 0.4pt,line join=round,line cap=round,fill=fillColor] (153.93, 91.71) circle (  1.16);

\path[draw=drawColor,line width= 0.4pt,line join=round,line cap=round,fill=fillColor] (154.30, 91.71) circle (  1.16);

\path[draw=drawColor,line width= 0.4pt,line join=round,line cap=round,fill=fillColor] (154.67, 91.69) circle (  1.16);

\path[draw=drawColor,line width= 0.4pt,line join=round,line cap=round,fill=fillColor] (155.04, 91.41) circle (  1.16);

\path[draw=drawColor,line width= 0.4pt,line join=round,line cap=round,fill=fillColor] (155.41, 91.41) circle (  1.16);

\path[draw=drawColor,line width= 0.4pt,line join=round,line cap=round,fill=fillColor] (155.78, 91.40) circle (  1.16);

\path[draw=drawColor,line width= 0.4pt,line join=round,line cap=round,fill=fillColor] (156.14, 91.35) circle (  1.16);

\path[draw=drawColor,line width= 0.4pt,line join=round,line cap=round,fill=fillColor] (156.50, 91.24) circle (  1.16);

\path[draw=drawColor,line width= 0.4pt,line join=round,line cap=round,fill=fillColor] (156.86, 91.21) circle (  1.16);

\path[draw=drawColor,line width= 0.4pt,line join=round,line cap=round,fill=fillColor] (157.21, 91.20) circle (  1.16);

\path[draw=drawColor,line width= 0.4pt,line join=round,line cap=round,fill=fillColor] (157.56, 91.14) circle (  1.16);

\path[draw=drawColor,line width= 0.4pt,line join=round,line cap=round,fill=fillColor] (157.91, 91.10) circle (  1.16);

\path[draw=drawColor,line width= 0.4pt,line join=round,line cap=round,fill=fillColor] (158.26, 91.04) circle (  1.16);

\path[draw=drawColor,line width= 0.4pt,line join=round,line cap=round,fill=fillColor] (158.61, 90.94) circle (  1.16);

\path[draw=drawColor,line width= 0.4pt,line join=round,line cap=round,fill=fillColor] (158.95, 90.93) circle (  1.16);

\path[draw=drawColor,line width= 0.4pt,line join=round,line cap=round,fill=fillColor] (159.29, 90.86) circle (  1.16);

\path[draw=drawColor,line width= 0.4pt,line join=round,line cap=round,fill=fillColor] (159.63, 90.83) circle (  1.16);

\path[draw=drawColor,line width= 0.4pt,line join=round,line cap=round,fill=fillColor] (159.97, 90.58) circle (  1.16);

\path[draw=drawColor,line width= 0.4pt,line join=round,line cap=round,fill=fillColor] (160.30, 90.50) circle (  1.16);

\path[draw=drawColor,line width= 0.4pt,line join=round,line cap=round,fill=fillColor] (160.63, 90.31) circle (  1.16);

\path[draw=drawColor,line width= 0.4pt,line join=round,line cap=round,fill=fillColor] (160.96, 90.23) circle (  1.16);

\path[draw=drawColor,line width= 0.4pt,line join=round,line cap=round,fill=fillColor] (161.29, 90.17) circle (  1.16);

\path[draw=drawColor,line width= 0.4pt,line join=round,line cap=round,fill=fillColor] (161.62, 90.11) circle (  1.16);

\path[draw=drawColor,line width= 0.4pt,line join=round,line cap=round,fill=fillColor] (161.95, 90.08) circle (  1.16);

\path[draw=drawColor,line width= 0.4pt,line join=round,line cap=round,fill=fillColor] (162.27, 89.80) circle (  1.16);

\path[draw=drawColor,line width= 0.4pt,line join=round,line cap=round,fill=fillColor] (162.59, 89.71) circle (  1.16);

\path[draw=drawColor,line width= 0.4pt,line join=round,line cap=round,fill=fillColor] (162.91, 89.68) circle (  1.16);

\path[draw=drawColor,line width= 0.4pt,line join=round,line cap=round,fill=fillColor] (163.23, 89.66) circle (  1.16);

\path[draw=drawColor,line width= 0.4pt,line join=round,line cap=round,fill=fillColor] (163.54, 89.60) circle (  1.16);

\path[draw=drawColor,line width= 0.4pt,line join=round,line cap=round,fill=fillColor] (163.85, 89.58) circle (  1.16);

\path[draw=drawColor,line width= 0.4pt,line join=round,line cap=round,fill=fillColor] (164.17, 89.55) circle (  1.16);

\path[draw=drawColor,line width= 0.4pt,line join=round,line cap=round,fill=fillColor] (164.48, 89.51) circle (  1.16);

\path[draw=drawColor,line width= 0.4pt,line join=round,line cap=round,fill=fillColor] (164.79, 89.50) circle (  1.16);

\path[draw=drawColor,line width= 0.4pt,line join=round,line cap=round,fill=fillColor] (165.09, 89.48) circle (  1.16);

\path[draw=drawColor,line width= 0.4pt,line join=round,line cap=round,fill=fillColor] (165.40, 89.42) circle (  1.16);

\path[draw=drawColor,line width= 0.4pt,line join=round,line cap=round,fill=fillColor] (165.70, 89.42) circle (  1.16);

\path[draw=drawColor,line width= 0.4pt,line join=round,line cap=round,fill=fillColor] (166.00, 89.34) circle (  1.16);

\path[draw=drawColor,line width= 0.4pt,line join=round,line cap=round,fill=fillColor] (166.30, 89.26) circle (  1.16);

\path[draw=drawColor,line width= 0.4pt,line join=round,line cap=round,fill=fillColor] (166.60, 89.25) circle (  1.16);

\path[draw=drawColor,line width= 0.4pt,line join=round,line cap=round,fill=fillColor] (166.90, 89.10) circle (  1.16);

\path[draw=drawColor,line width= 0.4pt,line join=round,line cap=round,fill=fillColor] (167.19, 89.04) circle (  1.16);

\path[draw=drawColor,line width= 0.4pt,line join=round,line cap=round,fill=fillColor] (167.49, 88.99) circle (  1.16);

\path[draw=drawColor,line width= 0.4pt,line join=round,line cap=round,fill=fillColor] (167.78, 88.73) circle (  1.16);

\path[draw=drawColor,line width= 0.4pt,line join=round,line cap=round,fill=fillColor] (168.07, 88.69) circle (  1.16);

\path[draw=drawColor,line width= 0.4pt,line join=round,line cap=round,fill=fillColor] (168.36, 88.69) circle (  1.16);

\path[draw=drawColor,line width= 0.4pt,line join=round,line cap=round,fill=fillColor] (168.65, 88.50) circle (  1.16);

\path[draw=drawColor,line width= 0.4pt,line join=round,line cap=round,fill=fillColor] (168.94, 88.50) circle (  1.16);

\path[draw=drawColor,line width= 0.4pt,line join=round,line cap=round,fill=fillColor] (169.22, 88.40) circle (  1.16);

\path[draw=drawColor,line width= 0.4pt,line join=round,line cap=round,fill=fillColor] (169.51, 88.39) circle (  1.16);

\path[draw=drawColor,line width= 0.4pt,line join=round,line cap=round,fill=fillColor] (169.79, 88.37) circle (  1.16);

\path[draw=drawColor,line width= 0.4pt,line join=round,line cap=round,fill=fillColor] (170.07, 88.36) circle (  1.16);

\path[draw=drawColor,line width= 0.4pt,line join=round,line cap=round,fill=fillColor] (170.35, 88.34) circle (  1.16);

\path[draw=drawColor,line width= 0.4pt,line join=round,line cap=round,fill=fillColor] (170.63, 88.32) circle (  1.16);

\path[draw=drawColor,line width= 0.4pt,line join=round,line cap=round,fill=fillColor] (170.91, 88.29) circle (  1.16);

\path[draw=drawColor,line width= 0.4pt,line join=round,line cap=round,fill=fillColor] (171.18, 88.22) circle (  1.16);

\path[draw=drawColor,line width= 0.4pt,line join=round,line cap=round,fill=fillColor] (171.46, 88.15) circle (  1.16);

\path[draw=drawColor,line width= 0.4pt,line join=round,line cap=round,fill=fillColor] (171.73, 88.15) circle (  1.16);

\path[draw=drawColor,line width= 0.4pt,line join=round,line cap=round,fill=fillColor] (172.00, 88.13) circle (  1.16);

\path[draw=drawColor,line width= 0.4pt,line join=round,line cap=round,fill=fillColor] (172.28, 88.08) circle (  1.16);

\path[draw=drawColor,line width= 0.4pt,line join=round,line cap=round,fill=fillColor] (172.55, 88.05) circle (  1.16);

\path[draw=drawColor,line width= 0.4pt,line join=round,line cap=round,fill=fillColor] (172.81, 87.93) circle (  1.16);

\path[draw=drawColor,line width= 0.4pt,line join=round,line cap=round,fill=fillColor] (173.08, 87.93) circle (  1.16);

\path[draw=drawColor,line width= 0.4pt,line join=round,line cap=round,fill=fillColor] (173.35, 87.91) circle (  1.16);

\path[draw=drawColor,line width= 0.4pt,line join=round,line cap=round,fill=fillColor] (173.61, 87.90) circle (  1.16);

\path[draw=drawColor,line width= 0.4pt,line join=round,line cap=round,fill=fillColor] (173.88, 87.89) circle (  1.16);

\path[draw=drawColor,line width= 0.4pt,line join=round,line cap=round,fill=fillColor] (174.14, 87.87) circle (  1.16);

\path[draw=drawColor,line width= 0.4pt,line join=round,line cap=round,fill=fillColor] (174.40, 87.80) circle (  1.16);

\path[draw=drawColor,line width= 0.4pt,line join=round,line cap=round,fill=fillColor] (174.66, 87.70) circle (  1.16);

\path[draw=drawColor,line width= 0.4pt,line join=round,line cap=round,fill=fillColor] (174.92, 87.69) circle (  1.16);

\path[draw=drawColor,line width= 0.4pt,line join=round,line cap=round,fill=fillColor] (175.18, 87.60) circle (  1.16);

\path[draw=drawColor,line width= 0.4pt,line join=round,line cap=round,fill=fillColor] (175.44, 87.55) circle (  1.16);

\path[draw=drawColor,line width= 0.4pt,line join=round,line cap=round,fill=fillColor] (175.69, 87.52) circle (  1.16);

\path[draw=drawColor,line width= 0.4pt,line join=round,line cap=round,fill=fillColor] (175.95, 87.49) circle (  1.16);

\path[draw=drawColor,line width= 0.4pt,line join=round,line cap=round,fill=fillColor] (176.20, 87.49) circle (  1.16);

\path[draw=drawColor,line width= 0.4pt,line join=round,line cap=round,fill=fillColor] (176.46, 87.44) circle (  1.16);

\path[draw=drawColor,line width= 0.4pt,line join=round,line cap=round,fill=fillColor] (176.71, 87.24) circle (  1.16);

\path[draw=drawColor,line width= 0.4pt,line join=round,line cap=round,fill=fillColor] (176.96, 87.22) circle (  1.16);

\path[draw=drawColor,line width= 0.4pt,line join=round,line cap=round,fill=fillColor] (177.21, 87.21) circle (  1.16);

\path[draw=drawColor,line width= 0.4pt,line join=round,line cap=round,fill=fillColor] (177.46, 87.20) circle (  1.16);

\path[draw=drawColor,line width= 0.4pt,line join=round,line cap=round,fill=fillColor] (177.71, 87.20) circle (  1.16);

\path[draw=drawColor,line width= 0.4pt,line join=round,line cap=round,fill=fillColor] (177.95, 87.11) circle (  1.16);

\path[draw=drawColor,line width= 0.4pt,line join=round,line cap=round,fill=fillColor] (178.20, 87.10) circle (  1.16);

\path[draw=drawColor,line width= 0.4pt,line join=round,line cap=round,fill=fillColor] (178.45, 87.09) circle (  1.16);

\path[draw=drawColor,line width= 0.4pt,line join=round,line cap=round,fill=fillColor] (178.69, 87.04) circle (  1.16);

\path[draw=drawColor,line width= 0.4pt,line join=round,line cap=round,fill=fillColor] (178.93, 87.01) circle (  1.16);

\path[draw=drawColor,line width= 0.4pt,line join=round,line cap=round,fill=fillColor] (179.18, 87.01) circle (  1.16);

\path[draw=drawColor,line width= 0.4pt,line join=round,line cap=round,fill=fillColor] (179.42, 86.99) circle (  1.16);

\path[draw=drawColor,line width= 0.4pt,line join=round,line cap=round,fill=fillColor] (179.66, 86.99) circle (  1.16);

\path[draw=drawColor,line width= 0.4pt,line join=round,line cap=round,fill=fillColor] (179.90, 86.97) circle (  1.16);

\path[draw=drawColor,line width= 0.4pt,line join=round,line cap=round,fill=fillColor] (180.14, 86.96) circle (  1.16);

\path[draw=drawColor,line width= 0.4pt,line join=round,line cap=round,fill=fillColor] (180.37, 86.92) circle (  1.16);

\path[draw=drawColor,line width= 0.4pt,line join=round,line cap=round,fill=fillColor] (180.61, 86.89) circle (  1.16);

\path[draw=drawColor,line width= 0.4pt,line join=round,line cap=round,fill=fillColor] (180.85, 86.88) circle (  1.16);

\path[draw=drawColor,line width= 0.4pt,line join=round,line cap=round,fill=fillColor] (181.08, 86.85) circle (  1.16);

\path[draw=drawColor,line width= 0.4pt,line join=round,line cap=round,fill=fillColor] (181.32, 86.71) circle (  1.16);

\path[draw=drawColor,line width= 0.4pt,line join=round,line cap=round,fill=fillColor] (181.55, 86.71) circle (  1.16);

\path[draw=drawColor,line width= 0.4pt,line join=round,line cap=round,fill=fillColor] (181.78, 86.66) circle (  1.16);

\path[draw=drawColor,line width= 0.4pt,line join=round,line cap=round,fill=fillColor] (182.01, 86.58) circle (  1.16);

\path[draw=drawColor,line width= 0.4pt,line join=round,line cap=round,fill=fillColor] (182.25, 86.54) circle (  1.16);

\path[draw=drawColor,line width= 0.4pt,line join=round,line cap=round,fill=fillColor] (182.48, 86.49) circle (  1.16);

\path[draw=drawColor,line width= 0.4pt,line join=round,line cap=round,fill=fillColor] (182.70, 86.49) circle (  1.16);

\path[draw=drawColor,line width= 0.4pt,line join=round,line cap=round,fill=fillColor] (182.93, 86.44) circle (  1.16);

\path[draw=drawColor,line width= 0.4pt,line join=round,line cap=round,fill=fillColor] (183.16, 86.42) circle (  1.16);

\path[draw=drawColor,line width= 0.4pt,line join=round,line cap=round,fill=fillColor] (183.39, 86.38) circle (  1.16);

\path[draw=drawColor,line width= 0.4pt,line join=round,line cap=round,fill=fillColor] (183.61, 86.35) circle (  1.16);

\path[draw=drawColor,line width= 0.4pt,line join=round,line cap=round,fill=fillColor] (183.84, 86.20) circle (  1.16);

\path[draw=drawColor,line width= 0.4pt,line join=round,line cap=round,fill=fillColor] (184.06, 86.20) circle (  1.16);

\path[draw=drawColor,line width= 0.4pt,line join=round,line cap=round,fill=fillColor] (184.29, 86.18) circle (  1.16);

\path[draw=drawColor,line width= 0.4pt,line join=round,line cap=round,fill=fillColor] (184.51, 86.11) circle (  1.16);

\path[draw=drawColor,line width= 0.4pt,line join=round,line cap=round,fill=fillColor] (184.73, 86.06) circle (  1.16);

\path[draw=drawColor,line width= 0.4pt,line join=round,line cap=round,fill=fillColor] (184.96, 86.06) circle (  1.16);

\path[draw=drawColor,line width= 0.4pt,line join=round,line cap=round,fill=fillColor] (185.18, 86.04) circle (  1.16);

\path[draw=drawColor,line width= 0.4pt,line join=round,line cap=round,fill=fillColor] (185.40, 86.03) circle (  1.16);

\path[draw=drawColor,line width= 0.4pt,line join=round,line cap=round,fill=fillColor] (185.62, 85.96) circle (  1.16);

\path[draw=drawColor,line width= 0.4pt,line join=round,line cap=round,fill=fillColor] (185.84, 85.95) circle (  1.16);

\path[draw=drawColor,line width= 0.4pt,line join=round,line cap=round,fill=fillColor] (186.05, 85.87) circle (  1.16);

\path[draw=drawColor,line width= 0.4pt,line join=round,line cap=round,fill=fillColor] (186.27, 85.86) circle (  1.16);

\path[draw=drawColor,line width= 0.4pt,line join=round,line cap=round,fill=fillColor] (186.49, 85.81) circle (  1.16);

\path[draw=drawColor,line width= 0.4pt,line join=round,line cap=round,fill=fillColor] (186.70, 85.79) circle (  1.16);

\path[draw=drawColor,line width= 0.4pt,line join=round,line cap=round,fill=fillColor] (186.92, 85.78) circle (  1.16);

\path[draw=drawColor,line width= 0.4pt,line join=round,line cap=round,fill=fillColor] (187.13, 85.77) circle (  1.16);

\path[draw=drawColor,line width= 0.4pt,line join=round,line cap=round,fill=fillColor] (187.35, 85.71) circle (  1.16);

\path[draw=drawColor,line width= 0.4pt,line join=round,line cap=round,fill=fillColor] (187.56, 85.71) circle (  1.16);

\path[draw=drawColor,line width= 0.4pt,line join=round,line cap=round,fill=fillColor] (187.77, 85.67) circle (  1.16);

\path[draw=drawColor,line width= 0.4pt,line join=round,line cap=round,fill=fillColor] (187.98, 85.66) circle (  1.16);

\path[draw=drawColor,line width= 0.4pt,line join=round,line cap=round,fill=fillColor] (188.20, 85.56) circle (  1.16);

\path[draw=drawColor,line width= 0.4pt,line join=round,line cap=round,fill=fillColor] (188.41, 85.53) circle (  1.16);

\path[draw=drawColor,line width= 0.4pt,line join=round,line cap=round,fill=fillColor] (188.62, 85.52) circle (  1.16);

\path[draw=drawColor,line width= 0.4pt,line join=round,line cap=round,fill=fillColor] (188.83, 85.51) circle (  1.16);

\path[draw=drawColor,line width= 0.4pt,line join=round,line cap=round,fill=fillColor] (189.03, 85.50) circle (  1.16);

\path[draw=drawColor,line width= 0.4pt,line join=round,line cap=round,fill=fillColor] (189.24, 85.50) circle (  1.16);

\path[draw=drawColor,line width= 0.4pt,line join=round,line cap=round,fill=fillColor] (189.45, 85.42) circle (  1.16);

\path[draw=drawColor,line width= 0.4pt,line join=round,line cap=round,fill=fillColor] (189.66, 85.41) circle (  1.16);

\path[draw=drawColor,line width= 0.4pt,line join=round,line cap=round,fill=fillColor] (189.86, 85.39) circle (  1.16);

\path[draw=drawColor,line width= 0.4pt,line join=round,line cap=round,fill=fillColor] (190.07, 85.37) circle (  1.16);

\path[draw=drawColor,line width= 0.4pt,line join=round,line cap=round,fill=fillColor] (190.27, 85.36) circle (  1.16);

\path[draw=drawColor,line width= 0.4pt,line join=round,line cap=round,fill=fillColor] (190.48, 85.35) circle (  1.16);

\path[draw=drawColor,line width= 0.4pt,line join=round,line cap=round,fill=fillColor] (190.68, 85.32) circle (  1.16);

\path[draw=drawColor,line width= 0.4pt,line join=round,line cap=round,fill=fillColor] (190.88, 85.23) circle (  1.16);

\path[draw=drawColor,line width= 0.4pt,line join=round,line cap=round,fill=fillColor] (191.09, 85.22) circle (  1.16);

\path[draw=drawColor,line width= 0.4pt,line join=round,line cap=round,fill=fillColor] (191.29, 85.16) circle (  1.16);

\path[draw=drawColor,line width= 0.4pt,line join=round,line cap=round,fill=fillColor] (191.49, 85.16) circle (  1.16);

\path[draw=drawColor,line width= 0.4pt,line join=round,line cap=round,fill=fillColor] (191.69, 85.15) circle (  1.16);

\path[draw=drawColor,line width= 0.4pt,line join=round,line cap=round,fill=fillColor] (191.89, 85.15) circle (  1.16);

\path[draw=drawColor,line width= 0.4pt,line join=round,line cap=round,fill=fillColor] (192.09, 85.00) circle (  1.16);

\path[draw=drawColor,line width= 0.4pt,line join=round,line cap=round,fill=fillColor] (192.29, 84.97) circle (  1.16);

\path[draw=drawColor,line width= 0.4pt,line join=round,line cap=round,fill=fillColor] (192.49, 84.92) circle (  1.16);

\path[draw=drawColor,line width= 0.4pt,line join=round,line cap=round,fill=fillColor] (192.69, 84.89) circle (  1.16);

\path[draw=drawColor,line width= 0.4pt,line join=round,line cap=round,fill=fillColor] (192.88, 84.88) circle (  1.16);

\path[draw=drawColor,line width= 0.4pt,line join=round,line cap=round,fill=fillColor] (193.08, 84.87) circle (  1.16);

\path[draw=drawColor,line width= 0.4pt,line join=round,line cap=round,fill=fillColor] (193.28, 84.84) circle (  1.16);

\path[draw=drawColor,line width= 0.4pt,line join=round,line cap=round,fill=fillColor] (193.47, 84.81) circle (  1.16);

\path[draw=drawColor,line width= 0.4pt,line join=round,line cap=round,fill=fillColor] (193.67, 84.79) circle (  1.16);

\path[draw=drawColor,line width= 0.4pt,line join=round,line cap=round,fill=fillColor] (193.86, 84.70) circle (  1.16);

\path[draw=drawColor,line width= 0.4pt,line join=round,line cap=round,fill=fillColor] (194.06, 84.65) circle (  1.16);

\path[draw=drawColor,line width= 0.4pt,line join=round,line cap=round,fill=fillColor] (194.25, 84.60) circle (  1.16);

\path[draw=drawColor,line width= 0.4pt,line join=round,line cap=round,fill=fillColor] (194.44, 84.54) circle (  1.16);

\path[draw=drawColor,line width= 0.4pt,line join=round,line cap=round,fill=fillColor] (194.63, 84.43) circle (  1.16);

\path[draw=drawColor,line width= 0.4pt,line join=round,line cap=round,fill=fillColor] (194.83, 84.42) circle (  1.16);

\path[draw=drawColor,line width= 0.4pt,line join=round,line cap=round,fill=fillColor] (195.02, 84.38) circle (  1.16);

\path[draw=drawColor,line width= 0.4pt,line join=round,line cap=round,fill=fillColor] (195.21, 84.30) circle (  1.16);

\path[draw=drawColor,line width= 0.4pt,line join=round,line cap=round,fill=fillColor] (195.40, 84.29) circle (  1.16);

\path[draw=drawColor,line width= 0.4pt,line join=round,line cap=round,fill=fillColor] (195.59, 84.24) circle (  1.16);

\path[draw=drawColor,line width= 0.4pt,line join=round,line cap=round,fill=fillColor] (195.78, 84.24) circle (  1.16);

\path[draw=drawColor,line width= 0.4pt,line join=round,line cap=round,fill=fillColor] (195.97, 84.18) circle (  1.16);

\path[draw=drawColor,line width= 0.4pt,line join=round,line cap=round,fill=fillColor] (196.16, 83.98) circle (  1.16);

\path[draw=drawColor,line width= 0.4pt,line join=round,line cap=round,fill=fillColor] (196.34, 83.89) circle (  1.16);

\path[draw=drawColor,line width= 0.4pt,line join=round,line cap=round,fill=fillColor] (196.53, 83.86) circle (  1.16);

\path[draw=drawColor,line width= 0.4pt,line join=round,line cap=round,fill=fillColor] (196.72, 83.85) circle (  1.16);

\path[draw=drawColor,line width= 0.4pt,line join=round,line cap=round,fill=fillColor] (196.91, 83.84) circle (  1.16);

\path[draw=drawColor,line width= 0.4pt,line join=round,line cap=round,fill=fillColor] (197.09, 83.81) circle (  1.16);

\path[draw=drawColor,line width= 0.4pt,line join=round,line cap=round,fill=fillColor] (197.28, 83.76) circle (  1.16);

\path[draw=drawColor,line width= 0.4pt,line join=round,line cap=round,fill=fillColor] (197.46, 83.74) circle (  1.16);

\path[draw=drawColor,line width= 0.4pt,line join=round,line cap=round,fill=fillColor] (197.65, 83.73) circle (  1.16);

\path[draw=drawColor,line width= 0.4pt,line join=round,line cap=round,fill=fillColor] (197.83, 83.70) circle (  1.16);

\path[draw=drawColor,line width= 0.4pt,line join=round,line cap=round,fill=fillColor] (198.01, 83.66) circle (  1.16);

\path[draw=drawColor,line width= 0.4pt,line join=round,line cap=round,fill=fillColor] (198.20, 83.52) circle (  1.16);

\path[draw=drawColor,line width= 0.4pt,line join=round,line cap=round,fill=fillColor] (198.38, 83.50) circle (  1.16);

\path[draw=drawColor,line width= 0.4pt,line join=round,line cap=round,fill=fillColor] (198.56, 83.45) circle (  1.16);

\path[draw=drawColor,line width= 0.4pt,line join=round,line cap=round,fill=fillColor] (198.74, 83.44) circle (  1.16);

\path[draw=drawColor,line width= 0.4pt,line join=round,line cap=round,fill=fillColor] (198.93, 83.42) circle (  1.16);

\path[draw=drawColor,line width= 0.4pt,line join=round,line cap=round,fill=fillColor] (199.11, 83.40) circle (  1.16);

\path[draw=drawColor,line width= 0.4pt,line join=round,line cap=round,fill=fillColor] (199.29, 83.39) circle (  1.16);

\path[draw=drawColor,line width= 0.4pt,line join=round,line cap=round,fill=fillColor] (199.47, 83.36) circle (  1.16);

\path[draw=drawColor,line width= 0.4pt,line join=round,line cap=round,fill=fillColor] (199.65, 83.35) circle (  1.16);

\path[draw=drawColor,line width= 0.4pt,line join=round,line cap=round,fill=fillColor] (199.83, 83.35) circle (  1.16);

\path[draw=drawColor,line width= 0.4pt,line join=round,line cap=round,fill=fillColor] (200.00, 83.34) circle (  1.16);

\path[draw=drawColor,line width= 0.4pt,line join=round,line cap=round,fill=fillColor] (200.18, 83.32) circle (  1.16);

\path[draw=drawColor,line width= 0.4pt,line join=round,line cap=round,fill=fillColor] (200.36, 83.30) circle (  1.16);

\path[draw=drawColor,line width= 0.4pt,line join=round,line cap=round,fill=fillColor] (200.54, 83.26) circle (  1.16);

\path[draw=drawColor,line width= 0.4pt,line join=round,line cap=round,fill=fillColor] (200.72, 83.24) circle (  1.16);

\path[draw=drawColor,line width= 0.4pt,line join=round,line cap=round,fill=fillColor] (200.89, 83.24) circle (  1.16);

\path[draw=drawColor,line width= 0.4pt,line join=round,line cap=round,fill=fillColor] (201.07, 83.22) circle (  1.16);

\path[draw=drawColor,line width= 0.4pt,line join=round,line cap=round,fill=fillColor] (201.24, 83.22) circle (  1.16);

\path[draw=drawColor,line width= 0.4pt,line join=round,line cap=round,fill=fillColor] (201.42, 83.19) circle (  1.16);

\path[draw=drawColor,line width= 0.4pt,line join=round,line cap=round,fill=fillColor] (201.59, 83.16) circle (  1.16);

\path[draw=drawColor,line width= 0.4pt,line join=round,line cap=round,fill=fillColor] (201.77, 83.15) circle (  1.16);

\path[draw=drawColor,line width= 0.4pt,line join=round,line cap=round,fill=fillColor] (201.94, 83.10) circle (  1.16);

\path[draw=drawColor,line width= 0.4pt,line join=round,line cap=round,fill=fillColor] (202.12, 83.09) circle (  1.16);

\path[draw=drawColor,line width= 0.4pt,line join=round,line cap=round,fill=fillColor] (202.29, 83.03) circle (  1.16);

\path[draw=drawColor,line width= 0.4pt,line join=round,line cap=round,fill=fillColor] (202.46, 82.85) circle (  1.16);

\path[draw=drawColor,line width= 0.4pt,line join=round,line cap=round,fill=fillColor] (202.64, 82.84) circle (  1.16);

\path[draw=drawColor,line width= 0.4pt,line join=round,line cap=round,fill=fillColor] (202.81, 82.83) circle (  1.16);

\path[draw=drawColor,line width= 0.4pt,line join=round,line cap=round,fill=fillColor] (202.98, 82.75) circle (  1.16);

\path[draw=drawColor,line width= 0.4pt,line join=round,line cap=round,fill=fillColor] (203.15, 82.75) circle (  1.16);

\path[draw=drawColor,line width= 0.4pt,line join=round,line cap=round,fill=fillColor] (203.32, 82.75) circle (  1.16);

\path[draw=drawColor,line width= 0.4pt,line join=round,line cap=round,fill=fillColor] (203.49, 82.73) circle (  1.16);

\path[draw=drawColor,line width= 0.4pt,line join=round,line cap=round,fill=fillColor] (203.66, 82.68) circle (  1.16);

\path[draw=drawColor,line width= 0.4pt,line join=round,line cap=round,fill=fillColor] (203.83, 82.66) circle (  1.16);

\path[draw=drawColor,line width= 0.4pt,line join=round,line cap=round,fill=fillColor] (204.00, 82.63) circle (  1.16);

\path[draw=drawColor,line width= 0.4pt,line join=round,line cap=round,fill=fillColor] (204.17, 82.61) circle (  1.16);

\path[draw=drawColor,line width= 0.4pt,line join=round,line cap=round,fill=fillColor] (204.34, 82.55) circle (  1.16);

\path[draw=drawColor,line width= 0.4pt,line join=round,line cap=round,fill=fillColor] (204.51, 82.54) circle (  1.16);

\path[draw=drawColor,line width= 0.4pt,line join=round,line cap=round,fill=fillColor] (204.68, 82.53) circle (  1.16);

\path[draw=drawColor,line width= 0.4pt,line join=round,line cap=round,fill=fillColor] (204.84, 82.53) circle (  1.16);

\path[draw=drawColor,line width= 0.4pt,line join=round,line cap=round,fill=fillColor] (205.01, 82.42) circle (  1.16);

\path[draw=drawColor,line width= 0.4pt,line join=round,line cap=round,fill=fillColor] (205.18, 82.36) circle (  1.16);

\path[draw=drawColor,line width= 0.4pt,line join=round,line cap=round,fill=fillColor] (205.34, 82.36) circle (  1.16);

\path[draw=drawColor,line width= 0.4pt,line join=round,line cap=round,fill=fillColor] (205.51, 82.32) circle (  1.16);

\path[draw=drawColor,line width= 0.4pt,line join=round,line cap=round,fill=fillColor] (205.68, 82.30) circle (  1.16);

\path[draw=drawColor,line width= 0.4pt,line join=round,line cap=round,fill=fillColor] (205.84, 82.23) circle (  1.16);

\path[draw=drawColor,line width= 0.4pt,line join=round,line cap=round,fill=fillColor] (206.01, 82.22) circle (  1.16);

\path[draw=drawColor,line width= 0.4pt,line join=round,line cap=round,fill=fillColor] (206.17, 82.05) circle (  1.16);

\path[draw=drawColor,line width= 0.4pt,line join=round,line cap=round,fill=fillColor] (206.34, 81.87) circle (  1.16);

\path[draw=drawColor,line width= 0.4pt,line join=round,line cap=round,fill=fillColor] (206.50, 81.86) circle (  1.16);

\path[draw=drawColor,line width= 0.4pt,line join=round,line cap=round,fill=fillColor] (206.66, 81.84) circle (  1.16);

\path[draw=drawColor,line width= 0.4pt,line join=round,line cap=round,fill=fillColor] (206.83, 81.74) circle (  1.16);

\path[draw=drawColor,line width= 0.4pt,line join=round,line cap=round,fill=fillColor] (206.99, 81.73) circle (  1.16);

\path[draw=drawColor,line width= 0.4pt,line join=round,line cap=round,fill=fillColor] (207.15, 81.68) circle (  1.16);

\path[draw=drawColor,line width= 0.4pt,line join=round,line cap=round,fill=fillColor] (207.31, 81.65) circle (  1.16);

\path[draw=drawColor,line width= 0.4pt,line join=round,line cap=round,fill=fillColor] (207.48, 81.60) circle (  1.16);

\path[draw=drawColor,line width= 0.4pt,line join=round,line cap=round,fill=fillColor] (207.64, 81.58) circle (  1.16);

\path[draw=drawColor,line width= 0.4pt,line join=round,line cap=round,fill=fillColor] (207.80, 81.57) circle (  1.16);

\path[draw=drawColor,line width= 0.4pt,line join=round,line cap=round,fill=fillColor] (207.96, 81.54) circle (  1.16);

\path[draw=drawColor,line width= 0.4pt,line join=round,line cap=round,fill=fillColor] (208.12, 81.51) circle (  1.16);

\path[draw=drawColor,line width= 0.4pt,line join=round,line cap=round,fill=fillColor] (208.28, 81.48) circle (  1.16);

\path[draw=drawColor,line width= 0.4pt,line join=round,line cap=round,fill=fillColor] (208.44, 81.41) circle (  1.16);

\path[draw=drawColor,line width= 0.4pt,line join=round,line cap=round,fill=fillColor] (208.60, 81.40) circle (  1.16);

\path[draw=drawColor,line width= 0.4pt,line join=round,line cap=round,fill=fillColor] (208.76, 81.39) circle (  1.16);

\path[draw=drawColor,line width= 0.4pt,line join=round,line cap=round,fill=fillColor] (208.92, 81.39) circle (  1.16);

\path[draw=drawColor,line width= 0.4pt,line join=round,line cap=round,fill=fillColor] (209.08, 81.38) circle (  1.16);

\path[draw=drawColor,line width= 0.4pt,line join=round,line cap=round,fill=fillColor] (209.24, 81.36) circle (  1.16);

\path[draw=drawColor,line width= 0.4pt,line join=round,line cap=round,fill=fillColor] (209.39, 81.36) circle (  1.16);

\path[draw=drawColor,line width= 0.4pt,line join=round,line cap=round,fill=fillColor] (209.55, 81.34) circle (  1.16);

\path[draw=drawColor,line width= 0.4pt,line join=round,line cap=round,fill=fillColor] (209.71, 81.28) circle (  1.16);

\path[draw=drawColor,line width= 0.4pt,line join=round,line cap=round,fill=fillColor] (209.87, 81.25) circle (  1.16);

\path[draw=drawColor,line width= 0.4pt,line join=round,line cap=round,fill=fillColor] (210.02, 81.15) circle (  1.16);

\path[draw=drawColor,line width= 0.4pt,line join=round,line cap=round,fill=fillColor] (210.18, 81.13) circle (  1.16);

\path[draw=drawColor,line width= 0.4pt,line join=round,line cap=round,fill=fillColor] (210.34, 81.02) circle (  1.16);

\path[draw=drawColor,line width= 0.4pt,line join=round,line cap=round,fill=fillColor] (210.49, 81.02) circle (  1.16);

\path[draw=drawColor,line width= 0.4pt,line join=round,line cap=round,fill=fillColor] (210.65, 80.98) circle (  1.16);

\path[draw=drawColor,line width= 0.4pt,line join=round,line cap=round,fill=fillColor] (210.80, 80.95) circle (  1.16);

\path[draw=drawColor,line width= 0.4pt,line join=round,line cap=round,fill=fillColor] (210.96, 80.95) circle (  1.16);

\path[draw=drawColor,line width= 0.4pt,line join=round,line cap=round,fill=fillColor] (211.11, 80.95) circle (  1.16);

\path[draw=drawColor,line width= 0.4pt,line join=round,line cap=round,fill=fillColor] (211.27, 80.91) circle (  1.16);

\path[draw=drawColor,line width= 0.4pt,line join=round,line cap=round,fill=fillColor] (211.42, 80.85) circle (  1.16);

\path[draw=drawColor,line width= 0.4pt,line join=round,line cap=round,fill=fillColor] (211.57, 80.83) circle (  1.16);

\path[draw=drawColor,line width= 0.4pt,line join=round,line cap=round,fill=fillColor] (211.73, 80.79) circle (  1.16);

\path[draw=drawColor,line width= 0.4pt,line join=round,line cap=round,fill=fillColor] (211.88, 80.75) circle (  1.16);

\path[draw=drawColor,line width= 0.4pt,line join=round,line cap=round,fill=fillColor] (212.03, 80.74) circle (  1.16);

\path[draw=drawColor,line width= 0.4pt,line join=round,line cap=round,fill=fillColor] (212.19, 80.69) circle (  1.16);

\path[draw=drawColor,line width= 0.4pt,line join=round,line cap=round,fill=fillColor] (212.34, 80.67) circle (  1.16);

\path[draw=drawColor,line width= 0.4pt,line join=round,line cap=round,fill=fillColor] (212.49, 80.66) circle (  1.16);

\path[draw=drawColor,line width= 0.4pt,line join=round,line cap=round,fill=fillColor] (212.64, 80.64) circle (  1.16);

\path[draw=drawColor,line width= 0.4pt,line join=round,line cap=round,fill=fillColor] (212.79, 80.42) circle (  1.16);

\path[draw=drawColor,line width= 0.4pt,line join=round,line cap=round,fill=fillColor] (212.94, 80.40) circle (  1.16);

\path[draw=drawColor,line width= 0.4pt,line join=round,line cap=round,fill=fillColor] (213.09, 80.37) circle (  1.16);

\path[draw=drawColor,line width= 0.4pt,line join=round,line cap=round,fill=fillColor] (213.25, 80.24) circle (  1.16);

\path[draw=drawColor,line width= 0.4pt,line join=round,line cap=round,fill=fillColor] (213.40, 80.11) circle (  1.16);

\path[draw=drawColor,line width= 0.4pt,line join=round,line cap=round,fill=fillColor] (213.55, 79.99) circle (  1.16);

\path[draw=drawColor,line width= 0.4pt,line join=round,line cap=round,fill=fillColor] (213.70, 79.93) circle (  1.16);

\path[draw=drawColor,line width= 0.4pt,line join=round,line cap=round,fill=fillColor] (213.84, 79.82) circle (  1.16);

\path[draw=drawColor,line width= 0.4pt,line join=round,line cap=round,fill=fillColor] (213.99, 79.64) circle (  1.16);

\path[draw=drawColor,line width= 0.4pt,line join=round,line cap=round,fill=fillColor] (214.14, 79.61) circle (  1.16);

\path[draw=drawColor,line width= 0.4pt,line join=round,line cap=round,fill=fillColor] (214.29, 79.57) circle (  1.16);

\path[draw=drawColor,line width= 0.4pt,line join=round,line cap=round,fill=fillColor] (214.44, 79.50) circle (  1.16);

\path[draw=drawColor,line width= 0.4pt,line join=round,line cap=round,fill=fillColor] (214.59, 79.46) circle (  1.16);

\path[draw=drawColor,line width= 0.4pt,line join=round,line cap=round,fill=fillColor] (214.74, 79.43) circle (  1.16);

\path[draw=drawColor,line width= 0.4pt,line join=round,line cap=round,fill=fillColor] (214.88, 79.39) circle (  1.16);

\path[draw=drawColor,line width= 0.4pt,line join=round,line cap=round,fill=fillColor] (215.03, 79.15) circle (  1.16);

\path[draw=drawColor,line width= 0.4pt,line join=round,line cap=round,fill=fillColor] (215.18, 79.04) circle (  1.16);

\path[draw=drawColor,line width= 0.4pt,line join=round,line cap=round,fill=fillColor] (215.32, 78.90) circle (  1.16);

\path[draw=drawColor,line width= 0.4pt,line join=round,line cap=round,fill=fillColor] (215.47, 78.89) circle (  1.16);

\path[draw=drawColor,line width= 0.4pt,line join=round,line cap=round,fill=fillColor] (215.62, 78.66) circle (  1.16);

\path[draw=drawColor,line width= 0.4pt,line join=round,line cap=round,fill=fillColor] (215.76, 78.66) circle (  1.16);

\path[draw=drawColor,line width= 0.4pt,line join=round,line cap=round,fill=fillColor] (215.91, 78.56) circle (  1.16);

\path[draw=drawColor,line width= 0.4pt,line join=round,line cap=round,fill=fillColor] (216.05, 78.55) circle (  1.16);

\path[draw=drawColor,line width= 0.4pt,line join=round,line cap=round,fill=fillColor] (216.20, 78.53) circle (  1.16);

\path[draw=drawColor,line width= 0.4pt,line join=round,line cap=round,fill=fillColor] (216.34, 78.53) circle (  1.16);

\path[draw=drawColor,line width= 0.4pt,line join=round,line cap=round,fill=fillColor] (216.49, 78.37) circle (  1.16);

\path[draw=drawColor,line width= 0.4pt,line join=round,line cap=round,fill=fillColor] (216.63, 78.33) circle (  1.16);

\path[draw=drawColor,line width= 0.4pt,line join=round,line cap=round,fill=fillColor] (216.78, 78.25) circle (  1.16);

\path[draw=drawColor,line width= 0.4pt,line join=round,line cap=round,fill=fillColor] (216.92, 78.24) circle (  1.16);

\path[draw=drawColor,line width= 0.4pt,line join=round,line cap=round,fill=fillColor] (217.06, 78.15) circle (  1.16);

\path[draw=drawColor,line width= 0.4pt,line join=round,line cap=round,fill=fillColor] (217.21, 78.10) circle (  1.16);

\path[draw=drawColor,line width= 0.4pt,line join=round,line cap=round,fill=fillColor] (217.35, 78.09) circle (  1.16);

\path[draw=drawColor,line width= 0.4pt,line join=round,line cap=round,fill=fillColor] (217.49, 78.05) circle (  1.16);

\path[draw=drawColor,line width= 0.4pt,line join=round,line cap=round,fill=fillColor] (217.64, 78.04) circle (  1.16);

\path[draw=drawColor,line width= 0.4pt,line join=round,line cap=round,fill=fillColor] (217.78, 78.03) circle (  1.16);

\path[draw=drawColor,line width= 0.4pt,line join=round,line cap=round,fill=fillColor] (217.92, 78.02) circle (  1.16);

\path[draw=drawColor,line width= 0.4pt,line join=round,line cap=round,fill=fillColor] (218.06, 77.84) circle (  1.16);

\path[draw=drawColor,line width= 0.4pt,line join=round,line cap=round,fill=fillColor] (218.20, 77.81) circle (  1.16);

\path[draw=drawColor,line width= 0.4pt,line join=round,line cap=round,fill=fillColor] (218.35, 77.76) circle (  1.16);

\path[draw=drawColor,line width= 0.4pt,line join=round,line cap=round,fill=fillColor] (218.49, 77.67) circle (  1.16);

\path[draw=drawColor,line width= 0.4pt,line join=round,line cap=round,fill=fillColor] (218.63, 77.65) circle (  1.16);

\path[draw=drawColor,line width= 0.4pt,line join=round,line cap=round,fill=fillColor] (218.77, 77.57) circle (  1.16);

\path[draw=drawColor,line width= 0.4pt,line join=round,line cap=round,fill=fillColor] (218.91, 77.37) circle (  1.16);

\path[draw=drawColor,line width= 0.4pt,line join=round,line cap=round,fill=fillColor] (219.05, 77.32) circle (  1.16);

\path[draw=drawColor,line width= 0.4pt,line join=round,line cap=round,fill=fillColor] (219.19, 77.27) circle (  1.16);

\path[draw=drawColor,line width= 0.4pt,line join=round,line cap=round,fill=fillColor] (219.33, 76.98) circle (  1.16);

\path[draw=drawColor,line width= 0.4pt,line join=round,line cap=round,fill=fillColor] (219.47, 76.91) circle (  1.16);

\path[draw=drawColor,line width= 0.4pt,line join=round,line cap=round,fill=fillColor] (219.61, 76.89) circle (  1.16);

\path[draw=drawColor,line width= 0.4pt,line join=round,line cap=round,fill=fillColor] (219.75, 76.55) circle (  1.16);

\path[draw=drawColor,line width= 0.4pt,line join=round,line cap=round,fill=fillColor] (219.89, 76.42) circle (  1.16);

\path[draw=drawColor,line width= 0.4pt,line join=round,line cap=round,fill=fillColor] (220.02, 76.38) circle (  1.16);

\path[draw=drawColor,line width= 0.4pt,line join=round,line cap=round,fill=fillColor] (220.16, 75.97) circle (  1.16);

\path[draw=drawColor,line width= 0.4pt,line join=round,line cap=round,fill=fillColor] (220.30, 75.95) circle (  1.16);

\path[draw=drawColor,line width= 0.4pt,line join=round,line cap=round,fill=fillColor] (220.44, 75.75) circle (  1.16);

\path[draw=drawColor,line width= 0.4pt,line join=round,line cap=round,fill=fillColor] (220.58, 75.67) circle (  1.16);

\path[draw=drawColor,line width= 0.4pt,line join=round,line cap=round,fill=fillColor] (220.71, 75.52) circle (  1.16);

\path[draw=drawColor,line width= 0.4pt,line join=round,line cap=round,fill=fillColor] (220.85, 75.51) circle (  1.16);

\path[draw=drawColor,line width= 0.4pt,line join=round,line cap=round,fill=fillColor] (220.99, 75.42) circle (  1.16);

\path[draw=drawColor,line width= 0.4pt,line join=round,line cap=round,fill=fillColor] (221.13, 75.02) circle (  1.16);

\path[draw=drawColor,line width= 0.4pt,line join=round,line cap=round,fill=fillColor] (221.26, 75.01) circle (  1.16);

\path[draw=drawColor,line width= 0.4pt,line join=round,line cap=round,fill=fillColor] (221.40, 74.91) circle (  1.16);

\path[draw=drawColor,line width= 0.4pt,line join=round,line cap=round,fill=fillColor] (221.53, 74.83) circle (  1.16);

\path[draw=drawColor,line width= 0.4pt,line join=round,line cap=round,fill=fillColor] (221.67, 74.79) circle (  1.16);

\path[draw=drawColor,line width= 0.4pt,line join=round,line cap=round,fill=fillColor] (221.81, 74.50) circle (  1.16);

\path[draw=drawColor,line width= 0.4pt,line join=round,line cap=round,fill=fillColor] (221.94, 74.35) circle (  1.16);

\path[draw=drawColor,line width= 0.4pt,line join=round,line cap=round,fill=fillColor] (222.08, 74.14) circle (  1.16);

\path[draw=drawColor,line width= 0.4pt,line join=round,line cap=round,fill=fillColor] (222.21, 73.98) circle (  1.16);

\path[draw=drawColor,line width= 0.4pt,line join=round,line cap=round,fill=fillColor] (222.35, 73.62) circle (  1.16);

\path[draw=drawColor,line width= 0.4pt,line join=round,line cap=round,fill=fillColor] (222.48, 73.27) circle (  1.16);

\path[draw=drawColor,line width= 0.4pt,line join=round,line cap=round,fill=fillColor] (222.62, 73.21) circle (  1.16);

\path[draw=drawColor,line width= 0.4pt,line join=round,line cap=round,fill=fillColor] (222.75, 73.03) circle (  1.16);

\path[draw=drawColor,line width= 0.4pt,line join=round,line cap=round,fill=fillColor] (222.88, 72.96) circle (  1.16);

\path[draw=drawColor,line width= 0.4pt,line join=round,line cap=round,fill=fillColor] (223.02, 72.84) circle (  1.16);

\path[draw=drawColor,line width= 0.4pt,line join=round,line cap=round,fill=fillColor] (223.15, 72.71) circle (  1.16);

\path[draw=drawColor,line width= 0.4pt,line join=round,line cap=round,fill=fillColor] (223.28, 72.20) circle (  1.16);

\path[draw=drawColor,line width= 0.4pt,line join=round,line cap=round,fill=fillColor] (223.42, 71.73) circle (  1.16);

\path[draw=drawColor,line width= 0.4pt,line join=round,line cap=round,fill=fillColor] (223.55, 71.67) circle (  1.16);

\path[draw=drawColor,line width= 0.4pt,line join=round,line cap=round,fill=fillColor] (223.68, 71.18) circle (  1.16);

\path[draw=drawColor,line width= 0.4pt,line join=round,line cap=round,fill=fillColor] (223.82, 70.87) circle (  1.16);

\path[draw=drawColor,line width= 0.4pt,line join=round,line cap=round,fill=fillColor] (223.95, 70.66) circle (  1.16);

\path[draw=drawColor,line width= 0.4pt,line join=round,line cap=round,fill=fillColor] (224.08, 69.92) circle (  1.16);

\path[draw=drawColor,line width= 0.4pt,line join=round,line cap=round,fill=fillColor] (224.21, 69.56) circle (  1.16);

\path[draw=drawColor,line width= 0.4pt,line join=round,line cap=round,fill=fillColor] (224.34, 68.70) circle (  1.16);

\path[draw=drawColor,line width= 0.4pt,line join=round,line cap=round,fill=fillColor] (224.48, 65.22) circle (  1.16);

\path[draw=drawColor,line width= 0.4pt,line join=round,line cap=round,fill=fillColor] (224.61, 64.91) circle (  1.16);

\path[draw=drawColor,line width= 0.4pt,line join=round,line cap=round,fill=fillColor] (224.74, 64.71) circle (  1.16);

\path[draw=drawColor,line width= 0.4pt,line join=round,line cap=round,fill=fillColor] (224.87, 62.00) circle (  1.16);

\path[draw=drawColor,line width= 0.4pt,line join=round,line cap=round,fill=fillColor] (225.00, 55.99) circle (  1.16);

\path[draw=drawColor,line width= 0.4pt,line join=round,line cap=round,fill=fillColor] (225.13, 47.20) circle (  1.16);

\path[draw=drawColor,line width= 0.4pt,line join=round,line cap=round,fill=fillColor] (225.26, 47.20) circle (  1.16);
\definecolor[named]{drawColor}{rgb}{0.00,0.00,0.00}
\definecolor[named]{fillColor}{rgb}{0.00,0.00,0.00}

\path[draw=drawColor,line width= 0.6pt,line join=round,fill=fillColor] ( 67.36,130.17) -- (232.78,130.17);

\node[text=drawColor,anchor=base east,inner sep=0pt, outer sep=0pt, scale=  0.85] at (229.28,132.32) {infeasible solutions};

\path[draw=drawColor,line width= 0.6pt,line join=round,line cap=round] ( 67.36, 38.91) rectangle (232.78,141.14);
\end{scope}
\begin{scope}
\path[clip] (  0.00,  0.00) rectangle (505.89,650.43);
\definecolor[named]{drawColor}{rgb}{0.00,0.00,0.00}

\node[text=drawColor,anchor=base east,inner sep=0pt, outer sep=0pt, scale=  0.80] at ( 61.96, 44.45) {0.00};

\node[text=drawColor,anchor=base east,inner sep=0pt, outer sep=0pt, scale=  0.80] at ( 61.96, 62.32) {0.01};

\node[text=drawColor,anchor=base east,inner sep=0pt, outer sep=0pt, scale=  0.80] at ( 61.96, 75.01) {0.05};

\node[text=drawColor,anchor=base east,inner sep=0pt, outer sep=0pt, scale=  0.80] at ( 61.96, 82.96) {0.10};

\node[text=drawColor,anchor=base east,inner sep=0pt, outer sep=0pt, scale=  0.80] at ( 61.96, 92.97) {0.20};

\node[text=drawColor,anchor=base east,inner sep=0pt, outer sep=0pt, scale=  0.80] at ( 61.96,105.58) {0.40};

\node[text=drawColor,anchor=base east,inner sep=0pt, outer sep=0pt, scale=  0.80] at ( 61.96,114.42) {0.60};

\node[text=drawColor,anchor=base east,inner sep=0pt, outer sep=0pt, scale=  0.80] at ( 61.96,121.47) {0.80};

\node[text=drawColor,anchor=base east,inner sep=0pt, outer sep=0pt, scale=  0.80] at ( 61.96,127.41) {1.00};
\end{scope}
\begin{scope}
\path[clip] (  0.00,  0.00) rectangle (505.89,650.43);
\definecolor[named]{drawColor}{rgb}{0.00,0.00,0.00}

\path[draw=drawColor,line width= 0.6pt,line join=round] ( 64.36, 47.20) --
	( 67.36, 47.20);

\path[draw=drawColor,line width= 0.6pt,line join=round] ( 64.36, 65.08) --
	( 67.36, 65.08);

\path[draw=drawColor,line width= 0.6pt,line join=round] ( 64.36, 77.77) --
	( 67.36, 77.77);

\path[draw=drawColor,line width= 0.6pt,line join=round] ( 64.36, 85.71) --
	( 67.36, 85.71);

\path[draw=drawColor,line width= 0.6pt,line join=round] ( 64.36, 95.72) --
	( 67.36, 95.72);

\path[draw=drawColor,line width= 0.6pt,line join=round] ( 64.36,108.33) --
	( 67.36,108.33);

\path[draw=drawColor,line width= 0.6pt,line join=round] ( 64.36,117.18) --
	( 67.36,117.18);

\path[draw=drawColor,line width= 0.6pt,line join=round] ( 64.36,124.22) --
	( 67.36,124.22);

\path[draw=drawColor,line width= 0.6pt,line join=round] ( 64.36,130.17) --
	( 67.36,130.17);
\end{scope}
\begin{scope}
\path[clip] (  0.00,  0.00) rectangle (505.89,650.43);
\definecolor[named]{drawColor}{rgb}{0.00,0.00,0.00}

\path[draw=drawColor,line width= 0.6pt,line join=round] (157.56, 35.91) --
	(157.56, 38.91);

\path[draw=drawColor,line width= 0.6pt,line join=round] (184.96, 35.91) --
	(184.96, 38.91);

\path[draw=drawColor,line width= 0.6pt,line join=round] (204.17, 35.91) --
	(204.17, 38.91);

\path[draw=drawColor,line width= 0.6pt,line join=round] (219.47, 35.91) --
	(219.47, 38.91);

\path[draw=drawColor,line width= 0.6pt,line join=round] (232.39, 35.91) --
	(232.39, 38.91);
\end{scope}
\begin{scope}
\path[clip] (  0.00,  0.00) rectangle (505.89,650.43);
\definecolor[named]{drawColor}{rgb}{0.00,0.00,0.00}

\node[text=drawColor,rotate= 50.00,anchor=base east,inner sep=0pt, outer sep=0pt, scale=  0.80] at (161.78, 29.97) {100};

\node[text=drawColor,rotate= 50.00,anchor=base east,inner sep=0pt, outer sep=0pt, scale=  0.80] at (189.18, 29.97) {200};

\node[text=drawColor,rotate= 50.00,anchor=base east,inner sep=0pt, outer sep=0pt, scale=  0.80] at (208.39, 29.97) {300};

\node[text=drawColor,rotate= 50.00,anchor=base east,inner sep=0pt, outer sep=0pt, scale=  0.80] at (223.69, 29.97) {400};

\node[text=drawColor,rotate= 50.00,anchor=base east,inner sep=0pt, outer sep=0pt, scale=  0.80] at (236.61, 29.97) {500};
\end{scope}
\begin{scope}
\path[clip] (  0.00,  0.00) rectangle (505.89,650.43);
\definecolor[named]{drawColor}{rgb}{0.00,0.00,0.00}

\node[text=drawColor,anchor=base,inner sep=0pt, outer sep=0pt, scale=  1.10] at (150.07,  8.40) {\# Instances};
\end{scope}
\begin{scope}
\path[clip] (  0.00,  0.00) rectangle (505.89,650.43);
\definecolor[named]{drawColor}{rgb}{0.00,0.00,0.00}

\node[text=drawColor,rotate= 90.00,anchor=base,inner sep=0pt, outer sep=0pt, scale=  1.10] at ( 37.74, 90.03) {1-(Best/Algorithm)};
\end{scope}
\begin{scope}
\path[clip] (  0.00,  0.00) rectangle (505.89,650.43);
\definecolor[named]{drawColor}{rgb}{0.00,0.00,0.00}

\node[text=drawColor,anchor=base,inner sep=0pt, outer sep=0pt, scale=  1.20] at (150.07,148.34) {$k=128$};
\end{scope}
\begin{scope}
\path[clip] (  0.00,  0.00) rectangle (505.89,650.43);
\definecolor[named]{drawColor}{rgb}{0.00,0.00,0.00}

\node[text=drawColor,anchor=base west,inner sep=0pt, outer sep=0pt, scale=  1.00] at (345.50,110.37) {\bfseries Algorithm};
\end{scope}
\begin{scope}
\path[clip] (  0.00,  0.00) rectangle (505.89,650.43);
\definecolor[named]{drawColor}{rgb}{0.89,0.10,0.11}
\definecolor[named]{fillColor}{rgb}{0.89,0.10,0.11}

\path[draw=drawColor,line width= 0.4pt,line join=round,line cap=round,fill=fillColor] (348.51,101.64) circle (  2.50);
\end{scope}
\begin{scope}
\path[clip] (  0.00,  0.00) rectangle (505.89,650.43);
\definecolor[named]{drawColor}{rgb}{0.65,0.34,0.16}
\definecolor[named]{fillColor}{rgb}{0.65,0.34,0.16}

\path[draw=drawColor,line width= 0.4pt,line join=round,line cap=round,fill=fillColor] (348.51, 91.40) circle (  2.50);
\end{scope}
\begin{scope}
\path[clip] (  0.00,  0.00) rectangle (505.89,650.43);
\definecolor[named]{drawColor}{rgb}{0.22,0.49,0.72}
\definecolor[named]{fillColor}{rgb}{0.22,0.49,0.72}

\path[draw=drawColor,line width= 0.4pt,line join=round,line cap=round,fill=fillColor] (348.51, 81.17) circle (  2.50);
\end{scope}
\begin{scope}
\path[clip] (  0.00,  0.00) rectangle (505.89,650.43);
\definecolor[named]{drawColor}{rgb}{0.30,0.69,0.29}
\definecolor[named]{fillColor}{rgb}{0.30,0.69,0.29}

\path[draw=drawColor,line width= 0.4pt,line join=round,line cap=round,fill=fillColor] (348.51, 70.93) circle (  2.50);
\end{scope}
\begin{scope}
\path[clip] (  0.00,  0.00) rectangle (505.89,650.43);
\definecolor[named]{drawColor}{rgb}{0.60,0.31,0.64}
\definecolor[named]{fillColor}{rgb}{0.60,0.31,0.64}

\path[draw=drawColor,line width= 0.4pt,line join=round,line cap=round,fill=fillColor] (348.51, 60.69) circle (  2.50);
\end{scope}
\begin{scope}
\path[clip] (  0.00,  0.00) rectangle (505.89,650.43);
\definecolor[named]{drawColor}{rgb}{1.00,0.50,0.00}
\definecolor[named]{fillColor}{rgb}{1.00,0.50,0.00}

\path[draw=drawColor,line width= 0.4pt,line join=round,line cap=round,fill=fillColor] (348.51, 50.45) circle (  2.50);
\end{scope}
\begin{scope}
\path[clip] (  0.00,  0.00) rectangle (505.89,650.43);
\definecolor[named]{drawColor}{rgb}{0.00,0.00,0.00}

\node[text=drawColor,anchor=base west,inner sep=0pt, outer sep=0pt, scale=  1.00] at (353.33, 98.20) {\KaHyPar{CA}};
\end{scope}
\begin{scope}
\path[clip] (  0.00,  0.00) rectangle (505.89,650.43);
\definecolor[named]{drawColor}{rgb}{0.00,0.00,0.00}

\node[text=drawColor,anchor=base west,inner sep=0pt, outer sep=0pt, scale=  1.00] at (353.33, 87.96) {\KaHyPar{MF}};
\end{scope}
\begin{scope}
\path[clip] (  0.00,  0.00) rectangle (505.89,650.43);
\definecolor[named]{drawColor}{rgb}{0.00,0.00,0.00}

\node[text=drawColor,anchor=base west,inner sep=0pt, outer sep=0pt, scale=  1.00] at (353.33, 77.72) {\hMetis{R}};
\end{scope}
\begin{scope}
\path[clip] (  0.00,  0.00) rectangle (505.89,650.43);
\definecolor[named]{drawColor}{rgb}{0.00,0.00,0.00}

\node[text=drawColor,anchor=base west,inner sep=0pt, outer sep=0pt, scale=  1.00] at (353.33, 67.48) {\hMetis{K}};
\end{scope}
\begin{scope}
\path[clip] (  0.00,  0.00) rectangle (505.89,650.43);
\definecolor[named]{drawColor}{rgb}{0.00,0.00,0.00}

\node[text=drawColor,anchor=base west,inner sep=0pt, outer sep=0pt, scale=  1.00] at (353.33, 57.25) {\PaToH{Q}};
\end{scope}
\begin{scope}
\path[clip] (  0.00,  0.00) rectangle (505.89,650.43);
\definecolor[named]{drawColor}{rgb}{0.00,0.00,0.00}

\node[text=drawColor,anchor=base west,inner sep=0pt, outer sep=0pt, scale=  1.00] at (353.33, 47.01) {\PaToH{D}};
\end{scope}
\end{tikzpicture}
 %
\caption{Min-Cut performance plots comparing \KaHyPar{MF} with \KaHyPar{CA} and
         other partitioners for different values of $k$.}
\label{fig:final_flow_k}
\end{figure}
\end{appendix}

%%%%%%%%%%%%%%%%%%%%%%%%%%%%%%%%%%%%%%%%%%%%%%%%%%%%%%%%%%%%%%%%%%%%%%

\end{document}
