In this thesis we developed a novel \emph{local search} technique based on \emph{Max-Flow-Min-Cut}
computations for multilevel hypergraph partitioning. We integrated our \emph{flow}-based
refinement framework into the $n$-level hypergraph partitioner \emph{KaHyPar} and show that
in combination with the \emph{FM} heurisitc our new approach produces the best known partitions
for a wide range of applications.\\
On the road to a practical implementation we developed several concepts to speed up flow
computations on a flow network of a hypergraph (see Section \ref{sec:related_lawler}).
One is to remove low-degree hypernodes from the network and instead insert a clique 
between all incident hyperedge nodes. We show that the number of nodes and edges could
be reduced, if the degree of a hypernode is smaller or equal than $3$. Further, we model a hyperedge
of size $2$ as an undirected flow edge. We combine both techniques in a \emph{Hybrid}-Network 
and show that maximum flow algorithms are up to a factor of $3$ faster compared to
the execution on the \emph{Lawler}-Network \cite{lawler1973} on real world benchmarks. \\
Our \emph{flow}-based refinement framework is based on the ideas of Sanders and Schulz
\cite{sanders2011engineering} (developed for multilevel graph partitioning). Given an already bipartitioned
hypergraph, we show how to configure the source and sink sets of the flow network 
of a subhypergraph such that a \emph{Max-Flow-Min-Cut} computation yields to a cut smaller 
or equal than the cut before on the original hypergraph. A main contribution is that we proof
that applying the source and sink set modelling approach of Sanders and Schulz one-to-one on 
hypergraphs results in cuts greater or equal than with our optimized defintion. This is due to
the fact that \emph{border hyperdges}, which induced sources and sinks, can be
splitted into three different disjoint subsets on hypergraphs. This distinction enables a more effective
configuration of the sources and sinks. Additionally, we explain how one can find all 
minimum $(s,t)$-cutsets with one maximum $(s,t)$-flow calculation on hypergraphs. \\
We integrated our framework into the $n$-level hypergraph partitioner
\emph{KaHyPar}. A \emph{flow}-based refinement is executed in $\log{n}$ levels of the multilevel
hierarchy between each adjacent block in the quotient graph. The pairwise block scheduling 
refinement is executed in rounds and terminates if none of the blocks changes anymore.
The sizes of the flow problems are choosen adaptively. If a \emph{flow} computation on two blocks
yields to an improvement the flow problem size is increased, otherwise it is decreased. 
Additionally, we try to automatically balance the partition after \emph{Max-Flow-Min-Cut}
computation by iterating over each minimum $(S,T)$-cutset. In the remaining levels, 
where no flow is performed, the classical \emph{FM} heuristic is used to improve the quality 
of a partition. An observation during implementation was that only a minority of 
the \emph{Max-Flow-Min-Cut} computations leads to an improvement of the original partition. 
Therefore, we implement several speed up heuristics which prevents the
execution of unnecessary pairwise \emph{flow} refinements. \\
Our new quality configuration \KaHyPar{MF} produced on $95\%$ of our benchmark instances
better partitions than our old baseline configuration \KaHyPar{CA}. On average the solution 
quality is $2\%$ better and only within a factor of $2$ slower. In comparison with other 
state-of-the-art hypergraph partitioner, \KaHyPar{MF} produced on $70\%$ of the benchmark instances
the best known partitions with a running time comparable to the direct $k$-way implementation
of \emph{hMetis}.


\subsection{Future Work}

 \begin{enumerate}
\item Minimize number of edges with clique expansion ($k \rightarrow k + 1$ clique
      expansion)
\item Extensive evaluation of more maximum flow algorithms and parameter tunning
\item Proof that there exists no source and sink sets $S$ and $T$ such that the maximum
      flow $f$ is $|f| < |f'|$ with $f'$ is a maximum flow defined in Equation \ref{S_final_border_hyperedges}
      and \ref{T_final_border_hyperedges}
\item Test several flow execution policies
\item More speed up heuristics
 \end{enumerate}