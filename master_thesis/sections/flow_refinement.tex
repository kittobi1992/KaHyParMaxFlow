We will now give a detailed overview of our \emph{flow}-based refinement framework. The main
idea is to extract a subhypergraph $H_{V'}$ from an already partitioned hypergraph $H$. 
$V'$ is chosen in such way that it is a subset of two
adjacent blocks $V_i$ and $V_j$. Afterwards, we transform the subhypergraph
$H_{V'}$ into one of the flow networks proposed in Section \ref{sec:opt_flow_network}.
We will show how to configure the sources $S$ and sinks $T$ of the corresponding flow 
network such that a minimum $(S,T)$-bipartition of $H_{V'}$ can be used to improve the connectivity 
metric of $H$ (see Section \ref{sec:source_and_sink}). Further, we describe how 
the ideas of Sanders and Schulz \cite{sanders2011engineering} presented in Section \ref{sec:flow_local_search_graph}
can be adapted to work in the $n$-level hypergraph partitioner \emph{KaHyPar}. 

\subsection{Source and Sink Configuration}
\label{sec:source_and_sink}

Let $H = (V,E,c,\omega)$ be a hypergraph and $\Pi_1$ be a bipartition of $H$.
In the following, we show how to configure the source set $S$ and sink set $T$ of the flow
network $T_L(H_{V'})$ of a subhypergraph $H_{V'}$ induced by $V' \subseteq V$. The goal is 
to improve a bipartition $\Pi_1$ of $H$ with a maximum $(S,T)$-flow calculation 
on $T_L(H_{V'})$ (with $f$ as maximum flow) such that after applying the minimum 
$(S,T)$-bipartition of $H_{V'}$ on $H$ the resulting bipartition $\Pi_2$ of $H$ has a 
cut less than or equal to the cut of $\Pi_1$. However, an important concept of this Section will
be the definition of the \emph{extension} of a subhypergraph (see \autoref{img:extension}).

\begin{figure}
\centering
\includegraphics[width=1.0\textwidth]{../img/source_and_sink_set/extension.eps}
\caption{Illustration of an \emph{extension} of a subhypergraph. Each hyperedge of the
         the hypergraph $H$ which is fully or partially contained in $H_{V'}$ is fully
         contained $Ex(H_{V'})$. The nodes not contained in the rectangle of the right figure
         are part of $V''$.}
\label{img:extension}
\end{figure}

\begin{definition}[Extension of a Subhypergraph]
\label{def:sub_extension}
We define the extension $\SubExtension$ of a subhypergraph $H_{V'}$ such that each hyperedge of $H = (V,E,c,\omega)$ which is
partially contained in $H_{V'}$ is fully contained in $\SubExtension$. More formally,
$\SubExtension = (V' \cup V'', E', c, \omega)$ with $V'' = \bigcup_{\substack{e \in E \\ e \setminus V' \neq \emptyset}} e \setminus V'$
and $E' = \{e \in E\ |\ e \subseteq (V' \cup V'')\}$.
\end{definition}

In the following, we will first define our source and sink set on flow network $T_L(\SubExtension)$ such that
each hypernode $v \in V''$ is either a source or sink node. We can show, that a maximum
$(S,T)$-flow on $T_L(\SubExtension)$ then improves a given bipartition $\Pi_1$ of $H$.
Afterwards, we will remove all hypernodes $v \in V''$ of $T_L(\SubExtension)$ with
Theorem \ref{theorem:heuer_network} which will result in a source set $S'$ and sink 
set $T'$ for $T_L(H_{V'})$. Moreover, the value of a maximum $(S,T)$-flow of $T_L(\SubExtension)$
and a maximum $(S',T')$-flow of $T_L(H_{V'})$ will be the same. \\
We introduce the extension of subhypergraph $H_{V'}$ because we have to redefine the defintion
of the cut of $H_{V'}$ related to a bipartition $\Pi$ of $H$ as follows:
\begin{equation}
\label{eq:sub_cut}
\omega_{H_{V'}}(\Pi) := \omega_{\SubExtension}(\Pi)
\end{equation}
Note, that the cut $\omega_{H_{V'}}$ is defined over the cut nets of $H$ (which are equal to 
the cut nets of $\SubExtension$). A cut hyperedge $e$ of $H$ is not necessarily a cut hyperedge
of $H_{V'}$. For example, if $e = \{v_1,v_2\}$ is a hyperedge with $v_1 \in V_1$ and $v_2 \in V_2$ and
$v_1 \in V'$ and $v_2 \notin V'$, then $e$ is cut in $H$, but not in $H_{V'}$, because
$v_2$ is not contained in $H_{V'}$. The reason that we still
define $e$ as cut hyperedge of $H_{V'}$ has to do with our problem statement, 
which we will define as follows:

\begin{problem}
\label{prob:ST} 
Given a subhypergraph $H_{V'}$ ($V' \subseteq V$) and bipartition $\Pi_1$ of hypergraph $H$.
How should $S$ and $T$ be defined such that after a maximum $(S,T)$-flow calculation (with 
$f$ as maximum flow) the resulting minimum $(S,T)$-bipartition $\Pi_2$ of $H$ satisfy
the following conditions:
\begin{enumerate}
\item $\omega_H(\Pi_2) \le \omega_H(\Pi_1)$
\item $\Delta_{H} := \omega_H(\Pi_1) - \omega_H(\Pi_2) = \omega_{H_{V'}}(\Pi_1) - |f| =: \Delta_{H_{V'}}$
\end{enumerate}
\end{problem}

The first condition ensures that a \emph{Max-Flow-Min-Cut} computation on $T_L(H_{V'})$ never 
increases the cut of $H$. While the second condition allows us to update the cut metric in
constant time via $\omega_H(\Pi_2) = w_H(\Pi_1) - \Delta_{H_{V'}}$, instead of having 
to sum up the weight of all cut hyperedges. Since we have to build the subhypergraph
$H_{V'}$ before each maximum flow computation, we can implicitly calculate $\omega_{H_{V'}}(\Pi_1)$.\\
Note, that the equality
\[\Delta_{H} := \omega_H(\Pi_1) - \omega_H(\Pi_2) = \omega_{H_{V'}}(\Pi_1) - \omega_{H_{V'}}(\Pi_2) := \Delta_{H_{V'}}\]
holds only if we define the cut of subhypergraph $H_{V'}$ over the cut hyperedges of $H$
(see Equation \ref{eq:sub_cut}). For example, if a \emph{Max-Flow-Min-Cut} computation
on $H_{V'}$ removes a hyperedge $e$ from the cut of $H_{V'}$, but $e$ is still cut on
$H$, then the equality did not hold if we define the cut of $H_{V'}$ over the cut
hyperedges of $H_{V'}$. Because $\Delta_{H_{V'}}$ is then equal to $1$ and $\Delta_{H}$
is equal to $0$. Further, if we can show that $|f| = \omega_{H_{V'}}(\Pi_2)$, we simultaneously show
that our source and sink set modeling approach satisfies condition (ii) 
$\Delta_H = \Delta_{H_{V'}}$. \\
%Before we start to define the source and sink sets, we need some important notations.
%Let $\Pi_1 = (V_1, V_2)$ be a bipartition of hypergraph \HypergraphDef. We define $V' \subseteq V$
%and the set of all \emph{border hyperedges} $\delta B = \{ e \in E\ |\ \exists u,v \in e: u \in V'\ \land\ v \notin V' \}$
%(see \autoref{img:balanced_bipartition}) which are the hyperedges of $H$ partially
%contained in the subhypergraph $H_{V'}$. Further, we divide $\delta B$ into two disjoint subsets:
%\begin{enumerate}
%\item Non-Cut hyperedges $e \in \delta B$ of $H$: $\delta B_1 = \{ e \in \delta B \ |\ e \subseteq V_1\ \lor\ e \subseteq V_2 \}$
%\item Cut hyperedges $e \in \delta B$ of $H$: $\delta B_2 = \delta B \setminus \delta B_1$
%\end{enumerate}
We will first define $S$ and $T$ for flow network
$T_L(\SubExtension)$, because for each maximum $(S,T)$-flow $f$ of $T_L(\SubExtension)$
and its corresponding minimum $(S,T)$-bipartition $\Pi_2$ the equality 
$|f| = \omega_{\SubExtension}(\Pi_2) = \omega_{H_{V'}}(\Pi_2)$ holds per Equation
\ref{eq:sub_cut} and due to the \emph{max-flow-min-cut} theorem \cite{ford1956maximal}. 
Defining a hypernode $v \in V_1$ resp. $v \in V_2$ as source resp. sink means, that
it cannot change its block after a \emph{Max-Flow-Min-Cut} computation.
However, we do not want that a hypernode $v \notin V'$ can change its block after
a \emph{Max-Flow-Min-Cut} computation on $T_L(\SubExtension)$, because such hypernodes
cannot move if we solve a flow problem on subhypergraph $H_{V'}$. Therefore, we define
all hypernodes $V'' \cap V_1$ as sources and all hypernodes $V'' \cap V_2$ as sinks 
($V''$ is defined in Defintion \ref{def:sub_extension}). More formally:
\begin{align}
S = V'' \cap V_1 \label{S_border_hyperedges}\\
T = V'' \cap V_2 \label{T_border_hyperedges}
\end{align}
The pins of a hyperedge which is partially contained in the extension
of the subhypergraph are not able to change its block after a \emph{Max-Flow-Min-Cut}
computation, because we define them either as source or sink. Consequently, 
no hyperedge partially contained in $\SubExtension$ can change its state 
from non-cut to cut and therefore it follows with the \emph{max-flow-min-cut} 
theorem \cite{ford1956maximal} that the inequality $\omega_H(\Pi_2) \le \omega_H(\Pi_1)$
holds. Since, only nets fully contained in $\SubExtension$ can change its state and
$|f| = \omega_{\SubExtension}(\Pi_2) = \omega_{H_{V'}}(\Pi_2)$, it holds that
$\Delta_H = \Delta_{H_{V'}}$. \\
We will now define the source and sink set of $T_L(H_{V'})$, which will be a consequence
of the proof of Theorem \ref{theorem:ST}.
\begin{align}
S' = \{\incoming{e}\ |\ e \in I(S)\} \label{eq:S}\\
T' = \{\outgoing{e}\ |\ e \in I(T)\} \label{eq:T}
\end{align}
For a source node $s \in S$ resp. sink node $t \in T$ of $T_L(\SubExtension)$ 
we define the corresponding \emph{incoming} resp. \emph{outgoing}
hyperedge nodes of all incident hyperedges $e \in I(s)$ resp. $e \in I(t)$ as 
source resp. sink nodes of $T_L(H_{V'})$.

\begin{theorem}
\label{theorem:ST}
The value of a maximum $(S,T)$-flow of $T_L(\SubExtension)$ is equal with
a maximum $(S',T')$-flow of $T_L(H_{V'})$.
\end{theorem}

\begin{proof}
The value of a maximum $(S,T)$-flow of $T_L(\SubExtension)$ is equal with a maximum
$(S',T')$-flow of $T_H(\SubExtension,V'')$ according to Theorem \ref{theorem:heuer_network}.
Since each $v \in V''$ is either a source or sink node of $T_L(\SubExtension)$, the 
removal of $v$ does not induce any additional edges in $T_H(\SubExtension,V'')$
(see Lemma \ref{lemma:source_and_sink_removal}). Therefore,
$T_H(\SubExtension,V'') = T_L(H_{V'})$.
\end{proof}

The value of a maximum $(S,T)$-flow of $T_L(\SubExtension)$ and a maximum $(S',T')$-flow
of $T_L(H_{V'})$ are equal. Since $S$ and $T$ satisfy condition (i) and (ii) of our 
problem statement, also $S'$ and $T'$ satisfy the two conditions. Furthermore, no hypernode
of $T_L(H_{V'})$ is either a source or sink node. Consequently, all hypernodes of $V'$
can change its block after a \emph{Max-Flow-Min-Cut} computation. According to the 
\emph{max-flow-min-cut} theorem, the value of the cut of bipartition $\Pi_2$ is the minimum
cut of all possible bipartitions with the restriction that only hypernodes of $V'$ can move.\\
\begin{figure}
\centering
\includegraphics[width=0.95\textwidth]{../img/source_and_sink_set/edge_size_two.eps}
\caption{Illustration of modeling hyperedges of size two if the incoming or outgoing
         hyperedge node is a source or a sink node of the flow problem.}
\label{img:edge_size_two}
\end{figure}
If we compare our source and sink set modeling approach with the one of Sanders and Schulz
\cite{sanders2011engineering} (see Section \ref{sec:balanced_bipartitioning}), we can show that
with our technique better minimum $(S,T)$-bipartitions are achievable. They define each node of the graph
as source resp. sink which is adjacent to a node not contained in the flow problem of block $V_1$
resp. $V_2$. Consequently, a non-cut edge of the graph partially contained in the flow problem
cannot become a cut edge. Therefore, their modeling approach satisfy condition (i) of our problem
statement. However, it might be more benificial if we move a node incident to a non-cut edge because
we remove instead more edges from cut. A graph can still be interpreted as hypergraph. Therefore,
we can use our modeling approach on network $T_L(H_{V'})$ with $S'$ and $T'$ as source and sink
set. Further, we can reduce the number of nodes and edges, if model each edge which is fully contained
in $H_{V'}$ as graph edge according to Lemma \ref{lemma:undirected_transformation}. Each edge
partially contained in the flow problem is modeled as hyperedge of size one (see \autoref{img:edge_size_two}).
All nodes incident to a non-cut edge which is partially contained in the flow problem
now are able to change its block and the corresponding minimum $(S',T')$-bipartition is minimum
among all possible bipartitions where only nodes of $V'$ can move. \\
However, we can model hyperedges of size one more efficiently (see \autoref{img:edge_size_two}).
If the incoming hyperedge node $\incoming{e}$ is a source node, we can replace the hyperedge of
size one with a directed edge $(\incoming{e},v)$ with $v \in e \cap V'$ and capacity $\omega(e)$.
If the outgoing hyperege node $\outgoing{e}$ is a sink node, we add a directed edge
$(v,\outgoing{e})$ with capacity $\omega(e)$. If $\incoming{e}$ and $\outgoing{e}$ is not a source
and sink node, we can remove the hyperedge from the flow problem. \\
With the given approach we can optimize the cut metric of a given
bipartition of a hypergraph $H$. We can transfer those results to improve
a $k$-way partition $\Pi = (V_1,\ldots,V_k)$ if the objective is the connectivity
metric. Let $V' \subseteq V_i \cup V_j$ be a subset of the hypernodes of two adjacent
blocks $V_i$ and $V_j$. If we optimize the cut of
subhypergraph $H_{V'}$ we simultaneously optimize the connectivity metric of $H$.
The reduction of the cut of $H_{V'}$ is then equal with the decrease in
the connectivity metric of $H$.

%\begin{figure}[ht!]
%\centering
%\includegraphics[width=0.95\textwidth]{../img/source_and_sink_set/st_modelling_summary.eps}
%\caption{Illustration of modeling sources and sinks defined in Equation \ref{S_border_hyperedges}
%         and \ref{T_border_hyperedges}. }
%\label{img:st_modelling_summary}
%\end{figure}

\subsection{Most Balanced Minimum Cuts on Hypergraphs}
\label{sec:mbmc_hypergraphs}

Picard and Queyranne \cite{picard1980structure} show that all minimum $(s,t)$-cuts 
of a graph $G$ are computable with one maximum $(s,t)$-flow computation by 
iterating through all \emph{closed node sets} of the residual graph of $G$. 
The corresponding algorithm is presented in Section \ref{sec:related_mbmc}. \\
We can apply the same algorithm on hypergraphs. A minimum-capacity $(s,t)$-cutset of \ShortT{L}
is equal with a minimum-weight $(s,t)$-cutset of $H$. With the algorithm
of Section \ref{sec:related_mbmc} we can find all minimum-capacities
$(s,t)$-cutsets of \ShortT{L}, which are also minimum-weight $(s,t)$-cutsets
of $H$. The corresponding minimum-weight $(s,t)$-bipartitions are all
\emph{closed node sets} of the residual graph of \ShortT{L}. 
However, when we use e.g., \ShortExT{H}{V'} (see Section \ref{sec:heuer_network})
or \ShortHybrid~(see Section \ref{sec:hybrid_network}) as underlying flow network,
some hypernodes are removed from the flow problem. Note, we can use 
Lemma \ref{lemma:bipartition_construction} to find all minimum
$(s,t)$-cutsets with one maximum $(s,t)$-flow calculation. However, the algorithm would
become a lot more complicated. Since the algorithm has a linear running time, we decided
to use simple trick. \\
After a maximum $(s,t)$-flow calculation (with $f$ as maximum flow)
on one of the two mentioned networks we insert all removed hypernodes with
their corresponding edges again into the residual graph of our flow network.
If we remove a hypernode, we insert a biclique between all \emph{incoming}
and \emph{outgoing} edges with \emph{infinite} capacity. Since $|f| < \infty$,
there cannot be any inserted edge which is the minimum-capacity $(s,t)$-cutset
regarding to maximum $(s,t)$-flow $f$. Otherwise, an edge of the biclique
would be in the minimum-capacity $(s,t)$-cutset, too and this would imply that
$|f| = \infty$. Further, the maximum $(s,t)$-flow $f$ is still maximal. If we
find an \emph{augmenting} path in the network with the inserted hypernodes, we can
map the path to an \emph{augmenting} path in the network without them
by simply using the shortcut edges instead of the inserted edges. Note, that
the mapping is valid because all edges involve in the transformation have
\emph{infinite} capacity. This is a contradiction that $f$ is a maximum $(s,t)$-flow. 


\subsection{A direct $k$-way Flow-Based Refinement Framework}
\label{sec:flow_local_search_hypergraph}

We have described how a hypergraph $H$ can be transformed into
a flow network \ShortT{L} such that each minimum-capacity $(S,T)$-cutset of \ShortT{L} is a 
minimum-weight  $(S,T)$-cutset of $H$ (see Section \ref{sec:related_lawler}). 
Additionaly, we present techniques to sparsify the
flow network \ShortT{L} \cite{lawler1973} to reduce the complexity of 
the flow problem (see Section \ref{sec:opt_flow_network}). 
Further, we show how to configure the source and sink sets of the flow network of a 
subhypergraph $H_{V'}$ (with $V' \subseteq V$) such that a \emph{Max-Flow-Min-Cut} 
computation improves a given bipartition of $H$ (see Section \ref{sec:source_and_sink}). 
Finally, we can enumerate all minimum-weight $(S,T)$-cutsets of a subhypergraph 
$H_{V'}$ with one maximum $(S,T)$-flow calculation \cite{picard1980structure}. \\
We will now present our direct $k$-way \emph{flow}-based refinement framework which we integrate
into the $n$-level hypergraph partitioner \emph{KaHyPar} \cite{heuer2017improving} 
(see Section \ref{sec:kahypar}). Our framework optimizes
the \emph{connectivity} metric. We use a similiar architecture as proposed
by Sanders and Schulz \cite{sanders2011engineering} (see Section 
\ref{sec:flow_local_search_graph}). The basic concepts of the framework are
illustrated in \autoref{img:flow_framework}. \\
We perform a \emph{flow}-based \emph{local search} on two adjacent blocks of
the partition $\Pi$. We embed our \emph{Max-Flow-Min-Cut} computations into an \emph{active block scheduling}
refinement strategy \cite{holtgrewe2010engineering} (see Section \ref{sec:abs}).
The algorithm starts by constructing the quotient graph $Q$ of $\Pi$ (see definition \ref{def:quotient_graph}). 
Afterwards, we iterate over all edges of $Q$ in random order. For each edge
$(V_i,V_j)$ of $Q$, we build a flow problem around the cut of the bipartition
induced by $V_i$ and $V_j$. We use two \BFS s, one only 
touches hypernodes of $V_i$ and the second only touches hypernodes of $V_j$.
The \BFS~is initialized with all hypernodes contained in a cut hyperedge
of the bipartition $(V_i,V_j)$. We embed the pairwise \emph{flow}-based refinement
into an \emph{adaptive flow iterations} strategy \cite{sanders2011engineering}
(see Section \ref{sec:adaptive_flow_iterations}) which also determines
the size of the flow problem. \\
After we define the subhypergraph $H_{V'}$, which we use to improve the bipartition
$(V_i,V_j)$, we construct one of the flow networks proposed in Section
\ref{sec:opt_flow_network} with sources $S$ and sinks $T$ defined in
Section \ref{sec:source_and_sink}. We implement two maximum flow algorithms.
One is a slightly modified \emph{augmenting path} algorithm of Edmond \& Karp
\cite{edmonds1972theoretical} (see Section \ref{sec:aug_path}) 
and the second is the \emph{Push-Relabel} algorithm of
Goldberg \& Tarjan \cite{cherkassky1997implementing,goldberg1988new} 
(see Section \ref{sec:push_relabel}). Since we have a 
\emph{Multi-Source-Multi-Sink} problem, we can find several \emph{augmenting paths}
with one \BFS. After we execute a \BFS~on the residual graph, we search 
as many as possible edge-disjoint paths in the resulting \BFS-tree connecting a source $s$
with a sink $t$. Our Goldberg \& Tarjan implementation uses a \emph{FIFO} queue and
the \emph{global relabeling} and \emph{gap} heuristic \cite{cherkassky1997implementing}.
After we determine a maximum $(S,T)$-flow on our flow network, we iterate over
all minimum $(S,T)$-bipartitions of $H_{V'}$ \cite{picard1980structure} and choose 
the \emph{Most Balanced Minimum Cut} (see Section \ref{sec:related_mbmc} and 
\ref{sec:mbmc_hypergraphs}) according to our \emph{balanced constraint}. \\
\emph{KaHyPar} is an $n$-level hypergraph partitioner ($|V| = n$) taking the 
multilevel paradigm to its extreme by removing only a single vertex in each level
of the hierarchy \cite{akhremtsev2017engineering} (see Section \ref{sec:kahypar}). 
During the refinement step $n$ local searches are instantiated. Therefore, 
using our \emph{flow}-based refinement as local search algorithm on each level is not 
applicable, because the performance slowdown would be tremendous. Therefore,
we introduce \emph{Flow Execution Policies}. One is to execute our \emph{flow}-based
refinement on each level $i$ where $i = \beta\cdot j$ with $j \in \mathbb{N}_+$ and
$\beta$ as a predefined tunning parameter. Another approach is to simulate a
multilevel partitioner with $\log(n)$ hierarchies. A \emph{flow}-based refinement is then
executed on each level $i$ where $i = 2^j$ with $j \in \mathbb{N}_+$. Each policy also
performs the \emph{active block scheduling} refinement strategy on the last level of the
hierarchy. In all remaining levels where no flow is executed, we can use an 
\emph{FM} algorithm 
\cite{akhremtsev2017engineering,fiduccia1988linear,sanchis1989multiple} (see Section 
\ref{sec:abs}). \\
An observation during the implementation of this framework was that only a minority
of the pairwise refinements based on flows yields an improvement
on hypergraph $H$. Thus, we introduce several rules which might prevent
\emph{unpromising} flow executions to improve the effectiveness ratio by simultaneously speeding up
the running time.

\begin{enumerate}
\item[(R1)] If a \emph{flow}-based refinement did not lead to an improvement on two blocks in all previous
            runs, we use the \emph{Max-Flow-Min-Cut} computations only in the first iteration of 
            \emph{active block scheduling}.
\item[(R2)] If the cut between two adjacent blocks in the quotient graph is small (e.g. $\le 10$) we
            skip the \emph{flow}-based refinement on the blocks except on the last level of the hierarchy.
\item[(R3)] If the value of the cut of a minimum $(S,T)$-bipartition of $H_{V'}$ is the same 
            as the cut before, we stop the \emph{adaptive flow iteration} strategy.
\end{enumerate}

\begin{figure}
\centering 
\includegraphics[width=1.0\textwidth]{../img/flow_local_search/flow_framework_hypergraph.eps}
\caption{Illustration of our \emph{flow}-based refinement framework for direct $k$-way hypergraph
         partitioning.}
\label{img:flow_framework}
\end{figure} 


% First Version of Section 5.1

%An example of a \emph{Max-Flow-Min-Cut} computation of $H_{V'}$ with $S$ and $T$ as source and
%sink set is illustrated in \autoref{img:border_hyperedges}.
%\begin{figure}
%\centering
%\includegraphics[width=0.95\textwidth]{../img/source_and_sink_set/border_hyperedges.eps}
%\caption{Non-cut \emph{border hyperedges} of $H$ and $H_{V'}$ induce source and sink hypernodes
%         in the flow problem.}
%\label{img:border_hyperedges}
%\end{figure}
%
%\begin{lemma}
%\label{cut_decrease_proof}
%Let $\Pi_1$ be a bipartition of $H$ and $T_L(H_{V'})$ the flow network of subhypergraph
%$H_{V'}$ with $S$ and $T$ as defined in Equation \ref{S1_border_hyperedges} and \ref{T1_border_hyperedges} (with $V' \subseteq V$).
%If $\Pi_2$ is a bipartition obtained by a maximum $(S,T)$-flow computation on $T_L(H_{V'})$,
%then the inequality $\omega_H(\Pi_2) \le \omega_H(\Pi_1)$ holds.
%\end{lemma}
%
%\begin{proof}
%A maximum $(S,T)$-flow computation on $T_L(H_{V'})$ yields a minimum $(S,T)$-cutset on 
%$H_{V'}$ \cite{ford1956maximal}. Thus, for all hyperedges $e \notin \delta B$ (fully contained in $H_{V'}$)
%which are cut in $\Pi_2$, the sum of their weight must be less or equal than the sum of all cut hyperedges
%$e \notin \delta B$ of bipartition $\Pi_1$. We have to show that a non-cut
%hyperedge $e \in \delta B_1$ of $\Pi_1 = (V_1,V_2)$ cannot become a cut hyperedge of
%$\Pi_2 = (V_1',V_2')$. Let $e \in \delta B_1$ be such a hyperedge. $e$ must be either a subset of $V_1$ or $V_2$, otherwise
%$e$ is a cut hyperedge. Let $e \subseteq V_1$, then $e \cap V' \subseteq S$ (see Equation \ref{S1_border_hyperedges}). 
%Defining a node $s \in S$ as source node means that it cannot change its block after a \emph{Max-Flow-Min-Cut}
%computation. Therefore, $e \subseteq V_1$ and $e \subseteq V_1' \Rightarrow e$ is a non-cut
%hyperedge of $\Pi_2$. The proof for $e \subseteq V_2$ is equivalent $\Rightarrow \omega_H(\Pi_2) 
%\le \omega_H(\Pi_1)$.
%\end{proof} 
%\begin{figure}[ht!]
%\centering
%\includegraphics[width=1.0\textwidth]{../img/source_and_sink_set/non_cut_flow_hyperedges.eps}
%\caption{In this example $e_1$ and $e_3$ are cut hyperedges of the hypergraph, but non-cut nets
%        of subhypergraph $H_{V'}$. Modeling the \emph{outgoing} resp.
%        \emph{incoming} hyperedge node of $e_1$ resp. $e_2$ as sink resp. source ensures
%        that $\Delta_H = \Delta_{H_{V'}}$.} 
%\label{img:non_cut_flow_hyperedges}
%\end{figure}
%
%In the next step, we will show how $S$ and $T$ can be extended to satisfy condition (ii)
%of Problem \autoref{prob:ST}. Currently, $|f| \le \omega_{H_{V'}}(\Pi_2)$ (without a prove).
%Obviously, some nodes are missing in $S$ and $T$. Consider \autoref{img:non_cut_flow_hyperedges}
%to understand which nodes are missing. Transformation $1$ illustrates our current modeling
%approach defined in Equation \ref{S1_border_hyperedges} and \ref{T1_border_hyperedges}. The maximum
%flow on this network is $|f| = 1$, but the resulting minimum $(S,T)$-bipartition $\Pi_2$ induce
%a cut of $\omega_{H}(\Pi_2) = 2$. This implies that $\Delta_H = 3 - 2 \neq 3 - 1 = \Delta_{H_{V'}}$.
%The hyperedges $e_1$ and $e_3$ are cut nets of $H$, but non-cut hyperedges of $H_{V'}$. 
%Therefore, $\Pi_1$ induce a cut of $1$ on $H_{V'}$ if we define the cut $\omega_{H_{V'}}(\Pi_2)$
%over the cut hyperedges of $H_{V'}$ instead of the cut hyperedges of $H$. In our example, we can
%remove $e_2$ from cut, but $e_1$ becomes a cut hyperedge of $H_{V'}$. Therefore, the value of the
%cut of $H_{V'}$ does not change, but the cut of $H$ does. $e_1$ is already a cut hyperedge of $H$ and $\Pi_2$ removes
%$e_2$ from the cut of $H$. Therefore, $\Delta_H = 1$. However, we defined $\omega_{H_{V'}}(\Pi_2)$
%over the cut hyperedges of $H$ and currently, we hvae $|f| = 1 \neq 2 = \omega_{H_{V'}}(\Pi_2)$. \\
%Transformation $2$ illustrates the adapted modeling approach for cut hyperedges of $H$.
%For each hyperedge $e \in \delta B_2$ with $e \setminus V' \cap V_1 \neq \emptyset$, 
%we add the \emph{incoming hyperedge node} $\incoming{e}$ to $S$.
%More formal:
%\begin{align}
%S = S_1 \cup \{\incoming{e} \in \delta B_2\ |\ e \setminus V' \cap V_1  \neq \emptyset\} \label{S_border_hyperedges}\\
%T = T_1 \cup \{\outgoing{e} \in \delta B_2\ |\ e \setminus V' \cap V_2 \neq \emptyset \} \label{T_border_hyperedges}
%\end{align}
%
%We have to show that for a maximum $(S,T)$-flow $f$ of $T_L(H_{V'})$ holds $|f| = \omega_{H_{V'}}(\Pi_2)$.
%The idea of the proof is to use the extension $\SubExtension$ of $H_{V'}$ and add 
%all $V'' \cap V_1$ to $S_1$ (see Equation \ref{S1_border_hyperedges}) and all $V'' \cap V_2$ to
%$T_1$ (see Equation \ref{T1_border_hyperedges}). We can show that for a maximum $(S_1,T_1)$-flow 
%$f'$ of $T_L(\SubExtension)$ holds that $|f'| = \omega_{H_{V'}}(\Pi_2)$.\\
%Afterwards, we use a technique to remove all $v \in V''$ of $T_L(\SubExtension)$ and show
%that the resulting flow network is $T_L(H_{V'})$ with $S$ and $T$ as source and sink set as
%defined in Equation \ref{S_border_hyperedges} and \ref{T_border_hyperedges}. Moreover, for a maximum
%$(S,T)$-flow $f$ then holds that $|f| = |f'| = \omega_{H_{V'}}(\Pi_2)$. \\
%Because of the complexity of the proof, we will introduce lemmas in the following which
%will simplify the proof of the main theorem. Consider \autoref{img:general_source_and_sink}~if you
%need an illustration for the following lemmas.
%
%\begin{figure}
%\centering
%\includegraphics[width=0.95\textwidth]{../img/source_and_sink_set/general_source_and_sink.eps}
%\caption{Illustration of the proof technique used in Lemma \ref{lemma:general_source_and_sink_removal}.
%         The green node $s$ is the super source of the flow problem. The blue nodes are source nodes of
%         the corresponding \emph{multi-source multi-sink} flow problem. The red nodes $u$ and $v$ are
%         contained in $\mathcal{R}(s_2)$. Therefore, edges $(s_2,u)$ and $(s_2,v)$ are removable.}
%\label{img:general_source_and_sink}
%\end{figure}
%
%\begin{lemma}[Source Edge Removal]
%\label{lemma:source_edge_removal}
%Let $f$ be a maximum $(S,T)$-flow of $G = (V,E,\capa)$. If there exists two edges $(s_1,v)$ and $(s_2,v)$
%with infinite capacity ($s_1,s_2 \in S$) we can either remove $(s_1,v)$ or $(s_2,v)$ from $G$
%without changing the amount of a maximum $(S,T)$-flow.
%\end{lemma}
%
%\begin{proof}
%Let $P = (s_1,v,\ldots)$ be an augmenting path of $G$. Replacing $s_1$ in $P$ with $s_2$ yields
%an augmenting path $P'$ of same length. The operation is valid because $\capacity{s_1,v} = \capacity{s_2,v} = \infty$. 
%If we execute Edmond and Karp's maximum flow algorithm we can map each augmenting path 
%$P$ to $P'$ and ensure that for a maximum $(S,T)$-flow $f$ follows that $f(s_1,v) = 0$. 
%Consequently, there exists maximum $(S,T)$-flows where either $f(s_1,v) = 0$ or $f(s_2,v) = 0$.
%Therefore, we can remove either $(s_1,v)$ or $(s_2,v)$ without changing the amount of
%a maximum $(S,T)$-flow.
%\end{proof}
%
%\begin{lemma}[Sink Edge Removal]
%\label{lemma:sink_edge_removal}
%Let $f$ be a maximum $(S,T)$-flow of $G$. If there exists two edges $(v,t_1)$ and $(v,t_2)$
%with infinite capacity ($t_1,t_2 \in T$) we can either remove $(v,t_1)$ or $(v,t_2)$ from $G$
%without changing the amount of a maximum $(S,T)$-flow.
%\end{lemma}
%
%\begin{proof}
%Equivalent to proof of Lemma \ref{lemma:source_edge_removal}.
%\end{proof}
%
%\begin{definition}[Removable Edges]
%We denote the set of all adjacent nodes $v$ of a source node $s$ resp. sink node $t$, where
%edge $(s,v)$ or $(v,t)$ is removable according to Lemma \ref{lemma:source_edge_removal}
%and \ref{lemma:sink_edge_removal}, with $\mathcal{R}(s)$ resp. $\mathcal{R}(t)$.
%\end{definition}
%
%The following lemma is a generalisation of Lemma \ref{lemma:source_and_sink_removal}. We will use
%the definition of $in(u)$ and $out(u)$ presented in Section \ref{sec:heuer_network}. Further,
%$G_{V'}$ is a subgraph of $G = (V,E)$ induce by $V' \subseteq V$ (see definition \ref{def:subgraph}).
%
%\begin{lemma}[General Source/Sink Node Removal]
%\label{lemma:general_source_and_sink_removal}
%Let $f$ be a maximum $(S,T)$-flow of $G = (V,E,\capa)$ with $|f| < \infty$ and
%$E_s \subseteq \mathcal{R}(s)$ and $E_t \subseteq \mathcal{R}(t)$ with $s \in S$
%and $t \in T$.
%If $s$ is a source node where all outgoing edges have infinte capacity and
%$t$ is a sink node where all incoming edges have infinte capacity, then
%$|f|$ is equal with the amount of a maximum $(S',T)$-flow of $G_{V\setminus \{s\}}$ and a maximum
%$(S,T')$-flow of $G_{V\setminus \{t\}}$, where $S' = (S\setminus \{s\}) \cup (out(s) \setminus E_s)$ 
%and $T' = (T \setminus \{t\}) \cup (in(t) \setminus E_t)$.
%\end{lemma}
%
%\begin{proof}
%$E_s$ is an arbitrary subset of $\mathcal{R}(s)$, where foreach $v \in E_s$ the edge $(s,v)$ is removable.
%$S'$ is the source set without node $s$ extended with all outgoing edges of 
%$s$ minus the removable edges $E_s$. With Lemma \ref{lemma:source_edge_removal}
%we can remove all edges $(s,v)$ with $v \in E_s$ from $G$ and obtain flow network $G'$. Finally, we can
%apply Lemma \ref{lemma:source_and_sink_removal} on $G'$ and obtain $G_{V\setminus \{s\}}$ with $(S',T)$ as source
%and sink set (see \autoref{img:general_source_and_sink}). All used Lemma's did not change the amount of a maximum flow. Therefore, a maximum
%$(S,T)$-flow of $G$ is equal with a maximum $(S',T)$-flow of $G_{V\setminus \{s\}}$. The proof for $t$ is
%equivalent.
%\end{proof}
%
%The proof of Lemma \ref{cut_decrease_proof} can be applied one-to-one on our new source and
%sink sets because $S_1 \subseteq S$ and $T_1 \subseteq T$. Therefore, $S$ and $T$ 
%as defined in Equation \ref{S_border_hyperedges} and \ref{T_border_hyperedges} satisfies
%condition (i) of Problem \ref{prob:ST}. We will show that for $S$ and $T$ 
%the equality $\Delta_H = \Delta_{H_{V'}}$ holds.
%
%\begin{theorem} 
%\label{lemma:delta_proof}
%Let $\Pi_1 = (V_1,V_2)$ be a bipartition of $H$ and $T_L(H_{V'})$ the flow network of subhypergraph
%$H_{V'}$ with $S$ and $T$ as defined in Equation \ref{S_border_hyperedges} and \ref{T_border_hyperedges} (with $V' \subseteq V$).
%If $\Pi_2$ is a bipartition obtained by a maximum $(S,T)$-flow computation on $T_L(H_{V'})$
%with $f$ as maximum flow,
%then $\omega_{H_{V'}}(\Pi_2) = |f|$ ($\Rightarrow \Delta_H = \Delta_{H_{V'}}$).
%\end{theorem}
%
%\begin{proof}
%Consider the extension $\SubExtension$ of subhypergraph $H_{V'}$ (see Definition \ref{def:sub_extension}).
%Each maximum $(S,T)$-flow $f'$ of $T_L(\SubExtension)$ is then equal with a minimum-weight
%$(S,T)$-cutset of $H_{V'}$ according to our definition of the cut $\omega_{H_{V'}}(\Pi_2)$ over
%the cut hyperedges of $H$. Because each hyperedge which is partially contained in $H_{V'}$
%is fully contained in $\SubExtension$. Therefore, it holds that $|f'| = \omega_{H_{V'}}(\Pi_2)$.
%However, we have to model some restrictions into our source and sink set of $T_L(\SubExtension)$.
%We will denote the source and sink set of $T_L(\SubExtension)$ with $S'$ and $T'$.
%Each hypernode contained in a non-cut border hyperedge $e \in \delta B_1$ should not be able to move
%such that we ensure that $e$ is not cut after a maximum $(S',T')$-flow calculation. Therefore,
%we add $S_1$ and $T_1$ to $S'$ and $T'$ (see Equation \ref{S1_border_hyperedges} and \ref{T1_border_hyperedges}).
%Further, all hypernodes $v \in V''$ (see definition \ref{def:sub_extension}) are not contained in
%$H_{V'}$. Consequently, they cannot change their block if we calculate a maximum $(S,T)$-flow of
%$T_L(H_{V'})$. Therefore, we add $V'' \cap V_1$ to $S'$ and $V'' \cap V_2$ to $T'$.
%With $S'$ and $T'$ as source and sink set we ensure that only hypernodes $v \in V'$ are able
%to move and since $S_1 \subseteq S'$ and $T_1 \subseteq T'$, we ensure that $\omega_H(\Pi_2) \le \omega_H(\Pi_1)$
%(see Lemma \ref{cut_decrease_proof}). \\
%In the following, we apply Lemma \ref{lemma:general_source_and_sink_removal} on all
%hypernodes $v \in V''$ such that the flow network $T_L(\SubExtension)$ converge against
%$T_L(H_{V'})$ with $S$ and $T$ as source and sink set as defined in Equation 
%\ref{S_border_hyperedges} and \ref{T_border_hyperedges} \emph{without} changing the
%amount of the maximum flow $f'$. Since $|f'| = \omega_{H_{V'}}(\Pi_2)$, for a maximum $(S,T)$-flow
%$f$ of $T_L(H_{V'})$ then holds $|f| = \omega_{H_{V'}}(\Pi_2)$. Per definition a node $v \in V''$
%is either a source or sink node. For each source node $s \in V''$ we have to 
%define a removable subset $E_s \subseteq \mathcal{R}(s)$ such that after removing all $s \in S' \cap V''$
%the resulting source set $S$ is equal to Equation \ref{S_border_hyperedges}. The technique
%for removing each sink node $t \in S' \cap V''$ will be equivalent. For a source node $s \in V''$
%we define $E_s = \{\incoming{e}\ |\ e \in I(s) \cap \delta B_1\}$. Remember, $\delta B_1$ contains
%all non-cut border hyperedges of $H$. Thus, for each $e \in I(s) \cap \delta B_1$ exists
%a source node $\bar{s} \in V'$ of $S_1$ such that edges $(s,\incoming{e})$ and $(\bar{s},\incoming{e})$
%are contained in $T_L(\SubExtension)$ (see \autoref{img:delta_proof_illustration}). Therefore, $E_s$ is a removable subset of $\mathcal{R}(s)$.
%For each $t \in T' \cap V''$ we define the removable subset $E_t = \{\outgoing{e}\ |\ e \in I(t) \cap \delta B_1\}$.
%Applying Lemma \ref{lemma:general_source_and_sink_removal} on all source nodes $s \in V''$
%with $E_s$ as removable subset and on all sink nodes $t \in V''$ with $E_t$ 
%as removable subset yield flow network $T_L(H_{V'})$ with $S$ and $T$ as source and sink 
%set as defined in Equation \ref{S_border_hyperedges} and \ref{T_border_hyperedges}.
%A hyperedge $e \in \delta B_1$ cannot become a source or sink node because if we remove a 
%source node $s \in e$ then $e'$ is in the removable subset $E_s$ (see \autoref{img:delta_proof_illustration}). The same holds
%for each sink node $t \in V''$. A hyperedge $e \in \delta B_2$ becomes a source node of $S$
%if we remove a source node $s \in e$ and a sink node of $T$ if we remove a 
%sink node $t \in e$ because $\forall e \in \delta B_2: e',e'' \notin E_s \cup E_t$. 
%Therefore, $S$ and $T$ are equal to our source and sink set definition and
%for a maximum $(S,T)$-flow $f$ it holds that $|f| = |f'| = \omega_{H_{V'}}(\Pi_1)$.
%\end{proof}
%
%\begin{figure}
%\centering
%\includegraphics[width=0.95\textwidth]{../img/source_and_sink_set/delta_proof_illustration.eps}
%\caption{Illustration how to remove a source node $v \in V''$. Note, the green node $s$
%         is the super source of the flow problem. Consequently, all nodes connected to $s$
%         are source nodes in the corresponding \emph{multi-source multi-sink} flow problem.}
%\label{img:delta_proof_illustration}
%\end{figure}


%\begin{figure}
%\centering
%\includegraphics[width=0.95\textwidth]{../img/source_and_sink_set/cut_border_hyperedges.eps}
%\caption{Illustration of modeling \emph{Cut Border Hyperedges} as sources and sinks. In this
%         example $e_1$ contains node from block $V_1$ and $V_2$ not contained in the flow problem. Therefore,
%         we can not remove $e_1$ from cut. Treating $e_1$ as a \emph{Border Hyperedge} would result
%         in Transformation $1$. This has the consequence that we are not able to remove $e_2$
%         from cut with a \emph{Max-Flow-Min-Cut} computation. Defining the \emph{incoming} resp.
%         \emph{outgoing} hyperedge of $e_1$ as source resp. sinks allows the corresponding hypernodes
%         of $e_1$ still to move. The consequence is that we can remove $e_2$ from cut with a
%         \emph{Max-Flow-Min-Cut} computation in Transformation $2$. }
%\label{img:cut_border_hyperedges}
%\end{figure}
