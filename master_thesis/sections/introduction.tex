Hypergraphs are a generalization of graphs, where each (hyper)edge can connect 
more than two (hyper)nodes. The $k$-way hypergraph partitioning problem is to 
partition the vertices of a hypergraph into $k$ disjoint, non-empty blocks such
that the size of each block is smaller than $1+\epsilon$ times the
average block size, while the goal is to simultaneously minimize an objective function. \\
Classical application areas can be found in \emph{VLSI} design, parallelization of
the Sparse Matrix-Vector Product and simplifying \emph{SAT} formulas 
\cite{karypis1999multilevel, mann2014formula, papa2007hypergraph}. The goal 
in \emph{VLSI} design is to partition a circuit into smaller units such that
the wires between the gates are as short as possible \cite{bulucc2016recent}.
A wire can connect more than two gates, therefore a hypergraph models a circuit more
accurately than a graph. In \emph{SAT} solving, hypergraph partitioning is used to
decompose a formula into smaller subformulas, which can be solved more easily \cite{mann2014formula}.\\
%Another interesting application area of hypergraph partitioning is \emph{Warehouse Planning}. 
%A warehouse consists of several storage spaces where products can be placed. 
%If a list of previous orders is available, we can interpret the products as 
%vertices and the orders as hyperedges. If we partition the hypergraph 
%into $k$ blocks, where $k$ is the number of storage spaces,
%we can place products in the warehouse such that products are close to each other if
%they are often ordered together.\\
Hypergraph partitioning is an NP-hard problem \cite{lengauer2012combinatorial} and
it is even NP-hard to find a good approximation \cite{bui1992finding}.
The most common heuristic used in state-of-the-art hypergraph partitioners is the
\emph{multilevel paradigm} \cite{catalyurek1999hypergraph, heuer2017improving, karypis1999multilevel}.
First, a sequence of smaller hypergraphs is generated by contracting a set of matchings between hypernode pairs
or clusters in each step (\emph{coarsening phase}). If the hypergraph is small enough, we can use expensive
heuristics to \emph{initial partition} it into $k$ blocks. Afterwards, the sequence
of smaller hypergraphs is \emph{uncontracted} in reverse order and, at each level, a 
\emph{local search} heuristic is used to improve the quality of the partition 
(\emph{refinement phase}). \\
There exist several \emph{local search} heuristics for improving hypergraph partitions.
One algorithm used in the state-of-the-art multilevel hypergraph partitioners is
the \emph{Fiduccia-Mattheyses} heuristic (FM). The \emph{FM} algorithm
maintains gain values (according to a objective function) of moving a node
from its current block to another block \cite{fiduccia1988linear}. A move is performed, 
if its gain value is maximum among all possible moves. The \emph{FM} heuristic is generally
intuitive, flexible in adapting to different optimization objectives, easy to implement and
relatively fast \cite{zhao2002effective}. \\
However, the gain of moving a (hyper)node to another block only depends on the state of the
incident (hyper)edges. Therefore, \emph{FM} has no \emph{global} view on the structure
of the problem. A move is performed \emph{locally} and \emph{greedily}.
Consequently, the algorithm tends to find locally optimal solutions, which quality heavily depends
on the initial partition \cite{dutt1997vlsi}. Therefore, multiple runs are needed to find
a solution close to the global optimum. The probability of finding a good approximation
of the global optimum significantly drops if we partition large hypergraphs \cite{dutt1997vlsi}.
If we execute \emph{FM} in the multilevel context, we partially solve the problem.
A move of an vertex in a \emph{coarsened} hypergraph corresponds to a movement
of a subset of the hypernodes on the \emph{original} hypergraph which allows a more 
effective exploration of the solution space \cite{papa2007hypergraph}. The quality of
the solution then depends more on the quality of the \emph{coarsening} rather than
on the \emph{initial partition}.\\
Further, a move of a node only influences the gain function if the state of an incident hyperedge
changes \emph{immediately} after a move. If a hyperedge contains vertices from two different blocks,
where only one hypernode is contained in the first and all remainings are in the second block,
then a move of that node contributes to the gain if the objective is e.g., \emph{cut} 
(sum of the weights of hyperedges which contains vertices of more than one block). 
Especially for large hyperedges, we often have to \emph{move} a sequence of nodes
such that a single move of a node finally contributes to the gain.
Therefore, the gain of most vertices is equal to zero in such cases \cite{mann2014formula}. 
Krishnamurthy \cite{krishnamurthy1984improved} points out that the quality 
in such situations highly depends on random choices made within
the algorithm. Therefore, he enhanced the \emph{FM} algorithm with a look-ahead scheme
such that in case of ties one can incoroperate \emph{future gains} into the decision \cite{krishnamurthy1984improved}.
However, the \emph{forecast} is limited by a predefined parameter. \\
\emph{FM}-based \emph{local search} algorithms have the above-mentioned disadvantages, because
they are \emph{move}-based and only incoroperate \emph{local} informations about the structure of the problem.
Finding a balanced global minimum cut of a (hyper)graph is NP-hard, but if we ask for a minimum
cut separating two vertices $s$ and $t$ the problem becomes solvable in polynomial time \cite{edmonds1972theoretical}.
The well-known \emph{max-flow min-cut} theorem \cite{ford1956maximal} establish an analogy
between the maximum flow from a source $s$ to a sink $t$ and the minimum cut separating
$s$ and $t$ in a graph. \emph{Flow}-based approaches are not \emph{move}-based and
incoroperate the \emph{global} structure of the problem.
Therefore, it overcomes the drawbacks of the \emph{FM} algorithm. However, it was overlooked for a long time because
it was perceived as computationally expensive and impractical for (hyper)graph partitioning
\cite{liu1998network}. \\ 
Sanders and Schulz \cite{sanders2011engineering} successfully integrated a \emph{flow}-based refinement
algorithm in their multilevel \emph{graph} partitioner. 
%The algorithm is also applicable to the more
%complicated direct $k$-way partitioning case. 
%In general, their basic approach is to 
%extract a subgraph around the cut of a bipartition and configure the source and sink sets
%of the flow problem such that a maximum flow calculation on the subgraph leads to a 
%smaller cut on the original graph. 
They combine the strength of \emph{flow}-based and \emph{FM}
\emph{local search} by executing both algorithms alternating throughout the multilevel hierarchy.
As a result their multilevel graph partitioner produces the best partitions for 
a wide range of graph partitioning benchmarks. 
Recently, several algorithms were developed to obtain a balanced bipartition of a hypergraph 
with \emph{Max-Flow-Min-Cut} computations \cite{liu1998network,patkar2004efficient,yang1996balanced}. 
A balanced $k$-way hypergraph partition with such an approach is currently only
obtainable by applying the bipartitioning algorithm recursively \cite{yang1996balanced}. 
The impact of a \emph{flow}-based \emph{local search} algorithm on the solution quality of 
a multilevel hypergraph partitioner has not been studied yet. 

\subsection{Problem Statement}

Motivated by the successfull integration of a \emph{flow}-based \emph{local search} algorithm in the multilevel graph 
partitioner \emph{KaFFPa} of Sanders and Schulz \cite{sanders2011engineering} to obtain balanced $k$-way
partitions, this thesis investigates the integration of such an approach into the multilevel hypergraph
partitioner \emph{KaHyPar} \cite{heuer2017improving}.\\
In the first step, we have to find an appropriate model of a hypergraph as flow network. Afterwards, 
we want to improve a given bipartition of a hypergraph with a \emph{Max-Flow-Min-Cut} 
computation by using the flow network of the previous step.
This work is the theorectical foundations for developing a \emph{flow}-based \emph{local search} 
algorithm which works in a multilevel hypergraph partitioner and improves a given
balanced $k$-way partition. The last step is to integrate the framework into the
$n$-level hypergraph partitioner \emph{KaHyPar} \cite{heuer2017improving} and evaluate
the performance on a large benchmark set in comparison to different state-of-the-art 
multilevel hypergraph partitioners. A major goal of this work is to outperform the 
latest version of \emph{KaHyPar} on most of the benchmark instances and simultaneously
ensure that the running time is only within a constant factor slower.

\subsection{Contributions}

We present several techniques to sparsify the state-of-the-art hypergraph flow network
modeling approach proposed by Lawler \cite{lawler1973}. Our experiments indicate that
maximum flow algorithms are up to a factor of $3$ faster with our new network. The
theorectical results, which leads to the presented sparsification techniques, are
of independent interest and can also be applied on general flow networks.
Further, we show how to configure a flow problem on a subhypergraph of $H$ such that a 
maximum $(S,T)$-flow yields an improved cut of a given bipartition. We choose $S$ and
$T$ in a way such that the value of the cut of $H$ after a \emph{Max-Flow-Min-Cut} computation on a subhypergraph 
can be calculated with the value of a maximum $(S,T)$-flow.
Our \emph{flow}-based \emph{local search} framework is inspired by algorithmic ideas of
Sanders and Schulz \cite{sanders2011engineering}. However, we generalize many results of
their work such that they are applicable on hypergraph partitioning. Further, we implement
several heuristics which might prevent unpromising \emph{Max-Flow-Min-Cut} computations 
throughout the multilevel hierarchy and show that they speed-up the framework by factor
of $2$ while maintaining the quality of the solutions on average.
We integrate our \emph{flow}-based \emph{local search} algorithm into the $n$-level hypergraph partitioner
\emph{KaHyPar} and show that \emph{flow}-based refinement in combination with the
\emph{FM} algorithm produces the best partitions on a majority of real world benchmarks
in comparison to other state-of-the-art hypergraph partitioners. 
Ccompared to $5$ different systems we achieve the best partitions on $70\%$ 
of $3222$ benchmark instances. In comparison to the 
latest version of \emph{KaHyPar}, our new approach produces solutions that are
$2\%$ better on average, while only incurring a slowdown by a factor of $2$.
Moreover, our partitioner has a comparable running time to the direct $k$-way
version of \emph{hMetis} and outperforms it on $82\%$ of the benchmark instances

\subsection{Outline}

We first introduce necessary notations and summarize related work in Sections \ref{sec:preliminaries}
and \ref{sec:related_work}. Afterwards, we describe techniques to sparsify the flow network
proposed by Lawler \cite{lawler1973} in Section \ref{sec:opt_flow_network}. In Section
\ref{sec:flow_refinement} we present our source and sink set modeling
approach and describe the integration of our \emph{flow}-based refinement framework into
the $n$-level hypergraph partitioner \emph{KaHyPar}. The experimental evaluation of our algorithm
is presented in Section \ref{sec:experiments}. Section \ref{sec:conclusion} concludes this thesis.