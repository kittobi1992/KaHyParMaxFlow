%%%%%%%%%%%%%%%%%%%%%%%%%%%%%%%%%%%%%%%%%%%%%%%%%%%%%%%%%%%%%%%%%%%%%%

\subsection{Maximum Flow Algorithms}
\label{sec:max_flow}

In Section \ref{sec:applications} we introduce the concept of flows in a network. We
will now present two approaches to solve the maximum flow problem.

\subsubsection{Augmenting-Path Algorithms}
\label{sec:aug_path}

An \emph{augmenting path} $P = \{v_1,\ldots,v_k\}$ is a path in the residual graph $G_f$ with $v_1 = s$ and 
$v_k = t$ \cite{edmonds1972theoretical}. \autoref{img:edmond_karp_example}~illustrates such a path.
For all edges $(u,v) \in G_f$ it holds that $r_f(u,v) > 0$.
Therefore we can increase the flow on all edges $(v_i,v_{i+1})$ by 
$\Delta f = \min_{i \in [1,\ldots,k-1]} r_f(v_i,v_{i+1})$. It can be shown that $f$ is not a
maximum flow if an augmenting path exists in $G_f$ \cite{edmonds1972theoretical}. \\
One way to calculate a maximum flow $f$ is to find augmenting paths in $G_f$ as
long as there exists one. The algorithm was established by Ford and Fulkerson \cite{ford1956maximal} and
consists of two phases. First, we search for an augmenting path $P = \{v_1,\ldots,v_k\}$
from $s$ to $t$, e.g., with a simple \DFS. Afterwards, we increase the flow on each
edge $(v_i,v_{i+1})$ by $\Delta f$ and decrease the flow on each reverse edge $(v_{i+1},v_i)$
by $\Delta f$. If the capacities are integral, the algorithm always terminates. Since we can find an augmenting
path in $G_f$ with a simple \DFS~in \BigO{|V|+|E|} time and increase the
flow on every path by at least one, the running time of the algorithm can be bounded by \BigO{|E||f_{max}|}.
We can construct instances where the running time is \BigO{|E||f_{max}|} or even instances where 
the maximum flow $|f_{max}|$ is exponential in the problem size \cite{edmonds1972theoretical}. \\
Edmond and Karp \cite{edmonds1972theoretical} improved Ford \& Fulkerson's algorithm by increasing the flow
along an augmenting path of minimal length. The shortest path from $s$ to $t$ in a 
graph with unit lengths can be found with a simple \BFS. It can be shown that the
total number of \emph{augmentations} is in \BigO{|V||E|}. The running time of Edmond \& Karp's
maximum flow algorithm is \BigO{|V||E|^2}. An exemplary execution of the algorithm
is presented in \autoref{img:edmond_karp_example}. \\
\begin{figure}
\centering
\includegraphics[width=0.9\textwidth]{../img/maximum_flow/edmond_karp_example.eps}
\caption{Execution of Edmond \& Karps maximum flow algorithm \cite{edmonds1972theoretical}.
         The network $G$ with its capacities $c$ (black values) and flow $f$ (red values) is illustrated
         on the left side. The residual graph $G_f$ with its \emph{residual capacities} $r_f$ (black values)
         is presented on the right side. In each step the current augmenting path in $G_f$ is highlighted
         by a red path. }
\label{img:edmond_karp_example}
\end{figure}
Boykov and Kolmogorov proposed a maximum flow algorithm based on augmenting path especially
designed for applications in computer vision \cite{boykov2004experimental}. Their basic idea is to 
grow two search trees simultanousely. One is starting from the source and one from the sink.
The two search trees maintain the invariant that all edges in the tree are non-saturated. More formally,
for an edge $e$ the residual capacity $r_f(e)$ must be greater than zero.
A node is added to one of the two trees if a non-saturated edge exists that connects
the node with one of the nodes in the search trees. If the two trees touch at a given node, we found
an augmenting path from the source to the sink. After we increase the flow along this
path, some of the edges in the two search trees are saturated. Therefore, the algorithm
tries to restore the search tree invariant by finding a new non-saturated edge for each node which
is connected through a saturated edge to the tree. If this is not possible, then the node is removed
from the tree. The algorithm has no polynomial complexity (worst case \BigO{|E||V|^2|f|}), but it
outperforms many state-of-the-art maximum flow algorithms on computer vision benchmarks \cite{boykov2004experimental}.\\
An extension of the algorithm of Boykov \& Kolmogorov is the \emph{incremental breadth-first search}
algorithm \cite{goldberg2011maximum}, which guarantes polynomial running time (\BigO{|V|^2|E|}).
The algorithm maintains two distance labels $d_s$ and $d_t$ for each node. For some values 
$D_s$ and $D_t$, the source tree contains all nodes up to a distance $D_s$ and the sink tree
up to distance $D_t$. Furthermore it maintains the invariant that $L = D_s + D_t + 1$ is a lower bound
for the shortest augmenting path. Initially, $d_s(s) = d_t(t) = 0$ and $D_s = D_t = 0$.
The algorithm works in passes and in a pass one of the two trees is chosen to grow. Assume, we have chosen
the source tree. Each node $u$ contained in the source tree with distance label $d_s(u) = D_s$ is
marked as \emph{active}. In a pass all \emph{active} nodes are processed. If an \emph{active} node
$u$ is adjacent to a node $v$ not contained in any of the two trees, we add $v$ to the source tree
and set $d_s(v) = d_s(u) + 1$. If $v$ is in the sink tree, we have found an augmenting path.
After augmenting along that path some of the nodes are not connected to the tree through a non-saturated
edge. For such a node $v$ the adjacency list is scanned and if an adjacent node $u$ exist with
$d_s(u) = d_s(v) - 1$ and $r_f(u,v) > 0$, the parent of $v$ is set to $u$. If such a node is not found,
we search for an adjacent node $u$ for which $d_s(u)$ is minimal and $r_f(u,v) > 0$. If such
a node is found, we set the parent of $v$ to $u$ and $d_s(v) = d_s(u) + 1$. Otherwise, $v$ is removed
from the source tree. If after a pass a node $v$ exists with $d_s(v) = D_s + 1$, $D_s$ is set to $D_s + 1$,
otherwise the algorithm terminates.

\subsubsection{Push-Relabel Algorithm}
\label{sec:push_relabel}

Goldberg and Tarjan \cite{goldberg1988new} proposed a maximum flow algorithm
not based on finding an augmenting path in the \emph{residual graph}. Instead, the idea is
to maintain a \emph{preflow} during the execution of the algorithm which satisfies the capacity 
constraints, but only a weakened form of the conservation of flow constraint:
\[\forall v \in V \setminus \{s,t\}: \sum_{u \in V} f(v,u) \le \sum_{u \in V} f(u,v)\]
The algorithm maintains a \emph{distance labeling} $d: V \rightarrow \mathbb{N}$ and an 
\emph{excess function} $e_f: V \rightarrow \mathbb{N}$. The \emph{distance labeling} satisfies
the following conditions: $d(s) = |V|$, $d(t) = 0$ and for each $(u,v) \in E_f$, $d(u) \le d(v) + 1$. We say an
residual edge $(u,v)$ is \emph{admissible} if $d(u) = d(v) + 1$. A node $v$ is \emph{active}
if $v \notin \{s,t\}$ and $e_f(v) > 0$.\\
Initially, all \emph{labels} and \emph{excess} values are set to zero except for the source node $s$ will be set to $d(s) = 1$
and $e_f(s) = \infty$. For each \emph{active} node $u$ the algorithm performs two update
operations, called \emph{push} and \emph{relabel}. The first operation pushes flow
over each \emph{admissible} edge $(u,v)$. After a \emph{push} $e_f(u) = e_f(u) - 
\min{(e_f(u),r_f(u,v))}$ and $e_f(v) = e_f(v) + \min{(e_f(u),r_f(u,v))}$. If there is no
\emph{admissible} edge, a \emph{relabel} operation is performed, which replaces $d(u)$ by
$\min_{(u,v) \in E_f} d(v) + 1$. The algorithm terminates, if none of the nodes is \emph{active}.
The worst case complexity of the algorithm is \BigO{n^3}. The running time can be reduced
to \BigO{n^2\log{n}} with \emph{Dynamic Trees} \cite{goldberg1988new, sleator1981data}, but this
implementation is not practical due to a large hidden constant factor.\\
The \emph{push-relabel} algorithm is one of the fastest maximum flow algorithms in practice
because there exist several speed-up techniques. The first one is
the \emph{global relabeling} heuristic which frequently updates the \emph{distance labels} by computing
the shortest path in the residual graph from all nodes to the sink \cite{cherkassky1997implementing}.
This can be done with a backward \emph{BFS} in linear time. This technique is performed periodically,
e.g., after $n$ \emph{relabel} operations. \\
The second heuristic is the \emph{gap heuristic} \cite{cherkassky1994fast,derigs1989implementing}.
If at a particular stage of the algorithm there is no node $u$ with $d(u) = g < n$, then for each node
$v$ with $g < d(v) < n$ the sink is not reachable anymore. Therefore, we can increase the \emph{distance
label} of all those nodes to $n$. To implement this heuristic, the algorithm maintains a linked list of
nodes with distance label $i$.

%%%%%%%%%%%%%%%%%%%%%%%%%%%%%%%%%%%%%%%%%%%%%%%%%%%%%%%%%%%%%%%%%%%%%%

\subsection{Modeling Flows on Hypergraphs}
\label{sec:related_lawler}

Finding a minimum $(s,t)$-cutset of a hypergraph \HypergraphDef~is close related to the problem of finding
a minimum $(s,t)$-vertex separator of the corresponding \emph{bipartite graph} representation $G_*(H)$ (see Section \ref{sec:hypergraph}).
Hu and Moerder \cite{HuMoerder85} introduce node capacities in $G_*(H)$. Each hyperedge
$e$ has a capacity equal to $\omega(e)$ and each hypernode has infinite capacity. 
Further, they show that a minimum-weight $(s,t)$-vertex separator in $G_*(H)$
is equal to a minimum-weight $(s,t)$-cutset of a hypergraph $H$. 
Finding such a separator is a flow problem and can be calculated with the flow network \ShortT{L} 
presented by Lawler \cite{lawler1973}:

\begin{definition}[Lawler Transformation]
Let $T_L$ be the transformation of a hypergraph \HypergraphDef~into 
a flow network \T{L} proposed by Lawler \cite{lawler1973}. \ShortT{L} is defined as follows:
\begin{enumerate}
\item $V_L = V \cup \bigcup\limits_{e \in E}\ \{\incoming{e}, \outgoing{e}\}$
\item $\forall e \in E:$ we add a directed edge $(\incoming{e},\outgoing{e})$ 
      with capacity $\capa_L(\incoming{e},\outgoing{e}) = \omega(e)$
\item $\forall v \in V$ and $\forall e \in I(v):$ we add two directed edges $(v, \incoming{e})$ and 
      $(\outgoing{e}, v)$ with capacity $\capa_L(v, \incoming{e}) = \capa_L(\outgoing{e},v) = \infty$.
\end{enumerate} 
\end{definition}
\begin{figure}
\centering
\includegraphics[width=0.8\textwidth]{../img/network_transformation/lawler_transformation.eps}
\caption{Transformation of a hypergraph into the flow network \ShortT{L} \cite{lawler1973}. The
capacity of the black edges in the flow network is $\infty$.}
\label{img:lawler_transformation}
\end{figure}
An example of this transformation is shown in \autoref{img:lawler_transformation}.
\ShortT{L} is nearly equivalent to the transformation $T_V(G)$ described in Defintion \autoref{def:vertex_seperator_transformation}
except that we do not have to split the hypernodes $v \in V$.
For all $e \in E$ there exist two corresponding nodes $\incoming{e},\outgoing{e} \in V_L$. $\incoming{e}$ 
is called \emph{incoming hyperedge node} and $\outgoing{e}$ is called \emph{outgoing hyperedge node}. \\
A hypernode cannot be in a minimum-capacity $(s,t)$-vertex separator because each $v \in V$ has
infinity capacity \cite{HuMoerder85}. Therefore, a minimum-capacity $(s,t)$-cutset 
of \ShortT{L} is equal to a minimum $(s,t)$-vertex separator of $G_*(H)$.
The resulting graph \ShortT{L} has $|V_L| = |V| + 2|E|$ nodes and $|E_L| = (2\bar{e}+1)|E|$ edges, where
$\bar{e}$ is the average size of a hyperedge \cite{pistorius2003}. Using \emph{Edmond-Karps}
maximum flow algorithm (see Section \ref{sec:aug_path}) on flow network \ShortT{L} 
takes time \BigO{|V|^2|E|^2} \cite{lawler1973}. \\
A minimum-weight $(s,t)$-cutset of $H$ can be found by simply mapping the minimum-capacity
$(s,t)$-cutset to their corresponding hyperedges in $H$ (see Section \ref{sec:applications}). 
The minimum-weight $(s,t)$-bipartition consists of all vertices $v \in V$ \emph{reachable} from $s$ in the 
\emph{residual graph} of \ShortT{L} and all hypernodes not \emph{reachable}
from $s$. \\ 
In \autoref{img:lawler_augmenting_example} we illustrate the structure of \ShortT{L} and demonstrate
what happens after we augment along a path in the Lawler-Network. This figure can be used as
a reference to illustrate the proofs of Section \ref{sec:opt_flow_network}.\\
\begin{figure}[t!]
\centering
\includegraphics[width=0.75\textwidth]{../img/network_transformation/lawler_augmenting_example.eps}
\caption{Illustration of the structure of a hyperedge $e$ in \ShortT{L} and the effect of augmenting
         along a path in the network. The labeling on an edge represents the flow $f(x,y)$ and
         the capacity $\capacity{x,y}$ denoted with $f(x,y)/\capacity{x,y}$. The labeling of the
         edges in the residual graph corresponds to the residual capacity $r_f(x,y)$. The red highlighted
         path represents an augmenting path.}
\label{img:lawler_augmenting_example} 
\end{figure}

%%%%%%%%%%%%%%%%%%%%%%%%%%%%%%%%%%%%%%%%%%%%%%%%%%%%%%%%%%%%%%%%%%%%%%

\subsection{Flow-based Refinement for Graph Partitioning}
\label{sec:flow_local_search_graph} 

It seems natural to utilize maximum flow computations to improve the cut metric of a 
given partition of a graph. Lang and Rao \cite{lang2004flow} use an approach,
called \emph{Max-Flow Quotient-cut Improvement} (MQI) to improve the quality
of a graph partition when metrics such as \emph{expansion} or \emph{conductance}
are used. For a given bipartition $(S,\bar{S})$, they find the best 
improvement among all bipartitions $(S',\bar{S'})$ such that $S' \subset S$
by solving a flow problem. Andersen and Lang \cite{andersen2008algorithm}
suggested a flow-based improvement algorithm, called \emph{Improve},
which works similar as MQI, but does not restrict the output of the 
partition to $S' \subset S$. However, both techniques can not guarantee 
that the resulting bipartition is balanced and are only applicable for $k=2$. \\
Schulz and Sanders \cite{sanders2011engineering} integrate a flow-based refinement algorithm 
in the \emph{multilevel graph partitioner} \emph{KaFFPa}. Their basic idea is
to extract a region $B$ around the cut of the graph and connect the \emph{border} 
of $B$ with the source resp. sink. $B$ is defined in such a way that the flow computation
yields a feasible cut according to the \emph{balance contraint}. Many ideas of this work are used in this
thesis and adapted to hypergraphs. Therefore we will give a detailed description
of the concepts and advanced techniques to improve graph partitions.

\subsubsection{Balanced Bipartitioning}
\label{sec:balanced_bipartitioning}
\begin{figure}
\centering
\includegraphics[width=0.6\textwidth]{../img/flow_local_search/balanced_bipartitioning.eps}
\caption{Configuration of a flow problem around the cut of graph $G$ \cite{sanders2011engineering}.}
\label{img:balanced_bipartition}
\end{figure}
Let $(V_1,V_2)$ be a balanced bipartition of a graph $G = (V,E,c,\omega)$. 
Further, let $P(v) = 1$, if $v \in V_1$ and
$P(v) = 2$ otherwise. We now explain how a given bipartition 
can be improved with flow computations. The technique can also be applied on a $k$-way 
partition by applying the approach on two adjacent blocks \cite{sanders2011engineering}. \\
Let $\delta := \{ u\ |\ \exists (u,v) \in E: P(u) \neq P(v) \}$ be the set of nodes
around the cut of $G$. For a set $B \subseteq V$ we define its border 
$\delta B := \{u \in B\ |\ \exists (u,v) \in E: v \notin B\}$.
The basic idea is to build a region $B$ around $\delta$
and connect all nodes in $\delta B \cap V_1$ to the source node $s$ and all nodes in 
$\delta B \cap V_2$ to the sink node $t$. \\
We can construct $B := B_1 \cup B_2$ with two \emph{Breadth First Searches} (\BFS). 
One is initialized with all nodes in $\delta \cap V_1$ and stops if $c(B_1)$ would 
exceed $(1+\epsilon) \lceil \frac{c(V)}{2} \rceil - c(V_2)$. The second is initialized with 
all nodes in $\delta \cap V_2$ and stops if $c(B_2)$ would exceed 
$(1+\epsilon) \lceil \frac{c(V)}{2} \rceil - c(V_1)$. The two \BFS s only touch nodes of $V_1$ resp. $V_2$
such that $B_1 \subseteq V_1$ and $B_2 \subseteq V_2$. The constraints for the weights of $B_1$
and $B_2$ guarantee that the bipartition is still balanced after a \emph{Max-Flow-Min-Cut}
computation. Connecting $s$ resp. $t$ to all border nodes $\delta B \cap V_1$ resp.
$\delta B \cap V_2$ ensures that a non-cut edge not contained in $G_B$ is not a cut edge after
assigning the minimum $(s,t)$-bipartition of subgraph $G_B$ to $G$. Consequently,
each minimum $(s,t)$-cutset in $G_B$ leads to a cut smaller or 
equal to the old cut of $G$. All concepts are illustrated in \autoref{img:balanced_bipartition}.


\subsubsection{Adaptive Flow Iterations}
\label{sec:adaptive_flow_iterations}

Sanders and Schulz \cite{sanders2011engineering} introduce several techniques to improve
this basic approach. If the \emph{Max-Flow-Min-Cut} computation on $G_B$ leads to an
improved cut, we can apply the method again. 
An extension of this approach is to iteratively adapt the
size of the flow problem based on the result of the maximum flow computation.
We define $\epsilon' := \alpha\epsilon$ for a $\alpha \ge 1$ and let the size of $B$ depend
on $\epsilon'$ rather than on $\epsilon$. If we find an improvement, we
increase $\alpha$ to $\min\{2\alpha, \alpha'\}$ where $\alpha'$ is a predefined upper bound
for $\alpha$. If not, we decrease the size of $\alpha$ to 
$\max\{\frac{\alpha}{2},1\}$. The approach is called
\emph{adaptive flow iterations} \cite{sanders2011engineering}.

\subsubsection{Most Balanced Minimum Cut}
\label{sec:related_mbmc}

Picard and Queyranne \cite{picard1980structure} show that all minimum $(s,t)$-cutsets 
are computable with one maximum $(s,t)$-flow computation.
An important concept used by them is the definition
of a \emph{closed node set} $C \subseteq V$ of a graph $G$.

\begin{definition}[Closed Node Set]
Let $G = (V,E)$ be a graph and $C \subseteq V$. $C$ is called a closed node set iff the 
condition $u \in C$ implies that for all edges $(u,v) \in E$ also $v \in C$.
\end{definition}

A \emph{closed node set} is illustrated in \autoref{img:closed_node_set}. A simple observation
is that all nodes on a cycle have to be in the same \emph{closed node set} per definition. Therefore
we can contract all \emph{Strongly Connected Components} (SCC) of $G$ with a linear time algorithm
proposed by Tarjan \cite{tarjan1972depth} and sweep over the contracted graph in reverse
topological order to enumerate all \emph{closed node sets}. If
we contract all SCCs of $G$ the resulting graph is a \emph{directed acyclic graph} (DAG). Therefore, 
a topological order exists. \\
With the Theorem of Picard and Queyranne \cite{picard1980structure} we can enumerate
all minimum $(s,t)$-cuts of $G$ with one maximum flow computation.

\begin{theorem}
\label{theorem:mbmc}
There is a $1$-$1$ correspondence between the minimum $(s,t)$-cuts of a graph and the closed node
sets containing $s$ in the residual graph of a maximum $(s,t)$-flow.
\end{theorem}

All \emph{closed node sets} in the residual graph of $G$ induce a minimum $(s,t)$-cutset on $G$.
They can be calculated with the algorithm described above using the residual graph of
$G$ as input. The running time of the algorithm is \BigO{|V| + |E|}. \\
A common problem of the \emph{adaptive flow iteration} approach (see Section \ref{sec:adaptive_flow_iterations}) is
that using a large $\alpha$ often leads to cuts in $G$ that violate the balanced constraint. 
We can enumerate all minimum $(s,t)$-cutsets with one maximum 
flow computation and therefore have a higher probability to find
a feasible partition after a \emph{Max-Flow-Min-Cut} computation. We refer to this method as
\emph{Most Balanced Minimum Cut} \cite{sanders2011engineering}.
\begin{figure}
\centering
\includegraphics[width=0.9\textwidth]{../img/flow_local_search/closed_node_set.eps}
\caption{$C = \{s,a,b,c\}$ is a \emph{closed node set} of graph $G$ (left side).
         After contracting all \emph{Strongly Connected Components}, we can enumerate all
         \emph{closed node sets} of $G$ by sweeping over the contracted graph 
         in reverse topological order (right side).}
\label{img:closed_node_set}
\end{figure}


\subsubsection{Active Block Scheduling}
\label{sec:abs}
\emph{Active Block Scheduling} is a \emph{quotient graph style refinement} technique for
$k$-way partitions \cite{holtgrewe2010engineering,sanders2011engineering}. 
The algorithm is organized in rounds and executes a two-way 
local improvement algorithm on each pair of adjacent
blocks in the \emph{quotient graph} where at least one of both is \emph{active}. 
Initially all blocks are \emph{active}. A block becomes \emph{inactive}
if none of its nodes move in a round. The algorithm
terminates, if all blocks are \emph{inactive}. \\
Fiduccia and Mattheyses \cite{fiduccia1988linear} introduce a linear time
two-way local search heuristic, called \emph{FM} heuristic, 
which is fundamental for many graph partitioning algorithms.
They define the gain $g(v)$ of a node $v \in V$ as the reduction of the cut metric when
moving $v$ from its current block to the other block. By maintaining the gains of the
nodes in a special data structure, called \emph{bucket queue}, they can find a maximum
gain node in constant time. After moving a maximum gain node, they are also able to update the
data structure in time equal to the number of adjacent nodes.\\
The local improvement algorithm (for \emph{Active Block Scheduling}) can either 
be an \emph{FM} local search or a flow-based approach or even a combination of 
both as proposed by Sanders and Schulz \cite{sanders2011engineering}.

%%%%%%%%%%%%%%%%%%%%%%%%%%%%%%%%%%%%%%%%%%%%%%%%%%%%%%%%%%%%%%%%%%%%%%

\subsection{Hypergraph Partitioning}

In this section, we review how most hypergraph partitioners solve the \emph{hypergraph
partitioning problem}. The most successful
approach is the \emph{multilevel paradigm} \cite{alpert1995recent,bader2013graph,
papa2007hypergraph} which we describe in Section \ref{sec:multilevel_paradigm}.
The algorithms presented in this thesis are integrated into $n$-level hypergraph partitioner \emph{KaHyPar}. Therefore, we
give a brief overview of implementation details of this framework in Section \ref{sec:kahypar}.

\subsubsection{Multilevel Paradigm}
\label{sec:multilevel_paradigm}

The \emph{multilevel paradigm} is a three phase algorithm to solve the \emph{hypergraph 
partitioning problem} (see \autoref{img:multilevel_paradigm}). In the first stage, called
\emph{coarsening phase}, vertex matchings or clusterings are calculated which are contracted. This process is
repeated until a predefined number of hypernodes remains. The sequence of successively
smaller hypergraphs is called \emph{levels}. If the hypergraph $H$ is small enough, expensive 
algorithms can be used to initially partition $H$ into $k$ blocks (\emph{Initial Partitioning}). Afterwards, we
\emph{uncontract} each \emph{level} in reverse order of \emph{contraction} by projecting
the partition to the next \emph{level}. After \emph{uncontraction} a \emph{refinement} heuristic can be
used to improve the quality of the current partition according
to an objective function. The most commonly used \emph{refinement} algorithm is the \emph{FM}
algorithm \cite{fiduccia1988linear}.

\begin{figure}
\centering
\includegraphics[width=1.0\textwidth]{../img/hypergraph_partitioning/multilevel.eps}
\caption{Multilevel Hypergraph Partitioning}
\label{img:multilevel_paradigm}
\end{figure}


\subsubsection{$n$-Level Hypergraph Partitioning}
\label{sec:kahypar}

\emph{KaHyPar} is a multilevel hypergraph partitioner in its most extreme version, which
removes only a single vertex in one \emph{level} of the hierarchy. It seems to be the method
of choice for optimizing cut- and the $(\lambda - 1)$-metric unless speed is more important than
quality \cite{heuer2017improving}. The framework provides a \emph{direct $k$-way} \cite{akhremtsev2017engineering} 
and a \emph{recursive bisection} mode, which recursively calculates bipartitions 
(with multilevel paradigm) until the hypergraph is divided into $k$ blocks 
\cite{schlag2016k}. \emph{KaHyPar} consists of four phases: \emph{Preprocessing}
and the three phases of the \emph{multilevel paradigm}. \\
In the \emph{preprocessing} step community structures of the hypergraph are detected. The
hypergraph is transformed into a bipartite graph $G_*(H)$ (see Section \ref{sec:hypergraph}) and
a community detection algorithm is executed which optimizes \emph{modularity} \cite{fortunato2010community,
heuer2017improving}. During the \emph{coarsening phase} contractions are restricted to vertices within the same 
community. The contraction partners are chosen according to the \emph{heavy-edge} rating function
$r(u,v) := \sum_{e \in I(u) \cap I(v)} \frac{\omega(e)}{|e|-1}$ \cite{karypis1999multilevel}. The
function prefers vertices which share a large number of heavy nets with small size. The contraction
algorithm works in passes. At the beginning of each pass a random permutation of the vertices 
is generated and for each vertex $u$, the contraction partner $v$ is determined according
to the \emph{heavy-edge} rating function \cite{schlag2016k}. A pass ends if each vertex was involved
in a contraction. The passes are repeated until only
$t = 160k$ hypernodes remains.
The \emph{initial partitioning} phase uses the \emph{recursive bisection} approach to calculate
a $k$-way partition in combination with a portfolio of initial partitioning techniques 
\cite{heuer2015engineering}. In the \emph{refinement phase}, a localized \emph{FM} search is started \cite{fiduccia1988linear},
initialized with the current uncontracted vertices. The \emph{local search} maintains $k$ \emph{priority
queues} (PQ) for each block $V_i$ exactly one \cite{akhremtsev2017engineering}. After a move,
the gains of all adjacent hypernodes are updated with a \emph{delta-gain} update strategy \cite{papa2007hypergraph}.
The recalculation of all gain values at the beginning of a \emph{FM} pass is one of the main bottlenecks
of the algorithm \cite{papa2007hypergraph}. Therefore, Schlag et al. \cite{akhremtsev2017engineering,
schlag2016k} introduce a \emph{gain cache}, which prevents
expensive recalculations of the corresponding gain function. The \emph{gain cache} is maintained
with \emph{delta-gain} updates in the same way as the \emph{PQs}. Further, the \emph{local search}
is stopped as soon as an improvement during an \emph{FM} pass becomes unlikely. This model is
called \emph{adaptive stopping rule} \cite{akhremtsev2017engineering}.
