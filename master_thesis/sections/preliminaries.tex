\subsection{Definitions and Terminology}\label{sec:prelim}
\subparagraph*{Notation and Definitions.}
An \textit{undirected hypergraph} $H=(V,E,c,\omega)$ is defined as a set of $n$ vertices $V$ and a
set of $m$ hyperedges/nets $E$ with vertex weights $c:V \rightarrow \mathbb{R}_{>0}$ and net 
weights $\omega:E \rightarrow \mathbb{R}_{>0}$, where each net is a subset of the vertex set $V$ (i.e., $e \subseteq V$). The vertices of a net are called \emph{pins}.
We use $P$ to denote the multiset of all pins in $H$.
We extend $c$ and $\omega$ to sets, i.e., $c(U) :=\sum_{v\in U} c(v)$ and $\omega(F) :=\sum_{e \in F} \omega(e)$.
A vertex $v$ is \textit{incident} to a net $e$ if $v \in e$. $\mathrm{I}(v)$ denotes the set of all incident nets of $v$. 
The \textit{degree} of a vertex $v$ is $d(v) := |\mathrm{I}(v)|$. 
The set $\Gamma(v) := \{ u~|~\exists~e \in E : \{v,u\} \subseteq e\}$ denotes the neighbors of $v$.
The \textit{size} $|e|$ of a net $e$ is the number of its pins. Nets of size one are called \emph{single-vertex} nets.
A \emph{$k$-way partition} of a hypergraph $H$ is a partition of its vertex set into $k$ \emph{blocks} $\mathrm{\Pi} = \{V_1, \dots, V_k\}$ 
such that $\bigcup_{i=1}^k V_i = V$, $V_i \neq \emptyset $ for $1 \leq i \leq k$ and $V_i \cap V_j = \emptyset$ for $i \neq j$.
We call a $k$-way partition $\mathrm{\Pi}$ \emph{$\mathrm{\varepsilon}$-balanced} if each block $V_i \in \mathrm{\Pi}$ satisfies the \emph{balance constraint}:
$c(V_i) \leq L_{\max} := (1+\varepsilon)\lceil \frac{c(V)}{k} \rceil$ for some parameter $\mathrm{\varepsilon}$. 
Given a $k$-way partition $\mathrm{\Pi}$, the number of pins of a net $e$ in block $V_i$ is defined as
$\mathrm{\Phi}(e,V_i) := |\{v \in V_i~|~v \in e \}|$. 
For each net $e$, $\mathrm{\Lambda}(e) := \{V_i~|~ \mathrm{\Phi}(e, V_i) > 0\}$ denotes the \emph{connectivity set} of $e$.
The \emph{connectivity} of a net $e$ is the cardinality of its connectivity set: $\mathrm{\lambda}(e) := |\mathrm{\Lambda}(e)|$.
A net is called \emph{cut net} if $\mathrm{\lambda}(e) > 1$.
The \emph{$k$-way hypergraph partitioning problem} is to find an $\varepsilon$-balanced $k$-way partition $\mathrm{\Pi}$ of a hypergraph $H$ that
minimizes an objective function over the cut nets for some $\varepsilon$.
Several objective functions exist in the literature~\cite{Alpert19951,Lengauer:1990}.
The most commonly used cost functions are the \emph{cut-net} metric $\text{cut}(\mathrm{\Pi}) := \sum_{e \in E'} \omega(e)$ and the
\emph{connectivity} metric $(\mathrm{\lambda} - 1)(\mathrm{\Pi}) := \sum_{e\in E'} (\mathrm{\lambda}(e) -1)~\omega(e)$, where $E'$ is the set of all cut nets~\cite{donath1988logic}.
In this paper, we use the connectivity-metric, which accurately models the total communication volume of parallel sparse matrix-vector multiplication~\cite{PaToH}.
Optimizing both objective functions is known to be NP-hard \cite{Lengauer:1990}.
\emph{Contracting} a pair of vertices $(u, v)$ means merging $v$ into $u$.
The weight of $u$ becomes $c(u) := c(u) + c(v)$.  We connect $u$ to the former neighbors $\Gamma(v)$ of $v$ by replacing 
$v$ with $u$ in all nets $e \in \mathrm{I}(v) \setminus \mathrm{I}(u)$ and remove $v$ from all nets $e \in \mathrm{I}(u) \cap \mathrm{I}(v)$.
\emph{Uncontracting} a vertex $u$ reverses the contraction.
The two most common ways to represent a hypergraph $H=(V,E,c,\omega)$ as an undirected graph are the \textit{clique} and the \textit{bipartite} representation~\cite{HuMoerder85}.
In the following, we use \emph{nodes} and \emph{edges} when referring to a graph representation and \emph{vertices}
and \emph{nets} when referring to $H$.
In the \textit{clique} graph $G_x(V,E_x \subseteq V^2)$ of $H$, each net is replaced with an edge for each
pair of vertices in the net: $E_x := \{(u,v) : u,v \in e, e \in E\}$. Thus the pins of a net $e$ with size $|e|$ form a $|e|$-clique in $G_x$.
In the \textit{bipartite} graph $G_*(V \dot\cup E, F)$ the vertices and nets of $H$ form the node set and for each net $e$ incident to a vertex $v$, we
add an edge $(e,v)$ to $G_*$. The edge set $F$ is thus defined as $F := \{(e,v)~|~e \in E, v \in e \}$. Each net in $E$ therefore corresponds to a star in $G_*$.
In both models, node weights $c$ and edge weights $\omega$ are chosen according to the problem domain~\cite{Hadley95}.
