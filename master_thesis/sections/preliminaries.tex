\subsection{Graphs}

\begin{definition}
A directed weighted graph $G = (V,E,c,\omega)$ is a set of nodes $V$ 
and a set of edges $E$ with a node weight function 
$c: V \rightarrow \mathbb{R}_{\ge 0}$ and an edge weight 
function $\omega: E \rightarrow \mathbb{R}_{\ge 0}$. A edge $e = (u,v)$ 
is a relation between two nodes $u,v \in V$.
\label{def:hypergraph}
\end{definition}

Two vertices $u$ and $v$ are \emph{adjacent}, if there exists an edge
$(u,v) \in E$. Two edges $e_1$ and $e_2$ are \emph{incident} to each other, if they
share a node. $I(v)$ denotes the set of all \emph{adjacent} nodes of 
$v$. The \emph{degree} of a node $v$ is $d(v) = |I(v)|$.

\begin{definition}
Given a directed graph $G = (V,E)$. A contraction of two nodes
$u$ and $v$ results in a new graph $G_{(u,v)} = (V\setminus\{v\},E')$, where
each edge of the form $(v,w)$ or $(w,v)$ in $E$ is replaced with an edge 
$(u,w)$ or $(w,u)$ in $E'$.
\label{def:contraction}
\end{definition}

A \emph{path} $P = (v_1,\ldots,v_k)$ is a sequence of nodes, where for
each $i \in [1,k-1]: (v_i,v_{i+1}) \in E$. A \emph{cycle} is a \emph{path}
$P = (v_1,\ldots,v_k)$ with $v_1 = v_k$. A \emph{strongly connected 
component} $C \subseteq V$ is a set of nodes where for each $u,v \in C$
exists a \emph{path} from $u$ to $v$. We can enumerate all \emph{strongly
connected components} (\emph{SCC}) in a directed graph $G$ with a linear time algorithm
proposed by Tarjan \cite{tarjan1972depth}. A directed graph $G$ without
any \emph{cycles} is called \emph{directed acyclic graph} (\emph{DAG}). On such
graphs we can define a \emph{topological order} $\gamma: V \rightarrow \mathbb{N}_+$ such
that for each $(u,v) \in E: \gamma(u) < \gamma(v)$. A \emph{topological order}
of a \emph{DAG} can be found in linear time with Kahn's algorithm \cite{kahn1962topological}.
We can transform a general directed graph $G$ into a \emph{DAG}, if we contract
each \emph{strongly connected component}. All concepts are illustrated in 
\autoref{img:scctoposort}.

\begin{figure}
\centering
\includegraphics[width=0.9\textwidth]{../img/preliminaries/scctoposort.eps}
\caption{Example of \emph{strongly connected components} of a directed graph and
         a \emph{topological order} on a \emph{directed acyclic graph}. Each 
         \emph{SCC} is marked in the same color.} 
\label{img:scctoposort}
\end{figure}

\begin{definition}
Let $G_{V'} = (V',E_{V'},c,\omega)$ be the subgraph of a graph $G$
induced by $V' \subseteq V$ with $E_{V'} = \{(u,v) \in E\ |\ u,v \in V'\}$.
\label{def:subgraph}
\end{definition}


\subsection{Flows and Applications}
\label{sec:applications}

Given a graph $G = (V,E,c)$ with capacity function $c: E \rightarrow \mathbb{R}_+$ and a source 
$s \in V$ and a sink $t \in V$. The maximum flow problem is about finding 
the maximum amount of flow from $s$ to $t$ in $G$. A flow is a function 
$f: E \rightarrow \mathbb{R}_+$, which have to satisfy the following constraints:
\begin{enumerate}
\item $\forall (u,v) \in E: f(u,v) \le c(u,v)$ (capacity constraint)
\item $\forall v \in V \setminus \{s,t\}: \sum_{(u,v) \in E} f(u,v) = \sum_{(v,u) \in E} f(v,u)$ (conservation of flow constraint)
\end{enumerate}
The capacity constraint restricts the flow on an edge $(u,v)$ by its capacity 
$c(u,v)$. Whereas the conservation of flow constraint ensures that the amount
of flow entering a node $v \in V \setminus \{s,t\}$ is the same as leaving a node.
The value of the flow is defined as $|f| = \sum_{(s,v) \in E} f(s,v) = \sum_{(v,t) \in E} f(v,t)$.
A flow $f$ is maximal, if there exists no other flow $f'$ with $|f'| > |f|$. \\
Another useful construct in connection with maximum flows, is the concept of the
\emph{residual graph} $G_f$ and the \emph{residual capacity} $r_f$ of a flow function $f$ on graph $G$.
The \emph{residual capacity} $r_f: V \times V \rightarrow \mathbb{R}_+$ is defined as follows:
\begin{enumerate}
\item $\forall (u,v) \in E: r_f(u,v) = c(u,v) - f(u,v)$
\item $\forall (u,v) \in E:$ If $f(u,v) > 0$ and $c(v,u) = 0$, then $r_f(v,u) = f(u,v)$
\end{enumerate}
For a edge $e = (u,v) \in E$ the residual capacity $r_f(u,v)$ is the remaining amount of 
flow which can be send over edge $e$. For each reverse edge $\overleftarrow{e} \notin E$ the
residual capacity $r_f(\overleftarrow{e})$ is the amount of flow which is send over $e$.
The \emph{residual graph} $G_f = (V,E_f,r_f)$ is the network containing all $(u,v) \in V \times V$
with $r_f(u,v) > 0$. More formal $E_f = \{(u,v)\ |\ r_f(u,v) > 0, (u,v) \in V \times V\}$.
\autoref{img:maximum_flow_example}~illustrates all presented concepts.

\begin{figure}
\centering
\includegraphics[width=0.9\textwidth]{../img/maximum_flow/maximum_flow_example.eps}
\caption{Figure illustrates concepts related to the maximum flow problem. A valid flow $f$ 
(red values) from $s$ to $t$ on a graph $G$ is shown on the left side. The corresponding
\emph{residual graph} $G_f$ with its \emph{residual capacities} (black values) 
is illustrated on the right side. The red highlighted path represents an \emph{augmenting path}
in $G$.}
\label{img:maximum_flow_example}
\end{figure}

The \emph{Max-Flow-Min-Cut}-Theorem is fundamental for many applications related to the maximum
flow problem \cite{ford1956maximal}.

\begin{theorem}
The value of a maximum $(s,t)$-flow obtainable in a graph $G$ is equal with the weight
of the minimum cutset in $G$ seperating $s$ and $t$.
\end{theorem}

Let $f$ be a maximum flow in a graph $G = (V,E,\omega)$ with $s \in V$ and $t \in V$. 
Further, let $A$ be the set containing all $v \in V$, which are \emph{reachable} from $s$
in $G_f$. A node $v$ is \emph{reachable} from a node $u$, if there exists a path from $u$
to $v$. Then the set of all cut edges between the bipartition $(A,V\setminus A)$ 
is a minimum-weight $(s,t)$-cutset \cite{ford2015flows}. $A$ can be calculated with a simple \BFS~in $G_f$ starting
from $s$. \\
From this analogy many solutions for related problems arose. Samples are listed below:
\begin{enumerate}
\item Maximum Bipartite-Matching
\item Minimum-Weight Vertex Seperator
\item Number of Edge-Disjoint Paths
\item Number of Vertex-Disjoint Paths
\end{enumerate}
Solutions for those problems sometimes involves a transformation $T$ of the graph $G$
into a flow network $T(G)$, such that the \emph{Max-Flow-Min-Cut}-Theorem is applicable. 
A problem important for this work is to find a minimum-weight $(s,t)$-vertex seperator
in a graph $G = (V,E,c)$ with $c: V \rightarrow \mathbb{R}_+$.

\begin{definition}
Let $G = (V,E,c)$ be a graph with $c: V \rightarrow \mathbb{R}_+$. $S \subseteq V$
is a vertex seperator for non-adjacent verticies $s \in V$ and $t \in V$ if the
removal of $S$ from graph $G$ seperates $s$ and $t$ ($s$ not \emph{reachable} from $t$).
A vertex seperator $S$ is a minimum-weight $(s,t)$-vertex seperator, if for all $S' \subseteq V$
$c(S) \le c(S')$.
\end{definition}

We can calculate a minimum-weight $(s,t)$-vertex seperator with a maximum flow
calculation in the following flow network (\todo{reference}):

\begin{definition}
\label{def:vertex_seperator_transformation}
Let $T_V$ be a transformation of a graph $G = (V,E,c)$ into 
a flow network $T_V(G) = (V_V, E_V, c_V)$ (with $c_V: E_V \rightarrow \mathbb{R}_+$). 
$T_V$ is defined as follows:
\begin{enumerate}
\item $V_V = \bigcup\limits_{v \in V}\ \{v', v''\}$
\item $\forall v \in V$ we add a directed edge $(v',v'')$
      with capacity $c_V(v',v'') = c(v)$
\item $\forall (u,v) \in E$ we add two directed edges $(u'', v')$ and 
      $(v'', u')$ with capacity $c_V(u'', v') = c_V(v'', u') = \infty$.
\end{enumerate} 
\end{definition}

The vertex seperator problem and transformation $T_V(G)$ are illustrated in \autoref{img:vertex_seperator_example}.
Obviously no edge between two adjacent nodes can be in a minimum-capacity $(s,t)$-cutset of $T_V(G)$,
because for all those edges the capacity is $\infty$. Therefore, the cutset must consist
of edges of the form $(v',v'')$. A minimum-weight $(s,t)$-vertex seperator can be calculated by
finding a maximum flow in $T_V(G)$, finding the minimum-capacity $(s,t)$-cutset with the procedure
described above and then map each cut edge $(v',v'')$ to their corresponding node $v$.\\
Given a set of sources $S$ and sinks $T$. The \emph{multi-source multi-sink} maximum flow problem is
about finding a maximum flow $f$ from all source nodes $s \in S$ to all sink nodes $t \in T$.
We can transform such a problem into a \emph{single-source single-sink} problem by adding
two additional nodes $s$ and $t$. We add a directed edge from $s$ to all source nodes $s' \in S$ 
and for all sink nodes $t' \in T$ a directed edge to $t$ with capacity 
$c(s,s') = c(t',t) = \infty$.

\begin{figure}
\centering
\includegraphics[width=0.9\textwidth]{../img/maximum_flow/vertex_seperator_example.eps}
\caption{ Illustration of the vertex seperator problem and the transformation $T_V(G)$ in which
          we can find a minimum vertex seperator with maximum flow computation. }
\label{img:vertex_seperator_example}
\end{figure}

\subsection{Hypergraphs}
\label{sec:hypergraph}

\begin{definition}
An undirected weighted hypergraph $H = (V,E,c,\omega)$ is a set of hypernodes $V$ 
and a set of hyperedges $E$ with a hypernode weight function 
$c: V \rightarrow \mathbb{R}_{\ge 0}$ and a hyperedge weight 
function $\omega: E \rightarrow \mathbb{R}_{\ge 0}$. A hyperedge $e$ 
is a subset of $V$ (formally: $\forall e \in E: e \subseteq V$).
\label{def:hypergraph}
\end{definition}

A hypergraph generalizes a graph by extending the definition of an edge, which 
can contain more than two nodes. Hyperedges are also called \emph{nets} and the hypernodes
of a net are called \emph{pins}. For a subset $V' \subseteq V$ and $E' \subseteq E$ we
define
\begin{align*}
c(V') = \sum_{v \in V'} c(v) \\
\omega(E') = \sum_{e \in E'} \omega(e)
\end{align*}

A vertex $v$ is \emph{incident} to a hyperedge $e$, if $v \in e$.
Two vertices $u$ and $v$ are \emph{adjacent}, if there exists an 
$e \in E$ such that $u,v \in e$. $I(v)$ denotes the set of all 
\emph{incident} nets of $v$. The \emph{degree} of a hypernode 
$v$ is $d(v) = |I(v)|$. The size of a net $e$ is the cardinality $|e|$.

\begin{definition}
Let $H_{V'} = (V',E_{V'},c,\omega)$ be the subhypergraph of a hypergraph $H$
induced by $V' \subseteq V$ with $E_{V'} = \{e \cap V'\ |\ e \in E: e 
\cap V' \neq \emptyset\}$.
\label{def:subhypergraph}
\end{definition}

A hypergraph $H = (V,E,c,\omega)$ can be represented as an undirected graph. 
There are two common transformations, called \emph{clique} and \emph{bipartite} 
representation \cite{HuMoerder85}. The \emph{clique} graph $G_x(H) = (V,E_x)$ models
each net $e$ as a clique between its pins. The \emph{bipartite} graph $G_*(H) = 
(V \cup E, E_*)$ contains all hypernodes and hyperedges as nodes and connects each
net $e$ with an undirected edge $\{e,v\}$ to all its pins $v \in e$. The two transformations
are illustrated in \autoref{img:hypergraph_transformation}.

\begin{figure}
\centering
\includegraphics[width=1.0\textwidth]{../img/preliminaries/hypergraph_transformation.eps}
\caption{Example of a hypergraph $H$ and its two corresponding graph representations.} 
\label{img:hypergraph_transformation}
\end{figure}

\subsection{Hypergraph Partitioning}
\label{sec:hypergraph_partitioning}

\begin{definition}
A $k$-way partition of a hypergraph $H$ is a partition of its hypernodes into
$k$ disjoint blocks $\Pi = \{V_1,\ldots,V_k\}$ such that $\bigcup_{i=1}^{k} V_i = V$
and $V_i \neq \emptyset$.
\label{def:kway_partition}
\end{definition}

For a $k$-way partition $\Pi = \{V_1,\ldots,V_k\}$, we define the \emph{connectivity set} of a
hyperedge $e$ with $\Lambda(e,\Pi) = \{V_i \in \Pi\ |\ V_i \cap e \neq \emptyset\}$. The \emph{connectivity}
of a net $e$ is $\lambda(e,\Pi) = |\Lambda(e,\Pi)|$. A hyperedge $e$ is \emph{cut}, if
$\lambda(e,\Pi) > 1$. $E(\Pi) = \{e\ |\ \lambda(e,\Pi) > 1\}$ is the set of all \emph{cut} 
nets. We say two blocks $V_i$ and $V_j$ are adjacent, if there exists a hyperedge
$e$ with $V_i,V_j \in \Lambda(e,\Pi)$.

\begin{definition}
For a $k$-way partition $\Pi = \{V_1,\ldots,V_k\}$ of a hypergraph $H$ 
the quotient graph $Q = (\Pi,E')$ is a undirected graph containing an 
edge between each pair of adjacent blocks of $\Pi$.
More formal, $E' = \{(V_i,V_j)\ |\ \exists e \in E: V_i,V_j \in \Lambda(e,\Pi)\}$
\label{def:quotient_graph}
\end{definition}

We say a $k$-way partition is $\epsilon$-balanced, if each block 
$V_i \in \Pi$ satisfies the \emph{balance constraint} 
$c(V_i) \le (1+\epsilon)\lceil\frac{c(V)}{k}\rceil$.

\begin{definition}
The $k$-way hypergraph partitioning problem is to find an $\epsilon$-balanced $k$-way
partition $\Pi$ of a hypergraph $H$ such that a certain objective function is minimized.
\label{def:kway_partitioning_problem}
\end{definition}

There exists several objective function in the hypergraph partitioning context,
which should either be minimized or maximized. The most popular objective function 
is the $\text{cut}$ metric (especially for \emph{graph partitioning}), which is defined as
\[\omega_H(\Pi) = \sum_{e \in E(\Pi)} \omega(e)\]
The goal is to minimize the sum of all \emph{cut} hyperedges. Another important metric
for this work is the $(\lambda - 1)$-metric or \emph{connectivity} metric, 
which is defined as
\[(\lambda - 1)_H(\Pi) = \sum_{e \in E} (\lambda(e) - 1)\omega(e)\]
The idea behind this function is to minimize the \emph{connectivity} of all hyperedges.