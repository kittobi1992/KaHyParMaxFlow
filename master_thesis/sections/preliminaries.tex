\subsection{Graphs}

\begin{definition}
A directed weighted graph $G = (V,E,c,\omega)$ is a set of nodes $V$ 
and a set of edges $E$ with a node weight function 
$c: V \rightarrow \mathbb{R}_{\ge 0}$ and an edge weight 
function $\omega: E \rightarrow \mathbb{R}_{\ge 0}$. A edge $e = (u,v)$ 
is a relation between two nodes $u,v \in V$.
\label{def:hypergraph}
\end{definition}

Two vertices $u$ and $v$ are \emph{adjacent}, if there exists an edge
$(u,v) \in E$. Two edges $e_1$ and $e_2$ are \emph{incident} to each other, if they
share a node. $I(v)$ denotes the set of all \emph{adjacent} nodes of 
$v$. The \emph{degree} of a node $v$ is $d(v) = |I(v)|$.

\begin{definition}
Given a directed graph $G = (V,E)$. A contraction of two nodes
$u$ and $v$ results in a new graph $G_{(u,v)} = (V\setminus\{v\},E')$, where
each edge of the form $(v,w)$ or $(w,v)$ in $E$ is replaced with an edge 
$(u,w)$ or $(w,u)$ in $E'$.
\label{def:contraction}
\end{definition}

A \emph{path} $P = (v_1,\ldots,v_k)$ is a sequence of nodes, where for
each $i \in [1,k-1]: (v_i,v_{i+1}) \in E$. A \emph{cycle} is \emph{path}
$P = (v_1,\ldots,v_k)$ with $v_1 = v_k$. A \emph{strongly connected 
component} $C \subseteq V$ is a set of nodes where for each $u,v \in C$
exists a \emph{path} from $u$ to $v$. We can enumerate all \emph{strongly
connected components} (\emph{SCC}) in a directed graph $G$ with a linear time algorithm
proposed by Tarjan \cite{tarjan1972depth}. A directed graph $G$ without
any \emph{cycles} is called \emph{directed acyclic graph} (\emph{DAG}). On such
graphs we can define a \emph{topological order} $\gamma: V \rightarrow \mathbb{N}_+$ such
that for each $(u,v) \in E: \gamma(u) < \gamma(v)$. A \emph{topological order}
of a \emph{DAG} can be found in linear time with Kahn's algorithm \cite{kahn1962topological}.
We can transform a general directed graph $G$ into a \emph{DAG}, if we contract
each \emph{strongly connected component}. All concepts are illustrated in 
\autoref{img:scctoposort}.

\begin{figure}
\centering
\includegraphics[width=0.9\textwidth]{../img/preliminaries/scctoposort.eps}
\caption{Example of \emph{strongly connected components} of a directed graph and
         a \emph{topological order} on a \emph{directed acyclic graph}. Each 
         \emph{SCC} is marked in the same color.} 
\label{img:scctoposort}
\end{figure}

\subsection{Hypergraphs}
\label{sec:hypergraph}

\begin{definition}
An undirected weighted hypergraph $H = (V,E,c,\omega)$ is a set of hypernodes $V$ 
and a set of hyperedges $E$ with a hypernode weight function 
$c: V \rightarrow \mathbb{R}_{\ge 0}$ and a hyperedge weight 
function $\omega: E \rightarrow \mathbb{R}_{\ge 0}$. A hyperedge $e$ 
is a subset of $V$ (formally: $\forall e \in E: e \subseteq V$).
\label{def:hypergraph}
\end{definition}

A hypergraph generalizes a graph by extending the definition of an edge, which 
can contain more than two nodes. Hyperedges are also called \emph{nets} and the hypernodes
of a net are called \emph{pins}. For a subset $V' \subseteq V$ and $E' \subseteq E$ we
define
\begin{align*}
c(V') = \sum_{v \in V'} c(v) \\
\omega(E') = \sum_{e \in E'} \omega(e)
\end{align*}

A vertex $v$ is \emph{incident} to a hyperedge $e$, if $v \in e$.
Two vertices $u$ and $v$ are \emph{adjacent}, if there exists an 
$e \in E$ such that $u,v \in e$. $I(v)$ denotes the set of all 
\emph{incident} nets of $v$. The \emph{degree} of a hypernode 
$v$ is $d(v) = |I(v)|$. The size of a net $e$ is the cardinality $|e|$.

\begin{definition}
Let $H_{V'} = (V',E_{V'},c,\omega)$ be the subhypergraph of a hypergraph $H$
induced by $V' \subseteq V$ with $E_{V'} = \{e \cap V'\ |\ e \in E: e 
\cap V' \neq \emptyset\}$.
\label{def:subhypergraph}
\end{definition}

A hypergraph $H = (V,E,c,\omega)$ can be represented as an undirected graph. 
There are two common transformations, called \emph{clique} and \emph{bipartite} 
representation \cite{HuMoerder85}. The \emph{clique} graph $G_x(H) = (V,E_x)$ models
each net $e$ as a clique between its pins. The \emph{bipartite} graph $G_*(H) = 
(V \cup E, E_*)$ contains all hypernodes and hyperedges as nodes and connects each
net $e$ with an undirected edge $\{e,v\}$ to all its pins $v \in e$. The two transformations
are illustrated in \autoref{img:hypergraph_transformation}.

\begin{figure}
\centering
\includegraphics[width=1.0\textwidth]{../img/preliminaries/hypergraph_transformation.eps}
\caption{Example of a hypergraph $H$ and its two corresponding graph representations.} 
\label{img:hypergraph_transformation}
\end{figure}

\subsection{Hypergraph Partitioning}

\begin{definition}
A $k$-way partition of a hypergraph $H$ is a partition of its hypernodes into
$k$ disjoint blocks $\Pi = \{V_1,\ldots,V_k\}$ such that $\bigcup_{i=1}^{k} V_i = V$
and $V_i \neq \emptyset$.
\label{def:kway_partition}
\end{definition}

For a $k$-way partition $\Pi = \{V_1,\ldots,V_k\}$, we define the \emph{connectivity set} of a
hyperedge $e$ with $\Lambda(e,\Pi) = \{V_i \in \Pi\ |\ V_i \cap e \neq \emptyset\}$. The \emph{connectivity}
of a net $e$ is $\lambda(e,\Pi) = |\Lambda(e,\Pi)|$. A hyperedge $e$ is \emph{cut}, if
$\lambda(e,\Pi) > 1$. $E(\Pi) = \{e\ |\ \lambda(e,\Pi) > 1\}$ is the set of all \emph{cut} 
nets. We say two blocks $V_i$ and $V_j$ are adjacent, if there exists a hyperedge
$e$ with $V_i,V_j \in \Lambda(e,\Pi)$.

\begin{definition}
For a $k$-way partition $\Pi = \{V_1,\ldots,V_k\}$ of a hypergraph $H$ 
the quotient graph $Q = (\Pi,E')$ is a undirected graph containing an 
edge between each pair of adjacent blocks of $\Pi$.
More formal, $E' = \{(V_i,V_j)\ |\ \exists e \in E: V_i,V_j \in \Lambda(e,\Pi)\}$
\label{def:quotient_graph}
\end{definition}

We say a $k$-way partition is $\epsilon$-balanced, if each block 
$V_i \in \Pi$ satisfies the \emph{balance constraint} 
$c(V_i) \le (1+\epsilon)\lceil\frac{c(V)}{k}\rceil$.

\begin{definition}
The $k$-way hypergraph partitioning problem is to find an $\epsilon$-balanced $k$-way
partition $\Pi$ of a hypergraph $H$ such that a certain objective function is minimized.
\label{def:kway_partitioning_problem}
\end{definition}

There exists several objective function in the hypergraph partitioning context,
which should either be minimized or maximized. The most popular objective function 
is the $\text{cut}$ metric (especially for \emph{graph partitioning}), which is defined as
\[\text{cut}(\Pi) = \sum_{e \in E(\Pi)} \omega(e)\]
The goal is to minimize the sum of all \emph{cut} hyperedges. Another important metric
for this work is the $(\lambda - 1)$-metric or \emph{connectivity} metric, 
which is defined as
\[(\lambda - 1)(\Pi) = \sum_{e \in E} (\lambda(e) - 1)\omega(e)\]
The idea behind this function is to minimize the \emph{connectivity} of all hyperedges.